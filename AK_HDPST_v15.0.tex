% ====================================
% AK High-Dimensional Projection Structural Theory v15.0 (pdfLaTeX)
% ====================================
\documentclass[11pt]{article}

% --- Encoding & Language ---
\usepackage[utf8]{inputenc}
\usepackage[T1]{fontenc}
\usepackage[english]{babel}
\usepackage{geometry}
\geometry{margin=1in}

% --- Core Math & Utilities (AMS first) ---
\usepackage{amsmath,amssymb,amsthm,amsfonts,mathtools}
\usepackage{mathrsfs}
\usepackage{bm}
\usepackage{stmaryrd}
\usepackage{changepage}
\usepackage{amscd}
\usepackage{textcomp}
\usepackage{etoolbox} % <-- for \forcsvlist
\usepackage{enumitem}

% --- Fonts (pdfLaTeX-friendly) ---
\usepackage{newtxtext,newtxmath}  % Times-like text & math
\usepackage{inconsolata}          % Monospace for listings

% --- Diagrams ---
\usepackage{tikz, tikz-cd}
\usetikzlibrary{
  matrix, arrows.meta, cd, calc, positioning, quotes,
  decorations.pathmorphing, decorations.markings,
  shapes.misc, shapes.geometric, arrows, fit, backgrounds, fadings
}
\usepackage{pgfplots}
\pgfplotsset{compat=1.18}
\usepackage[all,cmtip]{xy}        % Xy-pic(tikz-cd と共存可)

% --- Tables ---
\usepackage{array,tabularx,booktabs}
% fixed-width paragraph columns
\newcolumntype{L}[1]{>{\raggedright\arraybackslash}p{#1}}
\newcolumntype{C}[1]{>{\centering\arraybackslash}p{#1}}
\newcolumntype{R}[1]{>{\raggedleft\arraybackslash}p{#1}}
% variable-width paragraph column (ragged-right)
\newcolumntype{Y}{>{\raggedright\arraybackslash}X}

% --- Monospace token helper (safe in math) ---
\newcommand{\fname}[1]{\text{\texttt{#1}}}

% --- Listings / Code ---
\usepackage{xcolor}
\usepackage{listings}
\usepackage{float}
\lstset{
  basicstyle=\ttfamily\small,
  keywordstyle=\color{blue},
  commentstyle=\color{gray},
  stringstyle=\color{orange},
  frame=single,
  breaklines=true,
  showstringspaces=false,
  captionpos=b,
  xleftmargin=1em,
  columns=flexible,
  literate=
    {∀}{{$\forall$}}1 {∃}{{$\exists$}}1
    {→}{{$\to$}}1 {←}{{$\leftarrow$}}1 {⟶}{{$\to$}}1 {⇒}{{$\Rightarrow$}}1 {⇔}{{$\Leftrightarrow$}}1 {↦}{{$\mapsto$}}1
    {α}{{$\alpha$}}1 {β}{{$\beta$}}1 {γ}{{$\gamma$}}1 {λ}{{$\lambda$}}1 {Π}{{$\Pi$}}1
    {ℕ}{{$\mathbb{N}$}}1 {ℤ}{{$\mathbb{Z}$}}1 {ℚ}{{$\mathbb{Q}$}}1 {ℝ}{{$\mathbb{R}$}}1
    {≤}{{$\le$}}1 {≥}{{$\ge$}}1 {≠}{{$\ne$}}1
    {∧}{{$\land$}}1 {∨}{{$\lor$}}1
    {ε}{{$\varepsilon$}}1 {μ}{{$\mu$}}1
    {⟨}{{$\langle$}}1 {⟩}{{$\rangle$}}1
    {…}{{$\ldots$}}1 {–}{{--}}1 {—}{{---}}1 {∞}{{$\infty$}}1
    {⊕}{{$\oplus$}}1 {⊗}{{$\otimes$}}1 {⊣}{{$\dashv$}}1 {⊢}{{$\vdash$}}1
    {≙}{{$\triangleq$}}1
    {·}{{$\cdot$}}1 {∙}{{$\cdot$}}1 {⋅}{{$\cdot$}}1 {•}{{$\bullet$}}1
    {・}{{$\cdot$}}1 {‧}{{$\cdot$}}1
    {×}{{$\times$}}1 {⨯}{{$\times$}}1
}

% --- Math Operators ---
\DeclareMathOperator{\Ext}{Ext}
\DeclareMathOperator{\Hom}{Hom}
\DeclareMathOperator{\Spec}{Spec}
\DeclareMathOperator{\colim}{colim}
\DeclareMathOperator{\PH}{PH}
\DeclareMathOperator{\Tor}{Tor}
\DeclareMathOperator{\rank}{rank}
\DeclareMathOperator{\im}{im}
\DeclareMathOperator{\id}{id}
\DeclareMathOperator{\Ker}{Ker}
\DeclareMathOperator{\Coker}{Coker}
\DeclareMathOperator{\Collapse}{Collapse}
\DeclareMathOperator{\Mot}{Mot}
\DeclareMathOperator{\GL}{GL}
\DeclareMathOperator{\RHom}{RHom}
\DeclareMathOperator{\HT}{HT}
\DeclareMathOperator{\gdim}{gdim}

\DeclareRobustCommand{\hyp}{\nobreakdash-}

% --- General Macros ---
\newcommand{\QQ}{\mathbb{Q}}
\newcommand{\RR}{\mathbb{R}}
\newcommand{\CC}{\mathbb{C}}
\newcommand{\ZZ}{\mathbb{Z}}
\newcommand{\TT}{\mathbb{T}}
\newcommand{\CollapseZone}{\mathfrak{C}}
\newcommand{\cF}{\mathcal{F}}
\newcommand{\cG}{\mathcal{G}}
\newcommand{\cE}{\mathcal{E}}
\newcommand{\cO}{\mathcal{O}}
\newcommand{\cD}{\mathcal{D}}
\newcommand{\cH}{\mathcal{H}}
\newcommand{\into}{\hookrightarrow}
\newcommand{\onto}{\twoheadrightarrow}
\newcommand{\eps}{\varepsilon}
\newcommand{\Sha}{\mathcal{X}} % (Tate--Shafarevich placeholder)
\newcommand{\CollapseCompatible}{\mathsf{CollapseCompatible}}
\newcommand{\CollapseTypeOf}[1]{\operatorname{CollapseType}(#1)}
\newcommand{\CollapseImage}{\operatorname{CollapseImage}}
\newcommand{\CollapseEnergy}{\mathcal{E}_{\mathrm{Coll}}}
\newcommand{\ptorsionfree}[1]{\ensuremath{#1}\nobreakdash-torsion-free}
\newcommand{\Pers}{\mathsf{Pers}}
\newcommand{\Vect}{\mathsf{Vect}}
\newcommand{\Ho}{\mathrm{Ho}}
\newcommand{\dq}{\textquotedbl}
\newcommand{\rH}{\mathrm{H}}
\newcommand{\timevar}{t} % use \HT(\timevar;F)
\newcommand{\T}{\mathsf{T}}
\newcommand{\C}{\mathsf{C}}
\newcommand{\Coll}{\mathsf{C}}
\newcommand{\Rfun}{\mathcal{R}}

% --- Theorem-like Environments ---
\numberwithin{equation}{section}
\newtheorem{theorem}{Theorem}[section]
\newtheorem{proposition}[theorem]{Proposition}
\newtheorem{lemma}[theorem]{Lemma}
\newtheorem{corollary}[theorem]{Corollary}
\newtheorem{axiom}{Axiom}[section]
\newtheorem{conjecture}{Conjecture}[section]
\theoremstyle{definition}
\newtheorem{definition}[theorem]{Definition}
\newtheorem{example}[theorem]{Example}
\newtheorem{remark}[theorem]{Remark}
\newtheorem{specification}[theorem]{Specification}
\newtheorem{declaration}[theorem]{Declaration}

% --- Unified notation ---
\DeclareRobustCommand{\FiltCh}[1]{\mathsf{FiltCh}(#1)}
\DeclareRobustCommand{\Perskft}{\Pers^{\mathrm{cons}}_{k}}
\DeclareRobustCommand{\Pk}{\mathbf{P}}
\DeclareRobustCommand{\Ttau}{\texorpdfstring{\ensuremath{\mathbf{T}_{\tau}}}{T\_\tau}}
\DeclareRobustCommand{\Ctau}{\texorpdfstring{\ensuremath{C_{\tau}}}{C\_\tau}}
\DeclareRobustCommand{\Dbplain}{D^{\mathrm{b}}}
\DeclareRobustCommand{\Dbkmod}{D^{\mathrm{b}}(k\text{-mod})}
\DeclareRobustCommand{\Db}{D^{\mathrm{b}}(k\text{-mod})}
\DeclareRobustCommand{\muc}{\mu_{\mathrm{Collapse}}}
\DeclareRobustCommand{\nuc}{\nu_{\mathrm{Collapse}}}
\DeclareRobustCommand{\fqi}{\text{f.q.i.}}
\DeclareRobustCommand{\LC}{\texorpdfstring{\ensuremath{\mathrm{(LC)}}}{(LC)}}
\DeclareRobustCommand{\Qtest}{\{\,k[0]\,\}}
\DeclareRobustCommand{\pos}[1]{\left(#1\right)_{+}}
\DeclareRobustCommand{\Trop}{\texorpdfstring{\ensuremath{\operatorname{Trop}}}{Trop}}
\DeclareRobustCommand{\Mirror}{\texorpdfstring{\ensuremath{\operatorname{Mirror}}}{Mirror}}
\newcommand{\ProofBadge}{\textsf{\footnotesize[Proof]}}
\newcommand{\SpecBadge}{\textsf{\footnotesize[Spec]}}

% --- Over/underfull mitigation ---
\usepackage{microtype}
\emergencystretch=2em

% --- Hyperlinks (load ONCE, near the end) ---
\usepackage[colorlinks=true, linkcolor=blue, citecolor=blue, urlcolor=blue]{hyperref}
\usepackage{xurl}                 % long URL/code line breaks
\urlstyle{tt}
\def\UrlBreaks{\do\/\do\_\do\.\do\-}

% --- Clever references (after hyperref) ---
\usepackage[capitalise,nameinlink,noabbrev]{cleveref}

% --- Float / page control ---
\usepackage{placeins}

% ======================================================
% === Paste-artifact guard: robust \n handling (HEAVY) ===
% ======================================================
% Baseline: treat \n as a space even if followed by a control sequence.
\providecommand{\n}{\unskip\space}

% Word-level fixes: define \n<Word> for many common starters (both cases).
\makeatletter
\newcommand{\patchn}[1]{\expandafter\def\csname n#1\endcsname{\unskip\space #1}}

% Frequent English/connective words and theorem-like labels
\forcsvlist{\patchn}{Hence,Therefore,Thus,Then,Proof,Remark,Theorem,Proposition,Lemma,Corollary,
Definition,Example,Declaration,Specification,Spec,Appendix,Section,Subsection,We,In,Let,Under,
Given,Assuming,Assumption,Fix,If,When,Where,While,For,From,By,On,Off,Also,Moreover,Conversely,
Equivalently,Finally,Next,First,Second,Third,Note,Notation,Convention,Claim,Case,Step,Algorithm,
Figure,Table,Thm,Prop,Lem,Cor,Def,Ex,Rem,Eq,Eqn,Ch,Chap,PF,BC,AK,NS,HT,PH}

\forcsvlist{\patchn}{hence,therefore,thus,then,proof,remark,theorem,proposition,lemma,corollary,
definition,example,declaration,specification,spec,appendix,section,subsection,we,in,let,under,
given,assuming,assumption,fix,if,when,where,while,for,from,by,on,off,also,moreover,conversely,
equivalently,finally,next,first,second,third,note,notation,convention,claim,case,step,algorithm,
figure,table,thm,prop,lem,cor,def,ex,rem,eq,eqn,ch,chap,pf,bc,ak,ns,ht,ph}

% Abbrev. with punctuation often seen after \n (e.g., \n(Proof))
\patchn{Proof)}\patchn{Remark)}\patchn{Theorem)}\patchn{Lemma)}\patchn{Corollary)}

% One-letter and digit fallbacks (covers \nA., \n1, etc. — NOT a full catch-all)
\newcommand{\patchnletter}[1]{\expandafter\def\csname n#1\endcsname{\unskip\space #1}}
\@tfor\@tempa:=ABCDEFGHIJKLMNOPQRSTUVWXYZabcdefghijklmnopqrstuvwxyz0123456789\do{%
  \expandafter\patchnletter\@tempa}
\makeatother



% === Document Metadata ===
\title{AK High-Dimensional Projection Structural Theory\\
\Large Version 15.0: Collapse Structures, Group Simplification, and Persistent Projection Geometry} 
\author{\textbf{Atsushi Kobayashi} \quad {\small (with ChatGPT Research Partner)}}
\date{August 2025}

% === Document Start ===
\begin{document}

\maketitle



\begin{abstract}
We present the AK High-Dimensional Projection Structural Theory (AK--HDPST) v15.0, a two-layer framework for functorial \emph{collapse} based on higher-dimensional projections with controlled obstruction removal. The deliverables are:
(i) a provable \emph{core} with machine-checkable statements, and
(ii) a \textbf{[Spec]} layer that safely extends the core under explicit, auditable hypotheses.

Core (constructible 1D persistence over a field).
We work in the one-parameter, constructible subcategory, where the exact Serre reflector \(\mathbf{T}_\tau\) deletes all bars of length \(\le \tau\) and is \(1\)-Lipschitz for interleavings; a filtered lift \(C_\tau\) exists up to filtered quasi-isomorphism (f.q.i.) with \(\mathbf{P}_i C_\tau \simeq \mathbf{T}_\tau \mathbf{P}_i\).
A \emph{collapse gate} is certified by a one-way bridge:
\(\mathrm{PH}_1(F)=0 \Rightarrow \Ext^1(\mathcal{R}(F),k)=0\) under \(t\)-exact realization of amplitude \(\le 1\) (no converse is used).
Directed systems are audited by windowed comparison maps \(\phi_{i,\tau}\); their kernel/cokernel diagnostics \((\mu,\nu)\) detect failure of limit collapse even when all finite levels collapse.
Updates are split by policy: deletion-type steps are \emph{non-increasing} after truncation for windowed persistence energies and spectral indicators; inclusion-type steps are \emph{non-expansive}.
All comparisons are performed \emph{after collapse} using the protocol
“for each \(t\) \(\to\) persistence \(\to\) \(\mathbf{T}_\tau\) \(\to\) compare,”
on MECE, right-open windows.

Pipeline control and reproducibility.
A per-window \(\delta\)-ledger aggregates continuation shifts, commutation defects, and numerical tolerances.
B--Gate\(^{+}\) enforces a safety margin \(\mathrm{gap}_\tau>\Sigma\delta\) per window, and Restart/Summability pastes windowed certificates to global ones.
Versioned logs (bars/spec/ext/phi) and cross-linked hashes enable auditability and re-runs.

[Spec] layer (non-expansive after truncation).
We include auditable extensions:
lax monoidal tensor/collapse with windowed energy bounds and a mono regime yielding post-collapse dominance;
projection formula/base change transported to persistence via the windowed protocol;
and Fukaya realizations with action filtration (continuation is \(1\)-Lipschitz; adding stops is deletion-type).
Mirror/Transfer pipelines contribute \(\delta\)-controlled commutation; a permitted-operations table maps each step to \(\delta\)-ledger entries and B--Gate\(^{+}\) usage.
All [Spec] items are explicitly marked, used only within their stated hypotheses, and audited by \((\mu,\nu)\).

Scope.
We do not claim \(\mathrm{PH}_1 \Leftrightarrow \Ext^1\), nor make statements beyond constructible 1D persistence over a field.
The organizing principle is a sharp boundary between theorems and [Spec] contracts: this enables safe reuse, cross-domain exploration, and machine-checkable testing while preserving verified guarantees.
\end{abstract}

% ------------------------------------------------------------
% Scope boilerplate (concise, for chapter-level reference)
% ------------------------------------------------------------
\medskip
\noindent\textbf{Scope boilerplate (all chapters).}\label{boilerplate:scope}
\begin{itemize}
  \item Base: one-parameter, constructible persistence over a field; bounded-window finiteness.
  \item Truncation: \(\mathbf{T}_\tau\) is exact, idempotent, \(1\)-Lipschitz, and deletes exactly the bars of length \(\le \tau\); \(C_\tau\) lifts it up to f.q.i. with \(\mathbf{P}_i C_\tau \simeq \mathbf{T}_\tau \mathbf{P}_i\).
  \item Layer discipline: equalities and (co)limit/pullback compatibilities are asserted at the persistence layer; filtered-complex statements are up to f.q.i.
  \item Policy: deletion-type updates are non-increasing after truncation; inclusion-type updates are non-expansive (stability only).
  \item Bridge: we use only \(\mathrm{PH}_1(F)=0 \Rightarrow \Ext^1(\mathcal{R}(F),k)=0\) under \(t\)-exact, amplitude \(\le 1\); no converse.
  \item Protocol: comparisons are windowed and performed “\(t\) \(\to\) persistence \(\to\) \(\mathbf{T}_\tau\) \(\to\) compare,” with per-window \(\delta\)-ledger and B--Gate\(^{+}\) (\(\mathrm{gap}_\tau>\Sigma\delta\)).
\end{itemize}



% ===========================
% Chapter 1: Collapse — Operational Definition and Scope (reinforced v15.0 → proof framework)
% ===========================
\section{Chapter 1: Collapse — Operational Definition and Scope}
\addcontentsline{toc}{section}{Collapse — Operational Definition and Scope}

\begin{remark}[Global guard-rails (reinforced)]\label{rk:global-guards}
(G1) \emph{Constructible regime.} All persistence objects lie in the one-parameter, \emph{constructible} subcategory $\Pers^{\mathrm{cons}}_k$ (finite critical set; Appendix~A).\footnote{$\Pers^{\mathrm{cons}}_k$ is abelian, every object admits an interval decomposition, and lengths are well-defined; see Crawley--Boevey (2015); Chazal--de~Silva--Glisse--Oudot (2016).}

(G2) \emph{Bridge scope (field coefficients only).} All $\Ext$-tests use \emph{field} coefficients, and the implication
\[
\mathrm{PH}_1(F)=0\ \Longrightarrow\ \Ext^1\!\big(\mathcal{R}(F),k\big)=0
\]
is \textbf{proved only in $D^{\mathrm{b}}(k\text{-mod})$} under $t$-exactness and amplitude $\le 1$ (Chapter~3, assumptions (B1)–(B3)). No converse is claimed, and the bridge is not extended outside this domain (uses of other coefficients, e.g.\ Novikov, are \textbf{[Spec]} only).

(G3) \emph{Monotonicity vs.\ stability.} For \emph{deletion-type} updates (Dirichlet restriction, principal submatrices/Schur complements, positive-semidefinite Loewner contractions, conservative averaging), windowed persistence energies and spectral tails/heat traces are \emph{non-increasing after truncation}. For \emph{inclusion-type} updates we claim only \emph{stability} (non-expansiveness). See Appendix~E for sufficient conditions and counterexamples.

(G4) \emph{Tower diagnostics after truncation.} The invariants $(\mu_{\mathrm{Collapse}},\nu_{\mathrm{Collapse}})$ are evaluated \emph{after} $\mathbf{T}_\tau$; under \textbf{objectwise, degreewise filtered colimits} in the constructible range (Appendix~A), the comparison maps $\phi_i$ are isomorphisms in $\Pers^{\mathrm{cons}}_k$ and hence $(\mu_{\mathrm{Collapse}},\nu_{\mathrm{Collapse}})=(0,0)$ (Appendix~D).

(G5) \emph{Derived geometric/sheaf realizations are [Spec].} Realizations into $D^{\mathrm{b}}\!\operatorname{Coh}$ or the bounded derived category of constructible étale sheaves $D^{\mathrm{b}}_{\mathrm{c}}(X_{\acute{e}t},\Lambda)$ appear only as \textbf{[Spec]} and rely on projection formula/base change summarized in Appendix~N; they do not enlarge the proven bridge beyond $D^{\mathrm{b}}(k\text{-mod})$.

(G6) \emph{Strength of functoriality.} Strong adjunction/commutation statements (beyond the persistence-layer identities stated below) are \textbf{[Spec]}; their proofs, limits of validity, and model choices are deferred to the appendices (see Appendices~B and~K for the implementable range and categorical architecture).

(G7) \emph{Windowed proof policy (MECE).} All claims are made \emph{per window} and reported as \emph{windowed certificates}. Domain windows are partitioned as half-open intervals $[u_k,u_{k+1})$ (right-inclusive convention), forming a \emph{MECE} cover of the monitored range; see Appendix~A for the window policy and coverage checks. Global statements are obtained by \emph{pasting} windowed certificates using the Restart/Summability policy (Chapter~4).

(G8) \emph{B-side single-layer judgement.} All accept/reject decisions (gates) are issued \emph{only} on the B-side \emph{after} collapse, i.e.\ on the single-layer objects $\mathbf{T}_\tau\mathbf{P}_i$ (equivalently $\mathbf{P}_i(C_\tau -)$). Equalities on filtered complexes hold only \emph{up to filtered quasi-isomorphism (f.q.i.)}.

(G9) \emph{Asymmetric mirror with $\delta$-ledger.} A-side (e.g.\ symplectic/Fukaya) may run multi-layer exploration, but each step is classified as \emph{deletion-type} (monotone), \emph{$\varepsilon$-continuation} (non-expansive), or \emph{inclusion-type} (stable only). Each step is followed by a collapse $C_{\tau_j}$ and measured on B-side. Non-commutation between $\Mirror$ and $C_{\tau_j}$ is \emph{externalized} as a \emph{$\delta$-ledger} with components $\delta^{\mathrm{alg}},\delta^{\mathrm{disc}},\delta^{\mathrm{meas}}$; along pipelines the total budget is \emph{additive} (Appendix~L).

(G10) \emph{Spectral indicators are auxiliary.} Spectral quantities (tails, heat traces) and the corresponding \emph{auxiliary spectral bars} (aux-bars; Chapter~11) are \emph{not} f.q.i.-invariants and are used as \emph{auxiliary gauges only}: \emph{monotone} for deletion-type steps and \emph{stable} under $\varepsilon$-continuations. Final gate decisions never rely solely on spectral indicators.

(G11) \emph{Reproducibility and logs.} Each run records the window partition, collapse thresholds, operations (with type and $\delta$-decomposition), B-side indicators (PH, Ext, tower), and gate verdict in a manifest (Appendix~G). Coverage and additivity checks (MECE, $\sum\delta$) are mandatory.

(G12) \emph{Ext usage is one-way only.} $\Ext^1$-tests are performed \emph{only} within the scope of (G2). No $\Ext\Rightarrow\mathrm{PH}$ nor global equivalence is asserted; any local equivalence (if available) is restricted to saturation windows and is reported explicitly as \textbf{[Spec]}.
\end{remark}

\noindent\emph{Scope boilerplate.} This chapter (and all that follow) adheres to the global \emph{Scope boilerplate} stated above: strong claims are confined to constructible $1$D persistence over a field; equalities and exactness are asserted \emph{at the persistence layer}; filtered-complex statements hold \emph{up to f.q.i.}; deletion-type updates are \emph{non-increasing} after truncation, inclusion-type updates are \emph{stability-only}; the bridge $\mathrm{PH}_1\Rightarrow \Ext^1$ is used only in $D^{\mathrm{b}}(k\text{-mod})$; windowed proof policy (MECE) and $\delta$-ledger are in force.

\subsection*{1.0. Windowed proof policy and B-Gate$^{+}$}
We formalize the windowed certificate and the standard acceptance gate used throughout the paper.

\begin{definition}[Windows (MECE), safety margin, and $\delta$-ledger]\label{def:windows-delta}
A \emph{domain window} is a half-open interval $W=[u,u')$ (right-inclusive). A windowing is \emph{MECE} if $\bigsqcup_k [u_k,u_{k+1})=[u_0,U)$, each interval is half-open with right-inclusion, and adjacent windows share endpoints. The \emph{collapse window} is given by a threshold $\tau>0$ (possibly varying per step but fixed for a gate). For a pipeline of A-side steps with per-step non-commutation budgets $\delta_j=\delta^{\mathrm{alg}}_j+\delta^{\mathrm{disc}}_j+\delta^{\mathrm{meas}}_j$, we set the \emph{pipeline budget}
\[
\Sigma\delta(i,\{\tau_j\})\ :=\ \sum_j \delta_j(i,\tau_j),
\]
which is used against the \emph{safety margin} $\mathrm{gap}_\tau$ determined by the window and the collapse policy. Mandatory reproducibility checks (Appendix~G) ensure MECE coverage and additivity $\sum_k\Sigma\delta_k=\Sigma\delta$.
\end{definition}

\begin{definition}[B-Gate$^{+}$ (standard gate on the B-side)]\label{def:bgate-plus}
Fix a domain window $W=[u,u')$, a collapse threshold $\tau>0$, and a degree $i$.
We say that \emph{B-Gate$^{+}$ passes on $(W,\tau,i)$} if the following hold \emph{on the B-side after collapse}:
\begin{enumerate}
\item $\mathrm{PH}_1(C_\tau F)=0$;
\item $\Ext^1(\mathcal{R}(C_\tau F),k)=0$ (only within the scope (G2));
\item The tower audit \emph{after} $\mathbf{T}_\tau$ yields $(\mu_{\mathrm{Collapse}},\nu_{\mathrm{Collapse}})=(0,0)$ (tail isomorphism of comparison maps $\phi$);
\item The \emph{safety margin} satisfies $\mathrm{gap}_\tau>\Sigma\delta(i,\{\tau_j\})$.
\end{enumerate}
Optionally (auxiliary), we may require the \emph{auxiliary spectral bars} (aux-bars, Chapter~11) to vanish after collapse on the chosen spectral window. The \emph{windowed certificate} for $(W,\tau)$ asserts “no failure with no false negative on $W$’’ when B-Gate$^{+}$ passes.
\end{definition}

\begin{remark}[Asymmetric mirror with pipeline error budget]\label{rk:asym-mirror}
Let $U_m,\dots,U_1$ be A-side steps (each classified as deletion-type or $\varepsilon$-continuation), and let $C_{\tau_j}$ be the stepwise collapses. For every $i$ and any \emph{fixed} B-side collapse threshold $\tau$,
\[
d_{\mathrm{int}}\!\Big(\n\mathbf{T}_\tau \mathbf{P}_i\big(\Mirror(C_{\tau_m}U_m\cdots C_{\tau_1}U_1F)\big),\ \n\mathbf{T}_\tau \mathbf{P}_i\big(C_{\tau_m}U_m\cdots C_{\tau_1}U_1\Mirror F\big)\n\Big)\ \le\ \sum_{j=1}^m \delta_j(i,\tau_j),
\]
with a bound \emph{uniform in $F$}. Subsequent $1$-Lipschitz post-processing on the persistence layer does not increase the right-hand side (Appendix~L). If a pair of collapsers $T_A,T_B$ are not provably commuting, we adopt the \emph{soft-commuting} policy: measure $\Delta_{\mathrm{comm}}:=d_{\mathrm{int}}(T_A T_B M,T_B T_A M)$; if $\Delta_{\mathrm{comm}}\le \eta$, we accept parallel collapse; otherwise we fall back to a fixed order and record $\Delta_{\mathrm{comm}}$ in $\delta^{\mathrm{alg}}$.
\end{remark}

\subsection*{1.1. Terminology and notation}
Fix a base field $k$. Let $\mathsf{Vect}_k$ denote finite-dimensional $k$-vector spaces and $D^{\mathrm{b}}(k\text{-mod})$ the bounded derived category of finite-dimensional $k$-modules.

\paragraph{Filtered chain complexes.}
Write $\mathsf{FiltCh}(k)$ for finite-type (filtered-constructible) filtered chain complexes 
$F=(C_\bullet,\{F^{t}C_\bullet\}_{t\in\mathbb{R}})$ over $k$, with 
$F^{t}C_\bullet\subseteq F^{t'}C_\bullet$ for $t\le t'$ and filtered chain maps as morphisms.
For each $i\in\mathbb{Z}$ (bounded in homological degree), degreewise homology yields a persistence module with structure maps induced by the inclusions $F^{t}C_\bullet\hookrightarrow F^{t'}C_\bullet$:
\[
\mathbf{P}_i(F)\colon \mathbb{R}\to \mathsf{Vect}_k,\qquad\nt\longmapsto \mathrm{H}_i(F^{t}C_\bullet).
\]
Thus $\mathbf{P}_i(F)\in \Pers^{\mathrm{cons}}_k$ (constructible on bounded windows) and its barcode is denoted by $\mathrm{PH}_i(F)$ (a multiset of intervals). 
Writing $\mathrm{PH}_1(F)=0$ means that $\mathbf{P}_1(F)$ is the zero object in 
$\Pers^{\mathrm{cons}}_k$ (equivalently, the barcode is empty).

\paragraph{Standing convention (constructible range).}
All persistence arguments take place in $\Pers^{\mathrm{cons}}_k\subset\Pers_k$ (finite critical set on bounded intervals). Abelianity, Serre subcategories, and exact localizations are used only within $\Pers^{\mathrm{cons}}_k$ (Appendix~A). When filtered (co)limits are used, they are computed objectwise in $[\mathbb{R},\mathsf{Vect}_k]$ and then verified to return to $\Pers^{\mathrm{cons}}_k$; we only assert equalities \emph{at the persistence layer}. On filtered complexes we retain the finite-type hypothesis and note that filtered colimits may exit finiteness. By convention we identify $\Pers^{\mathrm{ft}}_k$ with $\Pers^{\mathrm{cons}}_k$ and henceforth write only $\Pers^{\mathrm{cons}}_k$.

\paragraph{Realization functor and test family.}
A realization is a $t$-exact functor $\mathcal{R}:\mathsf{FiltCh}(k)\to D^{\mathrm{b}}(k\text{-mod})$ of amplitude $\le 1$. We fix the \emph{minimal} test family $\mathcal{Q}=\{k[0]\}$ for $\Ext^1$-vanishing tests.

\paragraph{Interleaving distance.}
Throughout, let $d_{\mathrm{int}}$ denote the interleaving (equivalently, bottleneck) distance on $\Pers^{\mathrm{cons}}_k$ (Appendix~A). All Lipschitz claims are with respect to $d_{\mathrm{int}}$.

\paragraph{Epistemic stance.}
We treat \emph{collapse} as a functorial simplification exposed via higher-dimensional projections and controlled obstruction removal. Operationally, we work with (i) observable indicators $(\mathrm{PH}_1,\ \Ext^1,\ \mu_{\mathrm{Collapse}},\ \nu_{\mathrm{Collapse}})$, (ii) an admissible entry gate (the \emph{Collapse Zone}), and (iii) a failure typology including \emph{invisible} limit obstructions. In addition, we \emph{enforce} a windowed proof policy (Definition~\ref{def:windows-delta}) and a single-layer judgement after collapse (B-side), with non-commutation externalized via a $\delta$-ledger.

\subsection*{1.2. Collapse (operational definition, v15.0)}
We define a metrically stable truncation on persistence and a filtered lift used for constructions.

\begin{definition}[Exact truncation on persistence and abbreviation]\label{def:Ttau}
For $\tau>0$, let
\[
\mathbf{T}_\tau:\ \Pers^{\mathrm{cons}}_k\longrightarrow (\mathsf{E}_\tau)^{\perp}
\]
be the exact reflective localization (Serre quotient) at the Serre subcategory generated by \emph{interval modules} of length $\le \tau$, i.e.\ $\mathbf{T}_\tau$ deletes precisely those \emph{finite} bars. Here $(\mathsf{E}_\tau)^{\perp}$ denotes the full \emph{$\tau$-local} (orthogonal) subcategory.\footnote{Existence and exactness hold in the constructible $1$D range; see Crawley--Boevey (2015); Chazal--de~Silva--Glisse--Oudot (2016). In general, ``$\tau$-local'' (orthogonal to $\mathsf{E}_\tau$ for both $\Hom$ and $\Ext^1$) is finer than ``$\tau$-torsion-free''.}
We use the shorthand $\Ttau:=\mathbf{T}_\tau$ upon first mention. Its $1$-Lipschitz property is recalled in Lemma~\ref{lem:shift}.
\end{definition}

\begin{definition}[Thresholded collapse lift on filtered complexes]\label{def:Ctau}
A \emph{thresholded collapse lift} is a functor
\[
C_\tau:\ \mathsf{FiltCh}(k)\longrightarrow \mathsf{FiltCh}(k)
\]
defined \emph{up to filtered quasi-isomorphism (f.q.i.)} such that, for every $i$,
\[
\mathbf{P}_i\!\big(C_\tau(F)\big)\ \cong\ \mathbf{T}_\tau\!\big(\mathbf{P}_i(F)\big)\quad\n\text{in }\Pers^{\mathrm{cons}}_k.
\]
All (co)limit and pullback compatibilities used in this work are applied \emph{at the persistence layer} via these identifications (see Chapter~2, §2.3).
\end{definition}

\begin{remark}[Endpoints and infinite bars]\label{rem:endpoints-infinite-bars}
Endpoint conventions and the treatment of infinite bars are centralized in Chapter~2 (see Chapter~2, Remark~\ref{rk:2-endpoints}); all later references defer to that remark without repetition.
\end{remark}

\begin{remark}[Lipschitz stability and scope]\label{rk:Ctau-stability}
$\mathbf{T}_\tau$ is $1$-Lipschitz for $d_{\mathrm{int}}$, hence $C_\tau$ is $1$-Lipschitz \emph{at the persistence level} via the identifications $\mathbf{P}_i\!\circ C_\tau\simeq \mathbf{T}_\tau\!\circ \mathbf{P}_i$. Compatibilities with filtered colimits and finite pullbacks are used only through $\mathbf{T}_\tau$ (Appendix~A). Spectral indicators (tails/heat traces) are \emph{not} f.q.i.-invariants; they are controlled by stability under a fixed policy $(\beta,M(\tau),s)$ on $L(C_\tau F)$ (cf.\ Chapter~11).
\end{remark}

\begin{definition}[PH--collapse and Ext--collapse]
For $F\in\mathsf{FiltCh}(k)$:
\begin{itemize}
  \item \emph{PH--collapse in degree $1$}: $\ \mathrm{PH}_1(F)=0$.
  \item \emph{Ext--collapse (tested against $\mathcal{Q}=\{k[0]\}$)}: $\ \Ext^1\!\big(\mathcal{R}(F),k\big)=0$.
\end{itemize}
\end{definition}

\begin{definition}[Collapse Zone (binary entry gate)]
Set
\[
\texttt{CollapseAdmissible}(F)\;:\!\!\iff\;\n\mathrm{PH}_1(F)=0\ \wedge\ \Ext^1\!\big(\mathcal{R}(F),k\big)=0.
\]
The \emph{Collapse Zone} is the full subcategory
\[
\mathfrak{C}\ :=\ \{\,F\in\mathsf{FiltCh}(k)\mid \texttt{CollapseAdmissible}(F)\,\}\subset \mathsf{FiltCh}(k).
\]
Under the hypotheses of Chapter~3 (assumptions (B1)–(B3)), PH--collapse implies Ext--collapse in $D^{\mathrm{b}}(k\text{-mod})$; no converse is asserted anywhere in this text.
\end{definition}

\begin{remark}[Invariance]
$\texttt{CollapseAdmissible}$ is invariant under filtered quasi-isomorphisms and under equivalences preserving $\mathbf{P}_i(-)$ and $\mathcal{R}(-)$ up to isomorphism in $D^{\mathrm{b}}(k\text{-mod})$.
\end{remark}

\paragraph{Abstract functorial viewpoint.}
Let $\mathsf{Triv}\subset\mathsf{FiltCh}(k)$ be the full subcategory of objects with vanishing $\mathrm{PH}_1$ and tested $\Ext^1$. If the inclusion $\iota:\mathsf{Triv}\hookrightarrow\mathsf{FiltCh}(k)$ admits a right adjoint $C$ (existence of a genuine right adjoint is \textbf{[Spec]} beyond the implementable range), we call such $C$ a \emph{collapser}; the unit $\eta_F:F\to \iota C(F)$ serves as a canonical collapse morphism. In practice, $C_\tau$ from Definition~\ref{def:Ctau} is the operative, metrically stable proxy.

\subsection*{1.3. The entry gate as a measurable predicate}
The predicate $\texttt{CollapseAdmissible}$ separates, by observable tests, objects whose first persistent topology and tested categorical extensions vanish:
\begin{itemize}
  \item \emph{Measurability:} $\mathrm{PH}_1{=}0$ is read off barcodes; $\Ext^1$-vanishing is checked against $\{k[0]\}$ in $D^{\mathrm{b}}(k\text{-mod})$.
  \item \emph{Functorial simplification:} $C_\tau$ erases bars of length $\le \tau$ and, under Chapter~3 hypotheses, preserves admissibility.
  \item \emph{Cross-domain interface:} the gate supports specifications (geometry, arithmetic, mirror/tropical, PDE) \emph{without} positing cross-domain equivalences.
\end{itemize}

\subsection*{1.4. Failure landscape and the invisible obstruction
\texorpdfstring{$(\mu_{\mathrm{Collapse}},\nu_{\mathrm{Collapse}})$}{(mu,nu)}}\label{sec:typeIV}
We classify failures of collapse and introduce tower-sensitive diagnostics.

\paragraph{Generic fiber dimension (intuition).}
For a constructible persistence module $M$, the \emph{generic fiber dimension} is the stabilized rank
\[
\mathrm{gdim}(M)\ :=\ \lim_{t\to+\infty}\dim_k M(t),
\]
which exists in the constructible setting. After truncation by $\mathbf{T}_\tau$, $\mathrm{gdim}$ coincides with the multiplicity of the infinite bar $I[0,\infty)$ in the barcode (Appendix~D, Remark~\ref{rem:D-generic-dim}).

\paragraph{Observable failure types.}
\begin{itemize}
  \item \textbf{Type I (Topological):} $\mathrm{PH}_1(F)\neq 0$.
  \item \textbf{Type II (Categorical):} $\Ext^1\!\big(\mathcal{R}(F),k\big)\neq 0$.
  \item \textbf{Type III ([Spec]-level instability):} admissibility is unstable under a prescribed operation (e.g.\ a specific pullback/filtered colimit) in a way detectable at finite level.\footnote{Stability specifications appear in Part~I; cross-domain specifications are in Part~II.}
\end{itemize}

\paragraph{Invisible failure along towers (Type IV).}
Let $\{F_n\}_{n\in\mathbb{N}}$ be a directed system in $\mathsf{FiltCh}(k)$ with limit $F_\infty$. For fixed $\tau>0$ and each degree $i$, the natural comparison \emph{after truncation by $\mathbf{T}_\tau$}
\[
\phi_i:\ \varinjlim_{n}\ \mathbf{T}_\tau\!\big(\mathbf{P}_i(F_n)\big)\ \longrightarrow\\n\mathbf{T}_\tau\!\big(\mathbf{P}_i(F_\infty)\big)
\]
induces the tower-sensitivity invariants
\[
\mu_{\mathrm{Collapse}}^{\,i}:=\dim_k\ker(\phi_i),\qquad\n\nu_{\mathrm{Collapse}}^{\,i}:=\dim_k\mathrm{coker}(\phi_i),
\]
interpreting $\dim_k$ as the \emph{generic fiber} dimension after truncation (multiplicity of $I[0,\infty)$ summands; Appendix~D, Remark~\ref{rem:D-generic-dim}). Define
\[
\mu_{\mathrm{Collapse}}:=\sum_i \mu_{\mathrm{Collapse}}^{\,i},\qquad\n\nu_{\mathrm{Collapse}}:=\sum_i \nu_{\mathrm{Collapse}}^{\,i}.
\]
We say $F_\infty$ exhibits \emph{invisible failure (Type IV)} if $(\mu_{\mathrm{Collapse}},\nu_{\mathrm{Collapse}})\neq (0,0)$. In that case, all finite layers may appear admissible while the limit fails the gate. Finiteness of the sums follows from constructibility and bounded homological degrees.

\medskip
\noindent\emph{Purpose and non-identity.} $(\mu_{\mathrm{Collapse}},\nu_{\mathrm{Collapse}})$ diagnose limit effects of collapse \emph{at the persistence layer}; they are \emph{not} classical Iwasawa invariants. A canonical refinement-limit example (Chapter~4) exhibits pure cokernel-type failure at fixed~$\tau$. They are invariant under filtered quasi-isomorphisms and under cofinal reindexing (Appendix~J).

\medskip
\noindent\emph{Technical provenance.} The maps $\phi_i$, shift–commutation and Lipschitz control (Lemma~\ref{lem:shift}), and stability/(co)limit compatibility (Proposition~\ref{prop:stability}) are developed in Chapter~2, §2.3. We use generic fiber dimensions as in Appendix~D, Remark~\ref{rem:D-generic-dim}.

\subsection*{1.5. Admissible A-side operations and single-layer judgement}
For completeness we record the permitted A-side steps and the single-layer policy.

\begin{definition}[Admissible A-side operations]\label{def:admissible-ops}
Each A-side step $U$ applied to filtered complexes is labeled as:
\begin{itemize}
  \item \emph{Deletion-type (monotone):} induces non-increase of windowed persistence energies and of auxiliary spectral counters after applying $C_\tau$ (Appendix~E). Examples: stop addition/sector shrinking (Appendix~O), mollification (low-pass), viscosity increase, threshold lowering.
  \item \emph{$\varepsilon$-continuation (non-expansive):} is $1$-Lipschitz for $d_{\mathrm{int}}$; any drift is bounded by the declared $\varepsilon$ (Appendix~O).
  \item \emph{Inclusion-type (stable only):} may increase indicators; only stability (non-expansiveness) is claimed; monotonicity assertions are not made.
\end{itemize}
Every step is followed by a collapse $C_{\tau_j}$ and all measurements are taken \emph{only} on the B-side single layer $\mathbf{T}_{\tau_j}\mathbf{P}_i$.
\end{definition}

\begin{declaration}[Windowed certificate and pasting]\label{dec:windowed-certificate}
If B-Gate$^{+}$ (Definition~\ref{def:bgate-plus}) passes on $(W,\tau,i)$, we issue a \emph{windowed regularity certificate} for $(W,\tau)$: “no failure with no false negative on $W$.” Global statements are obtained by pasting windowed certificates along a MECE partition using the Restart Lemma and the Summability policy (Chapter~4). All cross-domain comparisons are performed after collapse, and any non-commutation is accounted for in the $\delta$-ledger.
\end{declaration}



% ===========================
% Chapter 2: Concrete Model — Finite-Type Filtered Chain Complexes and Thresholded Collapse (reinforced)
% ===========================
\section{Chapter 2: Concrete Model — Finite-Type Filtered Chain Complexes and Thresholded Collapse}
\addcontentsline{toc}{section}{Concrete Model — Finite-Type Filtered Chain Complexes and Thresholded Collapse}

\subsection*{2.1. The category \texorpdfstring{$\mathsf{FiltCh}(k)$}{FiltCh(k)} and persistence modules}
Fix a field \(k\). Let \(\mathsf{Ch}^{b}(k)\) be the category of bounded chain complexes of finite-dimensional \(k\)-vector spaces. A \emph{finite-type filtered chain complex} is a pair
\[
F=(C_\bullet,\{F^{t}C_\bullet\}_{t\in\mathbb{R}})
\]
where \(F^{t}C_\bullet\subseteq F^{t'}C_\bullet\) for \(t\le t'\), the filtration is exhaustive and left-bounded, and, for each \(i\), the persistence module
\[
\mathrm{H}_i(F):\ \mathbb{R}\longrightarrow \mathsf{Vect}_k,\qquad t\longmapsto \mathrm{H}_i(F^{t}C_\bullet)
\]
is pointwise finite-dimensional with finitely many critical parameters on compact intervals. Denote by \(\mathsf{FiltCh}(k)\) the category of such \(F\) with filtration-preserving chain maps. For each \(i\), let
\[
\mathbf{P}_i:\ \mathsf{FiltCh}(k)\longrightarrow \Pers^{\mathrm{cons}}_k
\]
be the functor to the abelian category \(\Pers^{\mathrm{cons}}_k\) of finite-type \(k\)-persistence modules (indexed by \((\mathbb{R},\le)\)), sending \(F\mapsto \mathrm{H}_i(F)\). We write \(\mathrm{PH}_i(F)\) for the barcode (multiset of intervals) of \(\mathbf{P}_i(F)\). Throughout, the interleaving (equivalently, bottleneck) distance on persistence modules is denoted \(d_{\mathrm{int}}\) (Appendix~A).

\paragraph{Standing convention (constructible range and notation).}
Throughout Part~I we work inside the constructible subcategory
\(\Pers^{\mathrm{cons}}_k\subset\Pers_k\) of persistence modules indexed by \((\mathbb{R},\le)\)
that have a finite critical set on every bounded interval (equivalently, p.f.d.\ with finitely many changes on compacts).
In this paper the symbol \(\Pers^{\mathrm{ft}}_k\) is \emph{identified} with \(\Pers^{\mathrm{cons}}_k\) by convention.
The category is abelian, admits Jordan–Hölder decompositions by interval modules, and carries a well-defined length.
All uses of abelianity, Serre subcategories, and exact localizations are made within \(\Pers^{\mathrm{cons}}_k\); see Appendix~A.
For filtered complexes we keep the finite-type hypothesis and record that filtered colimits may leave finiteness; cf.\ Remark~\ref{rem:filtered-colimits-finite-type}.
In what follows, the phrase “finite-type’’ is used synonymously with “constructible’’ in this sense.
Filtered (co)limits, when used, are computed objectwise in \([\mathbb{R},\mathsf{Vect}_k]\) and then verified to return to \(\Pers^{\mathrm{cons}}_k\) (Appendix~A); no claim is made outside this regime.

\begin{remark}[Constructible abelian setting]\label{rem:constructible-abelian}
Within \(\Pers^{\mathrm{cons}}_k\), kernels and cokernels are computed pointwise and preserve finiteness, hence \(\Pers^{\mathrm{cons}}_k\) is abelian with interval decompositions and Serre localizations by bar length (Appendix~A). For each \(\tau>0\), the full subcategory \(\mathsf{E}_\tau\) generated by interval modules of length \(\le \tau\) is a hereditary Serre (localizing) subcategory in this constructible \(1\)D setting; see Crawley--Boevey (2015); Chazal--de~Silva--Glisse--Oudot (2016).
\end{remark}

\begin{remark}[Windowed proof policy (MECE) in the concrete model]\label{rem:ch2-mece}
All statements in this chapter are meant to be applied \emph{per window} (Chapter~1, Def.~1.0).
A \emph{domain window} is a half-open interval \([u_k,u_{k+1})\) (right-inclusive).
A windowing is \emph{MECE} if \(\bigsqcup_k [u_k,u_{k+1})=[u_0,U)\), intervals are ordered and adjacent windows meet only in their endpoints (right-inclusive convention).
Two mandatory coverage checks are recorded in each run (Appendix~G): (i) \(\sum_k (u_{k+1}-u_k)=U-u_0\) (total length equals the sum of window lengths), (ii) the number of events (births/deaths counted with multiplicity) on \([u_0,U)\) equals the sum of per-window counts up to rounding tolerance.
Collapse thresholds and spectral bins are \emph{fixed on each window} and used only \emph{after collapse} at the persistence layer.
\end{remark}

\subsection*{2.2. The thresholded collapse functor \texorpdfstring{$C_\tau$}{C\_\texttau}}
Fix a threshold \(\tau>0\). We first define a truncation on persistence modules and then lift it to filtered complexes.

\paragraph{Ephemeral part and localization (constructible $1$D case).}
Let \(\mathsf{E}_\tau\subset \Pers^{\mathrm{cons}}_k\) be the Serre subcategory generated by \emph{interval modules} of length \(\le \tau\).
The reflective localization
\[
\mathbf{T}_\tau:\ \Pers^{\mathrm{cons}}_k\longrightarrow (\mathsf{E}_\tau)^{\perp}
\]
is \emph{exact in the constructible $1$D range} and \(1\)-Lipschitz for the interleaving distance \(d_{\mathrm{int}}\) (Appendix~A). Here \((\mathsf{E}_\tau)^{\perp}\) denotes the \emph{$\tau$-local} (orthogonal) subcategory.%
\footnote{In general, “$\tau$-local’’ (orthogonal to \(\mathsf{E}_\tau\) for both \(\Hom\) and \(\Ext^1\)) is finer than “$\tau$-torsion-free’’. We consistently use the $\tau$-local (orthogonal) target \((\mathsf{E}_\tau)^{\perp}\).}
We interpret “deleting bars of length \(\le\tau\)” via this exact localization: for any \(M\in\Pers^{\mathrm{cons}}_k\),
\[
\mathbf{T}_\tau(M)=M/E_\tau(M),\qquad E_\tau(M)\ \text{ the maximal \(\tau\)-ephemeral subobject.}
\]

\begin{remark}[Endpoints and infinite bars; Chapter~2]\label{rk:2-endpoints}
\(\mathbf{T}_\tau\) deletes precisely the \emph{finite} bars of length \(\le\tau\); infinite bars are invariant. The choice of open/closed endpoint conventions does not affect \(\mathbf{T}_\tau\), since interleaving-equivalence classes (hence bar lengths) are preserved (Appendix~A; see also Appendix~A, Remark~\ref{A:rk:endpoints}).
\end{remark}

\paragraph{Lifting to filtered complexes.}
Choose once and for all a functor
\[
\mathcal{U}:\ \Pers^{\mathrm{cons}}_k\longrightarrow \mathsf{FiltCh}(k)
\]
such that \(\mathbf{P}_i\big(\mathcal{U}(M)\big)\cong M\) for every \(M\) concentrated in degree \(i\), and \(\mathcal{U}\) sends finite direct sums of degree-concentrated modules to finite direct sums of elementary filtered complexes built by gluing interval complexes.%
\footnote{Such a \(\mathcal{U}\) is canonical up to filtered quasi-isomorphism and preserves finite limits/colimits up to filtered quasi-isomorphism. See, e.g., W.~Crawley--Boevey (2015) and F.~Chazal–V.~de~Silva–S.~Oudot (2016) for constructible decompositions underpinning such realizations; Appendix~B records the lifting–coherence hypotheses used in this paper.}
For \(F\in \mathsf{FiltCh}(k)\) define \(C_\tau(F)\) as any object of \(\mathsf{FiltCh}(k)\) equipped with filtered quasi-isomorphisms in each degree \(i\)
\[
\mathbf{P}_i\big(C_\tau(F)\big)\ \xrightarrow{\ \cong\ }\ \mathbf{T}_\tau\!\big(\mathbf{P}_i(F)\big).
\]
Such a choice exists and is unique up to filtered quasi-isomorphism; we call any functorial choice (up to filtered quasi-isomorphism) a \emph{thresholded collapse functor} and continue to write \(C_\tau:\mathsf{FiltCh}(k)\to\mathsf{FiltCh}(k)\).

\subsection*{2.3. Stability and basic calculus of \texorpdfstring{$C_\tau$}{C\_\texttau}}
We record the properties needed later; all statements are with respect to equality in \(\Pers^{\mathrm{cons}}_k\) and to filtered quasi-isomorphism in \(\mathsf{FiltCh}(k)\).

\begin{lemma}[Shift–commutation and \(1\)-Lipschitz]\label{lem:shift}
Let \(S^\varepsilon\) denote the \(\varepsilon\)-shift on persistence modules. Then
\[
\mathbf{T}_\tau\circ S^\varepsilon\ \cong\ S^\varepsilon\circ \mathbf{T}_\tau.
\]
Consequently, \(\mathbf{T}_\tau\) preserves \(\varepsilon\)-interleavings and is \(1\)-Lipschitz for the interleaving (equivalently, bottleneck) distance \(d_{\mathrm{int}}\). (Cf.\ the standard interleaving stability argument; Appendix~A.)
\end{lemma}

\begin{proposition}[Stability, calculus, and (co)limit compatibility]\label{prop:stability}
Let \(\tau,\sigma>0\) and \(F,G\in\mathsf{FiltCh}(k)\).
\begin{enumerate}
  \item \textup{(Non-expansiveness)} For every \(i\),
  \[
  d_{\mathrm{int}}\!\big(\mathbf{P}_i(C_\tau F),\ \mathbf{P}_i(C_\tau G)\big)\ \le\ d_{\mathrm{int}}\!\big(\mathbf{P}_i(F),\ \mathbf{P}_i(G)\big).
  \]
  \item \textup{(Monotonicity and idempotence)} If \(\tau\le\sigma\), then for every \(i\) there is a natural epimorphism
  \[
  \mathbf{P}_i(C_\tau F)\ \twoheadrightarrow\ \mathbf{P}_i(C_\sigma F),
  \]
  and \(C_\tau\circ C_\sigma\ \simeq\ C_{\max\{\tau,\sigma\}}\ \simeq\ C_\sigma\circ C_\tau\) up to filtered quasi-isomorphism \emph{(equivalently, \(\mathbf{T}_\tau\circ \mathbf{T}_\sigma=\mathbf{T}_{\max\{\tau,\sigma\}}\) at the persistence layer)}.
  \item \textup{(Exactness on persistence via localization)} The subcategory \(\mathsf{E}_\tau\subset \Pers^{\mathrm{cons}}_k\) of \(\tau\)-ephemeral modules is hereditary Serre (localizing). Interpreting \(\mathbf{T}_\tau\) as the localization functor
  \[
  \Pers^{\mathrm{cons}}_k\ \longrightarrow\ \Pers^{\mathrm{cons}}_k/\mathsf{E}_\tau
  \]
  composed with the canonical identification of the quotient with its reflective subcategory of \(\tau\)-local (orthogonal) objects, \(\mathbf{T}_\tau\) is exact and preserves finite limits and colimits. \emph{Pointwise exactness in \(\mathsf{Vect}_k\) yields, for each \(t\), a long exact sequence; assembling in \(t\) gives a natural long exact sequence of persistence modules.} Consequently, after applying \(\mathbf{P}_\ast\) \emph{degreewise} to a short exact sequence of filtered complexes, exactness is preserved under \(\mathbf{T}_\tau\) \emph{at the persistence layer}. No claim of short exactness for \(\mathbf{P}_i\circ C_\tau\) is made in general.
  \item \textup{(Filtered colimits; \textbf{[Spec]})} If \(\{F_\lambda\}_{\lambda\in\Lambda}\) is a filtered diagram in \(\mathsf{FiltCh}(k)\) whose colimit is computed degreewise on chains and filtrations, then for every \(i\),
  \[
  \mathbf{P}_i\!\big(C_\tau(\varinjlim_\Lambda F_\lambda)\big)\ \cong\ \varinjlim_\Lambda\, \mathbf{P}_i\!\big(C_\tau(F_\lambda)\big),
  \]
  hence, in the implementable range and up to filtered quasi-isomorphism, \(C_\tau\) preserves filtered colimits. \textbf{Equality is asserted at the persistence layer only.}
  \item \textup{(Finite pullbacks; \textbf{[Spec]})} If finite pullbacks in \(\mathsf{FiltCh}(k)\) are computed degreewise and the chosen \(\mathcal{U}\) preserves finite limits up to filtered quasi-isomorphism, then for every pullback square \(F\times_H G\) one has
  \[
  \mathbf{P}_i\!\big(C_\tau(F\times_H G)\big)\ \cong\ \mathbf{P}_i\!\big(C_\tau(F)\times_{\,C_\tau(H)} C_\tau(G)\big),
  \]
  so, in the implementable range and up to filtered quasi-isomorphism, \(C_\tau\) preserves finite pullbacks. \textbf{Equality is asserted at the persistence layer only.}
\end{enumerate}
\end{proposition}

\begin{remark}[Scope of limit/pullback compatibilities]
Compatibilities of \(C_\tau\) with finite pullbacks and filtered colimits are used at the
persistence layer via \(\mathbf{T}_\tau\circ \mathbf{P}_i\).
\textbf{Statements at the filtered-complex level are adopted as [Spec] and hold up to filtered quasi-isomorphism} under the lifting–coherence hypothesis; see Appendix~B.
\end{remark}

\begin{remark}[Monotonicity vs.\ stability; deletion- vs.\ inclusion-type]\label{rk:monotone-stable}
For \emph{deletion-type} updates satisfying Appendix~E (Dirichlet restriction/absorbing boundaries, principal submatrices/Schur complements, positive-semidefinite Loewner contractions, conservative averaging), windowed persistence energies and spectral tails/heat traces computed on \(L(C_\tau F)\) are \emph{non-increasing after truncation}. For \emph{inclusion-type} updates we claim only \emph{stability} (non-expansiveness). Spectral indicators are not f.q.i.-invariants; they are controlled by a fixed policy \((\beta,M(\tau),s)\) as detailed in Chapter~11 (see also Appendix~E for sufficient conditions and counterexamples).
\end{remark}

\begin{declaration}[Spec–\texorpdfstring{$C_\tau$}{C\_\texttau} calculus]\label{spec:Ctau-calc}
The filtered lift \(C_\tau\) realizes \(\mathbf{T}_\tau\) degreewise \emph{up to filtered quasi-isomorphism}.
Compatibility with filtered colimits and finite pullbacks is used \emph{at the persistence layer} via
\(\mathbf{P}_i\!\circ C_\tau\simeq \mathbf{T}_\tau\!\circ \mathbf{P}_i\) (Appendix~B).
\textbf{No stronger claim is made at the filtered-complex level.}
\end{declaration}

\subsection*{2.4. Windowing (MECE), $\tau$-adaptation, and spectral bins}
We formalize the operating policies required for windowed certificates and for the interaction with spectral auxiliaries (Chapter~11).

\begin{definition}[Domain windows (MECE) and coverage checks]\label{def:ch2-mece}
A \emph{domain windowing} is a finite or countable family of half-open intervals \(\{[u_k,u_{k+1})\}_k\) with right-inclusion, such that \(\bigsqcup_k [u_k,u_{k+1})=[u_0,U)\) (disjoint union of intervals) and \(u_k<u_{k+1}\) for all \(k\).
We enforce the coverage checks:
\[
\sum_k (u_{k+1}-u_k)=U-u_0,\qquad\n\#\mathrm{Events}([u_0,U))=\sum_k \#\mathrm{Events}([u_k,u_{k+1}))\ \ (\pm\ \text{rounding}).
\]
Here \(\#\mathrm{Events}\) counts births and deaths (with multiplicity) observed by \(\mathbf{P}_i\) on the specified window. All verdicts are issued per window, and global statements are obtained by pasting certificates along the windowing as in Chapter~4.
\end{definition}

\begin{definition}[Collapse thresholds and $\tau$-adaptation]\label{def:tau-adapt}
Fix a collapse threshold \(\tau>0\) on each domain window. The default adaptation ties \(\tau\) to the numerical/filtration resolution:
\[
\tau\ =\ \alpha\cdot \max\{\Delta t,\Delta x\}\qquad (\alpha>0\ \text{fixed per run}),
\]
or any comparable rule documented in the run manifest (Appendix~G). A \emph{$\tau$-sweep} is a discrete set \(\{\tau_\ell\}\) on which \((\mu_{\mathrm{Collapse}},\nu_{\mathrm{Collapse}})\) and B-Gate\(^{+}\) are evaluated. A \emph{stable band} is a contiguous range of \(\tau\)-values on which \((\mu_{\mathrm{Collapse}},\nu_{\mathrm{Collapse}})=(0,0)\) holds (Appendix~J); windowed certificates are issued within stable bands.
\end{definition}

\begin{definition}[Spectral bins and auxiliary bars (aux-bars)]\label{def:aux-bars}
For a spectral operator with (ascending) spectrum \((\lambda_m)_{m\ge 1}\), fix a bin width \(\beta>0\) and a range \([a,b]\). Define half-open bins \(I_r=[a+r\beta,a+(r+1)\beta)\) (right-inclusion) for \(r=0,1,\dots,\lfloor (b-a)/\beta\rfloor-1\), and count occupancies \(E_r:=\#\{m\mid \lambda_m\in I_r\}\). Underflow \(\{\lambda_m<a\}\) and overflow \(\{\lambda_m\ge b\}\) are recorded explicitly to ensure coverage. Along a discrete index (e.g.\ step number), the sequences \((E_r(j))_j\) define \emph{auxiliary spectral bars (aux-bars)} as maximal consecutive runs on which \(E_r(j)>0\); their \emph{lifetimes} are measured in the discrete index (rescaled when needed). After applying \(C_\tau\) on the persistence side, the indicators derived from aux-bars are evaluated as \emph{auxiliary gauges}: \emph{monotone} under deletion-type steps, \emph{stable} under \(\varepsilon\)-continuations (Appendix~E). They never replace the B-side gate.
\end{definition}

\begin{remark}[Separation of roles: persistence vs.\ spectral auxiliaries]\label{rem:sep-persist-spectral}
Gate decisions are made on the persistence layer after collapse, never on spectral auxiliaries alone. Aux-bars are used as supportive evidence (e.g.\ “aux-bars=0 after $C_\tau$ on the spectral window’’), with their binning policy \((\beta,[a,b])\) recorded in the manifest. This maintains the invariance guarantees of \(\mathbf{T}_\tau\) and the reproducibility of windowed certificates.
\end{remark}

\subsection*{2.5. Collapse Admissibility (v15.0)}
Let \(\mathcal{R}:\mathsf{FiltCh}(k)\to D^{\mathrm{b}}(k\text{-mod})\) be a fixed \(t\)-exact realization functor of amplitude \(\le 1\), and fix the minimal test family \(\mathcal{Q}=\{k[0]\}\) (cf.\ Chapter~1 and Appendix~C).

\begin{definition}[Admissibility predicate (v15.0)]
For \(F\in\mathsf{FiltCh}(k)\) set
\[
\texttt{CollapseAdmissible}(F)\ :\!\!\iff\ \mathrm{PH}_1(F)=0\ \ \wedge\ \ \mathrm{Ext}^1\!\big(\mathcal{R}(F),k\big)=0.
\]
Under the bridging assumptions (B1)–(B3) of Chapter~3 (in particular, $t$-exactness and amplitude $\le 1$), \(\mathrm{PH}_1(F)=0\Rightarrow \mathrm{Ext}^1(\mathcal{R}(F),k)=0\) in \(D^{\mathrm{b}}(k\text{-mod})\).
\end{definition}

\begin{definition}[Robust admissibility at scale \texorpdfstring{$\varepsilon$}{\(\varepsilon\)}]
Fix a tolerance \(\varepsilon>0\). We say \(F\) is \emph{\(\varepsilon\)-robustly collapse-admissible} if
\[
\mathrm{PH}_1\!\big(C_\varepsilon(F)\big)=0\quad\text{and}\quad \mathrm{Ext}^1\!\big(\mathcal{R}(C_\varepsilon(F)),k\big)=0.
\]
This weakens the binary gate by discarding bars shorter than \(\varepsilon\). The choice of \(\varepsilon\) reflects the admissible noise level for applications.
\end{definition}

\begin{remark}[Usage and guarantees]
The predicate separates objects with vanishing first persistent homology and vanishing tested \(\mathrm{Ext}^1\) from those retaining obstructions. By Proposition~\ref{prop:stability}, \(C_\tau\) provides a metrically stable simplification compatible with filtered colimits and finite pullbacks (up to filtered quasi-isomorphism), furnishing a robust pre-processing step toward admissibility checks. In practice, one fixes an application-level noise scale \(\varepsilon\) and tests admissibility on \(C_\varepsilon(F)\); the one-way bridge then applies verbatim in \(D^{\mathrm{b}}(k\text{-mod})\). No global equivalence \(\mathrm{PH}_1\Leftrightarrow \mathrm{Ext}^1\) is asserted in this work; the one-way implication under explicit bridging hypotheses is established later.
\end{remark}

\begin{remark}[Formalizability]
The ingredients of §2 are directly formalizable: (i) \(\mathsf{E}_\tau\) is a localizing (hereditary Serre) subcategory in the constructible range; (ii) the reflector \(\mathbf{T}_\tau\) is exact and preserves finite limits/colimits; (iii) the shift–commutation Lemma~\ref{lem:shift} implies \(1\)-Lipschitz continuity in the interleaving metric; and (iv) Proposition~\ref{prop:stability} encodes the (co)limit and pullback compatibility. These specifications can be encoded in Coq/Lean as axiomatized interfaces for the persistence layer and lifted to filtered complexes (Appendices~A–B).
\end{remark}

\begin{lemma}[\texorpdfstring{$\varepsilon$}{\(\varepsilon\)}-survival under interleaving perturbations]
If \(d_{\mathrm{int}}(\mathbf{P}_i(F),\mathbf{P}_i(G))\le \varepsilon\), then for every bar \(b\) of \(\mathbf{P}_i(F)\)
and every \(\tau_0>0\), letting \(\ell_{\tau_0}(b)\) denote the \([0,\tau_0]\)-clipped length of \(b\), the following holds:
if \(\ell_{\tau_0}(b)>2\varepsilon\), then \(b\) has a corresponding bar in \(\mathbf{T}_{\tau_0}(\mathbf{P}_i(G))\)
whose \([0,\tau_0]\)-clipped length is at least \(\ell_{\tau_0}(b)-2\varepsilon\).\footnote{The “corresponding bar’’ arises from the standard matching used in the interleaving stability proof; see Appendix~I.}
\end{lemma}
\noindent Proof in Appendix~I.

\begin{remark}[Operating summary for Chapter~2]\label{rem:ch2-summary}
On each domain window \([u_k,u_{k+1})\), fix a collapse threshold \(\tau\) (adapted to resolution as in Definition~\ref{def:tau-adapt}) and, if spectral auxiliaries are used, fix a bin policy \((\beta,[a,b])\) (Definition~\ref{def:aux-bars}). Apply any A-side step (deletion-type or \(\varepsilon\)-continuation), then collapse \(C_\tau\), and \emph{measure} solely on the B-side \(\mathbf{T}_\tau\mathbf{P}_i\). Record the \(\delta\)-ledger for non-commutation and verify the B-Gate\(^{+}\) conditions (Chapter~1). Issue a windowed certificate when B-Gate\(^{+}\) passes. Global claims are obtained by pasting certificates along the MECE partition using Restart and Summability (Chapter~4).
\end{remark}



% ===========================
% Chapter 3: A One-Way Bridge PH$_1\!\to$ Ext$^1$ and the Hypothesis Scheme (reinforced)
% ===========================
\section{Chapter 3: A One-Way Bridge \texorpdfstring{$\mathrm{PH}_1\!\Rightarrow\!\mathrm{Ext}^1$}{PH1⇒Ext1} and the Hypothesis Scheme}
\addcontentsline{toc}{section}{A One-Way Bridge $\mathrm{PH}_1\Rightarrow\mathrm{Ext}^1$ and the Hypothesis Scheme}

\noindent\textbf{Scope note (reinforced windowed policy).}
All statements of this chapter lie within the constructible $1$D regime of Chapter~2 and use field coefficients.
The implication $\mathrm{PH}_1\Rightarrow \Ext^1$ is \emph{proved only in} $D^{\mathrm{b}}(k\text{-mod})$ under (B1)–(B3).
Every claim is issued \emph{per domain window} (half-open, right-inclusive; Chapter~1, Def.~1.0 and Chapter~2, Remark~\ref{rem:ch2-mece}); gate decisions are taken \emph{only} on the B-side after collapse, i.e.\ on single-layer objects $\mathbf{T}_\tau\mathbf{P}_i$ (equivalently $\mathbf{P}_i(C_\tau -)$).
Filtered-complex equalities hold only up to filtered quasi-isomorphism (f.q.i.).

\subsection*{3.0. Windowed usage and gate integration}
This chapter provides the \emph{PH$_1\!\Rightarrow\!\Ext^1$} bridge used by B-Gate$^{+}$ (Chapter~1, Def.~1.0) \emph{after} applying the collapse $C_\tau$ on a fixed window $W=[u,u')$ and a fixed threshold $\tau>0$:
\begin{itemize}
  \item On $(W,\tau)$, if $\mathrm{PH}_1(C_\tau F)=0$, then (under (B1)–(B3)) $\Ext^1(\mathcal{R}(C_\tau F),k)=0$.
  \item Together with $(\mu_{\mathrm{Collapse}},\nu_{\mathrm{Collapse}})=(0,0)$ at $(W,\tau)$ and the safety margin $\mathrm{gap}_\tau>\Sigma\delta$, this yields B-Gate$^{+}$ passing and thus a \emph{windowed certificate} on $(W,\tau)$ (Chapter~1, Def.~1.0).
  \item Outside the scope of (B1)–(B3), $\Ext^1$ is \emph{not} used; B-Gate$^{+}$ may be evaluated with PH and tower-only parts (cf.\ Chapter~1).
\end{itemize}
All cross-domain comparisons (PF/BC, Mirror/Transfer) are performed \emph{after collapse}, and any non-commutation is recorded in the $\delta$-ledger (Appendix~L).

\subsection*{3.1. Bridging Hypotheses (B1–B3)}
We fix the notation and conventions of Chapter~2. In particular, \(k\) is a field, \(\mathsf{FiltCh}(k)\) denotes finite-type (constructible) filtered chain complexes, \(\mathbf{P}_i(F)\) is the degree-\(i\) persistence module with barcode \(\mathrm{PH}_i(F)\), and \(\mathcal{R}:\mathsf{FiltCh}(k)\to D^{\mathrm{b}}(k\text{-mod})\) is a fixed $t$-exact realization functor.\footnote{We work in cohomological amplitude \([-1,0]\): amplitude control and $t$-exactness place $\mathcal{R}(F)$ in $D^{[-1,0]}$ and identify the edge map; see Appendix~C.}

\begin{description}
\item[\normalfont (B1) Finite-type over a field.]
\(F\in\mathsf{FiltCh}(k)\) with pointwise finite-dimensional persistence. Filtered \emph{(co)limits} of constructible persistence modules are computed objectwise in \([\mathbb{R},\mathsf{Vect}_k]\) and used only under the scope policy of Appendix~A (compute in the functor category and verify return to \(\Pers^{\mathrm{cons}}_k\)); no claim is made outside this regime. Equalities are asserted only at the persistence layer.

\item[\normalfont (B2) Amplitude \(\boldsymbol{\le 1}\) and identification of the \(\boldsymbol{H^{-1}}\)-edge.]
There is a natural isomorphism
\[
H^{-1}\!\big(\mathcal{R}(F)\big)\ \cong\ \varinjlim_{t}\ H_1(F^{t}C_\bullet),
\]
and \(\mathcal{R}(F)\in D^{[-1,0]}(k\text{-mod})\).
Equivalently, \(\mathcal{R}(F)\) admits a two-term model
\[
\mathcal{R}(F)\ \simeq\ \bigl[\, \varinjlim_{t} H_1(F^{t}C_\bullet)\ \xrightarrow{\,d\,}\ \varinjlim_{t} H_0(F^{t}C_\bullet)\, \bigr],
\]
concentrated in cohomological degrees \((-1,0)\), functorial in \(F\).
Here exactness of filtered colimits in \(\mathsf{Vect}_k\) and the \(t\)-exactness of \(\mathcal{R}\) ensure functoriality in \(F\); all statements are confined to \(D^{[-1,0]}(k\text{-mod})\) (see Appendix~C).

\item[\normalfont (B3) Edge identification for degree \(1\) with \(Q=k\).]
For any \(A\in D^{[-1,0]}(k\text{-mod})\),
\[
\Ext^1(A,k)\ \cong\ \Hom\!\big(H^{-1}(A),k\big),
\]
naturally in \(A\), by applying \(\mathbf{R}\!\operatorname{Hom}(-,k)\) to the truncation triangle \(\tau_{\le -1}A\to A\to \tau_{\ge 0}A\to\) and using that \(k\) is a field; cf.\ Appendix~C.
\end{description}

\begin{remark}[On (B2) and the edge identification]\label{rk:B2-edge}
We use a two-term realization with amplitude \(\le 1\) so that \(H^{-1}(\mathcal{R}(F))\cong \varinjlim_t H_1(F^tC_\bullet)\), relying on exactness of filtered colimits in \(\mathsf{Vect}_k\). Over a field, for \(A\in D^{[-1,0]}\) one has \(\Ext^1(A,k)\cong \Hom(H^{-1}(A),k)\). Proof details and naturality are in Appendix~C. \emph{The bridge of this chapter is proved only in \(D^{\mathrm{b}}(k\text{-mod})\). Uses of other coefficient fields (e.g.\ Novikov) appear only as \textbf{[Spec]} and do not extend the proven bridge.}
\end{remark}

\begin{remark}[Outside the bridge domain]\label{rk:outside-bridge}
Derived geometric/sheaf or symplectic/Floer realizations may be employed as \textbf{[Spec]} (Appendix~N/O), but the implication \(\mathrm{PH}_1\Rightarrow\Ext^1\) is \emph{not} asserted in those targets. All proofs here remain in \(D^{\mathrm{b}}(k\text{-mod})\).
\end{remark}

\subsection*{3.2. The Bridge \texorpdfstring{$\mathrm{PH}_1\!\Rightarrow\!\mathrm{Ext}^1$}{PH1⇒Ext1}}
Recall that \(\mathrm{PH}_1(F)=0\) means the degree-\(1\) persistence module vanishes; equivalently, \(H_1(F^tC_\bullet)=0\) for all \(t\).

\begin{theorem}[One-way bridge]\label{thm:PH1-to-Ext1}
Assume \textup{(B1)–(B3)}. If \(\mathrm{PH}_1(F)=0\), then
\[
\Ext^1\!\big(\mathcal{R}(F),k\big)\ =\ 0.
\]
\end{theorem}

\noindent\emph{Equivalently}, under \textup{(B2)}–\textup{(B3)} the conclusion reads \(H^{-1}\!\big(\mathcal{R}(F)\big)=0\).

\begin{proof}
\(\mathrm{PH}_1(F)=0\) implies \(H_1(F^tC_\bullet)=0\) for all \(t\), hence \(\varinjlim_t H_1(F^tC_\bullet)=0\). By \textup{(B2)}, \(H^{-1}(\mathcal{R}(F))\cong \varinjlim_t H_1(F^tC_\bullet)=0\). By \textup{(B3)}, \(\Ext^1(\mathcal{R}(F),k)\cong \Hom(H^{-1}(\mathcal{R}(F)),k)=0\). All steps are natural in \(F\).
\end{proof}

\begin{corollary}[Robust bridge at scale \(\varepsilon\)]\label{cor:robust-bridge}
For any \(\varepsilon>0\), if \(\mathrm{PH}_1\!\big(C_\varepsilon(F)\big)=0\), then
\[
\Ext^1\!\big(\mathcal{R}(C_\varepsilon(F)),k\big)\ =\ 0.
\]
\textbf{In particular, robust admissibility is always tested after truncation} (apply $C_\varepsilon$ first).
All identifications are natural in \(F\) and compatible with morphisms in \(\mathsf{FiltCh}(k)\).
\end{corollary}

\begin{proof}
Apply Theorem~\ref{thm:PH1-to-Ext1} to \(C_\varepsilon(F)\); the hypotheses are preserved by \(C_\varepsilon\) (Chapter~2). This uses that \(C_\varepsilon\) preserves constructibility and that, under the lifting–coherence hypothesis, the \(t\)-exact realization \(\mathcal{R}\) keeps amplitude \(\le 1\) (Appendix~B; cf.\ Chapter~2, §§2.2–2.3).
\end{proof}

\subsection*{3.3. Naturality, stability, and windowed gate usage}
\begin{proposition}[Naturality of the edge isomorphisms (B2)–(B3)]
Under \textup{(B1)–(B3)}, the edge isomorphisms in \textup{(B2)} and \textup{(B3)} are natural in \(F\). For any morphism \(f:F\to G\) in \(\mathsf{FiltCh}(k)\) the diagram
\[
\begin{tikzcd}\n\rH^{-1}\!\big(\mathcal{R}(F)\big) \arrow[r,"\sim"] \arrow[d,"\rH^{-1}(\mathcal{R}(f))"'] &\n\varinjlim_t \,\rH_1(F^tC_\bullet) \arrow[d,"\varinjlim_t\, \rH_1(f^t)"] \\\n\rH^{-1}\!\big(\mathcal{R}(G)\big) \arrow[r,"\sim"] &\n\varinjlim_t \,\rH_1(G^tC_\bullet)\n\end{tikzcd}
\]
commutes, and the identifications in \textup{(B3)} are functorial in \(A\) and compatible with morphisms in \(D^{[-1,0]}\).
\end{proposition}

\begin{remark}[Stability via thresholded collapse and B-Gate$^{+}$]
By Chapter~2, \(\mathbf{T}_\tau\) is \(1\)-Lipschitz for the interleaving distance \(d_{\mathrm{int}}\) and compatible with filtered colimits at the persistence layer (up to filtered quasi-isomorphism; Appendix~B). Hence the premise \(\mathrm{PH}_1(C_\varepsilon(F))=0\) is metrically stable under interleaving perturbations, and the conclusion of Corollary~\ref{cor:robust-bridge} is invariant under functorial choices of \(C_\varepsilon\). On a window \((W,\tau)\), this yields a robust method to discharge the Ext-part of B-Gate$^{+}$ once the PH-part is verified on the collapsed object.
\end{remark}

\begin{remark}[Degree specificity]
The bridge in this chapter concerns degree \(1\) only; no claim is made for \((\mathrm{PH}_i \Rightarrow \Ext^i)\) with \(i\neq 1\).
\end{remark}

\subsection*{3.4. Scope and limitations}
The bridge is strictly one-way: no claim is made that \(\Ext^1(\mathcal{R}(F),k)=0\) implies \(\mathrm{PH}_1(F)=0\).
Failure modes (e.g.\ invisible obstructions beyond the amplitude window) and counterexamples to the converse are documented in Appendix~D, including subsection \textbf{D.4} (\emph{Counterexamples to the converse}). Type~IV/tower artifacts are detected by \((\mu_{\mathrm{Collapse}},\nu_{\mathrm{Collapse}})\); see Appendix~D.1–D.3 and Remark~\ref{rem:D-generic-dim} for the generic-fiber (multiplicity of \(I[0,\infty)\)) interpretation.

\subsection*{3.5. Windowed saturation and domain-restricted triggers \textup{[Spec]}}
The following \textbf{[Spec]} items record useful strengthening \emph{on fixed windows} that may apply in restricted domains; they are \emph{not} part of the proved bridge.

\begin{declaration}[Saturation gate: local equivalence \textup{[Spec]}]\label{dec:saturation-local}
On a saturation window (Chapter~11), where the event set stabilizes, the inter-window drift is $\le\eta$, and the edge gap exceeds $\eta$, it is \textbf{[Spec]}-admissible to adopt a temporary local equivalence:
\[
\mathrm{PH}_1(C_{\tau^\ast}F)=0\quad\Longleftrightarrow\quad \Ext^1\!\big(\mathcal{R}(C_{\tau^\ast}F),k\big)=0.
\]
This policy is \emph{window-local} and must be explicitly logged; it does not extend the global one-way bridge outside the saturation conditions.
\end{declaration}

\begin{declaration}[Trigger pack (domain-restricted) \textup{[Spec]}]\label{dec:trigger-pack}
In restricted regimes (e.g.\ 2D flows, small-data critical spaces, simplified arithmetic models), one may prepare \emph{trigger lemmas} asserting that analytic deviations imply failure of B-Gate$^{+}$ on the window, e.g.:
\[
\text{(blow-up/instability on $W$)}\ \Rightarrow\ \bigl(\mathrm{PH}_1(C_\tau F)>0\ \ \text{or}\ \ \mu_{\mathrm{Collapse}}>0\ \ \text{or}\ \ \mathrm{aux\!-\!bars}>0\bigr).
\]
Such triggers (when available) supply \emph{partial necessity}, improving the diagnostic power of the gate. They must be stated with explicit assumptions and logged as \textbf{[Spec]}.
\end{declaration}

\subsection*{3.6. Interaction with PF/BC, Mirror, and the $\delta$-ledger}
PF/BC isomorphisms are applied \emph{per $t$}, transported to the persistence layer, and then to the collapsed layer (Appendix~N).
Mirror/Transfer comparisons are performed only after $C_\tau$, and any non-commutation with collapse is recorded in the $\delta$-ledger (Appendix~L).
The Ext-part of B-Gate$^{+}$ is always checked \emph{on the collapsed object} and only within the $t$-exact, amplitude $\le 1$ scope. No $\Ext\Rightarrow\mathrm{PH}$ claim is made.

\subsection*{3.7. Formalizability}
Hypotheses \textup{(B1)–(B3)} and Theorem~\ref{thm:PH1-to-Ext1} admit direct formalization: \textup{(B2)} via an explicit two-term model for \(\mathcal{R}(F)\) and exactness of filtered colimits in \(\mathsf{Vect}_k\); \textup{(B3)} via truncation in \(D^{[-1,0]}\) and the edge of the long exact sequence. Skeletons and lemma names are listed in Appendix~F for Coq/Lean integration. Windowed usage (B-Gate$^{+}$), the MECE policy, and the $\delta$-ledger are recorded as operational axioms in the formalization stubs (Appendix~F), with proofs confined to the persistence layer and the derived category of $k$-modules.



% ===========================
% Chapter 4: Failure Lattice and the Tower-Sensitivity Invariant μ_Collapse  (reinforced)
% ===========================
\section{Chapter 4: Failure Lattice and the Tower-Sensitivity Invariant \texorpdfstring{$\mu_{\mathrm{Collapse}}$}{mu\_Collapse}}
\addcontentsline{toc}{section}{Failure Lattice and the Tower-Sensitivity Invariant $\mu_{\mathrm{Collapse}}$}

\paragraph{Standing hypotheses and scope.}
We work in the constructible (finite-type) range of persistence (Chapter~2, §2.1), adopt the bridging hypotheses \textup{(B1)–(B3)} from Chapter~3 with the minimal test family $\mathcal{Q}=\{k[0]\}$, and fix a $t$-exact realization $\mathcal{R}:\mathsf{FiltCh}(k)\to D^{\mathrm{b}}(k\text{-mod})$. All uses of filtered (co)limits are asserted \emph{at the persistence layer} (Appendix~A). The endpoint convention and the treatment of infinite bars are as in Chapter~2, Remark~\ref{rk:2-endpoints}. Monotonicity claims for indicators apply only to \emph{deletion-type} updates; inclusion-type updates are \emph{stability-only} (Appendix~E). Every claim is \emph{windowed} (Chapter~1, Def.~1.0; Chapter~2, §2.4), and all gate decisions are taken \emph{only} on the B-side after collapse (single layer $\mathbf{T}_\tau\mathbf{P}_i$).

\medskip
\noindent\emph{(B2) (edge identification, recall).}
There is a natural isomorphism $H^{-1}(\mathcal{R}(F))\cong \varinjlim_t H_1(F^tC_\bullet)$ and $\mathcal{R}(F)\in D^{[-1,0]}$; see Appendix~C.

\subsection*{4.1. Failure lattice and observable vs.\ invisible modes}
We organize collapse failures as follows (cf.\ Chapter~1, §1.4):

\begin{itemize}
  \item \textbf{Type I (Topological):} $\mathrm{PH}_1(F)\neq 0$.
  \item \textbf{Type II (Categorical):} $\Ext^1\!\big(\mathcal{R}(F),k\big)\neq 0$ (with $\mathcal{Q}=\{k[0]\}$).
  \item \textbf{Type III (Functorial/[Spec]):} admissibility unstable under a prescribed operation (e.g.\ a given pullback/filtered colimit) at finite level.
  \item \textbf{Type IV (Invisible/tower-level):} all finite layers appear admissible while the limit is not; detected by the tower-sensitivity invariants below.
\end{itemize}

Types~I–II are \emph{observable} on a single object; Type~III is \emph{specification-level}; Type~IV is \emph{invisible} at finite layers and requires tower diagnostics. We write $I[a,b)$ for the interval module (constructible, p.f.d.) supported on $[a,b)$; endpoint choices do not affect lengths (Chapter~2, Remark~\ref{rk:2-endpoints}).

\begin{remark}[Specification-level failures and functorial calculus]
Type~III is analyzed using Chapter~2, §2.3: non-expansiveness, shift–commutation (Lemma~\ref{lem:shift}), and the persistence-layer (co)limit/pullback compatibilities in Proposition~\ref{prop:stability}\,(4)(5). Statements at the filtered-complex layer are \emph{adopted as} \textbf{[Spec]} and used only up to filtered quasi-isomorphism (Appendix~B).
\end{remark}

\begin{remark}[Model towers]
Pure-kernel, pure-cokernel, and mixed toy towers, together with the vanishing regime under constructible filtered colimits, are illustrated in Appendix~D (D.1–D.3). Counterexamples to the converse $\Ext^1{=}0 \Rightarrow \mathrm{PH}_1{=}0$ are in D.4.
\end{remark}

\subsection*{4.2. The invariants \texorpdfstring{$\mu_{\mathrm{Collapse}}$}{mu\_Collapse} and \texorpdfstring{$\nu_{\mathrm{Collapse}}$}{nu\_Collapse}}
\emph{Recall (generic fiber dimension).} As in Chapter~1, §1.4 and Appendix~D, Remark~\ref{rem:D-generic-dim}, for a constructible module $M$ the \emph{generic fiber dimension} is $\mathrm{gdim}(M)=\lim_{t\to+\infty}\dim_k M(t)$; after truncation by $\mathbf{T}_\tau$, it equals the multiplicity of the infinite bar $I[0,\infty)$ in the barcode.

\medskip
Fix $\tau>0$. Let $\{F_n\}_{n\in\mathbb{N}}$ be a directed system in $\mathsf{FiltCh}(k)$ with colimit $F_\infty$. For each degree $i$ consider the comparison map of truncated persistence modules
\[
\phi_{i,\tau}:\ \varinjlim_{n}\ \mathbf{T}_\tau\!\big(\mathbf{P}_i(F_n)\big)\ \longrightarrow\ \mathbf{T}_\tau\!\big(\mathbf{P}_i(F_\infty)\big),
\]
where $\mathbf{T}_\tau$ is the exact reflector deleting all bars of length $\le\tau$ (Chapter~2, §2.2) and $\mathbf{P}_i$ is degreewise persistence.

\begin{definition}[Tower-sensitivity invariants at scale $\tau$]\label{def:mu-nu}
For each $i$ set
\[
\mu_{i,\tau}\ :=\ \dim_k \ker(\phi_{i,\tau}),\qquad\n\nu_{i,\tau}\ :=\ \dim_k \mathrm{coker}(\phi_{i,\tau}),
\]
and define the totals
\[
\muc\ :=\ \sum_i \mu_{i,\tau},\qquad\n\nuc\ :=\ \sum_i \nu_{i,\tau}.
\]
\emph{Here $\dim_k$ denotes the \emph{generic fiber} dimension after truncation, i.e.\ the multiplicity of $I[0,\infty)$ summands in the barcode of the indicated kernel/cokernel (Appendix~D, Remark~\ref{rem:D-generic-dim}).} Since complexes are bounded in homological degrees and barcodes are constructible, both sums are finite on bounded $\tau$-windows.
\end{definition}

\noindent\emph{Notation.} We suppress the explicit $\tau$–dependence of $\muc,\nuc$ when the scale is clear from context.

\begin{proposition}[Generic dimension equals infinite-bar multiplicity]\label{prop:generic-dimension-barcode}
In the constructible range, for any morphism $\psi:M\to N$ in $\Pers^{\mathrm{cons}}_k$ the generic fiber dimension
$\dim_k\ker(\psi)$ (resp.\ $\dim_k\mathrm{coker}(\psi)$) equals the multiplicity of $I[0,\infty)$ in the barcode of $\ker(\psi)$ (resp.\ $\mathrm{coker}(\psi)$). The same holds after truncation by $\mathbf{T}_\tau$.
\end{proposition}

\noindent\emph{Proof sketch.} Kernels and cokernels are computed pointwise and stay constructible (Chapter~2, Remark~\ref{rem:constructible-abelian}), so their barcodes are well-defined. Interval Jordan–Hölder theory identifies the generic rank with the $I[0,\infty)$ multiplicity; truncation preserves $I[0,\infty)$ and removes only finite bars (Chapter~2, §2.2). Details are in Appendix~D (Remark~\ref{rem:D-generic-dim}).

\begin{remark}[Well-definedness, invariance, and scale behavior]
The maps $\phi_{i,\tau}$ are natural in the tower, independent of filtered representatives, and invariant under cofinal reindexing (Appendix~J). Consequently, $\mu_{i,\tau}$ and $\nu_{i,\tau}$ are invariant under filtered quasi-isomorphisms. \textbf{As $\tau$ varies, no general monotonicity of $\mu_{i,\tau}$ or $\nu_{i,\tau}$ is asserted;} one monitors them across a discrete sweep of $\tau$ for scale robustness.
\end{remark}

\begin{proposition}[Vanishing under constructible filtered colimits]\label{prop:mu-vanishing}
Assume the colimit of the tower $\{F_n\}$ is computed \emph{objectwise on chains and filtrations} and that all $\mathbf{P}_i(F_n)$ remain in $\Pers^{\mathrm{cons}}_k$. Then for every $\tau>0$ and every $i$ the map $\phi_{i,\tau}$ is an isomorphism, hence
\[
\muc=\nuc=0.
\]
\end{proposition}

\begin{proof}
In $[\mathbb{R},\mathsf{Vect}_k]$, filtered colimits are exact, giving $\mathbf{P}_i(F_\infty)\cong \varinjlim_n \mathbf{P}_i(F_n)$. Since $\mathbf{T}_\tau$ is the exact reflective localization at the Serre subcategory generated by length-$\le\tau$ intervals, it preserves filtered colimits. Therefore $\mathbf{T}_\tau(\mathbf{P}_i(F_\infty))\cong \varinjlim_n \mathbf{T}_\tau(\mathbf{P}_i(F_n))$ and $\phi_{i,\tau}$ is an isomorphism. Equalities are asserted \emph{at the persistence layer}; see Appendix~A for the scope policy on returning to $\Pers^{\mathrm{cons}}_k$.
\end{proof}

\noindent A basic calculus (subadditivity under composition, additivity under finite sums, cofinal invariance) is collected in Appendix~J.

\subsection*{4.3. Type IV: finite admissibility need not pass to the limit}
The following precise form of the invisible failure principle works already at the persistence layer and then lifts.

\begin{proposition}[Type IV: finite-level admissibility may fail at the limit]\label{prop:type4}
In the filtered index category $\mathbb{N}\cup\{\infty\}$ with cone apex $\infty$, there exists a tower $\{F_n\}$ and $\tau>0$ such that
\[
\forall n:\ \ \mathrm{PH}_1\!\big(C_\tau(F_n)\big)=0\quad\text{and}\quad \Ext^1\!\big(\mathcal{R}(C_\tau(F_n)),k\big)=0,
\]
yet
\[
\mathrm{PH}_1\!\big(C_\tau(F_\infty)\big)\neq 0\quad\text{and thus}\quad \Ext^1\!\big(\mathcal{R}(C_\tau(F_\infty)),k\big)\neq 0.
\]
Moreover one can arrange $\muc=0$ and $\nuc>0$ (pure cokernel-type mismatch).
\end{proposition}

\begin{proof}[Proof (persistence model; then lift)]
Fix $\tau>0$ and set $\ell_n=\tau-\frac{1}{n}\uparrow\tau$. Let $M_n:=I[0,\ell_n)$ with structure maps $M_n\hookrightarrow M_{n+1}$. Then $\mathbf{T}_\tau(M_n)=0$ for all $n$, while $\varinjlim_n M_n\cong I[0,\tau)$. Cone the diagram to the apex module $N:=I[0,\infty)$ via $M_n\hookrightarrow I[0,\tau)\hookrightarrow N$, so the colimit of the extended diagram is $N$. Choose $F_n,F_\infty$ with $\mathbf{P}_1(F_n)\cong M_n$, $\mathbf{P}_1(F_\infty)\cong N$ (Chapter~2, §2.2). Then $\mathbf{T}_\tau(\mathbf{P}_1(F_n))=0$ while $\mathbf{T}_\tau(\mathbf{P}_1(F_\infty))\cong I[0,\infty)$. Thus $\mathrm{PH}_1(C_\tau(F_n))=0$ for all $n$, but $\mathrm{PH}_1(C_\tau(F_\infty))\neq 0$. By \textup{(B2)}–\textup{(B3)} this forces $\Ext^1\!\big(\mathcal{R}(C_\tau(F_\infty)),k\big)\neq 0$. Here $\ker(\phi_{1,\tau})=0$ and $\mathrm{coker}(\phi_{1,\tau})\neq 0$, so $\muc=0$ and $\nuc>0$.
\end{proof}

\begin{figure}[t]
\centering
\begin{tikzcd}[row sep=1.0em, column sep=2.0em]
I[0,\tau-\tfrac{1}{1}) \arrow[r, hook] \arrow[dr] &
I[0,\tau-\tfrac{1}{2}) \arrow[r, hook] 
& \cdots \arrow[r, hook] \arrow[dr] &
I[0,\tau-\tfrac{1}{n}) \arrow[r, hook] 
& \cdots \arrow[r, hook] \arrow[dr] &
I[0,\tau) \arrow[r, hook] \arrow[dr] &
I[0,\infty) \\\n& \mathbf{T}_\tau(\cdot)=0 & & \mathbf{T}_\tau(\cdot)=0 & &
\mathbf{T}_\tau(\cdot)=0 & \mathbf{T}_\tau(\cdot)=I[0,\infty)
\end{tikzcd}
\caption{Type~IV intuition (after truncation by $\mathbf{T}_\tau$): finite layers vanish, while the apex produces an infinite bar.}
\label{fig:typeIV-intuition}
\end{figure}

\subsection*{4.4. A natural refinement-limit example (pure cokernel type)}
\begin{example}[Resolution refinement producing a limit infinite bar]
Fix $\tau>0$. Consider a mesh-refinement sequence of scalar fields whose degree-$1$ persistence at level $n$ has a single bar $[0,\tau-\delta_n)$ with $\delta_n\downarrow 0$. Each finite layer satisfies $\mathbf{T}_\tau(\mathbf{P}_1(F_n))=0$. In the refinement limit the coherent structure becomes persistent across all thresholds, yielding an infinite bar in degree $1$: $\mathbf{T}_\tau(\mathbf{P}_1(F_\infty))\cong I[0,\infty)$. This is the natural analogue of Proposition~\ref{prop:type4} in resolution limits (see Appendix~D for diagrams). It realizes a \emph{pure cokernel} Type~IV failure ($\muc=0$, $\nuc>0$).
\end{example}

\begin{remark}[When invisible failure is excluded]
If the tower satisfies the hypotheses of Proposition~\ref{prop:mu-vanishing} (and Proposition~\ref{prop:stability}\,(4) in Chapter~2), then $\phi_{i,\tau}$ is an isomorphism for all $i,\tau$, so no Type~IV occurs: for every $\tau>0$,
\[
\bigl(\forall n:\ \mathrm{PH}_1(C_\tau(F_n))=0\bigr)\ \Longrightarrow\ \mathrm{PH}_1(C_\tau(F_\infty))=0,
\]
and, under \textup{(B1)–(B3)}, the corresponding $\Ext^1$-vanishing propagates along the tower.
\end{remark}

\subsection*{4.5. Restart lemma, summability, and pasting of windowed certificates}
We formalize the pasting principle used to obtain global conclusions from windowed certificates.

\begin{definition}[Per-window safety margin and pipeline budget]\label{def:window-budget}
Let $\{[u_k,u_{k+1})\}_k$ be a MECE partition of the monitored range (Chapter~2, Def.~\ref{def:ch2-mece}). On each window $W_k=[u_k,u_{k+1})$, fix a collapse threshold $\tau_k>0$ and define the \emph{per-window pipeline budget}
\[
\Sigma\delta_k(i)\ :=\ \sum_{j\in J_k} \bigl(\delta^{\mathrm{alg}}_{j}(i,\tau_k)+\delta^{\mathrm{disc}}_{j}(i,\tau_k)+\delta^{\mathrm{meas}}_{j}(i,\tau_k)\bigr),
\]
where $J_k$ indexes the A-side steps applied before the B-side gate on $W_k$. The \emph{per-window safety margin} $\mathrm{gap}_{\tau_k}>0$ is the admissible slack chosen for the gate on $W_k$ (Chapter~1, Def.~1.0).
\end{definition}

\begin{lemma}[Restart lemma (window-to-window inheritance)]\label{lem:restart}
Fix a degree $i$. Suppose that on $W_k$ the B-Gate$^{+}$ passes with safety margin $\mathrm{gap}_{\tau_k}>\Sigma\delta_k(i)$, and that the next window $W_{k+1}$ is reached via a finite number of admissible A-side steps, each either deletion-type or an $\varepsilon$-continuation, followed by collapse $C_{\tau_{k+1}}$ with $\tau_{k+1}$ chosen by an adaptation rule (Chapter~2, Def.~\ref{def:tau-adapt}). Then there exists $\kappa\in (0,1]$, depending only on the admissible step class and the adaptation policy, such that the safety margin on $W_{k+1}$ satisfies
\[
\mathrm{gap}_{\tau_{k+1}}\ \ge\ \kappa\ \bigl(\mathrm{gap}_{\tau_k}-\Sigma\delta_k(i)\bigr).
\]
In particular, if $\mathrm{gap}_{\tau_k}-\Sigma\delta_k(i)>0$, then $\mathrm{gap}_{\tau_{k+1}}>0$ and B-Gate$^{+}$ can pass on $W_{k+1}$ provided the new budget $\Sigma\delta_{k+1}(i)$ is sufficiently small.
\end{lemma}

\noindent\emph{Proof sketch.} Deletion-type steps are non-increasing after collapse (Chapter~2, Remark~\ref{rk:monotone-stable}); $\varepsilon$-continuations are $1$-Lipschitz hence introduce a controlled drift proportional to the declared $\varepsilon$. The adaptation of $\tau$ preserves the scale of the clipped indicators. Aggregating the drifts yields a multiplicative retention $\kappa$ of the effective margin.

\begin{definition}[Summability policy]\label{def:summability}
A run satisfies the \emph{summability policy} if the sequence of per-window budgets on a MECE partition obeys
\[
\sum_k \Sigma\delta_k(i)\ <\ \infty
\]
for the degrees $i$ on which gates are evaluated. A sufficient design pattern is a geometric decay of step sizes (e.g.\ $\tau_k=\tau_0\rho^k$, $\beta_k=\beta_0\rho^k$ with $\rho\in (0,1)$) and/or a geometric damping of the number and strength of $\varepsilon$-continuations per window, all recorded in the manifest (Appendix~G).
\end{definition}

\begin{theorem}[Pasting windowed certificates]\label{thm:pasting}
Let $\{[u_k,u_{k+1})\}_k$ be a MECE partition, and suppose that on each window $W_k$ the B-Gate$^{+}$ passes with $\mathrm{gap}_{\tau_k}>\Sigma\delta_k(i)$. If the summability policy holds (Definition~\ref{def:summability}) and the restart lemma (Lemma~\ref{lem:restart}) is applicable at each transition, then the concatenation of windowed certificates yields a global certificate on $\bigcup_k [u_k,u_{k+1})$, i.e.\ no failure occurs with no false negative on the union of windows for the monitored degrees $i$.
\end{theorem}

\noindent\emph{Proof sketch.} By Lemma~\ref{lem:restart}, positive safety margin propagates to the next window with a controlled retention factor. Summability of drifts ensures that the cumulative loss of margin remains bounded and does not exhaust the initial slack. The tower audit per window ensures $(\mu,\nu)=(0,0)$ at each scale and excludes Type~IV; the MECE coverage check (Chapter~2, Def.~\ref{def:ch2-mece}) guarantees the union covers the monitored range without gaps or double counting.

\subsection*{4.6. Stable bands and $\tau$-sweeps}
We record the $\tau$-selection policy used to stabilize the tower audit.

\begin{definition}[Stable band]\label{def:stable-band}
For a fixed window $W$ and degree $i$, a \emph{stable band} is a contiguous range $B\subset (0,\infty)$ such that for all $\tau\in B$ the comparison maps $\phi_{i,\tau}$ are isomorphisms and hence $(\mu_{i,\tau},\nu_{i,\tau})=(0,0)$. A \emph{$\tau$-sweep} is a discrete set $\{\tau_\ell\}$ used to probe $(\mu_{i,\tau_\ell},\nu_{i,\tau_\ell})$; a band is declared stable when the sweep detects $(0,0)$ on a consecutive subarray and the outcome persists under refinement of the sweep.
\end{definition}

\begin{remark}[Using stable bands]
Windowed certificates are issued within stable bands only. Outside a stable band, the collapse threshold or the pipeline must be redesigned, or the window must be refined. This stabilizes the tower audit across steps and prevents spurious Type~IV readings due to poor scale selection.
\end{remark}

\subsection*{4.7. Summary}
The failure lattice separates observable (Types~I–II), specification-level (Type~III), and invisible tower effects (Type~IV). The invariants $(\muc,\nuc)$ are principled tower diagnostics: they \emph{vanish} under constructible filtered colimits (Proposition~\ref{prop:mu-vanishing}) and, when positive, certify instability of persistence under passage to limits. Toy and natural refinement towers show that finite-level admissibility does not generally imply limit-level admissibility, even with stable truncation $C_\tau$ (used \emph{up to f.q.i.}). The restart lemma (Lemma~\ref{lem:restart}) and the summability policy (Definition~\ref{def:summability}) provide a robust pasting principle (Theorem~\ref{thm:pasting}) for \emph{windowed certificates}; stable bands (Definition~\ref{def:stable-band}) guide the $\tau$-selection for reliable tower audits. A calculus for $(\mu,\nu)$ (subadditivity, additivity, cofinal invariance) is recorded in Appendix~J, and counterexamples to converse statements appear in Appendix~D (see especially Remark~\ref{rem:D-generic-dim} and D.4). These invariants are \emph{not} the classical Iwasawa $\mu,\lambda$ invariants.



% ===========================
% Chapter 5: Functoriality, Set-Theoretic Coherence, and Formalization Specifications (Proof/Spec) — reinforced
% ===========================
\section{Chapter 5: Functoriality, Set-Theoretic Coherence, and Formalization Specifications (Proof/Spec)}
\addcontentsline{toc}{section}{Functoriality, Set-Theoretic Coherence, and Formalization Specifications (Proof/Spec)}

All adjunction and (co)limit statements in this chapter are made in the \emph{implementable range} and inside
$\mathrm{Ho}(\mathsf{FiltCh}(k))$, \emph{up to filtered quasi\hyp isomorphism} (Appendix~B). Equalities are asserted \emph{at the persistence layer}. We retain the standing conventions of Chapters~1–4: constructible range, field coefficients, $t$-exact realization, and the after\hyp truncation policy. Endpoint conventions and infinite bars are as in Chapter~2, Remark~\ref{rk:2-endpoints}. \emph{Monotonicity claims apply only to deletion\hyp type updates; inclusion\hyp type updates are stability\hyp only} (Appendix~E). All statements are \emph{windowed} and gates are evaluated \emph{after collapse} (B-side single layer).

\subsection*{5.1 Exactness, (Co)Limit Behavior, and a Right\hyp Adjoint Collapse in the Implementable Range (up to f.q.i.)}

Let $k$ be a field. Recall: $\mathsf{FiltCh}(k)$ is the category of finite\hyp type (constructible) filtered chain complexes; $\mathbf{P}_i$ is degreewise persistence; $\mathbf{T}_\tau$ is the exact truncation deleting all bars of length $\le\tau$ (Chapter~2, §2.2); $C_\tau$ is any filtered lift of $\mathbf{T}_\tau$ (Chapter~2, §§2.2–2.3; always \emph{up to f.q.i.}); and $\mathcal{R}:\mathsf{FiltCh}(k)\to D^{\mathrm{b}}(k\text{-mod})$ is $t$-exact. We also keep the minimal test family $\mathcal{Q}=\{k[0]\}$.

\paragraph{Persistence\hyp level (reflective) adjunction.}
Let $\mathsf{Pers}^{\mathrm{cons}}_k$ be the abelian category of constructible $k$-persistence modules and $\mathsf{Pers}^{\mathrm{cons}}_{k,\tau\text{-tf}}\subset \mathsf{Pers}^{\mathrm{cons}}_k$ the full subcategory of $\tau$-torsion-free objects (no composition factors of length $\le\tau$). As in Chapter~2, §§2.2–2.3:
\begin{itemize}
  \item $\mathsf{E}_\tau\subset \mathsf{Pers}^{\mathrm{cons}}_k$ (generated by interval modules of length $\le\tau$) is hereditary Serre (localizing).
  \item The reflector
  \[
  \mathbf{T}_\tau:\ \mathsf{Pers}^{\mathrm{cons}}_k\longrightarrow \mathsf{Pers}^{\mathrm{cons}}_{k,\tau\text{-tf}}
  \]
  is exact and exhibits a \emph{reflective} adjunction $\mathbf{T}_\tau\dashv \iota_\tau$ with the inclusion
  $\iota_\tau:\mathsf{Pers}^{\mathrm{cons}}_{k,\tau\text{-tf}}\hookrightarrow \mathsf{Pers}^{\mathrm{cons}}_k$.
  Consequently, $\mathbf{T}_\tau$ preserves finite limits and colimits and is $1$-Lipschitz for $d_{\mathrm{int}}$
  (Lemma~\ref{lem:shift}, Proposition~\ref{prop:stability}(1),(3)). \emph{As $\mathbf{T}_\tau$ is exact (left and right exact) in this abelian setting, it preserves both finite limits and finite colimits.}
\end{itemize}

\paragraph{Filtered\hyp complex level (operational coreflection; implementable range, up to f.q.i.).}
Define the full subcategory
\[
\mathsf{S}_\tau\ :=\ \Big\{\,F\in \mathsf{FiltCh}(k)\ \Big|\ \forall i,\ \mathbf{P}_i(F)\ \text{is $\tau$-torsion-free and }\ \Ext^1\!\big(\mathcal{R}(F),k\big)=0\,\Big\},
\]
and let $\mathsf{S}_\tau^{\mathrm{h}}$ be its image in $\mathrm{Ho}(\mathsf{FiltCh}(k))$.
For later comparison, recall the “trivial-at-$\tau$” gate
\[
\mathsf{Triv}_\tau\ :=\ \Big\{\,F\ \Big|\ \mathrm{PH}_1\big(C_\tau(F)\big)=0\ \ \text{and}\ \ \mathrm{Ext}^1\!\big(\mathcal{R}(F),k\big)=0\,\Big\},\n\qquad\n\mathsf{Triv}_\tau^{\mathrm{h}}\subset \mathsf{S}_\tau^{\mathrm{h}}.
\]

\begin{proposition}[Operational collapse as a right adjoint (implementable range; up to f.q.i.; Proof/Spec)]\label{prop:operational-coreflection}
Assume \textup{(B1)–(B3)} (Chapter~3) and the lifting–coherence hypothesis (Appendix~B).
Then there exists a functor
\[
\mathsf{C}_\tau^{\mathrm{comb}}:\ \mathrm{Ho}(\mathsf{FiltCh}(k))\longrightarrow \mathsf{S}_\tau^{\mathrm{h}}
\]
and a natural transformation $\eta:\mathrm{Id}\Rightarrow \iota\,\mathsf{C}_\tau^{\mathrm{comb}}$ (with $\iota:\mathsf{S}_\tau^{\mathrm{h}}\hookrightarrow \mathrm{Ho}(\mathsf{FiltCh}(k))$ the inclusion) such that, \emph{in this regime},
\begin{enumerate}
  \item \textup{(Adjunction)} $\mathsf{C}_\tau^{\mathrm{comb}}$ is right adjoint to $\iota$:
  \[
  \mathrm{Hom}\!\big(\iota(G),F\big)\ \cong\ \mathrm{Hom}\!\big(G,\mathsf{C}_\tau^{\mathrm{comb}}(F)\big)\qquad(G\in\mathsf{S}_\tau^{\mathrm{h}}).
  \]
  \item \textup{(Compatibility; persistence layer)} For each $i$, $\mathbf{P}_i\!\big(\mathsf{C}_\tau^{\mathrm{comb}}(F)\big)\cong \mathbf{T}_\tau\!\big(\mathbf{P}_i(F)\big)$ in $\Pers^{\mathrm{cons}}_k$.
  \item \textup{(Compatibility; realization layer)} $\mathcal{R}\!\big(\mathsf{C}_\tau^{\mathrm{comb}}(F)\big)\cong \tau_{\ge 0}\,\mathcal{R}(F)$ in $D^{\mathrm{b}}(k\text{-mod})$.
  \item \textup{(Soundness)} $\mathsf{C}_\tau^{\mathrm{comb}}(F)\in \mathsf{S}_\tau^{\mathrm{h}}$ for all $F$, and
  \[
  \mathsf{C}_\tau^{\mathrm{comb}}(F)\in \mathsf{Triv}_\tau^{\mathrm{h}}\n  \quad\Longleftrightarrow\quad\n  \mathbf{T}_\tau\!\big(\mathbf{P}_1(F)\big)=0\n  \ \ \big(\text{i.e.\ } \mathrm{PH}_1(C_\tau(F))=0\big).
  \]
\end{enumerate}
\emph{\textbf{All equalities above are asserted at the persistence layer, and all filtered-complex statements are in $\mathrm{Ho}(\mathsf{FiltCh}(k))$ up to f.q.i. only.}}
\end{proposition}

\begin{proof}[Proof sketch]
At persistence level, use $\mathbf{T}_\tau$ (implemented by its lift $C_\tau$) so that
$\mathbf{P}_i(C_\tau F)\cong \mathbf{T}_\tau(\mathbf{P}_i F)$.
At realization level, use the coreflection $\tau_{\ge 0}$ to enforce amplitude $\le 1$,
under which $\Ext^1(-,k)=0\iff H^{-1}(-)=0$ (Chapter~3, (B2)–(B3)).
The lifting–coherence hypothesis furnishes functorial comparison maps
$\mathcal{R}\circ C_\tau \Rightarrow \tau_{\ge 0}\circ \mathcal{R}$ (up to f.q.i.),
allowing a cartesian assembly into $\mathsf{C}_\tau^{\mathrm{comb}}$ valued in $\mathsf{S}_\tau^{\mathrm{h}}$.
The unit $\eta$ and the universal property of the right orthogonal to $k[1]$
(i.e.\ objects $A$ with $\Hom(A,k[1])=0$) yield the adjunction within this implementable range;
triangle identities hold up to f.q.i. in $\mathrm{Ho}(\mathsf{FiltCh}(k))$.
\end{proof}

\begin{proposition}[Operational coreflection, \textbf{[Spec]} (implementable range; up to f.q.i.)]\label{prop:operational-coreflection-spec}
Within the implementable range (Appendix~B), use the explicit recipe
\[
F\longmapsto \widehat F\quad\text{with}\quad\n\mathbf{P}_i(\widehat F)\cong \mathbf{T}_\tau(\mathbf{P}_iF),\qquad\n\mathcal{R}(\widehat F)\simeq \tau_{\ge 0}\mathcal{R}(F).
\]
Treat $F\mapsto \widehat F$ as a right adjoint to $\mathsf{S}_\tau^{\mathrm{h}}\hookrightarrow \mathrm{Ho}(\mathsf{FiltCh}(k))$
\emph{in this regime}. \emph{\textbf{Equalities are asserted at the persistence layer; filtered-complex identities are taken in $\mathrm{Ho}$ up to f.q.i.}} No general existence is claimed beyond the implementable range.
\end{proposition}

\begin{corollary}[Limit/(co)limit behavior of $\mathsf{C}_\tau^{\mathrm{comb}}$ (persistence layer)]
Under Proposition~\ref{prop:stability}(4),(5), $\mathsf{C}_\tau^{\mathrm{comb}}$ preserves finite limits
(in particular, finite pullbacks) \emph{up to filtered quasi\hyp isomorphism} at the filtered\hyp complex level; at the
\emph{persistence level} one has, for every degree $i$,
\[
\mathbf{P}_i\!\big(\mathsf{C}_\tau^{\mathrm{comb}}(\varinjlim_\Lambda F_\lambda)\big)\ \cong\ \n\varinjlim_\Lambda\, \mathbf{P}_i\!\big(\mathsf{C}_\tau^{\mathrm{comb}}(F_\lambda)\big).
\]
Hence $\mathsf{C}_\tau^{\mathrm{comb}}$ is $1$-Lipschitz at the persistence layer and inherits exactness via $\mathbf{T}_\tau$ (Chapter~2, §2.3). \emph{\textbf{All equalities are stated at the persistence layer; no metric or exactness statement is made in $\mathrm{Ho}(\mathsf{FiltCh}(k))$ beyond up to f.q.i. compatibility.}} No claim is made about preservation of filtered colimits in $\mathrm{Ho}(\mathsf{FiltCh}(k))$ beyond this persistence\hyp level compatibility.
\end{corollary}

\begin{remark}[Scope of colimit claims]
Since $\tau_{\ge 0}$ is a right adjoint, it need not commute with filtered colimits; all colimit statements are therefore restricted to the \textbf{persistence layer} (via $\mathbf{T}_\tau$ and $\mathbf{P}_i$).
\end{remark}

\subsection*{5.2 \texorpdfstring{[Spec]}{[Spec]} Coq/Lean Contracts: Stability, (Co)Limits, Bridge, and $\delta$-Commutation}

\noindent\emph{Identifiers are indicative}; concrete names may follow local conventions (e.g.\ Lean/mathlib namespaces). Appendix~F lists one possible naming scheme. \emph{\textbf{All equalities are asserted at the persistence layer; formal objects at the filtered level are considered in $\mathrm{Ho}$ up to f.q.i.}} 

\begin{specification}[Persistence truncation]\label{spec:pers-Ttau}
\begin{itemize}
  \item \texttt{pers\_Ttau\_exact}: For every short exact $0\to M'\to M\to M''\to 0$ in $\mathsf{Pers}^{\mathrm{cons}}_k$, the sequence $0\to \mathbf{T}_\tau(M')\to \mathbf{T}_\tau(M)\to \mathbf{T}_\tau(M'')\to 0$ is exact.
  \item \texttt{pers\_Ttau\_lipschitz}: $d_{\mathrm{int}}\!\big(\mathbf{T}_\tau M,\mathbf{T}_\tau N\big)\le d_{\mathrm{int}}(M,N)$.
  \item \texttt{pers\_Ttau\_pres\_colim\_pullback}: $\mathbf{T}_\tau$ preserves filtered colimits and finite limits \emph{(in the constructible range)}.
  \item \texttt{pers\_Ttau\_compose}: $\mathbf{T}_\tau\circ \mathbf{T}_\sigma=\mathbf{T}_{\max\{\tau,\sigma\}}$.
\end{itemize}
\end{specification}

\begin{specification}[Filtered\hyp complex level]\label{spec:filtered-level}
Equalities are stated at the persistence layer via $\mathbf{P}_i$; at the filtered\hyp complex level they hold up to filtered quasi\hyp isomorphism under the lifting–coherence hypothesis.
\begin{itemize}
  \item \texttt{Ctau\_lift}: For each $i$, $\mathbf{P}_i(C_\tau F)\cong \mathbf{T}_\tau(\mathbf{P}_i(F))$.
  \item \texttt{Ctau\_colim}: $\mathbf{P}_i\big(C_\tau(\varinjlim_\Lambda F_\lambda)\big)\cong \varinjlim_\Lambda \mathbf{P}_i(C_\tau(F_\lambda))$.
  \item \texttt{Ctau\_pullback}: $\mathbf{P}_i\big(C_\tau(F\times_H G)\big)\cong \mathbf{P}_i\big(C_\tau(F)\times_{C_\tau(H)}C_\tau(G)\big)$.
\end{itemize}
\end{specification}

\begin{specification}[$\delta$-ledger and quantitative commutation]\label{spec:delta-ledger}
\begin{itemize}
  \item \texttt{delta\_2cell\_mirror\_collapse}: For each $i$ and $\tau$, a natural $2$-cell $\Mirror\circ C_\tau \Rightarrow C_\tau\circ \Mirror$ with a uniform bound $\delta(i,\tau)\ge 0$ in $d_{\mathrm{int}}$.
  \item \texttt{delta\_pipeline\_additivity}: For a pipeline of steps indexed by $j$, $d_{\mathrm{int}}(\text{lhs},\text{rhs})\le \sum_j \delta_j(i,\tau_j)$ (additive budget).
  \item \texttt{delta\_lipschitz\_post}: Post\hyp processing by $1$-Lipschitz persistence maps (e.g.\ $\mathbf{T}_\tau,\mathbf{P}_i,S^\varepsilon$) does not increase the bound.
  \item \texttt{torsion\_nest\_commute}: If torsions $T_A,T_B$ satisfy $E_A\subseteq E_B$ (nested Serre subcategories), then $T_A\circ T_B=T_B\circ T_A=T_{A\vee B}$.
  \item \texttt{soft\_commuting\_policy}: Define $\Delta_{\mathrm{comm}}:=d_{\mathrm{int}}(T_A T_B M,T_B T_A M)$. If $\Delta_{\mathrm{comm}}\le \eta$, accept \emph{soft\hyp commuting}; else fall back to a fixed order and record $\Delta_{\mathrm{comm}}$ into $\delta^{\mathrm{alg}}$.
\end{itemize}
\end{specification}

\begin{specification}[Bridge and admissibility]\label{spec:bridge}
\begin{itemize}
  \item \texttt{PH1\_to\_Ext1\_under\_B}: Under (B1)–(B3) with $\mathcal{Q}=\{k\}$, if $\mathrm{PH}_1(F)=0$ then $\mathrm{Ext}^1(\mathcal{R}(F),k)=0$ (Theorem~\ref{thm:PH1-to-Ext1}).
  \item \texttt{admissible\_robust\_eps}: If $\mathrm{PH}_1(C_\varepsilon F)=0$ then $\mathrm{Ext}^1(\mathcal{R}(C_\varepsilon F),k)=0$.
\end{itemize}
\end{specification}

\begin{specification}[Tower diagnostics]\label{spec:mu-nu}
For a filtered diagram $\{F_\lambda\}$ with limit $F$ and each degree $i$,
\begin{itemize}
  \item \texttt{mu\_def}: $\mu^{\,i}=\dim_k\ker\!\big(\varinjlim_\lambda \mathbf{T}_\tau(\mathbf{P}_i(F_\lambda))\to \mathbf{T}_\tau(\mathbf{P}_i(F))\big)$, and $\ \muc=\sum_i \mu^{\,i}$.
  \item \texttt{nu\_def}: $\nu^{\,i}=\dim_k\mathrm{coker}\!\big(\varinjlim_\lambda \mathbf{T}_\tau(\mathbf{P}_i(F_\lambda))\to \mathbf{T}_\tau(\mathbf{P}_i(F))\big)$, and $\ \nuc=\sum_i \nu^{\,i}$.
  \item \texttt{mu\_nu\_finite}: The sums are finite (bounded complexes).
  \item \texttt{mu\_nu\_vanish\_under\_finite\_type\_colim}: Under Proposition~\ref{prop:stability}(4) hypotheses, $\phi_i$ is an isomorphism, hence $\muc=\nuc=0$.
\end{itemize}
Here $\dim_k$ denotes the \emph{generic fiber} dimension after truncation (Appendix~D, Remark~\ref{rem:D-generic-dim}).
\end{specification}

\begin{specification}[Combined collapse coreflection]\label{spec:combined}
\begin{itemize}
  \item \texttt{Ccomb\_adjunction}: In $\mathrm{Ho}(\mathsf{FiltCh}(k))$, the inclusion $\iota:\mathsf{S}_\tau^{\mathrm{h}}\hookrightarrow \mathrm{Ho}(\mathsf{FiltCh}(k))$ admits a right adjoint $\mathsf{C}_\tau^{\mathrm{comb}}$ \emph{in the implementable range}.
  \item \texttt{Ccomb\_compat}: $\mathbf{P}_i(\mathsf{C}_\tau^{\mathrm{comb}}F)\cong \mathbf{T}_\tau(\mathbf{P}_iF)$ and $\mathcal{R}(\mathsf{C}_\tau^{\mathrm{comb}}F)\cong \tau_{\ge 0}\mathcal{R}(F)$.
  \item \texttt{Ccomb\_lipschitz\_pers}: $d_{\mathrm{int}}(\mathbf{P}_i(\mathsf{C}_\tau^{\mathrm{comb}}F),\mathbf{P}_i(\mathsf{C}_\tau^{\mathrm{comb}}G))\le d_{\mathrm{int}}(\mathbf{P}_i(F),\mathbf{P}_i(G))$.
\end{itemize}
\end{specification}

\subsection*{5.3 Minimal Foundations: ZFC and Dependent Type Theory}

\paragraph{ZFC assumptions (minimal).}
\begin{itemize}
  \item \textbf{(S1) Smallness.} All categories used are locally small. Persistence modules are functors $(\mathbb{R},\le)\to \mathsf{Vect}_k$ with p.f.d.\ fibres and finitely many critical parameters on compacts; this yields a small skeleton per object.
  \item \textbf{(S2) Abelian context.} $\mathsf{Pers}^{\mathrm{cons}}_k$ is abelian; kernels/cokernels are computed pointwise and remain finite\hyp type (Chapter~2, §2.1).
  \item \textbf{(S3) Serre localization.} For each $\tau>0$, $\mathsf{E}_\tau$ is hereditary Serre (localizing); $\mathbf{T}_\tau$ exists and is exact.
  \item \textbf{(S4) (Co)limits.} $\mathsf{Vect}_k$ has all small (co)limits; filtered colimits commute with homology; the functors $\mathbf{P}_i$ commute with filtered colimits \emph{under the scope policy of Appendix~A} (compute in the functor category and verify return to $\mathsf{Pers}^{\mathrm{cons}}_k$).
  \item \textbf{(S5) Boundedness.} All chain complexes are bounded; all sums $\sum_i(\cdot)$ are finite.
\end{itemize}

\paragraph{Dependent type theory (Coq/Lean) — minimal requirements.}
\begin{itemize}
  \item \textbf{(T1) Universes.} One universe level suffices for the small categories at hand; universe polymorphism is optional.
  \item \textbf{(T2) Exactness and localization.} Abelian categories, Serre subcategories, and exact localizations to define $\mathbf{T}_\tau$ and prove exactness.
  \item \textbf{(T3) $t$-structures.} The standard $t$-structure on $D^{\mathrm{b}}(k\text{-mod})$ with right adjoint truncation $\tau_{\ge 0}$.
  \item \textbf{(T4) Metrics on persistence.} A formal interleaving metric and non\hyp expansiveness of $\mathbf{T}_\tau$ (Lemma~\ref{lem:shift}).
  \item \textbf{(T5) (Co)limits.} Implementations of filtered colimits and finite limits and their compatibility with $\mathbf{P}_i$, $\mathbf{T}_\tau$, $C_\tau$ (Proposition~\ref{prop:stability}).
  \item \textbf{(T6) Homotopy category.} A relative\hyp category or model\hyp structure presentation of $\mathsf{FiltCh}(k)$ with weak equivalences the filtered quasi\hyp isomorphisms, to interpret $\mathrm{Ho}(\mathsf{FiltCh}(k))$ and construct right adjoints therein.
\end{itemize}

\paragraph{Set\hyp theoretic coherence of adjunctions.}
At persistence level, $\mathbf{T}_\tau\dashv \iota_\tau$ is a reflection. At realization level, $\tau_{\ge 0}$ is the right adjoint coreflection to the inclusion $D^{\ge 0}\hookrightarrow D^{\mathrm{b}}(k\text{-mod})$.
Proposition~\ref{prop:operational-coreflection} combines these into a right adjoint
$\mathsf{C}_\tau^{\mathrm{comb}}$ to the inclusion of $\mathsf{S}_\tau^{\mathrm{h}}$ in the homotopy category, respecting the stability and (co)limit properties in Chapter~2. No non\hyp canonical barcode decomposition is required, and (S1)–(S5) ensure size coherence.

\begin{remark}[Recommended practice for applications]
For scale selection, monitor $\mu_{\mathrm{Collapse}}(\tau)$ and $\nu_{\mathrm{Collapse}}(\tau)$ across a range of $\tau$'s; use Proposition~\ref{prop:stability}(4),(5) to certify regimes where tower effects vanish ($\muc=\nuc=0$). This aligns run\hyp time tests with the provable guarantees of $\mathsf{C}_\tau^{\mathrm{comb}}$.
\end{remark}

\subsection*{5.4 Declarations for External Realizations and Operational Recipe (\textbf{[Spec]})}

\begin{declaration}[Spec–Derived realizations]\label{spec:derived-geo}
We may use $\mathcal{R}_{\mathrm{coh}}:\mathsf{FiltCh}(k)\to D^{\mathrm{b}}\mathrm{Coh}(X)$ or
$\mathcal{R}_{\acute{e}t}:\mathsf{FiltCh}(k)\to D^{\mathrm{b}}_{\mathrm{c}}(X_{\acute{e}t},\Lambda)$ with \emph{field} $\Lambda$ as specifications.
Projection formula and base change are assumed as in Appendix~N \emph{(PF/BC checked per Appendix~N’s table of assumptions)}.
\textbf{The bridge $\mathrm{PH}_1\Rightarrow\Ext^1$ is proved only in $D^{\mathrm{b}}(k\text{-mod})$} (Chapter~3); these external realizations do not extend the proven bridge.
\end{declaration}

\begin{declaration}[Spec–Operational coreflection]\label{spec:operational-coreflection}
In applications we operate with the recipe
\[
F\ \longmapsto\ \big(C_\tau(F)\ \text{at persistence}\big)\quad\text{and}\quad\n\big(\tau_{\ge 0}\mathcal{R}(F)\ \text{at realization}\big).
\]
Right\hyp adjoint phrasing is used at the \textbf{[Spec]} level; coherence and limits of validity are summarized in Appendix~B.
\end{declaration}

\subsection*{5.5 $\delta$-Budget Naturalities and the Pipeline Error Budget}

\begin{definition}[Natural $2$-cell and $\delta$-ledger]\label{def:delta-2cell}
For each collapse threshold $\tau>0$ and each degree $i$, a \emph{natural $2$-cell}
\[
\epsilon_{i,\tau}:\ \Mirror\circ C_\tau\ \Rightarrow\ C_\tau\circ \Mirror
\]
is equipped with a uniform bound $\delta(i,\tau)\ge 0$ in interleaving distance:
\[
d_{\mathrm{int}}\!\Big(\mathbf{T}_\tau \mathbf{P}_i(\Mirror(C_\tau F)),\ \mathbf{T}_\tau \mathbf{P}_i(C_\tau(\Mirror F))\Big)\ \le\ \delta(i,\tau),
\]
for all $F$ in the implementable range. We decompose $\delta(i,\tau)=\delta^{\mathrm{alg}}(i,\tau)+\delta^{\mathrm{disc}}(i,\tau)+\delta^{\mathrm{meas}}(i,\tau)$ and record each component in a \emph{$\delta$-ledger}. The bound is uniform in $F$ and invariant under filtered quasi\hyp isomorphisms.
\end{definition}

\begin{proposition}[Pipeline error budget: additivity and $1$-Lipschitz post\hyp processing]\label{prop:pipeline-budget}
Let $U_m,\dots,U_1$ be A\hyp side steps with per\hyp step collapses $C_{\tau_j}$ and bounds $\delta_j(i,\tau_j)$ as in Definition~\ref{def:delta-2cell}. Then, for any fixed B\hyp side collapse threshold $\tau$,
\[
d_{\mathrm{int}}\!\Big(\n\mathbf{T}_\tau \mathbf{P}_i\big(\Mirror(C_{\tau_m}U_m\cdots C_{\tau_1}U_1F)\big),\\n\mathbf{T}_\tau \mathbf{P}_i\big(C_{\tau_m}U_m\cdots C_{\tau_1}U_1\Mirror F\big)\n\Big)\ \le\ \sum_{j=1}^m \delta_j(i,\tau_j).
\]
Moreover, any subsequent processing by maps that are $1$-Lipschitz for $d_{\mathrm{int}}$ (e.g.\ shifts $S^\varepsilon$, further truncations $\mathbf{T}_{\tau'}$, degree projections $\mathbf{P}_i$) does not increase the right\hyp hand side.
\end{proposition}

\begin{proof}[Proof sketch]
Compose the natural $2$-cells $\epsilon_{i,\tau_j}$ along the pipeline and apply the triangle inequality in $d_{\mathrm{int}}$, using that each $\mathbf{T}_\tau$ (and $S^\varepsilon$) is $1$-Lipschitz (Lemma~\ref{lem:shift}, Proposition~\ref{prop:stability}(1)).
\end{proof}

\begin{remark}[Logging policy and safety margin]
Per window $W$ and threshold $\tau$, the gate uses the \emph{safety margin} $\mathrm{gap}_\tau$ and the total budget $\Sigma\delta(i)=\sum_j \delta_j(i,\tau_j)$; acceptance requires $\mathrm{gap}_\tau>\Sigma\delta(i)$ (Chapter~1, B\hyp Gate$^+$). The full decomposition $(\delta^{\mathrm{alg}},\delta^{\mathrm{disc}},\delta^{\mathrm{meas}})$ is stored in the run manifest (Appendix~G).
\end{remark}

\subsection*{5.6 Commutable Torsion: Adoption Policy, Tests, and Order Control}

\begin{definition}[Torsion reflectors and nesting]\label{def:torsion-reflectors}
Let $T_A,T_B$ be exact reflectors on $\mathsf{Pers}^{\mathrm{cons}}_k$ obtained by localizing at hereditary Serre subcategories $E_A,E_B$ (e.g.\ length thresholds, birth\hyp window deletions). We say $T_A,T_B$ are \emph{nested} if $E_A\subseteq E_B$ or $E_B\subseteq E_A$.
\end{definition}

\begin{proposition}[Order independence under nesting]\label{prop:nested-commute}
If $T_A,T_B$ are nested, then
\[
T_A\circ T_B\ =\ T_B\circ T_A\ =\ T_{A\vee B},
\]
where $E_{A\vee B}$ is the Serre subcategory generated by $E_A\cup E_B$. In particular, for $1$D length thresholds, $\mathbf{T}_\tau\circ \mathbf{T}_\sigma=\mathbf{T}_{\max\{\tau,\sigma\}}$.
\end{proposition}

\begin{proof}[Proof sketch]
In an abelian setting, localizing at nested Serre subcategories is idempotent and order\hyp independent; the join $E_{A\vee B}$ is Serre, and the universal property of the reflector identifies both composites with $T_{A\vee B}$.
\end{proof}

\begin{definition}[Quantifying non\hyp commutation: A/B test and soft\hyp commuting]\label{def:soft-commute}
For arbitrary reflectors $T_A,T_B$, define the \emph{commutation defect} on $M\in\Pers^{\mathrm{cons}}_k$ by
\[
\Delta_{\mathrm{comm}}(M;A,B)\ :=\ d_{\mathrm{int}}\!\big(T_A T_B M,\ T_B T_A M\big).
\]
Given a tolerance $\eta\ge 0$, we declare \emph{soft\hyp commuting} if $\Delta_{\mathrm{comm}}(M;A,B)\le \eta$ on the relevant dataset. Otherwise we \emph{fall back} to a fixed order (e.g.\ $T_A$ then $T_B$) and record $\Delta_{\mathrm{comm}}$ into $\delta^{\mathrm{alg}}$ in the ledger.
\end{definition}

\begin{remark}[Multi\hyp axis torsion and scope]
When $E_A,E_B$ are not nested (e.g.\ mixing length truncation with birth\hyp window deletion or spectral\hyp policy driven deletions), we make no general commutation claim. The soft\hyp commuting A/B test provides an \emph{operational} criterion; failing it enforces a deterministic order with explicit $\delta^{\mathrm{alg}}$ accountability. This policy is mandatory outside the $1$D length\hyp threshold axis.
\end{remark}

\begin{corollary}[Compatibility with the pipeline budget]
If $T_A,T_B$ are used in a pipeline and are not nested, then the differential outcome between the two orders contributes a budget term $\Delta_{\mathrm{comm}}$ to $\delta^{\mathrm{alg}}$, and Proposition~\ref{prop:pipeline-budget} applies additively together with the other $2$-cell bounds.
\end{corollary}

\subsection*{5.7 Worked micro\hyp example (policy illustration)}
Let $T^{\mathrm{len}}_\tau$ be the length threshold reflector and $T^{\mathrm{birth}}_{[u,u')}$ the deletion of bars with birth outside a fixed domain window. These are not nested in general. Compute $\Delta_{\mathrm{comm}}(M; \mathrm{len},\mathrm{birth})$ on a sample and pick $\eta$ (e.g.\ numerical tolerance). If $\Delta_{\mathrm{comm}}\le \eta$, adopt soft\hyp commuting and treat $T^{\mathrm{len}}_\tau$ and $T^{\mathrm{birth}}_{[u,u')}$ as parallel. Otherwise, fix an order (e.g.\ apply $T^{\mathrm{birth}}$ then $T^{\mathrm{len}}$), log $\Delta_{\mathrm{comm}}$ into $\delta^{\mathrm{alg}}$, and proceed with the B\hyp Gate$^+$ evaluation on the collapsed single layer.

\subsection*{5.8 Summary}
We established a coherent functorial core for collapse within the constructible regime: a persistence\hyp level exact reflector $\mathbf{T}_\tau$, an operational right adjoint collapse in $\mathrm{Ho}$ (up to f.q.i.), and formal contracts for stability and bridge usage. The \emph{$\delta$-budget} is made \emph{natural} via $2$-cells measuring the non\hyp commutation of Mirror/Transfer with collapse, with \emph{additive} pipeline accounting and $1$-Lipschitz post\hyp processing. For torsion reflectors, \emph{order independence} is guaranteed under \emph{nesting}, while non\hyp nested cases are governed by an A/B \emph{soft\hyp commuting} policy and a deterministic fallback with explicit $\delta^{\mathrm{alg}}$ logging. These reinforced specifications ensure that all gate decisions remain on the B\hyp side after collapse, within a reproducible, windowed, and metrically stable framework that integrates seamlessly with the tower diagnostics of Chapter~4 and the realization bridge of Chapter~3.



% ===========================
% Chapter 6: Geometric Collapse (Program/Spec) — reinforced
% ===========================
\section{Chapter 6: Geometric Collapse (Program/Spec)}
\addcontentsline{toc}{section}{Geometric Collapse (Program/Spec)}

\noindent\textbf{Monotonicity policy (after truncation).} Deletion\hyp type updates are \emph{non\hyp increasing} for windowed persistence energies and spectral indicators; inclusion\hyp type updates are \emph{stability\hyp only} (non\hyp expansive). See Appendix~E for sufficient conditions and counterexamples.

\subsection*{6.0. Standing hypotheses and admissible geometric realization}
We work over a fixed field \(k\) and adopt the notation and hypotheses of Part~I. In particular,
\(\FiltCh{k}\) denotes finite\hyp type filtered chain complexes over \(k\),
\(\mathbf{P}_i:\FiltCh{k}\to\Perskft\) the degreewise persistence functor,
and we write \(\Ttau:=\mathbf{T}_\tau\) for the bar\hyp deletion (Serre) localization at scale \(\tau\ge 0\) (allowing \(\Ttau=\mathrm{Id}\) when \(\tau=0\)).
Its filtered lift \(\Ctau\) is used \emph{up to filtered quasi\hyp isomorphism} (Chapter~2, §§2.2–2.3).
The realization \(\Rfun:\FiltCh{k}\to D^{\mathrm{b}}(k\text{-mod})\) is \(t\)\hyp exact.
All statements in this chapter lie in the constructible range (we identify \(\Perskft\) with the constructible subcategory).
Unless explicitly marked \textbf{[Spec]}, \emph{equalities and Lipschitz claims are asserted only at the persistence layer}; identities at the filtered\hyp complex layer hold \emph{up to filtered quasi\hyp isomorphism}.
Kernel/cokernel diagnostics \((\muc,\nuc)\) are computed from the comparison maps
\[
  \phi_{i,\tau}:\ \varinjlim\nolimits_\lambda \Ttau\!\big(\mathbf{P}_i(F_\lambda)\big)\ \longrightarrow\ \Ttau\!\big(\mathbf{P}_i(\varinjlim\nolimits_\lambda F_\lambda)\big),
\]
with \(\dim_k\) interpreted as the \emph{generic\hyp fiber} dimension after truncation (multiplicity of \(I[0,\infty)\)); see Appendix~D, Remark~\ref{rem:D-generic-dim}.

\begin{definition}[Admissible geometric realization]\label{def:geom-real}
Let \(\mathsf{Geom}\) be a geometric input category (e.g.\ metric or metric\hyp measure spaces with \(1\)\hyp Lipschitz maps; triangulated manifolds with mesh\hyp refinement maps; weighted graphs with contraction/sparsification maps). An \emph{admissible geometric realization} is a functor
\[
  \mathcal{G}:\ \mathsf{Geom}\longrightarrow \FiltCh{k}
\]
such that:
\emph{(i)} \(\mathcal{G}\) is functorial and sends non\hyp expansive maps to filtered chain maps whose images under each \(\mathbf{P}_i\) are \(1\)\hyp Lipschitz for the interleaving distance;
\emph{(ii)} degreewise finite\hyp type is preserved;
\emph{(iii)} subsampling/refinement maps are carried to filtered maps that, for each fixed \(\tau\), induce filtered quasi\hyp isomorphisms after applying \(\Ctau\).
\end{definition}

\begin{remark}[Program posture and bridges]\label{rk:LC}
All specifications are asserted within the \emph{implementable range} of Part~I: (co)limit and stability statements are restricted to the persistence layer; the lifting–coherence hypothesis \LC\ is assumed for comparing \(\Ctau\) on \(\FiltCh{k}\) with effects after realization \(\Rfun\).
No equivalence \(\mathrm{PH}_1\Leftrightarrow\Ext^1\) is claimed; only the one\hyp way bridge under \textup{(B1)–(B3)} from Part~I is used.
The obstruction \(\muc\) is \emph{distinct} from the classical Iwasawa \(\mu\)\hyp invariant.
\end{remark}

\begin{remark}[Stability vs.\ monotonicity; spectral policy]\label{rem:stability-vs-monotonicity}
Non\hyp expansive maps ensure stability (non\hyp expansiveness) of all indicators.
Under \emph{deletion\hyp type} updates satisfying Appendix~E (Dirichlet restriction, principal submatrices/Schur complements, Loewner contractions, and—in the symplectic setting—stop additions/Liouville contractions), spectral tails and windowed energies are \emph{non\hyp increasing}. Inclusion\hyp type updates guarantee only \emph{stability}.
Spectral indicators are \emph{not} f.q.i.\ invariants; throughout we treat them as \emph{stable under a fixed normalization policy} and evaluate them on \(L(\Ctau F)\) (see Chapter~11).
\end{remark}

All monotonicity claims are interpreted after truncation by \Ttau.

\subsection*{6.1. Monitored indicators and energies}
Fix an admissible \(\mathcal{G}\) and write \(F=\mathcal{G}(X)\in \FiltCh{k}\).

\begin{definition}[Persistence energies]\label{def:PE}
Let \(\mathcal{B}_i(F)\) be the multiset of intervals of \(\mathbf{P}_i(F)\).
For \(\alpha>0\) (default \(\alpha=1\)) and a truncation window \([0,\tau]\), define
\begin{equation}\label{eq:PE-alpha-trunc}
\begin{aligned}
\mathrm{PE}_{i,\alpha}^{\le\tau}(F)
&:= \sum_{[b,d)\in \mathcal{B}_i(F)}
   \bigl(\min\{d,\tau\}-\min\{b,\tau\}\bigr)_{+}^{\alpha},\\\n(x)_{+}&:=\max\{x,0\}.
\end{aligned}
\end{equation}
By default \(\alpha=1\), and \(\mathrm{PE}^{\le\tau}(F):=\sum_i \mathrm{PE}_{i}^{\le\tau}(F)\).
\emph{All energies are evaluated on the truncated barcode: \(\mathrm{PE}_{i,\alpha}^{\le\tau}(F)=\mathrm{PE}_{i,\alpha}^{\le\tau}(\Ctau F)\) with \(\Ttau\mathbf{P}_i(F)=\mathbf{P}_i(\Ctau F)\).}
\end{definition}

\begin{definition}[Spectral indicators]\label{def:spectral}
Let \(L(\Ctau F)\) be a combinatorial Hodge Laplacian on the truncated complex \(\Ctau F\) (normalized, with the Euclidean inner product on chains). Denote the non\hyp decreasing spectrum by \((\lambda_m(\Ctau F))_{m\ge 0}\). For \(\beta>0\) and an \emph{integer} cutoff \(M(\tau)\in\mathbb{N}\), define the spectral tail
\[
  \mathrm{ST}_{\beta}^{\ge M(\tau)}(F)\ :=\ \sum_{m\ge M(\tau)} \lambda_m(\Ctau F)^{-\beta},\n  \qquad\n  \mathrm{HT}(t;F)\ :=\ \mathrm{Tr}\big(e^{-tL(\Ctau F)}\big)\ \ (t>0),
\]
with zero modes excluded (or replaced by the Moore–Penrose pseudoinverse). Qualitative specifications are invariant under these standard choices; the policy \((\beta,M(\tau),t)\) is fixed across a run (Appendix~G; Chapter~11).
\end{definition}

\begin{remark}[Convergence, parameterization, and logging]\label{rk:ST-conv}
Choose \(\beta\) and \(M(\tau)\) to ensure convergence (typical \(\beta\in\{1,2\}\), \(M(\tau)=\lfloor c\,\tau^{\gamma}\rfloor\) with \(c>0\), \(\gamma\in(0,2]\)).
When sweeping \(\tau\), take \(M(\tau)\) non\hyp decreasing to avoid artificial discontinuities.
Normalization, zero\hyp mode handling, and the window policy are fixed and logged with \((\beta,M(\tau),t)\).
\end{remark}

\begin{definition}[\(\Ext^1\)\hyp collapse at scale]\label{def:ext-collapse}
Writing \(\Rfun(F)\in D^{\mathrm{b}}(k\text{-mod})\), we say \emph{\(\Ext^1\)\hyp collapse holds at scale \(\tau\)} if, for all \(Q\in\Qtest:=\{k[0]\}\),
\[
  \Ext^1\!\big(\Rfun(\Ctau F),\, Q\big)\ =\ 0.
\]
\end{definition}

\subsection*{6.2. Stability under filtered colimits (geometry level)}
Let \(L_i(\Ctau F)\) denote the normalized combinatorial Hodge Laplacian in degree \(i\) on \(\Ctau F\), with nondecreasing positive spectrum \(\bigl(\lambda_{i,m}(\Ctau F)\bigr)_{m\ge 0}\). For brevity we suppress \(i\) and write \(L(\Ctau F)\), \(\mathrm{ST}_{\beta}^{\ge M(\tau)}(F)\), \(\mathrm{HT}(t;F)\) when the degree is clear from context.
\begin{declaration}[Specification: Stability under filtered colimits in geometry]\label{spec:geom-colim}
Assume a filtered diagram \(\{F_\lambda\}\) in \(\FiltCh{k}\) remains degreewise finite\hyp type; filtered (co)limits are computed objectwise in \([\mathbb{R},\mathsf{Vect}_k]\) and used only under the scope policy of Appendix~A (compute in the functor category and verify return to \(\Pers^{\mathrm{cons}}_k\)).
Then, for each fixed \(\tau\), the induced maps
\[
  \phi_{i,\tau}:\ \varinjlim\nolimits_\lambda \Ttau\!\big(\mathbf{P}_i(F_\lambda)\big)\ \xrightarrow{\ \cong\ }\ \Ttau\!\big(\mathbf{P}_i(\varinjlim\nolimits_\lambda F_\lambda)\big)
\]
are isomorphisms; hence \(\muc=\nuc=0\) at that scale. The conclusion holds pointwise along any discrete \(\tau\)\hyp sweep.
\end{declaration}

When aggregating across degrees, the degree set and aggregation policy (per-degree vs. summed) are fixed and logged.

\begin{remark}[Endpoints and infinite bars]\label{rk:endpoints-ch6}
Endpoint conventions (open/closed) and the treatment of infinite bars are as in Chapter~2, Remark~\ref{rk:2-endpoints}; \(\Ttau\) deletes only finite bars of length \(\le\tau\).
\end{remark}

\subsection*{6.3. Joint monitoring and programmatic guarantees}
\begin{declaration}[Specification: Geometric collapse indicators]\label{spec:geom-indicators}
Under \LC\ and within the implementable range, along geometric degenerations \emph{compute and record}:
\begin{enumerate}
  \item \(\Ttau\mathbf{P}_i(F)\) and the truncated energies \(\mathrm{PE}_i^{\le\tau}\) on \(\Ttau\mathbf{P}_i(F)=\mathbf{P}_i(\Ctau F)\);
  \item spectral indicators \(\mathrm{ST}_{\beta}^{\ge M(\tau)}\) or \(\mathrm{HT}(t;\cdot)\) on \(L(\Ctau F)\) (parameters as in Remark~\ref{rk:ST-conv});
  \item the \(\Ext^1\)\hyp check \(\Ext^1\big(\Rfun(\Ctau F),Q\big)=0\) for \(Q\in\Qtest\).
\end{enumerate}
The \emph{stable regime} is declared where \((\muc,\nuc)=(0,0)\) and (1)–(3) hold jointly.
\end{declaration}

\begin{remark}[Saturation gate (reference; see Chapter~11)]\label{rk:sat-gate-ch6}
We follow the Chapter~11 policy for a window \([0,\tau^\ast]\): \emph{(i)} eventually the maximal finite bar length in \(\mathbf{T}_{\tau^\ast}\mathbf{P}_i(F_t)\) is \(\le \eta\); \emph{(ii)} eventually \(d_{\mathrm{int}}\!\big(\mathbf{T}_{\tau^\ast}\mathbf{P}_i(F_t),\mathbf{T}_{\tau^\ast}\mathbf{P}_i(F_{t'})\big)\le \eta\); \emph{(iii)} the edge gap \(\delta:=\tau^\ast-\max\{b_r<\tau^\ast\}\) satisfies \(\delta>\eta\).
This chapter \emph{uses the gate only as a reference}; the quantitative policy and its verification are centralized in Chapter~11.
\end{remark}

\begin{conjecture}[Geometry\texorpdfstring{$\to$}{->}AK collapse propagation]\label{conj:geom-prop}
If items \emph{(1)}–\emph{(3)} of Declaration~\ref{spec:geom-indicators} hold with \((\muc,\nuc)=(0,0)\) along a non\hyp expansive degeneration over a \(\tau\)\hyp interval, then
\[
  \text{geometric collapse}\ \Longrightarrow\ \text{persistence\ energy\ decay}\ \Longrightarrow\ \text{spectral decay}\ \Longrightarrow\ \text{\(\Ext\)\hyp collapse at scale},
\]
compatibly with \LC.
\end{conjecture}

\subsection*{6.4. Scope, design patterns, and diagrams}
\begin{declaration}[Specification: Scope of admissible degenerations]\label{spec:scope}
The program encompasses: \emph{(a)} metric(\hyp measure) collapses modeled by subsampling and \(1\)\hyp Lipschitz retractions; \emph{(b)} simplicial refinements with bounded local degree; \emph{(c)} graph sparsifications preserving the normalized Laplacian construction and the \(1\)\hyp Lipschitz property of \(\mathcal{G}\), thereby keeping each \(\mathbf{P}_i\) non\hyp expansive under these maps.
Each case is functorially embedded by an admissible \(\mathcal{G}\).
\end{declaration}

\begin{center}
\begin{tikzcd}[
  row sep=1.6em, column sep=3.4em,
  cells={nodes={font=\footnotesize}},
  every label/.append style={font=\scriptsize},
  scale=0.92, transform shape
]
{\text{Geometric input } X} \arrow[r, "\mathcal{G}"] 
  & {F\in \FiltCh{k}} \arrow[r, "\mathbf{P}_i"] 
    & {\mathbf{P}_i(F)} \arrow[r, "\Ttau"] \arrow[d, dashed, "\mathrm{PE}_i^{\le\tau}"']
      & {\Ttau\mathbf{P}_i(F)} \arrow[d, dashed, "\mathrm{PE}_i^{\le\tau}"] \\\n{} & {} & {\text{Spectral } L(\Ctau F)}
      \arrow[r, dashed, "\text{heat trace on }L(\Ctau F)"] 
      & {{\mathrm{HT},\,\mathrm{ST}_{\beta}}} \\\n{} & {\Rfun(F)}
      \arrow[from=3-2, to=1-3, bend left=20, dashed, "\LC"]
      \arrow[from=3-2, to=3-4, dashed, "{\begin{matrix}\Ctau\text{ then}\\ \Ext^1(-,Q)\end{matrix}}"]
  & {}
  & {{\Ext^1\big(\Rfun(\Ctau F),Q\big)=0\ \ (\text{check})}}
\end{tikzcd}
\end{center}

\subsection*{6.5. Failure geometry and diagnostics}
\begin{definition}[Geometric failure types at scale]\label{def:geom-failure}
Within the monitored window, a sample is \emph{Type~IV at scale \(\tau\)} if \(\mathrm{PE}^{\le\tau}\) and spectral indicators decay while \((\muc,\nuc)\neq(0,0)\).
The \emph{pure cokernel type} denotes \(\muc=0\) and \(\nuc>0\).
\end{definition}

\begin{declaration}[Specification: Diagnostic actions]\label{spec:diagnostic}
When \((\muc,\nuc)\neq(0,0)\), refine the index diagram or adjust \(\tau\)\hyp sweep granularity until either \emph{(a)} the obstruction vanishes, or \emph{(b)} the failure persists across refinements, in which case the regime is recorded as non\hyp collapsible at the monitored scale.
\end{declaration}

\subsection*{6.6. Symplectic hook: Fukaya realization (\textbf{[Spec]})}
\begin{declaration}[Spec--Fukaya realization]\label{spec:fukaya}
Let \(\mathsf{Symp}^{\mathrm{adm}}\) be exact/monotone Liouville domains or sectors with stops.
\(\mathcal{G}_{\mathrm{Fuk}}\) assigns action\hyp filtered Floer complexes on a fixed window \([a,\tau]\) with \(a\le \tau\) over a field.
Assume:
\emph{(F1)} finite action spectrum in \([a,\tau]\);
\emph{(F2)} continuation maps shift actions by \(\le\varepsilon\) uniformly (hence are \(1\)\hyp Lipschitz for interleavings);
\emph{(F3)} stop additions/Liouville contractions are deletion\hyp type (Appendix~E).
Then for each degree \(i\) and scale \(\tau\) the comparison maps \(\phi_{i,\tau}\) are isomorphisms, hence \((\mu,\nu)=(0,0)\) on the monitored window.
Proof sketches and scope limits appear in Appendix~O.
\end{declaration}

\begin{remark}[Scope and bridge domain]
The specification above does \emph{not} extend the proved bridge beyond \(D^{\mathrm{b}}(k\text{-mod})\);
it provides a stable geometric hook whose persistence\hyp level behavior feeds the Part~I pipeline.
\end{remark}

\subsection*{6.7. Permitted operations catalog and $\delta$-ledger (reinforced policy)}\label{subsec:ops-delta}
We record the admissible A\hyp side operations, their expected persistence\hyp level behavior \emph{after collapse}, and the mandatory $\delta$ logging.

\begin{definition}[Permitted operations]\label{def:ops-catalog}
Each A\hyp side step $U$ is labeled:
\begin{itemize}
  \item \emph{Deletion\hyp type (monotone).} Examples: stop addition / sector shrinking (symplectic), mollification (low\hyp pass filtering), viscosity increment (PDE), threshold lowering, filter upper\hyp cap. \emph{Guarantee:} after applying $C_\tau$, windowed persistence energies and spectral auxiliaries (aux\hyp bars) are \emph{non\hyp increasing} (Appendix~E).
  \item \emph{$\varepsilon$-continuation (non\hyp expansive).} Examples: small Hamiltonian continuation; micro time\hyp step; minor stop shift. \emph{Guarantee:} $d_{\mathrm{int}}(\mathbf{P}_i(F),\mathbf{P}_i(UF))\le \varepsilon$; after $C_\tau$, indicators are \emph{stable} up to the prescribed $\varepsilon$.
  \item \emph{Inclusion\hyp type (stable only).} Examples: domain enlargement, inclusion maps not covered by the deletion\hyp type list. \emph{Guarantee:} no monotonicity claim; only stability (non\hyp expansiveness) if the induced map is $1$-Lipschitz on persistence.
\end{itemize}
\end{definition}

\begin{declaration}[Mandatory $\delta$-ledger]\label{dec:delta-ledger-ch6}
For each step $U$ with collapse $C_{\tau}$ and a fixed degree $i$, record a three\hyp part non\hyp commutation budget
\[
\delta(i,\tau)\ =\ \delta^{\mathrm{alg}}(i,\tau)\ +\ \delta^{\mathrm{disc}}(i,\tau)\ +\ \delta^{\mathrm{meas}}(i,\tau),
\]
where $\delta^{\mathrm{alg}}$ is the theoretical Mirror/Transfer–Collapse mismatch, $\delta^{\mathrm{disc}}$ the discretization error, and $\delta^{\mathrm{meas}}$ the numerical/estimation error. The per\hyp window pipeline budget is $\Sigma\delta(i)=\sum_{U\in W}\delta(i,\tau)$ and must satisfy $\mathrm{gap}_\tau>\Sigma\delta(i)$ to pass B\hyp Gate$^{+}$ (Chapter~1).
\end{declaration}

\subsection*{6.8. Gate template (per step, per window) and saturation usage}\label{subsec:gate-template}
The following operational template is used for each A\hyp side step within a fixed domain window $W=[u,u')$ and a fixed collapse threshold $\tau>0$:

\begin{enumerate}
  \item \emph{Apply step $U$ and collapse.} Execute $U$ (labeled as in Definition~\ref{def:ops-catalog}), then apply $C_\tau$.
  \item \emph{Measure on B\hyp side single layer.} Compute $\Ttau\mathbf{P}_i(F)$, $\mathrm{PE}_i^{\le\tau}$, spectral indicators on $L(C_\tau F)$ under the fixed policy, and (if in scope) $\Ext^1(\Rfun(C_\tau F),k)$.
  \item \emph{Record $\delta$.} Append $\delta^{\mathrm{alg}},\delta^{\mathrm{disc}},\delta^{\mathrm{meas}}$ for this step to the per\hyp window ledger and update $\Sigma\delta(i)$.
  \item \emph{Evaluate B\hyp Gate$^{+}$.} Check PH1=0, (if in scope) Ext1=0, $(\mu,\nu)=(0,0)$ for the window and degree $i$, and enforce $\mathrm{gap}_\tau>\Sigma\delta(i)$.
  \item \emph{Log verdict.} If all pass, issue a windowed certificate; otherwise, classify failure (Type I–IV) and proceed with diagnostics (Declaration~\ref{spec:diagnostic}).
\end{enumerate}
On windows declared \emph{saturated} in the sense of Chapter~11, one may use the saturation gate (Chapter~11) as a reference to shorten step (4) (remain within its quantitative policy).

\subsection*{6.9. Windowed workflow and logging (MECE enforcement)}\label{subsec:window-workflow}
Let \(\{[u_k,u_{k+1})\}_k\) be a MECE partition (Chapter~2, Def.~\ref{def:ch2-mece}). For each window:
\begin{itemize}
  \item Fix \(\tau\) by the adaptation rule (Chapter~2, Def.~\ref{def:tau-adapt}); if spectral auxiliaries are used, fix \((\beta,[a,b])\).
  \item Run the gate template (Subsection~\ref{subsec:gate-template}) for each step; aggregate $\Sigma\delta(i)$ and evaluate B\hyp Gate$^{+}$.
  \item Record coverage checks (sum of lengths; sum of events) and all parameters in the manifest (Appendix~G).
\end{itemize}
Global claims are obtained by pasting windowed certificates via Restart (Lemma~\ref{lem:restart}) and Summability (Definition~\ref{def:summability}); $\tau$ is selected inside stable bands (Definition~\ref{def:stable-band}). When multiple torsion reflectors are used (e.g.\ length plus birth window), apply the soft\hyp commuting policy (Chapter~5, Definition~\ref{def:soft-commute}); otherwise fix a deterministic order and record the commutation defect in $\delta^{\mathrm{alg}}$.

\subsection*{6.10. Compliance checklist (per run)}\label{subsec:compliance}
\begin{enumerate}
  \item MECE windows recorded; coverage checks pass.
  \item Collapse threshold \(\tau\) adapted to resolution; spectral bin policy fixed and logged.
  \item Each step labeled (deletion/\(\varepsilon\)/inclusion) with 1\hyp Lipschitz rationale; \(\delta^{\mathrm{alg}},\delta^{\mathrm{disc}},\delta^{\mathrm{meas}}\) recorded.
  \item Indicators computed on B\hyp side single layer only; B\hyp Gate$^{+}$ evaluated (PH1/Ext1/$(\mu,\nu)$, safety margin).
  \item Tower audit $(\mu,\nu)=(0,0)$ on the window; stable band identified for \(\tau\).
  \item Verdict (accept/reject) and failure type logged; Restart/Summability plan updated for the next window.
\end{enumerate}

\subsection*{6.11. Summary}
This chapter specifies the operational program for geometric collapse in the implementable range. Admissible realizations (Definition~\ref{def:geom-real}) feed the persistence layer, where collapse $C_\tau$ is applied and all indicators are computed on the B\hyp side single layer. Deletion\hyp type steps are \emph{monotone} after collapse; $\varepsilon$–continuations are \emph{stable}. Mirror/Transfer non\hyp commutation with collapse is \emph{externalized} via a \(\delta\)–ledger and accumulated additively along pipelines. Windowed certificates are issued per MECE window by B\hyp Gate$^{+}$; global claims are obtained by pasting certificates using Restart and Summability, with $\tau$ selected inside stable bands. The soft\hyp commuting policy (Chapter~5) governs multi\hyp axis torsions. All assertions remain confined to the persistence layer and respect the one\hyp way bridge (Chapter~3) and the tower calculus (Chapter~4).



```latex
% ============================================================
% Chapter 7: Arithmetic Layers and Iwasawa Refinement (Design) — reinforced
% ============================================================
\section{Chapter 7: Arithmetic Layers and Iwasawa Refinement (Design)}
\addcontentsline{toc}{section}{Arithmetic Layers and Iwasawa Refinement (Design)}

\noindent\textbf{Index separation.} The collapse obstruction \(\muc\) used in this chapter is a persistence\hyp level diagnostic and is \emph{unrelated} to the classical Iwasawa \(\mu\)\nobreakdash invariant; no identity or implication between them is asserted (see also §\ref{rk:separation}).

\begin{remark}[Monotonicity convention]
Throughout this chapter we adopt the corrected monotonicity convention of
Chapter~6, Remark~\ref{rem:stability-vs-monotonicity}:
\emph{deletion\hyp type} updates are non\hyp increasing for spectral tails and windowed energies,
while \emph{inclusion\hyp type} updates are only stable (non\hyp expansive);
see Appendix~E for sufficient conditions and counterexamples.
\end{remark}

\subsection*{7.0. Standing hypotheses and admissible arithmetic realization}
All statements in this chapter are made within the \emph{constructible range}
(we identify \(\Pers^{\mathrm{ft}}_k\) with the constructible subcategory as in Chapters~2 and~6).
Fix a base field \(k\) and adopt the notation and posture of Part~I:
\(\FiltCh{k}\) denotes finite\hyp type filtered chain complexes,
\(\mathbf{P}_i:\FiltCh{k}\to\Pers^{\mathrm{cons}}_k\) the degreewise persistence functor,
and we write \(\Ttau:=\mathbf{T}_\tau\) for the Serre (bar\hyp deletion) reflector at scale \(\tau\ge 0\) (with \(\Ttau=\mathrm{Id}\) at \(\tau=0\)).
Its filtered lift \(C_\tau\) is used \emph{up to filtered quasi\hyp isomorphism} (Chapter~2, §§2.2–2.3).
A fixed realization \(\Rfun:\FiltCh{k}\to D^{\mathrm{b}}(k\text{-mod})\) is \(t\)\hyp exact.
Unless explicitly marked \textbf{[Spec]}, \emph{equalities and Lipschitz claims are asserted only at the persistence layer};
at the filtered\hyp complex layer they hold \emph{up to filtered quasi\hyp isomorphism}.
Endpoint conventions and the treatment of infinite bars are as in Chapter~2, Remark~\ref{rk:2-endpoints}.

Arithmetic input is organized as towers
\[
  \mathbb{T}\ :=\ \{X_t\}_{t\in I}\ \longrightarrow\ X_\infty,
\]
indexed by a directed set \(I\cup\{\infty\}\) with transition maps \(X_{t'}\to X_t\) for \(t'\ge t\) (e.g.\ norm/corestriction, specialization, level\hyp lowering).
Typical instances include cyclotomic/ray\hyp class towers of number fields, modular\hyp level towers, or Selmer\hyp complex towers.
Filtered (co)limits, when used, are computed objectwise in \([\mathbb{R},\mathsf{Vect}_k]\) and used only under the scope policy of Appendix~A (compute in the functor category and verify return to \(\Pers^{\mathrm{cons}}_k\)); no claim is made outside this regime.

\begin{definition}[Admissible arithmetic realization]\label{def:arith-real}
An \emph{admissible arithmetic realization} is a functor
\[
\begin{aligned}\n\mathcal{A}\colon\ & \mathsf{ArithTower}\longrightarrow \FiltCh{k},\\[-0.25em]\n& \mathbb{T}\longmapsto F_\bullet=\{F_t\}_{t\in I\cup\{\infty\}},\n\end{aligned}
\]
subject to:
\begin{enumerate}
  \item \textbf{Functoriality \& non\hyp expansiveness (persistence):} each transition \(X_{t'}\!\to X_{t}\) with \(t'\ge t\) induces a filtered chain map \(F_{t'}\!\to F_{t}\) such that, for every degree \(i\),
  \[
    d_{\mathrm{int}}\!\big(\mathbf{P}_i(F_{t'}),\mathbf{P}_i(F_{t})\big)\ \le\ \varepsilon_{t',t},\qquad \varepsilon_{t',t}\ge 0.
  \]
  In \emph{deletion\hyp type} steps satisfying Appendix~E (Dirichlet restriction, principal submatrices/Schur complements, certain contractions) one often has \(\varepsilon_{t',t}=0\); in general we assume the bound above (non\hyp expansive updates up to f.q.i.).
  \item \textbf{Finite\hyp type preservation \& colimits:} each \(F_t\) is degreewise finite\hyp type; degreewise filtered colimits in \(\FiltCh{k}\) are computed objectwise (Appendix~A).
  \item \textbf{Realization coherence:} a fixed \(t\)\hyp exact \(\Rfun\) is used for all \(t\), with comparison maps compatible with the lifting–coherence hypothesis \LC, so that functorially (up to f.q.i.)
  \[
    \Rfun(C_\tau F_t)\ \simeq\ \tau_{\ge 0}\,\Rfun(F_t).
  \]
  \item \textbf{Endpoints:} bars use the Part~I open/closed endpoint policy; \(\Ttau\) deletes only finite bars of length \(\le\tau\) (Chapter~2, §2.2).
\end{enumerate}
\end{definition}

\begin{remark}[Cone extension for the tower]\label{rk:cone-extension}
We work in the filtered index category \(I\cup\{\infty\}\) with \(t\le \infty\) and \emph{cone maps} \(X_t\to X_\infty\).
The realization \(\mathcal{A}\) carries these to filtered maps \(F_t\to F_\infty\), yielding the comparison maps in Definition~\ref{def:phi-munu}, mirroring Chapter~4.
\end{remark}

\subsection*{7.1. Class/Selmer visualization at the persistence layer}
\begin{definition}[Arithmetic visualization data]\label{def:vis}
Given \(\mathbb{T}\mapsto F_\bullet\) via \(\mathcal{A}\), define for each \(t\in I\) and degree \(i\):
\[
  \text{barcode } \mathcal{B}_i(F_t):=\mathrm{bars}\big(\mathbf{P}_i(F_t)\big),\qquad\n  \text{truncated energies } \mathrm{PE}_i^{\le\tau}(F_t)\ \text{ as in §6.1}.
\]
All measurements are computed on the \emph{truncated barcodes} \(\Ttau\mathbf{P}_i(F_t)\); equivalently on \(C_\tau F_t\).
We use the default \(\alpha=1\) unless stated otherwise (cf.\ §6.1).
Interpretation: \(\mathcal{B}_i(F_t)\) and \(\mathrm{PE}_i^{\le\tau}(F_t)\) serve as \emph{visual proxies} for growth/stability patterns of arithmetic invariants (e.g.\ class/Selmer\hyp like data) along the tower.
\end{definition}

\begin{remark}[Spectral layer and \(\Ext^1\)\hyp check]\label{rk:spectral-ext}
Form the normalized Hodge Laplacian \(L_i(C_\tau F_t)\) in degree \(i\) and record spectral tails/heat traces as in §6.1, with positive\hyp eigenvalue summation and standard convergence choices.
At the categorical layer, check \(\Ext^1\!\big(\Rfun(C_\tau F_t),Q\big)=0\) for \(Q\in\Qtest=\{k[0]\}\).
These three families (persistence, spectral, categorical) are monitored jointly.
\end{remark}

\subsection*{7.2. Tower diagnostics and obstructions}
\begin{definition}[Tower comparison and obstruction indices]\label{def:phi-munu}
For each degree \(i\) and scale \(\tau\), the comparison map
\[
  \phi_{i,\tau}:\ \varinjlim_{t\in I}\ \Ttau\!\big(\mathbf{P}_i(F_t)\big)\ \longrightarrow\ \Ttau\!\big(\mathbf{P}_i(F_\infty)\big)
\]
yields obstruction counts
\[
  \mu_{i,\tau}:=\dim_k\ker\phi_{i,\tau}\footnote{Here \(\dim_k\) denotes the \emph{generic\hyp fiber} dimension after truncation, i.e.\ the multiplicity of \(I[0,\infty)\) summands; see Appendix~D, Remark~\ref{rem:D-generic-dim}.},\qquad\n  \nu_{i,\tau}:=\dim_k\,\mathrm{coker}\big(\phi_{i,\tau}\big),\qquad\n  \muc:=\sum_{i}\mu_{i,\tau},\ \ \nuc:=\sum_i \nu_{i,\tau}.
\]
Since complexes are bounded in homological degrees, the sums are finite. They are invariant under \emph{filtered quasi\hyp isomorphisms} and under \emph{cofinal reindexing} of the tower (Appendix~J).
\emph{We suppress the explicit \(\tau\)\nobreakdash dependence of the totals and write \(\mu_{\mathrm{Collapse}}\), \(\nu_{\mathrm{Collapse}}\) when the scale is clear from context.}
\end{definition}

\begin{declaration}[Spec–Arithmetic towers (non\hyp expansion)]\label{spec:arith-nonexp}
Index transitions are non\hyp expansive in the interleaving sense of Definition~\ref{def:arith-real}\,(1), with shifts \(\varepsilon_{t',t}\) uniformly controlled (e.g.\ \(\sup_{t'\ge t}\varepsilon_{t',t}<\infty\)).
Under finite\hyp type, objectwise filtered colimits (Appendix~A), each \(\phi_{i,\tau}\) is an isomorphism (cf.\ Chapter~4, Proposition~\ref{prop:mu-vanishing}), hence \((\muc,\nuc)=(0,0)\) at the monitored scales.
\end{declaration}

\begin{declaration}[Specification: Tower stability at the persistence layer]\label{spec:tower-stability}
Assume Definition~\ref{def:arith-real}\,(2) and that degreewise filtered colimits in \(\FiltCh{k}\) are computed objectwise \emph{under the scope policy of Appendix~A}.
Then, for each fixed \(\tau\) and all degrees \(i\),
\[
  \phi_{i,\tau}:\ \varinjlim_{t\in I} \Ttau\!\big(\mathbf{P}_i(F_t)\big)\ \xrightarrow{\ \cong\ }\ \Ttau\!\big(\mathbf{P}_i(F_\infty)\big)
\]
is an isomorphism. Consequently, \((\muc,\nuc)=(0,0)\) at scale \(\tau\).
Under a discrete sweep of \(\tau\), the conclusion holds pointwise at each monitored scale.
\end{declaration}

\begin{remark}[Excluding Type~IV under tower stability]\label{rk:exclude-typeIV}
Under Declaration~\ref{spec:tower-stability}, we have \((\muc,\nuc)=(0,0)\) at each fixed \(\tau\); hence Type~IV cannot occur at that scale.
\end{remark}

\begin{remark}[Failure patterns]\label{rk:fail}
If \((\muc,\nuc)\neq(0,0)\), record the failure type: \emph{pure cokernel} \((\muc=0,\nuc>0)\), \emph{pure kernel} \((\muc>0,\nuc=0)\), or \emph{mixed} \((\muc>0,\nuc>0)\).
Type~IV at scale \(\tau\) denotes decay of persistence/spectral indicators with \((\muc,\nuc)\neq(0,0)\).
\end{remark}

\begin{example}[Toy towers at the persistence layer]\label{ex:toy-towers}
Fix \(\tau>0\).
\emph{(Pure cokernel.)} Let \(\mathbf{P}_1(F_t)=I[0,\tau-\tfrac{1}{t})\) for $t\in\mathbb{N}$ with inclusions and set \(\mathbf{P}_1(F_\infty)=I[0,\infty)\).
Then \(\Ttau(\mathbf{P}_1(F_t))=0\) for all $t$ while \(\Ttau(\mathbf{P}_1(F_\infty))\cong I[0,\infty)\), hence \(\muc=0\) and \(\nuc>0\) at scale \(\tau\).
\emph{(Pure kernel.)} Dually, let \(\mathbf{P}_1(F_t)=\bigoplus_{j=1}^{t} I[0,\tau)\) with epimorphic transitions \(\mathbf{P}_1(F_{t+1})\twoheadrightarrow \mathbf{P}_1(F_t)\) and set \(\mathbf{P}_1(F_\infty)=0\).
Then \(\muc>0\) and \(\nuc=0\) at scale \(\tau\).
In both cases the cone extension (Remark~\ref{rk:cone-extension}) furnishes the maps to \(F_\infty\) and hence the \(\phi_{i,\tau}\).
\end{example}

\subsection*{7.3. Non\hyp identity with classical Iwasawa \(\mu\)}
\begin{remark}[Separation of indices]\label{rk:separation}
The persistence obstruction \(\muc\) is defined from kernels/cokernels of \(\phi_{i,\tau}\) between \emph{truncated} persistence modules, whereas the classical Iwasawa \(\mu\)\nobreakdash invariant measures \(p\)\nobreakdash primary growth for \(\mathbb{Z}_p[[T]]\)\nobreakdash modules.
No identity or implication between the two is asserted; any relation, if present, is programmatic and confined to \textbf{[Conjecture]} statements.
\end{remark}

\begin{remark}[On saturation gates]
Saturation/plateau criteria and their use as \textbf{[Spec]} binary gates are organized in Chapter~11; we make no chapter\hyp local gate declaration here.
\end{remark}

\subsection*{7.4. Program specifications for arithmetic towers}
\begin{declaration}[Specification: Admissible indexings and maps]\label{spec:indexing}
An indexing of the tower by conductor, level, or height that renders the transition maps non\hyp expansive (hence \(1\)\hyp Lipschitz under each \(\mathbf{P}_i\) in the interleaving sense of Definition~\ref{def:arith-real}(1)) is \emph{admissible}.
Under such indexings, energy and spectral indicators are stable (non\hyp expansive) in general and \emph{non\hyp increasing for deletion\hyp type steps} (Appendix~E), up to f.q.i.; no non\hyp increase is claimed for inclusion\hyp type updates.
\end{declaration}

\subsection*{7.5. Conjectural propagation along arithmetic towers}
\begin{conjecture}[AK–Arithmetic tower propagation]\label{conj:arith-prop}
Assume an admissible arithmetic realization $\mathcal{A}$ and \LC.
If, along a non\hyp expansive tower segment and for a scale interval in $\tau$, we have $(\muc,\nuc)=(0,0)$ and the persistence energies (deletion\hyp type: non\hyp increasing; general: stable) together with the spectral indicators are controlled as above, then the arithmetic visualization stabilizes at that scale:
the proxies registered by persistence/spectral layers remain bounded, and the categorical check
\(\Ext^1\big(\Rfun(C_\tau F_t),Q\big)=0\) persists along the segment.
No number\hyp theoretic identity, and no identification with the classical Iwasawa invariants, is asserted.
\end{conjecture}

\subsection*{7.6. Diagram and data flow}

\begin{tikzcd}[
  ampersand replacement=\&,
  column sep=1.8em, row sep=1.8em,
  cells={nodes={font=\footnotesize}},
  every label/.append style={font=\scriptsize},
  scale=0.92, transform shape
]
{\text{Arithmetic tower } \mathbb{T}} \arrow[r, "\mathcal{A}"]
  \& {\{F_t\}\subset \FiltCh{k}} \arrow[r, "\mathbf{P}_i"]
    \& {\mathbf{P}_i(F_t)} \arrow[r, "\Ttau"]
       \arrow[from=1-3, to=2-3, dashed, "{\mathrm{PE}_i^{\le \tau}}"']
      \& {\Ttau\mathbf{P}_i(F_t)}
       \arrow[from=1-4, to=2-4, dashed, "{\mathrm{PE}_i^{\le \tau}}"] \\\n{\phantom{\cdot}} \& {\phantom{\cdot}}
  \& {\text{Spectral } L(C_\tau F_t)}
    \arrow[r, dashed, "{\text{heat trace on }L(C_\tau F_t)}"]
  \& {\mathrm{HT},\ \mathrm{ST}_{\beta}} \\\n{\phantom{\cdot}} \& {\Rfun(F_t)}
  \arrow[from=3-2, to=1-3, bend left=40, dashed, "\LC"]
  \arrow[from=3-2, to=3-4, dashed, "{C_\tau\ \text{ then }\Ext^1(-,Q)}"]
  \& {\phantom{\cdot}} \& {\Ext^1(\Rfun(C_\tau F_t),Q)=0} \\\n{\phantom{\cdot}} \& {\varinjlim_{t}\, \Ttau\mathbf{P}_i(F_t)}
  \arrow[from=4-2, to=4-4, "\phi_{i,\tau}"]
  \& {\phantom{\cdot}} \& {\Ttau\mathbf{P}_i(F_\infty)}
    \arrow[from=4-4, to=5-2, bend right=25, dashed, swap, "{\text{log }(\muc,\nuc)}"] \\\n{\phantom{\cdot}} \& {\text{failure log (pure/mixed)}} \& {\phantom{\cdot}} \& {\phantom{\cdot}}
\end{tikzcd}

\subsection*{7.7. Minimal assumptions per arithmetic class (design templates)}
\begin{declaration}[Specification: Template hypotheses]\label{spec:templates}
For practical deployment, the following minimal templates ensure admissibility (one\hyp line concrete instances shown):
\begin{itemize}
  \item \textbf{(MM spaces from arithmetic data).} Index by conductor/level; realize transitions as \(1\)\hyp Lipschitz retractions between metric(\hyp measure) models (e.g.\ modular curves under level\hyp lowering with Gromov–Hausdorff \(1\)\hyp Lipschitz maps); preserve finite\hyp type per degree.
  \item \textbf{(Simplicial/complex models).} Use bounded\hyp degree subdivisions for level changes (e.g.\ barycentric refinement at fixed depth); ensure objectwise degreewise colimits; non\hyp expansiveness under each \(\mathbf{P}_i\).
  \item \textbf{(Graphs/quotients).} Sparsify while preserving normalized Laplacians and the \(1\)\hyp Lipschitz property of \(\mathcal{G}\)/\(\mathcal{A}\) (e.g.\ degree\hyp bounded sparsification of Cayley graphs); compute spectra on \(C_\tau F_t\).
\end{itemize}
\end{declaration}

\subsection*{7.8. Reproducibility and logs}
\begin{remark}[Run logs and parameters]\label{rk:logs}
For each run, log: the tower index range \(t\in[t_{\min},t_{\max}]\), the scale sweep \(\tau\in[\tau_{\min},\tau_{\max}]\) with step, spectral parameters \((\beta,M(\tau),t_{\mathrm{HT}})\), and the obstruction tuple \((\muc,\nuc)\) per \(\tau\) (with failure type).
Record also the degree set used for aggregation (per\hyp degree vs.\ summed across \(i\)) to ensure consistent replays.
These logs are part of the program specification and enable exact reruns.
\end{remark}

\subsection*{7.9. Final guard-rails}
\begin{remark}[Scope and non-claims]\label{rk:nonclaims}
This chapter provides a design blueprint at the persistence/spectral/categorical layers for arithmetic towers.
It does \emph{not} assert number\hyp theoretic identities or decide deep conjectures; all forward\hyp looking statements are explicitly labeled \textbf{[Conjecture]} and rely on the implementable range and \LC.
No claim of \(\mathrm{PH}_1\Leftrightarrow\Ext^1\) is made; only the one\hyp way bridge under \textup{(B1)–(B3)} from Part~I is used.
\end{remark}

% -------------------- Reinforcements: height windows, PF/BC, commutativity, δ-ledger, gates --------------------

\subsection*{7.10. Height windows (MECE), PF/BC audit, and $\delta$-naturality}
\begin{definition}[Height windows and MECE partition]\label{def:height-windows}
Let the index set \(I\) carry a \emph{height} function \(h:I\to \mathbb{R}\) (e.g.\ conductor/level/weight) that is non\hyp decreasing along transitions. A \emph{height windowing} is a MECE partition \(\{W_k=[u_k,u_{k+1})\}_k\) of the height range such that the subdiagram of indices \(\{t\in I:\ h(t)\in W_k\}\) is filtered. All audits, gates, and certificates are performed \emph{per window}.
\end{definition}

\begin{declaration}[PF/BC audit after collapse]\label{dec:pf-bc-audit}
For external comparison functors (Projection Formula/Base Change) denoted \(\mathrm{PF},\mathrm{BC}\) at the arithmetic layer, we \emph{first} pass to persistence and \emph{then} collapse:
\[
X_t\ \xrightarrow{\ \mathcal{A}\ }\ F_t\ \xrightarrow{\ \mathbf{P}_i\ }\ \mathbf{P}_i(F_t)\ \xrightarrow{\ \Ttau\ }\ \Ttau\mathbf{P}_i(F_t).
\]
Pseudonaturality and PF/BC equalities are checked \emph{after} \(\Ttau\), i.e.\ on \(\Ttau\mathbf{P}_i(F_t)\), uniformly on each window. Any mismatch is recorded in the \(\delta\)\hyp ledger as
\[
\delta^{\mathrm{alg}}_{\mathrm{PF/BC}}(i,\tau;W_k)\ :=\ d_{\mathrm{int}}\!\Big(\Ttau \mathbf{P}_i\big(\mathrm{PF/BC}\circ \mathcal{A}\big),\ \Ttau \mathbf{P}_i\big(\mathcal{A}\circ \mathrm{PF/BC}\big)\Big).
\]
Discretization and measurement contributions are added as \(\delta^{\mathrm{disc}},\delta^{\mathrm{meas}}\); the per\hyp window budget is \(\Sigma\delta(i)\) (cf.\ Chapter~5, Specification~\ref{spec:delta-ledger} and Chapter~6, Declaration~\ref{dec:delta-ledger-ch6}).
\end{declaration}

\begin{remark}[Mirror/level transfer and $\delta$-ledger]
Let \(\Mirror\) denote level transfer (e.g.\ norm, corestriction, specialization). We measure the $2$-cell defect \(\epsilon_{i,\tau}:\Mirror\circ C_\tau\Rightarrow C_\tau\circ \Mirror\) after collapse with bound \(\delta(i,\tau)\) as in Chapter~5, Definition~\ref{def:delta-2cell}, and add it additively to the window ledger. Pipeline additivity and $1$-Lipschitz post\hyp processing follow from Proposition~\ref{prop:pipeline-budget}.
\end{remark}

\subsection*{7.11. Commutativity and pseudonaturality tests after collapse}
\begin{declaration}[Pseudonaturality verification policy]\label{dec:pseudonat}
All naturality/compatibility diagrams involving level transfer, PF/BC, or auxiliary reflectors are verified \emph{on the collapsed persistence layer} (\(\Ttau\mathbf{P}_i(F_t)\)). This avoids pre\hyp collapse torsion noise and aligns the audit with the gate posture (B\hyp side single layer).
\end{declaration}

\begin{definition}[A/B commutativity test and fallback]\label{def:ab-test}
Given two persistence\hyp level reflectors \(T_A,T_B\) (e.g.\ length\hyp threshold and birth\hyp window) we define
\[
\Delta_{\mathrm{comm}}(M;A,B)\ :=\ d_{\mathrm{int}}\!\big(T_A T_B M,\ T_B T_A M\big).
\]
On each height window \(W_k\) we run the A/B test on \(M=\Ttau\mathbf{P}_i(F_t)\). If \(\Delta_{\mathrm{comm}}\le \eta\) (tolerance), we accept \emph{soft\hyp commuting} (Chapter~5, Definition~\ref{def:soft-commute}); otherwise we fix a deterministic order (e.g.\ \(T_B\circ T_A\)), and record \(\Delta_{\mathrm{comm}}\) into \(\delta^{\mathrm{alg}}\).
\end{definition}

\begin{remark}[Nested torsions and order independence]
If the Serre classes are nested, order independence holds and no A/B test is required (Chapter~5, Proposition~\ref{prop:nested-commute}); otherwise soft\hyp commuting governs adoption.
\end{remark}

\subsection*{7.12. Gate template for arithmetic windows and saturation usage}
\begin{enumerate}
  \item \textbf{Window selection.} Choose a height window \(W_k=[u_k,u_{k+1})\) (Definition~\ref{def:height-windows}); fix \(\tau\) inside a stable band (Chapter~4, Definition~\ref{def:stable-band}).
  \item \textbf{Collapse then measure.} For each \(t\) with \(h(t)\in W_k\), compute \(\Ttau\mathbf{P}_i(F_t)\), energies \(\mathrm{PE}_i^{\le\tau}\), spectral indicators on \(L(C_\tau F_t)\), and (if in scope) \(\Ext^1(\Rfun(C_\tau F_t),k)\).
  \item \textbf{$\delta$ logging.} Audit PF/BC and Mirror transfer after collapse (Declaration~\ref{dec:pf-bc-audit}); run A/B tests (Definition~\ref{def:ab-test}); accumulate \(\Sigma\delta(i)\).
  \item \textbf{B\hyp Gate$^{+}$.} Require: \(\mathrm{PH}_1(C_\tau F_t)=0\), \((\mu,\nu)=(0,0)\) per window (Declarations~\ref{spec:tower-stability}), Ext\(^1\) pass (if checked), and safety margin \(\mathrm{gap}_\tau>\Sigma\delta(i)\).
  \item \textbf{Certificate \& paste.} Issue the window certificate; paste across windows via Restart and Summability (Chapter~4, Lemma~\ref{lem:restart}, Definition~\ref{def:summability}).
\end{enumerate}
On windows declared \emph{saturated} (Chapter~11), the gate may reference the saturation criteria directly.

\subsection*{7.13. Compliance checklist (arithmetic run)}
\begin{enumerate}
  \item Height windows form a MECE partition; coverage log recorded.
  \item \(\tau\)\hyp sweep and stable bands documented; spectral parameters fixed.
  \item PF/BC and Mirror audits executed \emph{after collapse}; \(\delta^{\mathrm{alg}},\delta^{\mathrm{disc}},\delta^{\mathrm{meas}}\) ledger complete.
  \item A/B commutativity tests run per window; soft\hyp commuting adopted or deterministic order fixed with \(\Delta_{\mathrm{comm}}\) logged.
  \item B\hyp side only measurements; B\hyp Gate$^{+}$ passed with safety margin; \((\muc,\nuc)=(0,0)\) per window.
  \item Certificates issued and pasted with Restart/Summability; failure types logged if any.
\end{enumerate}
```


```latex
% ============================================================
% Chapter 8: Mirror/Tropical Collapse (Weak Group Collapse) — reinforced
% ============================================================
\section{Chapter 8: Mirror/Tropical Collapse (Weak Group Collapse)}
\addcontentsline{toc}{section}{Mirror/Tropical Collapse (Weak Group Collapse)}

\noindent\textbf{Windowed policy and B-side judgement.}
All statements in this chapter are \emph{windowed} (Chapter~1, Def.~1.0; Chapter~2, Remark~\ref{rk:ch2-mece}).
Gate decisions are taken \emph{only} on the B-side after collapse, i.e.\ on single-layer objects
$\mathbf{T}_\tau\mathbf{P}_i$ (equivalently $\mathbf{P}_i(C_\tau-)$).
Filtered-complex equalities hold only up to f.q.i.\ (Appendix~B).

\begin{remark}[Monotonicity convention]
Throughout this chapter we adopt the convention of
Chapter~6, Remark~\ref{rem:stability-vs-monotonicity}:
\emph{deletion\hyp type} updates are non\hyp increasing for spectral tails and windowed energies,
while \emph{inclusion\hyp type} updates are only stable (non\hyp expansive); see Appendix~E.
\end{remark}

\subsection*{8.0. Standing hypotheses, realizations, and $\delta$-ledger}
We fix a field \(k\) and work within the \emph{implementable range} of Part~I.
All statements lie in the \emph{constructible range} (we identify \(\Pers^{\mathrm{ft}}_k\) with the constructible subcategory as in Chapter~6).
Let \(\FiltCh{k}\) denote finite\hyp type filtered chain complexes,
\(\mathbf{P}_i:\FiltCh{k}\to\Pers^{\mathrm{cons}}_k\) the degreewise persistence functor, and write
\(\Ttau:=\mathbf{T}_\tau\) for the Serre (bar\hyp deletion) reflector at scale \(\tau\ge 0\).
Its filtered lift \(\Ctau\) is used \emph{up to filtered quasi\hyp isomorphism} (Chapter~2, §§2.2–2.3).
A fixed \(t\)\hyp exact realization \(\Rfun:\FiltCh{k}\to D^{\mathrm{b}}(k\text{-mod})\) is retained, and \LC\ holds whenever \(\Ctau\) is compared with \(\tau_{\ge 0}\!\circ\!\Rfun\).
\emph{Equalities and Lipschitz statements are asserted only at the persistence layer; at the filtered\hyp complex layer they hold up to filtered quasi\hyp isomorphism.}
Endpoint conventions and the treatment of infinite bars are as in Chapter~2, Remark~\ref{rk:2-endpoints}.
Kernel/cokernel diagnostics \((\mu_{\mathrm{Collapse}},\nu_{\mathrm{Collapse}})\) at scale \(\tau\) are computed from comparison maps as in Chapter~4, §4.2, with \(\dim_k\) interpreted as the \emph{generic\hyp fiber} dimension after truncation (multiplicity of \(I[0,\infty)\)); see Appendix~D, Remark~\ref{rem:D-generic-dim}.

\noindent\textbf{Quantitative commutation reference.}
When needed, we assume a natural \(2\)\hyp cell \(\theta:\Mirror\!\circ\! C_\tau\Rightarrow C_\tau\!\circ\!\Mirror\) whose effect at persistence is controlled by \(\delta(i,\tau)\ge 0\), \emph{uniform in the input \(F\)}, and that \(\Mirror\) is \(1\)\hyp Lipschitz; see Appendix~L (hypotheses \textup{(H1)}–\textup{(H2)}). The total non\hyp commutation budget along a pipeline is the additive sum \(\Sigma\delta(i)=\sum_j \delta_j(i,\tau_j)\) (Chapter~5, Proposition~\ref{prop:pipeline-budget}).

\medskip
We consider admissible realizations (Chapter~6, Definition~\ref{def:geom-real}; Chapter~7, Definition~\ref{def:arith-real})
\[
  \mathsf{Geom}_A\ \xrightarrow{\ \mathcal{G}_A\ }\ \FiltCh{k},\qquad\n  \mathsf{Geom}_B\ \xrightarrow{\ \mathcal{G}_B\ }\ \FiltCh{k}.
\]
A \emph{tropical base contraction} at parameter \(\lambda\in(0,1]\) is an endofunctor
\[
  \Trop_\lambda:\ \mathsf{Geom}_A\longrightarrow \mathsf{Geom}_A
\]
whose induced filtered map on \(F:=\mathcal{G}_A(X)\) is non\hyp expansive degreewise under each \(\mathbf{P}_i\) and monotone (deletion\hyp type) when \(\lambda\) decreases. A \emph{mirror transfer} is a functor
\[
  \Mirror:\ \FiltCh{k}\longrightarrow \FiltCh{k}
\]
that is non\hyp expansive for each \(\mathbf{P}_i\), compatible with \(\Ctau\) up to f.q.i., and subject to \LC\ for comparisons after realization \(\Rfun\). No categorical equivalence is assumed.

\begin{remark}[Endpoints and infinite bars]\label{rk:8-endpoints}
All statements are insensitive to open/closed endpoints, and infinite bars are not removed by \(\Ttau\);
windowed indicators clip their contributions (cf.\ Chapter~6).
\end{remark}

\begin{remark}[Cone extension for the tropical flow]\label{rk:8-cone}
For a directed parameter set \(\Lambda\subset(0,1]\) with \(\lambda'\le \lambda\), adjoin a terminal element \(\lambda_\ast\) (formally representing \(\lambda\to 0\)) and cone maps \(\Trop_{\lambda}\Rightarrow \Trop_{\lambda_\ast}\).
Under \(\mathcal{G}_A\), these induce filtered maps \(F_\lambda\to F_{\lambda_\ast}\), providing the comparison maps used to compute \((\mu_{\mathrm{Collapse}},\nu_{\mathrm{Collapse}})\) at fixed \(\tau\) along the \(\lambda\)\nobreakdash tower (Chapter~4).
\end{remark}

\subsection*{8.1. Tropical contraction and barcode shortening}
\begin{definition}[Uniform shortening proxy \textbf{[Spec]}]\label{def:shorten}
Let \(F\in\FiltCh{k}\) and fix \(\tau\ge 0\).
We say that a filtered map \(F\to F'\) \emph{uniformly shortens} degreewise barcodes at factor \(\kappa\in(0,1]\) \emph{up to f.q.i.} if, for every degree \(i\), the multiset of lengths in \(\Ttau\mathbf{P}_i(F')\) is obtained from that of \(\Ttau\mathbf{P}_i(F)\) by multiplying lengths by \(\le \kappa\) and possibly deleting some bars, modulo filtered quasi\hyp isomorphisms.
Infinite bars remain unaffected; \emph{\textbf{shortening is enforced only within the monitored window \([0,\tau]\) after applying \(\Ttau\)}} (windowing clips contributions at \(\tau\)).
The factor may depend on \(i\) and \(\tau\); we allow \(\kappa=\kappa_i(\tau)\) implicitly.
\emph{This is a model\hyp dependent operational specification; no canonical form is claimed outside the implementable range.}
\end{definition}

\begin{declaration}[Specification: Tropical reduction vs.\ barcode shortening]\label{spec:trop-short}
Within the implementable range, the tropical base contraction \(X\mapsto \Trop_\lambda(X)\) induces, for \(\lambda'\le \lambda\), filtered maps
\[
  \mathcal{G}_A(\Trop_{\lambda'}X)\ \longrightarrow\ \mathcal{G}_A(\Trop_{\lambda}X)
\]
that uniformly shorten degreewise barcodes at a factor \(\kappa(\lambda',\lambda)\le 1\) up to f.q.i.
Consequently, for fixed \(\tau\) the truncated energies \(\mathrm{PE}_i^{\le\tau}\) are non\hyp increasing along \(\lambda\searrow 0\), and strictly decreasing whenever \(\kappa(\lambda',\lambda)<1\) on a subset of bars whose cumulative length within the \(\tau\)\nobreakdash window has positive proportion of the total windowed bar length.
\end{declaration}

\subsection*{8.2. Weak group collapse: proxies on automorphism groupoids}
\begin{definition}[Automorphism groupoid and linear proxies]\label{def:group-proxy}
Let \(\mathsf{Aut}(F)\) be the groupoid of filtered self\hyp maps of \(F\) in \(\FiltCh{k}\).
Fix a scale \(\tau\) and degree \(i\).
Choose an interval\hyp decomposition model (up to f.q.i.) of the truncated persistence module
\[
  \Ttau\mathbf{P}_i(F)\ \cong\ \bigoplus_{b\in \mathcal{B}_{i,\tau}(F)} I_b,
\]
where \(\mathcal{B}_{i,\tau}(F)\) is the multiset of bars surviving under \(\Ttau\).
Define the \emph{barcode vector space}
\[
  V_{i,\tau}\ :=\ \bigoplus_{b\in \mathcal{B}_{i,\tau}(F)} k\cdot e_b,
\]
one basis vector \(e_b\) per bar \(b\) (so \(\dim_k V_{i,\tau}=\#\,\mathcal{B}_{i,\tau}(F)\), counted with multiplicity).
Any \(g\in \mathsf{Aut}(F)\) induces an automorphism of \(\Ttau\mathbf{P}_i(F)\), hence (after choosing a decomposition) a linear map on \(V_{i,\tau}\), well\hyp defined up to conjugacy.
This yields a functor (well\hyp defined up to conjugacy)
\[
  \rho_{i,\tau}:\ \mathsf{Aut}(F)\longrightarrow \mathsf{GL}(V_{i,\tau}).
\]
For a finite generating set \(S\subset \mathsf{Aut}(F)\) we define the \emph{linear proxies}:
\begin{itemize}
  \item \emph{spectral radius bound} \(\rho_{\max,i,\tau}(S):=\sup_{g\in S}\ \mathrm{spr}\big(\rho_{i,\tau}(g)\big)\) (computed over an algebraic closure of \(k\) if needed);
  \item \emph{unipotent length} \(\mathrm{nilp}_{i,\tau}(S)\): the smallest \(m\) such that for every \(g\in S\), the unipotent part of \(\rho_{i,\tau}(g)\) satisfies \((\rho_{i,\tau}(g)-I)^{m}=0\).
\end{itemize}
We call \(F\) \emph{weakly group\hyp collapsed at scale \(\tau\)} if for some finite \(S\),
\[
  \rho_{\max,i,\tau}(S)\le 1\ \text{ for all \(i\)},\qquad \sup_i \mathrm{nilp}_{i,\tau}(S)<\infty.
\]
This expresses the intuition that, \emph{after truncation by \(\Ttau\)}, the linear action on the bar basis is \emph{semi\hyp contractive} (spectral radius \(\le 1\)) with \emph{finite\hyp length unipotence} uniformly in \(i\).
\end{definition}

\begin{remark}[Meaning of non\hyp expansive]\label{rk:nonexp-meaning}
Here “non\hyp expansive’’ means the induced maps on \(\Ttau\mathbf{P}_i(\,{\cdot}\,)\) are \(1\)\nobreakdash Lipschitz for the interleaving distance \(d_{\mathrm{int}}\) for all degrees \(i\) and the fixed scale \(\tau\).
\end{remark}

\begin{remark}[Independence up to conjugacy]\label{rk:conj-inv}
The construction of \(V_{i,\tau}\) uses a choice of interval decomposition; different choices yield conjugate linear representations.
The quantities \(\rho_{\max,i,\tau}(S)\) and \(\mathrm{nilp}_{i,\tau}(S)\) are conjugacy invariants and hence well\hyp defined up to f.q.i.
\end{remark}

\begin{declaration}[Specification: Shortening \(\Rightarrow\) weak group collapse]\label{spec:short-to-weak}
Assume \((\mu_{\mathrm{Collapse}},\nu_{\mathrm{Collapse}})=(0,0)\) at scale \(\tau\) and that \(F\to F'\) uniformly shortens barcodes at factor \(\kappa<1\) up to f.q.i.\ (Def.~\ref{def:shorten}).
Then for any finite \(S\subset \mathsf{Aut}(F')\) consisting of non\hyp expansive maps,
\[
  \rho_{\max,i,\tau}(S)\le 1\quad \text{and}\quad \mathrm{nilp}_{i,\tau}(S)\ \text{is uniformly bounded in \(i\)}.
\]
Hence \(F'\) is weakly group\hyp collapsed at scale \(\tau\).
\emph{All quantities are \(\tau\)\nobreakdash dependent; weak group collapse is certified on each monitored window separately.}
\end{declaration}

\begin{remark}[Guard\hyp rails]\label{rk:guard}
The notion above is a \emph{persistence\hyp level proxy}; no claim is made about abstract group trivialization.
No equivalence \(\mathrm{PH}_1\Leftrightarrow\Ext^1\) is used; categorical checks remain one\hyp way as in Part~I.
\end{remark}

\subsection*{8.3. Mirror transfer of indicators and pipeline error budget}
\begin{declaration}[Spec–Mirror non\hyp expansion and $2$-cell]\label{spec:mirror}
Assume a functor \(\Mirror\) that is non\hyp expansive for each \(\mathbf{P}_i\) and a natural transformation \(\Mirror\!\circ\! \Ctau \Rightarrow \Ctau\!\circ\!\Mirror\) \emph{up to filtered quasi\hyp isomorphism}.
Then all persistence\hyp level indicators after \(\Ttau\) transfer with non\hyp expansive bounds (Appendix~L).
\emph{All statements are at the persistence layer in the constructible range, up to f.q.i.; equalities are asserted only at the persistence layer.}
\end{declaration}

\begin{theorem}[Pipeline error budget (Mirror $\times$ Collapse)]\label{thm:pipe-budget-8}
Let \(U_m,\dots,U_1\) be A\hyp side steps (each deletion\hyp type or \(\varepsilon\)-continuation), with interleaved collapses \(C_{\tau_j}\).
Assume natural $2$-cells \(\epsilon_{i,\tau_j}:\Mirror\circ C_{\tau_j}\Rightarrow C_{\tau_j}\circ \Mirror\) with bounds \(\delta_j(i,\tau_j)\).
Then, for any fixed B\hyp side threshold \(\tau\) and any degree \(i\),
\[
d_{\mathrm{int}}\!\Big(\n\mathbf{T}_\tau \mathbf{P}_i\big(\Mirror(C_{\tau_m}U_m\cdots C_{\tau_1}U_1F)\big),\ \n\mathbf{T}_\tau \mathbf{P}_i\big(C_{\tau_m}U_m\cdots C_{\tau_1}U_1\Mirror F\big)\n\Big)\ \le\ \sum_{j=1}^m \delta_j(i,\tau_j).
\]
Post\hyp processing by $1$-Lipschitz persistence maps (shifts $S^\varepsilon$, further truncations $\mathbf{T}_{\tau'}$, degree projections $\mathbf{P}_i$) does not increase the bound.
\end{theorem}

\begin{remark}[Soft\hyp commuting and $\delta^{\mathrm{alg}}$ accounting]\label{rk:soft-comm-8}
If a second reflector $T_B$ (e.g.\ birth\hyp window deletion) is involved together with the length\hyp threshold reflector $T_\tau$, run the A/B commutativity test (Chapter~5, Definition~\ref{def:soft-commute}) on \(\Ttau\mathbf{P}_i(F)\) per window and degree.
If \(\Delta_{\mathrm{comm}}\le \eta\) we accept soft\hyp commuting; otherwise we fix a deterministic order and record \(\Delta_{\mathrm{comm}}\) into $\delta^{\mathrm{alg}}$.
\end{remark}

{\let\n\relax \def\n{\unskip\space}
\begin{remark}[Mirror transfer with quantitative commutation]\label{rk:mirror-quant}
Under Declaration~\ref{spec:mirror}, \emph{in the constructible range and up to isomorphism in \(\Pers^{\mathrm{ft}}_k\)} there are natural identifications, for fixed \(\tau\) and all \(i\),
\[
  \mathbf{P}_i(\Mirror(\Ctau F))\ \cong\ \mathbf{P}_i(\Ctau(\Mirror F))\ \cong\ \Ttau\,\mathbf{P}_i(\Mirror F),
\]
and the truncated interleaving distances satisfy
\[
  d_{\mathrm{int}}\!\big(\Ttau\mathbf{P}_i(\Mirror F),\,\Ttau\mathbf{P}_i(\Mirror G)\big)\n  \ \le\\n  d_{\mathrm{int}}\!\big(\mathbf{P}_i(F),\,\mathbf{P}_i(G)\big).
\]
If, in addition, \(\Mirror\) is \(1\)\nobreakdash Lipschitz, then for the same input \(F\),
\[
d_{\mathrm{int}}\!\big(\Ttau\,\mathbf{P}_i\big((\Mirror\!\circ\! \Ctau)F\big),\,\Ttau\,\mathbf{P}_i\big((\Ctau\!\circ\!\Mirror)F\big)\big)=0
\]
up to f.q.i., and for possibly different inputs \(F,G\) one has
\begin{align}
& d_{\mathrm{int}}\!\big(\Ttau\,\mathbf{P}_i((\Mirror\!\circ\! \Ctau)F),\,\Ttau\,\mathbf{P}_i((\Ctau\!\circ\!\Mirror)G)\big)
\notag\\[-2pt]\n&\qquad\le\ d_{\mathrm{int}}\!\big(\mathbf{P}_i(F),\,\mathbf{P}_i(G)\big)
     \;+\; d_{\mathrm{int}}\!\big(\mathbf{P}_i(\Mirror F),\,\mathbf{P}_i(\Mirror G)\big)\n\label{eq:mirror-ctau-stability}\\[-2pt]\n&\qquad\le\ 2\,d_{\mathrm{int}}\!\big(\mathbf{P}_i(F),\,\mathbf{P}_i(G)\big).\n\notag\n\end{align}\nHence the indicators \(\{\mathrm{PE}_i^{\le\tau},\ \mathrm{ST}_{\beta}^{\ge M(\tau)},\ \Ext^1(\Rfun(\Ctau{-}),Q)\}\) are preserved across \(\Mirror\) with non\hyp expansive bounds (Appendix~L).\n\end{remark}\n}\n\n\begin{conjecture}[Mirror correspondences under collapse monitoring]\label{conj:mirror}\nAssuming \((\mu_{\mathrm{Collapse}},\nu_{\mathrm{Collapse}})=(0,0)\) and \LC, mirror correspondences preserve the monitored indicators and propagate weak group collapse (Def.~\ref{def:group-proxy}) across \(\Mirror\) along the same \(\tau\)\nobreakdash range.\n\end{conjecture}\n\n\subsection*{8.3.1.\ Spec–Saturation gate (tropical/mirror)}\n\begin{declaration}[Saturation gate \textbf{[Spec]} (tropical/mirror)]\label{gate:8-saturation}\nFix \(\tau^\ast>0\) and parameters \(\eta,\delta>0\).\nOn the window \([0,\tau^\ast]\), assume:\n(i) eventually the maximal \emph{finite} bar length in \(\mathbf{T}_{\tau^\ast}\mathbf{P}_i(F_\lambda)\) is \(\le \eta\);\n(ii) eventually \(d_{\mathrm{int}}\!\big(\mathbf{T}_{\tau^\ast}\mathbf{P}_i(F_\lambda),\mathbf{T}_{\tau^\ast}\mathbf{P}_i(F_{\lambda'})\big)\le \eta\);\n(iii) the \emph{edge gap} to the window, \(\delta:=\tau^\ast-\max\{b_r<\tau^\ast\}\), satisfies \(\delta>\eta\).\nThen, \textbf{within this window only}, we adopt the temporary policy\n\[\n\mathrm{PH}_1(\C_{\tau^\ast}F_\lambda)=0\quad\Longleftrightarrow\quad \Ext^1(\Rfun(\C_{\tau^\ast}F_\lambda),k)=0.
\]
This chapter references the gate; quantitative verification and usage are centralized in Chapter~11.
\end{declaration}

\subsection*{8.4. Monitoring protocol for tropical/mirror flows}
\begin{declaration}[Specification: Monitoring protocol]\label{spec:prot}
Fix a sweep \(\lambda\searrow 0\) and a finite set of scales \(\tau\).
For each sample:
\begin{enumerate}
  \item \emph{Compute and record} \(\Ttau\mathbf{P}_i\big(\mathcal{G}_A(\Trop_\lambda X)\big)\) and \(\mathrm{PE}_i^{\le\tau}\) on the truncated barcodes (equivalently on \(\Ctau\)).
  \item \emph{Compute and record} spectral indicators \(\mathrm{ST}_{\beta}^{\ge M(\tau)}\), \(\mathrm{HT}(t;\cdot)\) on \(L(\Ctau(\mathcal{G}_A(\Trop_\lambda X)))\) with a fixed \((\beta,M(\tau),t)\) policy.
  \item \emph{Check} \(\Ext^1\big(\Rfun(\Ctau{-}),Q\big)=0\) for \(Q\in\Qtest\).
  \item \emph{Evaluate} \((\mu_{\mathrm{Collapse}},\nu_{\mathrm{Collapse}})\) from the \(\lambda\)\nobreakdash tower via comparison maps at \(\tau\) (Remark~\ref{rk:8-cone}); \emph{these obstructions are invariant under filtered quasi\hyp isomorphisms and under cofinal reindexing} (Appendix~J).
  \item \emph{Group proxies:} choose a finite \(S\subset\mathsf{Aut}\big(\mathcal{G}_A(\Trop_\lambda X)\big)\) (e.g.\ monodromies/symmetries), and record \(\rho_{\max,i,\tau}(S)\) and \(\mathrm{nilp}_{i,\tau}(S)\) on \(V_{i,\tau}\).
  \item \emph{Mirror transfer:} apply \(\Mirror\) and repeat (1)–(5) on the mirror side.
\end{enumerate}
The \emph{stable regime} at \(\tau\) is where \((\mu_{\mathrm{Collapse}},\nu_{\mathrm{Collapse}})=(0,0)\) and indicators in (1)–(3) are non\hyp increasing along \(\lambda\searrow 0\);
\emph{weak group collapse} is declared when (5) meets the thresholds in Def.~\ref{def:group-proxy}.
\end{declaration}

\subsection*{8.5. Diagram (tropical flow, indicators, mirror transfer)}
\begin{center}
\begin{tikzcd}[
  ampersand replacement=\&,
  column sep=1.6em, row sep=2.0em,
  cells={nodes={font=\footnotesize}},
  every label/.append style={font=\scriptsize},
  scale=0.9, transform shape
]
X \arrow[r, "\mathcal{G}_A"] \arrow[d, "{\Trop_\lambda}"']
  \& F \arrow[r, "\mathbf{P}_i"] 
    \& \mathbf{P}_i(F) \arrow[r, "\Ttau"] \arrow[d, dashed, "{\mathrm{PE}_i^{\le \tau}}"']
      \& \Ttau\mathbf{P}_i(F) \arrow[d, dashed, "{\mathrm{PE}_i^{\le \tau}}"] \arrow[r, dashed, "{\Mirror}"] 
        \& \Ttau\mathbf{P}_i(\Mirror F) \arrow[d, dashed, "{\mathrm{PE}_i^{\le \tau}}"] \\\n\Trop_\lambda X \arrow[r, "\mathcal{G}_A"']
  \& F_\lambda \arrow[r, "\mathbf{P}_i"']
    \& \mathbf{P}_i(F_\lambda) \arrow[r, "\Ttau"']
      \& \Ttau\mathbf{P}_i(F_\lambda) \arrow[r, dashed, "{\Mirror}"']
        \& \Ttau\mathbf{P}_i(\Mirror F_\lambda) \\\n \& {} \& \text{Spectral } L(\Ctau F) \arrow[r, dashed, "{\text{heat trace on }L(\Ctau F)}"] \&
    {\mathrm{HT},\ \mathrm{ST}_{\beta}} \& \\\n\& \Rfun(F_\lambda) \arrow[uu, bend left=40, dashed, "\LC"] 
    \arrow[rr, dashed, "{\Ctau\ \text{ then }\Ext^1(-,Q)}"']
      \&\& \Ext^1(\Rfun(\Ctau F_\lambda),Q)=0 \\\n\& \text{group proxies on }V_{i,\tau}
   \arrow[uu, dashed, swap, "{\rho_{\max},\ \mathrm{nilp}}", shift left=1.5ex, pos=0.55]
\end{tikzcd}
\end{center}

\subsection*{8.6. Toy instance (persistence layer)}
\begin{example}[Uniform shortening under tropical scaling]\label{ex:trop}
Let \(\mathbf{P}_i(F)\) have bars of lengths \(\{\ell_j\}_j\) in \([0,\tau]\).
Suppose \(\Trop_\lambda\) induces \(\ell_j\mapsto \ell'_j\) with \(\ell'_j\le \kappa\ell_j\) for a fixed \(\kappa<1\) on a subset \(S\) of bars whose cumulative length within the \(\tau\)\nobreakdash window satisfies \(\sum_{j\in S}\ell_j>0\)
(equivalently, a positive fraction of the total windowed bar length), and deletions otherwise.
Then \(\mathrm{PE}_i^{\le\tau}\) strictly decreases and, if \((\mu_{\mathrm{Collapse}},\nu_{\mathrm{Collapse}})=(0,0)\) at \(\tau\), Def.~\ref{def:group-proxy} yields weak group collapse for non\hyp expansive automorphisms.
\end{example}

\subsection*{8.7. Final guard\hyp rails}
\begin{remark}[Scope and non\hyp claims]\label{rk:8-guard}
All claims are at the persistence/spectral/categorical layers within the implementable range, with \LC\ in force when comparing after realization.
No number\hyp theoretic or group\hyp theoretic decision is asserted; “weak group collapse’’ is a persistence\hyp level \emph{linear} proxy.
No claim of \(\mathrm{PH}_1\Leftrightarrow\Ext^1\) is made; only the one\hyp way bridge of Part~I is used.
The obstruction \(\mu_{\mathrm{Collapse}}\) is unrelated to the classical Iwasawa \(\mu\)\nobreakdash invariant.
\end{remark}
```


% ============================================================
% Chapter 9: Langlands Collapse (Three Layers)
% ============================================================
\section{Chapter 9: Langlands Collapse (Three Layers)}
\addcontentsline{toc}{section}{Langlands Collapse (Three Layers)}

\noindent\textbf{Scope note (PF/BC, truncation, and comparison order).}
Projection Formula and Base Change (PF/BC) are referenced from Appendix~N and are applied \emph{objectwise in \(t\)} in the functor category \([\mathbb{R},\mathsf{Vect}_k]\).
All comparisons are then made \emph{at the persistence layer after truncation by \(\Ttau\)}.
Concretely, for any morphism of filtered objects computed via PF/BC, the standard operating procedure is:
\[
\text{for each } t \ \Longrightarrow\ \text{apply } \mathbf{P}_i \ \Longrightarrow\ \text{apply } \Ttau \ \Longrightarrow\ \text{compare in } \Perskft.
\]
All inter\hyp layer equalities and Lipschitz statements are asserted only \emph{after truncation} and hold up to f.q.i.\ (pseudonatural, not strict).
When energies/indicators are involved, we require a \emph{fixed window, fixed \(\tau\), and a fixed \(\delta\)\nobreakdash policy} on both sides of any comparison; see Remark~\ref{rk:9-delta-policy}.
Endpoint conventions and infinite bars follow Chapter~2, Remark~\ref{rk:2-endpoints}.

\begin{remark}[Monotonicity convention]
Throughout this chapter we adopt the convention of
Chapter~6, Remark~\ref{rem:stability-vs-monotonicity}:
\emph{deletion\hyp type} updates are non\hyp increasing for spectral tails and windowed energies,
while \emph{inclusion\hyp type} updates are only stable (non\hyp expansive); see Appendix~E.
\end{remark}

\subsection*{9.0. Standing hypotheses and admissible Langlands realizations}
We fix a base field \(k\) and work within the \emph{implementable range} of Part~I.
\emph{All statements in this chapter are made within the constructible range}
(we identify \(\Perskft\) with the constructible subcategory as in Chapter~6).
Let \(\FiltCh{k}\) denote finite\hyp type filtered chain complexes and
\(\mathbf{P}_i:\FiltCh{k}\to\Perskft\) the degreewise persistence functor; we write
\(\Ttau:=\mathbf{T}_\tau\) for the Serre (bar\hyp deletion) reflector at scale \(\tau\ge 0\),
and use its filtered lift \(\Ctau\) \emph{up to filtered quasi\hyp isomorphism} (Chapter~2, §§2.2–2.3).
A fixed \(t\)\hyp exact realization \(\Rfun:\FiltCh{k}\to D^{\mathrm{b}}(k\text{-mod})\) is retained, and the lifting–coherence hypothesis \(\LC\) is assumed when comparing \(\Ctau\) with \(\tau_{\ge 0}\!\circ\!\Rfun\).
\emph{Equalities and Lipschitz statements are asserted only at the persistence layer; at the filtered\hyp complex layer they hold up to filtered quasi\hyp isomorphism.}

We consider three data layers
\[
\mathsf{Gal}\ \xrightarrow{\ \mathsf{Trans}\ }\ \mathsf{Par}\ \xrightarrow{\ \mathsf{Funct}\ }\ \mathsf{Aut},
\]
heuristically “Galois \(\to\) Transfer \(\to\) Functoriality”.
An \emph{admissible Langlands triple of realizations} consists of functors
\[
\mathcal{L}_{\mathrm{Gal}},\ \mathcal{L}_{\mathrm{Tr}},\ \mathcal{L}_{\mathrm{Aut}}:\ \FiltCh{k}\longrightarrow\FiltCh{k},
\]
subject to:
\begin{itemize}
  \item \textbf{Non\hyp expansiveness.} Each layer induces filtered maps whose images under every \(\mathbf{P}_i\) are \(1\)\nobreakdash Lipschitz for the interleaving distance; deletion\hyp type updates (Appendix~E) yield non\hyp increase of windowed indicators up to f.q.i., while inclusion\hyp type updates guarantee only stability.
  \item \textbf{Compatibility with \(\Ctau\).} For each degree \(i\) there are natural identifications in \(\Perskft\),
  \(\mathbf{P}_i(\Ctau{-})\cong \Ttau\,\mathbf{P}_i(-)\).
  \item \textbf{Finite\hyp type \& (co)limits.} Outputs are degreewise finite\hyp type; degreewise filtered (co)limits in \(\FiltCh{k}\) are computed objectwise in \([\mathbb{R},\mathsf{Vect}_k]\) and used only under the scope policy of Appendix~A (compute in the functor category and verify return to \(\Pers^{\mathrm{cons}}_k\)).
  \item \textbf{Realization coherence.} The \(t\)\hyp exact \(\Rfun\) is compatible with \(\LC\) across layers, so functorially up to f.q.i.,
  \(\Rfun(\Ctau F)\simeq \tau_{\ge 0}\,\Rfun(F)\).
\end{itemize}
Kernel/cokernel diagnostics \((\mu_{\mathrm{Collapse}},\nu_{\mathrm{Collapse}})\) are always taken \emph{after truncation}, with \(\dim_k\) interpreted as \emph{generic\hyp fiber} dimension (multiplicity of \(I[0,\infty)\)); see Appendix~D, Remark~\ref{rem:D-generic-dim}.

\begin{remark}[Operational comparison policy (PF/BC and collapse ordering)]\label{rk:9-operational}
Given any PF/BC computation (Appendix~N) producing a morphism or correspondence at the filtered\hyp complex level, we adopt the mandatory comparison order:
\[
\text{(i) objectwise in \(t\)}\ \Longrightarrow\ \text{(ii) apply }\mathbf{P}_i\ \Longrightarrow\ \text{(iii) apply }\Ttau\ \Longrightarrow\ \text{(iv) compare in }\Perskft.
\]
All “same\hyp scale” claims are checked with the \emph{same window}, the \emph{same \(\tau\)}, and the \emph{same \(\delta\)\nobreakdash policy}; see Remark~\ref{rk:9-delta-policy}.
\end{remark}

\begin{declaration}[Spec–Derived Langlands transfers]\label{spec:9-derived}
Besides \(\Rfun:\FiltCh{k}\!\to\!D^{\mathrm{b}}(k\text{-mod})\), we may use (as \textbf{[Spec]})
\[
\mathsf{R}_{\mathrm{coh}}:\ \FiltCh{k}\to D^{\mathrm{b}}\!\operatorname{Coh}(\mathfrak{X}),\qquad\n\mathsf{R}_{\acute{e}t}:\ \FiltCh{k}\to D^{\mathrm{b}}_{\mathrm{c}}(\mathfrak{Y}_{\acute{e}t},\Lambda)
\]
with \emph{field} coefficients \(\Lambda\), where \(\mathfrak{X}\), \(\mathfrak{Y}\) are parameter/automorphic stacks.
Projection Formula and Base Change are assumed \emph{exactly as tabulated in Appendix~N} (including proper/smooth hypotheses and the standard \(t\)\nobreakdash structure/degree normalizations).
Under these assumptions, the normalized transfers (degree\hyp normalized pullback/pushforward, kernel transforms/Hecke correspondences satisfying PF/BC, normalized parabolic induction/Jacquet) are \emph{non\hyp expansive at the persistence layer after truncation}: for each \(i,\tau\) and any such transfer \(\Phi\),
\[
d_{\mathrm{int}}\!\big(\Ttau\mathbf{P}_i(\Phi F),\ \Ttau\mathbf{P}_i(\Phi G)\big)\ \le\ d_{\mathrm{int}}\!\big(\Ttau\mathbf{P}_i(F),\ \Ttau\mathbf{P}_i(G)\big).
\]
All claims in this declaration are persistence\hyp level \emph{specifications}; the bridge \(\mathrm{PH}_1\Rightarrow\Ext^1\) is proved only in \(D^{\mathrm{b}}(k\text{-mod})\).
\end{declaration}

\begin{declaration}[Deletion\hyp type operations (PDE)]\label{spec:9-pde}
Inter\hyp /intra\hyp layer maps implemented by PDE\hyp style operations satisfying Appendix~E
(Dirichlet restriction/absorbing boundaries, positive semidefinite Loewner contractions, principal submatrices/Schur complements)
are treated as \emph{deletion\hyp type} and make windowed energies and spectral tails \emph{non\hyp increasing} after truncation.
Inclusion\hyp type updates are asserted only to be \emph{stable} (non\hyp expansive).
\end{declaration}

\begin{remark}[Endpoints and infinite bars]\label{rk:9-endpoints}
Open/closed endpoint conventions are immaterial; infinite bars are not removed by \(\Ttau\) and are clipped by windowed indicators (cf.\ Chapter~6).
\end{remark}

\begin{remark}[Indexing and cone extension]\label{rk:9-cone}
Let \(I\) be a directed index (e.g.\ level, conductor, degree of base\hyp change, auxiliary height).
Adjoin a terminal element \(\infty\) and cone maps \(t\to \infty\) in \(I\cup\{\infty\}\).
Under the realizations, these yield filtered maps \(F_t\to F_\infty\) per layer/degree, providing the comparison maps for \((\mu_{\mathrm{Collapse}},\nu_{\mathrm{Collapse}})\) at fixed~\(\tau\) (Chapter~4).
\end{remark}

\begin{remark}[Inter\hyp layer comparison and pseudonaturality]\label{rk:9-compare}
To speak of commutativity “up to f.q.i.” across layers after truncation, fix comparison natural transformations
\[
\alpha:\ \mathcal{L}_{\mathrm{Tr}}\circ \mathsf{Trans}\ \Longrightarrow\ \mathcal{L}_{\mathrm{Gal}},\qquad\n\beta:\ \mathcal{L}_{\mathrm{Aut}}\circ \mathsf{Funct}\ \Longrightarrow\ \mathcal{L}_{\mathrm{Tr}},
\]
which become filtered quasi\hyp isomorphisms degreewise after applying \(\Ctau\).
Equivalently, at the persistence layer there are natural isomorphisms
\[
\Ttau\mathbf{P}_i\big(\mathcal{L}_{\mathrm{Tr}}\!\circ\!\mathsf{Trans}(-)\big)\ \cong\ \Ttau\mathbf{P}_i\big(\mathcal{L}_{\mathrm{Gal}}(-)\big),\quad\n\Ttau\mathbf{P}_i\big(\mathcal{L}_{\mathrm{Aut}}\!\circ\!\mathsf{Funct}(-)\big)\ \cong\ \Ttau\mathbf{P}_i\big(\mathcal{L}_{\mathrm{Tr}}(-)\big),
\]
and these assemble to \emph{pseudonatural equivalences} between the truncated persistence\hyp level functors.
\end{remark}

\begin{remark}[Unified \(\delta\)\nobreakdash policy: budgets, aggregation, and A/B testing]\label{rk:9-delta-policy}
We fix once and for all a \emph{\(\delta\)\nobreakdash policy} at scale \(\tau\), denoted \(\boldsymbol{\delta}=(\delta_{\mathrm{int}},\delta_{\mathrm{win}},\delta_{\mathrm{spec}})\), governing tolerances for:
\begin{itemize}
  \item interleaving distance comparisons, \(\delta_{\mathrm{int}}\),
  \item windowed energy measurements (discretization/rounding/finite sampling), \(\delta_{\mathrm{win}}\),
  \item spectral/heat normalization differences, \(\delta_{\mathrm{spec}}\).
\end{itemize}
For a composite along the three layers, we \emph{aggregate} per\hyp step budgets additively:
\[
\boldsymbol{\delta}_{\mathrm{tot}}\ =\ \boldsymbol{\delta}^{(\mathsf{Gal}\to\mathsf{Par})}\ +\ \boldsymbol{\delta}^{(\mathsf{Par}\to\mathsf{Aut})}.
\]
All equality/commutativity checks after truncation are validated with the \emph{same window and \(\tau\)} under the \emph{same} \(\boldsymbol{\delta}_{\mathrm{tot}}\).
Pseudonaturality is \emph{A/B tested} after collapse: for any two paths \(\gamma_A,\gamma_B\) between the same endpoints in the three\hyp layer diagram, the outputs
\(\Ttau\mathbf{P}_i(\gamma_A(-))\) and \(\Ttau\mathbf{P}_i(\gamma_B(-))\) are compared in \(\Perskft\) with tolerance \(\delta_{\mathrm{int,tot}}\), and their indicators are compared with \(\delta_{\mathrm{win,tot}},\delta_{\mathrm{spec,tot}}\).
When \(\boldsymbol{\delta}_{\mathrm{tot}}=\mathbf{0}\), equality is taken in \(\Perskft\) (up to isomorphism).
\end{remark}

\subsection*{9.1. Persistence\hyp layer interface for the three layers}
For an object \(x\) in a given layer and realization \(F=\mathcal{L}_{\ast}(x)\), we monitor degreewise
\[
\Ttau\,\mathbf{P}_i(F),\qquad\n\mathrm{PE}_i^{\le \tau}(F),\qquad\n\mathrm{ST}_{\beta_{\mathrm{spec}}}^{\ge M(\tau)}(F),\ \mathrm{HT}(s;F),\qquad\n\Ext^1\big(\Rfun(\Ctau F),Q\big)=0\ (Q\in\Qtest).
\]
\emph{Energies \(\mathrm{PE}_i^{\le\tau}\) are computed on the truncated barcodes \(\Ttau\mathbf{P}_i(F)\) (equivalently on \(\Ctau F\)).
Spectral indicators are computed on \(L(\Ctau F)\) and treated as \emph{stable under a fixed normalization policy} (not f.q.i.\ invariants; cf.\ Chapter~11).}
(Here \(\beta_{\mathrm{spec}}>0\) denotes the spectral tail exponent, to avoid symbol clash with other layer parameters.)
All metric comparisons in this interface are made under the fixed \(\boldsymbol{\delta}\)\nobreakdash policy of Remark~\ref{rk:9-delta-policy}.

\subsection*{9.2. Diagnostics along the index \(I\)}
For a fixed layer and degree \(i\), at scale \(\tau\) define
\[
\phi_{i,\tau}:\ \varinjlim_{t\in I}\ \Ttau\,\mathbf{P}_i(F_t)\ \longrightarrow\ \Ttau\,\mathbf{P}_i(F_\infty),
\]
\[
\mu_{i,\tau}:=\dim_k\ker\phi_{i,\tau}\footnote{Here \(\dim_k\) denotes the \emph{generic\hyp fiber} dimension after truncation, i.e.\ the multiplicity of \(I[0,\infty)\) summands; see Appendix~D, Remark~\ref{rem:D-generic-dim}.},\qquad\n\nu_{i,\tau}:=\dim_k\mathrm{coker}(\phi_{i,\tau}),
\]
and \(\muc:=\sum_i\mu_{i,\tau}\), \(\nuc:=\sum_i\nu_{i,\tau}\) (finite by bounded degrees).
\emph{We suppress the explicit \(\tau\)\nobreakdash dependence of the totals when the scale is clear from context.}
Moreover, the obstructions \((\muc,\nuc)\) are \emph{invariant under filtered quasi\hyp isomorphisms and under cofinal reindexing of the tower} (Appendix~J).

\begin{remark}[Window/scale/metric alignment for diagnostics]\label{rk:9-alignment}
Whenever \(\phi_{i,\tau}\) is formed, both the colimit and target are computed at the same \(\tau\) with the same window convention, and all distances/energies/indicators are evaluated with the same \(\boldsymbol{\delta}\)\nobreakdash policy.
This alignment is mandatory for the failure log in §\ref{sec:9.5}.
\end{remark}

\subsection*{9.3. Propagation across the three layers}
\begin{declaration}[Specification: Propagation diagram under \((\muc,\nuc)=(0,0)\)]\label{spec:9-prop}
Assume \((\muc,\nuc)=(0,0)\) at a fixed \(\tau\) for the three layers, \(\LC\), and the inter\hyp layer comparison data of Remark~\ref{rk:9-compare}.
Then
\[
\mathsf{Gal}\ \xrightarrow{\ \mathsf{Trans}\ }\ \mathsf{Par}\ \xrightarrow{\ \mathsf{Funct}\ }\ \mathsf{Aut}
\]
commutes, \emph{after truncation}, up to isomorphism in \(\Perskft\) in each degree \(i\): along any path of layer maps, the induced morphisms on \(\Ttau\,\mathbf{P}_i(-)\) agree up to isomorphism in \(\Perskft\).
Any obstruction to commutativity at this level is detected as kernels/cokernels in the monitored degrees and is recorded by \((\muc,\nuc)\).
Moreover, under the unified \(\delta\)\nobreakdash policy, any residual slack is accounted for by \(\boldsymbol{\delta}_{\mathrm{tot}}\); when \(\boldsymbol{\delta}_{\mathrm{tot}}=\mathbf{0}\), comparisons are strict (up to isomorphism) in \(\Perskft\).
\end{declaration}

\begin{remark}[Excluding Type~IV at fixed \(\tau\)]\label{rk:9-typeIV}
By Declaration~\ref{spec:9-prop}, \((\muc,\nuc)=(0,0)\) at fixed \(\tau\) excludes Type~IV at that scale within each layer and along their composites.
\end{remark}

\subsection*{9.4. Monitoring protocol (Langlands three\hyp layer)}
\begin{declaration}[Specification: Joint monitoring protocol]\label{spec:9-protocol}
Fix scales \(\tau\in[\tau_{\min},\tau_{\max}]\) and an index range \(t\in I\).
Fix a window convention and a \(\boldsymbol{\delta}\)\nobreakdash policy.
For each layer \(\ast\in\{\mathrm{Gal},\mathrm{Tr},\mathrm{Aut}\}\) and sample \(t\):
\begin{enumerate}
  \item \emph{Compute and record} \(\Ttau\,\mathbf{P}_i(F^{(\ast)}_t)\) and truncated energies \(\mathrm{PE}_i^{\le \tau}\) on \(\Ttau\,\mathbf{P}_i(F^{(\ast)}_t)\) (equivalently on \(\Ctau F^{(\ast)}_t\)), with window/\(\tau\) fixed and window tolerances within \(\delta_{\mathrm{win}}\).
  \item \emph{Compute and record} spectral indicators on \(L(\Ctau F^{(\ast)}_t)\) with a fixed \((\beta_{\mathrm{spec}},M(\tau),s_{\mathrm{HT}})\) policy (heat time \(s_{\mathrm{HT}}>0\)), with normalization tolerances within \(\delta_{\mathrm{spec}}\).
  \item \emph{Check} \(\Ext^1\big(\Rfun(\Ctau F^{(\ast)}_t),Q\big)=0\) for \(Q\in\Qtest\) under the same truncation \(\Ctau\).
  \item \emph{Evaluate} \((\muc,\nuc)\) via the comparison maps \(\phi_{i,\tau}\) along \(t\in I\) (Remark~\ref{rk:9-alignment}); record failure types (pure/mixed). Interleaving distances are compared within \(\delta_{\mathrm{int}}\).
  \item \emph{Cross\hyp layer check.} Verify non\hyp expansiveness of layer transitions under \(\mathbf{P}_i\) and compatibility with \(\Ctau\) up to f.q.i.; test commutativity up to f.q.i.\ on \(\Ttau\,\mathbf{P}_i(-)\).
  If the comparison transformations of Remark~\ref{rk:9-compare} fail after truncation beyond \(\boldsymbol{\delta}_{\mathrm{tot}}\), record a Type~III (spec\hyp mismatch) entry in the failure log.
\end{enumerate}
The \emph{collapse\hyp stable regime} at \(\tau\) is declared where \((\muc,\nuc)=(0,0)\) simultaneously across layers and (1)–(3) hold jointly within the \(\boldsymbol{\delta}\)\nobreakdash policy.
\end{declaration}

\subsection*{9.5. Diagram (three layers, indicators, obstructions)}\label{sec:9.5}
\begin{center}
\begin{tikzcd}[
  column sep=5.5em, row sep=1.8em,
  cells={nodes={font=\footnotesize}},
  every label/.append style={font=\scriptsize},
  scale=0.9, transform shape
]
\text{Galois} \arrow[r, "\mathsf{Trans}"]
  & \text{Transfer} \arrow[r, "\mathsf{Funct}"]
    & \text{Automorphic} \\\n\mathcal{L}_{\mathrm{Gal}} \arrow[r, dashed, "{\text{non-expansive},\ \Ctau\text{-compat}}"']
  & \mathcal{L}_{\mathrm{Tr}} \arrow[r, dashed, "{\text{non-expansive},\ \Ctau\text{-compat}}"']
    & \mathcal{L}_{\mathrm{Aut}} \\\n\mathbf{P}_i \arrow[r]
  & \mathbf{P}_i \arrow[r]
    & \mathbf{P}_i \\\n\Ttau\,\mathbf{P}_i \arrow[r, dashed, "{\cong\ \text{in }\Perskft\ \text{after truncation}}"]
  & \Ttau\,\mathbf{P}_i \arrow[r, dashed, "{\cong\ \text{in }\Perskft\ \text{after truncation}}"]
    & \Ttau\,\mathbf{P}_i \arrow[d, dashed, "\mathrm{PE}_i^{\le \tau}"] \\\n& & \text{heat trace on }L(\Ctau F^{(\ast)}_t),\ \ \Ext^1(\Rfun(\Ctau F^{(\ast)}_t),k)=0 \arrow[d, dashed] \\\n\varinjlim_{t}\,\Ttau\,\mathbf{P}_i(F_t) \arrow[rr, "\phi_{i,\tau}"]
  & & \Ttau\,\mathbf{P}_i(F_\infty) \arrow[dl, bend right=25, dashed, "{(\muc,\nuc)\ \text{log}}"'] \\\n& \text{failure log (pure/mixed)} &
\end{tikzcd}
\end{center}

\subsection*{9.6. Stability, non\hyp expansiveness, and \(\delta\)\nobreakdash aggregation}
\begin{declaration}[Layerwise non\hyp expansiveness with \(\delta\)\nobreakdash aggregation]\label{spec:9-delta-agg}
Let \(\Phi_1:\mathsf{Gal}\!\to\!\mathsf{Par}\) and \(\Phi_2:\mathsf{Par}\!\to\!\mathsf{Aut}\) be admissible layer maps (normalized transfers satisfying PF/BC per Appendix~N, or PDE\hyp style deletions per Appendix~E).
Then for each \(i,\tau\),
\[
d_{\mathrm{int}}\!\Big(\Ttau\mathbf{P}_i\big(\Phi_2\!\circ\!\Phi_1(F)\big),\ \Ttau\mathbf{P}_i\big(\Phi_2\!\circ\!\Phi_1(G)\big)\Big)\n\ \le\ d_{\mathrm{int}}\!\big(\Ttau\mathbf{P}_i(F),\ \Ttau\mathbf{P}_i(G)\big),
\]
and any empirical or normalization slack is bounded by the aggregated tolerance
\(\delta_{\mathrm{int,tot}}=\delta_{\mathrm{int}}^{(\Phi_1)}+\delta_{\mathrm{int}}^{(\Phi_2)}\).
For deletion\hyp type steps, windowed energies and spectral tails are non\hyp increasing after truncation; for inclusion\hyp type steps they are stable within the aggregated budgets
\(\delta_{\mathrm{win,tot}},\delta_{\mathrm{spec,tot}}\).
\end{declaration}

\subsection*{9.7. Conjectural stability of functorial transfer}
\begin{conjecture}[Stability of functorial transfer under collapse]\label{conj:9-stability}
Within the implementable range, assume non\hyp expansive layer maps, \(\LC\), and \((\muc,\nuc)=(0,0)\) for a scale interval in \(\tau\).
Then the functorial transfer maps are stabilized at that scale: persistence energies are non\hyp increasing in the deletion\hyp type case (Appendix~E) and stable in general, spectral indicators do not grow, and the categorical checks \(\Ext^1\big(\Rfun(\Ctau -),Q\big)=0\) persist across the three layers.
All comparisons are performed after truncation, with the same window, the same \(\tau\), and the same \(\boldsymbol{\delta}\)\nobreakdash policy; \(\boldsymbol{\delta}_{\mathrm{tot}}\) is the sum of per\hyp step budgets across layers.
No number\hyp theoretic identity is asserted.
\end{conjecture}

\subsection*{9.8. Guard\hyp rails, A/B testing, and non\hyp claims}
\begin{remark}[A/B pseudonaturality test after collapse]\label{rk:9-AB}
For two composites \(\gamma_A,\gamma_B\) through the three layers,
pseudonaturality after truncation is tested as follows:
\[
d_{\mathrm{int}}\!\Big(\Ttau\mathbf{P}_i(\gamma_A(F)),\ \Ttau\mathbf{P}_i(\gamma_B(F))\Big)\ \le\ \delta_{\mathrm{int,tot}},\quad\n\big|\mathrm{PE}_i^{\le\tau}(\gamma_A)-\mathrm{PE}_i^{\le\tau}(\gamma_B)\big|\ \le\ \delta_{\mathrm{win,tot}},
\]
and spectral/heat indicators differ by at most \(\delta_{\mathrm{spec,tot}}\), all with the same window and \(\tau\).
When \(\boldsymbol{\delta}_{\mathrm{tot}}=\mathbf{0}\), the two outputs are isomorphic in \(\Perskft\).
Failures beyond budget are logged as Type~III (spec\hyp mismatch) unless accounted for by \((\muc,\nuc)\).
\end{remark}

\begin{remark}[Scope and non\hyp claims]\label{rk:9-guard}
All statements are at the persistence/spectral/categorical layers and use only the one\hyp way bridge under (B1)–(B3) from Part~I; no claim of \(\mathrm{PH}_1\Leftrightarrow\Ext^1\) is made.
Obstructions are recorded by \((\muc,\nuc)\); these are unrelated to classical Iwasawa \(\mu\).
Saturation gates (binary \(\mathrm{PH}_1\)\(\Leftrightarrow\)\(\Ext^1\) policies) are organized in Chapter~11.
This chapter provides a design/specification blueprint and does not decide Langlands correspondence.
\end{remark}

% ----------------------
% Appendix-style addenda (local to Chapter 9, optional in build)
% ----------------------
\subsection*{9.A. Local clarifications for implementers (non-normative)}
\begin{remark}[Windows and energies]\label{rk:9-windows}
A window convention is a choice of measurable subset \(W_\tau\subset\mathbb{R}_{\ge 0}\) (e.g.\ \([0,\tau]\) with endpoint policy) used to compute \(\mathrm{PE}_i^{\le\tau}\) on \(\Ttau\mathbf{P}_i(F)\).
All cross\hyp layer energy comparisons must share the same \(W_\tau\) and endpoint policy.
\end{remark}

\begin{remark}[Spectral normalization]\label{rk:9-spec-norm}
Spectral indicators (tails \(\mathrm{ST}_{\beta_{\mathrm{spec}}}^{\ge M(\tau)}\), heat traces \(\mathrm{HT}(s;{-})\)) are computed on \(L(\Ctau F)\) with a fixed normalization (e.g.\ reference operator, scaling of time \(s\), and tail threshold \(M(\tau)\)).
Stability is asserted only with respect to that fixed normalization and within \(\delta_{\mathrm{spec}}\).
\end{remark}

\begin{remark}[Practical PF/BC workflow]\label{rk:9-pfbc-workflow}
To minimize ambiguity: implement PF/BC objectwise in \(t\), realize via \(\mathbf{P}_i\), truncate by \(\Ttau\), then compare in \(\Perskft\) under the fixed window/\(\boldsymbol{\delta}\) policy.
Do not compare before truncation or with mismatched window/\(\tau\).
\end{remark}



```latex
% ============================================================
% Chapter 10: PDE Program (Navier--Stokes, etc.) — fully integrated (v15.0)
% ============================================================
\section{Chapter 10: PDE Program (Navier--Stokes, etc.)}
\addcontentsline{toc}{section}{PDE Program (Navier--Stokes, etc.)}

\begin{remark}[Stability vs.\ monotonicity; corrected]
For non\hyp expansive maps, indicators are stable (non\hyp expansive).
Deletion\hyp type operations satisfying Appendix~E (e.g.\ Dirichlet restriction, principal submatrices/Schur complements, positive\hyp semidefinite Loewner contractions)
make spectral tails and windowed energies \emph{non\hyp increasing}.
Inclusion\hyp type updates generally do not guarantee non\hyp increase; we only claim stability.
\end{remark}

\subsection*{10.0. Standing hypotheses and admissible PDE realizations}
We fix a field \(k\) and work within the \emph{implementable range} of Part~I.
\emph{All statements in this chapter are made within the constructible range}
(we identify \(\Perskft\) with the constructible subcategory as in Chapter~6).
Let \(\FiltCh{k}\) be finite\hyp type filtered chain complexes, and \(\mathbf{P}_i:\FiltCh{k}\to\Perskft\) the degreewise persistence functor.
We write \(\Ttau:=\mathbf{T}_\tau\) for the Serre (bar\hyp deletion) reflector at scale \(\tau\ge 0\), and use its filtered lift \(\Ctau\) \emph{up to filtered quasi\hyp isomorphism} (Chapter~2, §§2.2–2.3).
A fixed \(t\)\hyp exact realization \(\Rfun:\FiltCh{k}\to D^{\mathrm{b}}(k\text{-mod})\) is retained; the lifting–coherence hypothesis \(\LC\) is assumed when comparing \(\Ctau\) with \(\tau_{\ge 0}\!\circ\!\Rfun\).
\emph{Equalities and Lipschitz claims are asserted only at the persistence layer; at the filtered\hyp complex layer they hold up to filtered quasi\hyp isomorphism.}
Endpoint conventions and infinite bars follow Chapter~2, Remark~\ref{rk:2-endpoints}.

PDE states are sampled along a directed index (time, resolution, or parameter). An \emph{admissible PDE realization} is a functor
\[
  \mathsf{PDE}\ \xrightarrow{\ \mathcal{P}\ }\ \FiltCh{k},\qquad U\longmapsto F=\mathcal{P}(U),
\]
satisfying:
\begin{itemize}
  \item \textbf{Finite\hyp type and (co)limits:} \(F\) is degreewise finite\hyp type; degreewise filtered (co)limits in \(\FiltCh{k}\) are computed objectwise in \([\mathbb{R},\mathsf{Vect}_k]\) and used only under the scope policy of Appendix~A (compute in the functor category and verify return to \(\Pers^{\mathrm{cons}}_k\)).
  \item \textbf{Non\hyp expansiveness under persistence:} along each directed update (e.g.\ time step, viscosity step, down\hyp /up\hyp sampling), the induced filtered map is non\hyp expansive degreewise under \(\mathbf{P}_i\);
  in \emph{deletion\hyp type} steps (Appendix~E) indicators are non\hyp increasing up to f.q.i., while inclusion\hyp type updates guarantee only stability.
  \item \textbf{Compatibility with truncation:} for each \(i\), naturally in \(\Perskft\),
  \[
    \mathbf{P}_i(\Ctau F)\ \cong\ \Ttau\,\mathbf{P}_i(F).
  \]
  \item \textbf{Realization coherence:} \(\Rfun\) is \(t\)\hyp exact and compatible with \(\LC\), so functorially up to f.q.i.,
  \(\Rfun(\Ctau F)\simeq \tau_{\ge 0}\,\Rfun(F)\).
\end{itemize}

\subsection*{10.1. Permitted operations and NS--specific examples (with CFL/CN controls)}
Each A\hyp side step \(U\) is labeled and immediately followed by collapse \(\Ctau\); all measurements and gate decisions are taken on the B\hyp side single layer \(\Ttau\mathbf{P}_i\) (Chapter~1, B\hyp Gate\(^{+}\)). The \emph{Courant number} \(\mathrm{CN}\) and \emph{CFL} condition are recorded in the run manifest (Appendix~G) and justify quantitative non\hyp expansiveness (\(\varepsilon\)\hyp interleavings) for time stepping.

\paragraph{Operation labels and NS examples.}
\begin{itemize}
  \item \emph{Deletion\hyp type (monotone):} low\hyp pass mollification (filter width \(\sigma\)), viscosity increment \(\nu\mapsto \nu+\delta\nu\), threshold lowering in levelset filtrations, Dirichlet/absorbing boundary introduction, conservative averaging, Schur complements on blocks of the discrete operators. After \(\Ctau\), windowed persistence energies and spectral tails/heat traces on \(L(\Ctau F)\) are \emph{non\hyp increasing} (Appendix~E).
  \item \emph{\(\varepsilon\)\hyp continuation (non\hyp expansive):} small time step \(\Delta t\) respecting CFL (e.g.\ \(\mathrm{CN}=\frac{u\,\Delta t}{\Delta x}\le \mathrm{CN}_{\max}\)), small parameter drifts (forcing amplitude, boundary condition perturbations), micro\hyp updates of numerical flux limiters. Collapse\hyp after stability holds with interleaving drift \(\varepsilon\!\sim\!C\Delta t\) (recorded).
  \item \emph{Inclusion\hyp type (stable only):} domain enlargement, mesh refinement without smoothing, addition of couplings/sources (as long as the induced filtered map is 1\hyp Lipschitz for \(\mathbf{P}_i\)). No monotonicity is claimed; stability only.
\end{itemize}
For each step, record in the \emph{\(\delta\)\hyp ledger} (Chapter~5; Appendix~L) the decomposition
\[
\delta(i,\tau)\ =\ \delta^{\mathrm{alg}}(i,\tau)\ +\ \delta^{\mathrm{disc}}(i,\tau)\ +\ \delta^{\mathrm{meas}}(i,\tau),
\]
where \(\delta^{\mathrm{alg}}\) is the algorithmic non\hyp commutation budget, \(\delta^{\mathrm{disc}}\) the discretization error (e.g.\ from CFL, quadrature, linear solves), and \(\delta^{\mathrm{meas}}\) the measurement error.

\begin{declaration}[Deletion\hyp type operations (PDE)]\label{spec:10-pde-del}
Operations covered by Appendix~E (Dirichlet restriction/absorbing boundaries, positive\hyp semidefinite Loewner contractions with trace monotonicity, principal submatrices and Schur complements, conservative averaging) are treated as \emph{deletion\hyp type}.
After truncation they are non\hyp expansive for each \(\mathbf{P}_i\), and windowed energies \(\mathrm{PE}_i^{\le\tau}\) as well as spectral tails/heat traces on \(L(\Ctau F)\) are \emph{non\hyp increasing}.
Inclusion\hyp type updates are asserted only to be stable (non\hyp expansive). See Appendix~E for the monotonicity lemmas used.
\end{declaration}

\begin{remark}[Quantitative non\hyp expansiveness]\label{rk:10-epsilon}
Let \(d_{\mathrm{int}}\) denote the interleaving distance on degreewise persistence. Along an update \(F_{s+1}\to F_s\), assume
\[
  d_{\mathrm{int}}\big(\mathbf{P}_i(F_{s+1}),\mathbf{P}_i(F_s)\big)\ \le\ \varepsilon_s\qquad(\varepsilon_s\ge 0).
\]
If \(\sup_s\varepsilon_s\le \varepsilon\), we call the tower \emph{\(\varepsilon\)\nobreakdash Lipschitz}. In deletion\hyp type steps typically \(\varepsilon_s=0\); inclusion\hyp type need not be zero. Time stepping under a CFL bound provides a concrete \(\varepsilon_s\sim C\Delta t\), recorded in the manifest.
\end{remark}

\begin{remark}[Truncation is \(1\)\nobreakdash Lipschitz]
Since \(\Ttau\) is \(1\)\nobreakdash Lipschitz for \(d_{\mathrm{int}}\), the same \(\varepsilon\)\nobreakdash Lipschitz control holds after truncation:
\[
d_{\mathrm{int}}\!\big(\Ttau\,\mathbf{P}_i(F_{s+1}),\ \Ttau\,\mathbf{P}_i(F_s)\big)\ \le\ \varepsilon_s.
\]
\end{remark}

\begin{remark}[Endpoints and infinite bars]\label{rk:10-endpoints}
Open/closed endpoint choices are immaterial; infinite bars are not removed by \(\Ttau\) and are clipped by windowed indicators (as in Chapter~6).
\end{remark}

\begin{remark}[Index set and cone extension]\label{rk:10-cone}
Work in the \emph{filtered index category} \(I\cup\{\infty\}\) with cone apex \(\infty\): for \(s\in I\), adjoin cone maps \(s\to\infty\).
Under \(\mathcal{P}\), these yield filtered maps \(F_s\to F_\infty\) used to define the comparison morphisms \(\phi_{i,\tau}\) at fixed~\(\tau\) (cf.\ Chapter~4).
\end{remark}

\subsection*{10.2. Construction principles for \texorpdfstring{$\mathcal{P}$}{P}}
We list domain\hyp agnostic templates; any one suffices for admissibility.
\begin{itemize}
  \item \textbf{Scalar\hyp field cubical pipeline.} From a field \(q\) (e.g.\ vorticity magnitude, enstrophy density, \(Q\)\nobreakdash criterion) on a grid, build a cubical filtration by superlevel/sublevel sets; chains are \(k\)\nobreakdash valued on cubes.
  \item \textbf{Graph/simplicial pipeline.} From point samples, build Vietoris–Rips/alpha complexes with scale \(\varepsilon\); chains are \(k\)\nobreakdash valued on simplices.
  \item \textbf{Hybrid pipeline.} Combine topology of coherent structures with connectivity of level sets; filtration is vectorized but evaluated degreewise.
\end{itemize}
All three preserve finite\hyp type per degree and admit non\hyp expansive updates for standard PDE integrators (viscous steps are smoothing; down\hyp sampling is deletion\hyp type).
\emph{Spectral proxies are computed on \(L(\Ctau F)\) (positive eigenvalues; zero modes excluded or via pseudoinverse).}

\begin{remark}[Normalization and logging]
Normalization (graph vs.\ Hodge, symmetric vs.\ random\hyp walk), zero\hyp mode handling, and the window policy are fixed throughout a run and recorded alongside \((\beta,M(\tau),t)\) (Appendix~G).
All spectral indicators are computed on \(L(\Ctau F)\) to align with the truncation window.
\emph{Spectral indicators are not f.q.i.\ invariants; we only claim stability under a fixed policy \((\beta,M(\tau),t)\) on \(L(\Ctau F)\) (cf.\ Chapter~11).}
\end{remark}

\subsection*{10.3. Indicators and diagnostics}
For each sample \(s\in I\) and degree \(i\) we monitor:
\[
  \Ttau\,\mathbf{P}_i(F_s),\qquad\n  \mathrm{PE}_i^{\le \tau}(F_s)\ \text{(truncated energies, computed on }\Ttau\,\mathbf{P}_i(F_s)\text{)},\qquad\n  \mathrm{ST}_\beta^{\ge M(\tau)}(F_s),\ \mathrm{HT}(t;F_s)\ \text{on }L(\Ctau F_s),
\]
together with the categorical check \(\Ext^1\big(\Rfun(\Ctau F_s),Q\big)=0\) for \(Q\in\Qtest\).
Equivalently, \(\mathrm{PE}_i^{\le\tau}(F_s)=\mathrm{PE}_i^{\le\tau}(\Ctau F_s)\).
For fixed \(\tau\), define the comparison map
\[
  \phi_{i,\tau}:\ \varinjlim_{s\in I}\ \Ttau\,\mathbf{P}_i(F_s)\ \longrightarrow\ \Ttau\,\mathbf{P}_i(F_\infty),
\]
and obstruction counts
\[
\mu_{i,\tau}=\dim_k\ker\phi_{i,\tau}\footnote{Here \(\dim_k\) denotes the \emph{generic\hyp fiber} dimension after truncation, i.e.\ the multiplicity of \(I[0,\infty)\) summands; see Appendix~D, Remark~\ref{rem:D-generic-dim}.},\qquad\n\nu_{i,\tau}=\dim_k\mathrm{coker}\,\phi_{i,\tau},
\]
with \(\muc=\sum_i\mu_{i,\tau}\), \(\nuc=\sum_i\nu_{i,\tau}\) (finite by bounded degrees).
\emph{The obstructions \((\muc,\nuc)\) are invariant under filtered quasi\hyp isomorphisms and under cofinal reindexing of the tower} (Appendix~J).

\begin{declaration}[Specification: Tower stability at the persistence layer]\label{spec:10-stab}
Under the finite\hyp type and objectwise degreewise\hyp colimit hypotheses, for each fixed \(\tau\) and all \(i\) the map
\[
  \phi_{i,\tau}:\ \varinjlim_{s}\ \Ttau\,\mathbf{P}_i(F_s)\ \xrightarrow{\ \cong\ }\ \Ttau\,\mathbf{P}_i(F_\infty)
\]
is an isomorphism; hence \((\muc,\nuc)=(0,0)\) at that scale and Type~IV is excluded at \(\tau\).
\end{declaration}

\subsection*{10.4. Trigger pack ([Spec], domain--restricted necessary conditions)}
We record PDE\hyp specific \emph{triggers} that indicate B\hyp Gate\(^{+}\) failure on a window \(W=[u,u')\) (Chapter~1), to be used as \textbf{[Spec]}. Typical instances:
\begin{itemize}
  \item \textbf{High\hyp frequency surge:} sustained growth of enstrophy or high\hyp wavenumber density in \(W\) \(\Rightarrow\) aux\hyp bars (Chapter~11) persist \(>0\) after \(\Ctau\) or \(\mu>0\) is detected at \(\tau\).
  \item \textbf{Under\hyp resolved advection:} CFL violation or \(\mathrm{CN}>\mathrm{CN}_{\max}\) \(\Rightarrow\) \(\varepsilon\)\hyp continuation drift \(\varepsilon\) exceeds the safety margin \(\mathrm{gap}_\tau\) and B\hyp Gate\(^{+}\) fails.
  \item \textbf{Unbalanced dissipation:} lack of smoothing under nominally viscous steps \(\Rightarrow\) non\hyp decrease of \(\mathrm{PE}_i^{\le\tau}\) or spectral tails; repeated violations within \(W\) mark the window as non\hyp regularizing.
\end{itemize}
All triggers are logged (Appendix~G) with quantitative thresholds and do \emph{not} replace B\hyp Gate\(^{+}\); they augment diagnostics.

\subsection*{10.5. Window pasting: Restart and Summability}
Let \(\{W_k=[u_k,u_{k+1})\}_k\) be a MECE partition (Chapter~2). On each \(W_k\), fix \(\tau_k\) (resolution\hyp adapted; Chapter~2) and compute the pipeline budget \(\Sigma\delta_k(i)=\sum_{U\in W_k}\delta_U(i,\tau_k)\). If B\hyp Gate\(^{+}\) passes with a safety margin \(\mathrm{gap}_{\tau_k}>\Sigma\delta_k(i)\), the Restart lemma (Chapter~4) yields
\[
\mathrm{gap}_{\tau_{k+1}}\ \ge\ \kappa\ \bigl(\mathrm{gap}_{\tau_k}-\Sigma\delta_k(i)\bigr)\quad(\kappa\in(0,1]).
\]
If moreover \(\sum_k\Sigma\delta_k(i)<\infty\) (Summability; e.g.\ geometric decay of \(\tau_k,\beta_k\)), windowed certificates paste to a global one (Chapter~4).

\subsection*{10.6. Persistence--guided regularization ([Spec])}
\begin{declaration}[Specification: Persistence\hyp guided regularization]\label{spec:10-PGR}
A numerical regime is \emph{persistence\hyp regularizing at scale \(\tau\)} if, along \(s\in I\),
\begin{enumerate}
  \item \(\mathrm{PE}_i^{\le \tau}\) are non\hyp increasing (strictly decreasing on steps with genuine deletion\hyp type smoothing),
  \item spectral indicators \(\mathrm{ST}_\beta^{\ge M(\tau)}\), \(\mathrm{HT}(t;\cdot)\) on \(\Ctau F_s\) are non\hyp increasing (stability in general, monotone decrease in smoothing steps),
  \item \((\muc,\nuc)=(0,0)\),
  \item \(\Ext^1\big(\Rfun(\Ctau F_s),Q\big)=0\) for \(Q\in\Qtest\).
\end{enumerate}
When these hold across a \(\tau\)\nobreakdash interval, the regime aligns with established regularization frameworks at that scale (domain\hyp specific hypotheses to be listed separately). No analytic identity is claimed.
\end{declaration}

\subsection*{10.7. AK--NS hypothesis (programmatic)}
\begin{conjecture}[AK--NS hypothesis]\label{conj:10-AKNS}
For Navier–Stokes\hyp type flows, under an admissible realization \(\mathcal{P}\) and \(\LC\), if a monitored segment satisfies Declaration~\ref{spec:10-PGR} across a \(\tau\)\nobreakdash interval, then the designed persistence structure \emph{collapses} at that scale (bars shorten/vanish in aggregate), spectral tails decay, and the categorical check persists.
Programmatically, this corresponds to convergence toward known regularity scenarios at that scale.
No equivalence \(\mathrm{PH}_1\Leftrightarrow\Ext^1\) is asserted; only the one\hyp way bridge under (B1)–(B3) is used.
\end{conjecture}

\subsection*{10.8. Gate template (PDE)}
On a fixed window \(W=[u,u')\), collapse threshold \(\tau>0\), and degree \(i\):
\begin{enumerate}
  \item Apply step \(U\) (labeled as above), then collapse \(\Ctau\).
  \item Measure on the B\hyp side single layer: \(\Ttau\mathbf{P}_i\), \(\mathrm{PE}_i^{\le\tau}\), spectral auxiliaries (aux\hyp bars; Chapter~11), and (in scope) \(\Ext^1\).
  \item Record \(\delta^{\mathrm{alg}},\delta^{\mathrm{disc}},\delta^{\mathrm{meas}}\) and update \(\Sigma\delta\).
  \item Evaluate B\hyp Gate\(^{+}\): require PH1\(=0\), (in scope) Ext1\(=0\), \((\mu,\nu)=(0,0)\) after \(\Ttau\), and \(\mathrm{gap}_\tau>\Sigma\delta\).
  \item Log verdict; issue a windowed certificate on success; otherwise classify failure (Type I–IV).
\end{enumerate}

\subsection*{10.9. Monitoring protocol (PDE)}
\begin{declaration}[Specification: Joint monitoring protocol]\label{spec:10-protocol}
Fix scales \(\tau\in[\tau_{\min},\tau_{\max}]\) and an index set \(I\) (time/resolution/parameter).
For each sample \(s\in I\):
\begin{enumerate}
  \item \emph{Compute and record} \(\Ttau\,\mathbf{P}_i(F_s)\) and \(\mathrm{PE}_i^{\le \tau}\) on \(\Ttau\,\mathbf{P}_i(F_s)\) (equivalently on \(\Ctau F_s\)).
  \item \emph{Compute and record} spectral indicators \(\mathrm{ST}_\beta^{\ge M(\tau)}\), \(\mathrm{HT}(t;F_s)\) on \(L(\Ctau F_s)\) with a fixed \((\beta,M(\tau),t)\) policy.
  \item \emph{Check} \(\Ext^1\big(\Rfun(\Ctau F_s),Q\big)=0\) for \(Q\in\Qtest\).
  \item \emph{Evaluate} \((\muc,\nuc)\) via \(\phi_{i,\tau}\) along \(s\in I\) and log failure types \emph{(pure kernel/cokernel/mixed)} if present.
  \item \emph{Stability declaration:} declare the \emph{persistence\hyp regularizing regime} where (1)–(4) hold across the monitored \(\tau\)\nobreakdash range.
\end{enumerate}
\end{declaration}

\subsection*{10.10. Diagram (PDE pipeline and diagnostics)}
\begin{center}
\begin{tikzcd}[
  ampersand replacement=\&,
  column sep=2.8em, row sep=2.2em,
  cells={nodes={font=\footnotesize}},
  every label/.append style={font=\scriptsize},
  scale=0.9, transform shape
]
% --- Row 1 ---
{\text{PDE state }U} \arrow[r, "\mathcal{P}"] \&
{F\in \FiltCh{k}} \arrow[r, "\mathbf{P}_i"] \&
{\mathbf{P}_i(F)} \arrow[r, "\Ttau"] \&
{\Ttau\,\mathbf{P}_i(F)} \\\n% --- Row 2 ---
\& \& {L(\Ctau F_s)} \arrow[r, dashed, "{\text{heat trace on }L(\Ctau F_s)}"] \&
{\mathrm{HT},\ \mathrm{ST}_{\beta}} \\\n% --- Row 3 ---
\& {\Rfun(F_s)} \& \& {\Ext^1(\Rfun(\Ctau F_s),k)=0} \\\n% --- Row 4 ---
\& {\varinjlim_{s}\,\Ttau\,\mathbf{P}_i(F_s)} \arrow[rr, "{\phi_{i,\tau}}"] \& \&
{\Ttau\,\mathbf{P}_i(F_\infty)} \\\n% --- Row 5 ---
\& {\text{failure log (pure kernel/cokernel/mixed)}} \& \&
%
% ===== extra arrows with absolute coordinates =====
\arrow[from=1-3, to=2-3, dashed, "{\mathrm{PE}_i^{\le \tau}}"']
\arrow[from=1-4, to=2-4, dashed, "{\mathrm{PE}_i^{\le \tau}}"]
\arrow[from=3-2, to=1-2, bend left=40, dashed, "\LC"]
\arrow[from=3-2, to=3-4, dashed, "{\Ctau\ \text{ then }\Ext^1(-,k)\ \text{test}}"]
\arrow[from=4-4, to=5-2, bend right=25, dashed, swap,
       "{(\muc,\nuc)\ \text{log (pure kernel/cokernel/mixed)}}"]
\end{tikzcd}
\end{center}

\subsection*{10.11. Toy instances (persistence layer)}
\begin{example}[Viscous smoothing]\label{ex:10-visc}
Let \(s\mapsto U_s\) be viscous steps for which the induced maps are deletion\hyp type on the filtration.
Then bar lengths within the \(\tau\)\nobreakdash window decrease (or vanish), \(\mathrm{PE}_i^{\le \tau}\) strictly decreases, and \((\muc,\nuc)=(0,0)\) at fixed \(\tau\) by Declaration~\ref{spec:10-stab}.
Spectral proxies are evaluated as tails/heat traces of \(L(\Ctau F_s)\).
\end{example}

\begin{example}[Refinement/averaging pair]
A refinement \(F\to F'\) (inclusion\hyp type) followed by conservative averaging \(F'\to \bar{F}\) (deletion\hyp type) yields a non\hyp expansive two\hyp step update.
Under stability, \(\mathrm{PE}_i^{\le \tau}\) is non\hyp increasing; failure logs isolate kernel/cokernel imbalance when present.
Spectral indicators are computed on \(L(\Ctau \bar{F})\).
\end{example}

\subsection*{10.12. Reproducibility (PDE)}
\begin{remark}[Run logs and parameters]\label{rk:10-logs}
For each run, log: index range \(s\in[s_{\min},s_{\max}]\) (e.g.\ time), scales \(\tau\in[\tau_{\min},\tau_{\max}]\) (step width), spectral parameters \((\beta,M(\tau),t)\), discretization choices (cubical/simplicial/hybrid), CFL/CN numbers, \emph{a barcode\hyp matching seed for reproducible vineyard tracking} (Appendix~G), \((\muc,\nuc)\) per \(\tau\) with failure types, and the \(\delta\)\hyp ledger decomposition at step level.
These logs enable exact reruns and pipeline audits.
\end{remark}

\noindent A minimal \texttt{run.yaml} PDE block:

\begin{verbatim}
windows:
  domain: [[0,1), [1,2), [2,3)]
  collapse_tau: 0.08
  spectral_bins: {a: 0.0, beta: 0.02, bins: 96, boundary: "right-open"}
coverage_check:
  length_sum: 3.0
  length_target: 3.0
  events_sum_equals_global: true
cfl:
  courant_number_max: 0.5
  courant_number_measured: 0.32
operations:
  - U: mollify; type: deletion; tau: 0.08; delta: {alg:0.004, disc:0.003, meas:0.001}
  - U: timestep; type: epsilon;  tau: 0.08; eps: 0.006; delta: {alg:0.000, disc:0.002, meas:0.001}
persistence:
  PH1_zero: true
  Ext1_zero: true
  mu: 0
  nu: 0
  phi_iso_tail: true
spectral:
  aux_bars_remaining: 0
budget:
  sum_delta: 0.011
  safety_margin: 0.025
gate:
  accept: true
\end{verbatim}

\subsection*{10.13. Guard\hyp rails and non\hyp claims}
\begin{remark}[Scope and non\hyp claims]\label{rk:10-guard}
All statements operate at the persistence/spectral/categorical layers in the implementable range.
No analytic regularity theorem is proved; the AK–NS hypothesis is programmatic.
No claim of \(\mathrm{PH}_1\Leftrightarrow\Ext^1\) is made; only the one\hyp way bridge under (B1)–(B3) is used.
The obstruction \(\muc\) is unrelated to classical Iwasawa \(\mu\).
\end{remark}

\subsection*{10.14. Completion note}
\begin{remark}[No further supplementation required]
This chapter fully integrates: (i) MECE windowing and resolution\hyp adapted \(\tau\) with stability bands (via Chapter~2 and Chapter~4), (ii) the permitted operations catalog with NS\hyp specific examples under CFL/CN controls and \(\delta\)\hyp ledger accounting (Chapter~5; Appendix~L), (iii) B\hyp side single\hyp layer gate B\hyp Gate\(^{+}\) with PH1/Ext1/(\(\mu,\nu\))/safety\hyp margin, (iv) triggers as \textbf{[Spec]} and a complete monitoring protocol, (v) Restart/Summability for window pasting, and (vi) reproducibility (run.yaml) with pipeline audit fields. All claims remain within the v15.0 guard\hyp rails and cross\hyp reference the proven core; no additional supplements are needed for operational use as a proof framework.
\end{remark}
```


% ============================================================
% Part III — Measurement / Implementation / Tests
% ============================================================
\section{Chapter 11: Collapse Energy, Spectral Indicators, and TDA Notes}
\addcontentsline{toc}{section}{Collapse Energy, Spectral Indicators, and TDA Notes}

\subsection*{11.0. Scope, standing hypotheses, and notation}
We work within the \emph{implementable range} of Part~I, using realizations into $\FiltCh{k}$ as in Chs.~6–10. All persistence quantities are computed degreewise \emph{after} truncation by $\Ctau$; spectral indicators use the normalized combinatorial Hodge Laplacian on $\Ctau F$; categorical checks use a fixed $t$-exact $\Rfun:\FiltCh{k}\to\Db(k\text{-mod})$, compatible with \LC. Filtered (co)limits, when used, are computed objectwise in $[\mathbb{R},\mathsf{Vect}_k]$ and used only under the scope policy of Appendix~A (compute in the functor category and verify return to $\Pers^{\mathrm{cons}}_k$). No claim of $\mathrm{PH}_1\Leftrightarrow \Ext^1$ is made; only the one-way bridge under (B1)–(B3) is used. The obstruction pair $(\muc,\nuc)$ is the \emph{collapse} diagnostic and is unrelated to the classical Iwasawa $\mu$.

\begin{remark}[Endpoints and infinite bars]\label{rk:11-endpoints}
Open/closed endpoint conventions are immaterial; infinite bars are not removed by $\Ttau$ and are clipped by the window $\tau$ in all windowed quantities (cf.\ Ch.~6).
\end{remark}

\subsection*{11.1. Collapse energy (windowed persistence energies)}
Fix a degree $i$, a scale $\tau\ge 0$, and an exponent $\alpha>0$ (default $\alpha=1$). Let $\mathcal{B}_{i}(F)$ be the barcode of $\mathbf{P}_i(F)$, and for a bar $b=[b_{\ell},b_{r})$ write its $\tau$-windowed length
\[
  \ell_{\tau}(b)\ :=\ \bigl(\min\{b_{r},\tau\}-\min\{b_{\ell},\tau\}\bigr)_{+},\qquad (x)_{+}:=\max\{x,0\}.
\]
Fix a weight function $w_i:\mathcal{B}_i(F)\to[0,\infty)$ (default $w_i\equiv 1$).
\begin{definition}[Windowed energies]\label{def:11-PE}
The (weighted) windowed energy at degree $i$ is
\[
  \mathrm{PE}_{i}^{\le \tau}(F;w_i,\alpha)\ :=\ \sum_{b\in \mathcal{B}_{i}(F)} w_i(b)\,\big(\ell_{\tau}(b)\big)^{\alpha}.
\]
With the unweighted shorthand $\mathrm{PE}_{i}^{\le \tau}(F):=\mathrm{PE}_{i}^{\le \tau}(F;1,\alpha)$ (default $\alpha=1$), the \emph{collapse energy} vector at $\tau$ is
\[
  \mathbf{CE}^{\le \tau}(F)\ :=\ \big(\mathrm{PE}_{i}^{\le \tau}(F)\big)_{i\in\mathbb{Z}},\qquad \|\mathbf{CE}^{\le \tau}(F)\|_{1}\ :=\ \sum_{i}\mathrm{PE}_{i}^{\le \tau}(F).
\]
All quantities are computed on the truncated barcode $\Ttau\mathbf{P}_i(F)$ (equivalently, on $\Ctau F$).
\end{definition}

\begin{remark}[Stability and monotonicity]
By Part~I (Lemma~\ref{lem:shift}, Proposition~\ref{prop:stability}), $1$-Lipschitz updates under $\mathbf{P}_i$ induce non\hyp expansive changes of $\mathrm{PE}_{i}^{\le \tau}$; in \emph{deletion\hyp type} updates, $\mathrm{PE}_{i}^{\le \tau}$ is non\hyp increasing up to \fqi.
Since $\Ttau$ is $1$-Lipschitz for the interleaving distance, the same Lipschitz control carries over after truncation:
\[
  d_{\mathrm{int}}\!\big(\Ttau\mathbf{P}_i(F'),\,\Ttau\mathbf{P}_i(F)\big)\ \le\ d_{\mathrm{int}}\!\big(\mathbf{P}_i(F'),\,\mathbf{P}_i(F)\big).
\]
Under finite-type and objectwise degreewise-colimit hypotheses, $(\muc,\nuc)=(0,0)$ at fixed $\tau$ excludes Type~IV at that scale (Ch.~4).
\end{remark}

\subsection*{11.2. Spectral indicators on \texorpdfstring{$\Ctau F$}{C\_tau F}}
Let $L(\Ctau F)$ be the normalized combinatorial Hodge Laplacian on $\Ctau F$ (per degree, with the standard Euclidean inner product on chains), and let $\{\lambda_{j}\}_{j\ge 1}$ denote its positive eigenvalues (zero modes excluded; alternatively treat by Moore–Penrose pseudoinverse).
\begin{definition}[Spectral tails and heat traces]\label{def:11-spectral}
Fix $\beta>0$ and a cutoff policy $M(\tau)\in\mathbb{N}$. Define
\[
  \mathrm{ST}_{\beta}^{\ge M(\tau)}(F)\ :=\ \sum_{j\ge M(\tau)} \lambda_{j}^{-\beta},\qquad \mathrm{HT}(t;F)\ :=\ \sum_{j\ge 1} e^{-t\,\lambda_{j}},\quad t>0.
\]
Typical policies: \emph{(i)} $M(\tau)=\lfloor c\,\tau^{\gamma}\rfloor$ with $c>0$, $\gamma\in[0,2]$; \emph{(ii)} $t\in[c_1\tau^{-2},c_2\tau^{-2}]$ with fixed $0<c_1\le c_2$.
\end{definition}

\begin{remark}[Stability]
Non\hyp expansive filtered maps on $\Ctau F$ induce $1$-Lipschitz updates at the persistence layer and stable (non\hyp expansive) spectral responses for $\mathrm{ST}_{\beta}^{\ge M(\tau)}$, $\mathrm{HT}(t;-)$, provided the policy $(\beta,M(\tau),t)$ is held fixed (cf.\ Chs.~6–10).
\end{remark}

\subsection*{11.3. Auxiliary spectral bars (aux-bars)}
We supplement spectral scalars by a \emph{binwise occupancy} summary along a discrete index (e.g.\ time, step number), evaluated on \emph{collapsed} Laplacians $L(\Ctau F)$.

\begin{definition}[Spectral bins, occupancies, and aux-bars]\label{def:11-aux-bars}
Fix a spectral window $[a,b]$ and a bin width $\beta>0$. Define right-open bins
\[
I_r\ :=\ [\,a+r\beta,\ a+(r+1)\beta\,),\qquad r=0,1,\dots,R-1,\ \ R:=\Big\lfloor\frac{b-a}{\beta}\Big\rfloor.
\]
For a sample index $j$ (e.g.\ time step), let $\{\lambda_m(j)\}_{m\ge 1}$ be the positive eigenvalues of $L(\Ctau F_j)$ and define \emph{occupancies}
\[
E_r(j)\ :=\ \#\{\,m\ \mid\ \lambda_m(j)\in I_r\,\},\qquad \text{with under/overflow counts }E_{<a}(j),\ E_{\ge b}(j)\ \text{logged explicitly}.
\]
For each fixed bin $r$, define an \emph{auxiliary spectral bar} (aux-bar) as a maximal consecutive run $J\subset\mathbb{Z}$ such that $E_r(j)>0$ for all $j\in J$. The \emph{lifetime} of the aux-bar is $|J|$ (or a rescaling thereof).
\end{definition}

\begin{remark}[Collapse-first, window policy, and stability]
Aux-bars are computed on $L(\Ctau F_j)$ to align with the persistence window; bin boundary is right-open; under/overflow are recorded to ensure coverage. Along \emph{deletion-type} updates (Appendix~E), the number of active bins and the total aux-bar mass are \emph{non\hyp increasing} after collapse; under \(\varepsilon\)\hyp continuations they are \emph{stable} (non\hyp expansive). Inclusion-type steps are asserted only \emph{stable} (no monotonicity).
\end{remark}

\begin{remark}[Lifetime scale and $\tau$]
When lifetimes are rescaled to a physical unit (e.g.\ time), choose the scale so that the aux-bar lifetime is commensurate with the collapse threshold $\tau$ (e.g.\ count only lifetimes $\ge c\tau$ for a chosen constant $c$). This prevents spurious short\hyp lived spectral flickers from driving gate decisions.
\end{remark}

\begin{declaration}[Gate usage of aux-bars]\label{dec:11-aux-gate}
B\hyp Gate\(^{+}\) decisions (Chapter~1) are made on the persistence layer; aux-bars are \emph{auxiliary}. A strengthened acceptance policy on a window $W$ may (optionally) require, in addition to B\hyp Gate\(^{+}\) passing, that \emph{all} aux-bars vanish on the monitored bins $I_r$ after collapse, i.e.\ $E_r(j)=0$ for all $r$ and $j\in W$. This strengthens practical robustness but does not supersede the persistence\hyp level verdict.
\end{declaration}

\subsection*{11.4. Categorical check (one-way bridge)}
With $\Qtest=\{k[0]\}$ fixed, we monitor the categorical obstruction
\[
  \Ext^1\big(\Rfun(\Ctau F),Q\big)=0\qquad(Q\in\Qtest).
\]
This check is performed \emph{after truncation} and used only in the one-way direction sanctioned in Part~I (under (B1)–(B3)); no converse or equivalence is claimed.

\subsection*{11.5. Collapse diagnostics along towers}
For an index category $I$ (time/resolution/parameter), adjoin a terminal element $\infty$ and cone maps to form comparison morphisms
\[
  \phi_{i,\tau}:\ \varinjlim_{t\in I}\ \Ttau\mathbf{P}_i(F_t)\ \longrightarrow\ \Ttau\mathbf{P}_i(F_\infty),
\]
with
\[
  \mu_{i,\tau}=\dim_k\ker\phi_{i,\tau}\footnote{Here $\dim_k$ denotes the \emph{generic fiber} dimension after truncation, i.e.\ the multiplicity of $I[0,\infty)$ summands; see Appendix~D, Remark~\ref{rem:D-generic-dim}.},\qquad\n  \nu_{i,\tau}=\dim_k\mathrm{coker}\,\phi_{i,\tau}\footnote{Here $\dim_k$ denotes the \emph{generic fiber} dimension after truncation, i.e.\ the multiplicity of $I[0,\infty)$ summands; see Appendix~D, Remark~\ref{rem:D-generic-dim}.},
\]
and $\muc=\sum_{i}\mu_{i,\tau}$, $\nuc=\sum_{i}\nu_{i,\tau}$ (finite by finite-typeness). Under the hypotheses of Part~I/Ch.~4, $\phi_{i,\tau}$ are isomorphisms and $(\muc,\nuc)=(0,0)$ at fixed $\tau$.

\subsection*{11.6. Specification: joint monitoring protocol}
\begin{declaration}[Specification: Joint monitoring protocol]\label{spec:11-monitor}
Fix a finite sweep $\tau\in[\tau_{\min},\tau_{\max}]$ and a policy $(\alpha, w_i;\ \beta, M(\tau), t)$.
For each sample $t\in I$ and degree $i$:
\begin{enumerate}
  \item \emph{Compute and record} $\Ttau\mathbf{P}_i(F_t)$; evaluate $\mathrm{PE}_{i}^{\le \tau}(F_t)$ on $\Ttau\mathbf{P}_i(F_t)$ (equivalently on $\Ctau F_t$).
  \item \emph{Compute and record} $\mathrm{ST}_{\beta}^{\ge M(\tau)}(F_t)$ and $\mathrm{HT}(t;\Ctau F_t)$ using $L(\Ctau F_t)$; compute aux-bars via Definition~\ref{def:11-aux-bars} on the same window $[a,b]$ and bin policy $\beta$.
  \item \emph{Check} $\Ext^1\big(\Rfun(\Ctau F_t),Q\big)=0$ for $Q\in\Qtest$.
  \item \emph{Evaluate} $(\muc,\nuc)$ at each $\tau$ via the comparison maps $\phi_{i,\tau}$; log failure type (pure kernel/cokernel/mixed) if $(\muc,\nuc)\neq(0,0)$.
  \item \emph{Declare stable regime} at $\tau$ when (1)–(3) hold jointly and $(\muc,\nuc)=(0,0)$. Optionally require aux-bars$=0$ (Declaration~\ref{dec:11-aux-gate}).
\end{enumerate}
All persistence-layer steps are asserted at the persistence layer and are invariant under \fqi\ by construction; spectral and aux-bar steps are not \fqi\ invariants but are \emph{stable} under the fixed policy \((\beta, M(\tau), t)\) on \(L(\Ctau F_t)\).
\end{declaration}

\subsection*{11.7. Noise tolerance and discretization rules}
\begin{declaration}[Specification: noise and discretization policy]\label{spec:11-noise}
Let $\varepsilon>0$ be the noise scale.
\begin{itemize}
  \item \textbf{Barcode denoising:} apply \emph{$\varepsilon$-clipping} on $\Ttau\mathbf{P}_i(F)$ by removing bars of length $\le \varepsilon$ within the $\tau$-window. This is stable under bottleneck perturbations $\le \varepsilon$ and preserves \fqi\ invariants.
  \item \textbf{Energy stability:} there exists $C_{i,\tau,\alpha}$ (depending on the number of bars in the window, their maximal multiplicity, and ambient dimension bounds) such that
  \[
    \big|\mathrm{PE}_{i}^{\le \tau}(F)-\mathrm{PE}_{i}^{\le \tau}(\tilde F)\big|\ \le\ C_{i,\tau,\alpha}\,\varepsilon^{\min\{1,\alpha\}}
  \]
  whenever $d_{\mathrm{int}}(\Ttau\mathbf{P}_i(F),\Ttau\mathbf{P}_i(\tilde F))\le \varepsilon$.
  \item \textbf{Spectral stabilization:} compute spectra on $L(\Ctau F)$; use the same $(\beta,M(\tau),t)$ across runs. Optional averaging over $N$ realizations reduces variance as $N^{-1/2}$. Aux-bar lifetimes are counted with the same binning policy; short lifetimes $\le 2$ frames may be ignored as noise (manifest must record the rule).
  \item \textbf{Resolution rule:} choose sampling so that the minimal feature length resolved is $\ge 3$ grid steps; sweep $\tau$ on a lattice $\Delta\tau$ satisfying $\Delta\tau\le \tfrac{1}{2}$ of the minimal resolvable bar length. Here the “bar-length quantum’’ refers to the minimal resolvable feature length induced by the sampling/grid spacing and the filtration step.
\end{itemize}
\end{declaration}

\subsection*{11.8. Categorical check (bridge) and saturation gate}
With $\Qtest=\{k[0]\}$ fixed, we monitor $\Ext^1(\Rfun(\Ctau F),Q)=0$ after truncation. The one-way bridge is used only under (B1)–(B3). For windows satisfying the \emph{saturation} conditions:

\begin{declaration}[Saturation gate \textbf{[Spec]}]\label{gate:11-saturation}
Fix $\tau^\ast>0$ and parameters $\eta,\delta>0$. On the window $[0,\tau^\ast]$, assume:
(i) eventually the maximal \emph{finite} bar length in $\mathbf{T}_{\tau^\ast}\mathbf{P}_i(F_t)$ is $\le \eta$;
(ii) eventually $d_{\mathrm{int}}\!\big(\mathbf{T}_{\tau^\ast}\mathbf{P}_i(F_t),\mathbf{T}_{\tau^\ast}\mathbf{P}_i(F_{t'})\big)\le \eta$;
(iii) the \emph{edge gap} to the window, $\delta:=\tau^\ast-\max\{b_r<\tau^\ast\}$, satisfies $\delta>\eta$.
Then, \textbf{within this window only}, we adopt the temporary binary policy
\[
\mathrm{PH}_1(\C_{\tau^\ast}F)=0\quad\Longleftrightarrow\quad \Ext^1(\Rfun(\C_{\tau^\ast}F),k)=0.
\]
\end{declaration}

\subsection*{11.9. Data formats, reproducibility, and minimal manifest}
\begin{remark}[Implementation notes]\label{rk:11-impl}
\emph{Artifacts.} (i) \texttt{bars.json/h5}: list of records $\langle i,\ b_{\ell},b_{r},\ w\rangle$ per degree; (ii) \texttt{spec.json/h5}: positive eigenvalues of $L(\Ctau F)$ per degree; (iii) \texttt{aux.json/h5}: aux-bar occupancies $E_r(j)$ with bin metadata; (iv) \texttt{ext.json}: boolean for $\Ext^1(\Rfun(\Ctau F),Q)$ with minimal witness; (v) \texttt{phi.json}: ranks of comparison maps and $(\mu_{i,\tau},\nu_{i,\tau})$. \emph{Run log.} Store (a) sweep $\tau_{\min}$:$\Delta\tau$:$\tau_{\max}$; (b) policy $(\alpha, w_i;\ \beta,M(\tau),t)$; (c) discretization (grid/complex, step sizes); (d) random seeds; (e) software versions; (f) $\delta$-ledger per step; (g) bin window $[a,b]$, bin width $\beta$, under/overflow counts. \emph{Invariance.} Persistence-layer quantities are taken on $\Ttau\mathbf{P}_i(-)$ or $\Ctau(-)$ and are invariant under \fqi. \emph{Reproducibility.} Provide a single manifest (\texttt{run.yaml}) that references all artifacts and declares the tower index set, cone extension, and the decision rule for stable regimes.
\end{remark}

\noindent A minimal \texttt{run.yaml} block:

\begin{verbatim}
windows:
  domain: [[0,1), [1,2), [2,3)]
  collapse_tau: 0.08
  spectral_bins: {a: 0.0, beta: 0.02, bins: 96, boundary: "right-open"}
coverage_check:
  length_sum: 3.0
  length_target: 3.0
  events_sum_equals_global: true
operations:
  - U: mollify; type: deletion; tau: 0.08; delta: {alg:0.004, disc:0.003, meas:0.001}
  - U: timestep; type: epsilon;  tau: 0.08; eps:0.006; delta: {alg:0.000, disc:0.002, meas:0.001}
persistence:
  PH1_zero: true
  Ext1_zero: true
  mu: 0
  nu: 0
  phi_iso_tail: true
spectral:
  ST_beta: 2
  ST_M_of_tau: "floor(0.5 * tau^1.5)"
  HT_t: [0.5*tau^-2, 1.0*tau^-2]
  aux_bars_remaining: 0
budget:
  sum_delta: 0.011
  safety_margin: 0.025
gate:
  accept: true
\end{verbatim}

\subsection*{11.10. Compliance checks and unit tests}
\begin{declaration}[Specification: minimal test suite]\label{spec:11-tests}
Every deployment must pass:
\begin{itemize}
  \item \textbf{Stability test.} For synthetic $\varepsilon$-perturbations, verify non\hyp expansiveness of $\mathrm{PE}_{i}^{\le \tau}$ and stability of spectral indicators and aux-bars under the fixed policy.
  \item \textbf{Monotone update test.} For a deletion-type update, confirm $\mathrm{PE}_{i}^{\le \tau}$ non-increase, spectral tail non-increase, aux-bars non-increase (active bins do not grow), and record $(\muc,\nuc)=(0,0)$ at fixed $\tau$.
  \item \textbf{Cone-extension test.} Verify $\phi_{i,\tau}$ is an isomorphism on a model tower (hence $(\muc,\nuc)=(0,0)$) and that Type~IV is excluded at that $\tau$.
  \item \textbf{Categorical check.} On a curated sample, confirm $\Ext^1(\Rfun(\Ctau F),Q)=0$ is stable under admissible \fqi\ updates.
\end{itemize}
\end{declaration}

% ============================================================
% 11.A  Formalization API (Lean/Coq stubs)
% ============================================================
\subsection*{11.A. Formalization API (Lean/Coq stubs)}
All items are stated in the constructible range and, at the filtered–complex layer, \emph{up to filtered quasi-isomorphism}. Identifiers are indicative.

\begin{specification}[Persistence truncation $\mathbf{T}_\tau$ — exactness and $1$-Lipschitz]
\begin{itemize}
  \item \texttt{pers\_Ttau\_exact}: $\mathbf{T}_\tau$ is exact on $\Pers^{\mathrm{cons}}_k$ (Serre localization at $\tau$).
  \item \texttt{pers\_Ttau\_lipschitz}: $d_{\mathrm{int}}(\mathbf{T}_\tau M,\mathbf{T}_\tau N)\le d_{\mathrm{int}}(M,N)$.
  \item \texttt{pers\_Ttau\_commute\_colim\_flim}: $\mathbf{T}_\tau$ commutes with filtered colimits and finite limits.
\end{itemize}
\end{specification}

\begin{specification}[Comparison maps $\phi$ — functoriality and cofinal invariance]
For a filtered diagram $\{F_t\}_{t\in I}$ and degree $i$,
\[
\phi_{i,\tau}:\ \varinjlim_{t}\ \mathbf{T}_\tau\mathbf{P}_i(F_t)\ \longrightarrow\ \mathbf{T}_\tau\mathbf{P}_i(F_\infty).
\]
\begin{itemize}
  \item \texttt{phi\_natural}: natural in (a) morphisms of diagrams; (b) $\tau$ via $\mathbf{T}_\tau\Rightarrow \mathbf{T}_\sigma$ for $\tau\le \sigma$.
  \item \texttt{phi\_cofinal}: invariant under cofinal reindexing $I\to I'$.
  \item \texttt{phi\_sum\_prod}: additive under finite direct sums; compatible with finite limits (pullbacks) at the persistence layer.
\end{itemize}
\end{specification}

\begin{specification}[Kernel/cokernel indices $(\mu,\nu)$ — computation and calculus]
\begin{itemize}
  \item \texttt{mu\_nu\_def}: $\mu_{i,\tau}=\dim_k\ker\phi_{i,\tau}$, $\ \nu_{i,\tau}=\dim_k\mathrm{coker}\,\phi_{i,\tau}$; $\ \muc=\sum_i\mu_{i,\tau}$, $\ \nuc=\sum_i\nu_{i,\tau}$ (finite by bounded degree).
  \item \texttt{mu\_nu\_vanish}: under degreewise objectwise filtered colimits (Part~I/Ch.~4), each $\phi_{i,\tau}$ is an isomorphism, hence $(\muc,\nuc)=(0,0)$.
  \item \texttt{mu\_nu\_calc}: subadditivity under composition, additivity under finite sums, and cofinal invariance (Appendix~J).
\end{itemize}
Here $\dim_k$ denotes the \emph{generic fiber} dimension after truncation (Appendix~D, Remark~\ref{rem:D-generic-dim}).
\end{specification}

\begin{specification}[Edge identification and bridge (B2)]
\begin{itemize}
  \item \texttt{edge\_iso\_B2}: natural isomorphism $H^{-1}\!\big(\Rfun(F)\big)\ \cong\ \varinjlim_{t}\ H_1\!\big(F^{t}C_\bullet\big)$ and $\Rfun(F)\in D^{[-1,0]}$.
  \item \texttt{ext1\_edge\_iso}: natural isomorphism $\Ext^1(\Rfun(\Ctau F),k)\ \cong\ \Hom\!\big(H^{-1}(\Rfun(\Ctau F)),k\big)$; in particular $H^{-1}=0$ iff $\Ext^1=0$ in this amplitude-$\le 1$ range.
\end{itemize}
\end{specification}

\begin{specification}[Aux-bars (measurement layer)]
\begin{itemize}
  \item \texttt{aux\_bins}: parameters $(a,b,\beta)$; right-open bins $I_r=[a+r\beta,a+(r+1)\beta)$; record under/overflow.
  \item \texttt{aux\_occupancy}: $E_r(j)=\#\{m\mid \lambda_m(j)\in I_r\}$ from the spectrum of $L(\Ctau F_j)$.
  \item \texttt{aux\_bars}: for fixed $r$, maximal consecutive runs $J$ with $E_r(j)>0$; lifetime $|J|$ (or rescaled).
  \item \texttt{aux\_monotone}: deletion-type steps $\Rightarrow$ non\hyp increase of active bins and total aux-bar mass; $\varepsilon$-continuations $\Rightarrow$ stability; inclusion-type $\Rightarrow$ stability only.
\end{itemize}
\end{specification}

\subsection*{11.11. Diagram (pipeline and logs)}
\begin{center}
\begin{tikzcd}[
  ampersand replacement=\&,
  column sep=2.4em, row sep=2.0em,
  cells={nodes={font=\footnotesize}},
  every label/.append style={font=\scriptsize},
  scale=0.9, transform shape
]
\text{Input }F \arrow[r, "\mathbf{P}_i"]
  \& \mathbf{P}_i(F) \arrow[r, "\Ttau"] \arrow[d, dashed, "{\mathrm{PE}_i^{\le \tau}}"]
    \& \Ttau\mathbf{P}_i(F) \arrow[d, dashed, "{\mathrm{PE}_i^{\le \tau}}"] \\\n\& L(\Ctau F) \arrow[r, dashed, "{\text{heat/aux on }L(\Ctau F)}"]
    \& \mathrm{HT},\ \mathrm{ST}_{\beta},\ \text{aux\hyp bars} \\\n\Rfun(F) \arrow[uu, bend left=40, dashed, "\LC"]
  \arrow[rr, dashed, "{\Ctau\ \text{ then }\Ext^1(-,Q)}"]
    \&\& \Ext^1(\Rfun(\Ctau F),Q)=0 \\\n\varinjlim_{t}\ \Ttau\mathbf{P}_i(F_t) \arrow[rr, "{\phi_{i,\tau}}"]
  \&\& \Ttau\mathbf{P}_i(F_\infty) \arrow[lld, bend right=25, swap, dashed, "{(\muc,\nuc)\ \text{log}}"] \\\n\text{failure log (pure kernel/cokernel/mixed)} \&\&
\end{tikzcd}
\end{center}

\subsection*{11.12. Completion note}
\begin{remark}[No further supplementation required]
This chapter fully integrates: (i) collapse energy and spectral indicators on $L(\Ctau F)$; (ii) auxiliary spectral bars (aux-bars) with binning, occupancies, lifetimes, and monotonicity/stability policies; (iii) the joint monitoring protocol with gate usage of aux-bars as optional auxiliary conditions; (iv) noise/discretization policies, including lifetime thresholds and reproducibility; (v) a minimal manifest schema and artifact set; (vi) a formalization stub for persistence, towers, bridge, and aux-bars. All claims remain within the v15.0 guard-rails (B-side single layer, PH1$\to$Ext1 one-way only, MECE windows, $\delta$-ledger, tower diagnostics); no additional supplementation is needed for operational use as a proof/measurement framework.
\end{remark}

\subsection*{11.13. Guard-rails}
\begin{remark}[Scope and non-claims]\label{rk:11-guard}
This chapter specifies measurement protocols, auxiliaries, and implementation notes at the persistence/spectral/categorical layers. No analytic regularity, group trivialization, or number-theoretic identity is asserted. All statements respect the guard-rails of Part~I; in particular, no claim of $\mathrm{PH}_1\Leftrightarrow \Ext^1$ is made, and $\muc$ differs from classical Iwasawa $\mu$.
\end{remark}



```latex
% ============================================================
% Part III — Measurement / Implementation / Tests
% ============================================================
\section{Chapter 12: Formal Test Suite and Open Problems}
\addcontentsline{toc}{section}{Formal Test Suite and Open Problems}

% --- Badge policy (explicit at chapter head) ---
\noindent\textbf{Badge policy.}
\begin{itemize}
  \item \textbf{[Prop]}: mathematics proved in Part~I (core results; cite exact source).
  \item \textbf{[Declaration]}: programmatic specification in the implementable range, verifiable by the test suite in this chapter.
  \item \textbf{[Conjecture]}: forward-looking statement; no claim beyond the stated scope.
\end{itemize}

% --- Notation & Conventions block for the whole chapter ---
\subsection*{12.0. Notation \& conventions}
\begin{itemize}
  \item \textbf{Constructible range.} We identify \(\Perskft\) with the constructible subcategory of \([\mathbb{R},\mathsf{Vect}_k]\) and use \(\Perskft\) uniformly (we do not use \(\Pers^{\mathrm{cons}}_k\) hereafter).
  \item \textbf{Truncation phrase.} “after applying \(\mathbf{T}_\tau\); equivalently on \(\Ctau F\)” is our standard phrase indicating that a quantity is computed at the persistence layer after truncation (hence equivalently on the filtered lift \(\Ctau F\)).
  \item \textbf{Generic–fiber dimension.} For a comparison map \(\phi_{i,\tau}\) at fixed \(\tau\), \(\dim_k\) denotes the \emph{generic–fiber dimension after truncation}, i.e.\ the multiplicity of \(I[0,\infty)\) summands in \(\Ttau\mathbf{P}_i(-)\); informally, the \(t\to\infty\) stable rank within the \(\tau\)–window.
  \item \textbf{Spectral ordering and norms.} Positive eigenvalues of \(L(\Ctau F)\) are listed in ascending order \(\lambda_1\le \lambda_2\le\cdots\). Matrix/operator norms are denoted by \(\|\cdot\|_{\mathrm{op}}\) (operator norm) and \(\|\cdot\|_{\mathrm{fro}}\) (Frobenius norm); each test specifies which is used and logs the choice.
  \item \textbf{Obstruction totals and macros.} We write \(\mu_{i,\tau}=\dim_k\ker\phi_{i,\tau}\), \(\nu_{i,\tau}=\dim_k\mathrm{coker}\,\phi_{i,\tau}\), and set the totals \(\muc:=\sum_i\mu_{i,\tau}\), \(\nuc:=\sum_i\nu_{i,\tau}\) (finite by bounded degree).
  \item \textbf{Endpoints.} Endpoint conventions and the handling of infinite bars follow the global policy (cf.\ Appendix~A, Remark~\ref{rk:A-endpoints}); infinite bars are not removed by \(\Ttau\) and are clipped by the window in all windowed quantities.
  \item \textbf{Non\hyp expansiveness.} We use the spelling “non\hyp expansive”/“non\hyp expansiveness” uniformly.
\end{itemize}

\subsection*{12.1. Badge inventory (representative items)}
\begin{center}
\begingroup
\renewcommand{\arraystretch}{1.12}
\begin{tabular}{@{}l p{0.74\textwidth}@{}}
\toprule
\textbf{Badge} & \textbf{Representative items (label / location)}\\
\midrule
\textbf{[Prop]} &
  Stability, idempotence, and exactness of $\Ttau$ (Prop.~\ref{prop:stability}, Ch.~2);\\
& Shift--commutation / $1$\nobreakdash-Lipschitz for $\Ttau$ (Lemma~\ref{lem:shift}, Ch.~2);\\
& Operational coreflection $\mathsf{C}_\tau^{\mathrm{comb}}$ on the implementable range
  (Prop.~\ref{prop:operational-coreflection}, Ch.~5);\\
& Tower diagnosis: $(\muc,\nuc)$ via cone extension; isomorphism criterion excluding Type~IV (Prop.~\ref{prop:mu-vanishing}, Ch.~4).\\
\textbf{[Thm]} &
  One\hyp way bridge: $\mathrm{PH}_1(F)=0 \Rightarrow \Ext^1(\mathcal{R}(F),k)=0$ under (B1)--(B3)\\
& \hspace{1.6em}(Thm.~\ref{thm:PH1-to-Ext1}, Ch.~3).\\
\textbf{[Declaration]} &
  Ch.~2: (co)limit and pullback compatibility \emph{at the persistence layer only} (after $\Ttau$);\\
& Ch.~6: filtered\hyp colimit stability in geometry; joint indicators; protocol (after truncation);\\
& Ch.~7: arithmetic tower stability; non\hyp identity of $\muc$ with Iwasawa $\mu$;\\
& Ch.~8: tropical shortening $\Rightarrow$ weak group collapse; mirror transfer \emph{non\hyp expansive after truncation};\\
& Ch.~9: three\hyp layer (Gal$\to$Trans$\to$Funct) compatibility as isomorphisms in $\Perskft$ \emph{after} $\Ttau\mathbf{P}_i$;\\
& Ch.~10: persistence\hyp guided regularization; AK--NS hypothesis (programmatic);\\
& Ch.~11: joint monitoring, noise/discretization policy, minimal test suite; central \textbf{Saturation gate} (Ch.~11.S).\\
\textbf{[Conjecture]} &
  Cross\hyp domain collapse propagation (Chs.~6--10); AK--NS (Ch.~10); mirror\hyp side propagation (Ch.~8);\\
& Functorial transfer stability (Ch.~9).\\
\bottomrule
\end{tabular}
\endgroup
\end{center}

\subsection*{12.2. Formal test suite (unit / integration / regression)}
All tests operate at the truncated persistence, spectral (on \(L(\Ctau F)\)), and categorical layers and are f.q.i.\ invariant at the persistence layer. A test \emph{passes} iff all stated pass\hyp criteria are met and logs are complete. Pass\hyp criteria must state whether indicators are evaluated \emph{per degree} or \emph{aggregated} across degrees; the choice must be fixed and logged for the run. Spectra use ascending order \(\lambda_1\le\lambda_2\le\cdots\), and the chosen norm \(\|\cdot\|_{\mathrm{op}}\) or \(\|\cdot\|_{\mathrm{fro}}\) must be declared and logged.

\paragraph{(T1) Stability under non\hyp expansive updates [Unit].}
\emph{Input:} pairs \(F\to F'\) with \(d_{\mathrm{int}}(\mathbf{P}_i(F),\mathbf{P}_i(F'))\le \varepsilon\).\\\n\emph{Assertions:} \(\big|\mathrm{PE}^{\le\tau}_i(F)-\mathrm{PE}^{\le\tau}_i(F')\big|\le C_{i,\tau,\alpha}\,\varepsilon^{\min\{1,\alpha\}}\) (computed after applying \(\Ttau\); equivalently on \(\Ctau F\)); spectra of \(L(\Ctau F)\) vs.\ \(L(\Ctau F')\) (eigenvalues ascending) satisfy the fixed \((\beta,M(\tau),t)\)\hyp policy stability bounds in the declared norm; \(\Ext^1(\Rfun(\Ctau-),Q)\) is \emph{stable under admissible f.q.i.\ updates} (for all \(Q\in\Qtest\)); in particular, if \(\Ext^1(\Rfun(\Ctau F),Q)=0\) at baseline, then \(\Ext^1(\Rfun(\Ctau F'),Q)=0\).\\\n\emph{Artifacts:} \texttt{bars.json}, \texttt{spec.json}, \texttt{ext.json}; \texttt{run.yaml} (norm and spectral policy recorded).

\paragraph{(T2) Monotone update (deletion\hyp /inclusion\hyp type) [Unit].}
\emph{Input:} \(F\to F'\) monotone.\\\n\emph{Assertions:} \emph{Deletion\hyp type:} \(\mathrm{PE}^{\le\tau}_i\) and spectral indicators are non\hyp increasing (after \(\Ttau\)); spectral eigenvalues are compared in ascending order in the declared norm. In the minimal two\hyp term cone tower (source \(F\), terminal \(F'\)) the comparison map \(\phi_{i,\tau}\) is an isomorphism, hence \((\muc,\nuc)=(0,0)\) at fixed \(\tau\).
\emph{Inclusion\hyp type:} stability (non\hyp expansiveness) only; no non\hyp increase is claimed.\\\n\emph{Artifacts:} \texttt{bars.json}, \texttt{spec.json}, \texttt{ext.json}, \texttt{phi.json}; \texttt{run.yaml} (update type, norm choice, and \((\muc,\nuc)\) recorded).

\paragraph{(T3) Filtered\hyp colimit stability [Integration].}
\emph{Input:} tower \(\{F_\lambda\}_\lambda\).\\\n\emph{Assertions:} for each fixed \(\tau\), comparison maps \(\phi_{i,\tau}:\varinjlim_\lambda \Ttau\mathbf{P}_i(F_\lambda)\xrightarrow{\cong}\Ttau\mathbf{P}_i(F_{\lambda_\ast})\); thus \((\muc,\nuc)=(0,0)\) and Type~IV excluded at that scale. \emph{Terminal symbol consistency:} within a run, the same terminal symbol (e.g.\ \(F_{\lambda_\ast}\) or \(F_\infty\)) is used and logged in \texttt{run.yaml}.\\\n\emph{Artifacts:} \texttt{phi.json} with ranks and \((\mu_{i,\tau},\nu_{i,\tau})\); \texttt{run.yaml} (terminal symbol).

\paragraph{(T4) Mirror/tropical pipeline [Integration].}
\emph{Input:} \(X\), tropical flow \(\Trop_\lambda\), realization \(F_\lambda\), mirror functor \(\Mirror\).\\\n\emph{Assertions:} tropical shortening factor \(\kappa\le 1\) implies non\hyp increase of \(\mathrm{PE}^{\le\tau}_i\) (after \(\Ttau\)); mirror transfer is \emph{non\hyp expansive after truncation} at the persistence level; group proxies (if used) meet weak\hyp collapse thresholds (Ch.~8). Spectral eigenvalues are reported in ascending order with the declared norm.\\\n\emph{Artifacts:} \texttt{bars.json}, \texttt{spec.json}, \texttt{ext.json}, \texttt{phi.json} (per \(\lambda\) and on the mirror side); \texttt{run.yaml}.

\paragraph{(T5) Three\hyp layer compatibility [Integration].}
\emph{Input:} Gal\(\to\)Trans\(\to\)Funct data with comparison natural transformations (Ch.~9).\\\n\emph{Assertions:} after applying \(\Ctau\) and \(\mathbf{P}_i\), commutativity holds up to isomorphism in \(\Perskft\) for each degree \(i\); indicators consistent across layers; failure logs record type if mismatches occur.\\\n\emph{Artifacts:} per\hyp layer \texttt{bars.json}, \texttt{spec.json}, \texttt{ext.json}; global \texttt{phi.json}; \texttt{run.yaml}.

\paragraph{(T6) PDE monitoring loop [Regression].}
\emph{Input:} index set \(I\) (time/resolution/parameter), realization \(\mathcal{P}\).\\\n\emph{Assertions:} protocol of Ch.~11 (Decl.~\ref{spec:11-monitor}) holds; stable regime declaration matches (logs \(\mathrm{PE}^{\le\tau}\), spectral indicators and aux-bars, \(\Ext^1\), \((\muc,\nuc)\)); pass\hyp criteria specify \emph{per\hyp degree vs.\ aggregated} reporting and are fixed and logged. Repeatability from \texttt{run.yaml}. Spectral eigenvalues ascending; norm choice declared.\\\n\emph{Artifacts:} \texttt{bars.json}, \texttt{spec.json}, \texttt{aux.json}, \texttt{ext.json}, \texttt{phi.json} over \(I\); \texttt{run.yaml}.

\paragraph{(T7) Saturation gate verification [Integration].}
\emph{Input:} a window \([0,\tau^\ast]\) with candidate saturation (Ch.~11.S).\\\n\emph{Assertions:} verify the quantitative saturation conditions on the window: (a) maximal \emph{finite} bar length \(\le \eta\); (b) interleaving bound \(\le \eta\) eventually; (c) edge gap \(\delta:=\tau^\ast-\max\{b_r<\tau^\ast\}>\eta\). Confirm the \textbf{[Spec]} adoption \(\mathrm{PH}_1(\C_{\tau^\ast}F)=0 \Leftrightarrow \Ext^1(\Rfun(\C_{\tau^\ast}F),k)=0\) is recorded and used only on the saturated window. Spectral reports use ascending ordering and the declared norm.\\\n\emph{Artifacts:} \texttt{bars.json}, \texttt{ext.json}; \texttt{run.yaml} (saturation parameters, norm, decision).

\paragraph{(T8) \(\varepsilon\)\hyp clipping regression [Unit].}
\emph{Input:} paired runs with unclipped vs.\ \(\varepsilon\)\hyp clipped \(\Ttau\mathbf{P}_i\) (Ch.~11.7).\\\n\emph{Assertions:} stability bound for \(\mathrm{PE}^{\le\tau}_i\) under clipping (theoretical upper bound matched up to tolerance); \((\muc,\nuc)\) are computed on unclipped data and remain identical across the pair; logs explicitly distinguish clipped vs.\ unclipped reporting.\\\n\emph{Artifacts:} \texttt{bars.json} (both variants), \texttt{phi.json}; \texttt{run.yaml}.

\paragraph{(T9) MECE window coverage \& event accounting [Unit].}
\emph{Input:} a MECE windowing \(\{[u_k,u_{k+1})\}_k\) and global range \([u_0,U)\).\\\n\emph{Assertions:} (a) \(\sum_k (u_{k+1}-u_k)=U-u_0\); (b) the total number of events (births/deaths counted with multiplicity) on \([u_0,U)\) equals the sum of per\hyp window counts up to rounding tolerance; (c) the collapse threshold \(\tau\) and spectral bin policy \((a,b,\beta)\) are \emph{identical} across windows unless explicitly recorded and justified in \texttt{run.yaml}.\\\n\emph{Artifacts:} \texttt{run.yaml} (windows/coverage\_check block), per\hyp window \texttt{bars.json}.

\paragraph{(T10) A/B commutativity test for reflectors [Unit/Integration].}
\emph{Input:} two persistence\hyp level reflectors \(T_A,T_B\) (e.g.\ length\hyp threshold and birth\hyp window), tolerance \(\eta\ge 0\).\\\n\emph{Assertions:} compute \(\Delta_{\mathrm{comm}}(M;A,B)=d_{\mathrm{int}}(T_A T_B M,\ T_B T_A M)\) on the relevant dataset \(M\). If \(\Delta_{\mathrm{comm}}\le \eta\), adopt \emph{soft\hyp commuting}; else fall back to a fixed order and record \(\Delta_{\mathrm{comm}}\) into \(\delta^{\mathrm{alg}}\) (Appendix~L). Pass if the outcome matches the configured policy in \texttt{run.yaml}.\\\n\emph{Artifacts:} \texttt{run.yaml} (A/B policy: \(\eta\), order, soft\hyp commuting flag), \texttt{bars.json} before/after.

\paragraph{(T11) Restart experiment \& Summability design [Integration/Regression].}
\emph{Input:} a MECE window sequence with collapse thresholds \(\tau_k\), budgets \(\Sigma\delta_k(i)\), and safety margins \(\mathrm{gap}_{\tau_k}\).\\\n\emph{Assertions:} (Restart) empirically verify \(\mathrm{gap}_{\tau_{k+1}}\ge \kappa(\mathrm{gap}_{\tau_k}-\Sigma\delta_k(i))\) for some \(\kappa\in(0,1]\) (record \(\kappa\)); (Summability) verify \(\sum_k \Sigma\delta_k(i)<\infty\) (e.g.\ geometric decay of \(\tau_k,\beta_k\) and bounded step counts); (Pasting) confirm that all B\hyp Gate\(^{+}\) certificates paste to a global certificate.\\\n\emph{Artifacts:} \texttt{run.yaml} (restart/summability fields), per\hyp window gate logs, cumulative certificate.

\paragraph{(T12) Trigger pack verification (domain\hyp restricted) [Integration].}
\emph{Input:} triggers declared for a domain (e.g.\ PDE, Ch.~10.4).\\\n\emph{Assertions:} for each trigger, verify that the declared analytical/numerical condition implies a B\hyp Gate\(^{+}\) failure on the window (e.g.\ sustained enstrophy surge \(\Rightarrow\) aux\hyp bars \(>0\) or \(\mu>0\)); record detection rate and false positives (if any). Triggers are \textbf{[Spec]} and complement (not replace) B\hyp Gate\(^{+}\).\\\n\emph{Artifacts:} \texttt{run.yaml} (trigger set, thresholds), \texttt{aux.json}, \texttt{phi.json}, gate verdicts.

\paragraph{(T13) $\delta$-ledger additivity \& pipeline budget [Integration].}
\emph{Input:} a pipeline of steps \(U_m,\dots,U_1\) with per\hyp step collapses \(C_{\tau_j}\) and bounds \(\delta_j(i,\tau_j)\).\\\n\emph{Assertions:} verify
\[
d_{\mathrm{int}}\!\Big(\Ttau \mathbf{P}_i\big(\Mirror(C_{\tau_m}\!\cdots C_{\tau_1}F)\big),\ \Ttau \mathbf{P}_i\big(C_{\tau_m}\!\cdots C_{\tau_1}\Mirror F\big)\Big)\ \le\ \sum_{j=1}^m \delta_j(i,\tau_j),
\]
and that any post\hyp processing by 1\hyp Lipschitz maps does not increase the bound (Appendix~L). Pass if the measured defect is \(\le\) the recorded budget.\\\n\emph{Artifacts:} \texttt{run.yaml} (\(\delta\)\hyp ledger), \texttt{bars.json} along the pipeline, distance logs.
  
\subsection*{12.3. Reproducibility and logs}
Every run ships with a manifest \texttt{run.yaml} declaring: sweep \(\tau_{\min}\!:\Delta\tau\!:\tau_{\max}\); spectral policy \((\beta,M(\tau),t)\); discretization (grid/complex, steps); seeds; software versions; tower index set and cone extension \emph{including the terminal symbol used (e.g.\ \(\lambda_\ast\) or \(\infty\))}; pass\hyp criteria (per\hyp degree vs.\ aggregated); norm choice \(\|\cdot\|_{\mathrm{op}}\) or \(\|\cdot\|_{\mathrm{fro}}\); A/B tolerance \(\eta\) (if applicable); Restart constants (\(\kappa\)) and Summability evidence; and file pointers to \texttt{bars.json}, \texttt{spec.json}, \texttt{aux.json}, \texttt{ext.json}, \texttt{phi.json} (optionally with an \texttt{.h5} mirror). All persistence quantities are computed after applying \(\Ttau\); equivalently on \(\Ctau F\); and remain invariant under f.q.i.\ at the persistence layer.

\begin{remark}[Audit checklist]\label{rk:12-audit}
(i) Constructibility (finite critical set) verified.\;
(ii) Field coefficients fixed (Novikov field allowed at \textbf{[Spec]}\hyp level).\;
(iii) Updates are \emph{deletion\hyp type} for monotonicity; inclusion\hyp type are stability\hyp only.\;
(iv) Interleaving shifts \(\varepsilon_n\) uniformly bounded (tower non\hyp expansion).\;
(v) After applying \(\Ttau\) record \(\mathrm{PE}\), heat trace/spectral tail, aux\hyp bars, \(\Ext^1\), and \((\mu,\nu)\) on the \emph{same window}.\;
(vi) \emph{LC activation order:} apply truncation \(\Ctau\), run persistence\hyp layer checks, then apply \(\Rfun\) with \(\LC\) to compute \(\Ext^1\) (one\hyp way bridge only).\;
(vii) For derived realizations, check PF/BC hypotheses (Appendix~N) \emph{before} kernel/Hecke/induction transfers; all non\hyp expansiveness claims are \emph{after truncation}.\;
(viii) Spectral eigenvalues reported in ascending order; chosen norm \(\|\cdot\|_{\mathrm{op}}\)/\(\|\cdot\|_{\mathrm{fro}}\) declared and logged; terminal symbol consistent across the run.\;
(ix) MECE coverage and event accounting satisfied;\;
(x) A/B commutativity test configured (\(\eta\)) and logged (soft\hyp commuting vs.\ fallback);\;
(xi) Restart/Summability evidenced with \(\kappa\) and \(\sum \Sigma\delta\).
\end{remark}

\noindent\emph{Manifest template (YAML).}
\small
\begin{verbatim}
coeff_field: "k"            # or "Novikov(q)" [Spec-level]
tau_window: [0.05, 1.0]     # start, end
tau_step: 0.05
spectral:
  tail_beta: 2
  tail_cutoff_M_of_tau: "floor(0.5 * tau^1.5)"
  heat_t: [0.5*tau^-2, 1.0*tau^-2]
  aux_bins: {a: 0.0, beta: 0.02, bins: 96, boundary: "right-open"}
tower:
  eps_interleave_max: 0.02
  terminal_symbol: "infty"   # or "lambda_star"
  cone_extension: true
ab_test:
  eta: 0.01
  policy: "soft-commuting"   # or "fallback:A_then_B"
restart_summability:
  kappa_min: 0.8
  sum_delta_bound: 0.05
windows:
  domain: [[0,1), [1,2), [2,3)]
  collapse_tau: 0.08
coverage_check:
  length_sum: 3.0
  length_target: 3.0
  events_sum_equals_global: true
record:
  bars: true
  PE: {report: "per-degree", clipping: "epsilon=0.02"}
  aux: {lifetime_min_frames: 3}
  heat_trace: {ordering: "ascending", norm: "op"}  # op or fro
  ext1: true
  mu_nu: true
notes: "deletion-type updates only; LC with Rfun after truncation; PF/BC verified where applicable"
\end{verbatim}
\normalsize

\subsection*{12.4. Open problems (selected)}
\begin{remark}[Open problems]\label{rk:12-open}
\hfill
\begin{enumerate}
  \item \textbf{Strengthening the bridge \( \mathrm{PH}_1\to \Ext^1 \).} Identify domain\hyp wise sufficient conditions (beyond (B1)–(B3)) ensuring vanishing of \(\Ext^1\) from quantitative decay of persistence energy/spectral tails. No converse/equivalence is claimed.
  \item \textbf{Persistence\hyp level colimit criteria.} Sharp hypotheses guaranteeing \((\muc,\nuc)=(0,0)\) across broader indexing classes (beyond objectwise degreewise colimits).
  \item \textbf{Failure lattice refinement.} Finer invariants separating pure/mixed failures and detecting “invisible’’ Type~IV precursors at nearby scales.
  \item \textbf{Spectral–persistence calibration.} Quantitative bounds linking the collapse energy \(\|\mathbf{CE}^{\le \tau}(F)\|_{1}\) and \(\mathrm{ST}_{\beta}^{\ge M(\tau)}\), robust to noise and discretization.
  \item \textbf{Weak group collapse.} Relate persistence\hyp level proxies (Ch.~8) to algebraic invariants (virtual nilpotence/solvabilization) without leaving the implementable range.
  \item \textbf{Arithmetic towers.} Domain\hyp specific templates ensuring tower stability and clarifying the relation (if any) between collapse diagnostics and Selmer/class growth; maintain the non\hyp identity with classical Iwasawa \(\mu\).
  \item \textbf{Langlands layer compatibility.} Minimal comparison data ensuring truncated commutativity across Gal\(\to\)Trans\(\to\)Funct in practical pipelines.
  \item \textbf{PDE program.} Conditions under which persistence\hyp guided regularization predicts classical regularity regimes while remaining purely programmatic.
  \item \textbf{Universality of \(\mathbf{T}_\tau\) as Serre localization.} Minimal assumptions (within constructible \(1\)D persistence) under which \(\mathbf{T}_\tau\) enjoys a universal property characterizing the implementable range.
\end{enumerate}
\end{remark}

\subsection*{12.5. Final guard\hyp rails}
\begin{remark}[Scope and non\hyp claims]\label{rk:12-guards}
All specifications are confined to the persistence/spectral/categorical layers in the implementable range and are verifiable by the test suite above. No number\hyp theoretic identity, analytic regularity theorem, or group trivialization is asserted. In particular, no claim of \(\mathrm{PH}_1\Leftrightarrow \Ext^1\) is made; only the one\hyp way implication under (B1)–(B3) is used. The obstruction \(\muc\) is a collapse diagnostic and is distinct from the classical Iwasawa \(\mu\).
\end{remark}

\subsection*{12.6. Conclusion}
This closing chapter consolidates a complete, testable interface for the program: a precise badge policy, a uniform notation layer, and a formal test suite spanning stability, monotone updates, filtered\hyp colimits, mirror/tropical flows, Langlands triples, and PDE pipelines. All persistence\hyp layer quantities are computed after applying \(\Ttau\); equivalently on \(\Ctau F\); spectral indicators are normalized (eigenvalues ascending; norm declared) with aux\hyp bar policies fixed; and categorical checks are performed only in the one\hyp way direction sanctioned by Part~I. Reproducibility is enforced by a single manifest and a minimal artifact set. With these guard\hyp rails, the \emph{implementable range} is not merely a blueprint but an executable methodology, inviting careful extensions along the open directions of Remark~\ref{rk:12-open}, while preserving the conservative stance articulated in Remark~\ref{rk:12-guards}.

\subsection*{12.7. Completion note}
\begin{remark}[No further supplementation required]
This chapter fully integrates: (i) MECE window tests and event accounting; (ii) A/B commutativity tests with tolerance and fallback; (iii) Restart experiments with Summability verification; (iv) Trigger pack validation; (v) $\delta$-ledger additivity checks; (vi) a minimal manifest with all audit fields. All items are consistent with the v15.0 guard\hyp rails and cross\hyp reference the proven core; no additional supplementation is needed for operational use as a formal test suite.
\end{remark}
```


% =========================
\section*{Notation and Conventions (reinforced v15.0)}
% =========================
\phantomsection
\addcontentsline{toc}{section}{Notation and Conventions}

\paragraph{Base field and ambient categories.}
Fix a coefficient field \(k\) (Appendices~N/O also admit a field \(\Lambda\); when used, replace \(k\) by \(\Lambda\) everywhere).
Let \(\mathsf{Vect}_k\) be the abelian category of finite\hyp dimensional \(k\)\hyp vector spaces.
Write \([\mathbb{R},\mathsf{Vect}_k]\) for functors \((\mathbb{R},\le)\to \mathsf{Vect}_k\).

\paragraph{Constructible persistence and standing identification.}
We write
\[
\Perskft\ \subset\ [\mathbb{R},\mathsf{Vect}_k]
\]
for the full subcategory of \emph{constructible} persistence modules (pointwise finite\hyp dimensional with locally finite critical set on bounded windows).
Throughout the paper we \emph{identify} the “finite\hyp type” category with this constructible subcategory and use the symbol \(\Perskft\) uniformly.

\paragraph{Filtered objects and persistence.}
\(\mathsf{FiltCh}(k)\) denotes filtered chain complexes of finite\hyp dimensional \(k\)\hyp spaces (filtered quasi\hyp isomorphism abbreviated f.q.i.).
For \(i\in\mathbb{Z}\) the degree–\(i\) persistence functor is
\[
\mathbf{P}_i:\ \mathsf{FiltCh}(k)\longrightarrow \Perskft,\qquad \mathbf{P}_i(F)(t)=\mathrm{H}_i(F^t).
\]
Realizations from other formalisms into \(\mathsf{FiltCh}(k)\) are denoted \(\mathcal{R}(-)\), \(\mathcal{F}(-)\) as appropriate.

\paragraph{Reflection/truncation.}
For \(\tau\ge 0\), let \(\mathsf{E}_\tau\subset\Perskft\) be the Serre subcategory generated by bars of length \(\le \tau\).
The reflector (truncation)
\[
\mathbf{T}_\tau:\ \Perskft\longrightarrow \mathsf{E}_\tau^\perp
\]
is exact, idempotent, and left adjoint to the inclusion \(\iota_\tau:\mathsf{E}_\tau^\perp\hookrightarrow\Perskft\) (Appendix~A, Theorem~\ref{A:thm:localization}); it is \(1\)\hyp Lipschitz for the interleaving metric (Appendix~A, Proposition~\ref{A:prop:lipschitz}). On filtered complexes we use a collapser \(C_\tau\) with a natural (up to f.q.i.) identification
\[
\mathbf{P}_i(C_\tau F)\ \cong\ \mathbf{T}_\tau(\mathbf{P}_iF)\qquad\text{(natural in \(F,i\)).}
\]

\paragraph{Interleaving metric and shifts.}
On \(\Perskft\) the interleaving metric \(d_{\mathrm{int}}\) equals the bottleneck distance in the constructible \(1\)D setting.
The time shift is \((S^\varepsilon M)(t):=M(t+\varepsilon)\); shifts commute canonically with \(\mathbf{T}_\tau\), hence \(\mathbf{T}_\tau\) is \(1\)\hyp Lipschitz.

\paragraph{Window clipping.}
For \(\tau\ge 0\) the window clip \(\mathbf{W}_{\le\tau}\) is restriction along \(i_{\le\tau}:[0,\tau]\hookrightarrow\mathbb{R}\) followed by extension by zero, viewed as an endofunctor \(\Perskft\to\Perskft\).
Restriction preserves \(\varepsilon\)\hyp interleavings since shifts commute with \(i_{\le\tau}^\ast\); thus \(\mathbf{W}_{\le\tau}\) is \(1\)\hyp Lipschitz (Appendix~I, Lemma~I:\,lem:clip-1lip).

\paragraph{“After truncation” (standard phrase).}
The phrase “\emph{after applying \(\mathbf{T}_\tau\); equivalently on \(\,C_\tau F\)}” indicates that a quantity is computed at the persistence layer on \(\mathbf{T}_\tau\mathbf{P}_i(F)\), which equals (by construction) \(\mathbf{P}_i(C_\tau F)\).

\paragraph{Barcodes, events, and endpoint convention.}
Barcodes use half\hyp open intervals \(I=[b,d)\) with \(d\in\mathbb{R}\cup\{\infty\}\) and multiplicity \(m(I)\in\mathbb{Z}_{\ge1}\).
Any consistent open/closed choice yields the same clipped lengths and event sets.
For \(\tau\ge 0\) the clipped length is
\[
\ell_{[0,\tau]}(I):=\max\{0,\min\{d,\tau\}-\max\{b,0\}\}.
\]
Given \(\tau_0>0\), the finite event set in degree \(i\) is
\[
\mathsf{Ev}_i(F;\tau_0)=\{0,\tau_0\}\ \cup\ \bigl(\{b\in[0,\tau_0]\}\cap\mathrm{births}\bigr)\ \cup\ \bigl(\{d\in[0,\tau_0]\}\cap\mathrm{deaths}\bigr).
\]
Endpoint conventions and the handling of infinite bars follow Appendix~A, Remark~\ref{A:rk:endpoints}.

\paragraph{Betti curves and Betti integral.}
\(\beta_i(F;t):=\dim_k \mathrm{H}_i(F^t)\) is c\`adl\`ag and piecewise constant on bounded windows.
The (clipped) Betti integral is
\[
\mathrm{PE}_i^{\le\tau}(F)=\int_0^\tau \beta_i(F;t)\,dt\ =\ \sum_{I\in\mathcal{B}_i(F)} m(I)\,\ell_{[0,\tau]}(I)\qquad\text{(Appendix~H).}
\]

\paragraph{Generic–fiber dimension.}
For \(M\in\Perskft\),
\[
\gdim(M):=\lim_{t\to+\infty}\dim_k M(t),
\]
which exists in the constructible range and, \emph{after applying \(\mathbf{T}_\tau\)}, equals the multiplicity of the infinite bar \(I[0,\infty)\) in the barcode.
If \(f:M\to N\) is a morphism, then for all \(t\gg0\),
\(\gdim(\ker f)=\dim_k\ker(f(t))\) and \(\gdim(\operatorname{coker} f)=\dim_k\operatorname{coker}(f(t))\).

\paragraph{Towers and diagnostics.}
A \emph{tower} is a directed system \(F=(F_n)_{n\in I}\) with colimit \(F_\infty\).
For \(i\in\mathbb{Z}\), \(\tau\ge 0\), the comparison map is
\[
\phi_{i,\tau}(F):\ \varinjlim_{n}\ \mathbf{T}_\tau\!\big(\mathbf{P}_i(F_n)\big)\longrightarrow \mathbf{T}_\tau\!\big(\mathbf{P}_i(F_\infty)\big).
\]
Set \(\mu_{i,\tau}(F):=\gdim\ker\phi_{i,\tau}(F)\) and \(\nu_{i,\tau}(F):=\gdim\operatorname{coker}\phi_{i,\tau}(F)\); the totals are
\[
\muc(F):=\sum_{i}\mu_{i,\tau}(F),\qquad \nuc(F):=\sum_{i}\nu_{i,\tau}(F),
\]
which are finite by bounded degree. Cofinal restriction leaves \((\muc,\nuc)\) unchanged; finite direct sums add; composition is subadditive (Appendix~J). If \(\phi_{i,\tau}\) is an isomorphism then \((\muc,\nuc)=(0,0)\) (Appendix~D/J; see Remark~\ref{rem:D-generic-dim}).

\paragraph{Windows (MECE), coverage, and \(\tau\)\hyp adaptation.}
A \emph{domain windowing} is a MECE partition \(\{[u_k,u_{k+1})\}_k\) (half\hyp open with right\hyp inclusion, consecutive, disjoint) of a monitored range \([u_0,U)\).
Coverage checks require \(\sum_k(u_{k+1}-u_k)=U-u_0\) and that the global event count equals the sum of per\hyp window counts (up to rounding).
Collapse thresholds \(\tau\) are \emph{resolution\hyp adapted} (e.g.\ \(\tau=\alpha\max\{\Delta t,\Delta x\}\)) and may be swept to identify \emph{stability bands} where \((\muc,\nuc)=(0,0)\).

\paragraph{Safety margin and pipeline budget.}
For a window \(W\) and degree \(i\), the \emph{safety margin} \(\mathrm{gap}_\tau>0\) is compared against the \emph{pipeline budget}
\[
\Sigma\delta(i,\tau)\ :=\ \sum_{U\in W}\Big(\delta_U^{\mathrm{alg}}(i,\tau)+\delta_U^{\mathrm{disc}}(i,\tau)+\delta_U^{\mathrm{meas}}(i,\tau)\Big),
\]
where \(\delta^{\mathrm{alg}}\) (algorithmic), \(\delta^{\mathrm{disc}}\) (discretization), \(\delta^{\mathrm{meas}}\) (measurement) are logged per step (Appendix~L). B\hyp Gate\(^{+}\) requires \(\mathrm{gap}_\tau>\Sigma\delta(i,\tau)\).

\paragraph{Deletion vs.\ inclusion (monotonicity policy).}
Deletion\hyp type operations (principal submatrices/Dirichlet, Schur complements, Loewner contractions, adding stops, shrinking sectors, etc.) are order\hyp decreasing in the relevant Loewner sense; indicators are \emph{non\hyp increasing}.
Inclusion\hyp type operations are asserted \emph{stable} (non\hyp expansive); no monotone inequality is claimed without extra structure (Appendix~E).
Loewner order \(A'\preceq A\) is used primarily in PSD contexts; for eigenvalue counts \(N_\theta\) we implicitly assume \(\theta\ge0\) when required.

\paragraph{Spectral notation and norms.}
Positive eigenvalues are listed in ascending order \(\lambda_1(A)\le\cdots\le\lambda_n(A)\).
Left/right counts at \(\theta\) are \(N_{\theta-0}(A):=\#\{j:\lambda_j(A)<\theta\}\), \(N_{\theta+0}(A):=\#\{j:\lambda_j(A)\le\theta\}\).
Clipped sum/deficit at level \(\tau\ge0\) are
\[
S^{\le\tau}(A):=\sum_{j=1}^n \min\{\lambda_j,\tau\},\qquad\nD^{<\tau}(A):=\sum_{j=1}^n (\tau-\lambda_j)_+.
\]
Operator/Frobenius norms are denoted \(\|\cdot\|_{\mathrm{op}}\), \(\|\cdot\|_{\mathrm{fro}}\) (Appendix~E); Appendix~G records these as strings \texttt{"op"}, \texttt{"fro"} in \texttt{run.yaml}/artifacts.

\paragraph{Auxiliary spectral bars (aux-bars).}
Fix a spectral window \([a,b]\) and bin width \(\beta>0\); define right\hyp open bins \(I_r=[a+r\beta,a+(r+1)\beta)\) and occupancies
\(E_r(j)=\#\{m\mid \lambda_m(j)\in I_r\}\) from the spectrum of \(L(\Ctau F_j)\) (under/overflow are logged).
For fixed \(r\), an \emph{aux-bar} is a maximal consecutive index run \(J\) with \(E_r(j)>0\), and its lifetime is \(|J|\) (or a fixed rescaling).
After collapse, aux\hyp bar mass and active bins are \emph{non\hyp increasing} under deletion\hyp type steps, \emph{stable} under \(\varepsilon\)\hyp continuations, and \emph{stable only} under inclusion\hyp type steps (Chapter~11).
Aux\hyp bars are \emph{auxiliary} for gates; B\hyp Gate\(^{+}\) may optionally require aux\hyp bars\(=0\) on the monitored bins.

\paragraph{Mirror/tropical; \(\delta\)\hyp commutation; A/B policy.}
\(\Mirror\) denotes a persistence\hyp level endofunctor (or a functor on filtered complexes composed with \(\mathbf{P}_i\)).
A natural \(2\)\hyp cell \(\theta:\Mirror\circ C_\tau \Rightarrow C_\tau\circ \Mirror\) controlled by \(\delta(i,\tau)\ge0\) yields
\[
d_{\mathrm{int}}\big(\mathbf{T}_\tau\mathbf{P}_i(\Mirror C_\tau F),\ \mathbf{T}_\tau\mathbf{P}_i(C_\tau\Mirror F)\big)\le \delta(i,\tau)\quad\text{(Appendix~L),}
\]
and pipeline bounds add: \(\sum_j\delta_j(i,\tau_j)\) (uniform in \(F\)); post\hyp processing by \(1\)\hyp Lipschitz maps does not increase the bound.
For two reflectors \(T_A,T_B\), the \emph{A/B test} measures
\(\Delta_{\mathrm{comm}}(M;A,B)=d_{\mathrm{int}}(T_AT_BM,T_BT_AM)\).
If \(\Delta_{\mathrm{comm}}\le \eta\), adopt \emph{soft\hyp commuting}; else fall back to a fixed order and record \(\Delta_{\mathrm{comm}}\) into \(\delta^{\mathrm{alg}}\).

\paragraph{Base change and projection formula.}
Proper base change and the projection formula are applied objectwise in \(t\) in the source category, then transported by \(\mathbf{P}_i\) and \(\mathbf{T}_\tau\).
Exactness of \(\mathbf{T}_\tau\) preserves short exact sequences induced by these isomorphisms (Appendix~N).

\paragraph{Fukaya realization.}
Action\hyp filtered Floer realizations are denoted \(\mathcal{F}(-)\).
Continuation maps with filtration increase \(\le\varepsilon\) produce \(\varepsilon\)\hyp interleavings on \(\mathbf{P}_i\) (and after \(\mathbf{T}_\tau\)); adding stops/shrinking sectors is treated as deletion\hyp type (Appendix~O).

\paragraph{\(\Ext^1\) tests and bridge scope.}
All \(\Ext^1\) tests are against the unit \(k[0]\) (or \(\Lambda[0]\) where applicable).
Where \(\mathcal{R}(C_\tau F)\) has amplitude \([-1,0]\), the edge identification \(\mathrm{PH}_1\Rightarrow\Ext^1\) is used only within its declared scope; no reverse implication is assumed.

\paragraph{Energy exponent, B\hyp Gate\(^{+}\), and type labels.}
\(\alpha>0\) is the fixed energy exponent (default \(\alpha=1\)).
\emph{B\hyp Gate\(^{+}\)} on a window requires (i) \(\mathrm{PH}_1=0\), (ii) (in scope) \(\Ext^1=0\), (iii) \((\muc,\nuc)=(0,0)\) after \(\mathbf{T}_\tau\), and (iv) \(\mathrm{gap}_\tau>\Sigma\delta(i,\tau)\); aux\hyp bars\(=0\) may be optionally required.
Labels \emph{Type I–II / Type III / Type IV} classify tower behaviors for diagnostics (Appendices~D/E) only.

\paragraph{Reproducibility (manifest).}
The manifest \texttt{run.yaml} records: the \(\tau\)\hyp sweep and windowing (MECE \& coverage checks), spectral policy \((\beta,M(\tau),t)\), aux\hyp bar bins \(([a,b],\beta)\), norm choice \(\|\cdot\|_{\mathrm{op}}\)/\(\|\cdot\|_{\mathrm{fro}}\), A/B tolerance \(\eta\), \(\delta\)\hyp ledger per step, Restart/Summability parameters, tower terminal symbol, seeds, and software versions (Appendix~G).

\medskip
\noindent\emph{Abbreviations.}
f.q.i.\ = filtered quasi\hyp isomorphism;\quad
c\`adl\`ag = right\hyp continuous with left limits;\quad
“window” \(=\) interval \([0,\tau]\) with \(\tau\ge0\).




% =========================
\appendix
\section*{Appendix A. Constructible Persistence: Abelianity and Serre Localization (reinforced)}
% =========================

\addcontentsline{toc}{section}{Appendix A. Constructible Persistence: Abelianity and Serre Localization}

Throughout this appendix, fix a field \(k\).
Write \(\Pers_k\) for the category of (right-continuous) persistence modules
\(M:(\mathbb{R},\le)\to\mathsf{Vect}_k\) with structure maps \(M(t\le t')\).
We denote by \(\Pers^{\mathrm{ft}}_k\subset\Pers_k\) the \emph{constructible} (finite-type) subcategory used in the main text.

\medskip
\noindent\textbf{Global conventions.}
(i) All \(\Ext\)-tests are taken against \(Q=k[0]\).
(ii) Windowed energies use an exponent \(\alpha>0\) (default \(\alpha=1\)).
(iii) References to appendices use the tilde style (e.g.\ Appendix~D); failure types use the dash style
\(\mathrm{Type\ I\text{--}II}\), \(\mathrm{Type\ III}\), \(\mathrm{Type\ IV}\).
(iv) For truncation we write \(\mathbf{T}_\tau\) (elsewhere we sometimes use the shorthand \(\Ttau=\mathbf{T}_\tau\)).
(v) When window partitions are used (MECE; §A.6), \emph{half-open with right-inclusion} is the standing endpoint convention for domain windows and spectral bins; coverage checks are mandatory.

% -------------------------
\subsection*{A.1. Constructible objects}
% -------------------------
\begin{definition}[Constructible / finite-type]\label{A:def:constructible}
A persistence module \(M\in\Pers_k\) is \emph{constructible} (finite-type) if on every bounded interval
\([a,b]\subset\mathbb{R}\) it has a \emph{finite critical set}: there exist
\(a=t_0<t_1<\dots<t_N=b\) such that each structure map
\(M(t\le t')\) is an isomorphism whenever \(t,t'\in (t_j,t_{j+1})\) for some \(j\).
Equivalently, \(M\) is pointwise finite-dimensional and admits a barcode decomposition as a
\emph{locally finite direct sum of interval modules}, i.e.\ only finitely many intervals intersect any bounded window.
We write \(\Pers^{\mathrm{ft}}_k\) for the full subcategory of such modules.
\end{definition}

\begin{remark}
In the 1D, field-coefficient, right-continuous setting, the equivalence above is standard (barcode decomposition), and all constructions below (kernels, cokernels, torsion, truncation) preserve constructibility and are determined by finitely many events on bounded windows; see the references cited at the end of this appendix.
\end{remark}

% -------------------------
\subsection*{A.2. Abelianity}
% -------------------------
\begin{proposition}\label{A:prop:abelian}
\(\Pers^{\mathrm{ft}}_k\) is an abelian category.
Moreover, for a morphism \(f:M\to N\) in \(\Pers^{\mathrm{ft}}_k\),
kernels and cokernels are computed pointwise in \(\mathsf{Vect}_k\) and remain constructible.
\end{proposition}

\begin{proof}
Evaluation at each \(t\in\mathbb{R}\) is exact in \(\mathsf{Vect}_k\), hence pointwise kernels and cokernels define functorial sub/quotient persistence modules.
Constructibility is preserved: on any bounded window one refines the break sets of \(M,N\) to a finite set controlling \(\Ker f\) and \(\Coker f\).
Exactness axioms follow objectwise; hence \(\Pers^{\mathrm{ft}}_k\) is abelian with pointwise exactness.
\end{proof}

% -------------------------
\subsection*{A.3. The \texorpdfstring{$\tau$}{tau}-ephemeral Serre subcategory}
% -------------------------
Fix \(\tau>0\).
Let \(I[a,b)\) denote the interval module supported on \([a,b)\) (with the usual endpoint convention).

\begin{definition}[\(\tau\)-ephemeral subcategory]\label{A:def:Etau}
Let \(\mathsf{E}_\tau\subset\Pers^{\mathrm{ft}}_k\) be the smallest full subcategory
containing all interval modules \(I[a,b)\) with length \(b-a\le \tau\) and closed under subobjects, quotients, and extensions.
We call \(\mathsf{E}_\tau\) the \emph{\(\tau\)-ephemeral} (or \(\tau\)-torsion) subcategory.
\end{definition}

\begin{lemma}\label{A:lem:Serre}
\(\mathsf{E}_\tau\) is a hereditary Serre subcategory of \(\Pers^{\mathrm{ft}}_k\):
it is closed under subobjects, quotients, and extensions, and subobjects of objects in \(\mathsf{E}_\tau\) remain in \(\mathsf{E}_\tau\).
\end{lemma}

\begin{proof}
By Definition~\ref{A:def:Etau} it is the Serre subcategory generated by length-\(\le\tau\) intervals.
Hereditariness follows because every subobject of a finite direct sum of length-\(\le\tau\) intervals admits a finite filtration whose composition factors have length \(\le\tau\).
\end{proof}

\begin{remark}[Endpoint conventions]\label{A:rk:endpoints}
All statements in this appendix are insensitive to the choice of open/closed endpoints on interval modules.
For definiteness we use half-open intervals \([a,b)\); changing endpoint conventions does not affect lengths, barcode decompositions, interleaving/bottleneck distances, or any categorical constructions below.
\end{remark}

\begin{remark}[1D constructible context]
In the one-parameter (1D) constructible setting considered here, \(\mathsf{E}_\tau\) is a hereditary torsion class inside the abelian category \(\Pers^{\mathrm{ft}}_k\), which is \emph{locally finite on bounded windows}.
This is standard in the barcode framework (see Crawley–Boevey 2015; Chazal–de~Silva–Glisse–Oudot 2016).
\end{remark}

% -------------------------
\subsection*{A.4. The reflector \texorpdfstring{$\mathbf{T}_\tau\dashv\iota_\tau$}{T\_\tau ⊣ ι\_\tau} and exactness (constructible 1D)}
% -------------------------
All localization statements in A.4–A.5 are made and used in the 1D constructible (finite-type) setting.

Let \(\iota_\tau:\Pers^{\mathrm{ft}}_{k,\tau\text{-loc}}\hookrightarrow\Pers^{\mathrm{ft}}_k\) be the inclusion of the full subcategory of \emph{\(\tau\)-local (orthogonal)} objects, i.e.\ those \(X\) such that
\[
\Hom(E,X)=0=\Ext^1(E,X)\qquad\text{for all }E\in\mathsf{E}_\tau.
\]
\footnote{In general, “\(\tau\)-local” (orthogonal) is strictly stronger than “\(\tau\)-torsion-free” (which only requires \(\Hom(\mathsf{E}_\tau,-)=0\)).
In 1D interval settings, \(\Ext^1\)-classes between short and long intervals can be nontrivial; hence “torsion-free \(\neq\) local’’ in general.}

\begin{theorem}[Exact reflective localization]\label{A:thm:localization}
The Serre quotient functor
\[
\pi_\tau:\ \Pers^{\mathrm{ft}}_k\ \longrightarrow\ \Pers^{\mathrm{ft}}_k/\mathsf{E}_\tau
\]
is exact. Moreover, in the 1D constructible setting there is a canonical exact equivalence of abelian categories
\[
\Pers^{\mathrm{ft}}_k/\mathsf{E}_\tau \ \simeq\ \Pers^{\mathrm{ft}}_{k,\tau\text{-loc}}.
\]
Composing \(\pi_\tau\) with this equivalence yields a functor
\[
\mathbf{T}_\tau:\ \Pers^{\mathrm{ft}}_k\ \longrightarrow\ \Pers^{\mathrm{ft}}_{k,\tau\text{-loc}}
\]
which is left adjoint to \(\iota_\tau\) and is exact (hence additive).
As a left adjoint, \(\mathbf{T}_\tau\) preserves all colimits that exist in \(\Pers^{\mathrm{ft}}_k\) (interpreted via Remark~\ref{A:rk:filtered-colimits});
moreover, an additive exact functor between abelian categories preserves \emph{finite limits and finite colimits}.
\end{theorem}

\begin{proof}
Since \(\mathsf{E}_\tau\) is Serre (Lemma~\ref{A:lem:Serre}), the abelian Serre quotient exists and \(\pi_\tau\) is exact.
By Gabriel localization for Serre subcategories, in our 1D constructible context the quotient \(\Pers^{\mathrm{ft}}_k/\mathsf{E}_\tau\) is equivalent to the full subcategory of \(\mathsf{E}_\tau\)-\emph{local (orthogonal)} objects, i.e.\ those \(X\) with \(\Hom(E,X)=\Ext^1(E,X)=0\) for all \(E\in\mathsf{E}_\tau\).
Transporting along this equivalence yields the adjunction \(\mathbf{T}_\tau\dashv\iota_\tau\).
Exactness and additivity imply preservation of finite limits and finite colimits; being a left adjoint implies preservation of (existing) colimits, with the filtered-colimit policy as in Remark~\ref{A:rk:filtered-colimits}.
\end{proof}

\begin{remark}[Filtered colimits: functor-category computation and return to constructible]
\label{A:rk:filtered-colimits}
Filtered colimits are computed objectwise in the functor category \([\mathbb{R},\mathsf{Vect}_k]\), and \(\mathbf{T}_\tau\) commutes with those colimits there (as a left adjoint).
A filtered colimit of constructible modules may exit \(\Pers^{\mathrm{ft}}_k\).
In applications we either: (i) restrict to towers that remain constructible degreewise; or (ii) compute in \([\mathbb{R},\mathsf{Vect}_k]\), apply \(\mathbf{T}_\tau\), and \emph{verify} that the result returns to \(\Pers^{\mathrm{ft}}_k\) (finite critical set on bounded windows).
\textbf{This policy is assumed throughout the paper whenever filtered colimits appear, and no claim is made outside this regime.}
\end{remark}

% -------------------------
\subsection*{A.5. Shift-commutation and 1-Lipschitz continuity}
% -------------------------
For \(\varepsilon\ge 0\), let \(S^\varepsilon:\Pers_k\to\Pers_k\) be the shift \((S^\varepsilon M)(t):=M(t+\varepsilon)\).

\begin{lemma}[Shift commutation]\label{A:lem:shift}
For all \(\varepsilon\ge 0\), there is a canonical isomorphism
\(\mathbf{T}_\tau\circ S^\varepsilon \;\cong\; S^\varepsilon\circ \mathbf{T}_\tau.\)
\end{lemma}

\begin{proof}
Shifts preserve constructibility and interval lengths and hence preserve \(\mathsf{E}_\tau\).
Therefore \(S^\varepsilon\) descends to the Serre quotient and commutes with \(\pi_\tau\); transporting across the equivalence with \(\Pers^{\mathrm{ft}}_{k,\tau\text{-loc}}\) gives the claim.
\end{proof}

\begin{proposition}[Non-expansiveness for the interleaving metric]\label{A:prop:lipschitz}
In the 1D constructible/barcode setting, \(\mathbf{T}_\tau\) is \(1\)-Lipschitz for the interleaving (equivalently, bottleneck) distance on \(\Pers^{\mathrm{ft}}_k\).
\end{proposition}

\begin{proof}
If \(M,N\) are \(\varepsilon\)-interleaved via shifts \(S^\varepsilon\), then by Lemma~\ref{A:lem:shift} the same diagrams exhibit an \(\varepsilon\)-interleaving between \(\mathbf{T}_\tau M\) and \(\mathbf{T}_\tau N\).
Thus \(d_{\mathrm{int}}(\mathbf{T}_\tau M,\mathbf{T}_\tau N)\le d_{\mathrm{int}}(M,N)\).
\end{proof}

\begin{remark}[Standard references]
The barcode decomposition and the resulting torsion/localization picture in 1D constructible persistence trace back to Crawley–Boevey (2015) and to the stability/structure framework summarized by Chazal–de~Silva–Glisse–Oudot (2016).
For Serre/Gabriel localization and the identification of the quotient with the \emph{local (orthogonal)} subcategory see, e.g., Gabriel (1962) or standard expositions (Popescu, \emph{Abelian Categories}; Stacks Project, Tag~02MO).
The \(1\)-Lipschitz property of \(\mathbf{T}_\tau\) follows from shift-commutation and the interleaving formalism in this 1D setting.
\end{remark}

% -------------------------
\subsection*{A.6. Windowing (MECE), coverage checks, and \texorpdfstring{$\tau$}{tau}-adaptation}
% -------------------------
\begin{definition}[MECE domain windowing and coverage]\label{A:def:MECE}
A \emph{domain windowing} is a finite or countable collection of half-open intervals with right-inclusion
\(\{[u_k,u_{k+1})\}_{k\in K}\) such that:
\begin{itemize}
  \item \emph{(Disjointness)} \([u_k,u_{k+1})\cap [u_\ell,u_{\ell+1})=\varnothing\) for \(k\neq \ell\);
  \item \emph{(Contiguity)} \(u_{k+1}=u_k+\mathrm{len}_k\) with \(\mathrm{len}_k>0\);
  \item \emph{(Coverage)} \(\bigsqcup_{k\in K}[u_k,u_{k+1})=[u_0,U)\) for some finite \(U>u_0\).
\end{itemize}
The \emph{coverage checks} require
\[
\sum_{k\in K}(u_{k+1}-u_k)=U-u_0,\qquad\n\#\mathrm{Events}([u_0,U))=\sum_{k\in K}\#\mathrm{Events}([u_k,u_{k+1}))\ \ (\pm\ \text{rounding}),
\]
where \(\#\mathrm{Events}([a,b))\) counts births/deaths with multiplicity recorded by the chosen endpoint convention.
\end{definition}

\begin{remark}[Alignment of windows and truncations]
When persistence and spectral measurements are combined, the domain windowing \(\{[u_k,u_{k+1})\}\), the collapse threshold(s) \(\tau\), and spectral bin windows \([a,b]\) must be \emph{fixed per window} and recorded. All B\hyp side measurements are taken \emph{after} applying \(\mathbf{T}_\tau\) and on the same window; clipping respects the half-open, right-inclusion convention.
\end{remark}

\begin{definition}[\(\tau\)-adaptation, sweep, and stability bands]\label{A:def:tau-adapt}
A collapse threshold \(\tau\) is \emph{resolution-adapted} if there is a constant \(\alpha>0\) such that
\[
\tau\ =\ \alpha\cdot \max\{\Delta t,\Delta x\},
\]
for the temporal/spatial mesh sizes \((\Delta t,\Delta x)\) of the data/solver. A \emph{\(\tau\)-sweep} is a discrete set \(\{\tau_\ell\}\) on which diagnostics \((\mu_{i,\tau_\ell},\nu_{i,\tau_\ell})\) are evaluated. A \emph{stability band} is a contiguous range \(B\subset (0,\infty)\) such that \(\phi_{i,\tau}\) is an isomorphism for all \(\tau\in B\) (hence \(\mu_{i,\tau}=\nu_{i,\tau}=0\)) and for all monitored degrees \(i\).
\end{definition}

\begin{remark}[Persistence/spectral window policy]
Spectral indicators and auxiliary spectral bars (Chapter~11) are computed on \(L(C_\tau F)\) over a spectral window \([a,b]\) with right-open bins of width \(\beta\). The pair \(([a,b],\beta)\) is fixed on each domain window and logged; under/overflow counts are recorded to ensure coverage.
\end{remark}

\begin{remark}[Restart/Summability interface]
Windowed certificates produced on MECE partitions paste globally if (i) the Restart inequality holds for safety margins from one window to the next, and (ii) the cumulative \(\delta\)-budget is summable (Chapter~4). The windowing and \(\tau\)-adaptation rules above are the only assumptions on Appendix~A required to support that interface.
\end{remark}

\medskip
\noindent\emph{Summary of Appendix A.}
\(\Pers^{\mathrm{ft}}_k\) is abelian with pointwise exactness; the \(\tau\)-ephemeral subcategory \(\mathsf{E}_\tau\) is hereditary Serre.
The Serre localization yields an \emph{exact} reflective localization \(\mathbf{T}_\tau\dashv\iota_\tau\) onto the \(\tau\)-\emph{local (orthogonal)} subcategory in the 1D constructible regime, preserving finite limits/finite colimits and (existing) colimits, commuting with shifts, hence \(1\)-Lipschitz.
Filtered colimits are handled in the functor category with an explicit return-to-constructible check as in Remark~\ref{A:rk:filtered-colimits}.
The windowing (MECE) layer (§A.6) adds a reproducible operational shell: half-open domain windows with coverage checks, resolution-adapted \(\tau\), \(\tau\)-sweeps, and stability bands; this integrates seamlessly with collapse, spectral auxiliaries, and the Restart/Summability pasting in Chapter~4.
```



% =========================
\appendix
\section*{Appendix B. Lifting \texorpdfstring{$\mathbf{T}_\tau$}{T\_\tau} to \texorpdfstring{$C_\tau$}{C\_\tau} and the Homotopy Setting (reinforced)}
% =========================

\addcontentsline{toc}{section}{Appendix B. Lifting $T_\tau$ to $C_\tau$ and the Homotopy Setting}

Throughout, fix a field \(k\).
Let \(\mathsf{FiltCh}(k)\) denote the category of \emph{bounded-in-degree} filtered chain complexes of finite-dimensional \(k\)-vector spaces with filtration-preserving chain maps.
(“Bounded” refers to homological degree; filtrations are assumed \emph{locally finite on bounded windows} as in Appendix~A.)
For each homological degree \(i\), write
\[
\mathbf{P}_i:\ \mathsf{FiltCh}(k)\longrightarrow \Pers^{\mathrm{ft}}_k,\qquad\nF\longmapsto \big(t\mapsto H_i(F^{t}C_\bullet)\big),
\]
the degreewise persistence functor into the constructible subcategory (Appendix~A).

\medskip
\noindent\textbf{Scope rule and monotonicity convention.}
Filtered (co)limits, when invoked, are computed objectwise in \([\mathbb{R},\mathsf{Vect}_k]\), and we then \emph{verify} that the result lies in (or returns to) \(\Pers^{\mathrm{ft}}_k\) (Appendix~A, Remark~\ref{A:rk:filtered-colimits}); no claim is made outside this regime.
Deletion-type updates are non-increasing for windowed energies and spectral tails after truncation, whereas inclusion-type updates are only stable (non-expansive); see Appendix~E.
Equalities at the filtered–complex layer are asserted \emph{up to filtered quasi-isomorphism (f.q.i.)}.
We use \(\mathbf{T}_\tau\) for truncation (occasionally \(\Ttau=\mathbf{T}_\tau\) as shorthand on first use).
Endpoint conventions follow Appendix~A (Remark~\ref{A:rk:endpoints}); in particular, infinite bars are not removed by \(\mathbf{T}_\tau\) and their contributions are clipped by windowing.

% -------------------------
\subsection*{B.1. The interval-realization assignment \texorpdfstring{$\mathcal{U}$}{U} (up to f.q.i.)}
% -------------------------
\begin{definition}[Elementary interval blocks (two-term/one-term model)]
Let \(I[a,b)\) be an interval module (fixed endpoint convention; Appendix~A, Remark~\ref{A:rk:endpoints}).
\begin{itemize}
  \item If \(b<+\infty\), realize \(I[a,b)\) in homological degree \(i\) by a \emph{two-term filtered block}
  \[
    k\cdot y \xrightarrow{\,d\,} k\cdot x,\qquad |y|=i+1,\ |x|=i,\quad
      \mathrm{fil}(x)=a,\ \mathrm{fil}(y)=b,\ d(y)=x,\ d(x)=0.
  \]
  Then \(x\) contributes a bar born at \(a\) and killed at \(b\).
  \item If \(b=+\infty\), realize \(I[a,\infty)\) by a \emph{one-term} block \(k\cdot x\) in degree \(i\) with \(\mathrm{fil}(x)=a\) and \(d=0\).
\end{itemize}
In all blocks, the differential preserves the filtration: \(d(F^t)\subseteq F^t\) for every \(t\).
Since \(a\le b\), for \(t\ge b\) one has \(x,y\in F^t\) and \(d(y)=x\in F^t\); for \(a\le t<b\) one has \(x\in F^t\), \(y\notin F^t\), whence \(d\) restricts to \(0\) on \(F^t\).
Taking \emph{locally finite on bounded windows} direct sums of such blocks and applying degree shifts produces a filtered complex whose persistence recovers the prescribed bars.
We call any such model an \emph{elementary interval complex} and denote a representative by \(\mathcal{I}[a,b)\).
\end{definition}

\begin{proposition}[Barcode realization for bounded families (up to f.q.i.)]\label{B:prop:U}
There exists an assignment
\[
\mathcal{U}:\ \Pers^{\mathrm{ft}}_k\longrightarrow \mathsf{FiltCh}(k)
\]
such that for any \emph{degree-bounded} family \(\{M_i\}_{i\in\mathbb{Z}}\) of constructible persistence modules
(only finitely many \(i\) nonzero) there are natural isomorphisms in \(\Pers^{\mathrm{ft}}_k\),
\[
\mathbf{P}_i\!\Big(\,\bigoplus_{j}\mathcal{U}(M_j)[-j]\Big)\ \cong\ M_i\qquad(\forall i).
\]
The construction is canonical \emph{up to} filtered quasi-isomorphism, additive, and functorial in the homotopy category \(\Ho(\mathsf{FiltCh}(k))\).
In particular, for a single module \(M\) realized in a base degree (say \(0\)) one has \(\mathbf{P}_0(\mathcal{U}(M))\cong M\) and \(\mathbf{P}_j(\mathcal{U}(M))=0\) for all \(j\neq 0\).
\end{proposition}

\begin{remark}[Pseudofunctoriality of \(\mathcal{U}\)]
The assignment \(\mathcal{U}\) extends to a \emph{pseudofunctor}
\(\mathcal{U}:\Pers^{\mathrm{ft}}_k\to \Ho(\mathsf{FiltCh}(k))\): on a morphism of persistence modules, choose interval decompositions and a bar-matching; the induced blockwise filtered chain map is well-defined in \(\Ho\) \emph{up to} f.q.i., and compositions are respected up to coherent isomorphism.
In dimension \(1\) with constructibility, the needed (pseudo)naturality follows from standard barcode calculus (e.g.\ Crawley--Boevey (2015); Chazal--de~Silva--Glisse--Oudot (2016)).
Consequently, constructions below that use \(\mathcal{U}\) on morphisms (e.g.\ \(C_\tau\)) are functorial on \(\Ho(\mathsf{FiltCh}(k))\).
\end{remark}

\begin{proof}[Proof sketch of Proposition~\ref{B:prop:U}]
Choose barcode decompositions \(M_i\simeq \bigoplus_{j\in J_i} I[a_{i,j},b_{i,j})\) (locally finite on bounded windows) and set
\(\bigoplus_{i}\bigoplus_{j\in J_i}\mathcal{I}[a_{i,j},b_{i,j})[-i]\).
Different decompositions yield filtered complexes that are f.q.i.-equivalent, hence define the same object (and morphisms) in \(\Ho(\mathsf{FiltCh}(k))\).
\end{proof}

% -------------------------
\subsection*{B.2. Filtered quasi-isomorphisms and \texorpdfstring{$\Ho(\mathsf{FiltCh}(k))$}{Ho(FiltCh(k))}}
% -------------------------
\begin{definition}[Filtered quasi-isomorphism]
A filtration-preserving chain map \(f:F\to G\) is a \emph{filtered quasi-isomorphism (f.q.i.)} if for every \(t\in\mathbb{R}\) the map \(F^{t}C_\bullet\to G^{t}C_\bullet\) is a quasi-isomorphism.
Equivalently, for all \(i\), \(\mathbf{P}_i(f)\) is an isomorphism in \(\Pers^{\mathrm{ft}}_k\).
\end{definition}

\begin{lemma}[Characterization of f.q.i.]
For bounded-in-degree filtered complexes of finite-dimensional vector spaces, a filtration-preserving chain map \(f:F\to G\) is an f.q.i. if and only if \(\mathbf{P}_i(f)\) is an isomorphism in \(\Pers^{\mathrm{ft}}_k\) for all \(i\).
\end{lemma}

\begin{proof}[Proof sketch]
If \(f\) is an f.q.i., then for each \(t\) the induced maps on homology are isomorphisms, hence \(\mathbf{P}_i(f)\) is pointwise an isomorphism and thus an isomorphism in \(\Pers^{\mathrm{ft}}_k\).
Conversely, if \(\mathbf{P}_i(f)\) is an isomorphism, then for each \(t\) the maps \(H_i(F^t)\to H_i(G^t)\) are isomorphisms for all \(i\); by boundedness and finite-dimensionality, \(f|_{F^t}\) is a quasi-isomorphism for every \(t\), hence \(f\) is an f.q.i.
\end{proof}

\begin{definition}[Homotopy category]
Let \(\Ho(\mathsf{FiltCh}(k))\) be the localization of \(\mathsf{FiltCh}(k)\) at f.q.i.’s.
Identities stated in \(\Ho(\mathsf{FiltCh}(k))\) are to be understood \emph{up to f.q.i.} at the model level.
All endofunctors considered below (e.g.\ \(C_\tau\) and Mirror/Transfer templates) preserve f.q.i.’s; thus they descend to \(\Ho(\mathsf{FiltCh}(k))\).
\end{definition}

% -------------------------
\subsection*{B.3. Lifting \texorpdfstring{$\mathbf{T}_\tau$}{T\_\tau} to \texorpdfstring{$C_\tau$}{C\_\tau} and (co)limit/pullback compatibilities}
% -------------------------

\paragraph*{Existence, functoriality, and uniqueness (homotopy-functor level).}
\begin{theorem}[Thresholded collapse in \(\Ho\)]\label{B:thm:Ctau}
For each \(\tau\ge 0\) there exists an endofunctor
\[
C_\tau:\ \Ho(\mathsf{FiltCh}(k))\longrightarrow \Ho(\mathsf{FiltCh}(k))
\]
and natural isomorphisms in \(\Pers^{\mathrm{ft}}_k\)
\[
\mathbf{P}_i\!\big(C_\tau(F)\big)\ \xrightarrow{\ \cong\ }\ \mathbf{T}_\tau\!\big(\mathbf{P}_i(F)\big)\qquad(\forall i,F),
\]
such that:
\begin{enumerate}\itemsep0.2em
  \item (\emph{Idempotence/monotonicity in \(\Ho\)})
  \(C_\tau\circ C_\sigma \simeq C_{\max\{\tau,\sigma\}}\simeq C_\sigma\circ C_\tau\).
  \item (\emph{Non-expansiveness at persistence})
  For all \(F,G\) and all \(i\),
  \[
  d_{\mathrm{int}}\!\big(\mathbf{P}_i(C_\tau F),\,\mathbf{P}_i(C_\tau G)\big)\ \le\ d_{\mathrm{int}}\!\big(\mathbf{P}_i(F),\,\mathbf{P}_i(G)\big).
  \]
\end{enumerate}
Moreover, any two lifts with these properties are uniquely isomorphic in \(\Ho(\mathsf{FiltCh}(k))\).
For \(\tau=0\), \(C_0\simeq \mathrm{id}\) in \(\Ho(\mathsf{FiltCh}(k))\).
\end{theorem}

\begin{proof}[Construction/Proof]
On objects: for each \(i\), replace \(\mathbf{P}_i(F)\) with \(\mathbf{T}_\tau(\mathbf{P}_i(F))\) (Appendix~A, Thm.~\ref{A:thm:localization}) and realize via \(\mathcal{U}\) (Proposition~\ref{B:prop:U}); \emph{assemble the total differential in a strictly block-diagonal way}, i.e.\ differentials are nonzero only inside the two-term blocks \((i+1)\to i\) representing finite bars (and the one-term blocks for infinite bars), while \emph{off-diagonal components between distinct bars or non-adjacent homological degrees are set to zero}. This preserves degreewise persistence exactly.
On morphisms \(f:F\to G\): apply \(\mathbf{T}_\tau\mathbf{P}_i(f)\) and lift blockwise via \(\mathcal{U}\) (pseudofunctoriality), taking direct sums over \(i\); this defines \(C_\tau(f)\) well-defined in \(\Ho\).
Idempotence/monotonicity reflect the corresponding properties of \(\mathbf{T}_\tau\) (a Serre exact reflector); non-expansiveness follows from the \(1\)-Lipschitz property of \(\mathbf{T}_\tau\) (Appendix~A, Prop.~\ref{A:prop:lipschitz}).
Uniqueness in \(\Ho\) follows from the uniqueness of \(\mathbf{T}_\tau\) at the persistence layer and of \(\mathcal{U}\) up to f.q.i.
\end{proof}

\paragraph*{(Co)limits and pullbacks: persistence layer is strict; filtered layer up to f.q.i.}
\begin{proposition}[Compatibility at the persistence layer]\label{B:prop:limits}
Assume filtered colimits in \(\mathsf{FiltCh}(k)\) are computed degreewise on chains/filtrations and the results return to \(\Pers^{\mathrm{ft}}_k\).
Then for every filtered diagram \(\{F_\lambda\}\) and every \(i\),
\[
\mathbf{P}_i\!\big(C_\tau(\varinjlim\nolimits_\lambda F_\lambda)\big)\ \cong\ \varinjlim\nolimits_\lambda\, \mathbf{P}_i\!\big(C_\tau(F_\lambda)\big)\quad\text{in } \Pers^{\mathrm{ft}}_k.
\]
If, in addition, \textup{[Spec]} finite pullbacks in \(\mathsf{FiltCh}(k)\) are computed degreewise and the interval realization \(\mathcal{U}\) preserves finite limits up to f.q.i.\ under the \emph{lifting–coherence} hypothesis \((\LC)\), then for any pullback square \(F\times_H G\),
\[
\mathbf{P}_i\!\big(C_\tau(F\times_H G)\big)\ \cong\ \mathbf{P}_i\!\big(C_\tau(F)\times_{C_\tau(H)} C_\tau(G)\big)\quad\text{in } \Pers^{\mathrm{ft}}_k.
\]
Similarly, finite pushouts behave dually at the persistence layer (under the same scope policy), since \(\mathbf{T}_\tau\) preserves finite colimits; \textbf{at the filtered-complex level all such compatibilities are asserted only \emph{up to f.q.i.}}.
\end{proposition}

\begin{proof}
Exactness (both left and right) and additivity of \(\mathbf{T}_\tau\) as a Serre localization (Appendix~A, Thm.~\ref{A:thm:localization}) yield preservation of finite limits/colimits; being a left adjoint, \(\mathbf{T}_\tau\) preserves (existing) colimits with the filtered-colimit policy of Appendix~A, Remark~\ref{A:rk:filtered-colimits}.
Applying \(\mathbf{P}_i\) gives the identities at the persistence layer.
At the filtered level, compatibilities hold \emph{up to f.q.i.} via realization by \(\mathcal{U}\) and the block-diagonal assembly in the construction of \(C_\tau\).
\end{proof}

\begin{remark}[On \((\LC)\)]
The hypothesis \((\LC)\) is a \emph{finite-diagram} coherence condition ensuring that interval realizations can be chosen compatibly (up to f.q.i.) with pullback/pushout shapes occurring in practice (e.g.\ fiber products along filtration-preserving maps).
It holds for the elementary block model above and finite matching diagrams where barcode maps are induced by monotone filtrations; we use it only in \textup{[Spec]} statements.
\end{remark}

\begin{remark}[Realization functor; comparison maps \textup{\lbrack}Spec\textup{\rbrack}]
Let \(\mathcal{R}:\mathsf{FiltCh}(k)\to D^{\mathrm{b}}(k\text{-mod})\) be the fixed \(t\)-exact realization from the main text.
Within the implementable range there are natural comparison morphisms
\[
\mathcal{R}\circ C_\tau\ \Longrightarrow\ \tau_{\ge 0}\circ \mathcal{R},
\]
compatible with \(\mathbf{P}_i\) after homology (Appendix~C).
Here \(\tau_{\ge 0}\) denotes truncation for the fixed \(t\)-structure.
These comparison maps are treated up to f.q.i.\ in \(\Ho(\mathsf{FiltCh}(k))\).
\end{remark}

% -------------------------
\subsection*{B.4. Non-expansive Mirror/Transfer templates \texorpdfstring{[Spec]}{[Spec]}}
% -------------------------
\begin{specification}[Mirror/Transfer endofunctors]\label{B:spec:mirror}
An endofunctor \(\Mirror:\mathsf{FiltCh}(k)\to \mathsf{FiltCh}(k)\) is \emph{admissible} if:
\begin{enumerate}\itemsep0.2em
  \item (\emph{Persistence non-expansiveness})
  For all \(F,G\) and every \(i\),
  \[
  d_{\mathrm{int}}\!\big(\mathbf{P}_i(\Mirror F),\,\mathbf{P}_i(\Mirror G)\big)\ \le\ d_{\mathrm{int}}\!\big(\mathbf{P}_i(F),\,\mathbf{P}_i(G)\big).
  \]
  \item (\emph{Constructible stability})
  \(\Mirror\) carries finite-type objects to finite-type objects degreewise.
  \item (\emph{f.q.i.-invariance})
  If \(f\) is an f.q.i., then \(\Mirror(f)\) is an f.q.i.; hence \(\Mirror\) descends to \(\Ho(\mathsf{FiltCh}(k))\).
  \item (\emph{Conditional commutation with \(C_\tau\)})
  There exists a natural 2-cell
  \[
  \theta:\ \Mirror\circ C_\tau\ \Rightarrow\ C_\tau\circ \Mirror
  \]
  whose effect at persistence is \(\delta\)-controlled:
  \[
  d_{\mathrm{int}}\!\Big(\mathbf{T}_\tau\,\mathbf{P}_i\big(\Mirror(C_\tau F)\big),\ \mathbf{T}_\tau\,\mathbf{P}_i\big(C_\tau(\Mirror F)\big)\Big)\ \le\ \delta(i,\tau)\quad(\forall i,F).
  \]
  The bound \(\delta(i,\tau)\) is \emph{uniform in \(F\)} and can be chosen to be \emph{additive} along pipelines (Appendix~L) and \emph{non-increasing under} any subsequent \(1\)\hyp Lipschitz persistence post\hyp processing (e.g.\ shifts, further truncations).
  In particular, if \(\theta\) induces isomorphisms after applying \(\mathbf{P}_i\), then one may take \(\delta(i,\tau)=0\).
\end{enumerate}
Under these assumptions, windowed indicators computed on \(C_\tau F\) (e.g.\ truncated energies) are stable (non-expansive) along \(\Mirror\).
Quantitative control of \(\delta\) follows Appendix~L (uniformity in \(F\), pipeline additivity, and non-increase under \(1\)\hyp Lipschitz post\hyp processing).
\end{specification}

\begin{remark}
Specification~\ref{B:spec:mirror}(4) is the only place where a quantitative commutation is used (cf.\ Appendix~L).
Without such a natural 2-cell and bound, no uniform control between \(\Mirror\circ C_\tau\) and \(C_\tau\circ \Mirror\) is asserted.
\end{remark}

% -------------------------
\subsection*{B.5. Quantitative commutation: uniformity, additivity, and post-processing stability}
% -------------------------
\begin{theorem}[Quantitative commutation in \(\Ho\) via \(\mathbf{P}_i\) and \(\mathbf{T}_\tau\)]\label{B:thm:quant}
Assume \textup{(i)} \(\Mirror\) is admissible (Specification~\ref{B:spec:mirror}) and \textup{(ii)} a natural 2-cell \(\theta:\Mirror\circ C_\tau \Rightarrow C_\tau\circ \Mirror\) with bound \(\delta(i,\tau)\ge 0\), uniform in \(F\).
Then, for all \(F\), degrees \(i\), and scales \(\tau\),
\[
  d_{\mathrm{int}}\!\Big(\mathbf{T}_\tau\,\mathbf{P}_i(\Mirror(C_\tau F)),\ \mathbf{T}_\tau\,\mathbf{P}_i(C_\tau(\Mirror F))\Big)\ \le\ \delta(i,\tau).
\]
Moreover, for a pipeline \(\Mirror_m,\dots,\Mirror_1\) of admissible endofunctors with 2-cell bounds \(\delta_j(i,\tau_j)\), one has the additive estimate
\[
  d_{\mathrm{int}}\!\Big(\mathbf{T}_\tau\,\mathbf{P}_i\big(\Mirror_m\cdots\Mirror_1(C_{\tau_m}\cdots C_{\tau_1}F)\big),\ \mathbf{T}_\tau\,\mathbf{P}_i\big(C_{\tau_m}\cdots C_{\tau_1}(\Mirror_m\cdots\Mirror_1 F)\big)\Big)\ \le\ \sum_{j=1}^m \delta_j(i,\tau_j),
\]
and any further \(1\)\hyp Lipschitz persistence post\hyp processing (e.g.\ shifts \(S^\varepsilon\), truncations \(\mathbf{T}_{\tau'}\), degree projections) does not increase the bound.
\end{theorem}

\begin{proof}[Proof sketch]
Apply the 2-cell bound at the persistence layer, then use the \(1\)\hyp Lipschitz property of \(\mathbf{T}_\tau\) (Appendix~A, Prop.~\ref{A:prop:lipschitz}). For the pipeline, compose the 2-cells and sum the bounds; post\hyp processing stability follows from non\hyp expansiveness of the applied persistence functors.
\end{proof}

% -------------------------
\subsection*{B.6. Commutable torsion reflectors and A/B policy (homotopy interface)}
% -------------------------
Let \(T_A,T_B:\Pers^{\mathrm{ft}}_k\to\Pers^{\mathrm{ft}}_k\) be exact reflectors (e.g.\ truncations by distinct torsion classes). Write \(E_A,E_B\) for the underlying Serre subcategories.

\begin{proposition}[Nested torsions \(\Rightarrow\) order independence]\label{B:prop:nested}
If \(E_A\subseteq E_B\) or \(E_B\subseteq E_A\), then
\[
T_A\circ T_B\ =\ T_B\circ T_A\ =\ T_{A\vee B},
\]
where \(E_{A\vee B}\) is the Serre subcategory generated by \(E_A\cup E_B\).
In particular, for 1D length thresholds, \(\mathbf{T}_\tau\circ \mathbf{T}_\sigma=\mathbf{T}_{\max\{\tau,\sigma\}}\).
\end{proposition}

\begin{proof}[Proof sketch]
Nested Serre subcategories yield idempotent, order\hyp independent localizations by the universal property of reflectors in abelian categories.
\end{proof}

\begin{definition}[A/B commutativity test and soft\hyp commuting policy]\label{B:def:ab}
For arbitrary reflectors \(T_A,T_B\) and \(M\in\Pers^{\mathrm{ft}}_k\), set
\[
\Delta_{\mathrm{comm}}(M;A,B):=d_{\mathrm{int}}(T_A T_B M,\ T_B T_A M).
\]
Given a tolerance \(\eta\ge 0\), we \emph{accept soft\hyp commuting} on a dataset if \(\Delta_{\mathrm{comm}}(M;A,B)\le \eta\) holds for the relevant instances \(M\). Otherwise, we \emph{fallback} to a fixed order (say \(T_B\circ T_A\)) and record \(\Delta_{\mathrm{comm}}\) into \(\delta^{\mathrm{alg}}\) in the \(\delta\)\hyp ledger (Appendix~L).
\end{definition}

\begin{remark}[Integration with the homotopy setting]
At the filtered\hyp complex level, apply \(C_\tau\) (up to f.q.i.) and then the persistence reflectors \(T_A,T_B\) on \(\mathbf{P}_i\). The A/B outcome is logged at the persistence layer; any induced algorithmic mismatch contributes to \(\delta^{\mathrm{alg}}(i,\tau)\), uniform in \(F\) by design (Appendix~L). When nestedness holds (Proposition~\ref{B:prop:nested}), the A/B test is not required.
\end{remark}

% -------------------------
\subsection*{B.7. Completion note}
% -------------------------
\begin{remark}[No further supplementation required]
This appendix fully integrates: (i) the lift \(C_\tau\) of the exact reflector \(\mathbf{T}_\tau\) to the homotopy setting (existence, functoriality, uniqueness up to f.q.i.; non\hyp expansiveness at persistence); (ii) strict persistence\hyp layer compatibilities with (co)limits and pullbacks (filtered\hyp complex level up to f.q.i.); (iii) admissible Mirror/Transfer templates with a quantitative 2\hyp cell bound \(\delta(i,\tau)\) that is \emph{uniform in \(F\)}, \emph{additive} along pipelines, and \emph{non\hyp increasing} under \(1\)\hyp Lipschitz post\hyp processing; and (iv) a commutable torsion policy: nested torsions imply order independence, otherwise an A/B test with soft\hyp commuting and a deterministic fallback, with differences logged into \(\delta^{\mathrm{alg}}\).
All statements are confined to the v15.0 guard\hyp rails (constructible \(1\)D persistence over a field; B\hyp side single layer after collapse; f.q.i.\ on filtered complexes), and no further supplementation is required for operational use in this framework.
\end{remark}



% =========================
\appendix
\section*{Appendix C. The Bridge \texorpdfstring{$\mathrm{PH}_1 \Rightarrow \Ext^1$}{PH1⇒Ext1} [Proof] (reinforced)}
% =========================

\addcontentsline{toc}{section}{Appendix C. The Bridge PH$_1\Rightarrow$Ext$^1$}

Throughout, fix a field \(k\).
Let \(\mathsf{FiltCh}(k)\) be the category of \emph{bounded-in-degree} filtered chain complexes of finite-dimensional \(k\)-vector spaces with filtration-preserving maps.
For \(F\in\mathsf{FiltCh}(k)\) and each degree \(i\), the degreewise persistence functor
\[
\mathbf{P}_i(F):\ \mathbb{R}\longrightarrow \mathsf{Vect}_k,\qquad\nt\longmapsto H_i(F^{t}C_\bullet)
\]
is assumed \emph{constructible} (pointwise finite-dimensional, with finitely many critical parameters on bounded windows), i.e.\ \(\mathbf{P}_i(F)\in \Pers^{\mathrm{ft}}_k\).
We also fix a \(t\)-exact realization functor
\[
\mathcal{R}:\ \mathsf{FiltCh}(k)\longrightarrow D^{\mathrm{b}}(k\text{-mod})
\]
into the bounded derived category of finite-dimensional \(k\)-vector spaces.

\medskip
\noindent\textbf{Bridge hypotheses, scope, and gate policy.}
The one-way bridge \(\mathrm{PH}_1\Rightarrow \Ext^1\) in Part~I uses assumptions \textup{(B1)–(B3)}:
\emph{(B1)} field coefficients and constructibility; \emph{(B2)} two-term amplitude (edge identification) for \(\mathcal{R}\); \emph{(B3)} functoriality/naturality.
All statements in this appendix operate under \textup{(B1)–(B3)} and within the implementable range of Appendix~A.
Filtered colimits, when used, are computed in the functor category \([\mathbb{R},\mathsf{Vect}_k]\) and must \emph{return} to \(\Pers^{\mathrm{ft}}_k\) once constructibility is verified (Appendix~A, Remark~\ref{A:rk:filtered-colimits}).
\textbf{Eligibility for B-Gate\(^{+}\):} the \(\Ext^1\)-test is included in B-Gate\(^{+}\) \emph{only} on windows/scales where \(\mathcal{R}(C_\tau F)\in D^{[-1,0]}(k\text{-mod})\) (amplitude \(\le 1\)); outside this amplitude regime \(\Ext^1\) may be logged but is \emph{not} used for gating.
\textbf{No reverse claim:} we never assert \(\Ext^1=0\Rightarrow \mathrm{PH}_1=0\).

\medskip
\noindent\textbf{Minimal test family.}
We test \(\Ext\)-vanishing only against the unit \(k[0]\):
\[
\text{``\(\Ext^1\)-collapse''}\quad\Longleftrightarrow\quad\n\Ext^1\!\big(\mathcal{R}(\bullet),\,k\big)=0.
\]
For objects \(A\in D^{[-1,0]}(k\text{-mod})\), Lemma~\ref{C:lem:edge-ext} gives
\(\Ext^1(A,k)\cong \Hom(H^{-1}(A),k)\); hence testing with \(k[0]\) is sufficient.

In the notation of the main text, \(\mathrm{PH}_1(F)=0\) means that the degree-\(1\) persistence module vanishes (equivalently, \(H_1(F^t)=0\) for all \(t\in\mathbb{R}\); see Chapter~2).

% -------------------------
\subsection*{C.1. Two-term amplitude and $t$-exactness}
% -------------------------

\begin{proposition}[Two-term amplitude]\label{C:prop:two-term}
Under \textup{(B2)} there is a natural isomorphism in \(D^{\mathrm{b}}(k\text{-mod})\)
\[
\mathcal{R}(F)\ \simeq\ \Big[\, H^{-1}\!\big(\mathcal{R}(F)\big)\ \xrightarrow{\,d\,}\ H^{0}\!\big(\mathcal{R}(F)\big)\,\Big],
\]
concentrated in cohomological degrees \([-1,0]\).
Equivalently, \(\mathcal{R}(F)\in D^{[-1,0]}(k\text{-mod})\).
\end{proposition}

\begin{proof}[Proof sketch]
By design and \textup{(B2)}, \(\mathcal{R}\) is \(t\)-exact for the standard \(t\)-structure and is built so that only \(H^{-1}\) and \(H^0\) can be nonzero.
Concretely, one may take a mapping-cone type model in which the persistence in degrees \(1\) and \(0\) is encoded by a two-term complex concentrated in cohomological degrees \(-1\) and \(0\), with \(H^{-1}\) corresponding to the stabilized degree-\(1\) layer and \(H^0\) to the degree-\(0\) layer.
Thus \(\tau_{\le 0}\mathcal{R}(F)=\mathcal{R}(F)=\tau_{\ge -1}\mathcal{R}(F)\), yielding amplitude \([-1,0]\).
\end{proof}

\begin{remark}
All statements below are invariant under filtered quasi-isomorphism on \(F\) and under isomorphism in \(D^{\mathrm{b}}(k\text{-mod})\) on \(\mathcal{R}(F)\).
\end{remark}

% -------------------------
\subsection*{C.2. The edge: \texorpdfstring{$H^{-1}(\mathcal{R}(F))\cong \varinjlim_t H_1(F^t)$}{H^{-1}(R(F)) ≅ colim H_1(F^t)} and naturality}
% -------------------------

\begin{proposition}[Edge identification and naturality]\label{C:prop:edge}
Under \textup{(B2)}, for every \(F\in \mathsf{FiltCh}(k)\) there is a natural isomorphism
\[
H^{-1}\!\big(\mathcal{R}(F)\big)\ \cong\ \varinjlim_{t\in\mathbb{R}}\, H_1(F^{t}C_\bullet).
\]
If \(f:F\to G\) is a filtration-preserving chain map, then the diagram
\[
\begin{tikzcd}[row sep=1.2em, column sep=2.1em]\nH^{-1}\!\big(\mathcal{R}(F)\big) \arrow[r, "\sim"] \arrow[d, "H^{-1}(\mathcal{R}(f))"'] &\n\varinjlim_{t}H_1(F^tC_\bullet) \arrow[d, "\varinjlim_{t} H_1(f^t)"] \\\nH^{-1}\!\big(\mathcal{R}(G)\big) \arrow[r, "\sim"] &\n\varinjlim_{t}H_1(G^tC_\bullet)\n\end{tikzcd}
\]
commutes.
\end{proposition}

\begin{proof}[Proof sketch]
Using the two-term model from Proposition~\ref{C:prop:two-term} and the construction of \(\mathcal{R}\), the degree \(-1\) cohomology identifies with the filtered colimit of \(H_1(F^t)\).
Exactness of filtered colimits in \(\mathsf{Vect}_k\), computed objectwise in \([\mathbb{R},\mathsf{Vect}_k]\) (Appendix~A, Remark~\ref{A:rk:filtered-colimits}), implies that filtered colimits commute with homology in this setting; hence the edge isomorphism.
Naturality in \(f\) follows from functoriality of homology and the universal property of colimits.
\end{proof}

% -------------------------
\subsection*{C.3. Computing \texorpdfstring{$\Ext^1$}{Ext1} for amplitude \([-1,0]\)}
% -------------------------

\begin{lemma}[Edge lemma for \(\Ext^1\)]\label{C:lem:edge-ext}
Let \(A\in D^{[-1,0]}(k\text{-mod})\).
Then there is a natural isomorphism
\[
\Ext^1(A,k)\ \cong\ \Hom\!\big(H^{-1}(A),\,k\big).
\]
\end{lemma}

\begin{proof}
Compute \(\Ext^1(A,k)=\Hom_{D^{\mathrm{b}}}(A,k[1])\) via the truncation triangle
\(\tau_{\le -1}A\to A\to \tau_{\ge 0}A\to (\tau_{\le -1}A)[1]\).
Since \(A\in D^{[-1,0]}\), \(\tau_{\le -1}A\simeq H^{-1}(A)[1]\) and \(\tau_{\ge 0}A\simeq H^{0}(A)[0]\).
Applying \(\Hom(-,k[1])\) and using that \(k\text{-mod}\) (finite-dimensional vector spaces over a field) is semisimple, we have \(\Ext^1(H^{0}(A),k)=0\).
Therefore
\[
\Hom(A,k[1])\ \cong\ \Hom\!\big(H^{-1}(A)[1],\,k[1]\big)\ \cong\ \Hom\!\big(H^{-1}(A),k\big),
\]
naturally in \(A\).
\end{proof}

\begin{theorem}[Bridge \(\mathrm{PH}_1\Rightarrow \Ext^1\)]\label{C:thm:bridge}
Let \(F\in \mathsf{FiltCh}(k)\).
If \(\mathrm{PH}_1(F)=0\) (equivalently, \(H_1(F^t)=0\) for all \(t\)), then
\[
\Ext^1\!\big(\mathcal{R}(F),\,k\big)\ =\ 0.
\]
\end{theorem}

\begin{proof}
By \(\mathrm{PH}_1(F)=0\) we have \(\varinjlim_t H_1(F^t)=0\).
Proposition~\ref{C:prop:edge} gives \(H^{-1}(\mathcal{R}(F))=0\).
Apply Lemma~\ref{C:lem:edge-ext} with \(A=\mathcal{R}(F)\in D^{[-1,0]}\) (Proposition~\ref{C:prop:two-term}) to obtain
\(\Ext^1(\mathcal{R}(F),k)\cong \Hom(H^{-1}(\mathcal{R}(F)),k)=0\).
\end{proof}

\begin{corollary}[Robust windowed version]
For any \(\varepsilon>0\), if \(\mathrm{PH}_1\!\big(C_\varepsilon(F)\big)=0\), then
\(\Ext^1\!\big(\mathcal{R}(C_\varepsilon(F)),\,k\big)=0\).
\end{corollary}

\begin{proof}
Apply Theorem~\ref{C:thm:bridge} to \(C_\varepsilon(F)\).
By Appendix~B, \(C_\varepsilon\) preserves the constructible range and \(\mathcal{R}(C_\varepsilon(F))\in D^{[-1,0]}\), so \textup{(B2)} applies verbatim.
\end{proof}

\begin{remark}[Application order for the robust gate]
In the robust/windowed regime we always apply \emph{persistence truncation first}, then realization, then the \(\Ext^1\)-test:
\[
F\ \xrightarrow{\ C_\varepsilon\ }\ C_\varepsilon(F)\ \xrightarrow{\ \mathcal{R}\ }\ \mathcal{R}(C_\varepsilon(F))\ \xrightarrow{\ \Ext^1(-,k)\ }\ 0.
\]
This order ensures both metric stability at the persistence layer and applicability of the two-term amplitude hypothesis for \(\mathcal{R}\).
\end{remark}

% -------------------------
\subsection*{C.4. Eligibility and B-Gate\texorpdfstring{$^{+}$}{+} policy}
% -------------------------

\begin{declaration}[Eligibility for using \(\Ext^1\) in B-Gate\(^{+}\)]\label{C:decl:eligibility}
Fix a collapse threshold \(\tau\ge 0\) and a window \([0,\tau]\).
The \(\Ext^1\)-test \(\Ext^1(\mathcal{R}(C_\tau F),k)=0\) is \emph{admitted into} B-Gate\(^{+}\) if and only if:
\begin{enumerate}\itemsep0.2em
  \item \(\mathcal{R}\) is \(t\)-exact and \(\mathcal{R}(C_\tau F)\in D^{[-1,0]}(k\text{-mod})\) (amplitude \(\le 1\));
  \item the scope rule of Appendix~A (constructibility and return from functor-category colimits) holds; and
  \item the test object is \(Q=k[0]\) (no other targets are used in gating).
\end{enumerate}
If any condition fails, the \(\Ext^1\)-value may be logged for diagnostics but is \emph{excluded} from the gate decision on that window.
\end{declaration}

\begin{remark}[Diagnostics outside the eligible regime]
Outside amplitude \(\le 1\), record a flag \texttt{ext1\_eligible:false} and a reason (e.g.\ \texttt{amplitude\_gt\_1}).
B-Gate\(^{+}\) then relies on the persistence layer (\(\mathrm{PH}_1=0\)), tower isomorphisms \((\mu,\nu)=(0,0)\), and the safety margin \(\mathrm{gap}_\tau>\Sigma\delta\) only (cf.\ Chapters~10–12).
\end{remark}

% -------------------------
\subsection*{C.5. Naturality and stability under admissible updates}
% -------------------------

\begin{proposition}[Naturality of the bridge]
For any filtration-preserving \(f:F\to G\) in \(\mathsf{FiltCh}(k)\), the diagram
\[
\begin{tikzcd}[row sep=1.2em, column sep=2.4em]\n\Ext^1\!\big(\mathcal{R}(F),k\big)\n  \arrow[r, "\sim"]\n  \arrow[d, "{\Ext^1(\mathcal{R}(f),\,k)}"'] &\n\Hom\!\big(H^{-1}(\mathcal{R}(F)),k\big)\n  \arrow[d, "{\Hom(H^{-1}(\mathcal{R}(f)),\,k)}"]\\\n\Ext^1\!\big(\mathcal{R}(G),k\big)\n  \arrow[r, "\sim"] &\n\Hom\!\big(H^{-1}(\mathcal{R}(G)),k\big)\n\end{tikzcd}
\]
commutes.
\end{proposition}

\begin{proof}
This follows from the functoriality of the truncation functors \(\tau_{\le -1}\), \(\tau_{\ge 0}\), the naturality in Lemma~\ref{C:lem:edge-ext}, and the edge identification in Proposition~\ref{C:prop:edge}.
\end{proof}

\begin{remark}[Stability under admissible updates]
When \(F\mapsto F'\) is an update that is \emph{non-expansive} at the persistence layer (e.g.\ deletion-type or \(\varepsilon\)-continuation post-collapse), and \(\mathcal{R}(C_\tau F),\mathcal{R}(C_\tau F')\in D^{[-1,0]}\), then the verdict \(\Ext^1(\mathcal{R}(C_\tau F),k)=0\) implies \(\Ext^1(\mathcal{R}(C_\tau F'),k)=0\) provided the edge groups remain isomorphic (or vanish) along the filtered colimit, by Proposition~\ref{C:prop:edge} and Lemma~\ref{C:lem:edge-ext}.
This stability is used only as a diagnostic convenience; gating always defers to eligibility (Declaration~\ref{C:decl:eligibility}).
\end{remark}

% -------------------------
\subsection*{C.6. Implementation and logging}
% -------------------------

\begin{remark}[Run-time policy and manifest fields]
When B-Gate\(^{+}\) is executed, enforce the order
\(\,F\to C_\tau F\to \mathcal{R}(C_\tau F)\to \Ext^1(-,k)\,\).
The manifest \texttt{run.yaml} must include:
\begin{itemize}
  \item \texttt{ext1\_eligible}: boolean; \texttt{true} iff \(\mathcal{R}(C_\tau F)\in D^{[-1,0]}\);
  \item \texttt{ext1\_used\_in\_gate}: boolean; \texttt{true} iff \texttt{ext1\_eligible} and the gate policy enables it;
  \item \texttt{amplitude}: reported as \([-1,0]\) or \texttt{>1} (if known by static design or a certified check);
  \item \texttt{gate\_order}: fixed as \texttt{collapse→realize→ext1};
  \item \texttt{q\_test}: fixed as \texttt{k[0]}.
\end{itemize}
Outside eligibility, set \texttt{ext1\_eligible:false}, \texttt{ext1\_used\_in\_gate:false}, and continue the gate using persistence/tower/safety criteria only.
\end{remark}

% -------------------------
\subsection*{C.7. Scope and non-claims}
% -------------------------

\begin{remark}[Scope and non-claims]
\textbf{The bridge \(\mathrm{PH}_1\Rightarrow \Ext^1\) is used and proved only in \(D^{\mathrm{b}}(k\text{-mod})\).}
The converse implication \(\Ext^1(\mathcal{R}(F),k)=0 \Rightarrow \mathrm{PH}_1(F)=0\) is \emph{false} in general (see Appendix~D, §D.4 for counterexamples).
All filtered colimit uses obey the scope rule stated at the start of this appendix (Appendix~A, Remark~\ref{A:rk:filtered-colimits}).
Derived realizations into other targets (e.g.\ coherent/étale) appear only as \textbf{[Spec]} in the main text and \emph{do not} extend the proved bridge.
\end{remark}

\medskip
\noindent\emph{Summary of Appendix C (reinforced).}
Under \((\mathrm{B}1)\)–\((\mathrm{B}3)\), the edge identification \(H^{-1}(\mathcal{R}(F))\cong \varinjlim_t H_1(F^t)\) and the amplitude \([-1,0]\) model yield the bridge \(\mathrm{PH}_1(F)=0\Rightarrow \Ext^1(\mathcal{R}(F),k)=0\).
The \(\Ext^1\)-test is included in B-Gate\(^{+}\) \emph{only} when \(\mathcal{R}(C_\tau F)\in D^{[-1,0]}\); otherwise it is excluded from the gate but may be logged.
No reverse implication is claimed.
All computations respect the collapse-first policy and the filtered-colimit scope of Appendix~A.



% =========================
\section*{Appendix D. Towers, \texorpdfstring{$\mu,\nu$}{mu,nu}, and Examples [Proof/Example] (reinforced)}
% =========================

\addcontentsline{toc}{section}{Appendix D. Towers, $\mu,\nu$, and Examples}
\refstepcounter{section} % ensure stable cross-references with unnumbered section

Throughout, fix a field \(k\).
We work in the constructible range (Appendix~A): for each degree \(i\),
\(\mathbf{P}_i:\mathsf{FiltCh}(k)\to\Pers^{\mathrm{ft}}_k\) sends a \emph{bounded-in-degree} filtered chain complex \(F\) to the persistence module
\(\big(t\mapsto H_i(F^{t}C_\bullet)\big)\).
The truncation \(\mathbf{T}_\tau:\Pers^{\mathrm{ft}}_k\to\Pers^{\mathrm{ft}}_{k,\tau\text{-loc}}\) is exact and \(1\)-Lipschitz (Appendix~A), and filtered colimits are computed objectwise in \([\mathbb{R},\mathsf{Vect}_k]\) with the scope rule of Appendix~A, Remark~\ref{A:rk:filtered-colimits}.
All statements at the filtered–complex layer are \emph{up to filtered quasi-isomorphism (f.q.i.)}; persistence-layer statements are strict in \(\Pers^{\mathrm{ft}}_k\).
All quantities below depend on the threshold \(\tau\); \textbf{no general monotonicity in \(\tau\) is asserted.}

\medskip
\noindent\textbf{Comparison map and obstruction indices.}
Let \(\{F_n\}_{n\in\mathbb{N}}\) be a directed system in \(\mathsf{FiltCh}(k)\).
Let \(F_\infty\) be an apex object equipped with a cocone \(F_n\to F_\infty\) (indexing category \(\mathbb{N}\cup\{\infty\}\) with unique morphisms \(n\to\infty\)).
For each \(i\) and threshold \(\tau>0\) set
\[
\phi_{i,\tau}:\ \varinjlim_{n\in\mathbb{N}}\, \mathbf{T}_\tau\!\big(\mathbf{P}_i(F_n)\big)\ \longrightarrow\ \mathbf{T}_\tau\!\big(\mathbf{P}_i(F_\infty)\big),
\]
the canonical comparison in \([\mathbb{R},\mathsf{Vect}_k]\).
Define the \emph{tower obstruction indices}
\[
\mu_{i,\tau}:=\dim_k\ker(\phi_{i,\tau}),\qquad\n\nu_{i,\tau}:=\dim_k\mathrm{coker}(\phi_{i,\tau}),\qquad\n\muc:=\sum_i\mu_{i,\tau},\ \ \nuc:=\sum_i\nu_{i,\tau}.
\]
These sums are finite because complexes are bounded in homological degrees (constructible range).
The pair \((\muc,\nuc)\) detects \emph{Type~IV} (tower-level) failures at scale~\(\tau\).\footnote{For composition of morphisms in finite-dimensional linear algebra one has subadditivity surrogates
\(\dim\ker(g\circ f)\le \dim\ker f+\dim\ker g\) and
\(\dim\operatorname{coker}(g\circ f)\le \dim\operatorname{coker} g+\dim\ker f\).
The calculus for \((\mu,\nu)\) in Appendix~J is the persistence-layer analogue after applying \(\mathbf{T}_\tau\) and taking generic-fiber dimensions.}

\begin{remark}[Generic dimension after truncation]\label{rem:D-generic-dim}
In \(\Pers^{\mathrm{ft}}_k\), after applying \(\mathbf{T}_\tau\) the kernel and cokernel of any morphism decompose (noncanonically) as finite direct sums of interval modules.
We write \(\dim_k(-)\) for the \emph{generic fiber} dimension, i.e.\ the multiplicity of the infinite bar \(I[0,\infty)\) in that decomposition.
Finite bars contribute zero generic fiber.
\end{remark}

\begin{remark}[Invariance of \((\muc,\nuc)\)]
The indices \((\muc,\nuc)\) are invariant under levelwise f.q.i.\ replacements of the tower and apex: if \(F_n\simeq_{\mathrm{f.q.i.}}F'_n\) and \(F_\infty\simeq_{\mathrm{f.q.i.}}F'_\infty\), then \(\mathbf{P}_i\) sends these to isomorphisms in \(\Pers^{\mathrm{ft}}_k\), hence \(\ker/\mathrm{coker}\) (and thus \(\mu,\nu\)) are unchanged.
They are also invariant under cofinal reindexing of the tower, since filtered colimits over cofinal subdiagrams are canonically isomorphic.
\end{remark}

\begin{figure}[t]
\centering
\begin{tikzcd}[row sep=1.0em, column sep=2.0em]
F_1 \arrow[r] \arrow[dr] &
F_2 \arrow[r] \arrow[dr] &
\cdots \arrow[r] &
F_n \arrow[r] \arrow[dr] &
\cdots \arrow[r] &
F_\infty \\\n& \mathbf{P}_i(F_1) \arrow[r] &
\mathbf{P}_i(F_2) \arrow[r] &
\cdots \arrow[r] &
\mathbf{P}_i(F_n) \arrow[r] &
\mathbf{P}_i(F_\infty)
\end{tikzcd}
\caption{A tower with apex \(F_\infty\) and its image under \(\mathbf{P}_i\).
The comparison \(\phi_{i,\tau}\) (defined after applying \(\mathbf{T}_\tau\)) measures the failure of the cocone to exhibit a colimit at scale \(\tau\).}
\end{figure}

% -------------------------
\subsection*{D.1. Toy towers: pure kernel / pure cokernel / mixed}
% -------------------------

\begin{example}[Pure cokernel at a fixed scale]\label{D:ex:pure-coker}
Fix \(\tau>0\) and degree \(i=1\).
Let \(\mathbf{P}_1(F_n)=I[0,\tau-\frac1n)\) with transition maps the evident inclusions \(I[0,\tau-\frac1n)\hookrightarrow I[0,\tau-\frac1{n+1})\).
Let \(F_\infty\) satisfy \(\mathbf{P}_1(F_\infty)=I[0,\infty)\).
Then \(\mathbf{T}_\tau(\mathbf{P}_1(F_n))=0\) for all \(n\), whereas \(\mathbf{T}_\tau(\mathbf{P}_1(F_\infty))\cong I[0,\infty)\).
Hence \(\phi_{1,\tau}:0\to I[0,\infty)\) has trivial kernel and nontrivial cokernel, so \(\mu_{1,\tau}=0\) and \(\nu_{1,\tau}=1\) (pure cokernel).
\end{example}

\begin{example}[Pure kernel at a fixed scale]\label{D:ex:pure-ker}
Fix \(\tau>0\).
Let \(\mathbf{P}_1(F_n)=I[0,\infty)\) for all \(n\), with transition maps the identities (a stationary directed system).
Let \(F_\infty\) satisfy \(\mathbf{P}_1(F_\infty)=0\), and take the cocone \(\mathbf{P}_1(F_n)\to \mathbf{P}_1(F_\infty)\) to be \(0\) for all \(n\).
Then \(\mathbf{T}_\tau(\mathbf{P}_1(F_n))\cong I[0,\infty)\) for all \(n\), so the source of \(\phi_{1,\tau}\) is \(I[0,\infty)\), while the target is \(0\).
Thus \(\phi_{1,\tau}: I[0,\infty)\to 0\) has nontrivial kernel and zero cokernel, hence \(\mu_{1,\tau}=1\), \(\nu_{1,\tau}=0\) (pure kernel).
\end{example}

\begin{example}[Mixed]\label{D:ex:mixed}
Fix \(\tau>0\) and set
\[
\mathbf{P}_1(F_n)\ =\ I[0,\tau-\tfrac1n)\ \oplus\ I[0,\infty),
\]
with transition maps the obvious inclusions on the first summand and the identities on the second.
Take \(F_\infty\) with \(\mathbf{P}_1(F_\infty)=I[0,\infty)\oplus 0\), using cocone maps that send the second summand to \(0\).
Then the first summand yields a cokernel contribution exactly as in Example~\ref{D:ex:pure-coker}, while the second yields a kernel contribution exactly as in Example~\ref{D:ex:pure-ker}.
Hence \(\mu_{1,\tau}=\nu_{1,\tau}=1\) (mixed).
\end{example}

All three persistence-level towers are realizable by filtered complexes via the interval-realization assignment \(\mathcal{U}\) (Appendix~B), up to f.q.i.; constructibility is preserved.

% -------------------------
\subsection*{D.2. When \texorpdfstring{$\phi_{i,\tau}$}{phi} is an isomorphism: \texorpdfstring{$(\muc,\nuc)=(0,0)$}{(mu,nu)=(0,0)}}
% -------------------------

\begin{proposition}[Isomorphism criterion]\label{D:prop:iso-zero}
Assume:
\begin{enumerate}
\item[(i)] degreewise filtered colimits in \(\mathsf{FiltCh}(k)\) are computed objectwise on chains and filtrations;
\item[(ii)] each \(\mathbf{P}_i(F_n)\) lies in \(\Pers^{\mathrm{ft}}_k\);
\item[(iii)] \(\mathbf{T}_\tau\) commutes with the filtered colimit of \(\{\mathbf{P}_i(F_n)\}\) in \([\mathbb{R},\mathsf{Vect}_k]\), and the result is constructible \emph{(Appendix~A, Thm.~\ref{A:thm:localization})}; and
\item[(iv)] the cocone exhibits a colimit at persistence level: the canonical map
\(\varinjlim_{n}\mathbf{P}_i(F_n)\xrightarrow{\ \cong\ }\mathbf{P}_i(F_\infty)\)
is an isomorphism in \([\mathbb{R},\mathsf{Vect}_k]\).
\end{enumerate}
Then for every \(i\) and \(\tau>0\), the comparison map \(\phi_{i,\tau}\) is an isomorphism in \(\Pers^{\mathrm{ft}}_k\).
Consequently \((\muc,\nuc)=(0,0)\).
\end{proposition}

\begin{remark}
Condition (iv) is automatic if \(F_\infty\) \emph{is} the colimit of \(\{F_n\}\) in a model of filtered complexes for which \(\mathbf{P}_i\) is computed objectwise and the scope rule of Appendix~A applies; no claim is made beyond that regime.
\end{remark}

% -------------------------
\subsection*{D.3. Sufficient conditions ensuring \texorpdfstring{$(\muc,\nuc)=(0,0)$}{(mu,nu)=(0,0)}}
% -------------------------

The summability condition \(\sum_{n} d_{\mathrm{int}}\!\big(\mathbf{P}_i(F_{n+1}),\mathbf{P}_i(F_n)\big)<\infty\) \emph{alone} does not guarantee \((\muc,\nuc)=(0,0)\); see §D.3.1.
The following hypotheses are sufficient.

\begin{theorem}\label{D:thm:sufficient}
Fix \(i\) and \(\tau>0\).
Each of the following implies that \(\phi_{i,\tau}\) is an isomorphism (hence \((\muc,\nuc)=(0,0)\)):
\begin{enumerate}
\item[(S1)] \textbf{Commutation and apex colimit:} \(\mathbf{T}_\tau\) commutes with the filtered colimit of \(\{\mathbf{P}_i(F_n)\}\) in \([\mathbb{R},\mathsf{Vect}_k]\), the outcome is constructible, and the cocone exhibits a colimit at persistence level (i.e.\ Proposition~\ref{D:prop:iso-zero}(iv) holds).
\item[(S2)] \textbf{No \(\tau\)-accumulation from below:} there exists \(\eta>0\) such that, for all sufficiently large \(n\), no bar in \(\mathbf{P}_i(F_n)\) has length in the half-open interval \((\tau-\eta,\tau)\).
Equivalently, there is no sequence of bar lengths strictly increasing to \(\tau\).
\item[(S3)] \textbf{\(\mathbf{T}_\tau\)-Cauchy with compatible cocone:} the sequence \(\mathbf{T}_\tau(\mathbf{P}_i(F_n))\) is Cauchy in the interleaving metric, and the cocone to \(\mathbf{T}_\tau(\mathbf{P}_i(F_\infty))\) identifies the metric limit with the colimit target.
(Here we use only the standard completeness/uniqueness of limits for p.f.d.\ barcodes under the bottleneck/interleaving metric.)
\end{enumerate}
\end{theorem}

\begin{proof}
(S1) is Proposition~\ref{D:prop:iso-zero}.
For (S2), the gap prevents creation at the apex of new bars of length \(>\tau\):
every long bar in \(\mathbf{T}_\tau(\mathbf{P}_i(F_\infty))\) must appear at some finite stage and stabilize, yielding bijectivity on interval factors.
For (S3), completeness of the space of p.f.d.\ persistence modules up to isometry implies a unique metric limit; the stated compatibility identifies it with the colimit target, so \(\phi_{i,\tau}\) is an isometry and hence an isomorphism in \(\Pers^{\mathrm{ft}}_k\).
\end{proof}

% -------------------------
\subsection*{D.3.1. A counterexample: $\sum d_{\mathrm{int}}<\infty$ yet $(\muc,\nuc)\neq(0,0)$}
% -------------------------

\begin{example}[Summable increments, pure cokernel at the apex]\label{D:ex:summable}
Fix \(\tau>0\) and set \(\ell_n=\tau-\sum_{m\ge n}2^{-m}\uparrow \tau\), so that
\(\sum_{n}(\ell_{n+1}-\ell_{n})=\sum_{n}2^{-n}<\infty\).
Let \(M_n:=I[0,\ell_n)\) with \(M_{n}\hookrightarrow M_{n+1}\) the standard inclusions.
Then
\[
d_{\mathrm{int}}(M_{n},M_{n+1})=\tfrac12(\ell_{n+1}-\ell_{n})=2^{-(n+1)},\qquad \sum_{n} d_{\mathrm{int}}(M_{n},M_{n+1})<\infty.
\]
Let \(\mathbf{P}_1(F_n)=M_n\), and choose an apex with \(\mathbf{P}_1(F_\infty)=I[0,\infty)\).
For every \(n\), \(\mathbf{T}_\tau(M_n)=0\), while \(\mathbf{T}_\tau(\mathbf{P}_1(F_\infty))=I[0,\infty)\).
Thus \(\muc=0\), \(\nuc=1\) (pure cokernel), despite the summable interleaving distances along the tower.
\end{example}

This shows that \(\sum d_{\mathrm{int}}<\infty\) alone is insufficient to force \((\muc,\nuc)=(0,0)\).

% -------------------------
\subsection*{D.4. Converse failures and the Type~IV catalog}
% -------------------------

\paragraph{D.4.1. \(\Ext^1=0\) does not imply \(\mathrm{PH}_1=0\).}
Let \(A\in D^{[-1,0]}(k\text{-mod})\) with \(H^{-1}(A)=0\) and \(H^0(A)\ne 0\), e.g.\ the stalk complex \(V[0]\) for a nonzero \(k\)-space \(V\).
Then \(\Ext^1(A,k)\cong \Hom(H^{-1}(A),k)=0\) by Appendix~C (Lemma~\ref{C:lem:edge-ext}).
Choose \(F\in\mathsf{FiltCh}(k)\) with \(\mathbf{P}_1(F)\neq 0\) (e.g.\ a single finite interval) and \(\mathcal{R}(F)\simeq A\); this can be arranged up to f.q.i.\ using the realization assignment \(\mathcal{U}\) (Appendix~B).
Hence \(\Ext^1(\mathcal{R}(F),k)=0\) while \(\mathrm{PH}_1(F)\neq 0\), refuting the converse of the bridge.

\paragraph{D.4.2. Type~IV (pure cokernel) at fixed \(\tau\).}
Example~\ref{D:ex:pure-coker} exhibits \(\muc=0\), \(\nuc>0\) with \(\mathbf{T}_\tau(\mathbf{P}_1(F_n))=0\) for all \(n\) but \(\mathbf{T}_\tau(\mathbf{P}_1(F_\infty))\neq 0\).
Thus finite layers appear admissible while the apex fails.

\paragraph{D.4.3. Type~IV (mixed).}
Example~\ref{D:ex:mixed} yields \(\muc>0\) and \(\nuc>0\) simultaneously, demonstrating that both kernel and cokernel defects can occur in the same tower.

\paragraph{D.4.4. Realization notes.}
All persistence-level constructions above are realizable by filtered complexes via \(\mathcal{U}\) (Appendix~B), up to f.q.i.; constructibility is preserved.

% -------------------------
\subsection*{D.5. Restart/Summability for window pasting}
% -------------------------

We formalize window pasting for towers equipped with stepwise error budgets (the \(\delta\)-ledger) and safety margins. All persistence-layer statements are made \emph{after} applying \(\mathbf{T}_\tau\).

\begin{definition}[Per-window safety margin and pipeline budget]\label{D:def:window-budget}
Let \(\{W_k=[u_k,u_{k+1})\}_{k\in K}\) be a MECE partition (Appendix~A, Definition~\ref{A:def:MECE}). On each window \(W_k\), fix a collapse threshold \(\tau_k>0\). For a degree \(i\), define the \emph{pipeline budget}
\[
\Sigma\delta_k(i)\ :=\ \sum_{U\in W_k}\Big(\delta^{\mathrm{alg}}_{U}(i,\tau_k)+\delta^{\mathrm{disc}}_{U}(i,\tau_k)+\delta^{\mathrm{meas}}_{U}(i,\tau_k)\Big),
\]
and the \emph{safety margin} \(\mathrm{gap}_{\tau_k}(i)>0\) as the configured slack for B\hyp Gate\(^{+}\) on \(W_k\) and degree \(i\).
\end{definition}

\begin{lemma}[Restart inequality]\label{D:lem:restart}
Assume that, on window \(W_k\), B\hyp Gate\(^{+}\) passes with \(\mathrm{gap}_{\tau_k}(i)>\Sigma\delta_k(i)\), and that the transition to \(W_{k+1}\) is realized by a finite composition of \emph{deletion-type} steps and \(\varepsilon\)\hyp continuations (both measured after \(\mathbf{T}_\tau\)). Then there exists \(\kappa\in(0,1]\), depending only on the admissible step class and the \(\tau\)-adaptation policy, such that
\[
\mathrm{gap}_{\tau_{k+1}}(i)\ \ge\ \kappa\bigl(\mathrm{gap}_{\tau_k}(i)-\Sigma\delta_k(i)\bigr).
\]
\end{lemma}

\begin{proof}[Proof sketch]
Deletion-type steps are non\hyp increasing for the monitored indicators after \(\mathbf{T}_\tau\) (Appendix~E), and \(\varepsilon\)\hyp continuations are \(1\)\hyp Lipschitz. Aggregating drifts yields the stated retention factor \(\kappa\).
\end{proof}

\begin{definition}[Summability]\label{D:def:summability}
A run satisfies \emph{Summability} (on a degree set \(I\subset\mathbb{Z}\)) if
\[
\sum_{k\in K}\Sigma\delta_k(i)\ <\ \infty\qquad(\forall\,i\in I).
\]
A sufficient design is a geometric decay of \(\tau_k\) (hence of spectral/temporal bins) and bounded per\hyp window step counts.
\end{definition}

\begin{theorem}[Pasting windowed certificates]\label{D:thm:pasting}
Let \(\{W_k\}_k\) be MECE, and on each \(W_k\) let B\hyp Gate\(^{+}\) pass with \(\mathrm{gap}_{\tau_k}(i)>\Sigma\delta_k(i)\) for all \(i\in I\). If the Restart inequality (Lemma~\ref{D:lem:restart}) holds at every transition and Summability (Definition~\ref{D:def:summability}) holds, then the concatenation of windowed certificates yields a global certificate on \(\bigcup_k W_k\) for the degrees \(i\in I\).
\end{theorem}

\begin{proof}[Proof sketch]
Iterate Lemma~\ref{D:lem:restart} and sum budgets; Summability ensures that the cumulative loss of the safety margin remains bounded, so positivity of the margin persists. MECE coverage (Appendix~A) ensures there are no gaps/overlaps.
\end{proof}

% -------------------------
\subsection*{D.6. Stability bands and \texorpdfstring{$\tau$}{tau}-sweeps}
% -------------------------

We record a practical criterion to certify \((\muc,\nuc)=(0,0)\) on contiguous ranges of \(\tau\).

\begin{definition}[Stability band via \(\tau\)-sweep]\label{D:def:stab-band}
Fix a window \(W\) and degree \(i\). Let \(\{\tau_\ell\}_{\ell=1}^L\) be an increasing \(\tau\)-sweep. A contiguous block \(\{\tau_a,\dots,\tau_b\}\) is a \emph{stability band} if
\[
\mu_{i,\tau_\ell}=\nu_{i,\tau_\ell}=0\quad\text{for all }\ \ell\in\{a,\dots,b\},
\]
and the verdict persists upon \emph{refining} the sweep (inserting new \(\tau\)-values) without introducing \(\mu\) or \(\nu\) in the band.
\end{definition}

\begin{proposition}[Robust detection of stability bands]\label{D:prop:robust-band}
Assume (S1)–(S3) of Theorem~\ref{D:thm:sufficient} hold on \(W\). Then any sufficiently fine \(\tau\)-sweep admits stability bands covering all \(\tau\) at which \(\phi_{i,\tau}\) is an isomorphism; conversely, detecting a stability band by a sweep and its refinement certifies \((\muc,\nuc)=(0,0)\) on the band.
\end{proposition}

\begin{proof}[Proof sketch]
Under (S1)–(S3), \(\phi_{i,\tau}\) is an isomorphism on open neighborhoods of the corresponding \(\tau\)’s. A fine sweep samples each neighborhood; refinement eliminates aliasing. The converse follows by definition.
\end{proof}

\begin{remark}[Caveat: non-monotonicity in \(\tau\)]
There is no general monotonicity of \(\mu_{i,\tau}\) or \(\nu_{i,\tau}\) in \(\tau\). Stability bands may be separated by isolated \(\tau\)-values where \(\phi_{i,\tau}\) fails to be an isomorphism.
\end{remark}

% -------------------------
\subsection*{D.7. Completion note}
% -------------------------

\begin{remark}[No further supplementation required]
This appendix now provides: (i) the definition and calculus of the tower obstruction indices \((\mu,\nu)\) (generic fiber dimensions after truncation), (ii) illustrative toy towers (pure kernel/cokernel/mixed) and a counterexample showing that \(\sum d_{\mathrm{int}}<\infty\) does not force \((\muc,\nuc)=(0,0)\), (iii) sufficient conditions (S1)–(S3) guaranteeing \((\muc,\nuc)=(0,0)\), (iv) a \emph{Restart/Summability} framework to paste windowed certificates into global ones, and (v) a robust \(\tau\)-sweep procedure and \emph{stability bands} to certify \((\muc,\nuc)=(0,0)\) on contiguous \(\tau\)-ranges. All statements are confined to the v15.0 guard-rails (constructible \(1\)D persistence over a field; persistence-layer equalities after truncation; f.q.i.\ on filtered complexes), and no further supplementation is required for operational use in the proof framework.
\end{remark}

\medskip
\noindent\textbf{Cross-module conventions.}
Ext-tests are always taken against \(k[0]\) (Appendix~C): \(\Ext^1(\mathcal{R}(C_\tau F),k)=0\).
When windowed energy summaries are referenced elsewhere, the exponent is uniformly \(\alpha>0\) (default \(\alpha=1\)).
Update monotonicity follows the global rule: \emph{deletion-type} updates are non-increasing for windowed energies and spectral tails after truncation, whereas \emph{inclusion-type} updates are stable (non-expansive); see Appendix~E.
Type labels follow the global convention \emph{Type~I–II / Type~III / Type~IV}.



% =========================
\section*{Appendix E. Spectral Indicators: Monotonicity, Stability, Counterexamples [Proof/Spec] (reinforced)}
% =========================
\phantomsection
\addcontentsline{toc}{section}{Appendix E. Spectral Indicators: Monotonicity, Stability, Counterexamples}
\refstepcounter{section} % stabilize cross-references with an unnumbered section

Throughout, \(k\) is a field and all matrices are real symmetric (Hermitian) and finite-dimensional.
For a Hermitian matrix \(H\), write its eigenvalues in \emph{nondecreasing} (ascending) order
\(\lambda_1(H)\le \cdots \le \lambda_n(H)\).
For a threshold \(\theta\in\mathbb{R}\), define the \emph{spectral counting indicator}
\[
N_\theta(H)\ :=\ \mathrm{rank}\,\mathbf{1}_{[\theta,\infty)}(H)\quad\text{(number of eigenvalues \(\ge\theta\), counted with multiplicity)}.
\]
For \(\tau>0\), define the \emph{clipped spectrum}
\(\mathrm{clip}_\tau(H):=(\min\{\lambda_i(H),\tau\})_{i=1}^n\), the \emph{clipped sum}
\(S^{\le \tau}(H):=\sum_{j=1}^n \min\{\lambda_j(H),\tau\}\), and the \emph{sub-threshold deficit}
\[
D^{<\tau}(H)\ :=\ \sum_{j=1}^n (\tau-\lambda_j(H))_+,\qquad x_+:=\max\{x,0\}.
\]
We use the operator norm \(\|\cdot\|_{\mathrm{op}}\) and Frobenius norm \(\|\cdot\|_{\mathrm{fro}}\).
All references to filtered colimits obey the scope rule in Appendix~A, Remark~\ref{A:rk:filtered-colimits}.
Cross-module conventions (used globally): tests of \(\Ext\) are taken against \(k[0]\) (Appendix~C), i.e.\ \(\Ext^1(\mathcal{R}(C_\tau F),k)=0\); energy exponents are uniform \(\alpha>0\) (default \(\alpha=1\)); type labels use \emph{Type~I--II / Type~III / Type~IV}.
When claims reference persistence/filtered complexes elsewhere, equalities are at the persistence layer (strict in \(\Pers^{\mathrm{ft}}_k\)); filtered–complex claims are \emph{up to f.q.i.}

\medskip
\noindent\textbf{Deletion vs.\ inclusion.}
When \(H\) arises by restricting admissible degrees of freedom, imposing Dirichlet constraints, eliminating internal dofs by shorting (Schur complement), or taking principal submatrices, we call conclusions \emph{deletion-type}.
When \(H\) is obtained by adding degrees of freedom, couplings, or enlarging a domain, we call them \emph{inclusion-type}.
Deletion-type updates admit one-sided monotonicity; inclusion-type updates admit only stability (non-expansive), unless additional order hypotheses are imposed.

% -------------------------
\subsection*{E.1. Deletion-type monotonicity (principal/Dirichlet, Schur complement, Loewner)}
% -------------------------

\begin{proposition}[Principal/Dirichlet restriction: interlacing and counting]\label{E:prop:dirichlet}
\textup{[Proof]}
Let \(A\in\mathbb{R}^{n\times n}\) be Hermitian and \(B\) a principal \((n-1)\times(n-1)\) submatrix (obtained, e.g., by pinning a coordinate—``Dirichlet restriction'').
Then Cauchy interlacing (with ascending ordering) holds:
\[
\lambda_1(A)\ \le\ \lambda_1(B)\ \le\ \lambda_2(A)\ \le\ \cdots\ \le\ \lambda_{n-1}(B)\ \le\ \lambda_n(A).
\]
In particular, for every \(\theta\in\mathbb{R}\),
\[
N_\theta(B)\ \le\ N_\theta(A).
\]
\emph{Proof.} Interlacing is classical; if \(B\) has at least \(j\) eigenvalues \(\ge\theta\), then by interlacing \(A\) has at least \(j\) eigenvalues \(\ge\theta\).
\end{proposition}

\begin{proposition}[Schur complement (shorting) monotonicity]\label{E:prop:schur}
\textup{[Proof]}
Partition \(M=\begin{psmallmatrix}A & B\\ B^\top & C\end{psmallmatrix}\succeq 0\) with \(C\succ 0\) and form the Schur complement \(S:=A-BC^{-1}B^\top\).
Then \(S\preceq A\).
Consequently, for all \(j\) and all \(\theta\ge 0\),
\[
\lambda_j(S)\ \le\ \lambda_j(A),\qquad N_\theta(S)\ \le\ N_\theta(A).
\]
\end{proposition}

\begin{proposition}[Loewner-order monotonicity]\label{E:prop:loewner}
\textup{[Proof]}
If \(0\preceq A\preceq B\) (Loewner order), then for each \(j\),
\(\lambda_j(A)\le \lambda_j(B)\) and, for every \(\theta\ge 0\), \(N_\theta(A)\le N_\theta(B)\).
\emph{Sketch.}
Weyl (min–max) monotonicity yields coordinatewise eigenvalue monotonicity; the counting inequality follows for \(\theta\ge 0\).
\end{proposition}

\begin{corollary}[Conservative averaging]\label{E:cor:avg}
\textup{[Proof]}
If \(A_1,\dots,A_m\succeq 0\) satisfy \(A_\ell\preceq A\) for all \(\ell\), then for any convex combination \(\bar A:=\sum_\ell w_\ell A_\ell\) with \(w_\ell\ge 0\), \(\sum_\ell w_\ell=1\),
\[
\bar A\ \preceq\ A\qquad\Rightarrow\qquad\n\lambda_j(\bar A)\le \lambda_j(A),\ \ N_\theta(\bar A)\le N_\theta(A)\ \ (\theta\ge 0).
\]
\end{corollary}

\begin{remark}[Orientation convention for ``deletions'']
Two Loewner orientations occur in practice.
\emph{Contractions} (e.g.\ Schur complements, Kron reduction) produce \(A'\preceq A\) (cf.\ Propositions~\ref{E:prop:schur}, \ref{E:prop:loewner}); \emph{hardening} operations (e.g.\ some PDE Dirichlet comparisons across different media) may yield \(A'\succeq A\) (reverse of Propositions~\ref{E:prop:dirichlet}, \ref{E:prop:loewner}).
All monotone statements below are given in both orientations where relevant.
\end{remark}

% -------------------------
\subsection*{E.2. Inclusion-type counterexamples}
% -------------------------

Deletion-type monotonicity does \emph{not} extend naively to inclusion-type operations (adding dofs, couplings, or enlarging a domain) without additional order hypotheses.

\begin{example}[Neumann/domain inclusion reverses direction]\label{E:ex:neumann}
\textup{[Spec]}
For the Neumann Laplacian on an interval, enlarging the domain decreases the nonzero eigenvalues:
on \([0,L]\), the first nonzero Neumann eigenvalue is \((\pi/L)^2\), so passing \(L:1\to 2\) reduces it from \(\pi^2\) to \((\pi/2)^2\).
Thus any “inclusion \(\Rightarrow\) increase” heuristic fails under Neumann-type constraints.
\end{example}

\begin{example}[Indefinite coupling can move eigenvalues both ways]\label{E:ex:indef}
\textup{[Proof]}
Let \(A=I_2=\mathrm{diag}(1,1)\) and
\(B=\begin{psmallmatrix}1 & M\\ M & 1\end{psmallmatrix}\) with \(M>1\).
Then \(B\) has eigenvalues \(1-M\) and \(1+M\), so for \(\theta=0\),
\(N_\theta(B)=1<N_\theta(A)=2\), while the top eigenvalue \(\lambda_2\) increases.
Without a Loewner relation (\(B-A\) indefinite), no monotone law survives.
\end{example}

\begin{example}[Principal extension lacks a fixed direction]\label{E:ex:principal-extend}
\textup{[Proof]}
Let \(B=[0]\) (eigenvalue \(0\)) and
\(A=\begin{psmallmatrix}0 & t\\ t & 0\end{psmallmatrix}\) with \(t\neq 0\).
Going from \(B\) to \(A\) (adding one dof and a coupling) produces eigenvalues \(-|t|\) and \(|t|\):
the maximum increases to \(|t|\), but the minimum decreases to \(-|t|\).
Hence no uniform increase/decrease holds under inclusion.
\end{example}

These examples justify restricting monotone claims to the deletion/Loewner settings used elsewhere in the paper.

% -------------------------
\subsection*{E.3. Continuity, stability, and truncated functionals}
% -------------------------

Write \(N_{\theta\pm 0}(A)\) for the left/right limits at \(\theta\) (no jump unless \(\theta\) is an eigenvalue).

\begin{proposition}[Weyl and Hoffman–Wielandt]\label{E:prop:weyl}
\textup{[Proof]}
For Hermitian \(A,B\in\mathbb{R}^{n\times n}\),
\[
\max_{1\le j\le n}\,|\lambda_j(A)-\lambda_j(B)|\ \le\ \|A-B\|_{\mathrm{op}},\qquad\n\Big(\sum_{j=1}^n |\lambda_j(A)-\lambda_j(B)|^2\Big)^{\!1/2}\ \le\ \|A-B\|_{\mathrm{fro}}.
\]
Hence \(A\mapsto(\lambda_1(A),\dots,\lambda_n(A))\) is \(1\)-Lipschitz from \((\|\cdot\|_{\mathrm{op}})\) into \((\mathbb{R}^n,\|\cdot\|_\infty)\).
\end{proposition}

\begin{corollary}[Lipschitz stability of clipped spectra]\label{E:cor:clip}
\textup{[Proof]}
For any \(\tau>0\) and Hermitian \(A,B\),
\[
\sum_{j=1}^n\Big|\min\{\lambda_j(A),\tau\}-\min\{\lambda_j(B),\tau\}\Big|\n\ \le\ \sum_{j=1}^n|\lambda_j(A)-\lambda_j(B)|\ \le\ \sqrt{n}\,\|A-B\|_{\mathrm{fro}}\ \le\ n\,\|A-B\|_{\mathrm{op}}.
\]
Consequently, \(S^{\le \tau}\) is \(\sqrt{n}\)–Lipschitz in \(\|\cdot\|_{\mathrm{fro}}\) and \(n\)–Lipschitz in \(\|\cdot\|_{\mathrm{op}}\).
\end{corollary}

\begin{proposition}[Semicontinuity of counting indicators]\label{E:prop:count-semi}
\textup{[Proof]}
If \(A_m\to A\) in operator norm and \(\theta\) is not an eigenvalue of \(A\), then \(N_\theta(A_m)=N_\theta(A)\) for all large \(m\) (local constancy).
In general,
\[
\limsup_{m\to\infty} N_\theta(A_m)\ \le\ N_{\theta-0}(A),\qquad\n\liminf_{m\to\infty} N_\theta(A_m)\ \ge\ N_{\theta+0}(A).
\]
\emph{Sketch.}
Combine Proposition~\ref{E:prop:weyl} with spectral-gap perturbation of spectral projectors; jumps occur only when eigenvalues cross \(\theta\).
\end{proposition}

\begin{proposition}[Truncated functionals: monotonicity and stability]\label{E:prop:trunc-func}
Fix $\tau>0$. For an $n\times n$ positive semidefinite (PSD) matrix $A$, set
\[
S^{\le\tau}(A):=\sum_{j=1}^n \min\{\lambda_j(A),\tau\},\n\qquad\nD^{<\tau}(A):=\sum_{j=1}^n (\tau-\lambda_j(A))_{+},
\]
and $N_\theta(A):=\#\{j:\lambda_j(A)\ge \theta\}$ for $\theta\ge 0$. Then:
\begin{enumerate}\setlength{\itemsep}{0.2em}
\item \emph{(Deletion; Loewner contraction $A'\preceq A$.)}
For all $j$, $\lambda_j(A')\le \lambda_j(A)$, hence $N_\theta(A')\le N_\theta(A)$ for every $\theta\ge 0$, and
\[
S^{\le\tau}(A')\le S^{\le\tau}(A),\qquad D^{<\tau}(A')\ge D^{<\tau}(A).
\]
\item \emph{(Deletion; Loewner hardening $A'\succeq A$.)}
All inequalities in \textup{(1)} reverse:
\[
\lambda_j(A')\ge \lambda_j(A),\quad N_\theta(A')\ge N_\theta(A)\ \ (\theta\ge 0),\quad\nS^{\le\tau}(A')\ge S^{\le\tau}(A),\quad D^{<\tau}(A')\le D^{<\tau}(A).
\]
\item \emph{(Lipschitz stability.)} For any Hermitian $A,B$,
\[
\bigl|D^{<\tau}(A)-D^{<\tau}(B)\bigr|\n\le \sum_{j=1}^n \bigl|\lambda_j(A)-\lambda_j(B)\bigr|\n\le \sqrt{n}\,\|A-B\|_{\mathrm{fro}}\n\le n\,\|A-B\|_{\mathrm{op}}.
\]
\end{enumerate}
\end{proposition}

\begin{proof}
Items (1) and (2) follow from the scalar monotonicity of
\(x\mapsto \mathbf{1}_{[\theta,\infty)}(x)\), \(x\mapsto \min\{x,\tau\}\), and \(x\mapsto (\tau-x)_+\),
together with Loewner/Weyl monotonicity for eigenvalues.
Item (3) follows from Proposition~\ref{E:prop:weyl} (Weyl’s eigenvalue perturbation inequality)
and the fact that \(x\mapsto(\tau-x)_+\) is \(1\)-Lipschitz.
\end{proof}

\begin{remark}
All spectral indicators used in the main text (counts above thresholds, clipped sums, and extremal eigenvalues of Dirichlet/principal/shorted forms) inherit the deletion-type monotonicities of §E.1 in the appropriate Loewner orientation and the stability bounds of §E.3.
Inclusion-type claims are never invoked without additional hypotheses (e.g.\ a Loewner-order relation or coercivity); cf.\ §E.2.
\end{remark}

% -------------------------
\subsection*{E.4. Auxiliary spectral bars (aux-bars): definition, stability, and policy \texorpdfstring{[Spec]}{[Spec]}}
% -------------------------

We formalize the optional \emph{auxiliary spectral bars} used as diagnostics alongside persistence.
All claims in this subsection are policy/\textbf{[Spec]} unless explicitly proved.

\paragraph{E.4.1. Binning and endpoint convention.}
Fix a spectral window \([a,b]\) with \(a<b\) and a bin width \(\beta>0\).
Let \(R:=\big\lfloor\frac{b-a}{\beta}\big\rfloor\).
Define \emph{half-open, right-attribution} bins
\[
I_r\ :=\ [\,a+r\beta,\ a+(r+1)\beta\,)\qquad (r=0,1,\dots,R-1).
\]
An eigenvalue at a \emph{right boundary} \(a+(r+1)\beta\) is counted in the \emph{next} bin \(I_{r+1}\) (the half-open convention enforces this).
We log \emph{underflow} \(U(H):=\#\{j:\lambda_j(H)<a\}\) and \emph{overflow} \(O(H):=\#\{j:\lambda_j(H)\ge b\}\).
For a Hermitian \(H\), define the \emph{bin occupancy}
\[
E_r(H)\ :=\ \#\{j:\lambda_j(H)\in I_r\}\qquad (r=0,\dots,R-1).
\]
The \emph{cumulative (upper-tail) profile} at the bin grid is
\[
C_r(H)\ :=\ \sum_{s=r}^{R-1} E_s(H)\ =\ N_{\,a+r\beta}(H)\ -\ O(H)\qquad (r=0,\dots,R-1),
\]
i.e.\ the count of eigenvalues in \([a+r\beta,\ b)\).

\paragraph{E.4.2. Aux-bars across an index (time/tower).}
Let \((H_j)_{j\in J}\) be a finite or countable sequence of Hermitian matrices (e.g.\ along a tower/window index; Appendix~A, §A.6).
For fixed \(r\), the set \(\{j\in J: E_r(H_j)>0\}\) decomposes into \emph{maximal consecutive runs} \(J_{r,\ell}\) (in the ambient order of \(J\)).
Each run \(J_{r,\ell}\) defines an \emph{aux-bar} \((r, J_{r,\ell})\) with \emph{lifetime} \(|J_{r,\ell}|\) (counting measure on discrete \(J\); a rescaled duration may be used if \(J\) is metrized).
We record:
\[
\mathrm{aux\text{-}count}:=\sum_{r,\ell} 1,\qquad\n\mathrm{aux\text{-}mass}:=\sum_{r} E_r(H_j)\ \ \text{(per index \(j\))},\qquad\n\mathrm{active\ bins}:=\#\{r: E_r(H_j)>0\}.
\]
Underflow/overflow are always logged in addition.

\paragraph{E.4.3. Monotonicity/stability—what is provable, what is policy.}
The following are \emph{provable} and used when monotonic guarantees are required:

\begin{proposition}[Cumulative-profile monotonicity under Loewner]\label{E:prop:aux-cum-mono}
\textup{[Proof]}
If \(A'\preceq A\) (Loewner contraction, PSD case) or if \(A'\) is a principal/Dirichlet restriction of \(A\), then for every \(r\),
\[
C_r(A')\ \le\ C_r(A).
\]
Equivalently, the entire upper-tail profile \(r\mapsto C_r(\cdot)\) is pointwise non-increasing under deletion-type updates.
\end{proposition}

\begin{proof}
By definition \(C_r(H)=N_{a+r\beta}(H)-O(H)\) and \(O(H)\ge 0\).
For \(\theta\ge 0\), \(N_\theta\) is monotone in Loewner order (Proposition~\ref{E:prop:loewner}); for Dirichlet/principal restriction, use interlacing (Proposition~\ref{E:prop:dirichlet}).
Thus \(N_{a+r\beta}(A')\le N_{a+r\beta}(A)\), and hence \(C_r(A')\le C_r(A)\).
\end{proof}

\begin{proposition}[Cumulative-profile stability under small perturbations]\label{E:prop:aux-cum-stab}
\textup{[Proof]}
Let \(\|A-B\|_{\mathrm{op}}\le \varepsilon\), and let \(q:=\lceil \varepsilon/\beta\rceil\).
Then for all \(r\),
\[
C_{\,r+q}(B)\ \le\ C_r(A)\ \le\ C_{\,\max\{0,r-q\}}(B).
\]
In particular, if \(\varepsilon<\beta\), then the cumulative profile can shift by at most one bin index.
\end{proposition}

\begin{proof}
Weyl’s inequality (Proposition~\ref{E:prop:weyl}) implies \(|\lambda_j(A)-\lambda_j(B)|\le \varepsilon\).
Hence any eigenvalue \(\ge a+r\beta\) for \(A\) is \(\ge a+(r-q)\beta\) for \(B\), and any eigenvalue \(\ge a+(r+q)\beta\) for \(B\) was \(\ge a+r\beta\) for \(A\).
Translating to counts yields the stated inequalities.
\end{proof}

\begin{remark}[Per-bin occupancies and active bins]
No general monotonicity holds for the number of \emph{per-bin} active bins \(r\) with \(E_r>0\) or for \(\mathrm{aux\text{-}count}\) across indices:
eigenvalues moving left under a deletion may split across bins or move into/out of \([a,b)\).
Accordingly, the \emph{cumulative} profile \(C_r\) is the \emph{only} quantity we use for guaranteed monotonicity.
All other aux-bar summaries are treated as diagnostics (\textbf{[Spec]}).
\end{remark}

\paragraph{E.4.4. Operational policy and B-side usage \texorpdfstring{[Spec]}{[Spec]}.}
- Deletion-type steps (principal/Dirichlet, Schur complement, Loewner contractions): enforce the monotone check on the \emph{cumulative} profile \(C_r\) (Proposition~\ref{E:prop:aux-cum-mono}). Per-bin plots and aux-bar lifetimes are allowed as diagnostics but not as proof obligations.

- \(\varepsilon\)-continuations: with a certified \(\|A_{j+1}-A_j\|_{\mathrm{op}}\le \varepsilon\), declare stability up to a bin shift \(q=\lceil \varepsilon/\beta\rceil\) using Proposition~\ref{E:prop:aux-cum-stab}. In manifests, record \texttt{eps\_cont\_shift\_bins}=\(q\).

- Inclusion-type steps: no monotone inequality is claimed; only stability bounds from §E.3 (Weyl/Hoffman–Wielandt) and Proposition~\ref{E:prop:aux-cum-stab} are used.

- Underflow/overflow: always log \((U,O)\). Gate policies may optionally require \(O=0\) (no mass beyond \(b\)) and \(C_{R-1}=0\) (no mass in the top bin) for conservative operation on the right tail.

- Zero-aux condition (\textbf{optional}): B\hyp Gate\(^{+}\) may require \(\mathrm{aux\text{-}count}=0\) on monitored bins; this is a \textbf{[Spec]} policy knob for deployments and is never used as a proof obligation.

\paragraph{E.4.5. Reproducibility fields.}
The manifest \texttt{run.yaml} should include:
\begin{itemize}
  \item \texttt{spec\_window}: \([a,b]\), \texttt{bin\_width}: \(\beta\), \texttt{bins}: \(R\), endpoint policy: \texttt{half-open/right-attribution};
  \item \texttt{underflow}/\texttt{overflow}: per index \(j\);
  \item \texttt{cum\_profile}: the sequence \(C_r(H_j)\) per \(j\); \texttt{aux\_bars}: list of runs \((r,J_{r,\ell})\) with lifetimes;
  \item \texttt{eps\_cont\_bound}: \(\varepsilon\) and derived \texttt{eps\_cont\_shift\_bins} \(= \lceil \varepsilon/\beta\rceil\);
  \item norm choice: \texttt{"op"} or \texttt{"fro"} (Appendix~G).
\end{itemize}

\medskip
\noindent\emph{Summary of Appendix E (reinforced).}
Deletion-type operations enjoy interlacing/Loewner monotonicity for counting indicators, clipped sums, and deficits; inclusion-type updates lack a fixed direction but remain stable under norm-bounded perturbations (Weyl/Hoffman–Wielandt).
Auxiliary spectral bars are defined via half-open, right-attribution binning on a fixed spectral window with explicit underflow/overflow counts.
For guaranteed monotonicity we monitor the \emph{cumulative} bin profile \(C_r=N_{a+r\beta}-O\), which is pointwise non-increasing under deletion-type steps and stable up to a bin shift under \(\varepsilon\)-continuations.
Per-bin occupancies, active-bin counts, and aux-bar lifetimes are treated as diagnostics \textbf{[Spec]} and are not used as proof obligations.
All logging fields needed for reproducibility are specified.



% =========================
\section*{Appendix F. Formalization Sketch (Lean/Coq) [Spec] (reinforced)}
% =========================
\phantomsection
\addcontentsline{toc}{section}{Appendix F. Formalization Sketch (Lean/Coq)}
\refstepcounter{section} % stabilize cross-references with an unnumbered section

This appendix provides an implementation-oriented \emph{Spec} for mechanizing the core claims of
Appendices~A–E in Lean/Coq: \emph{Serre localization} and the \emph{reflector} \(\mathbf{T}_\tau\) (exact, idempotent),
its \emph{$1$-Lipschitz} property on barcodes, tower diagnostics \((\mu,\nu)\) via a comparison map \(\phi_{i,\tau}\),
and the edge identification supporting the one-way bridge \(\mathrm{PH}_1\Rightarrow\Ext^1\).
We work in the \emph{constructible} (p.f.d.) range and adhere to the \emph{filtered colimit scope rule}
(Appendix~A, Remark~\ref{A:rk:filtered-colimits}). Cross-module conventions: \(\Ext\)-tests are always
against \(k[0]\) (Appendix~C), i.e.\ \(\Ext^1(\mathcal{R}(C_\tau F),k)=0\); the energy exponent is globally
\(\alpha>0\) (default \(\alpha=1\)); type labels use \emph{Type I--II / Type III / Type IV}.
Spectral monotonicity is invoked only for \emph{deletion-type} operations; inclusion-type operations are
used solely with stability bounds (Appendix~E).

\subsection*{F.1. Environment and objects [Spec]}
Fix a field \(k\).
Let \(\mathsf{Vect}_k\) be the abelian category of finite-dimensional \(k\)-vector spaces and
\([\mathbb{R},\mathsf{Vect}_k]\) the functor category (index \((\mathbb{R},\le)\)).
Let \(\Pers^{\mathrm{ft}}_k\subset[\mathbb{R},\mathsf{Vect}_k]\) be the full subcategory of \emph{constructible}
persistence modules (barcodes locally finite on bounded windows).
Let \(\mathsf{FiltCh}(k)\) be filtered chain complexes of finite-dimensional \(k\)-spaces, bounded in homological degree,
with filtration-preserving maps. For \(i\in\mathbb{Z}\) write \(\mathbf{P}_i:\mathsf{FiltCh}(k)\to\Pers^{\mathrm{ft}}_k\)
for the degree–\(i\) persistence functor, and \(\mathbf{T}_\tau:\Pers^{\mathrm{ft}}_k\to\Pers^{\mathrm{ft}}_k\) for the
\(\tau\)-truncation (collapse) functor (Appendix~A).

\begin{remark}[Generic fiber dimension and stabilization]\label{F:rk:generic-fiber}
We adopt Appendix~D, Remark~\ref{rem:D-generic-dim}. For \(M\in\Pers^{\mathrm{ft}}_k\), the \emph{generic fiber dimension}
is the multiplicity of the infinite interval \(I[0,\infty)\) in the barcode of \(M\); equivalently,
\[
\mathrm{gdim}(M)\ =\ \lim_{t\to +\infty}\dim_k M(t),
\]
which stabilizes in the constructible range.
After applying \(\mathbf{T}_\tau\), kernels and cokernels again lie in \(\Pers^{\mathrm{ft}}_k\), and \(\mathrm{gdim}\)
is computed there.
\end{remark}

\begin{specification}[Stabilization lemma for constructible modules]\label{F:spec:stabilize}
If \(M\in\Pers^{\mathrm{ft}}_k\), then there exist \(T_0\in\mathbb{R}\) and \(c\in\mathbb{N}\) such that
\(\dim_k M(t)=c=\mathrm{gdim}(M)\) for all \(t\ge T_0\).
\emph{Use:} define \(\mathrm{gdim}(M)\) by this stabilized value \(c\) and prove iso-invariance of \(\mathrm{gdim}\).
\end{specification}

\subsection*{F.2. Serre subcategory and localization [Spec]}
Let \(\mathsf{E}_\tau\subset \Pers^{\mathrm{ft}}_k\) be the Serre subcategory generated by intervals of length \(\le\tau\).
By Appendix~A, \(\mathsf{E}_\tau\) is hereditary Serre and the inclusion
\(\iota_\tau:\mathsf{E}_\tau^\perp\hookrightarrow \Pers^{\mathrm{ft}}_k\) admits an exact left adjoint
\(\mathbf{T}_\tau:\Pers^{\mathrm{ft}}_k\to \mathsf{E}_\tau^\perp\) (reflector), inducing an equivalence
\[
\Pers^{\mathrm{ft}}_k/\mathsf{E}_\tau\ \simeq\ \mathsf{E}_\tau^\perp.
\]
Here \(\mathsf{E}_\tau^\perp\) is the \(\tau\)-\emph{local} (orthogonal) subcategory.
Basic laws (API):
\[
\mathbf{T}_\tau\circ \mathbf{T}_\tau \cong \mathbf{T}_\tau,\qquad \mathbf{T}_\tau \dashv \iota_\tau .
\]
Moreover, \(\mathbf{T}_\tau\) is exact and preserves finite (co)limits (Appendix~A).

\subsection*{F.3. Lean~4 sketch (\textsf{mathlib} style) [Spec]}
\begin{verbatim}
-- F.3.1 Categories and constructibility
namespace AK
open scoped BigOperators Classical
noncomputable section

variable (k : Type*) [Field k]

/-- Abelian category of f.d. k-vector spaces (schematic alias). -/
abbrev Vect := FinVect k

/-- Functor category [ℝ, Vect_k] (schematic index for (ℝ, ≤)). -/
structure RIdx := (α : Type) (str : Preorder α)   -- [Spec]
abbrev Diag := (RIdx → Vect k)                    -- [ℝ, Vect_k]

/-- Constructible persistence modules (p.f.d. on bounded windows). -/
abbrev Pers := { M : Diag k // Constructible M }  -- Pers^{ft}_k

-- F.3.2 Serre subcategory E_τ and reflector T_τ
def shortBar (τ : ℝ≥0) (I : Interval) : Prop := I.length ≤ τ
def Eτ (τ : ℝ≥0) : SerreSubcategory (Pers k) := by admit     -- [Spec]

/-- Reflector \mathbf{T}_τ : Pers → E_τ^⊥ (τ-local, orthogonal). -/
noncomputable def Tτ (τ : ℝ≥0) : Pers k ⥤ (Eτ k τ).orthogonal := by admit
theorem Tτ_exact (τ) : (Tτ k τ).IsExact := by admit
theorem Tτ_idem  (τ) : (Tτ k τ) ⋙ (Tτ k τ) ≅ (Tτ k τ) := by admit
/-- Inclusion ι_τ : E_τ^⊥ ↪ Pers. -/
noncomputable def iotaτ (τ) : (Eτ k τ).orthogonal ⥤ Pers k := by admit
theorem Tτ_adj   (τ) : (Tτ k τ) ⊣ (iotaτ k τ) := by admit
theorem localization_equiv (τ) :
  (Pers k) ⧸ (Eτ k τ) ≌ (Eτ k τ).orthogonal := by admit

-- F.3.3 Interleaving metric, shifts, and non-expansiveness of T_τ
/-- Pseudometric space of constructible persistence modules (interleaving). -/
class Interleaving (C : Type*) :=
  (dist : C → C → ℝ≥0∞)
  (isPseudoMetric : PseudoMetricSpace C)
def d_int := (Interleaving.dist : Pers k → Pers k → ℝ≥0∞)

/-- Shift functor on persistence modules (index translation by ε). -/
noncomputable def Shift (ε : ℝ≥0) : Pers k ⥤ Pers k := by admit   -- [Spec]
axiom shift_zero : Shift k (0:ℝ≥0) ≅ �� (Pers k)
axiom shift_add  : ∀ ε δ, Shift k ε ⋙ Shift k δ ≅ Shift k (ε + δ)

/-- T_τ commutes with shifts up to iso (Appendix A). -/
axiom shift_comm (τ ε) :
  Shift k ε ⋙ (Tτ k τ) ≅ (Tτ k τ) ⋙ Shift k ε

/-- Non-expansiveness: ε-interleavings transport through T_τ via shift_comm. -/
theorem Tτ_nonexpansive (τ) :
  ∀ M N : Pers k, d_int k ((Tτ k τ).obj M) ((Tτ k τ).obj N) ≤ d_int k M N := by
  admit

-- F.3.4 Exactness of filtered colimits and return-to-constructible
/-- Exactness of filtered colimits in Vect_k. -/
theorem filtered_colim_exact :
  ∀ {J} [IsFiltered J] (F : J ⥤ Vect), ExactFilteredColim F := by admit  -- [Spec]

/-- Colimits computed in [ℝ, Vect_k] with return to Pers^{ft}_k (scope rule). -/
axiom return_to_constructible :
  ∀ (D : SomeFilteredDiagram), Constructible (colim D)                      -- [Spec]

-- F.3.5 Persistence from filtered complexes and φ_{i,τ}
abbrev FiltCh := FiltChCat k
def P_i (i : ℤ) : FiltCh k ⥤ Pers k := by admit   -- degreewise H_i

/-- ✅ Order fixed: P_i ⋙ T_τ (persistence then truncation). -/
def TτP (i : ℤ) (τ : ℝ≥0) : FiltCh k ⥤ (Eτ k τ).orthogonal :=
  (P_i k i) ⋙ (Tτ k τ)

/-- Towers with a chosen cocone to the apex. -/
structure Tower :=
  (obj : ℕ → FiltCh k)
  (map : ∀ n, obj n ⟶ obj (n+1))
  (apex : FiltCh k)
  (cocone : ∀ n, obj n ⟶ apex)

/-- Iterated composition along the tower from n to m (n ≤ m). -/
def iter_mor (T : Tower k) : ∀ {n m}, n ≤ m → T.obj n ⟶ T.obj m
| n, n, _ => �� _
| n, (m+1), h =>
  have : n ≤ m := Nat.le_of_lt_succ h
  iter_mor (T := T) this ≫ T.map m

/-- Cofinal reindexing of towers via a strictly monotone r : ℕ → ℕ. -/
def reindex (T : Tower k) (r : ℕ → ℕ) (h : StrictMono r) : Tower k :=
{ obj    := fun n => T.obj (r n),
  map    := fun n =>
    (let hlt : r n < r (n+1) := h (Nat.lt_succ_self n)
     let hle : r n ≤ r (n+1) := Nat.le_of_lt hlt
     iter_mor (T := T) hle),
  apex   := T.apex,
  cocone := fun n => T.cocone (r n) }

/-- Morphisms of towers (levelwise maps compatible with edges and cocones). -/
structure TowerHom (T T' : Tower k) :=
  (α : ∀ n, T.obj n ⟶ T'.obj n)
  (square : ∀ n, T.map n ≫ α (n+1) = α n ≫ T'.map n)
  (cocone_comm : ∀ n, α n ≫ T'.cocone n = T.cocone n)

/-- Comparison map φ_{i,τ} induced by the cocone. -/
def phi (i : ℤ) (τ : ℝ≥0) (T : Tower k) :
  colim (fun n ↦ (TτP k i τ).obj (T.obj n))
       ⟶ (TτP k i τ).obj T.apex := by
  admit

/-- Naturality of φ with respect to morphisms of towers. -/
theorem phi_natural :
  ∀ {i τ} {T T' : Tower k} (h : TowerHom k T T'),
    (colim.map h.α) ≫ phi k i τ T' = phi k i τ T := by
  admit

-- F.3.6 Generic fiber dimension and (μ,ν)
theorem gdim_stabilizes (M : Pers k) :
  ∃ (T0 : ℝ) (c : ℕ), ∀ t, t ≥ T0 → dim (M.val t) = c := by admit

noncomputable def gdim (M : Pers k) : ℕ :=
  let ⟨T0, c, _⟩ := gdim_stabilizes k M; c

theorem gdim_iso_invariant {X Y : Pers k} (e : X ≅ Y) : gdim k X = gdim k Y := by admit

noncomputable def mu (i : ℤ) (τ : ℝ≥0) (T : Tower k) : ℕ :=
  gdim k (kernel (phi k i τ T))
noncomputable def nu (i : ℤ) (τ : ℝ≥0) (T : Tower k) : ℕ :=
  gdim k (cokernel (phi k i τ T))

theorem mu_nu_fqi_invariant :
  ∀ {i τ T T'}, (T ≅ T') →
    mu k i τ T = mu k i τ T' ∧ nu k i τ T = nu k i τ T' := by admit

theorem mu_nu_cofinal_invariant :
  ∀ {i τ T} (r : ℕ → ℕ) (h : StrictMono r),
    mu k i τ T = mu k i τ (reindex k T r h) ∧
    nu k i τ T = nu k i τ (reindex k T r h) := by admit

-- F.3.7 Sufficient conditions (Appendix D, §D.3)
axiom S1_commutes (i : ℤ) (τ : ℝ≥0) (T : Tower k) :
  (colim (fun n ↦ (TτP k i τ).obj (T.obj n))) ≅ (TτP k i τ).obj T.apex

theorem mu_nu_vanish_of_S1 (i τ T) : mu k i τ T = 0 ∧ nu k i τ T = 0 := by
  admit

-- F.3.8 Edge identification PH₁ ⇒ Ext¹ (Appendix C)
def ℛ : FiltCh k ⥤ Derived k := by admit            -- t-exact, amplitude [-1,0]

theorem edge_iso (F : FiltCh k) :
  H^{-1} (ℛ.obj F) ≅ colim t, H_1 (F^t) := by admit  -- natural in F

theorem ext1_edge (A : Derived k) (hA : A ∈ D^{[-1,0]}) :
  Ext^1(A, (of k).obj k) ≅ Hom(H^{-1}(A), (of k).obj k) := by admit
end AK
\end{verbatim}

\subsection*{F.4. Coq sketch (\textsf{mathcomp}/\textsf{coq-cat-theory} style) [Spec]}
\begin{verbatim}
From mathcomp Require Import all_ssreflect all_algebra.
From CoqCT Require Import Category Abelian Functor Limits Colimits.
Set Implicit Arguments. Unset Strict Implicit. Unset Printing Implicit Defensive.

Module AK.

(* F.4.1 Categories and constructibility *)
Parameter k : fieldType.
Axiom Vect : AbelianCat.                  (* f.d. k-vector spaces *)
Axiom Rposet : PreOrder.                  (* (ℝ, ≤), schematic [Spec] *)
Definition Diag := FunctorCat Rposet Vect.
Parameter Constructible : Diag -> Prop.
Record Pers := { M : Diag; pfd : Constructible M }.  (* Pers^{ft}_k *)

(* F.4.2 Serre subcategory E_τ and reflector T_τ *)
Parameter tau : R.                        (* threshold *)
Axiom Eτ : SerreSubcat Pers.
Axiom Tτ : Functor Pers (Orthogonal Eτ).        (* exact, idempotent reflector *)
Axiom iotaτ : Functor (Orthogonal Eτ) Pers.     (* inclusion E_τ^⊥ ↪ Pers *)
Axiom Tτ_exact : ExactFunctor Tτ.
Axiom Tτ_idem  : FunctorComp Tτ Tτ ≅ Tτ.
Axiom Tτ_adj   : Adjunction Tτ iotaτ.
Axiom localization_equiv :
  SerreLocalization Pers Eτ (Orthogonal Eτ).    (* τ-local (orthogonal) *)

(* F.4.3 Interleaving metric, shifts, and non-expansiveness *)
Parameter dint : Pers -> Pers -> R.             (* ℝ≥0∞, schematic *)
Parameter Shift : R -> Functor Pers Pers.
Axiom shift_zero : Shift 0 ≅ Id.
Axiom shift_add  : forall e d, FunctorComp (Shift e) (Shift d) ≅ Shift (e + d).
Axiom shift_comm :
  forall eps, FunctorComp (Shift eps) Tτ ≅ FunctorComp Tτ (Shift eps).
Axiom Tτ_nonexpansive :
  forall (X Y : Pers), dint (Tτ X) (Tτ Y) <= dint X Y.

(* F.4.4 Towers, P_i ⋙ T_τ, and φ_{i,τ} *)
Parameter FiltCh : Type.
Parameter P_i : Z -> Functor FiltCh Pers.

Definition TτP (i : Z) : Functor FiltCh (Orthogonal Eτ) :=
  FunctorComp (P_i i) Tτ.                  (* ✅ order: persistence then truncation *)

Record Tower := {
  obj : nat -> FiltCh;
  mor : forall n, obj n ⟶ obj n.+1;
  apex : FiltCh;
  cocone : forall n, obj n ⟶ apex }.

Parameter iter_mor : forall (T : Tower) n m, n ≤ m -> obj T n ⟶ obj T m.

Parameter StrictMono : (nat -> nat) -> Prop.
Parameter StrictMono_step : forall r, StrictMono r -> forall n, r n < r (n.+1).

Definition reindex (T : Tower) (r : nat -> nat) (h : StrictMono r) : Tower :=
  {| obj := fun n => obj T (r n);
     mor := fun n =>
       let hlt := StrictMono_step _ h n in
       iter_mor T (r n) (r n.+1) (leq_trans (ltnW hlt) (leqnn _));
     apex := apex T;
     cocone := fun n => cocone T (r n) |}.

Record TowerHom (T T' : Tower) := {
  α : forall n, obj T n ⟶ obj T' n;
  square : forall n, mor T n ;; α (n.+1) = α n ;; mor T' n;
  cocone_comm : forall n, α n ;; cocone T' n = cocone T n }.

Parameter phi :
  forall (i : Z) (T : Tower),
    Colim (fun n => TτP i (obj T n)) ⟶ TτP i (apex T).

Axiom phi_natural :
  forall i (T T' : Tower) (h : TowerHom T T'),
    compose (phi i T') (colim_map h.(α)) = phi i T.

(* F.4.5 Generic fiber dimension and (μ,ν) *)
Axiom gdim_stabilizes :
  forall (X : Pers), exists T0 c, forall t, t ≥ T0 -> dim (M X t) = c.
Parameter gdim : Pers -> nat.  (* c from gdim_stabilizes *)
Axiom gdim_iso_invariant : forall X Y, X ≅ Y -> gdim X = gdim Y.

Definition mu i (T : Tower) := gdim (Ker (phi i T)).
Definition nu i (T : Tower) := gdim (Coker (phi i T)).

Theorem mu_nu_fqi_invariant :
  forall i T T', fqi_equiv T T' -> mu i T = mu i T' /\ nu i T = nu i T'.
Proof. admit. Qed.

Theorem mu_nu_cofinal_invariant :
  forall i T (r : nat -> nat), StrictMono r ->
    mu i T = mu i (reindex T r H) /\ nu i T = nu i (reindex T r H).
Proof. admit. Qed.

(* F.4.6 Sufficient conditions (Appendix D, §D.3) *)
Axiom S1_commutes : forall i T, IsIso (phi i T).
Theorem mu_nu_vanish_of_S1 i T :
  mu i T = 0 /\ nu i T = 0.
Proof. admit. Qed.

(* F.4.7 Edge identification PH1 ⇒ Ext1 (Appendix C) *)
Parameter Rfun : Functor FiltCh (DerivedCat Vect).
Axiom Rfun_t_exact : t_exact Rfun.  (* amplitude in [-1,0] *)
Axiom edge_iso :
  forall F, Hm1 (Rfun F) ≅ Colim_t (H1 (F^t)).
Axiom ext1_edge :
  forall A, in_amplitude A (-1) 0 ->
    Ext1 A (embed k) ≅ Hom (Hm1 A) (embed k).

End AK.
\end{verbatim}

\subsection*{F.5. Scope, exactness, and constructibility [Spec]}
\textbf{Scope rule.} All filtered (co)limit computations are performed \emph{objectwise} in
\([\mathbb{R},\mathsf{Vect}_k]\), where filtered colimits are exact; they are invoked only under
Appendix~A, Remark~\ref{A:rk:filtered-colimits}. Whenever the result might exit \(\Pers^{\mathrm{ft}}_k\),
we either (i) verify constructibility, or (ii) compute outside and \emph{return} via \(\mathbf{T}_\tau\)
or an explicit finite-type truncation. No claim is made outside this regime.

\subsection*{F.6. What is proved, what is assumed [Spec]}
\begin{itemize}
\item (\emph{Localization}) \(\mathsf{E}_\tau\) is hereditary Serre; the reflector \(\mathbf{T}_\tau\) exists, is exact,
idempotent, and induces \(\Pers^{\mathrm{ft}}_k/\mathsf{E}_\tau\simeq \mathsf{E}_\tau^\perp\) (\(\tau\)-local/orthogonal).
\item (\emph{Stability}) \(\mathbf{T}_\tau\) is \(1\)-Lipschitz for the interleaving metric; the proof proceeds via
shift functors and natural isomorphisms \(\mathrm{Shift}_\varepsilon\circ \mathbf{T}_\tau\simeq \mathbf{T}_\tau\circ \mathrm{Shift}_\varepsilon\) (Appendix~A, Prop.~\ref{A:prop:lipschitz}).
\item (\emph{Towers}) The comparison map \(\phi_{i,\tau}\) is functorial in the cocone; under any of (S1)–(S3)
of Appendix~D (commutation, no \(\tau\)-accumulation, Cauchy+compatibility) it is an isomorphism, hence
\((\mu,\nu)=(0,0)\). The indices \((\mu,\nu)\) are invariant under filtered quasi-isomorphisms and cofinal
reindexings. \emph{Finiteness:} bounded-in-degree complexes imply only finitely many nonzero \(\mu_{i,\tau},\nu_{i,\tau}\),
so \(\sum_i \mu_{i,\tau}\) and \(\sum_i \nu_{i,\tau}\) are finite.
\item (\emph{Bridge}) For \(F\in\mathsf{FiltCh}(k)\), \(\mathcal{R}\) is \(t\)-exact of amplitude \([-1,0]\);
\(H^{-1}(\mathcal{R}(F))\simeq \varinjlim_t H_1(F^t)\) and
\(\Ext^1(A,k)\simeq \Hom(H^{-1}(A),k)\) for \(A\in D^{[-1,0]}\), hence
\(\mathrm{PH}_1(F)=0\Rightarrow \Ext^1(\mathcal{R}(F),k)=0\) (Appendix~C).
\end{itemize}

\subsection*{F.7. Minimal test fixtures [Spec]}
Encode the toy towers of Appendix~D: pure cokernel (Type~IV), pure kernel (stationary summand with zero cone),
and the mixed direct sum. Verify that \(\mu,\nu\) match the \(\mathrm{gdim}\) of \(I[0,\infty)\) after \(\mathbf{T}_\tau\),
and include regressions:
(i) cofinal reindexing \(n\mapsto n+1\) (invariance of \((\mu,\nu)\));
(ii) (S1) case where \(\ker\phi_{i,\tau}\) and \(\mathrm{coker}\,\phi_{i,\tau}\) vanish;
(iii) small perturbations preserving the non-expansiveness of \(\mathbf{T}_\tau\).
Optionally add one instance each for (S2) (no \(\tau\)-accumulation) and (S3) (\(\mathbf{T}_\tau\)-Cauchy + compatible cocone).

\subsection*{F.8. Notes on libraries and portability [Spec]}
The Lean sketch targets \textsf{mathlib} (abelian categories, Serre subcategories, localization, derived categories).
The Coq sketch targets \textsf{mathcomp}+\textsf{coq-category-theory} (or \textsf{UniMath}). Nontrivial steps are
isolated behind \texttt{admit} with explicit references to Appendix~A–E. Replacing them by library lemmas yields
a complete development. For Lean’s subtype \texttt{Pers}, minor coercions may be required when using
\texttt{Interleaving.dist}; a thin wrapper or instance will resolve this in practice.
In Lean, define \(\phi_{i,\tau}\) via \texttt{Limits.colimit.desc} and use \texttt{colimit.hom\_ext} for naturality;
in Coq, use \texttt{Colim.desc}/\texttt{colim\_map} with a right-to-left \texttt{compose} convention.

% -------------------------
\subsection*{F.9. Windows (MECE), B-Gate\texorpdfstring{$^{+}$}{+}, $\delta$-commutation, Restart, Summability, and A/B Tests [Spec]}
% -------------------------

\paragraph{Lean~4 stubs.}
\begin{verbatim}
namespace AK

/-- Domain window (half-open, right-inclusion). -/
structure Window where
  u    : ℝ
  u'   : ℝ
  len_pos : u' > u

/-- MECE windowing over [u0, U). -/
structure Windowing where
  u0    : ℝ
  U     : ℝ
  cover : List Window
  disjoint : Pairwise (fun W1 W2 => W1 ≠ W2 → (W1.u' ≤ W2.u ∨ W2.u' ≤ W1.u))
  exact_cover : (∑ w in cover, (w.u' - w.u)) = (U - u0)
  right_inclusion : True    -- bookkeeping: half-open, right-inclusion

/-- Coverage check: global events equals sum of per-window events. -/
def coverage_ok (Ev_global : ℕ) (Ev_win : List ℕ) : Prop :=
  Ev_global = (Ev_win.foldl (·+·) 0)

/-- δ-ledger per step (algorithmic / discretization / measurement). -/
structure Delta where
  alg  : ℝ≥0
  disc : ℝ≥0
  meas : ℝ≥0
  sum  : ℝ≥0 := alg + disc + meas

/-- Additive pipeline budget on a window. -/
def budget (ds : List Delta) : ℝ≥0 := ds.foldl (·+·) 0

/-- B-Gate⁺ input bundle. -/
structure GateInput where
  tau     : ℝ≥0
  deg     : ℤ
  ph1zero : Bool
  ext1_ok : Bool      -- only if eligible (Appendix C)
  mu      : ℕ
  nu      : ℕ
  gap     : ℝ≥0
  dsum    : ℝ≥0

/-- B-Gate⁺ verdict. -/
def BGATE_plus (gin : GateInput) : Bool :=
  gin.ph1zero ∧ gin.ext1_ok ∧ (gin.mu = 0) ∧ (gin.nu = 0) ∧ (gin.gap > gin.dsum)

/-- δ-commutation contract at the persistence layer (uniform in F). -/
structure DeltaComm where
  delta : (ℤ → ℝ≥0 → ℝ≥0)    -- (i, τ) ↦ δ(i, τ)
  uniform_in_F : True
  additive     : True         -- pipeline subadditivity
  post_stable  : True         -- non-increasing under 1-Lipschitz post-processing

/-- A/B commutativity test with tolerance η. -/
def AB_soft_commuting
  (Delta_comm : Pers k → ℝ≥0) (eta : ℝ≥0) : Prop :=
  ∀ (M : Pers k), Delta_comm M ≤ eta

/-- Restart: next safety margin retains κ of the current margin minus budget. -/
def restart_ok (kappa : ℝ≥0) (gap_next gap_curr dsum : ℝ≥0) : Prop :=
  gap_next ≥ kappa * (gap_curr - dsum)

/-- Summability of budgets (per degree). -/
def summable_budget (budgets : ℕ → ℝ≥0) : Prop :=
  ∃ B : ℝ≥0, ∀ n, (∑ i in Finset.range n, budgets i) ≤ B

end AK
\end{verbatim}

\paragraph{Coq stubs.}
\begin{verbatim}
Module AK_MECE.

Record window := { u : R; u' : R; len_pos : (u' > u)%R }.

Record windowing := {
  u0 : R; U : R; cover : seq window;
  disjoint :
    all (fun p => (let: (w1, w2) := p in (~~ (w1 == w2))
                   ==> ( (w1.(u') <= w2.(u)) || (w2.(u') <= w1.(u)) ))) (all_pairs cover);
  exact_cover :
    \sum_(w <- cover) (w.(u') - w.(u)) = (U - u0);
  right_inclusion : True
}.

Definition coverage_ok (Ev_global : nat) (Ev_win : seq nat) :=
  Ev_global == \sum_(e <- Ev_win) e.

Record delta := { alg : R; disc : R; meas : R }.
Definition dsum (d : delta) : R := d.(alg) + d.(disc) + d.(meas).
Definition budget (ds : seq delta) : R := \sum_(d <- ds) (dsum d).

Record gate_input := {
  tau : R; deg : Z; ph1zero : bool; ext1_ok : bool;
  mu : nat; nu : nat; gap : R; dsum_win : R
}.

Definition BGATE_plus (g : gate_input) :=
  ph1zero g && ext1_ok g && (mu g == 0) && (nu g == 0) && (gap g > dsum_win g).

Record delta_comm := {
  delta : Z -> R -> R; (* (i, tau) ↦ δ(i, τ) *)
  uniform_in_F : True; additive : True; post_stable : True
}.

Definition AB_soft_commuting
  (Delta_comm : Pers -> R) (eta : R) : Prop :=
  forall M, (Delta_comm M <= eta)%R.

Definition restart_ok (kappa gap_next gap_curr dsum : R) :=
  (gap_next >= kappa * (gap_curr - dsum))%R.

Definition summable_budget (budgets : nat -> R) :=
  exists B, forall n, (\sum_(i < n) budgets i <= B)%R.

End AK_MECE.
\end{verbatim}

\subsection*{F.10. Logging, reproducibility, and test harness [Spec]}
A minimal test harness should expose:
\begin{itemize}
  \item \textbf{Windowing API:} constructors/checkers for MECE partitions and coverage; \texttt{collapse\_tau} policy per window; spectral bin policy \(([a,b],\beta)\).
  \item \textbf{Gate API:} a \texttt{BGATE\_plus} predicate that consumes PH1/Ext1/(\(\mu,\nu\))/safety\hyp margin/\(\delta\)-budget and yields a Boolean verdict per window/degree, with manifest fields (\texttt{ext1\_eligible}, \texttt{ext1\_used\_in\_gate}) as in Appendix~C.
  \item \textbf{\(\delta\)-commutation API:} a structure capturing \(\delta(i,\tau)\) with \emph{uniformity in \(F\)}, \emph{pipeline additivity}, and \emph{post\hyp processing stability}; metrized A/B tests with tolerance \(\eta\) and soft\hyp commuting/fallback logging (Appendix~B/L).
  \item \textbf{Restart/Summability:} verification routines for the inequality \(\mathrm{gap}_{k+1}\ge \kappa(\mathrm{gap}_k-\Sigma\delta_k)\) and the series \(\sum_k\Sigma\delta_k<\infty\) (Appendix~D).
  \item \textbf{Tower diagnostics:} wrappers for \(\phi_{i,\tau}\), \(\mu_{i,\tau},\nu_{i,\tau}\), stability-band detection across \(\tau\)-sweeps, and cofinal invariance checks.
  \item \textbf{Artifacts/manifest:} standardized JSON/YAML fields for windows, coverage checks, operations, \(\delta\)-ledger, persistence/spectral/aux-bars, gate verdicts, and seeds/versions (Appendix~G).
\end{itemize}

\subsection*{F.11. Outcome and completion note}
The stubs above, together with §§F.1–F.8, provide a complete formalization \textbf{Spec} for:
(i) the categorical spine (\(\mathbf{T}_\tau\), exactness, and \(1\)\hyp Lipschitz),
(ii) tower diagnostics \((\mu,\nu)\) with invariances and sufficient conditions,
(iii) the bridge \(\mathrm{PH}_1\Rightarrow \Ext^1\) under amplitude \(\le 1\),
(iv) the operational layer (MECE windows, B\hyp Gate\(^{+}\), \(\delta\)\hyp commutation, Restart/Summability, A/B tests), and
(v) reproducibility/logging contracts.
No further supplementation is required for mechanizing the v15.0 guard-rails in Lean/Coq; replacing \texttt{admit} by library lemmas or local developments yields a fully checked artifact consistent with the main text.



% =========================
\section*{Appendix G. Reproducibility: Logs and Schemas [Spec]}
% =========================
\phantomsection
\addcontentsline{toc}{section}{Appendix G. Reproducibility: Logs and Schemas}
\refstepcounter{section}
\label{G:repro}

This appendix specifies the provenance log (\texttt{run.yaml}) and the machine-readable
schemas for artifacts produced in this work—barcodes (\texttt{bars}), spectral indicators
(\texttt{spec}), Ext-tests (\texttt{ext}), and tower comparison maps (\texttt{phi}).
All files may be emitted in either JSON or HDF5; JSON keys coincide with HDF5
group/dataset names. Colimits are used only under the scope policy
(Appendix~A, Remark~\ref{A:rk:filtered-colimits}). Type labels follow \emph{Type I--II / Type III / Type IV}.
Cross-module conventions: the Ext-test is always against \(k[0]\), i.e.\ \(\Ext^1(\mathcal{R}(C_\tau F),k)=0\)
(with \(C_\tau\) understood up to f.q.i.\ on \(\Ho(\mathsf{FiltCh}(k))\)); the energy exponent satisfies
\(\alpha>0\) (default \(\alpha=1\)). Spectral monotonicity is asserted only for \emph{deletion-type} operations
(Dirichlet/principal/Loewner), with directions fixed by Appendix~E; inclusion-type operations
are used solely with stability bounds.

\paragraph{New in this version (2025-03-15).}
In addition to the original specification, \textbf{this version} extends \texttt{run.yaml} with:
\emph{(i)} \texttt{windows} (domain/collapse/spectral), \emph{(ii)} \texttt{coverage\_check},
\emph{(iii)} \texttt{operations} (with \texttt{U}, \texttt{type}, \(\tau\), and \(\delta\) breakdown),
\emph{(iv)} \texttt{persistence} (summaries: \texttt{PH1\_zero}/\texttt{Ext1\_zero}/\(\mu\)/\(\nu\)/\texttt{phi\_iso\_tail}),
\emph{(v)} \texttt{spectral.aux\_bars\_remaining}, \emph{(vi)} \texttt{budget} (\texttt{sum\_delta}/\texttt{safety\_margin}),
and \emph{(vii)} \texttt{gate.accept}. We also add optional but recommended fields for containerization,
timing, and failure capture to strengthen reproducibility.

\subsection*{G.1. Provenance, determinism, and gating}
Each run records (i) source/inputs, (ii) algorithmic choices and thresholds, (iii) numeric
tolerances and units, (iv) code/environment fingerprints, (v) RNG details, (vi) strong identifiers
(content hashes) for all artifacts, and (vii) \emph{gating} decisions that determine acceptance of results.
Randomness is controlled by explicit seeds. Floating-point claims report both an \emph{asserted} tolerance
and a \emph{measured} slack. \emph{Windows} (domain/collapse/spectral) must be declared, and a
\emph{coverage check} attests that all measured quantities fall inside their stated windows.
A \emph{budget} aggregates operation-level error contributions \(\delta\) and yields a
\emph{safety margin} relative to the governing tolerance; finally, \texttt{gate.accept} records the run-level
decision (accept/reject) together with reasons.

\subsection*{G.2. \texttt{run.yaml} schema (versions, units, windows, operations, budget, gate)}
\paragraph{Intent.} A single file per execution, sufficient to reproduce the pipeline end-to-end, including
all windows, coverage checks, operation logs, and the final acceptance gate.

\noindent\textbf{Canonical layout (YAML).}
\begin{verbatim}
version: 1
schema_version: "2025-03-15"
run_id: "2025-03-15T09:12:07Z-7f5c1b1"
seed: 1337
rng:
  python: "default_rng"
  numpy:  "PCG64"
platform:
  os: "Ubuntu 22.04"
  cpu: "Intel(R) Xeon(R) Platinum 8370C"
  cuda: "12.2"
  blas: "OpenBLAS 0.3.23"
  hdf5: "1.14.3"          # HDF5 library version (mandatory)
  lapack: "OpenBLAS-LAPACK"
  glibc: "2.35"
  kernel: "5.15.0-105"
  locale: "C.UTF-8"
env:
  python: "3.11.7"
  packages:
    numpy: "1.26.4"
    scipy: "1.13.1"
    h5py:  "3.10.0"
    networkx: "3.2.1"
  threads:
    OMP_NUM_THREADS: 1
    MKL_NUM_THREADS: 1
    OPENBLAS_NUM_THREADS: 1
container:
  image: "docker.io/example/persistence:2025.03"
  digest: "sha256:deadbeef..."    # content-addressed image digest
git:
  repo: "git@host:ak/persistence.git"
  commit: "a1b2c3d4"
units:
  filtration: "dimensionless"     # e.g. "height", "time"
  eigenvalues: "dimensionless"    # e.g. "1/length^2" for Laplacians
windows:
  domain:
    filtration_range: [0.0, 2.0]  # units: units.filtration; closed interval semantics
    degrees: [0, 2]               # inclusive range of homological degrees considered
  collapse:
    tau_sweep: [0.25, 0.50, 1.00] # list of τ values evaluated (subset may be used)
  spectral:
    range: [0.0, 2.0]             # units: units.eigenvalues; closed interval semantics
    order: "ascending"            # expected storage order for eigen arrays
coverage_check:
  domain_window_covers_bars: true
  spectral_window_covers_thetas: true
  collapse_tau_sweep_covers_reports: true
inputs:
  dataset: "AK-bench-v3"
  graphs:
    - path: "data/G_001.edgelist"
      hash: "sha256:..."
  filters:
    type: "height"
    params: { axis: 2 }
pipeline:
  metric: "interleaving"          # or "bottleneck" (exactly one)
  stages:
    - name: "barcode"
      params: { field: "k", reduction: "clearing" }
    - name: "collapse"            # C_τ (up to f.q.i.)
      params: { tau: 0.50 }
    - name: "spec"
      params:
        window: [0.0, 2.0]        # reporting window; closed interval semantics
        norm: "fro"               # "fro" ≡ ‖·‖_fro, "op" ≡ ‖·‖_op (Appendix E)
        order: "ascending"        # storage order for eigen arrays
        clip: 1.00                # clip acts on eigenvalues; window gates reporting
        loewner_assumption: "A'⪯A"   # enum: "A'⪯A" | "A'⪰A" | "none" (Appendix E)
        eig_solver:
          method: "lanczos"
          k: 128
          maxiter: 1000
          tol: 1e-12
          reorthogonalize: true
          rng_seed: 1337
    - name: "ext-test"            # Ext^1(R(C_τ F), k)
      params: { amplitude_check: true }
operations:
  - step: 1
    U: [0,1,3]                    # index set or label of affected subset
    type: "inclusion"             # enum: inclusion|projection|quasi_iso|
                                  #       filtration_preserving_map|schur_complement|other
    tau: 0.50
    delta:
      distance:
        interleaving: 0.050       # present iff pipeline.metric == "interleaving"
      sources:
        discretization: 0.030
        rounding: 1.0e-12
        heuristic: 0.020
      total: 0.050
      note: "Edge contraction in subgraph U"
  - step: 2
    U: "V\\W"
    type: "schur_complement"
    tau: 0.50
    delta:
      distance:
        interleaving: 0.100
      sources:
        elimination: 0.080
        rounding: 0.020
      total: 0.100
persistence:
  PH1_zero: true                  # H_1 extinguished in the reported window
  Ext1_zero: true                 # Ext^1 vanishes (as reported in ext artifact)
  mu: 1                           # generic kernel dimension after truncation
  nu: 0                           # generic cokernel dimension after truncation
  phi_iso_tail:
    passed: false
    i: 1
    tau: 0.50
spectral:
  aux_bars_remaining: 0           # count of barcode intervals not consumed by spectral mapping
thresholds:
  alpha: 1.0                      # α>0 (default α=1)
  tol:
    distance:
      interleaving: 1e-6          # present iff pipeline.metric == "interleaving"
      # bottleneck: 1e-6          # present iff pipeline.metric == "bottleneck"
    eig: 1e-8
    witness: 1e-9
budget:
  sum_delta:
    distance:
      interleaving: 0.150         # sum over operations[*].delta.distance.interleaving
  safety_margin: 0.850            # 1 - sum_delta / thresholds.tol.distance.interleaving
  rationale: "All deltas accounted for; slack remains >0"
gate:
  accept: true
  reason: "Coverage ok; safety margin positive; all assertions satisfied"
serialization:
  float_dtype: "ieee754-f64-le"
  json_sort_keys: true
  hdf5_canonical:
    compression: { algo: "gzip", level: 4 }
    shuffle: false
    fletcher32: false
    track_times: false
    fillvalue: 0.0
    string_encoding: "utf8-fixed"     # fixed-length UTF-8 strings for bitwise reproducibility
    chunk_shapes:
      bars: { i: 4096, birth: 4096, death: 4096, death_is_inf: 4096, mult: 4096 }
      spec_eigs: { eig: 4096 }
      spec_Ntheta: { theta: 512, left: 512, right: 512 }
      phi_idx: { i: 256, tau: 256, iso: 256, mu: 256, nu: 256 }
cache:
  enabled: true
  dir: ".cache/run_7f5c1b1"
timing:
  wallclock_s: 123.4
  cpu_s: 456.7
  stages:
    barcode: 12.3
    collapse: 4.5
    spec: 80.0
    ext_test: 2.0
status:
  success: true
  errors: []
outputs:
  bars: "out/bars_7f5c1b1.json"
  spec: "out/spec_7f5c1b1.json"
  ext:  "out/ext_7f5c1b1.json"
  phi:  "out/phi_7f5c1b1.h5"
\end{verbatim}

\noindent\textbf{JSON Schema (abridged; enforces metric–tolerance coupling, units, windows, operations, budget, and enums).}
\begin{verbatim}
{
  "type":"object",
  "required":[
    "version","schema_version","run_id","seed","platform",
    "env","git","units","windows","coverage_check","inputs",
    "pipeline","operations","persistence","spectral","thresholds",
    "budget","gate","serialization","outputs"
  ],
  "properties":{
    "version":{"type":"integer","enum":[1]},
    "schema_version":{"type":"string"},
    "run_id":{"type":"string"},
    "seed":{"type":"integer"},
    "rng":{"type":"object"},
    "platform":{"type":"object","required":["os","cpu","blas","hdf5"]},
    "env":{"type":"object"},
    "container":{"type":"object",
      "properties":{"image":{"type":"string"},"digest":{"type":"string"}}},
    "git":{"type":"object","required":["repo","commit"]},
    "units":{"type":"object",
      "properties":{"filtration":{"type":"string"},"eigenvalues":{"type":"string"}},
      "required":["filtration","eigenvalues"]},
    "windows":{"type":"object","required":["domain","collapse","spectral"],
      "properties":{
        "domain":{"type":"object","required":["filtration_range","degrees"],
          "properties":{
            "filtration_range":{"type":"array","items":{"type":"number"},"minItems":2,"maxItems":2},
            "degrees":{"type":"array","items":{"type":"integer"},"minItems":2,"maxItems":2}
          }},
        "collapse":{"type":"object",
          "properties":{"tau_sweep":{"type":"array","items":{"type":"number"}}},
          "required":["tau_sweep"]},
        "spectral":{"type":"object","required":["range","order"],
          "properties":{
            "range":{"type":"array","items":{"type":"number"},"minItems":2,"maxItems":2},
            "order":{"type":"string","enum":["ascending","descending"]}
          }}
      }},
    "coverage_check":{"type":"object",
      "properties":{
        "domain_window_covers_bars":{"type":"boolean"},
        "spectral_window_covers_thetas":{"type":"boolean"},
        "collapse_tau_sweep_covers_reports":{"type":"boolean"}
      },
      "required":["domain_window_covers_bars","spectral_window_covers_thetas",
                  "collapse_tau_sweep_covers_reports"]},
    "inputs":{"type":"object"},
    "pipeline":{"type":"object",
      "properties":{
        "metric":{"type":"string","enum":["interleaving","bottleneck"]},
        "stages":{"type":"array","minItems":1}}},
    "operations":{"type":"array","items":{
      "type":"object",
      "properties":{
        "step":{"type":"integer"},
        "U":{"oneOf":[{"type":"array","items":{"type":"integer"}},{"type":"string"}]},
        "type":{"type":"string",
          "enum":["inclusion","projection","quasi_iso","filtration_preserving_map",
                  "schur_complement","other"]},
        "tau":{"type":"number"},
        "delta":{"type":"object","properties":{
          "distance":{"type":"object",
            "properties":{"interleaving":{"type":"number"},"bottleneck":{"type":"number"}}},
          "sources":{"type":"object"},
          "total":{"type":"number"},
          "note":{"type":"string"}
        },"required":["total"]}}
      },
      "required":["step","type","tau","delta"]
    }},
    "persistence":{"type":"object",
      "properties":{
        "PH1_zero":{"type":"boolean"},
        "Ext1_zero":{"type":"boolean"},
        "mu":{"type":"integer","minimum":0},
        "nu":{"type":"integer","minimum":0},
        "phi_iso_tail":{"type":"object",
          "properties":{"passed":{"type":"boolean"},"i":{"type":"integer"},"tau":{"type":"number"}},
          "required":["passed"]}
      },
      "required":["PH1_zero","Ext1_zero","mu","nu","phi_iso_tail"]},
    "spectral":{"type":"object",
      "properties":{"aux_bars_remaining":{"type":"integer","minimum":0}},
      "required":["aux_bars_remaining"]},
    "thresholds":{"type":"object",
      "properties":{
        "alpha":{"type":"number","exclusiveMinimum":0},
        "tol":{"type":"object","properties":{
          "distance":{"type":"object"}, "eig":{"type":"number"}, "witness":{"type":"number"}},
          "required":["distance","eig","witness"]}},
      "required":["alpha","tol"]},
    "budget":{"type":"object",
      "properties":{
        "sum_delta":{"type":"object",
          "properties":{"distance":{"type":"object"}},
          "required":["distance"]},
        "safety_margin":{"type":"number"},
        "rationale":{"type":"string"}
      },
      "required":["sum_delta","safety_margin"]},
    "gate":{"type":"object",
      "properties":{"accept":{"type":"boolean"},"reason":{"type":"string"}},
      "required":["accept"]},
    "serialization":{"type":"object",
      "properties":{
        "float_dtype":{"type":"string","enum":["ieee754-f64-le"]},
        "json_sort_keys":{"type":"boolean"},
        "hdf5_canonical":{"type":"object",
          "required":["compression","shuffle","fletcher32","track_times",
                      "fillvalue","string_encoding","chunk_shapes"],
          "properties":{"string_encoding":{"type":"string","const":"utf8-fixed"}}}},
      "required":["float_dtype","json_sort_keys","hdf5_canonical"]},
    "cache":{"type":"object"},
    "timing":{"type":"object"},
    "status":{"type":"object",
      "properties":{"success":{"type":"boolean"},"errors":{"type":"array","items":{"type":"string"}}}},
    "outputs":{"type":"object","required":["bars","spec","ext","phi"]}
  },
  "allOf":[
    {"if":{"properties":{"pipeline":{"properties":{"metric":{"const":"interleaving"}}}}},
     "then":{"properties":{
       "thresholds":{"properties":{"tol":{"properties":{
         "distance":{"type":"object","required":["interleaving"]}}}}}},
       "budget":{"properties":{"sum_delta":{"properties":{
         "distance":{"type":"object","required":["interleaving"]}}}}}
     }}},
    {"if":{"properties":{"pipeline":{"properties":{"metric":{"const":"bottleneck"}}}}},
     "then":{"properties":{
       "thresholds":{"properties":{"tol":{"properties":{
         "distance":{"type":"object","required":["bottleneck"]}}}}}},
       "budget":{"properties":{"sum_delta":{"properties":{
         "distance":{"type":"object","required":["bottleneck"]}}}}}
     }}}
  ]
}
\end{verbatim}

\paragraph{Semantics of new fields.}
\begin{itemize}
\item \textbf{\texttt{windows}} fix the measurement scope: \texttt{domain.filtration\_range} (in \texttt{units.filtration}),
      \texttt{domain.degrees} (inclusive range of homological degrees), \texttt{collapse.tau\_sweep} (evaluated \(\tau\)),
      and \texttt{spectral.range} (in \texttt{units.eigenvalues}) with the expected eigenvalue storage \texttt{order}.
\item \textbf{\texttt{coverage\_check}} records that actual query points (birth/death coordinates, \(\theta\)'s, \(\tau\)'s)
      lie inside their respective windows. Violations should set the booleans to \texttt{false} and document in \texttt{status.errors}.
\item \textbf{\texttt{operations}} log each structural modification with the affected index set \texttt{U}, the operation \texttt{type},
      the \(\tau\) at which it is interpreted, and a \(\delta\)-\emph{breakdown} partitioned into \texttt{sources} (free-form key/value)
      summing to \texttt{total}. If a persistence distance is in force, encode the contribution under \texttt{delta.distance.*}.
\item \textbf{\texttt{persistence}} summarizes key outcomes used later: whether \( \mathrm{PH}_1 \) vanishes in the window,
      whether \( \Ext^1 \) vanishes (as asserted by the Ext-test), the generic fiber dimensions \((\mu,\nu)\) after truncation,
      and whether a \emph{tail} segment of the tower comparison \(\phi_{i,\tau}\) is an isomorphism (\texttt{phi\_iso\_tail}).
\item \textbf{\texttt{spectral.aux\_bars\_remaining}} counts residual barcode intervals that were not consumed by spectral
      diagnostics under the declared window (helps detect incomplete spectral coverage).
\item \textbf{\texttt{budget}} aggregates \(\delta\) across \texttt{operations} into \texttt{sum\_delta}, then computes a
      \texttt{safety\_margin} relative to the governing \texttt{thresholds.tol.distance.*}. Positive margin implies
      the budget remains inside tolerance.
\item \textbf{\texttt{gate.accept}} is the final decision bit for the run. It should be \texttt{true} only if coverage checks pass,
      computed slacks are nonnegative, and the budget’s safety margin is positive; the \texttt{reason} field explains the decision.
\item \textbf{\texttt{container}}, \texttt{timing}, and \texttt{status} improve portability and diagnosis (recommended).
\end{itemize}

\subsection*{G.3. \texttt{bars} (barcodes) schema}
\paragraph{Semantics.} A constructible barcode is a multiset of half-open intervals \(I=[b,d)\) with degree \(i\).
Deaths may be \(+\infty\).

\noindent\textbf{JSON layout (with infinity convention, units, and cross-link).}
\begin{verbatim}
{
  "meta": {
    "schema_version": "2025-03-15",
    "field": "k",
    "filtration_units": "dimensionless",
    "endpoint_convention": "[b,d) (see Chapter~2)",
    "infinity": { "json": "inf" },
    "float_dtype": "ieee754-f64-le",
    "string_encoding": "utf8-fixed",
    "links": {
      "run_id": "2025-03-15T09:12:07Z-7f5c1b1",
      "run_yaml_hash": "sha256:...run"
    }
  },
  "bars": [
    { "i": 0, "birth": 0.0, "death": 0.3,   "mult": 1 },
    { "i": 1, "birth": 0.2, "death": "inf", "mult": 1 }
  ],
  "hash": "sha256:...bars"
}
\end{verbatim}

\noindent\textbf{JSON Schema snippet for \texttt{bars.bars[*].death}.}
\begin{verbatim}
"bars":{"type":"array","items":{
  "type":"object",
  "properties":{
    "i":{"type":"integer"},
    "birth":{"type":"number"},
    "death":{"oneOf":[{"type":"number"},{"type":"string","enum":["inf"]}]},
    "mult":{"type":"integer","minimum":1}
  },
  "required":["i","birth","death","mult"]
}}
\end{verbatim}

\noindent\textbf{HDF5 layout (split representation; no numeric sentinel).}
\begin{itemize}[leftmargin=*, itemsep=0.1em]
  \item Datasets:
    \path{/bars/i} (\texttt{int32}),
    \path{/bars/birth} (\texttt{float64}),
    \path{/bars/death} (\texttt{float64}),
    \path{/bars/death_is_inf} (\texttt{bool}),
    \path{/bars/mult} (\texttt{int32}).
  \item Attributes:
    \path{/bars}.attrs[\texttt{field}=\texttt{"k"}],
    \texttt{filtration\_units},
    \texttt{schema\_version},
    \texttt{float\_dtype},
    \texttt{death\_encoding}=\texttt{"split\_scalar\_bool"},
    \texttt{string\_encoding}=\texttt{"utf8-fixed"}.
\end{itemize}

\noindent\textbf{Invariants.} Local finiteness on bounded windows (Appendix~A). Optional sorting by \((i,b,d)\);
multiplicities aggregate identical intervals. Report-window compliance is checked via
\nolinkurl{run.yaml.coverage_check.domain_window_covers_bars}.

\subsection*{G.4. \texttt{spec} (spectral indicators) schema}
\paragraph{Semantics.} Spectral features: clipped sums, counts above/below thresholds with left/right limits,
and deletion-type monotonicity diagnostics. Matrices are identified by content hashes.

\noindent\textbf{JSON layout (ascending storage; \(N_{\theta\pm 0}\); solver params; coverage; cross-link).}
\begin{verbatim}
{
  "meta": {
    "schema_version": "2025-03-15",
    "eigen_units": "dimensionless",
    "order": "ascending",           // storage order
    "sorted": true,                 // eigs verified nondecreasing
    "Ntheta_convention": { "left": "N_{θ-0}", "right": "N_{θ+0}" },
    "window": { "range": [0.0, 2.0], "semantics": "closed" },
    "norm": "fro",                  // "fro" ≡ ‖·‖_fro, "op" ≡ ‖·‖_op (Appendix E)
    "clip_tau": 1.0,                // clips eigenvalues; window gates reporting
    "tol_eig": 1e-8,
    "loewner_assumption": "A'⪯A",  // enum: "A'⪯A" | "A'⪰A" | "none"
    "aux_bars_remaining": 0,        // mirrors run.yaml.spectral.aux_bars_remaining
    "coverage_check": { "thetas_in_window": true },
    "string_encoding": "utf8-fixed",
    "eig_solver": {
      "method": "lanczos", "k": 128, "maxiter": 1000,
      "tol": 1e-12, "reorthogonalize": true, "rng_seed": 1337
    },
    "links": {
      "run_id": "2025-03-15T09:12:07Z-7f5c1b1",
      "run_yaml_hash": "sha256:...run"
    }
  },
  "operators": [
    {
      "id": "sha256:...A",
      "kind": "laplacian_dirichlet",
      "n": 500,
      "spectrum": { "eigs": [0.10, 0.12, 0.45, ...] },  // Dirichlet: min eigenvalue > 0
      "clip":     { "tau": 1.0, "sum": 37.219, "l1": 37.219 }
    },
    {
      "id": "sha256:...B",
      "kind": "principal_submatrix",
      "parent": "sha256:...A",
      "N_theta": [
        { "theta": 0.20, "left": 17, "right": 16 },
        { "theta": 0.50, "left": 10, "right": 10 }
      ],
      "monotonicity": { "type": "deletion", "passed": true }
    }
  ],
  "hash": "sha256:...spec"
}
\end{verbatim}

\noindent\textbf{HDF5 layout.}
\begin{itemize}
  \item \path|/spec/ops/{id}/eig| (float64, ascending),
        \path|/spec/ops/{id}/clip/sum| (float64),
        \path|/spec/ops/{id}/Ntheta/theta|, \path|/spec/ops/{id}/Ntheta/left|, \path|/spec/ops/{id}/Ntheta/right|
        (parallel datasets).
  \item Attributes: \texttt{kind}, \texttt{parent}, \texttt{norm}
        \( \in \{ \text{\texttt{"fro"}}, \text{\texttt{"op"}} \} \) with mapping to Appendix~E,
        \texttt{order}=\texttt{"ascending"}, \texttt{sorted} (bool), \texttt{eigen\_units}, \texttt{tol\_eig},
        \texttt{schema\_version}, \texttt{loewner\_assumption}
        \( \in \{ \text{\texttt{"A' $\,\preceq\,$ A"}}, \text{\texttt{"A' $\,\succeq\,$ A"}}, \text{\texttt{"none"}} \} \),
        \texttt{aux\_bars\_remaining} (int32), \texttt{coverage\_thetas\_in\_window} (bool),
        \texttt{string\_encoding}=\texttt{"utf8-fixed"},
        and a nested \texttt{eig\_solver} group mirroring the JSON.
\end{itemize}

\subsection*{G.5. \texttt{ext} (Ext-test) schema}
\paragraph{Semantics.} Outcome of the one-way bridge \(\mathrm{PH}_1\Rightarrow \Ext^1\) for \(C_\tau F\), with
amplitude checks for \(\mathcal{R}\) and recorded assumptions.

\noindent\textbf{JSON layout (with assumptions and cross-links).}
\begin{verbatim}
{
  "meta": {
    "schema_version": "2025-03-15",
    "field":"k", "alpha": 1.0,
    "assumptions": {
      "field_is_k": true,
      "constructible_verified": true,
      "t_exact_and_amp_le_1": true
    },
    "string_encoding": "utf8-fixed",
    "links": {
      "run_id": "2025-03-15T09:12:07Z-7f5c1b1",
      "run_yaml_hash": "sha256:...run"
    }
  },
  "tau": 0.50,
  "amplitude": { "ok": true, "range": [-1, 0] },
  "Hminus1": { "dim": 0, "witness_norm": 0.0 },
  "Ext1":    { "dim": 0, "passed": true, "tol": 1e-9, "slack": 0.0 },
  "links":   { "bars": "sha256:...bars", "phi": "sha256:...phi" },
  "hash": "sha256:...ext"
}
\end{verbatim}

\noindent\textbf{HDF5 layout.}
\begin{itemize}
\item Scalars: \texttt{/ext/tau} (float64), \texttt{/ext/Hminus1/dim} (int32),
\texttt{/ext/Ext1/dim} (int32), \texttt{/ext/Ext1/passed} (bool), \texttt{/ext/Ext1/tol} (float64),
\texttt{/ext/Ext1/slack} (float64).
\item Attributes: \texttt{field}=\texttt{"k"}, \texttt{alpha}=float64, \texttt{schema\_version},
\texttt{assumptions/*} (bools as above), \texttt{string\_encoding}=\texttt{"utf8-fixed"}.
\end{itemize}

\subsection*{G.6. \texttt{phi} (tower comparison) schema}
\paragraph{Semantics.} Encodes \(\phi_{i,\tau}\) for towers, together with \((\mu,\nu)\) as generic-fiber
dimensions after truncation, and structural flags for the sufficiency hypotheses (Appendix~D, §D.3).

\noindent\textbf{JSON layout (with \(\tau\) sweep, edge kinds, witnesses, iso-tail, cross-link).}
\begin{verbatim}
{
  "meta": {
    "schema_version": "2025-03-15",
    "definition": "phi_{i,τ}: colim T_τ P_i(F_n) → T_τ P_i(F_∞)",
    "scope": "colim in [ℝ,Vect_k], return-to-constructible policy",
    "tau_sweep": [0.25, 0.50, 1.00],
    "edge_kinds": ["inclusion","projection","quasi_iso",
                   "filtration_preserving_map","schur_complement","other"],
    "string_encoding": "utf8-fixed",
    "links": {
      "run_id": "2025-03-15T09:12:07Z-7f5c1b1",
      "run_yaml_hash": "sha256:...run"
    }
  },
  "indices": [
    {
      "i": 1, "tau": 0.50,
      "iso": false,
      "mu": 1, "nu": 0,
      "flags": { "S1_commutes": false, "S2_noAccum": true, "S3_Cauchy": false },
      "witness": { "ker_generic_dim": 1, "coker_generic_dim": 0 },
      "iso_tail": { "passed": false }   // mirrors run.yaml.persistence.phi_iso_tail
    }
  ],
  "tower": {
    "nodes": [
      { "n": 0, "id": "sha256:...F0" },
      { "n": 1, "id": "sha256:...F1" }
    ],
    "edges": [
      { "src": 0, "dst": 1, "kind": "inclusion" }
    ],
    "limit": { "id": "sha256:...Finf" }
  },
  "hash": "sha256:...phi"
}
\end{verbatim}

\noindent\textbf{HDF5 layout.}
\begin{itemize}
\item \texttt{/phi/idx/i} (int32), \texttt{/phi/idx/tau} (float64),
\texttt{/phi/idx/iso} (bool), \texttt{/phi/idx/mu} (int32), \texttt{/phi/idx/nu} (int32).
\item \texttt{/phi/idx/flags}: \texttt{S1\_commutes}, \texttt{S2\_noAccum}, \texttt{S3\_Cauchy} (bool).
\item Optional witnesses: \texttt{/phi/idx/ker\_generic\_dim}, \texttt{/phi/idx/coker\_generic\_dim}.
\item Optional tail: \texttt{/phi/idx/iso\_tail/passed} (bool).
\item Optional tower edges: \texttt{/phi/tower/edges/src,dst} (int32),
\texttt{/phi/tower/edges/kind} (fixed-length UTF-8 string; attribute \texttt{string\_encoding}=\dq utf8-fixed\dq).
\item Attributes: \texttt{schema\_version}, \texttt{string\_encoding}=\texttt{"utf8-fixed"},
\texttt{tau\_sweep} (float64 array).
\end{itemize}

\subsection*{G.7. Content hashing and serialization/canonicalization}
Each artifact carries a content hash \texttt{sha256:...} over its canonical serialization
(JSON with sorted keys; HDF5 with \emph{fixed} dataset/attribute creation order, chunk shapes,
compression and filters). All floating datasets are little-endian IEEE-754 double (float64).
Cross-file links (\texttt{bars} \(\leftrightarrow\) \texttt{phi} \(\leftrightarrow\) \texttt{ext} \(\leftrightarrow\) \texttt{spec})
use these hashes exclusively.
\emph{JSON numeric policy:} finite numbers only; positive infinity is encoded as the string
\texttt{"inf"} where applicable (see \texttt{bars.meta.infinity}). HDF5 encodes \(+\infty\) via the
\emph{split representation} \texttt{/bars/death} (float64) + \texttt{/bars/death\_is\_inf} (bool).
\emph{Strings:} all JSON strings and HDF5 string datasets/attributes are \emph{fixed-length} UTF-8
(\texttt{string\_encoding}=\texttt{"utf8-fixed"}) to ensure bitwise reproducibility.
\emph{HDF5 canonicalization:} for bit-wise reproducibility, set \texttt{track\_times=false},
\texttt{shuffle=false}, \texttt{fletcher32=false}, \texttt{fillvalue=0.0}, compression to GZIP level~4,
and use the \texttt{chunk\_shapes} recorded in \texttt{run.yaml}; create datasets and attributes in the
order shown in this appendix.

\subsection*{G.8. Numeric tolerances, budgets, and audit trail}
Every quantitative claim includes:
{\setlength{\emergencystretch}{2em}%
\begin{itemize}
\item \textbf{tolerance} (\texttt{tol}) declared in \texttt{run.yaml};
\item \textbf{slack} (\texttt{slack}) measured margin to the decision boundary;
\item \textbf{norm} used for the bound (\texttt{fro} or \texttt{op}), consistent with Appendix~E
(\texttt{fro} \(\equiv\) \(\|\cdot\|_{\mathrm{fro}}\), \texttt{op} \(\equiv\) \(\|\cdot\|_{\mathrm{op}}\));
\item \textbf{metric} consistency for persistence distances (\texttt{interleaving} or \texttt{bottleneck}),
with \texttt{thresholds.\allowbreak tol.\allowbreak distance} carrying the matching key exactly once;
\item \textbf{budget} aggregation: \texttt{operations[*].delta} contributions are summed into
\texttt{budget.sum\_delta}, and \texttt{budget.safety\_margin} is computed against the governing tolerance;
\item \textbf{solver} details for spectral computations (Lanczos parameters, RNG seed);
\item \textbf{windows and coverage}: \texttt{windows.*} declare scopes; \texttt{coverage\_check.*} record pass/fail;
\item \textbf{threading} environment (\texttt{OMP\_\allowbreak NUM\_\allowbreak THREADS}, 
\texttt{MKL\_\allowbreak NUM\_\allowbreak THREADS},
\texttt{OPENBLAS\_\allowbreak NUM\_\allowbreak THREADS}) and BLAS/LAPACK/HDF5 versions;
\item \textbf{gate} decision: \texttt{gate.accept} with \texttt{gate.reason} summarizes acceptance criteria.
\end{itemize}
}

\subsection*{G.9. Minimal reproducibility checklist}
\begin{enumerate}
\item Preserve \texttt{run.yaml} and all emitted \texttt{bars/spec/ext/phi} files (JSON or HDF5).
\item Confirm \(\alpha>0\) (default \(\alpha=1\)) and field \(k\) are consistent across files.
\item Verify content hashes and all cross-links resolve; each artifact should carry \texttt{meta.links.run\_id}
and \texttt{run\_yaml\_hash}.
\item Check that \texttt{pipeline.metric} matches \texttt{thresholds.tol.distance} (exactly one of
\texttt{interleaving}/\texttt{bottleneck} present), and that norms/tolerances are consistent.
\item Verify declared \textbf{windows} (\texttt{windows.domain/spectral/collapse}) and that \texttt{coverage\_check.*} is true.
\item Verify eigenvalue \emph{order} metadata: \texttt{spec.meta.order}=\texttt{"ascending"},
\texttt{spec.meta.sorted}=true, and HDF5 eigen arrays are non-decreasing.
\item For Dirichlet Laplacians (\texttt{kind}=\texttt{"laplacian\_dirichlet"}), confirm the minimal eigenvalue is \(>0\).
\item Ensure each recorded \(\theta\) used for \(N_\theta\) lies within \texttt{spec.meta.window.range}
(or document otherwise); recompute spectral indicators using the recorded norm/tolerance/solver settings,
and check slack \(\ge 0\). Compare \texttt{spec.meta.aux\_bars\_remaining} with \texttt{run.yaml.spectral.aux\_bars\_remaining}.
\item Re-evaluate \(\phi_{i,\tau}\) under the scope policy; confirm \((\mu,\nu)\) match generic-fiber counts
(\(\mathrm{gdim}\) as in Appendix~D, Remark~\ref{rem:D-generic-dim}); check \texttt{iso\_tail.passed} and
\texttt{run.yaml.persistence.phi\_iso\_tail}.
\item For Ext-tests, verify amplitude \([-1,0]\), the assumptions flags in \texttt{ext.meta.assumptions},
and confirm \(\Ext^1(\mathcal{R}(C_\tau F),k)=0\) entries; check \texttt{run.yaml.persistence.Ext1\_zero}.
\item Validate HDF5 canonicalization: chunk shapes, compression, filters, \texttt{string\_encoding="utf8-fixed"},
and \texttt{track\_times=false} match \texttt{run.yaml.hdf5\_canonical}; death is represented via the split
float/bool fields (no numeric sentinel).
\item Inspect \texttt{operations} and \texttt{budget}: the sum of per-operation \(\delta\) must match
\texttt{budget.sum\_delta}; compute the safety margin and verify \texttt{gate.accept} consistency.
\item Record container details (image \& digest) and timing; if unavailable, capture OS and library versions precisely.
\end{enumerate}

\medskip
\noindent\textbf{Outcome.}
The versioned schemas above—now with (i) collision-free \(+\infty\) handling in HDF5,
(ii) explicit HDF5 canonicalization parameters and fixed-length UTF-8 strings, (iii) recorded bridge/constructibility
assumptions for Ext-tests, (iv) \(\tau\)-sweep support for towers, (v) \emph{Dirichlet} spectra respecting
\(\lambda_{\min}>0\), (vi) norm labels consistent with Appendix~E, (vii) an explicit enum for the Loewner orientation,
and additionally (viii) \emph{windows} with \emph{coverage checks}, (ix) an \emph{operations} log with \(\delta\)-breakdowns,
(x) a run-level \emph{budget} and \emph{safety margin}, (xi) \emph{persistence} and \emph{spectral} summaries for quick audits,
and (xii) a final \emph{gate} decision—are sufficient to regenerate all figures and claims in the main text from first principles,
within the constructible regime and under the filtered-colimit policy, while making accept/reject criteria explicit and auditable.



% =========================
\section*{Appendix H. Betti Integral and Finite $\tau$-Events [Proof]}
% =========================
\phantomsection
\addcontentsline{toc}{section}{Appendix H. Betti Integral and Finite $\tau$-Events}
\refstepcounter{section}
\label{H:betti-integral}

\paragraph{Standing conventions.}
We work over a field \(k\). All persistence modules are constructible (locally finite on bounded windows); filtered colimits, when used, are taken under the scope policy of Appendix~A, Remark~\ref{A:rk:filtered-colimits}. Endpoint conventions follow Appendix~A, Remark~\ref{A:rk:endpoints} (we use half-open intervals \([b,d)\); any consistent choice is immaterial below). Global conventions: \(\Ext^1\)-tests are always against \(k[0]\) (we write \(\Ext^1(\mathcal{R}(C_\tau F),k)=0\), with \(C_\tau\) understood up to f.q.i.\ on \(\Ho(\mathsf{FiltCh}(k))\)); the energy exponent satisfies \(\alpha>0\) (default \(\alpha=1\)); tilde references and type dashes (Type I--II / Type III / Type IV) are used uniformly.

\subsection*{H.1. Betti curves and the Betti integral}
Let \(F\) be a filtered chain complex (or a filtered object realising a persistence module) with degree-\(i\) persistence module \(\mathbf{P}_i(F)\).
Write its barcode as a locally finite multiset
\[
\mathbf{P}_i(F)\ \cong\ \bigoplus_{I\in \mathcal{B}_i(F)} I^{\oplus m(I)},\qquad I=[b,d)\ \text{with } d\in\mathbb{R}\cup\{\infty\},\ m(I)\in\mathbb{Z}_{\ge 1}.
\]
Define the Betti curve \(\beta_i(t):=\dim_k H_i(F^t)\) and the \emph{Betti integral} up to \(\tau\ge 0\) by
\[
\mathrm{PE}_i^{\le \tau}(F)\ :=\ \int_{0}^{\tau} \beta_i(t)\,dt.
\]
Under constructibility, \(\beta_i\) is piecewise constant and càdlàg, and on any bounded window only finitely many bars meet.

\begin{theorem}[Betti integral = clipped barcode mass]\label{H:thm:betti-integral}
For every \(\tau\ge 0\),
\[
\mathrm{PE}_i^{\le \tau}(F)\ =\ \sum_{I\in \mathcal{B}_i(F)} m(I)\cdot \lambda\!\left(I\cap[0,\tau]\right),
\]
where \(\lambda\) is Lebesgue measure and
\(\lambda([b,d)\cap[0,\tau])=\max\{0,\min\{d,\tau\}-\max\{b,0\}\}\) (with \(\min\{\infty,\tau\}=\tau\)). In particular, an infinite bar alive at \(0\) contributes its clipped length \(\tau\).
\end{theorem}

\begin{proof}
By constructibility and local finiteness, on every bounded window the sum \(\sum_{I\in\mathcal{B}_i(F)} m(I)\,\mathbf{1}_I(t)\) has only finitely many nonzero terms, hence defines a nonnegative measurable function; moreover, for a.e.\ \(t\) we have
\(\beta_i(t)=\sum_{I\in\mathcal{B}_i(F)} m(I)\,\mathbf{1}_I(t)\).
Tonelli’s theorem applies on bounded windows (finite sum of nonnegative measurable summands), giving
\[
\int_{0}^{\tau}\!\beta_i(t)\,dt\ =\sum_{I} m(I)\int_{0}^{\tau}\!\mathbf{1}_{I}(t)\,dt\ =\sum_{I} m(I)\,\lambda\!\left(I\cap[0,\tau]\right).
\]
\end{proof}

\begin{corollary}[Monotonicity, (a.e.) derivative, piecewise linearity]\label{H:cor:pl}
The map \(\tau\mapsto \mathrm{PE}_i^{\le \tau}(F)\) is non\hyp decreasing, continuous, and piecewise linear on every bounded interval. Its derivative satisfies
\[
\frac{d}{d\tau}\,\mathrm{PE}_i^{\le \tau}(F)=\beta_i(\tau)\quad\text{for a.e.\ }\tau,
\]
and at event points (births/deaths) the right derivative equals \(\beta_i(\tau)\) while the left derivative equals \(\beta_i(\tau-)\). All breakpoints on \([0,\tau_0]\) lie in \(\{0,\tau_0\}\cup\{b\in[0,\tau_0]\}\cup\{d\in[0,\tau_0]\}\).
\end{corollary}

\begin{remark}[Endpoint and baseline conventions]\label{H:rk:endpoints}
The choice of half-open intervals \([b,d)\) is immaterial for \(\mathrm{PE}_i^{\le \tau}\) and the breakpoint set: any consistent open/closed endpoint convention yields the same integral and event locations (cf.\ Appendix~A, Remark~\ref{A:rk:endpoints} and the global Notation and Conventions).
The baseline \(0\) is a reference; negative births are allowed and handled by the intersection \(I\cap[0,\tau]\).
\end{remark}

\subsection*{H.2. Finite \texorpdfstring{$\tau$}{tau}-events and finite checking sets}
Fix \(\tau_0>0\). Define the finite \(\tau\)-event set
\[
\mathsf{Ev}_i(F;\tau_0)\ :=\ \{0,\tau_0\}\ \cup\ \bigl(\{b\mid [b,d)\in\mathcal{B}_i(F)\}\cap[0,\tau_0]\bigr)\ \cup\ \bigl(\{d\mid [b,d)\in\mathcal{B}_i(F)\}\cap[0,\tau_0]\bigr).
\]

\begin{proposition}[Finite checking set]\label{H:prop:finite-check}
Let \(g:[0,\tau_0]\to\mathbb{R}\) be continuous and affine on each connected component of \([0,\tau_0]\setminus \mathsf{Ev}_i(F;\tau_0)\) (e.g.\ a piecewise linear benchmark with breakpoints in \(\mathsf{Ev}_i\)). Then, for either inequality direction,
\[
\mathrm{PE}_i^{\le \tau}(F)\ \ge\ g(\tau)\quad(\text{resp.\ }\le)\quad\text{for all }\tau\in[0,\tau_0]
\]
holds if and only if it holds for all \(\tau\in \mathsf{Ev}_i(F;\tau_0)\).
\end{proposition}

\begin{remark}[Algorithmic evaluation]\label{H:rk:algo}
Collect births/deaths intersecting \([0,\tau_0]\), sort to form \(\mathsf{Ev}_i(F;\tau_0)\), and precompute segment slopes via \(\beta_i\); then \(\mathrm{PE}_i^{\le \tau}\) evaluates in a single pass. Complexity: forming/sorting \(\mathsf{Ev}_i\) is \(O(M\log M)\) for \(M\) events; evaluation is linear thereafter. For full reproducibility, record your window definition and event counts in the run manifest (Appendix~G).
\end{remark}

\subsection*{H.3. Consequences for shifts, truncations, and window variation}
\paragraph{(i) Equivariance under shifts.}
For \(\varepsilon\in\mathbb{R}\), let the shift \(S^\varepsilon\) act by \((S^\varepsilon F)^t:=F^{t+\varepsilon}\); then \(\beta_i^{S^\varepsilon}(t)=\beta_i(t+\varepsilon)\) and, for \(\sigma\ge 0\),
\[
\mathrm{PE}_i^{\le \sigma}(S^\varepsilon F)\ =\int_{0}^{\sigma}\beta_i(t+\varepsilon)\,dt\ =\int_{\varepsilon}^{\sigma+\varepsilon}\beta_i(u)\,du\ =\mathrm{PE}_i^{\le \sigma+\varepsilon}(F)-\mathrm{PE}_i^{\le \varepsilon}(F).
\]
Thus \(\mathrm{PE}_i^{\le \sigma}\) is generally \emph{not} invariant for a fixed window under shifting \(F\); it is equivariant under the simultaneous shift of both the module and the integration window.

\paragraph{(ii) Truncation monotonicity (deletion-type).}
Let \(\mathbf{T}_{\tau'}\) denote the bar-deletion truncation at scale \(\tau'>0\) on \(\Pers^{\mathrm{ft}}_k\) (Appendix~A). Then, for every \(\sigma>0\),
\[
\mathrm{PE}_i^{\le \sigma}\!\big(\mathbf{T}_{\tau'}(\mathbf{P}_i(F))\big)\ \le\ \mathrm{PE}_i^{\le \sigma}\!\big(\mathbf{P}_i(F)\big),
\]
because \(\mathbf{T}_{\tau'}\) \emph{deletes precisely} the finite bars of length \(\le \tau'\) while leaving longer (including infinite) bars unchanged. Moreover, \(\mathbf{T}_{\tau'}\) is \(1\)-Lipschitz in the interleaving metric (Appendix~A, Proposition~\ref{A:prop:lipschitz}).

\paragraph{(iii) Lipschitz in the window parameter.}
For \(0\le s\le \tau\),
\[
\bigl|\mathrm{PE}_i^{\le \tau}(F)-\mathrm{PE}_i^{\le s}(F)\bigr|\ =\int_{s}^{\tau}\beta_i(t)\,dt\ \le (\tau-s)\cdot \sup_{t\in[s,\tau]}\beta_i(t),
\]
and the supremum is finite on bounded windows by constructibility. Use the same window policy (MECE, right-open) you record in Appendix~G to ensure comparability across runs.

\medskip
\noindent\textbf{Summary.}
The Betti integral equals the clipped barcode mass (Theorem~\ref{H:thm:betti-integral}); consequently \(\mathrm{PE}_i^{\le \tau}\) is continuous, non\hyp decreasing, and piecewise linear, with breakpoints among births/deaths. Any affine-on-components constraint can be verified on a finite event set (Proposition~\ref{H:prop:finite-check}). Under shifts, \(\mathrm{PE}_i^{\le \sigma}\) obeys the explicit equivariance formula above; under deletion-type truncations \(\mathbf{T}_{\tau'}\) it is nonincreasing and enjoys \(1\)-Lipschitz stability (Appendix~A, Proposition~\ref{A:prop:lipschitz}). All uses of (co)limits respect the global scope policy (Appendix~A, Remark~\ref{A:rk:filtered-colimits}).



% =========================
\section*{Appendix I. $\varepsilon$-Survival Lemma and Grid-to-Continuum [Proof/Spec]}
% =========================
\phantomsection
\addcontentsline{toc}{section}{Appendix I. $\varepsilon$-Survival Lemma and Grid-to-Continuum}
\refstepcounter{section}
\label{I:eps-survival}

\paragraph{Standing conventions.}
We work over a field \(k\). All persistence modules are constructible (locally finite on bounded windows). Any use of filtered colimits follows the scope policy of Appendix~A, Remark~\ref{A:rk:filtered-colimits}. Global conventions: \(\Ext^1\)-tests are always against \(k[0]\) (so we write \(\Ext^1(\mathcal{R}(C_\tau F),k)=0\)); the energy exponent satisfies \(\alpha>0\) (default \(\alpha=1\)); tilde references (\(\text{Appendix}\,\tilde{}\)) and type dashes (Type I--II / Type III / Type IV) are used uniformly. We write \(d_{\mathrm{int}}\) for the interleaving metric (coinciding with the bottleneck distance in the constructible 1D case).

\begin{remark}[Endpoint conventions are immaterial]\label{I:rk:endpoints}
We take intervals half-open \([b,d)\) with \(d\in\mathbb{R}\cup\{\infty\}\). Any consistent open/closed endpoint choice yields the same clipped lengths and event sets; all statements below are invariant under this choice. Use the same right-open convention as Appendix~G for run logs.
\end{remark}

\subsection*{I.1. Window clipping, nonexpansivity, and $\varepsilon$-survival}
Let \(\Pers^{\mathrm{ft}}_k\) be the category of constructible persistence modules. For \(\tau\ge 0\), let
\[
i_{\le\tau}:\ ([0,\tau],\le)\hookrightarrow(\mathbb{R},\le)
\]
be the fully faithful inclusion of posets, and write \(i_{\le\tau}^{\ast}\) for restriction along \(i_{\le\tau}\).
We identify modules on \([0,\tau]\) with modules on \(\mathbb{R}\) supported in \([0,\tau]\) via \emph{extension by zero} along \(i_{\le\tau}\); denote this functor by \((i_{\le\tau})^{0}_!\).
Define the \emph{window clip} endofunctor
\[
\mathbf{W}_{\le \tau}\ :=\ (i_{\le\tau})^{0}_!\circ i_{\le\tau}^{\ast}:\ \Pers^{\mathrm{ft}}_k\longrightarrow \Pers^{\mathrm{ft}}_k.
\]
In barcode terms, this coincides with interval intersection \(I\mapsto I\cap[0,\tau]\) (discarding empties). For a bar \(I=[b,d)\) its clipped length is
\[
\ell_{[0,\tau]}(I):=\lambda\!\left(I\cap[0,\tau]\right)=\max\{0,\min\{d,\tau\}-\max\{b,0\}\},
\]
with \(\min\{\infty,\tau\}=\tau\). Since only finitely many bars intersect any bounded window, \(\mathbf{W}_{\le\tau}\) \emph{preserves constructibility}.

\begin{lemma}[Clipping is $1$-Lipschitz]\label{I:lem:clip-1lip}
For all \(M,N\in\Pers^{\mathrm{ft}}_k\) and \(\tau\ge 0\),
\[
d_{\mathrm{int}}\bigl(\mathbf{W}_{\le \tau}M,\ \mathbf{W}_{\le \tau}N\bigr)\ \le\ d_{\mathrm{int}}(M,N).
\]
\end{lemma}

\begin{proof}[Proof sketch]
Restriction along \(i_{\le\tau}\) preserves \(\varepsilon\)-interleavings: the shift functors commute with pullback and the interleaving triangles restrict. Composing with extension by zero yields an \(\varepsilon\)-interleaving on \(\mathbb{R}\) supported in \([0,\tau]\). Equivalently in barcode terms (constructible case), clippingと matchingを先後どちらにしても費用は増えません。
\end{proof}

\begin{remark}[Shift commutation with clipping]\label{I:rk:shift-clip}
For every \(\varepsilon\ge 0\) there is a canonical isomorphism
\[
S^\varepsilon\circ \mathbf{W}_{\le\tau}\ \cong\ \mathbf{W}_{\le\tau}\circ S^\varepsilon,
\]
since restriction along \(i_{\le\tau}\) and extension by zero commute with the time-shift endofunctors. In particular, any \(\varepsilon\)-interleaving data transport through \(\mathbf{W}_{\le\tau}\) via these identifications.
\end{remark}

\begin{lemma}[$\varepsilon$-survival lemma]\label{I:lem:survive}
Let \(M,N\in\Pers^{\mathrm{ft}}_k\) with \(d_{\mathrm{int}}(M,N)\le\varepsilon\). Fix \(\tau_0>0\).
\begin{enumerate}\itemsep0.2em
\item \textbf{Quantitative form.} For \emph{any} bar-matching that witnesses \(d_{\mathrm{int}}(M,N)\le\varepsilon\) (not necessarily unique), if a bar \(I\) of \(M\) is matched to a bar \(J\) of \(N\), then
\[
\ell_{[0,\tau_0]}(J)\ \ge\ \max\bigl\{\,\ell_{[0,\tau_0]}(I)-2\varepsilon,\ 0\,\bigr\}.
\]
\item \textbf{Nonvanishing form.} If \(\ell_{[0,\tau_0]}(I)>2\varepsilon\) for some bar \(I\) in \(M\), thenその相方 \(J\) は \(\ell_{[0,\tau_0]}(J)>0\) となる。従って \(\mathbf{W}_{\le \tau_0}N\neq 0\).
\item \textbf{Multiplicity lower bound.} If at least \(r\) bars of \(M\) have clipped length \(>\!2\varepsilon\), then at least \(r\) bars of \(N\) remain nonzero after clipping to \([0,\tau_0]\) (counted with multiplicity).
\end{enumerate}
\end{lemma}

\begin{proof}[Proof sketch]
In the constructible 1D case \(d_{\mathrm{int}}\) equals the bottleneck distance; endpoints of matched bars move by at most \(\varepsilon\). Hence the intersection length with \([0,\tau_0]\) can shrink by at most \(\varepsilon\) at each end, giving (1). Diagonal points have zero clipped length, so they cannot match bars with clipped length \(>\!2\varepsilon\); (2) and (3) follow.
\end{proof}

\begin{remark}[Notation]\label{I:rk:notation}
We reserve \(\mathbf{T}_\tau\) for the Serre reflection that removes bars of length \(\le\tau\) (Appendix~A, Theorem~\ref{A:thm:localization}, and Proposition~\ref{A:prop:lipschitz} for its \(1\)-Lipschitz property). Window clipping is denoted \(\mathbf{W}_{\le\tau}\). The present statements concern \(\mathbf{W}_{\le\tau}\). Appendix~H relates clipping to the Betti integral.
\end{remark}

\subsection*{I.2. Grid-to-continuum transfer}
Let \(F\) be a filtered object with degree-\(i\) persistence \(\mathbf{P}_i(F)\).
Suppose \(F_h\) is a discretization at mesh \(h\) with \(\mathbf{P}_i(F_h)\) and a certified bound
\[
d_{\mathrm{int}}\bigl(\mathbf{P}_i(F_h),\ \mathbf{P}_i(F)\bigr)\ \le\ \varepsilon(h).
\]

\begin{theorem}[Grid-to-continuum survival]\label{I:thm:g2c}
Fix \(\tau_0>0\) and \(r\in\mathbb{Z}_{\ge 1}\).
If \(\mathbf{W}_{\le \tau_0}\mathbf{P}_i(F_h)\) contains at least \(r\) bars of clipped length \(\ge 2\varepsilon(h)+\eta\) for some margin \(\eta>0\), then
\[
\mathbf{W}_{\le \tau_0}\mathbf{P}_i(F)\quad\text{contains at least \(r\) nonzero bars (with multiplicity), each of clipped length }\ge \eta.
\]
In particular, nonvanishing on the grid with margin \(2\varepsilon(h)+\eta\) implies nonvanishing in the continuum window \([0,\tau_0]\) (no false negatives). Record \(\varepsilon(h)\), \(\tau_0\), and the event counts per window in the run manifest (Appendix~G) for reproducibility.
\end{theorem}

\begin{proof}
Apply Lemma~\ref{I:lem:survive} with \(M=\mathbf{P}_i(F_h)\), \(N=\mathbf{P}_i(F)\), and \(\varepsilon=\varepsilon(h)\). Bars with clipped length \(>\!2\varepsilon(h)\) cannot be matched to the diagonal; at least \(r\) such bars survive with length \(\ge\eta\), counted with multiplicity.
\end{proof}

\begin{remark}[How to use]\label{I:rk:practice}
On the grid: compute \(\mathbf{P}_i(F_h)\), clip to \([0,\tau_0]\), and select bars of length \(\ge 2\varepsilon(h)+\eta\). If at least \(r\) such bars occur, then in the continuum \(\mathbf{W}_{\le \tau_0}\mathbf{P}_i(F)\) has at least \(r\) nonzero bars (each of length \(\ge \eta\)). The same finite event set from Appendix~H (births/deaths within \([0,\tau_0]\)) suffices to certify affine-on-components benchmarks for all \(\tau\in[0,\tau_0]\).
\end{remark}

\subsection*{I.3. Variants, sharpness, and compatibility}
\begin{itemize}\itemsep0.25em
\item \emph{Sharpness.} \textbf{The constant \(2\) is optimal}: a bar with \(\ell_{[0,\tau_0]}=2\varepsilon\) can be shifted by \(\varepsilon\) at each endpoint so that its clip collapses to length \(0\).
\item \emph{Compatibility with functors.} By Lemma~\ref{I:lem:clip-1lip} and Appendix~A, Proposition~\ref{A:prop:lipschitz}, composition with \(1\)-Lipschitz endofunctors (e.g.\ reflections \(\mathbf{T}_\tau\); mirror/transfer maps under their stated [Spec] hypotheses) preserves the survival inequalities.
\item \emph{Towers.} In a tower \(F_n\to F_\infty\), if \(\varepsilon_n\to 0\) with \(d_{\mathrm{int}}(\mathbf{P}_i(F_n),\mathbf{P}_i(F_\infty))\le \varepsilon_n\), then any uniform margin \(>\!0\) propagates from grid to continuum by Theorem~\ref{I:thm:g2c} (cf.\ Appendix~D for tower diagnostics).
\end{itemize}

\paragraph{Reference line for Chapter 4.5.}
\emph{“By Appendix~I (the $\varepsilon$-Survival Lemma, Lemma~\ref{I:lem:survive}), any grid-detected bar in the window \([0,\tau_0]\) with clipped length \(>2\varepsilon(h)\) persists in the continuum window; we therefore certify the claim at scale \(\tau_0\) from the discretization.”}

\medskip
\noindent\textbf{Summary.}
Clipping is restriction to \([0,\tau]\) followed by extension by zero to \(\mathbb{R}\), hence an endofunctor \(\mathbf{W}_{\le\tau}\) on \(\Pers^{\mathrm{ft}}_k\); it preserves constructibility and is \(1\)-Lipschitz (Lemma~\ref{I:lem:clip-1lip}). Under an \(\varepsilon\)-interleaving, clipped lengths degrade by at most \(2\varepsilon\); hence grid detections with margin \(>\!2\varepsilon\) transfer to continuum without false negatives (Lemma~\ref{I:lem:survive}, Theorem~\ref{I:thm:g2c}). These are consistent with the global scope policy and the finite-event structure of Appendices~A and~H.



% =========================
\section*{Appendix J. Calculus of $\mu,\nu$ [Proof + Stability Bands + Window Pasting]}
% =========================
\phantomsection
\addcontentsline{toc}{section}{Appendix J. Calculus of $\mu,\nu$}
\refstepcounter{section}
\label{J:calc}

\paragraph{Standing conventions.}
We work over a field \(k\).
All persistence modules are constructible (locally finite on bounded windows).
Filtered colimits are computed in the functor category \([\mathbb{R},\mathsf{Vect}_k]\) under the scope policy of Appendix~A, Remark~\ref{A:rk:filtered-colimits}, and then (when stated) returned to the constructible range.
The reflection \(\mathbf{T}_\tau\dashv \iota_\tau\) is exact and \(1\)-Lipschitz (Appendix~A, Theorem~\ref{A:thm:localization} and Proposition~\ref{A:prop:lipschitz}).
Global conventions: \(\Ext^1\) is always against \(k[0]\); the energy exponent \(\alpha>0\) (default \(\alpha=1\)).
All window policies follow the MECE (mutually exclusive–collectively exhaustive), right-open convention used throughout (Appendix~G).

\paragraph{Setup and notation.}
Fix \(i\in\mathbb{Z}\).
Let \(F=(F_n)_{n\in I}\) be a directed system (``tower'') of filtered objects for which \(\mathbf{P}_i(F_n)\in\Pers^{\mathrm{ft}}_k\), and let \(F_\infty\) be its colimit with cocone maps \(F_n\to F_\infty\).
For \(\tau\ge 0\) consider the \emph{comparison map} in the functor category
\begin{equation}\label{J:eq:phi}
\phi_{i,\tau}(F):\quad
\varinjlim_{n}\ \mathbf{T}_\tau\big(\mathbf{P}_i(F_n)\big)\ \longrightarrow\ \mathbf{T}_\tau\big(\mathbf{P}_i(F_\infty)\big).
\end{equation}
Define
\begin{equation}\label{J:eq:mu-nu}
\mu_{i,\tau}(F)\ :=\ \dim_k \ker \phi_{i,\tau}(F),\qquad
\nu_{i,\tau}(F)\ :=\ \dim_k \operatorname{coker} \phi_{i,\tau}(F),
\end{equation}
where \(\dim_k\) denotes the \emph{generic fiber} dimension in \(\Pers^{\mathrm{ft}}_k\) (i.e.\ the stabilized right-tail dimension \(\lim_{t\to+\infty}\dim_k(-)(t)\), equivalently the multiplicity of \(I[0,\infty)\) after applying \(\mathbf{T}_\tau\); cf.\ Appendix~D, Remark~\ref{rem:D-generic-dim}).\footnote{For a constructible module \(M\), the generic fiber dimension equals \(\dim_k M(t)\) for all sufficiently large \(t\), which stabilizes. After applying \(\mathbf{T}_\tau\), it coincides with the multiplicity of \(I[0,\infty)\) summands. Kernels/cokernels are taken in \(\Pers^{\mathrm{ft}}_k\), where they decompose into finite direct sums of interval modules.}
We also write the \emph{totals}
\begin{equation}\label{J:eq:totals}
\muc(F)\ :=\ \sum_{i}\mu_{i,\tau}(F),\qquad
\nuc(F)\ :=\ \sum_{i}\nu_{i,\tau}(F),
\end{equation}
which are finite because the complexes are bounded in homological degrees (constructible range).
All quantities depend on \(\tau\); no general monotonicity in \(\tau\) is asserted.

\subsection*{J.1. Subadditivity under composition}
We formalize ``composition of towers'' via morphisms of directed systems.

\begin{definition}[Morphisms of towers]
A morphism \(u:F\to G\) consists of maps \(u_n:F_n\to G_n\) commuting with the structure maps (in the index category) and inducing a canonical map on colimits \(u_\infty:F_\infty\to G_\infty\).
Given \(u:F\to G\) and \(v:G\to H\), the composite \(v\circ u:F\to H\) is defined degreewise.
\end{definition}

\begin{lemma}[Functoriality of comparison maps]\label{J:lem:functoriality}
Under the scope policy, each morphism \(u:F\to G\) induces a (canonical) morphism of comparison maps
\[
\phi_{i,\tau}(u):\ \phi_{i,\tau}(F)\ \Longrightarrow\ \phi_{i,\tau}(G),
\]
natural in both \(i\) and \(\tau\) (here \(\Longrightarrow\) indicates a canonical comparison morphism, natural in \(i,\tau\)).
Moreover, for composable \(u:F\to G\), \(v:G\to H\),
\[
\phi_{i,\tau}(v\circ u)\ =\ \phi_{i,\tau}(v)\ \circ\ \phi_{i,\tau}(u).
\]
\end{lemma}

\begin{proof}[Proof sketch]
Apply \(\mathbf{P}_i\), then \(\mathbf{T}_\tau\), to the maps on systems and pass to filtered colimits in \([\mathbb{R},\mathsf{Vect}_k]\); exactness of \(\mathbf{T}_\tau\) and functoriality of colimits yield naturality and compatibility with composition.
\end{proof}

\begin{theorem}[Subadditivity under composition]\label{J:thm:subadd}
Let \(u:F\to G\) and \(v:G\to H\) be morphisms of towers and fix \(\tau\ge 0\).
Then
\[
\mu_{i,\tau}(v\circ u)\ \le\ \mu_{i,\tau}(u)\ +\ \mu_{i,\tau}(v),\qquad\n\nu_{i,\tau}(v\circ u)\ \le\ \nu_{i,\tau}(u)\ +\ \nu_{i,\tau}(v).
\]
\end{theorem}

\begin{proof}
By Lemma~\ref{J:lem:functoriality} we reduce to linear estimates on the generic fiber.
In \([\mathbb{R},\mathsf{Vect}_k]\), kernels and cokernels are computed pointwise; applying \(\mathbf{T}_\tau\) preserves exactness.
In the constructible range the (right-tail) generic fiber dimension equals the stabilized pointwise dimension.
Thus, evaluating the comparison maps at any sufficiently large parameter \(t\) (where stabilization holds) and writing \(X_t\xrightarrow{f_t}Y_t\xrightarrow{g_t}Z_t\) for the induced linear maps, we have the elementary inequalities
\[
\dim\ker(g_t\!\circ\! f_t)\ \le\ \dim\ker f_t\ +\ \dim\ker g_t,\qquad\n\dim\operatorname{coker}(g_t\!\circ\! f_t)\ \le\ \dim\operatorname{coker} f_t\ +\ \dim\ker g_t,
\]
and taking these stabilized values yields the stated bounds for the generic fiber dimensions in~\eqref{J:eq:mu-nu}.\footnote{Finite-dimensional linear algebra: for \(X\xrightarrow{f}Y\xrightarrow{g}Z\), one has \(\ker(g\!\circ\! f)\subseteq \ker f \oplus f^{-1}(\ker g)\) and \(\operatorname{coker}(g\!\circ\! f)\) controlled by \(\operatorname{coker} f\) and \(\ker g\), giving the stated subadditivity bounds on dimensions.}
\end{proof}

\subsection*{J.2. Additivity under finite direct sums}

\begin{proposition}[Direct-sum additivity]\label{J:prop:sum}
Let \(F=F^{(1)}\oplus F^{(2)}\) be the levelwise direct sum of towers (same index category) and similarly for the colimit.
Then for every \(\tau\ge 0\),
\[
\mu_{i,\tau}(F)\ =\ \mu_{i,\tau}\!\big(F^{(1)}\big)\ +\ \mu_{i,\tau}\!\big(F^{(2)}\big),\qquad\n\nu_{i,\tau}(F)\ =\ \nu_{i,\tau}\!\big(F^{(1)}\big)\ +\ \nu_{i,\tau}\!\big(F^{(2)}\big).
\]
In particular, the totals satisfy
\[
\muc(F)\ =\ \muc\!\big(F^{(1)}\big)\ +\ \muc\!\big(F^{(2)}\big),\qquad\n\nuc(F)\ =\ \nuc\!\big(F^{(1)}\big)\ +\ \nuc\!\big(F^{(2)}\big).
\]
\end{proposition}

\begin{proof}
\(\mathbf{P}_i\) and \(\mathbf{T}_\tau\) preserve finite direct sums; filtered colimits \emph{commute with finite direct sums} in \([\mathbb{R},\mathsf{Vect}_k]\).
Hence \(\phi_{i,\tau}(F)\) is block-diagonal with blocks \(\phi_{i,\tau}\!\big(F^{(1)}\big)\) and \(\phi_{i,\tau}\!\big(F^{(2)}\big)\), and kernels/cokernels (hence generic fiber dimensions) add; summing over \(i\) gives the identities for \(\muc,\nuc\).
\end{proof}

\subsection*{J.3. Cofinal invariance}

\begin{definition}[Cofinal subindexing]
Let \(I\) be the directed index category for \(F\).
A full subcategory \(J\subset I\) is cofinal if for every \(i\in I\) there exists \(j\in J\) with a morphism \(i\to j\).
The restricted tower \(F|_J\) has the same colimit as \(F\) in \([\mathbb{R},\mathsf{Vect}_k]\).
\end{definition}

\begin{theorem}[Cofinal invariance]\label{J:thm:cofinal}
Let \(J\subset I\) be cofinal. Then for all \(\tau\ge 0\),
\[
\mu_{i,\tau}\big(F|_J\big)=\mu_{i,\tau}(F),\qquad\n\nu_{i,\tau}\big(F|_J\big)=\nu_{i,\tau}(F),
\]
hence also \(\muc(F|_J)=\muc(F)\) and \(\nuc(F|_J)=\nuc(F)\).
In particular, passing to any cofinal tail of a sequential tower leaves \((\muc,\nuc)\) unchanged.
\end{theorem}

\begin{proof}
Cofinal restriction does not change colimits in \([\mathbb{R},\mathsf{Vect}_k]\).
Thus the source and target of \(\phi_{i,\tau}\) (Eq.~\eqref{J:eq:phi}) remain unchanged, as does \(\phi_{i,\tau}\) itself; kernels/cokernels (hence \(\mu_{i,\tau},\nu_{i,\tau}\), and the totals) agree.
\end{proof}

\subsection*{J.4. Consequences and remarks}
\begin{itemize}\itemsep0.25em
  \item \emph{Triangle-type bounds.} Combining Theorem~\ref{J:thm:subadd} across a chain of morphisms yields \((\muc,\nuc)\)-control along multi-stage pipelines, useful when towers factor through preprocessing steps (e.g.\ mirror/transfer under the [Spec] hypotheses of Appendix~L).
  \item \emph{Finite decomposition.} Iterating Proposition~\ref{J:prop:sum} shows that \((\muc,\nuc)\) is additive across any finite direct-sum decomposition of towers, enabling modular bookkeeping.
  \item \emph{Cofinal tails for convergence.} Theorem~\ref{J:thm:cofinal} justifies replacing a tower by any convenient tail when verifying sufficient conditions for \((\muc,\nuc)=(0,0)\) (Appendix~D, §D.3).
  \item \emph{No general monotonicity in \(\tau\).} Absent additional hypotheses, \(\tau\mapsto(\mu_{i,\tau},\nu_{i,\tau})\) (hence \(\tau\mapsto(\muc,\nuc)\)) need not be monotone; only the sufficient conditions of Appendix~D (e.g.\ commutation, no \(\tau\)-accumulation, Cauchy+compatibility) guarantee \((\muc,\nuc)=(0,0)\) at fixed \(\tau\).
  \item \emph{Invariance under filtered quasi-isomorphism.} As in Appendix~D (and the formalization sketch of Appendix~F), \((\muc,\nuc)\) is invariant under replacing each \(F_n\) up to f.q.i., since \(\mathbf{P}_i\) and \(\mathbf{T}_\tau\) respect such replacements and kernels/cokernels (hence generic fiber dimensions) are preserved under isomorphism in \(\Pers^{\mathrm{ft}}_k\).
\end{itemize}

% -------------------------
\subsection*{J.5. $\tau$-sweep and stability bands}
% -------------------------
We formalize the practical, windowed method for selecting scales \(\tau\) at which tower effects vanish.

\begin{definition}[Stability band for a fixed window and degree]\label{J:def:stab-band}
Fix a window (MECE, right-open) and a degree \(i\).
A \emph{\(\tau\)-sweep} is a finite or countable increasing array \(\{\tau_\ell\}_{\ell\in L}\subset(0,\infty)\).
A contiguous subarray \(\{\tau_a,\ldots,\tau_b\}\) is a \emph{stability band} if
\[
\mu_{i,\tau_\ell}(F)=\nu_{i,\tau_\ell}(F)=0\quad\text{for all }\ \ell\in\{a,\ldots,b\},
\]
and the verdict persists under refinement of the sweep (inserting new \(\tau\)-values) without creating nonzero \(\mu_{i,\tau}\) or \(\nu_{i,\tau}\) in the band.
\end{definition}

\begin{proposition}[Robust detection under sweep refinement]\label{J:prop:robust-band}
Assume one of the sufficient conditions of Appendix~D, §D.3 holds (e.g.\ commutation (S1), no \(\tau\)-accumulation (S2), or \(\mathbf{T}_\tau\)-Cauchy with compatible cocone (S3)). Then for each \(i\) there exists a neighborhood of any \(\tau_0\) where \(\phi_{i,\tau}\) is an isomorphism, hence \((\mu_{i,\tau},\nu_{i,\tau})=(0,0)\). A sufficiently fine \(\tau\)-sweep detects such neighborhoods as stability bands and remains stable under refinement.
\end{proposition}

\begin{remark}[Aggregating across degrees]
Applications often monitor either a fixed finite degree set \(I\subset\mathbb{Z}\) or the total \(\muc,\nuc\). A \emph{joint} stability band is the intersection of degreewise bands; it is nonempty whenever each degree admits a band with nontrivial overlap. Report the monitored degree set in the run manifest (Appendix~G).
\end{remark}

% -------------------------
\subsection*{J.6. Window pasting via Restart and Summability}
% -------------------------
We now integrate a \emph{Restart} inequality and a \emph{Summability} condition to paste windowed certificates into global ones. The $\delta$-ledger (algorithmic/discretization/measurement) is as in Appendix~G; Mirror/Transfer commutation defects are treated via Appendix~L.

\begin{definition}[Per-window pipeline budget and safety margin]\label{J:def:budget-gap}
Let \(\{W_k=[u_k,u_{k+1})\}_k\) be a MECE partition (right-open).
On window \(W_k\), let \(\tau_k>0\) be the selected collapse threshold (possibly from a stability band) and define the \emph{pipeline budget}
\[
\Sigma\delta_k(i)\ :=\ \sum_{U\in W_k}\Big(\delta^{\mathrm{alg}}_{U}(i,\tau_k)+\delta^{\mathrm{disc}}_{U}(i,\tau_k)+\delta^{\mathrm{meas}}_{U}(i,\tau_k)\Big),
\]
with \(\delta\)-components recorded per operation (Appendix~G). The \emph{safety margin} \(\mathrm{gap}_{\tau_k}(i)>0\) is the configured slack for B-Gate\(^{+}\) on \(W_k\) and degree \(i\) (Chapter~1).
\end{definition}

\begin{lemma}[Restart inequality]\label{J:lem:restart}
Assume that on \(W_k\) the B-Gate\(^{+}\) passes with \(\mathrm{gap}_{\tau_k}(i)>\Sigma\delta_k(i)\) and that the transition to \(W_{k+1}\) is realized by a finite composition of \emph{deletion-type} steps and \(\varepsilon\)-continuations (both measured after \(\mathbf{T}_\tau\)). Then there exists \(\kappa\in(0,1]\), depending only on the admissible step class and the \(\tau\)-adaptation policy, such that
\[
\mathrm{gap}_{\tau_{k+1}}(i)\ \ge\ \kappa\ \bigl(\mathrm{gap}_{\tau_k}(i)-\Sigma\delta_k(i)\bigr).
\]
\end{lemma}

\begin{proof}[Proof sketch]
Deletion-type steps are nonincreasing for monitored indicators after \(\mathbf{T}_\tau\) (Appendix~E); \(\varepsilon\)-continuations are \(1\)-Lipschitz, so the drift is bounded by the declared \(\varepsilon\). Aggregating drifts across the finite composition yields a multiplicative retention factor \(\kappa\).
\end{proof}

\begin{definition}[Summability]\label{J:def:summability}
A run satisfies \emph{Summability} (on degrees \(i\in I\)) if
\[
\sum_{k}\Sigma\delta_k(i)\ <\ \infty\qquad(\forall\,i\in I).
\]
A sufficient design pattern is geometric decay of thresholds (e.g.\ \(\tau_k=\tau_0\rho^k\), \(\rho\in(0,1)\)), combined with bounded per-window operation counts, as recorded in Appendix~G.
\end{definition}

\begin{theorem}[Pasting windowed certificates]\label{J:thm:pasting}
Let \(\{W_k\}_k\) be MECE. Suppose that on each \(W_k\) the B-Gate\(^{+}\) passes with \(\mathrm{gap}_{\tau_k}(i)>\Sigma\delta_k(i)\) (for all \(i\in I\)), that the Restart inequality (Lemma~\ref{J:lem:restart}) holds at every transition, and that Summability (Definition~\ref{J:def:summability}) is satisfied. Then the concatenation of per-window certificates yields a global certificate on \(\bigcup_k W_k\) for the monitored degrees \(i\in I\).
\end{theorem}

\begin{proof}[Proof sketch]
Iterate Lemma~\ref{J:lem:restart} and sum budgets; Summability ensures that the cumulative loss of safety margin remains bounded, so positivity of the margin persists. MECE coverage (Appendix~A) ensures there are no gaps or overlaps; stability bands (if used) fix \(\tau_k\) within regimes where \((\mu,\nu)=(0,0)\).
\end{proof}

% -------------------------
\subsection*{J.7. Practical detection, logging, and reproducibility}
% -------------------------
In practice, one proceeds along a windowed run (Appendix~G) as follows:
\begin{enumerate}\itemsep0.25em
  \item \emph{Select \(\tau_k\)} on window \(W_k\) via a \(\tau\)-sweep; identify a stability band by Proposition~\ref{J:prop:robust-band}.
  \item \emph{Evaluate \((\mu,\nu)\)} at the chosen \(\tau_k\) and record \(\phi_{i,\tau_k}\)-ranks, \(\mu_{i,\tau_k},\nu_{i,\tau_k}\) in \texttt{phi} artifacts (Appendix~G).
  \item \emph{Log the pipeline budget} \(\Sigma\delta_k(i)\) and \(\mathrm{gap}_{\tau_k}(i)\) in \texttt{run.yaml} (Appendix~G) and check the Restart inequality.
  \item \emph{Paste certificates} across windows using Theorem~\ref{J:thm:pasting}; ensure that coverage checks and MECE conventions pass (Appendix~G).
\end{enumerate}
Mirror/Transfer non-commutation is measured after collapse via a natural 2-cell and contributes additively to \(\Sigma\delta\) (Appendix~L).

% -------------------------
\subsection*{J.8. Edge cases and pitfalls}
% -------------------------
\begin{itemize}\itemsep0.25em
  \item \emph{Near-threshold accumulation.} If bars approach length \(\tau\) from below, stability bands may be thin or absent; operate with finer sweeps and report \((\mu,\nu)\) sensitivity.
  \item \emph{Metric drift without Summability.} If \(\sum_k\Sigma\delta_k(i)\) diverges, the safety margin can be exhausted despite per-window passes; redesign the pipeline (e.g.\ impose geometric decay).
  \item \emph{Degree mixing.} When aggregating \(\muc,\nuc\) across degrees, a band for totals may hide degreewise failures. Prefer reporting both per-degree and totals (Appendix~G).
\end{itemize}

% -------------------------
\subsection*{J.9. Formalization pointers (Lean/Coq) and API alignment}
% -------------------------
A minimal formalization (Appendix~F) provides:
\begin{itemize}\itemsep0.25em
  \item \texttt{phi} as \(\varinjlim \mathbf{T}_\tau\mathbf{P}_i(F_n)\to \mathbf{T}_\tau\mathbf{P}_i(F_\infty)\), with naturality in tower morphisms, composition, and cofinal reindexing.
  \item \texttt{mu}, \texttt{nu} as generic-fiber dimensions of kernel/cokernel; subadditivity, additivity, and invariance lemmas (Theorems~\ref{J:thm:subadd}, \ref{J:prop:sum}, \ref{J:thm:cofinal}).
  \item Stability-band predicates over discrete \(\tau\)-sweeps (Definition~\ref{J:def:stab-band}) and refinement-stability (Proposition~\ref{J:prop:robust-band}).
  \item Restart/Summability contracts (\(\kappa\), \(\sum\Sigma\delta\)) and a pasting theorem as in Theorem~\ref{J:thm:pasting}. These align with the \texttt{run.yaml} schema (Appendix~G).
\end{itemize}

\medskip
\noindent\textbf{Summary.}
Within the constructible range and the filtered-colimit scope of Appendix~A, the comparison map \(\phi_{i,\tau}\) is functorial in the tower and compatible with composition and finite sums; \(\mu\) and \(\nu\) (and their totals \(\muc,\nuc\)) inherit subadditivity under composition, additivity under finite direct sums, and invariance under cofinal reindexing (and under f.q.i.\ replacements).
On top of this, \(\tau\)-sweeps with stability bands provide an operational tool to select robust scales where \((\mu,\nu)=(0,0)\), while Restart and Summability furnish a principled pasting mechanism for windowed certificates across MECE partitions, with all budgets and outcomes logged reproducibly (Appendix~G).



% =========================
\section*{Appendix K. Idempotent (Co)Monads for Collapse (up to f.q.i.) [Spec + Soft-Commuting Policy]}
% =========================
\addcontentsline{toc}{section}{Appendix K. Idempotent (Co)Monads for Collapse (up to f.q.i.)}
\label{K:monads}

\paragraph{Standing conventions.}
We work over a field \(k\).
All persistence modules are constructible (locally finite on bounded windows).
Filtered (co)limits are computed in the functor category \([\mathbb{R},\mathsf{Vect}_k]\) under the scope policy of Appendix~A, Remark~\ref{A:rk:filtered-colimits}; when stated, the result is returned to the constructible range.
Reflection \(\mathbf{T}_\tau\dashv \iota_\tau\) onto the \(\tau\)\hyp local (orthogonal) subcategory \((\mathsf{E}_\tau)^\perp\subset\Pers^{\mathrm{ft}}_k\) is exact and \(1\)\hyp Lipschitz (Appendix~A, Theorem~\ref{A:thm:localization}, Proposition~\ref{A:prop:lipschitz}).
Global conventions: \(\Ext^1\) is always against \(k[0]\); the energy exponent is \(\alpha>0\) (default \(\alpha=1\)); windows are MECE and right\hyp open (Appendix~G).

\subsection*{K.1. (Persistence layer) the idempotent monad \(\ \iota_\tau\mathbf{T}_\tau\)}
Let \(\iota_\tau:(\mathsf{E}_\tau)^\perp\hookrightarrow \Pers^{\mathrm{ft}}_k\) be the fully faithful inclusion and \(\mathbf{T}_\tau:\Pers^{\mathrm{ft}}_k\to(\mathsf{E}_\tau)^\perp\) its left adjoint (Appendix~A, Theorem~\ref{A:thm:localization}). Consider the endofunctor
\[
\mathbf{M}_\tau\ :=\ \iota_\tau\circ \mathbf{T}_\tau:\ \Pers^{\mathrm{ft}}_k\longrightarrow \Pers^{\mathrm{ft}}_k.
\]

\begin{theorem}[Idempotent monad on \(\Pers^{\mathrm{ft}}_k\)]\label{K:thm:monad}
With unit \(\eta:\mathrm{Id}\Rightarrow \iota_\tau\mathbf{T}_\tau\) and multiplication
\(\mu:\mathbf{M}_\tau^2=\iota_\tau\mathbf{T}_\tau\iota_\tau\mathbf{T}_\tau\xRightarrow{\ \iota_\tau\,\varepsilon\,\mathbf{T}_\tau\ }\iota_\tau\mathbf{T}_\tau=\mathbf{M}_\tau\),
induced by the counit \(\varepsilon:\mathbf{T}_\tau\iota_\tau\Rightarrow \mathrm{Id}\) on \((\mathsf{E}_\tau)^\perp\), the triple \((\mathbf{M}_\tau,\eta,\mu)\) is a monad on \(\Pers^{\mathrm{ft}}_k\).
Moreover, \(\mu\) is a natural isomorphism (idempotence).
\end{theorem}

\begin{proof}[Proof sketch]
Standard for reflective subcategories with fully faithful right adjoint: triangle identities yield the monad axioms; idempotence follows because \(\varepsilon\) is an isomorphism on \((\mathsf{E}_\tau)^\perp\).
\end{proof}

\begin{proposition}[Exactness and non\hyp expansiveness]\label{K:prop:exact-lip}
\(\mathbf{M}_\tau\) is exact on \(\Pers^{\mathrm{ft}}_k\) and \(1\)\hyp Lipschitz for the interleaving/bottleneck metric:
\[
d_{\mathrm{int}}\!\big(\mathbf{M}_\tau M,\ \mathbf{M}_\tau N\big)\ \le\ d_{\mathrm{int}}(M,N)\qquad(M,N\in\Pers^{\mathrm{ft}}_k).
\]
\end{proposition}

\begin{proof}[Proof sketch]
Exactness comes from exactness of \(\mathbf{T}_\tau\) and \(\iota_\tau\). Non\hyp expansiveness follows from shift–commutation of \(\mathbf{T}_\tau\) (Appendix~A, Proposition~\ref{A:prop:lipschitz}).
\end{proof}

\begin{remark}[Algebras over \(\mathbf{M}_\tau\)]
\(\mathbf{M}_\tau\)\hyp algebras \(\mathbf{M}_\tau M \to M\) correspond to \(\tau\)\hyp collapsed objects, i.e.\ the essential image of \(\iota_\tau\).
Eilenberg--Moore/Kleisli descriptions are standard.
\end{remark}

\subsection*{K.2. (Ho(\(\mathsf{FiltCh}(k)\)) — implementable range, up to f.q.i.) the idempotent comonad \(\ \iota\circ C_\tau^{\mathrm{comb}}\) [Spec]}
\emph{[Spec]} There exists an implementable subcategory \(\Ho(\mathsf{FiltCh}(k))_\tau^{\mathrm{comb}}\subset \Ho(\mathsf{FiltCh}(k))\) (``\(\tau\)\hyp combinatorial collapses'') and a fully faithful inclusion
\[
\iota:\ \Ho(\mathsf{FiltCh}(k))_\tau^{\mathrm{comb}}\hookrightarrow \Ho(\mathsf{FiltCh}(k))
\]
admitting a right adjoint (coreflector) \(C_\tau^{\mathrm{comb}}:\Ho(\mathsf{FiltCh}(k))\to \Ho(\mathsf{FiltCh}(k))_\tau^{\mathrm{comb}}\), natural up to filtered quasi\hyp isomorphism (Appendix~B). Define
\[
\mathbf{G}_\tau\ :=\ \iota\circ C_\tau^{\mathrm{comb}}:\ \Ho(\mathsf{FiltCh}(k))\longrightarrow \Ho(\mathsf{FiltCh}(k)).
\]

\begin{theorem}[Idempotent comonad on \(\Ho(\mathsf{FiltCh}(k))\) up to f.q.i.]\label{K:thm:comonad}
Assuming \(\iota\dashv C_\tau^{\mathrm{comb}}\), with counit \(\varepsilon:\mathbf{G}_\tau\Rightarrow \mathrm{Id}\) and comultiplication
\(\delta:\mathbf{G}_\tau\xRightarrow{\ \iota\,\eta\,C_\tau^{\mathrm{comb}}\ }\mathbf{G}_\tau^2\)
(\(\eta\) the unit on \(\Ho(\mathsf{FiltCh}(k))_\tau^{\mathrm{comb}}\)), the triple \((\mathbf{G}_\tau,\varepsilon,\delta)\) is a comonad on \(\Ho(\mathsf{FiltCh}(k))\).
Moreover, \(\delta\) is a natural isomorphism (idempotence).
All identities hold in \(\Ho(\mathsf{FiltCh}(k))\) and are invariant under filtered quasi\hyp isomorphisms.
\end{theorem}

\begin{proof}[Proof sketch]
Standard for coreflective subcategories with fully faithful left adjoint; idempotence follows since \(\eta\) is an isomorphism on \(\Ho(\mathsf{FiltCh}(k))_\tau^{\mathrm{comb}}\).
\end{proof}

\begin{proposition}[Compatibility with persistence]\label{K:prop:compat}
For any filtered complex \(F\),
\[
\mathbf{P}_i\big(\mathbf{G}_\tau F\big)\ \cong\ \mathbf{M}_\tau\big(\mathbf{P}_i(F)\big)\qquad\text{naturally in \(i\) and \(F\) in }\Pers^{\mathrm{ft}}_k,
\]
and, after \(\mathbf{P}_i\),
\[
d_{\mathrm{int}}\!\big(\mathbf{P}_i(\mathbf{G}_\tau F),\ \mathbf{P}_i(\mathbf{G}_\tau G)\big)\ \le\ d_{\mathrm{int}}\!\big(\mathbf{P}_i(F),\ \mathbf{P}_i(G)\big).
\]
\end{proposition}

\begin{proof}[Proof sketch]
\(C_\tau^{\mathrm{comb}}\) lifts \(\mathbf{T}_\tau\) (Appendix~B), hence \(\iota C_\tau^{\mathrm{comb}}\) lifts \(\iota_\tau \mathbf{T}_\tau\) at the persistence level; non\hyp expansiveness follows from Proposition~\ref{K:prop:exact-lip}.
\end{proof}

\begin{remark}[Scope and usage]\label{K:rk:scope}
The comonad \(\mathbf{G}_\tau\) is asserted only on the \emph{implementable range (up to f.q.i.)}, which suffices for algorithms and stability statements. All filtered (co)limit claims comply with Appendix~A, Remark~\ref{A:rk:filtered-colimits}.
\end{remark}

% ------------------------------------------------------------
\subsection*{K.3. Multi-axis torsion reflectors and soft-commuting policy}
% ------------------------------------------------------------
Beyond the length–threshold reflector \( \mathbf{T}_\tau \), one may consider other exact reflectors \(T_A,T_B:\Pers^{\mathrm{ft}}_k\to\Pers^{\mathrm{ft}}_k\) arising from hereditary Serre subcategories \(E_A,E_B\) (e.g.\ birth–window deletion, lifespan windowing, support cuts). This subsection specifies order–independence in the nested case and an \emph{operational} ``soft\hyp commuting'' policy in the general case.

\begin{definition}[Exact reflectors from Serre classes]
Let \(E\subset \Pers^{\mathrm{ft}}_k\) be a hereditary Serre subcategory. The Serre localization yields an exact reflector \(T_E:\Pers^{\mathrm{ft}}_k\to E^\perp\) left adjoint to the inclusion \(E^\perp\hookrightarrow \Pers^{\mathrm{ft}}_k\). We write \(T_A:=T_{E_A}\), \(T_B:=T_{E_B}\), and \(E_{A\vee B}\) for the Serre subcategory generated by \(E_A\cup E_B\).
\end{definition}

\begin{proposition}[Nested torsions \(\Rightarrow\) order independence]\label{K:prop:nested}
If \(E_A\subseteq E_B\) or \(E_B\subseteq E_A\), then
\[
T_A\circ T_B\ =\ T_B\circ T_A\ =\ T_{A\vee B}
\]
as exact reflectors on \(\Pers^{\mathrm{ft}}_k\).
In particular, for the 1D length thresholds, \(\mathbf{T}_\tau\circ \mathbf{T}_\sigma=\mathbf{T}_{\max\{\tau,\sigma\}}\).
\end{proposition}

\begin{proof}[Proof sketch]
In an abelian category, localization at nested Serre subcategories is idempotent and order–independent; both composites identify with the reflector at the join \(E_{A\vee B}\).
\end{proof}

\begin{definition}[A/B commutation defect and soft\hyp commuting policy]\label{K:def:soft}
For arbitrary exact reflectors \(T_A,T_B\) and a dataset \(M\in\Pers^{\mathrm{ft}}_k\), define the \emph{commutation defect}
\[
\Delta_{\mathrm{comm}}(M;A,B)\ :=\ d_{\mathrm{int}}\big(T_AT_BM,\ T_BT_AM\big).
\]
Fix a tolerance \(\eta\ge 0\) and a window (MECE, right–open). We say \(T_A,T_B\) are \emph{soft\hyp commuting} on the dataset if \(\Delta_{\mathrm{comm}}(M;A,B)\le \eta\) holds on the relevant instances (per window). Otherwise we \emph{fallback} to a fixed order (e.g.\ \(T_B\circ T_A\)).
\end{definition}

\begin{declaration}[Operational policy and $\delta$–ledger integration]\label{K:dec:soft-policy}
Within a window \(W\) and for a monitored degree \(i\):
\begin{enumerate}\itemsep0.2em
  \item \emph{Test.} Measure \(\Delta_{\mathrm{comm}}(M;A,B)\) at the persistence layer after applying any prerequisite collapse(s), using the window policy (Appendix~G). If \(\Delta_{\mathrm{comm}}\le \eta\), adopt \emph{soft\hyp commuting} and allow either order in that window.
  \item \emph{Fallback.} If \(\Delta_{\mathrm{comm}}>\eta\), fix an order (e.g.\ \(T_B\circ T_A\)) and \emph{record} the observed \(\Delta_{\mathrm{comm}}\) as an \(\delta^{\mathrm{alg}}\) entry in the \(\delta\)\hyp ledger (Appendix~G).
  \item \emph{Pipeline additivity.} Over a pipeline of such steps across windows, add the recorded defects to the window budgets \(\Sigma\delta\) (Appendix~G) and consume them in the Restart/Summability calculus (Appendix~J).
  \item \emph{Reproducibility.} Log \(\eta\), the chosen order (or the soft\hyp commuting verdict), and the measured \(\Delta_{\mathrm{comm}}\) per window in \texttt{run.yaml} (Appendix~G).
\end{enumerate}
\end{declaration}

\begin{remark}[Scope and guard\hyp rails]
We do \emph{not} assert any a priori commutation for non\hyp nested torsions; order control is purely \emph{operational} via the A/B test. All distances are measured \emph{after} truncation by \(\mathbf{T}_\tau\) and on the B\hyp side single layer, consistent with Chapter~1 and Appendix~L.
\end{remark}

% ------------------------------------------------------------
\subsection*{K.4. Interaction with Mirror/Transfer and pipeline $\delta$-budget}
% ------------------------------------------------------------
Let \(\Mirror\) be an admissible endofunctor at the filtered level (or on \(\Pers^{\mathrm{ft}}_k\)) satisfying the quantitative commutation hypotheses of Appendix~L: \(1\)\hyp Lipschitz on \(\mathbf{P}_i\) and a natural \(2\)\hyp cell \(\Mirror\circ C_\tau \Rightarrow C_\tau\circ \Mirror\) with a uniform bound \(\delta(i,\tau)\ge 0\) in \(d_{\mathrm{int}}\).

\begin{proposition}[Budget accounting with multiple (co)reflectors and Mirror]\label{K:prop:budget}
In a pipeline that interleaves exact reflectors \(T_\bullet\) (including \(\mathbf{T}_\tau\)) and Mirror/Transfer steps, the total commutation/ordering slack budget on a window is bounded by the \emph{additive} sum of:
\begin{itemize}\itemsep0.2em
  \item each Mirror–Collapse defect \(\delta(i,\tau)\) (Appendix~L),
  \item each A/B commutation defect \(\Delta_{\mathrm{comm}}(M;A,B)\) when soft\hyp commuting fails (Declaration~\ref{K:dec:soft-policy}),
\end{itemize}
and is \emph{non\hyp increasing} under any subsequent \(1\)\hyp Lipschitz post\hyp processing (e.g.\ further truncations, degree projections) at the persistence layer.
\end{proposition}

\begin{proof}[Proof sketch]
Apply the triangle inequality in \(d_{\mathrm{int}}\) along the pipeline, and use Appendix~L for Mirror–Collapse defects. A/B defects are measured directly at the persistence layer. \(1\)\hyp Lipschitz post\hyp processing does not increase distances (Appendix~A, Proposition~\ref{A:prop:lipschitz}).
\end{proof}

\begin{remark}[Window coherence]
All A/B tests and Mirror–Collapse commutation checks must be performed on the \emph{same} window (MECE, right\hyp open) and the \emph{same} collapse thresholds \(\tau\) as logged in \texttt{run.yaml} (Appendix~G) to ensure consistent budget accounting and pasting (Appendix~J).
\end{remark}

% ------------------------------------------------------------
\subsection*{K.5. Windowed usage and reproducibility (run.yaml alignment)}
% ------------------------------------------------------------
For each window \(W\) record in \texttt{run.yaml} (Appendix~G):
\begin{itemize}\itemsep0.2em
  \item the set of reflectors used (e.g.\ \texttt{length}, \texttt{birth\_window}), their order, and whether \emph{soft\hyp commuting} was adopted (\texttt{ab\_test} block with tolerance \(\eta\));
  \item the measured \(\Delta_{\mathrm{comm}}\) (if any), logged into the \(\delta\)\hyp ledger under \(\delta^{\mathrm{alg}}\);
  \item the Mirror/Transfer commutation bounds \(\delta(i,\tau)\) (Appendix~L) aggregated into \(\Sigma\delta\);
  \item the B\hyp Gate\(^{+}\) verdict and the safety margin \(\mathrm{gap}_\tau\), to be consumed by Restart/Summability (Appendix~J).
\end{itemize}

% ------------------------------------------------------------
\subsection*{K.6. Formalization stubs (Lean/Coq) [Spec]}
% ------------------------------------------------------------
A minimal API (cf.\ Appendix~F) includes:
\begin{itemize}\itemsep0.2em
  \item \texttt{reflector\_E} for a hereditary Serre subcategory \(E\), yielding an exact reflector \(T_E\) and the monad \(\iota_E T_E\) (idempotent);
  \item a lemma \texttt{nested\_torsion\_order\_indep} stating \(T_A\circ T_B=T_B\circ T_A=T_{A\vee B}\) when \(E_A\subseteq E_B\) or \(E_B\subseteq E_A\);
  \item a \texttt{comm\_defect} functional producing \(\Delta_{\mathrm{comm}}(M;A,B)=d_{\mathrm{int}}(T_AT_BM,T_BT_AM)\) with a policy hook for \emph{soft\hyp commuting} vs.\ fallback;
  \item a Mirror–Collapse \(2\)\hyp cell contract \texttt{mirror\_collapse\_delta} yielding \(\delta(i,\tau)\) and the additive pipeline bound under \(1\)\hyp Lipschitz post\hyp processing.
\end{itemize}

% ------------------------------------------------------------
\subsection*{K.7. Edge cases and pitfalls}
% ------------------------------------------------------------
\begin{itemize}\itemsep0.2em
  \item \emph{Non\hyp exact ``reflectors''.} The policy assumes exact reflectors (Serre localization). Heuristic truncations without exactness may break monad laws and invalidate stability claims.
  \item \emph{Inconsistent windows.} Running A/B tests or Mirror–Collapse checks on mismatched windows/\(\tau\) invalidates budget accounting and the pasting guarantee (Appendix~J).
  \item \emph{Chained non\hyp nested axes.} With \(T_A,T_B,T_C\) mutually non\hyp nested, pairwise soft\hyp commuting does not imply global confluence. Adopt a fixed canonical order (e.g.\ by axis priority), A/B test adjacent pairs, and log residuals.
\end{itemize}

\medskip
\noindent\textbf{Summary.}
On the persistence side, collapse is governed by the idempotent monad \(\iota_\tau\mathbf{T}_\tau\), exact and \(1\)\hyp Lipschitz.
On the filtered–homotopy side (implementable range, up to f.q.i.), collapse is governed by the idempotent comonad \(\iota\circ C_\tau^{\mathrm{comb}}\).
For \emph{multi–axis} torsion reflectors, order independence holds under nesting; otherwise, an A/B \emph{soft\hyp commuting} policy with a measured defect \(\Delta_{\mathrm{comm}}\) provides operational control, integrating into the additive pipeline \(\delta\)\hyp budget with Mirror/Transfer commutation (Appendix~L).
All steps are carried out \emph{after} collapse on the B\hyp side single layer, logged per window (Appendix~G), and pasted globally via Restart/Summability (Appendix~J).



% =========================
\section*{Appendix L. Quantitative Commutation for Mirror/Tropical [Spec + Pipeline Budget + A/B Policy]}
% =========================
\phantomsection
\addcontentsline{toc}{section}{Appendix L. Quantitative Commutation for Mirror/Tropical}
\refstepcounter{section}
\label{L:quant-comm}

\paragraph{Standing conventions.}
We work over a field \(k\); all persistence modules are constructible.
Filtered (co)limits are computed in \([\mathbb{R},\mathsf{Vect}_k]\) under the scope policy of Appendix~A, Remark~\ref{A:rk:filtered-colimits}, and then returned to the constructible range.
The interleaving metric \(d_{\mathrm{int}}\) (=\ bottleneck) is used throughout.
The truncation \(\mathbf{T}_\tau\) is exact and \(1\)-Lipschitz; shift–commutation implies \(1\)-Lipschitz (Appendix~A, Proposition~\ref{A:prop:lipschitz}).
On the filtered-complex side we write \(C_\tau\); on the persistence side \(\mathbf{T}_\tau\).
Global conventions: \(\Ext^1\) is always against \(k[0]\); the energy exponent satisfies \(\alpha>0\) (default \(\alpha=1\)); windows are MECE and right–open (Appendix~G).

% -------------------------
\subsection*{L.1. Hypotheses}
% -------------------------
\emph{[Spec]} Let \(\Mirror\) be a functor on filtered complexes (or, via \(\mathbf{P}_i\), a persistence-layer endofunctor). Assume:

\begin{itemize}\itemsep0.35em
  \item[(H1)] \textbf{\(1\)-Lipschitz of \(\Mirror\).} For every degree \(i\),
  \[
    d_{\mathrm{int}}\!\Big(\mathbf{P}_i(\Mirror F),\ \mathbf{P}_i(\Mirror G)\Big)
      \ \le\ d_{\mathrm{int}}\!\Big(\mathbf{P}_i(F),\ \mathbf{P}_i(G)\Big)
      \qquad\text{for all filtered complexes }F,G.
  \]
  \emph{(No extra assumption is needed for \(C_\tau\): by Appendix~B one has
  \(\mathbf{P}_i(C_\tau F)\cong \mathbf{T}_\tau(\mathbf{P}_iF)\), and \(\mathbf{T}_\tau\) is \(1\)-Lipschitz by Appendix~A, Proposition~\ref{A:prop:lipschitz}.)}
  \item[(H2)] \textbf{\(\delta\)-controlled natural \(2\)-cell.} There exists a natural \(2\)-cell
  \[
    \theta:\ \Mirror\!\circ\! C_\tau\ \Rightarrow\ C_\tau\!\circ\!\Mirror
  \]
  (interpreted up to f.q.i.\ in \(\Ho(\mathsf{FiltCh}(k))\); cf.\ Appendix~B) such that, \emph{uniformly in \(F\)}, for all \(i,\tau\),
  \[
    d_{\mathrm{int}}\!\Big(\mathbf{P}_i(\Mirror(C_\tau F)),\ \mathbf{P}_i(C_\tau(\Mirror F))\Big)\ \le\ \delta(i,\tau).
  \]
\end{itemize}

\begin{remark}[Interpretation]
(H2) measures, at the persistence layer, the failure of \(\Mirror\) and \(C_\tau\) to commute; the control \(\delta(i,\tau)\ge 0\) may depend on \((i,\tau)\) but not on \(F\). It is an external assumption; no intrinsic value of \(\delta\) is claimed without additional structure.
\end{remark}

\begin{remark}[Strict commutation as a special case]
If \(\delta(i,\tau)=0\) (e.g.\ \(\theta\) induces isomorphisms after applying \(\mathbf{P}_i\)), then \(\Mirror\) and \(C_\tau\) commute up to isomorphism at the truncated persistence layer in degree \(i\) and window \(\tau\).
\end{remark}

% -------------------------
\subsection*{L.2. Conditional bound}
% -------------------------
\begin{theorem}[Quantitative commutation under (H1)–(H2), uniform in \(F\)]\label{L:thm:cond}
Under \emph{(H1)} and \emph{(H2)}, for all filtered complexes \(F\), all \(i\), and all \(\tau\),
\[
  d_{\mathrm{int}}\!\Big(
      \mathbf{T}_\tau\,\mathbf{P}_i(\Mirror(C_\tau F))\ ,\ 
      \mathbf{T}_\tau\,\mathbf{P}_i(C_\tau(\Mirror F))\n  \Big)\ \le\ \delta(i,\tau),
\]
and the bound is \emph{uniform in \(F\)}.
\end{theorem}

\begin{proof}[Proof sketch]
By (H2),
\(d_{\mathrm{int}}\big(\mathbf{P}_i(\Mirror(C_\tau F)),\mathbf{P}_i(C_\tau(\Mirror F))\big)\le\delta(i,\tau)\).
Applying \(\mathbf{T}_\tau\), which is \(1\)-Lipschitz (Appendix~A, Proposition~\ref{A:prop:lipschitz}), yields the stated inequality.
Since \(\mathbf{T}_\tau\) is exact, \emph{no additional stability loss} is introduced for downstream kernel/cokernel diagnostics.\footnote{Exactness does not turn a small distance bound into equality of kernels/cokernels; it ensures that truncation itself adds no extra error to such diagnostics.}
\end{proof}

\begin{corollary}[Stability under additional \(1\)-Lipschitz post-processing]
If \(\Phi\) is any endofunctor on persistence modules with \(d_{\mathrm{int}}(\Phi X,\Phi Y)\le d_{\mathrm{int}}(X,Y)\), then for all \(i,\tau,F\),
\[
  d_{\mathrm{int}}\!\Big(
      \Phi\,\mathbf{T}_\tau\,\mathbf{P}_i(\Mirror(C_\tau F))\ ,\ 
      \Phi\,\mathbf{T}_\tau\,\mathbf{P}_i(C_\tau(\Mirror F))\n  \Big)\ \le\ \delta(i,\tau).
\]
\end{corollary}

\begin{corollary}[Two-stage pipeline; additive control]\label{L:cor:additive}
Let \(\Mirror_1,\Mirror_2\) satisfy \emph{(H1)} and admit \(2\)-cells with controls \(\delta_1,\delta_2\) as in \emph{(H2)}. Then for all \(i,\tau,F\),
\[
  d_{\mathrm{int}}\!\Big(
      \mathbf{T}_\tau\,\mathbf{P}_i(\Mirror_2\Mirror_1(C_\tau F))\ ,\ 
      \mathbf{T}_\tau\,\mathbf{P}_i(C_\tau(\Mirror_2\Mirror_1 F))\n  \Big)\ \le\ \delta_1(i,\tau)+\delta_2(i,\tau).
\]
\emph{Proof sketch.} Insert the intermediate term
\(\mathbf{T}_\tau\mathbf{P}_i(\Mirror_2(C_\tau\Mirror_1 F))\) and apply the triangle inequality. Use Theorem~\ref{L:thm:cond} for \(\Mirror_1\) with the \(1\)-Lipschitz post-processing \(\Phi=\mathbf{T}_\tau\mathbf{P}_i\Mirror_2\) to bound the first leg by \(\delta_1\), and Theorem~\ref{L:thm:cond} for \(\Mirror_2\) to bound the second leg by \(\delta_2\).
\end{corollary}

\begin{corollary}[Strict commutation and tower diagnostics]
If \(\delta(i,\tau)=0\) for all \(i,\tau\) (i.e.\ \(\Mirror\) and \(C_\tau\) commute up to isomorphism at the truncated persistence layer), then for any tower \(\{F_n\}\to F_\infty\) and any \(i,\tau\),
the comparison maps computed after either order agree up to isomorphism:
\[
\phi_{i,\tau}\big(\Mirror(C_\tau F_\bullet)\big)\ \cong\ \phi_{i,\tau}\big(C_\tau(\Mirror F_\bullet)\big).
\]
Consequently, the tower obstruction indices are preserved,
\[
\mu_{i,\tau}\big(\Mirror(C_\tau F_\bullet)\big)=\mu_{i,\tau}\big(C_\tau(\Mirror F_\bullet)\big),\qquad\n\nu_{i,\tau}\big(\Mirror(C_\tau F_\bullet)\big)=\nu_{i,\tau}\big(C_\tau(\Mirror F_\bullet)\big),
\]
and likewise for the totals \((\muc,\nuc)\).
\end{corollary}

\begin{remark}[Typical sources of \(\delta=0\)]
If \(\Mirror\) preserves the filtration and there is a filtration–natural isomorphism \(\Mirror\circ C_\tau \cong C_\tau\circ\Mirror\) (e.g.\ \(\Mirror\) is induced by a reindexing monotone endomorphism of \((\mathbb{R},\le)\) that fixes the window cut, or by a direct-sum decomposition functor that respects sublevel sets), then \(\delta(i,\tau)=0\) for all \(i,\tau\).
\end{remark}

\begin{remark}[Uniformity across degrees]
If \(\Mirror\) admits a degreewise \(1\)-Lipschitz constant independent of \(i\) on a fixed finite set of degrees \(I\subset\mathbb{Z}\), then the commutation control may be chosen \(\delta(i,\tau)\le \delta_I(\tau)\) uniformly for \(i\in I\).
\end{remark}

% -------------------------
\subsection*{L.3. Counterexample notes (necessity of assumptions)}
% -------------------------
\begin{remark}[Why the hypotheses are needed]
Without a \(\delta\)-controlled natural \(2\)-cell (H2), individual non-expansiveness of \(\Mirror\) and \(C_\tau\) does \emph{not} constrain the distance between the two composites:
triangle inequalities do not relate \( \Mirror\!\circ\! C_\tau \) and \( C_\tau\!\circ\!\Mirror\) in the absence of a comparison map.
Concretely, one can build (Appendix~D) towers whose bar lengths accumulate from below at \(\tau\) (Type~IV phenomena).
Collapsing first may erase the near-\(\tau\) mass, while mirroring first may convert it into persistent features that survive collapse; the gap can be made arbitrarily large despite each functor being \(1\)-Lipschitz on its own.
Thus no quantitative bound of the form in Theorem~\ref{L:thm:cond} is available in general unless (H2) (or an equivalent commutation control) is imposed.
\end{remark}

\begin{remark}[Scope]
All claims are made under the constructible scope and filtered-(co)limit policy of Appendix~A.
When \(\Mirror\) is only defined up to filtered quasi-isomorphism, (H2) is interpreted in \(\Ho(\mathsf{FiltCh}(k))\), and distances are taken after applying \(\mathbf{P}_i\).
\end{remark}

% -------------------------
\subsection*{L.4. Pipeline error budget and windowed accounting}
% -------------------------
We integrate quantitative commutation into a \emph{windowed} error budget that is additive along pipelines, uniform in the input \(F\), and non-increasing under any subsequent \(1\)\hyp Lipschitz post\hyp processing at the persistence layer.

\begin{definition}[Per-step commutation bounds and window budget]\label{L:def:budget}
Consider a pipeline on a window \(W=[u,u')\) consisting of steps
\[
\Pi\ :=\ \Big(\ \cdots\ \rightarrow\ \Mirror_j\ \xrightarrow{}\ C_{\tau_j}\ \xrightarrow{}\ T_{A_j/B_j}\ \xrightarrow{}\ \cdots\ \Big),
\]
where \(\Mirror_j\) are Mirror/Transfer steps, \(C_{\tau_j}\) are collapses (filtered lift of \(\mathbf{T}_{\tau_j}\)), and \(T_{A_j/B_j}\) are exact reflectors on \(\Pers^{\mathrm{ft}}_k\) (Appendix~K). For each Mirror–Collapse pair, let \(\delta_j(i,\tau_j)\) be the (H2) bound on degree \(i\). For each pair of reflectors \(T_A,T_B\) that are applied in some order with an \emph{A/B test} (Appendix~K), let \(\Delta_{\mathrm{comm}}(M;A,B)\) be the measured commutation defect (Definition~\ref{K:def:soft}); when \(\Delta_{\mathrm{comm}}>\eta\), we fallback to a fixed order and record \(\Delta_{\mathrm{comm}}\) as an \(\delta^{\mathrm{alg}}\) contribution. The \emph{window budget} is the additive sum of all such contributions on \(W\):
\[
\Sigma\delta_W(i)\ :=\ \sum_{j}\delta_j(i,\tau_j)\ +\ \sum_{\text{A/B tested pairs}}\ \Delta_{\mathrm{comm}}(M;A,B)\ \ \ (\text{when exceeded tolerance}).
\]
\end{definition}

\begin{theorem}[Pipeline budget: additivity, uniformity, and post-processing stability]\label{L:thm:budget}
Fix a window \(W\), a monitored degree \(i\), and a pipeline \(\Pi\) as in Definition~\ref{L:def:budget}. For any filtered input \(F\),
\[
d_{\mathrm{int}}\!\Big(
      \mathbf{T}_\tau\,\mathbf{P}_i(\Pi_{\text{lhs}}(F))\ ,\ 
      \mathbf{T}_\tau\,\mathbf{P}_i(\Pi_{\text{rhs}}(F))\n  \Big)\ \le\ \Sigma\delta_W(i),
\]
where the left/right versions aggregate Mirror–Collapse orderings and A/B reflector orderings according to the window’s operational choices. The bound:
\begin{itemize}\itemsep0.2em
  \item is \emph{additive} in the per-step Mirror–Collapse bounds \(\delta_j(i,\tau_j)\) and the reflector A/B defects \(\Delta_{\mathrm{comm}}\);
  \item is \emph{uniform in \(F\)} (each \(\delta_j\) is uniform by (H2), and \(\Delta_{\mathrm{comm}}\) are measured at the persistence layer independent of representatives up to f.q.i.);
  \item is \emph{non-increasing} under any subsequent \(1\)\hyp Lipschitz post\hyp processing on the persistence layer (e.g.\ further truncations \(\mathbf{T}_{\tau'}\), degree projections \(\mathbf{P}_i\), shifts \(S^\varepsilon\)).
\end{itemize}
\end{theorem}

\begin{proof}[Proof sketch]
Compose the natural \(2\)\hyp cells for Mirror–Collapse steps and measure each A/B reflector defect in \(d_{\mathrm{int}}\); apply the triangle inequality to obtain additivity. Uniformity follows from (H2) being uniform in \(F\) and from the fact that A/B defects are computed \emph{after} applying \(\mathbf{P}_i\) (hence f.q.i.\ invariant). Post\hyp processing stability comes from the \(1\)\hyp Lipschitz property of the applied persistence functors (Appendix~A).
\end{proof}

\begin{remark}[run.yaml alignment and B-Gate\(^{+}\)]
Record each \(\delta_j(i,\tau_j)\) and \(\Delta_{\mathrm{comm}}\) per window in the \(\delta\)\hyp ledger (Appendix~G); aggregate into \(\Sigma\delta_W(i)\) and compare to the safety margin \(\mathrm{gap}_\tau\) for B-Gate\(^{+}\) on the same window and degree (Chapter~1). This integrates with Restart/Summability for pasting (Appendix~J).
\end{remark}

% -------------------------
\subsection*{L.5. Operational A/B policy (test \(\Rightarrow\) fallback \(\Rightarrow\) \(\delta_{\mathrm{alg}}\) accounting)}
% -------------------------
We adopt the \emph{soft\hyp commuting} policy of Appendix~K at the persistence layer after collapse, with explicit windowed logging.

\begin{definition}[A/B commutativity test]\label{L:def:ab}
Given exact reflectors \(T_A,T_B\) (Appendix~K) and a dataset \(M\in\Pers^{\mathrm{ft}}_k\) on window \(W\), define
\[
\Delta_{\mathrm{comm}}(M;A,B)\ :=\ d_{\mathrm{int}}\big(T_AT_BM,\ T_BT_AM\big).
\]
Fix a tolerance \(\eta\ge 0\). If \(\Delta_{\mathrm{comm}}\le \eta\), we say \(T_A,T_B\) are \emph{soft\hyp commuting} on \(W\).
\end{definition}

\begin{declaration}[Windowed A/B procedure]\label{L:dec:ab}
On a window \(W\) and degree \(i\):
\begin{enumerate}\itemsep0.2em
  \item \emph{Measure} \(\Delta_{\mathrm{comm}}(M;A,B)\) on \(\mathbf{T}_\tau\mathbf{P}_i\) (same window and \(\tau\) as the budget).
  \item \emph{Accept} soft\hyp commuting if \(\Delta_{\mathrm{comm}}\le\eta\); otherwise \emph{fallback} to a fixed order (e.g.\ \(T_B\circ T_A\)).
  \item \emph{Record} \(\Delta_{\mathrm{comm}}\) as an \(\delta^{\mathrm{alg}}\) contribution if soft\hyp commuting fails; add it to \(\Sigma\delta_W(i)\).
  \item \emph{Log} \(\eta\), the decision (soft\hyp commute or fixed order), and the numerical \(\Delta_{\mathrm{comm}}\) in \texttt{run.yaml} (Appendix~G).
\end{enumerate}
\end{declaration}

\begin{proposition}[Combined bound with A/B policy]\label{L:prop:combined}
For a window \(W\) with the A/B procedure of Declaration~\ref{L:dec:ab}, the pipeline bound in Theorem~\ref{L:thm:budget} holds with
\[
\Sigma\delta_W(i)\ =\ \sum_{j}\delta_j(i,\tau_j)\ +\ \sum_{\text{A/B fails}} \Delta_{\mathrm{comm}}(M;A,B),
\]
and is non\hyp increasing under any subsequent \(1\)\hyp Lipschitz persistence post\hyp processing.
\end{proposition}

\begin{proof}[Proof sketch]
Immediate from Theorem~\ref{L:thm:budget} by including the A/B residuals as per\hyp step additive costs.
\end{proof}

\begin{remark}[Best practices]
Use a canonical ordering for multiple axes (e.g.\ priority by axis type) and apply A/B tests only to adjacent pairs; record all policies and tolerances. For nested torsions (Appendix~K, Proposition~\ref{K:prop:nested}), skip the test (order independence holds).
\end{remark}

% -------------------------
\subsection*{L.6. Edge cases, pitfalls, and window coherence}
% -------------------------
\begin{itemize}\itemsep0.25em
  \item \emph{Mismatched windows or thresholds.} All commutation bounds must be computed on the \emph{same} window and collapse thresholds \(\tau\) as used for B-Gate\(^{+}\) and for the \(\delta\)\hyp ledger; otherwise budget accounting and pasting (Appendix~J) become invalid.
  \item \emph{Degree mixing.} When reporting a total over degrees (e.g.\ \(\muc,\nuc\)), a small A/B residual in one degree can be obscured. Keep both per\hyp degree and aggregated logs (Appendix~G).
  \item \emph{Cascaded non\hyp commutation.} Pairwise soft\hyp commuting does not imply global commutation across three or more axes; use a fixed canonical order and log all residuals.
\end{itemize}

\medskip
\noindent\textbf{Summary.}
If \(\Mirror\) is \(1\)\hyp Lipschitz and there is a natural \(2\)\hyp cell controlling its commutator with \(C_\tau\) by \(\delta(i,\tau)\) (uniform in \(F\)), then truncation preserves this control:
\[
d_{\mathrm{int}}\Big(\mathbf{T}_\tau\mathbf{P}_i(\Mirror C_\tau F),\ \mathbf{T}_\tau\mathbf{P}_i(C_\tau\Mirror F)\Big)\ \le\ \delta(i,\tau).
\]
For multi-stage pipelines the controls \emph{add}, are \emph{uniform in \(F\)}, and are \emph{non\hyp increasing} under \(1\)\hyp Lipschitz post\hyp processing (Theorem~\ref{L:thm:budget}).
Operationally, A/B tests for non\hyp nested reflectors provide a \emph{soft\hyp commuting} decision per window; failures fall back to a fixed order and are logged as \(\delta^{\mathrm{alg}}\) contributions (Declaration~\ref{L:dec:ab}, Proposition~\ref{L:prop:combined}).
All items are computed \emph{after} collapse on the B\hyp side single layer, logged per window (\texttt{run.yaml}, Appendix~G), and pasted globally via Restart/Summability (Appendix~J).



% =========================
\section*{Appendix M. (Optional) Lax Monoidal Compatibility [Spec + Windowed Usage + Budget Integration]}
% =========================
\phantomsection
\addcontentsline{toc}{section}{Appendix M. (Optional) Lax Monoidal Compatibility}
\refstepcounter{section}
\label{M:lax-monoidal}

\paragraph{Standing conventions.}
We work over a field \(k\); all persistence modules are \emph{constructible} (locally finite on bounded windows).
Filtered (co)limits are computed objectwise in \([\mathbb{R},\mathsf{Vect}_k]\) under the scope policy of Appendix~A, Remark~\ref{A:rk:filtered-colimits}, and then returned to the constructible range when stated.
The interleaving metric \(d_{\mathrm{int}}\) (=\ bottleneck in the constructible 1D setting) is used throughout.
Truncation \(\mathbf{T}_\tau\) is exact and \(1\)\hyp Lipschitz; shift–commutation implies \(1\)\hyp Lipschitz (Appendix~A, Proposition~\ref{A:prop:lipschitz}).
On the filtered–complex side we write \(C_\tau\) (defined up to f.q.i.); on the persistence side \(\mathbf{T}_\tau\).
Global conventions: \(\Ext^1\) is always taken against \(k[0]\); the energy exponent satisfies \(\alpha>0\) (default \(\alpha=1\)); windows are MECE and right–open (Appendix~G).

\paragraph{Scope.}
All claims in this appendix are \textbf{[Spec]}; we provide windowed, reproducible contracts that are compatible with the proven core (Appendix~A/H/J) and with the operational budget calculus (Appendix~L), but we do \emph{not} assert strict monoidality of collapse beyond the stated assumptions.

% -------------------------
\subsection*{M.1. Hypotheses and the laxator}
% -------------------------
Let \(\otimes\) denote the \emph{pointwise} tensor on filtered complexes and on persistence modules:
\((M\otimes N)(t)\coloneqq M(t)\otimes_k N(t)\).
Assume:
\begin{itemize}\itemsep0.35em
  \item[(M1)] \textbf{Exact, constructible tensor.} The tensor \(\otimes\) is exact and biadditive, and preserves constructibility: on any bounded window only finitely many bars meet, hence \(\dim_k(M(t)\otimes_k N(t))<\infty\) and the set of critical parameters is locally finite (finite union). By local finiteness, sums/integrals interchange as in Appendix~H (Tonelli applies on bounded windows). A finite union of locally finite event sets is locally finite; ergo \(\otimes\) preserves constructibility.
  \item[(M2)] \textbf{Künneth over a field (pointwise).} For each \(t\),
  \[
  H_i\!\big((F\otimes G)^t\big)\ \cong\ \bigoplus_{p+q=i} H_p(F^t)\otimes_k H_q(G^t),
  \]
  naturally in \((F,G)\) and \(t\) (Tor-terms vanish over a field).
  \item[(M3)] \textbf{Lax compatibility for collapse.} There is a natural transformation (laxator) in the homotopy setting (up to f.q.i.; Appendix~B)
  \[
     \lambda_{\tau,F,G}:\ C_\tau(F\otimes G)\ \Longrightarrow\ C_\tau F\ \otimes\ C_\tau G,
  \]
  natural in \(F,G\) and \(\tau\).
\end{itemize}
We do \emph{not} assume invertibility or metric control of \(\lambda\) unless stated.
When needed, we strengthen (M3) to:
\[
\text{(M3$^{+}$):\quad for all \(t,i\), }H_i(\lambda_{\tau,F,G}^t)\text{ is a monomorphism (equivalently, rank-non\hyp decreasing).}
\]

\begin{remark}[Intervals under tensor]\label{M:rk:interval}
For interval modules over \(k\),
\(k_{[a,b)}\otimes k_{[c,d)} \cong k_{[\max\{a,c\},\,\min\{b,d\})}\) if the intervals overlap, and \(0\) otherwise.
Thus tensor intersects lifespans; this restates (M1)–(M2) at the barcode level (Tor vanishes).
\end{remark}

\begin{remark}[Monotonicity scope]
Tensor is neither purely deletion–type nor inclusion–type in general (cf.\ Appendix~E). This appendix confines monotone statements to \emph{windowed energy upper bounds} and to the \emph{(M3$^{+}$)} regime where the laxator is pointwise mono.
\end{remark}

% -------------------------
\subsection*{M.2. Windowed energy via overlap integrals}
% -------------------------
Recall the clipped Betti integral (Appendix~H): for a degree \(i\) and window \([0,\sigma]\),
\[
\mathrm{PE}_i^{\le \sigma}(F)\ :=\ \int_0^\sigma \beta_i(F;t)\,dt,\qquad \beta_i(F;t):=\dim_k H_i(F^t).
\]

\begin{theorem}[Convolution-type upper bound]\label{M:thm:conv}
Under \emph{(M1)–(M2)}, for all filtered complexes \(F,G\), degrees \(i\), and windows \(\sigma>0\),
\[
  \mathrm{PE}_i^{\le \sigma}(F\otimes G)\ \le\ \sum_{p+q=i}\ \int_0^\sigma \beta_p(F;t)\,\beta_q(G;t)\,dt,
\]
and the sum \(\sum_{p+q=i}\) is finite for each fixed \(i\) (bounded homological degree in our constructible setting).
Consequently,
\[
  \mathrm{PE}_i^{\le \sigma}(F\otimes G)\ \le\ \sum_{p+q=i}\Big(\sup_{t\in[0,\sigma]}\beta_q(G;t)\Big)\,\mathrm{PE}_p^{\le \sigma}(F),
\]
and symmetrically in \((F,p)\leftrightarrow(G,q)\).
\end{theorem}

\begin{proof}[Proof sketch]
By (M2), \(\beta_i(F\otimes G;t)=\sum_{p+q=i}\beta_p(F;t)\beta_q(G;t)\) whenever Künneth isomorphism holds; we keep the \(\le\) sign to cover non\hyp Künneth contexts. Integrate over \(t\in[0,\sigma]\) (Tonelli). Finiteness follows from constructibility.
\end{proof}

\begin{remark}[When equality holds]\label{M:rk:eq}
With field coefficients and pointwise Künneth isomorphism, equality holds:
\(\mathrm{PE}_i^{\le \sigma}(F\otimes G)=\sum_{p+q=i}\int_0^\sigma \beta_p(F;t)\beta_q(G;t)\,dt.\)
\end{remark}

\begin{corollary}[Bounds for collapsed tensors]\label{M:cor:collapsed}
With \emph{(M1)–(M3)} and any \(\tau,\sigma>0\),
\[
  \mathrm{PE}_i^{\le \sigma}\!\big(C_\tau F \otimes C_\tau G\big)\ \le\ \sum_{p+q=i}\ \int_0^\sigma \beta_p(C_\tau F;t)\,\beta_q(C_\tau G;t)\,dt.
\]
If, in addition, \emph{(M3$^{+}$)} holds (pointwise homology-mono for \(\lambda_{\tau,F,G}\)), then
\[
  \mathrm{PE}_i^{\le \sigma}\!\big(C_\tau(F\otimes G)\big)\ \le\ \mathrm{PE}_i^{\le \sigma}\!\big(C_\tau F \otimes C_\tau G\big).
\]
\end{corollary}

\begin{proof}[Proof sketch]
Apply Theorem~\ref{M:thm:conv} to \(C_\tau F,C_\tau G\). Exactness and \(1\)\hyp Lipschitz stability of \(C_\tau\) justify using their Betti curves. Under (M3$^{+}$), \(H_i(\lambda_{\tau,F,G}^t)\) is mono, hence the windowed integral cannot increase along \(\lambda\).
\end{proof}

% -------------------------
\subsection*{M.3. Windowed usage and run.yaml alignment}
% -------------------------
All energy bounds and comparisons in this appendix are \emph{windowed}. Use the same MECE, right–open windows declared in \texttt{run.yaml} (Appendix~G), and record:
\begin{itemize}\itemsep0.25em
  \item the tensor experiment context (pairs \((F,G)\), monitored degrees \(i\), windows \([0,\sigma]\), and collapse thresholds \(\tau\));
  \item whether Künneth (M2) is assumed/verified on the chosen windows;
  \item whether (M3) and optionally (M3$^{+}$) are assumed/verified on the chosen windows (e.g.\ monomorphy checks of \(H_i(\lambda^t)\) at sampled \(t\));
  \item numeric tolerances for spectra/Betti integrations and any clipping thresholds (Appendix~G).
\end{itemize}
When combining with reflectors or Mirror/Transfer in a pipeline, log the order and budget contributions per window (Appendix~L/K).

% -------------------------
\subsection*{M.4. Interaction with $\tau$-sweep and stability bands}
% -------------------------
Fix a window and degree \(i\). A \(\tau\)-sweep (Appendix~J) detects stability bands where \((\mu,\nu)=(0,0)\) at scale \(\tau\).
Within such a band, the energy comparison
\[
\mathrm{PE}_i^{\le \sigma}\!\big(C_\tau(F\otimes G)\big)\ \le\ \mathrm{PE}_i^{\le \sigma}\!\big(C_\tau F \otimes C_\tau G\big)
\]
(in the (M3$^{+}$) regime) is particularly robust: \(\phi_{i,\tau}\) is an isomorphism and budget–driven drift is controlled by Appendix~L. Record the chosen \(\tau\) from the band in \texttt{run.yaml}.

% -------------------------
\subsection*{M.5. Interaction with reflectors and pipeline budget}
% -------------------------
Lax monoidal steps may be interleaved with exact reflectors (Appendix~K) and Mirror/Transfer (Appendix~L). The \emph{pipeline budget} (Appendix~L, Theorem~\ref{L:thm:budget}) captures commutation defects additively; subsequent \(1\)\hyp Lipschitz processing (e.g.\ \(\mathbf{T}_\tau\), degree projections) cannot increase the budget.
If a quantitative bound for \(\lambda\) at the persistence layer is available (e.g.\ via a measured \(d_{\mathrm{int}}\) between \(\mathbf{T}_\tau\mathbf{P}_i(C_\tau(F\otimes G))\) and \(\mathbf{T}_\tau\mathbf{P}_i(C_\tau F\otimes C_\tau G)\)), include it as a \(\delta\)\hyp term in the window’s \(\delta\)\hyp ledger (Appendix~G) and integrate it into \(\Sigma\delta\) (Appendix~L). If no metric control is available, rely solely on the energy inequalities above and budget only Mirror/Transfer and A/B residuals.

% -------------------------
\subsection*{M.6. Edge cases and pitfalls}
% -------------------------
\begin{itemize}\itemsep0.25em
  \item \emph{Non\hyp exact ``tensor'' surrogates.} The contracts assume exactness (M1); heuristic aggregations that are not exact can break Künneth and invalidate the overlap bounds.
  \item \emph{Unverified Künneth.} If (M2) is not verified, keep the \(\le\) form (Theorem~\ref{M:thm:conv}) and avoid equality claims.
  \item \emph{Missing monomorphy.} Without (M3$^{+}$), the comparison \(C_\tau(F\otimes G)\to C_\tau F\otimes C_\tau G\) need not be energy–nonincreasing; only the two–sided inequality from Theorem~\ref{M:thm:conv} is safe.
  \item \emph{Window mismatch.} All measurements (Betti integrals, \(\lambda\) checks) must use the same MECE, right–open windows and the same \(\tau\) as the rest of the run (Appendix~G); otherwise pasting (Appendix~J) becomes invalid.
\end{itemize}

% -------------------------
\subsection*{M.7. Examples}
% -------------------------
\begin{itemize}\itemsep0.25em
  \item \emph{Single\hyp interval bars.} If \(\mathbf{P}_p(F)=k_{[a,b)}\), \(\mathbf{P}_q(G)=k_{[c,d)}\) and other degrees vanish, then
  \[
  \mathbf{P}_{p+q}(F\otimes G)\cong k_{[\max\{a,c\},\min\{b,d\})},\quad \mathrm{PE}_{p+q}^{\le \sigma}(F\otimes G)=\lambda\!\big([\max\{a,c\},\min\{b,d\})\cap[0,\sigma]\big).
  \]
  \item \emph{Collapsed intervals.} If \(b-a\le \tau\) or \(d-c\le \tau\), then \(C_\tau F\) or \(C_\tau G\) kills the corresponding bar, so the overlap integral bounds reduce accordingly; if both are killed, the right–hand side is \(0\) and the left must be \(0\) under (M3$^{+}$).
\end{itemize}

% -------------------------
\subsection*{M.8. Formalization stubs (Lean/Coq) [Spec]}
% -------------------------
A minimal API (cf.\ Appendix~F) includes:
\begin{itemize}\itemsep0.25em
  \item an exact, biadditive \texttt{tensor} on \(\Pers^{\mathrm{ft}}_k\) preserving constructibility and a \texttt{kun\_neth} lemma yielding \(\beta_i(F\otimes G;t)=\sum_{p+q=i}\beta_p(F;t)\beta_q(G;t)\) (under field coefficients);
  \item a \texttt{laxator} natural transformation \(\lambda_{\tau,F,G}:C_\tau(F\otimes G)\Rightarrow C_\tau F\otimes C_\tau G\);
  \item an optional \texttt{laxator\_mono} hypothesis (M3$^{+}$) as a degreewise mono condition;
  \item the overlap--integral bounds \texttt{PE\_conv\_bound} and \texttt{PE\_collapsed\_bound} matching Theorem~\ref{M:thm:conv} and Corollary~\ref{M:cor:collapsed}, with window parameters and MECE policy as explicit arguments;
  \item hooks to record optional metric gaps as \(\delta\)\hyp terms for the pipeline budget (Appendix~L).
\end{itemize}

\medskip
\noindent\textbf{Summary.}
Under exact pointwise tensor and a collapse–compatible laxator \(C_\tau(F\otimes G)\Rightarrow C_\tau F\otimes C_\tau G\), the clipped Betti integral of a tensor admits robust, \emph{windowed} upper bounds expressed by overlap integrals of Betti curves. With the additional (M3$^{+}$) monomorphy, \(C_\tau(F\otimes G)\) is energy–dominated by \(C_\tau F\otimes C_\tau G\) on each window. These contracts are compatible with \(\tau\)\hyp sweeps and stability bands (Appendix~J), with the pipeline error budget and A/B policy (Appendix~L/K), and with the reproducibility schema (Appendix~G). No stronger monoidality is claimed beyond these [Spec] bounds, and no further supplementation is required for operational, windowed use within the v15.0 guard–rails.



% =========================
\section*{Appendix N. Projection Formula and Base Change [Spec + Windowed Protocol + Budget Integration]}
% =========================
\phantomsection
\addcontentsline{toc}{section}{Appendix N. Projection Formula and Base Change}
\refstepcounter{section}
\label{N:pf-bc}

\paragraph{Standing conventions.}
We work over a coefficient \emph{field} \(\Lambda\) (e.g.\ a base field \(k\) or, at [Spec]-level, a Novikov field), and all statements below are phrased uniformly for \(\Lambda\).
All persistence modules are constructible (locally finite on bounded windows).
Filtered (co)limits are computed objectwise in \([\mathbb{R},\mathsf{Vect}_\Lambda]\) under the scope policy of Appendix~A, Remark~\ref{A:rk:filtered-colimits}, and then (when stated) returned to the constructible range.
The interleaving metric \(d_{\mathrm{int}}\) (=\ bottleneck in the constructible \(1\)D setting) is used throughout.
Truncation \(\mathbf{T}_\tau\) is exact and \(1\)\hyp Lipschitz (Appendix~A, Proposition~\ref{A:prop:lipschitz}); we keep the filtered-complex functor \(C_\tau\) \emph{up to f.q.i.} and write \(\mathbf{P}_i\) for degree–\(i\) persistence.
Global conventions: \(\Ext^1\) is always taken against \(\Lambda[0]\) (so we write \(\Ext^1(\mathcal{R}(C_\tau F),\Lambda)=0\)); the energy exponent satisfies \(\alpha>0\) (default \(\alpha=1\)); windows are MECE and right–open (Appendix~G).
References to “infinite bars/generic dimension” point to Appendix~D, Remark~\ref{rem:D-generic-dim}.
Monotonicity claims follow the global policy: deletion-type only (nonincreasing), inclusion-type merely stable/nonexpansive (Appendix~E).

% Number subsections in Appendix N as N.1, N.2, ...
\setcounter{subsection}{0}
\renewcommand\thesubsection{N.\arabic{subsection}}
\makeatletter
\renewcommand\@seccntformat[1]{\csname the#1\endcsname.\quad}
\makeatother

% -------------------------
\subsection{Hypotheses (PF/BC layer) and normalizations}
\label{N:hyp}
% -------------------------
We fix a class of filtered spaces and maps \(f:X\to Y\) for which the usual six–functor formalism is available on
\(D^b_c(\mathrm{Shv}_\Lambda(-))\) (bounded derived category of constructible \(\Lambda\)–sheaves), and adopt:

\begin{itemize}\itemsep0.35em
  \item[(N0)] \textbf{Coefficients.} \(\Lambda\) is a field; all objects have \emph{finite Tor-dimension} (no \(\mathrm{Tor}\) corrections).
  \item[(N1)] \textbf{Finiteness/constructibility.} All sheaves are constructible; the standard \(t\)-structure is used; per-object (co)homology is finite dimensional.
  \item[(N2)] \textbf{Proper/smooth hypotheses.} Projection formula (PF) and base change (BC) are taken under the usual hypotheses:
  \begin{itemize}\itemsep0.1em
    \item PF: for \(f\) \emph{proper}, \(Rf_\ast(A\otimes^\mathbf{L} f^\ast B)\simeq Rf_\ast A\otimes^\mathbf{L} B\).
    \item BC: for a Cartesian square with \(f\) proper (or smooth with the appropriate \(f^!\) variant),
    \(Lg^\ast Rf_\ast A \simeq Rf'_\ast Lg'{}^\ast A\).
  \end{itemize}
  \item[(N3)] \textbf{Degree normalization and objectwise evaluation.}
  We use the \emph{cohomological} convention on \(D^b_c\) and evaluate realizations \emph{objectwise in \(t\)}:
  \[
     \mathcal{R}(F)^t \cong \mathcal{R}(F^t),\qquad
       \mathbf{P}_i(F)(t)\ \cong\ H_i(F^t)\ \cong\ H^{-i}\!\big(\mathcal{R}(F^t)\big).
  \]
  Hence \(\mathbf{P}_i\) reads off the \((-i)\)-th cohomology sheaf along the filtration.
  Any geometric shift from \(f^!\) (smooth case) is \emph{absorbed} by this bookkeeping.
  \item[(N4)] \textbf{Tensor.} The tensor is pointwise in \(t\):
  \((A\otimes^\mathbf{L} B)^t \cong A^t\otimes^\mathbf{L} B^t\), exact over the field \(\Lambda\);
  constructibility is preserved (Appendix~H justifies Tonelli on bounded windows).
\end{itemize}

\begin{remark}[Scope and return to constructible]
All PF/BC comparisons below are formed in the derived category, computed objectwise in \(t\), and then passed to the persistence layer via \(\mathbf{P}_i\).
Any filtered colimit is taken in \([\mathbb{R},\mathsf{Vect}_\Lambda]\) under Appendix~A’s scope policy and \emph{returned to} \(\Pers^{\mathrm{ft}}_\Lambda\) (by verification of constructibility or by applying \(\mathbf{T}_\tau\)).
\end{remark}

% -------------------------
\subsection{Projection formula / base change at the persistence layer}
\label{N:pfbc-persistence}
% -------------------------
Let \(f:X\to Y\) and a Cartesian square
\[
\vcenter{\xymatrix{\nX'\ar[r]^{g'}\ar[d]_{f'} & X\ar[d]^f\\\nY'\ar[r]^g & Y\n}}
\]
satisfy \textup{(N0)–(N2)}.
For filtered complexes \(F\) on \(X\) and \(G\) on \(Y\), write \(\mathcal{R}(F),\mathcal{R}(G)\) for their realizations in \(D^b_c\), computed objectwise in \(t\).

\begin{theorem}[PF/BC transported to \(\mathbf{P}_i\) and \(\mathbf{T}_\tau\) \textup{[Spec]}]\label{N:thm:pf-bc}
Under \textup{(N0)--(N4)} the following canonical isomorphisms hold, \emph{natural in}
\((i,\tau,f,g,F,G)\) and \emph{up to f.q.i.} on the filtered--complex side; they are asserted \emph{after truncation by \(\mathbf{T}_\tau\)}:
\begin{align*}
\textup{(PF)}\quad
& \mathbf{T}_\tau\,\mathbf{P}_i\!\big(Rf_\ast(\mathcal{R}(F)\otimes^{\mathbf{L}} f^\ast\mathcal{R}(G))\big)
 \cong
 \mathbf{T}_\tau\,\mathbf{P}_i\!\big(Rf_\ast\mathcal{R}(F)\otimes^{\mathbf{L}}\mathcal{R}(G)\big),\\[0.25em]
\textup{(BC)}\quad
& \mathbf{T}_\tau\,\mathbf{P}_i\!\big(Lg^\ast Rf_\ast \mathcal{R}(F)\big)
 \cong
 \mathbf{T}_\tau\,\mathbf{P}_i\!\big(Rf'_\ast Lg'{}^\ast \mathcal{R}(F)\big).
\end{align*}
\end{theorem}

\begin{proof}[Proof sketch]
PF and BC are canonical isomorphisms in \(D^{b}_{c}\) under \textup{(N0)--(N2)}.
Evaluate objectwise in \(t\) (N3), identify \(\mathbf{P}_i\) with \((-i)\)-cohomology in \(t\), and then apply \(\mathbf{T}_\tau\).
Exactness of \(\mathbf{T}_\tau\) (Appendix~A, Theorem~\ref{A:thm:localization}) preserves short exact sequences induced by PF/BC on cohomology sheaves; hence isomorphisms descend to the persistence layer \emph{after truncation}.
Naturality in \((f,g,F,G)\) follows from naturality of PF/BC; naturality in \((i,\tau)\) is clear from functoriality of \(\mathbf{P}_i\) and \(\mathbf{T}_\tau\).
\end{proof}

\begin{corollary}[Compatibility with collapse at the filtered-complex layer]
Assume in addition that \(C_\tau\) realizes \(\mathbf{T}_\tau\) after applying \(\mathbf{P}_i\) (Appendix~B) up to f.q.i.
Then the PF/BC isomorphisms of Theorem~\ref{N:thm:pf-bc} hold with \(\mathcal{R}(C_\tau F)\) in place of \(\mathbf{T}_\tau\,\mathbf{P}_i(\mathcal{R}(F))\), \emph{after truncation by \(\mathbf{T}_\tau\) and up to f.q.i.\ on filtered complexes}.
\end{corollary}

\begin{remark}[What is \emph{not} claimed]
We do \emph{not} assert global Lipschitz control for PF/BC beyond the \(1\)\nobreakdash-Lipschitz behavior of \(\mathbf{T}_\tau\) (Appendix~A) and any additional [Spec] commutation controls (Appendix~L). PF/BC are exact identities at the sheaf level; any persistence--level discrepancy indicates violation of hypotheses or implementation drift and must be logged as such (see \S\ref{N:window-protocol}).
\end{remark}

% -------------------------
\subsection{Windowed protocol and reproducible audit}
\label{N:window-protocol}
% -------------------------
All PF/BC audits are \emph{windowed}. The mandatory comparison order is:
\begin{equation*}
\boxed{\ \text{for each } t\ \Longrightarrow\ \text{apply } \mathbf{P}_i\ \Longrightarrow\ \text{apply } \mathbf{T}_\tau\ \Longrightarrow\ \text{compare in }\Pers^{\mathrm{ft}}_\Lambda\ }.
\end{equation*}

Use the \emph{same} MECE, right–open windows and the \emph{same} \(\tau\) as the rest of the run (Appendix~G). In \texttt{run.yaml}, record per window:
\begin{itemize}\itemsep0.25em
  \item the PF/BC hypothesis set used (proper/smooth, finite Tor, degree normalization);
  \item the functors and objects compared (e.g.\ \(Rf_\ast(\mathcal{R}(F)\otimes f^\ast\mathcal{R}(G))\) vs.\ \(Rf_\ast\mathcal{R}(F)\otimes\mathcal{R}(G)\));
  \item the verdict (isomorphism detected) and, if any numerical/non–ideal drift is observed \emph{after truncation}, its breakdown into \(\delta^{\mathrm{disc}}\) and \(\delta^{\mathrm{meas}}\) together with tolerance(s);
  \item the window and \(\tau\) used for the comparison, matching those used by B–Gate\(^{+}\) and the \(\delta\)–ledger (Appendix~G).
\end{itemize}
When PF/BC hypotheses are satisfied, the algebraic contribution \(\delta^{\mathrm{alg}}\) is \(0\) by design; only discretization/measurement components may be nonzero.

% -------------------------
\subsection{Budget integration and window pasting}
\label{N:budget}
% -------------------------
PF/BC isomorphisms themselves contribute \(\delta^{\mathrm{alg}}=0\) when the hypotheses hold. However, in a \emph{pipeline} that also includes Mirror/Transfer steps and non–nested reflectors (Appendix~L/K), the \(\delta\)–budget on a window \(W\) is:
\[
\Sigma\delta_W(i)\ =\ \sum_{\text{Mirror--Collapse steps}} \delta_j(i,\tau_j)\ +\ \sum_{\text{A/B fails}} \Delta_{\mathrm{comm}}(M;A,B)\ +\ \sum_{\text{PF/BC}} \big(\delta^{\mathrm{disc}}+\delta^{\mathrm{meas}}\big),
\]
with \(\delta^{\mathrm{disc}},\delta^{\mathrm{meas}}\) for PF/BC typically small or null. This budget is \emph{additive}, \emph{uniform in \(F\)} for Mirror–Collapse (Appendix~L), and \emph{non–increasing} under any subsequent \(1\)\hyp Lipschitz persistence post–processing (Appendix~L, Theorem~\ref{L:thm:budget}). Window pasting then follows from Restart/Summability (Appendix~J): log \(\Sigma\delta_W(i)\) and \(\mathrm{gap}_\tau\) per window and verify \(\mathrm{gap}_\tau>\Sigma\delta_W(i)\).

% -------------------------
\subsection{Ext–tests under change of functor / coefficients}
\label{N:ext-tests}
% -------------------------
PF/BC isomorphisms transport \(\Ext^1\)–tests along canonical identifications:

\begin{proposition}[Portability of the \(\Ext^1\)–test \textup{(sheaf layer)}]
Under \textup{(N0)–(N2)} and Theorem~\ref{N:thm:pf-bc}, for any PF/BC isomorphism \(A \xrightarrow{\sim} B\) in \(D^{b}_{c}\), there is a natural isomorphism
\[
\Ext^1(A,\Lambda)\ \xrightarrow{\ \sim\ }\ \Ext^1(B,\Lambda).
\]
In particular, if \(\Ext^1(\mathcal{R}(C_\tau F),\Lambda)=0\), then \(\Ext^1\) also vanishes for any PF/BC partner of \(\mathcal{R}(C_\tau F)\).
\end{proposition}

\begin{remark}[Bridge stays one–way]
The one–way bridge \(\mathrm{PH}_1\Rightarrow \Ext^1\) (Appendix~C) is unchanged: PF/BC \emph{transport} the test across equivalent sheaf–theoretic descriptions. No converse or new implication is claimed.
\end{remark}

% -------------------------
\subsection{Implementation notes and checkpoints}
\label{N:impl}
% -------------------------
\begin{itemize}\itemsep0.35em
  \item \textbf{Finite windows / constructibility.} On bounded \(t\)–windows, bar events are finite (Appendix~H); PF/BC are computed objectwise in \(t\) and preserved by \(\mathbf{T}_\tau\).
  \item \textbf{Exactness bookkeeping.} Reductions to persistence use only: (i) PF/BC hold in \(D^b_c\); (ii) \(\mathbf{P}_i\) reads off \((-i)\)–cohomology in \(t\); (iii) \(\mathbf{T}_\tau\) is exact (Appendix~A, Theorem~\ref{A:thm:localization}) and \(1\)–Lipschitz (Appendix~A, Proposition~\ref{A:prop:lipschitz}); (iv) filtered colimits respect the scope policy (Appendix~A, Remark~\ref{A:rk:filtered-colimits}).
  \item \textbf{Proper/smooth reminder.} We use the cohomological convention, and any \(f^!\)–induced shifts in the smooth case are absorbed by (N3). PF/BC are invoked under the proper/smooth hypotheses (N2).
  \item \textbf{Window coherence.} All PF/BC audits must use the \emph{same} windows and \(\tau\) as those employed by B–Gate\(^{+}\) and the \(\delta\)–ledger (Appendix~G); otherwise budget accounting and pasting (Appendix~J) become invalid.
\end{itemize}

% -------------------------
\subsection{Formalization stubs (Lean/Coq) [Spec]}
\label{N:formal}
% -------------------------
A minimal API (cf.\ Appendix~F) includes:
\begin{itemize}\itemsep0.25em
  \item \texttt{pf\_iso}: \(Rf_\ast(A\otimes f^\ast B)\cong Rf_\ast A\otimes B\) under properness; \texttt{bc\_iso}: \(Lg^\ast Rf_\ast A\cong Rf'_\ast Lg'{}^\ast A\) for Cartesian squares (and smooth variants with \(f^!\));
  \item \texttt{to\_pers}: objectwise evaluation in \(t\) plus \(\mathbf{P}_i\) extraction and \(\mathbf{T}_\tau\) application to transport PF/BC to \(\Pers^{\mathrm{ft}}_\Lambda\);
  \item \texttt{pfbc\_pers\_nat}: naturality of these isomorphisms in \((i,\tau,f,g,F,G)\);
  \item hooks to log any residual numeric slack (post–truncation) as \(\delta^{\mathrm{disc}}\), \(\delta^{\mathrm{meas}}\) for the window budget (Appendix~L).
\end{itemize}

\medskip
\noindent\textbf{Summary.}
Under the standard PF/BC hypotheses over a field \(\Lambda\), projection formula and base change descend—via objectwise evaluation in \(t\), \(\mathbf{P}_i\), and the exact truncation \(\mathbf{T}_\tau\)—to canonical, \emph{natural} isomorphisms at the persistence layer, uniformly in \((i,\tau,f,g,F,G)\).
Comparisons must follow the windowed protocol “for each \(t\) \(\to\) persistence \(\to\) collapse \(\to\) compare,” with the \emph{same} windows and \(\tau\) as the rest of the run; any residual numerical drift (post–truncation) is accounted for in the \(\delta\)–ledger and aggregated in the pipeline budget.
These audits integrate seamlessly with Mirror/Transfer commutation (Appendix~L), multi–axis reflectors (Appendix~K), and Restart/Summability pasting (Appendix~J), while keeping the one–way bridge \(\mathrm{PH}_1\Rightarrow \Ext^1\) intact (Appendix~C).



% =========================
\section*{Appendix O. Fukaya Realization \& Stability [Spec + Permitted Ops + $\delta$-Ledger + B-Gate$^{+}$]}
% =========================
\phantomsection
\addcontentsline{toc}{section}{Appendix O. Fukaya Realization \& Stability}
\refstepcounter{section}
\label{O:fukaya}

\paragraph{Standing conventions.}
We work over a coefficient \emph{field} \(\Lambda\) (e.g.\ a ground field \(k\) or a Novikov field).
All persistence modules are constructible (locally finite on bounded windows).
Filtered (co)limits are computed objectwise in \([\mathbb{R},\mathsf{Vect}_\Lambda]\) under the scope policy of Appendix~A, Remark~\ref{A:rk:filtered-colimits}, and then (when stated) returned to the constructible range.
The interleaving metric \(d_{\mathrm{int}}\) (=\ bottleneck in the constructible \(1\)D setting) is used throughout.
Truncation \(\mathbf{T}_\tau\) is exact and \(1\)\hyp Lipschitz (Appendix~A, Proposition~\ref{A:prop:lipschitz}).
Global conventions: \(\Ext^1\) is always taken against \(\Lambda[0]\); the energy exponent satisfies \(\alpha>0\) (default \(\alpha=1\)).
References to “generic fiber dimension / infinite bars” point to Appendix~D, Remark~\ref{rem:D-generic-dim}.
Monotonicity claims follow the global policy: \emph{deletion-type only} (nonincreasing), inclusion-type merely stable/nonexpansive (Appendix~E).\footnote{Degree normalizations (e.g.\ Maslov grading shifts) are bookkeeping choices and have no effect on the windowed-counting or interleaving statements below.}
Windows are MECE and right–open; bars are half-open \([b,d)\) (Appendix~H/G).

% -------------------------
\subsection*{O.1. Realization functor and hypotheses}
% -------------------------
\emph{[Spec]} Fix a Liouville/Weinstein sector \((X,\lambda)\) with (possibly) a system of \emph{stops}.
Let \(\mathsf{Fuk}(X;\mathrm{stops})\) denote a wrapped/exact/monotone Fukaya-type category where Floer-theoretic chain models admit an \emph{action filtration}.
We adopt the convention that the \emph{action value is the filtration parameter \(t\)}, increasing in the direction of larger action (so sublevel sets \(F^{t}\) correspond to action \(\le t\)).
We package the chain-level construction into a realization functor
\[
\mathcal{F}:\ \text{(geometric input)}\longrightarrow \mathsf{FiltCh}(\Lambda),
\]
natural in continuation data and stop operations, with degree–\(i\) persistence
\(\mathbf{P}_i\big(\mathcal{F}(-)\big)\in \Pers^{\mathrm{ft}}_\Lambda\).
We assume:

\begin{itemize}\setlength{\itemsep}{0.35em}
  \item[(O0)] \textbf{Coefficients/admissibility.} \(\Lambda\) is a field; in the monotone/exact cases and with admissible almost complex structures, action and index filtrations are well defined; continuation solutions have finite energy.

  \item[(O1)] \textbf{Constructibility (action window).} On every bounded action window \([0,\sigma]\) the Floer complexes have finitely many generators and finitely many break times; hence \(\mathbf{P}_i(\mathcal{F}(-))\) is constructible. \emph{(This local finiteness holds equally with Novikov coefficients on bounded windows.)}

  \item[(O2)] \textbf{Continuation shift bound.} Any continuation map for a homotopy of data with controlled action shift \(\varepsilon\) induces a filtered chain map whose filtration increase is \(\le \varepsilon\).

  \item[(O3)] \textbf{Stop operations are deletion-type.} Adding a stop or shrinking a sector removes generators and/or increases differentials in a way that corresponds to a \emph{deletion-type} operation at the persistence layer: in any fixed action window no new bars are created.

  \item[(O4)] \textbf{Up to f.q.i.} Chain models are considered up to filtered quasi-isomorphism; all claims are invariant under f.q.i.\ by exactness of \(\mathbf{T}_\tau\) and the scope policy.
\end{itemize}

\begin{remark}[Endpoint conventions \& windows]
Half-open \([b,d)\) bars and right–open windows ensure MECE coverage; events at the right boundary are attributed to the next window (Appendix~H/G).
\end{remark}

% -------------------------
\subsection*{O.2. Stability: continuation and stops}
% -------------------------
\begin{theorem}[Continuation \(1\)-Lipschitz]\label{O:thm:cont}
Under \textup{(O2)}, for any two realizations related by a continuation with action shift \(\varepsilon\),
\[
d_{\mathrm{int}}\!\Big(\mathbf{P}_i(\mathcal{F}_0),\ \mathbf{P}_i(\mathcal{F}_1)\Big)\ \le\ \varepsilon,\qquad\nd_{\mathrm{int}}\!\Big(\mathbf{T}_\tau\mathbf{P}_i(\mathcal{F}_0),\ \mathbf{T}_\tau\mathbf{P}_i(\mathcal{F}_1)\Big)\ \le\ \varepsilon
\]
for all \(i,\tau\ge 0\).
\end{theorem}

\begin{proof}[Proof sketch]
A filtered chain map of filtration increase \(\le\varepsilon\) implements the required shift–commutation with the time-shift functors, hence yields an \(\varepsilon\)-interleaving; the first inequality follows.
Applying \(\mathbf{T}_\tau\) preserves the bound by its \(1\)\hyp Lipschitz property (Appendix~A, Proposition~\ref{A:prop:lipschitz}).
\end{proof}

\begin{proposition}[Deletion-type monotonicity for stops]\label{O:prop:stops}
Under \textup{(O3)}, adding a stop or shrinking a sector induces a map at the persistence layer that is deletion-type:
for every \(i\) and \(\tau\ge 0\),
\[
\mathbf{T}_\tau\,\mathbf{P}_i\big(\mathcal{F}_{\text{with stop}}\big)\ \preceq\ \mathbf{T}_\tau\,\mathbf{P}_i\big(\mathcal{F}_{\text{without stop}}\big),
\]
and all deletion-type monotone indicators (Appendix~E) are nonincreasing in this operation.
\end{proposition}

\begin{remark}[Inclusion-type caution]
Operations that \emph{enlarge} the admissible region or \emph{remove} stops need not be monotone; we only assert nonexpansivity via continuation when an explicit shift control is available (Appendix~E, §E.2).
\end{remark}

% -------------------------
\subsection*{O.3. Towers, comparison map, and diagnostics}
% -------------------------
Let \(F=(F_n)_{n\in I}\) be a directed system of geometric inputs (e.g.\ refining Hamiltonians/perturbations or nested stop systems) with colimit \(F_\infty\).
Apply \(\mathcal{F}\) and \(\mathbf{P}_i\) to obtain a tower in \(\Pers^{\mathrm{ft}}_\Lambda\).
For \(\tau\ge 0\) consider the comparison map (Appendix~J)
\[
\phi_{i,\tau}(F):\quad\n\varinjlim_n\ \mathbf{T}_\tau\!\big(\mathbf{P}_i(\mathcal{F}(F_n))\big)\ \longrightarrow\ \mathbf{T}_\tau\!\big(\mathbf{P}_i(\mathcal{F}(F_\infty))\big).
\]

\begin{theorem}[When the comparison is an isomorphism]\label{O:thm:phi-iso}
If the tower admits continuation controls \(\varepsilon_n\to 0\) with
\(d_{\mathrm{int}}\!\big(\mathbf{P}_i(\mathcal{F}(F_n)),\mathbf{P}_i(\mathcal{F}(F_\infty))\big)\le \varepsilon_n\)
and satisfies any of the sufficient hypotheses of Appendix~D, §D.3 (e.g.\ commutation (S1), no \(\tau\)-accumulation (S2), or Cauchy+compatibility (S3)), then \(\phi_{i,\tau}(F)\) is an isomorphism for all \(\tau\ge 0\).
Consequently,
\[
\mu_{i,\tau}(F)=\nu_{i,\tau}(F)=0,
\]
where \((\mu,\nu)\) are the generic fiber dimensions of the kernel/cokernel (Appendix~D, Remark~\ref{rem:D-generic-dim}).
\end{theorem}

\begin{proof}[Proof sketch]
Combine Theorem~\ref{O:thm:cont} with the tower criteria of Appendix~D, §D.3, and then apply Appendix~J to conclude \(\phi_{i,\tau}\) is an isomorphism and \((\mu,\nu)=(0,0)\).
\end{proof}

\begin{corollary}[Grid \(\Rightarrow\) continuum survival]
In discretization towers (mesh \(h\to 0\)) with certified continuation bounds \(\varepsilon(h)\), bars detected in a fixed window \([0,\tau_0]\) with clipped length \(>\!2\varepsilon(h)\) persist in the limit (Appendix~I, Theorem~I:\ref{I:thm:g2c}).
\end{corollary}

% -------------------------
\subsection*{O.4. Permitted-operations table (windowed, post-collapse) and $\delta$-ledger}
% -------------------------
All comparisons follow the protocol “for each \(t\) \(\to\) persistence \(\to\) collapse \(\mathbf{T}_\tau\) \(\to\) compare,” on MECE, right–open windows and a fixed \(\tau\) (Appendix~G/N).
The following table summarizes permitted operations, their type, quantitative contracts (after collapse), and how to record them in the \(\delta\)–ledger.

\begin{center}
\footnotesize
\setlength{\tabcolsep}{4pt}       % 列間余白を少し詰める(任意)
\renewcommand{\arraystretch}{1.1} % 行間(任意)
\begin{tabularx}{\linewidth}{@{} L{.26\linewidth} L{.14\linewidth} Xr Xr @{}}
\toprule
Operation & Type & Quantitative contract\\(after \(\mathbf{T}_\tau\)) & Ledger entry \\
\midrule
Add stop / shrink sector & Deletion & Nonincreasing deletion indicators; no new bars on window & \(\delta^{\mathrm{alg}}=0\);\; record \(\delta^{\mathrm{disc}},\delta^{\mathrm{meas}}\) if any \\
Continuation (tame homotopy) & Shift & \(d_{\mathrm{int}}\le \varepsilon\) (Theorem~\ref{O:thm:cont}) & \(\delta^{\mathrm{alg}}=\varepsilon\) \\
Hamiltonian change (bounded action drift) & Shift & \(d_{\mathrm{int}}\le \varepsilon\) & \(\delta^{\mathrm{alg}}=\varepsilon\) \\
Almost complex structure change (tame) & Shift & \(d_{\mathrm{int}}\le \varepsilon\) & \(\delta^{\mathrm{alg}}=\varepsilon\) \\
Regrading / Maslov shift & Bookkeeping & Isometry (reindexed degrees) & \(\delta^{\mathrm{alg}}=0\) \\
Monotone time reparam.\ (order-preserving) & Reindex & Isometry after reindex normalization & \(\delta^{\mathrm{alg}}=0\) \\
Mirror/Transfer post-processing & External & See Appendix~L, \(\delta(i,\tau)\) control if available & \(\delta^{\mathrm{alg}}=\delta(i,\tau)\) \\
Non-nested reflectors (if used) & External & A/B test, soft-commuting or fallback (Appendix~K/L) & \(\delta^{\mathrm{alg}}=\Delta_{\mathrm{comm}}\) if fallback \\
\bottomrule
\end{tabularx}
\end{center}


Per window \(W=[u,u')\) and degree \(i\), aggregate the budget
\[
\Sigma\delta_W(i)\ :=\ \sum_{\text{continuations}}\varepsilon\ +\ \sum_{\text{Mirror--Collapse}}\delta(i,\tau)\ +\ \sum_{\text{A/B fails}}\Delta_{\mathrm{comm}}\ +\ \sum_{\text{audits}}(\delta^{\mathrm{disc}}+\delta^{\mathrm{meas}}),
\]
where “audits” includes PF/BC checks if present (Appendix~N) and any numerical integration tolerances (Appendix~G).

% -------------------------
\subsection*{O.5. B-Gate$^{+}$, restart, and summability (window pasting)}
% -------------------------
We adopt B-Gate\(^{+}\) with a per-window safety margin \(\mathrm{gap}_\tau(i)>0\).
On window \(W\) and degree \(i\), the gate \emph{passes} if
\[
\mathrm{gap}_\tau(i)\ >\ \Sigma\delta_W(i).
\]
Across consecutive windows \((W_k)_k\), assume:
(i) transitions are finite compositions of deletion-type steps and \(\varepsilon\)-continuations (measured post-collapse), and
(ii) Summability holds \(\sum_k \Sigma\delta_{W_k}(i)<\infty\) (Appendix~J).
Then the Restart inequality of Appendix~J yields for some \(\kappa\in(0,1]\),
\[
\mathrm{gap}_{\tau_{k+1}}(i)\ \ge\ \kappa\Big(\mathrm{gap}_{\tau_k}(i)-\Sigma\delta_{W_k}(i)\Big),
\]
and positivity of the margin persists along the pipeline.
Hence per-window certificates paste to a global certificate on \(\bigcup_k W_k\) (Appendix~J, Theorem~J:\ref{J:thm:pasting}).

% -------------------------
\subsection*{O.6. Windowed usage and run.yaml alignment}
% -------------------------
Record in \texttt{run.yaml} per window and degree:
\begin{itemize}\itemsep0.25em
  \item the operation sequence (stops/sector changes, continuations) with quantitative parameters (\(\varepsilon\), thresholds \(\tau\), sweep settings);
  \item the \(\delta\)–ledger entries and their sum \(\Sigma\delta_W(i)\);
  \item the B-Gate\(^{+}\) safety margin \(\mathrm{gap}_\tau(i)\) and pass/fail verdict;
  \item any external steps (Mirror/Transfer, reflectors) with A/B policy data (\(\eta\), \(\Delta_{\mathrm{comm}}\)) (Appendix~K/L);
  \item constructibility checks: upper bounds on generators and event counts (Appendix~H).
\end{itemize}
All diagnostics (\(\mu,\nu\), comparison maps \(\phi_{i,\tau}\)) are computed \emph{after} truncation \(\mathbf{T}_\tau\) and logged with the same window and \(\tau\).

% -------------------------
\subsection*{O.7. Failure modes and audit checklist}
% -------------------------
\noindent\emph{Failure modes (outside our scope).}
\begin{itemize}\itemsep0.25em
  \item \textbf{Loss of filtration control.} Non-exact data or bubbling may invalidate (O2); interleaving bounds then fail.
  \item \textbf{Near-threshold accumulation.} Type~IV behavior (Appendix~D) may produce bar-length accumulation at \(\tau\); comparison maps need not stabilize without (S2)/(S3).
  \item \textbf{Inclusion-type operations.} Removing stops or enlarging sectors can increase features; only stability (not monotonicity) is claimed, and only when continuation control is present.
\end{itemize}

\noindent\emph{Audit checklist (runtime verifications).}
\begin{enumerate}\itemsep0.25em
  \item Record continuation shift bounds \(\varepsilon\) and certify \(d_{\mathrm{int}}\)-nonexpansivity (Appendix~A).
  \item Verify constructibility on each window \([0,\sigma]\) (finite generators/events; Appendix~H) and log per-window counts (Appendix~G).
  \item For stop additions/sector shrinkage, mark operation as deletion-type and apply Appendix~E indicators.
  \item For towers, log \(\varepsilon_n\), check (S1)/(S2)/(S3) where applicable, and compute \((\mu,\nu)\); \(\phi_{i,\tau}\) iso \(\Rightarrow (\mu,\nu)=(0,0)\) (Appendix~J).
  \item If external functors are used, run Mirror/Transfer commutation audits and A/B tests (Appendix~L/K) and bookkeep residuals in \(\Sigma\delta_W(i)\).
\end{enumerate}

% -------------------------
\subsection*{O.8. Formalization stubs (Lean/Coq) [Spec]}
% -------------------------
A minimal API (cf.\ Appendix~F) includes:
\begin{itemize}\itemsep0.25em
  \item \texttt{fukaya\_realize}: action-filtered chain model \(\mathcal{F}\) up to f.q.i., with constructibility on bounded windows;
  \item \texttt{cont\_eps}: continuation maps with filtration increase \(\le\varepsilon\) yielding \(d_{\mathrm{int}}\le\varepsilon\) (Theorem~\ref{O:thm:cont});
  \item \texttt{stop\_delete}: deletion-type morphisms for stops/sector-shrink (Proposition~\ref{O:prop:stops});
  \item \texttt{tower\_phi\_iso}: sufficient criteria to conclude \(\phi_{i,\tau}\) isomorphism and \((\mu,\nu)=(0,0)\) (Appendix~D/J);
  \item hooks for \(\delta\)–ledger entries and B-Gate\(^{+}\) checks; restart/summability contracts (Appendix~J).
\end{itemize}

\medskip
\noindent\textbf{Summary.}
Floer-theoretic realizations with action filtration yield constructible persistence (\textup{O1}); continuation with shift \(\varepsilon\) is \(1\)\hyp Lipschitz at the persistence level (\textup{O2} \(\Rightarrow\) Theorem~\ref{O:thm:cont}); adding stops or shrinking sectors is deletion-type and hence nonincreasing for all deletion indicators (Proposition~\ref{O:prop:stops} and Appendix~E).
A windowed, post-collapse permitted-ops table prescribes how to assign \(\delta\)–ledger entries; B-Gate\(^{+}\) requires \(\mathrm{gap}_\tau>\Sigma\delta\) per window and pastes via Restart/Summability (Appendix~J).
Under standard tower hypotheses (Appendix~D), the comparison map \(\phi_{i,\tau}\) is an isomorphism and \((\mu,\nu)=(0,0)\) (Theorem~\ref{O:thm:phi-iso}); grid-to-continuum survival follows from Appendix~I.
All items respect the MECE/right–open window policy, are evaluated “\(t\)\(\to\)\(\mathbf{P}_i\)\(\to\)\(\mathbf{T}_\tau\)\(\to\)compare,” and integrate with the pipeline budget and A/B policy (Appendix~L/K) in a reproducible \texttt{run.yaml} workflow (Appendix~G).



% -------------------------
\section*{Concluding Remarks and Acknowledgments}
% -------------------------
\addcontentsline{toc}{section}{Concluding Remarks and Acknowledgments}

\paragraph{Coefficients.}
Unless otherwise stated, we work over a \emph{field}. In the provable core (bridge in Appendix~C) the target is \(D^{\mathrm{b}}(k\text{-mod})\); in [Spec] appendices that use sheaf/Fukaya realizations we write the coefficient field as \(\Lambda\). This is only a notational distinction; all persistence-layer claims remain within the constructible 1D setting over a field.

\paragraph{What we proved (core).}
Within the constructible (p.f.d.) 1D persistence setting over a field, we established:
\begin{itemize}
  \item An exact, idempotent, and \(1\)\hyp Lipschitz Serre reflector \(\mathbf{T}_\tau\) that deletes all bars of length \(\le\tau\), and its filtered lift \(C_\tau\) (up to f.q.i.), with \(\mathbf{P}_i(C_\tau F)\cong \mathbf{T}_\tau(\mathbf{P}_iF)\).
  \item A one\hyp way bridge \(\mathrm{PH}_1(F)=0 \Rightarrow \Ext^1(\mathcal{R}(F),k)=0\) under amplitude \([-1,0]\) and \(t\)\hyp exactness (Appendix~C).
  \item Tower diagnostics \((\mu,\nu)\) from the comparison maps \(\phi_{i,\tau}\), with subadditivity under composition, additivity under finite sums, and cofinal invariance (Appendix~J); when \(\phi_{i,\tau}\) is an isomorphism, \((\mu,\nu)=(0,0)\). A windowed calculus augments these with \(\tau\)\hyp sweeps and \emph{stability bands} (Appendix~J) identifying robust scales where \((\mu,\nu)=(0,0)\).
  \item Stability and calculus for windowed indicators: Betti integral \(=\) clipped barcode mass (Appendix~H); \(\varepsilon\)\hyp survival and grid\hyp to\hyp continuum transfer (Appendix~I); \emph{deletion\hyp type monotonicity for spectral indicators} and general \emph{non\hyp expansiveness} bounds (Appendix~E).
  \item A windowed pasting mechanism: B–Gate\(^{+}\) with safety margin \(\mathrm{gap}_\tau>\Sigma\delta\) per window, and \emph{Restart + Summability} to paste windowed certificates globally (Appendix~J).
\end{itemize}

\paragraph{What remains \textup{[Spec]}.}
We provided auditable specifications—kept \emph{non\hyp expansive after truncation}—for:
\begin{itemize}
  \item Mirror/tropical flows with \(\delta\)\hyp controlled commutation and a \emph{pipeline error budget} that is additive, uniform in the input, and non\hyp increasing under \(1\)\hyp Lipschitz post\hyp processing (Appendix~L).
  \item A \emph{soft\hyp commuting} A/B policy for multi–axis exact reflectors, with tolerance, fallback order, and explicit logging of the commutation defect \(\Delta_{\mathrm{comm}}\) as \(\delta^{\mathrm{alg}}\) (Appendix~K).
  \item Lax monoidal bounds for tensor/collapse with windowed energy inequalities and a pointwise–mono regime (M3\(^{+}\)) yielding dominance after collapse; optional metric gaps can be integrated into the budget (Appendix~M).
  \item Projection formula/base change transported to persistence via the windowed protocol “\(t\)\(\to\)\(\mathbf{P}_i\)\(\to\)\(\mathbf{T}_\tau\)\(\to\)compare” (Appendix~N).
  \item Fukaya realizations with action filtration, continuation stability, and stop\hyp addition as deletion\hyp type; a permitted–ops table mapping each step to \(\delta\)\hyp ledger entries and B–Gate\(^{+}\) usage (Appendix~O).
\end{itemize}
These items are explicitly marked \textbf{[Spec]} and used only within the stated scope and controls.

\paragraph{Scope and guard\hyp rails.}
All claims respect:
\begin{itemize}
  \item Constructible \(1\)D persistence over a field; filtered (co)limits computed in \([\mathbb{R},\mathsf{Vect}]\) under the scope policy and returned to \(\Pers^{\mathrm{ft}}\) by verification or \(\mathbf{T}_\tau\).
  \item \(\mathbf{T}_\tau\) exact/idempotent/\(1\)\hyp Lipschitz; shift–commutation; deletion\hyp vs.\ inclusion\hyp type policy (\emph{deletion\hyp type monotonicity} vs.\ \emph{non\hyp expansiveness} only).
  \item B–side \emph{single–layer} judgement \emph{after collapse}; all tower diagnostics and budgets are evaluated on \(\mathbf{T}_\tau\mathbf{P}_i\).
  \item The bridge remains strictly one\hyp way; no claim of \(\mathrm{PH}_1\Leftrightarrow \Ext^1\).
  \item MECE, right–open windows; window–coherent comparisons and budgets; \(\tau\)\hyp sweeps with stability bands.
\end{itemize}

\paragraph{Reproducibility and formalization.}
Appendix~G specifies versioned logs and schemas (\texttt{bars}/\texttt{spec}/\texttt{ext}/\texttt{phi}) with canonical serialization and cross\hyp linked hashes; the \(\delta\)\hyp ledger records algorithmic/discretization/measurement components per window, including Mirror–Collapse \(\delta(i,\tau)\) and A/B–test \(\Delta_{\mathrm{comm}}\).
Appendix~F sketches a Lean/Coq development for the core: Serre localization and \(\mathbf{T}_\tau\); \(1\)\hyp Lipschitz; comparison maps and \((\mu,\nu)\); the bridge’s edge identification; and API stubs for budget accounting, A/B policy, PF/BC transport, and lax monoidal contracts.
A minimal test suite (Appendix~G/12) validates stability, monotone updates, filtered–colimit behavior, Mirror/tropical pipelines, three–layer correspondences, PDE workflows, saturation gates, A/B soft\hyp commuting, and Restart/Summability pasting.

\paragraph{Limitations.}
We do not assert metric Lipschitz control beyond \(\mathbf{T}_\tau\) (and any stated \(\delta\)\hyp controls).
Spectral indicators are not f.q.i.\ invariants; they are evaluated under fixed policies with \emph{deletion\hyp type} monotonicity and general \emph{non\hyp expansiveness} bounds (Appendix~E).
We avoid extending the bridge, strong adjunctions, or strict monoidality beyond the proven/declared domains.
PF/BC and lax monoidal statements remain windowed [Spec] contracts; all budgets are computed after collapse on the B–side single layer.

\paragraph{Future directions.}
\begin{itemize}
  \item Strengthening quantitative links between persistence energies and spectral tails under robust hypotheses and verified window policies.
  \item Broader, verifiable criteria for \((\mu,\nu)=(0,0)\) (beyond S1–S3) and refined tower diagnostics across scales; automated detection of stability bands.
  \item Domain templates (arithmetic/Langlands/PDE/Fukaya) with standardized \(\delta\)\hyp controls, PF/BC audits, and A/B policies; richer run\hyp level manifests.
  \item Expanded formalization: shift/interleaving libraries, PF/BC transports, survival lemmas, pipeline budget calculus, and soft\hyp commuting in proof assistants.
\end{itemize}

\paragraph{Acknowledgments.}
This manuscript was prepared solely by the author together with an AI assistant (ChatGPT) via an iterative chat\hyp based workflow; no other contributors were involved. Any remaining errors are ours.

\medskip
\noindent\textbf{Final note.}
The separation between the provable core and \textbf{[Spec]} extensions is the organizing principle of this work:
it enables safe reuse, cross\hyp domain exploration, and reproducible evaluation while preserving mathematically verified guarantees.
Windowed B–side judgement (collapse first), explicit \(\delta\)\hyp budgets, A/B soft\hyp commuting, and Restart/Summability pasting provide an end\hyp to\hyp end methodology that is both conservative and extensible within the v15.0 guard–rails.



\end{document}