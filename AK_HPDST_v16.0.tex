% ====================================
% AK High-Dimensional Projection Structural Theory v16.0 (pdfLaTeX)
% ====================================
\documentclass[11pt]{article}

% --- Encoding & Language ---
\usepackage[utf8]{inputenc}
\usepackage[T1]{fontenc}
\usepackage[english]{babel}
\usepackage{geometry}
\geometry{margin=1in}

% --- Core Math & Utilities (AMS first) ---
\usepackage{amsmath,amssymb,amsthm,amsfonts,mathtools}
\usepackage{mathrsfs}
\usepackage{bm}
\usepackage{stmaryrd}
\usepackage{changepage}
\usepackage{amscd}
\usepackage{textcomp}
\usepackage{etoolbox} % <-- for \forcsvlist
\usepackage{enumitem}
\usepackage{array}
\usepackage{hyphenat} % one\hyp way などのため
% --- Fonts (pdfLaTeX-friendly) ---
\usepackage{newtxtext,newtxmath}  % Times-like text & math
\usepackage{inconsolata}          % Monospace for listings

% --- Diagrams ---
\usepackage{tikz, tikz-cd}
\usetikzlibrary{
  matrix, arrows.meta, cd, calc, positioning, quotes,
  decorations.pathmorphing, decorations.markings,
  shapes.misc, shapes.geometric, arrows, fit, backgrounds, fadings
}
\usepackage{pgfplots}
\pgfplotsset{compat=1.18}
\usepackage[all,cmtip]{xy}        % Xy-pic(tikz-cd と共存可)
\numberwithin{equation}{section}
\theoremstyle{plain}
\theoremstyle{definition}
\theoremstyle{remark}


% --- Tables ---
\usepackage{array,tabularx,booktabs}
% fixed-width paragraph columns
\newcolumntype{L}[1]{>{\raggedright\arraybackslash}p{#1}}
\newcolumntype{C}[1]{>{\centering\arraybackslash}p{#1}}
\newcolumntype{R}[1]{>{\raggedleft\arraybackslash}p{#1}}
% variable-width paragraph column (ragged-right)
\newcolumntype{Y}{>{\raggedright\arraybackslash}X}

% --- Monospace token helper (safe in math) ---
\newcommand{\fname}[1]{\text{\texttt{#1}}}

% --- Listings / Code ---
\usepackage{xcolor}
\usepackage{listings}
\usepackage{float}
\lstset{
  basicstyle=\ttfamily\small,
  keywordstyle=\color{blue},
  commentstyle=\color{gray},
  stringstyle=\color{orange},
  frame=single,
  breaklines=true,
  showstringspaces=false,
  captionpos=b,
  xleftmargin=1em,
  columns=flexible,
  literate=
    {∀}{{$\forall$}}1 {∃}{{$\exists$}}1
    {→}{{$\to$}}1 {←}{{$\leftarrow$}}1 {⟶}{{$\to$}}1 {⇒}{{$\Rightarrow$}}1 {⇔}{{$\Leftrightarrow$}}1 {↦}{{$\mapsto$}}1
    {α}{{$\alpha$}}1 {β}{{$\beta$}}1 {γ}{{$\gamma$}}1 {λ}{{$\lambda$}}1 {Π}{{$\Pi$}}1
    {ℕ}{{$\mathbb{N}$}}1 {ℤ}{{$\mathbb{Z}$}}1 {ℚ}{{$\mathbb{Q}$}}1 {ℝ}{{$\mathbb{R}$}}1
    {≤}{{$\le$}}1 {≥}{{$\ge$}}1 {≠}{{$\ne$}}1
    {∧}{{$\land$}}1 {∨}{{$\lor$}}1
    {ε}{{$\varepsilon$}}1 {μ}{{$\mu$}}1
    {⟨}{{$\langle$}}1 {⟩}{{$\rangle$}}1
    {…}{{$\ldots$}}1 {–}{{--}}1 {—}{{---}}1 {∞}{{$\infty$}}1
    {⊕}{{$\oplus$}}1 {⊗}{{$\otimes$}}1 {⊣}{{$\dashv$}}1 {⊢}{{$\vdash$}}1
    {≙}{{$\triangleq$}}1
    {·}{{$\cdot$}}1 {∙}{{$\cdot$}}1 {⋅}{{$\cdot$}}1 {•}{{$\bullet$}}1
    {・}{{$\cdot$}}1 {‧}{{$\cdot$}}1
    {×}{{$\times$}}1 {⨯}{{$\times$}}1
}

% --- Math Operators ---
\DeclareMathOperator{\Ext}{Ext}
\DeclareMathOperator{\Hom}{Hom}
\DeclareMathOperator{\Spec}{Spec}
\DeclareMathOperator{\colim}{colim}
\DeclareMathOperator{\PH}{PH}
\DeclareMathOperator{\Tor}{Tor}
\DeclareMathOperator{\rank}{rank}
\DeclareMathOperator{\im}{im}
\DeclareMathOperator{\id}{id}
\DeclareMathOperator{\Ker}{Ker}
\DeclareMathOperator{\Coker}{Coker}
\DeclareMathOperator{\Collapse}{Collapse}
\DeclareMathOperator{\Mot}{Mot}
\DeclareMathOperator{\GL}{GL}
\DeclareMathOperator{\RHom}{RHom}
\DeclareMathOperator{\HT}{HT}
\DeclareMathOperator{\gdim}{gdim}

\DeclareRobustCommand{\hyp}{\nobreakdash-}

% --- General Macros ---
\newcommand{\QQ}{\mathbb{Q}}
\newcommand{\RR}{\mathbb{R}}
\newcommand{\CC}{\mathbb{C}}
\newcommand{\ZZ}{\mathbb{Z}}
\newcommand{\TT}{\mathbb{T}}
\newcommand{\CollapseZone}{\mathfrak{C}}
\newcommand{\cF}{\mathcal{F}}
\newcommand{\cG}{\mathcal{G}}
\newcommand{\cE}{\mathcal{E}}
\newcommand{\cO}{\mathcal{O}}
\newcommand{\cD}{\mathcal{D}}
\newcommand{\cH}{\mathcal{H}}
\newcommand{\into}{\hookrightarrow}
\newcommand{\onto}{\twoheadrightarrow}
\newcommand{\eps}{\varepsilon}
\newcommand{\Sha}{\mathcal{X}} % (Tate--Shafarevich placeholder)
\newcommand{\CollapseCompatible}{\mathsf{CollapseCompatible}}
\newcommand{\CollapseTypeOf}[1]{\operatorname{CollapseType}(#1)}
\newcommand{\CollapseImage}{\operatorname{CollapseImage}}
\newcommand{\CollapseEnergy}{\mathcal{E}_{\mathrm{Coll}}}
\newcommand{\ptorsionfree}[1]{\ensuremath{#1}\nobreakdash-torsion-free}
\newcommand{\Pers}{\mathsf{Pers}}
\newcommand{\Vect}{\mathsf{Vect}}
\newcommand{\Ho}{\mathrm{Ho}}
\newcommand{\dq}{\textquotedbl}
\newcommand{\rH}{\mathrm{H}}
\newcommand{\timevar}{t} % use \HT(\timevar;F)
\newcommand{\T}{\mathsf{T}}
\newcommand{\C}{\mathsf{C}}
\newcommand{\Coll}{\mathsf{C}}
\newcommand{\Rfun}{\mathcal{R}}

\theoremstyle{plain}
\theoremstyle{definition}

\newcommand{\kk}{k}
\newcommand{\xto}{\xrightarrow}
\newcommand{\xtoH}{\xRightarrow{\ \simeq\ }}
\newcommand{\xtoM}{\xRightarrow{\ \mono\ }}
\newcommand{\mono}{\mathrm{mono}}
\newcommand{\epi}{\mathrm{epi}}
\newcommand{\DD}{\mathcal{D}}
\newcommand{\cC}{\mathcal{C}}
\newcommand{\supp}{\mathrm{supp}}
% Windowed energy / Betti
\newcommand{\PE}{\mathrm{PE}}
\newcommand{\betti}{\beta}
% Collapse and truncation (chain-level vs persistence-level)
\newcommand{\CT}{C_{\tau}}
% Interleaving metric
\newcommand{\dint}{d_{\mathrm{int}}}

% --- Theorem-like Environments ---
\numberwithin{equation}{section}
\newtheorem{theorem}{Theorem}[section]
\newtheorem{proposition}[theorem]{Proposition}
\newtheorem{lemma}[theorem]{Lemma}
\newtheorem{corollary}[theorem]{Corollary}
\newtheorem{axiom}{Axiom}[section]
\newtheorem{conjecture}{Conjecture}[section]
\theoremstyle{definition}
\newtheorem{definition}[theorem]{Definition}
\newtheorem{example}[theorem]{Example}
\newtheorem{remark}[theorem]{Remark}
\newtheorem{specification}[theorem]{Specification}
\newtheorem{declaration}[theorem]{Declaration}

% --- Unified notation ---
\DeclareRobustCommand{\FiltCh}[1]{\mathsf{FiltCh}(#1)}
\DeclareRobustCommand{\Perskft}{\Pers^{\mathrm{cons}}_{k}}
\DeclareRobustCommand{\Pk}{\mathbf{P}}
\DeclareRobustCommand{\Ttau}{\texorpdfstring{\ensuremath{\mathbf{T}_{\tau}}}{T\_\tau}}
\DeclareRobustCommand{\Ctau}{\texorpdfstring{\ensuremath{C_{\tau}}}{C\_\tau}}
\DeclareRobustCommand{\Dbplain}{D^{\mathrm{b}}}
\DeclareRobustCommand{\Dbkmod}{D^{\mathrm{b}}(k\text{-mod})}
\DeclareRobustCommand{\Db}{D^{\mathrm{b}}(k\text{-mod})}
\DeclareRobustCommand{\muc}{\mu_{\mathrm{Collapse}}}
\DeclareRobustCommand{\nuc}{\nu_{\mathrm{Collapse}}}
\DeclareRobustCommand{\fqi}{\text{f.q.i.}}
\DeclareRobustCommand{\LC}{\texorpdfstring{\ensuremath{\mathrm{(LC)}}}{(LC)}}
\DeclareRobustCommand{\Qtest}{\{\,k[0]\,\}}
\DeclareRobustCommand{\pos}[1]{\left(#1\right)_{+}}
\DeclareRobustCommand{\Trop}{\texorpdfstring{\ensuremath{\operatorname{Trop}}}{Trop}}
\DeclareRobustCommand{\Mirror}{\texorpdfstring{\ensuremath{\operatorname{Mirror}}}{Mirror}}
\newcommand{\ProofBadge}{\textsf{\footnotesize[Proof]}}
\newcommand{\SpecBadge}{\textsf{\footnotesize[Spec]}}

% Theorem environments (AIM/IMRN-friendly)
\numberwithin{equation}{section}
\theoremstyle{plain}
\theoremstyle{definition}
\theoremstyle{remark}

% Macros (notations consistent across the paper)
\newcommand{\bbR}{\mathbb{R}}
\newcommand{\bbN}{\mathbb{N}}
\newcommand{\Vectk}{\mathsf{Vect}_k}
\newcommand{\Dbk}{D^{\mathrm{b}}(k\text{-mod})}
\newcommand{\Perscons}{\Pers^{\mathrm{cons}}_k}
\newcommand{\Persft}{\Pers^{\mathrm{ft}}_k}
\newcommand{\Crop}{\mathbf{W}}
\newcommand{\Res}{\mathrm{Res}}
\newcommand{\intdist}{d_{\mathrm{int}}}
\newcommand{\Ecat}{\mathsf{E}_\tau}
\newcommand{\Orth}{(\mathsf{E}_\tau)^{\perp}}
\newcommand{\Len}{\Lambda_{\mathrm{len}}}
% Shortcuts
\newcommand{\qand}{\quad\text{and}\quad}
\newcommand{\qtext}[1]{\quad\text{#1}\quad}
\newcommand{\ang}[1]{\langle #1\rangle}
\newcommand{\ol}[1]{\overline{#1}}

\newcommand{\Trans}{\mathsf{Trans}}
\newcommand{\Funct}{\mathsf{Funct}}

\newcommand{\loc}{\mathrm{loc}}
\newcommand{\MECE}{\mathrm{MECE}}
\newcommand{\clip}{\mathrm{clip}}
\newcommand{\len}{\mathrm{len}}

\newcommand{\NN}{\mathbb{N}}
\newcommand{\EE}{\mathbb{E}}
\newcommand{\op}{\mathrm{op}}
\newcommand{\Pfun}{\mathbf{P}}
\newcommand{\Ufun}{\mathcal{U}}

\providecommand{\Cfun}[1]{\mathsf{C}_{#1}}
\providecommand{\Tfun}[1]{\mathbf{T}_{#1}}

\providecommand{\Ctau}{\Cfun{\tau}}
\providecommand{\Csigma}{\Cfun{\sigma}}
\providecommand{\Ttau}{\Tfun{\tau}}
\providecommand{\Tsigma}{\Tfun{\sigma}}
\providecommand{\intdist}{d_{\mathrm{int}}}  % 未定義なら

\DeclareMathOperator{\coker}{coker} % \ker は標準定義済み、\coker は自前で用意

% Defect オブジェクト名(下付き/上付きを右肩につけたいので \mathsf 体で変数扱い)
\providecommand{\Defect}{\mathsf{Defect}}

% もし使っていれば安全側で:
\providecommand{\muc}{\mu_{\mathrm{Collapse}}}
\providecommand{\nuc}{\nu_{\mathrm{Collapse}}}
% --- Over/underfull mitigation ---
\usepackage{microtype}
\emergencystretch=2em

% --- Hyperlinks (load ONCE, near the end) ---
\usepackage[colorlinks=true, linkcolor=blue, citecolor=blue, urlcolor=blue]{hyperref}
\usepackage{xurl}                 % long URL/code line breaks
\urlstyle{tt}
\def\UrlBreaks{\do\/\do\_\do\.\do\-}

\usepackage{tabularx,booktabs,array}
\newcolumntype{L}[1]{>{\raggedright\arraybackslash}p{#1}}
% (X は tabularx で定義済み。右端も左揃えにしたいなら)
\newcolumntype{Y}{>{\raggedright\arraybackslash}X}

% --- Clever references (after hyperref) ---
\usepackage[capitalise,nameinlink,noabbrev]{cleveref}

% --- Float / page control ---
\usepackage{placeins}

% ======================================================
% === Paste-artifact guard: robust \n handling (HEAVY) ===
% ======================================================
% Baseline: treat \n as a space even if followed by a control sequence.
\providecommand{\n}{\unskip\space}

% Word-level fixes: define \n<Word> for many common starters (both cases).
\makeatletter
\newcommand{\patchn}[1]{\expandafter\def\csname n#1\endcsname{\unskip\space #1}}

% Frequent English/connective words and theorem-like labels
\forcsvlist{\patchn}{Hence,Therefore,Thus,Then,Proof,Remark,Theorem,Proposition,Lemma,Corollary,
Definition,Example,Declaration,Specification,Spec,Appendix,Section,Subsection,We,In,Let,Under,
Given,Assuming,Assumption,Fix,If,When,Where,While,For,From,By,On,Off,Also,Moreover,Conversely,
Equivalently,Finally,Next,First,Second,Third,Note,Notation,Convention,Claim,Case,Step,Algorithm,
Figure,Table,Thm,Prop,Lem,Cor,Def,Ex,Rem,Eq,Eqn,Ch,Chap,PF,BC,AK,NS,HT,PH}

\forcsvlist{\patchn}{hence,therefore,thus,then,proof,remark,theorem,proposition,lemma,corollary,
definition,example,declaration,specification,spec,appendix,section,subsection,we,in,let,under,
given,assuming,assumption,fix,if,when,where,while,for,from,by,on,off,also,moreover,conversely,
equivalently,finally,next,first,second,third,note,notation,convention,claim,case,step,algorithm,
figure,table,thm,prop,lem,cor,def,ex,rem,eq,eqn,ch,chap,pf,bc,ak,ns,ht,ph}

% Abbrev. with punctuation often seen after \n (e.g., \n(Proof))
\patchn{Proof)}\patchn{Remark)}\patchn{Theorem)}\patchn{Lemma)}\patchn{Corollary)}

% One-letter and digit fallbacks (covers \nA., \n1, etc. — NOT a full catch-all)
\newcommand{\patchnletter}[1]{\expandafter\def\csname n#1\endcsname{\unskip\space #1}}
\@tfor\@tempa:=ABCDEFGHIJKLMNOPQRSTUVWXYZabcdefghijklmnopqrstuvwxyz0123456789\do{%
  \expandafter\patchnletter\@tempa}
\makeatother



% === Document Metadata ===
\title{AK High-Dimensional Projection Structural Theory\\
\Large Version 15.0: Collapse Structures, Group Simplification, and Persistent Projection Geometry} 
\author{\textbf{Atsushi Kobayashi} \quad {\small (with ChatGPT Research Partner)}}
\date{August 2025}

% === Document Start ===
\begin{document}

\maketitle



\begin{abstract}
We present AK--HDPST v16.0, a two-layer functorial framework for controlled collapse in constructible one-parameter persistence over a field. The Core layer supplies machine-checkable statements; the [Spec] layer consists of auditable, non-expansive extensions used only under explicit hypotheses.

On the Core side we show that the exact Serre reflector \(\mathbf{T}_\tau\), which deletes precisely the bars of length \(\le \tau\), is idempotent and \(1\)-Lipschitz for interleavings, and admits a filtered lift \(C_\tau\) unique up to filtered quasi-isomorphism with \(\mathbf{P}_i C_\tau \simeq \mathbf{T}_\tau \mathbf{P}_i\). We certify collapse gates via a one-way bridge: if \(\mathrm{PH}_1(F)=0\) then \(\operatorname{Ext}^1(\mathcal{R}(F),k)=0\) under \(t\)-exact realizations of amplitude \(\le 1\) (no converse used). Directed systems are audited by windowed comparison maps \(\phi_{i,\tau}\); kernel/cokernel diagnostics \((\mu,\nu)\) detect failures of limit collapse even when all finite stages collapse. All comparisons are performed after truncation by the windowed protocol \(t \mapsto\) persistence \(\mapsto \mathbf{T}_\tau \mapsto\) compare on mutually exclusive, collectively exhaustive right-open windows. Policy-wise, deletion-type updates are non-increasing after truncation, whereas inclusion-type updates are non-expansive.

For pipeline control and reproducibility, a per-window \(\delta\)-ledger aggregates continuation shifts, commutation defects, and numerical tolerances. A safety margin B--Gate\(^{+}\) enforces \(\mathrm{gap}_\tau > \sum \delta\) per window, and Restart/Summability paste windowed certificates into global ones. Versioned logs and cross-linked hashes enable auditability and exact re-runs.

The [Spec] layer (non-expansive after truncation) includes: a lax monoidal tensor/collapse with windowed energy bounds and a mono regime yielding post-collapse dominance; projection formula and base change transported to persistence via the windowed protocol; and Fukaya-category realizations with action filtration (continuation \(1\)-Lipschitz; adding stops deletion-type). Mirror/Transfer pipelines contribute \(\delta\)-controlled commutation, with a permitted-operations table mapped to the \(\delta\)-ledger and B--Gate\(^{+}\).

Scope and limitations: we work only with constructible \(1\)D persistence over a field and do not claim \(\mathrm{PH}_1 \Leftrightarrow \operatorname{Ext}^1\). The organizing principle is a sharp boundary between theorems and [Spec] contracts, enabling safe reuse, cross-domain exploration, and machine-checkable testing while preserving verified guarantees.
\end{abstract}



% ===========================
% Chapter 1 : Collapse — Operational Definition and Scope
% ===========================
\section{Chapter 1: Collapse — Operational Definition and Scope}
\addcontentsline{toc}{section}{Collapse — Operational Definition and Scope}

%------------------------------------------
% Scope Box (short, IMRN/AIM-friendly)
%------------------------------------------
\noindent\fbox{%
\parbox{\textwidth}{%
\textbf{Scope Box (Global Guard-Rails, short version).}
All statements in this chapter (and throughout) are made under the following guard-rails:
\begin{itemize}[leftmargin=1.25em]
  \item Constructible 1D persistence over a \emph{field} (coefficients vary but remain fields).
  \item All equalities and exactness claims are asserted \emph{after collapse at the persistence layer}; filtered complex statements hold \emph{up to filtered quasi-isomorphism (f.q.i.)}.
  \item Non-commutation and implementation errors are externalized in a \(\delta\)-ledger with additive budget along pipelines.
  \item \emph{Windowed (MECE) proof policy} with restart and summability; \textbf{windows are right-open} half-open intervals \([u,u')\).
  \item The bridge \(\PH_1 \Rightarrow \Ext^1\) is used only in \(\Dbk\) under \(t\)-exactness and amplitude \(\le 1\); \emph{no converse is claimed}.
\end{itemize}
Detailed expansions of these guard-rails and their implementable ranges appear in the appendices.%
}%
}

\medskip

\subsection*{1.0. Windowed proof policy, Overlap Gate, Window Stack (WinFib), and B-Gate\(^{+}\)}
We formalize the window policy, the \emph{Overlap Gate} for gluing along overlaps (charts \(\times\) windows), the \emph{Window Stack} as a Grothendieck-fibrational bookkeeping device, and the standard single-layer acceptance gate on the B-side.

\begin{definition}[Windows (MECE), right-open endpoints, safety margin, and \(\delta\)-ledger]\label{def:windows-delta}
A \emph{domain window} is a right-open half-open interval \(W=[u,u')\subset\bbR\). A windowing is \emph{MECE} if a family \(\{[u_k,u_{k+1})\}_k\) forms a disjoint cover \([u_0,U)=\bigsqcup_k [u_k,u_{k+1})\) with shared endpoints only. The \emph{collapse window} is set by a threshold \(\tau>0\) (fixed within a gate, possibly varying across steps of a pipeline).

For an A-side pipeline with per-step non-commutation/error budgets \(\delta_j=\delta^{\mathrm{alg}}_j+\delta^{\mathrm{disc}}_j+\delta^{\mathrm{meas}}_j\), write the \emph{pipeline budget}
\[
\Sigma\delta(i,\{\tau_j\})\ :=\ \sum_j \delta_j(i,\tau_j).
\]
The \emph{safety margin} \(\mathrm{gap}_\tau\) is determined by the window and collapse policy and must dominate error accumulation. Mandatory logs ensure MECE coverage and additivity \(\sum_k \Sigma\delta_k=\Sigma\delta\).
\end{definition}

\begin{definition}[Window restriction (cropping) functor]\label{def:cropping}
For a window \(W=[u,u')\), define the exact endofunctor \(\Crop_{W}:\Perscons\to \Perscons\) by restricting bars to \(W\) (bars fully outside \(W\) are deleted; bars intersecting \(W\) are cut at the endpoints). On barcodes this is literal cropping; on persistence modules it is precomposition with the inclusion \(W\hookrightarrow\bbR\) and extension by zero outside \(W\). \(\Crop_{W}\) is \(1\)-Lipschitz for \(\intdist\).
\end{definition}

\begin{definition}[Overlap Gate]\label{def:overlap-gate}
Let \(\{(X_\alpha,W_\alpha)\}_{\alpha\in A}\) be a local cover, where \(X_\alpha\) is a chart (domain piece) and \(W_\alpha=[u_\alpha,u'_\alpha)\) a right-open window. For degree \(i\) and a fixed \(\tau>0\), write
\[
\mathcal{B}_{\alpha,i}\ :=\ \Ttau\,\Crop_{W_\alpha}\big(\mathbf{P}_i(F|_{X_\alpha})\big)\ \in\ \Perscons.
\]
For any overlapping pair \((\alpha,\beta)\) with non-empty overlap
\[
\Omega_{\alpha\beta}\ :=\ X_\alpha\cap X_\beta\ \times\ \bigl(W_\alpha\cap W_\beta\bigr),
\]
the \emph{Overlap Gate} \(\texttt{OG}(\alpha,\beta;i,\tau)\) \emph{passes} if
\begin{enumerate}[label=(\roman*),leftmargin=1.25em]
\item \(\Crop_{W_\alpha\cap W_\beta}\mathcal{B}_{\alpha,i}\) and \(\Crop_{W_\alpha\cap W_\beta}\mathcal{B}_{\beta,i}\) are \emph{isomorphic in} \(\Perscons\) up to budget:
\[
\intdist\!\Big(\Crop_{W_\alpha\cap W_\beta}\mathcal{B}_{\alpha,i},\ \Crop_{W_\alpha\cap W_\beta}\mathcal{B}_{\beta,i}\Big)\ \le\ \Sigma\delta_{\alpha\beta}(i,\tau),
\]
with \(\Sigma\delta_{\alpha\beta}\) the overlap-specific budget recorded in the \(\delta\)-ledger;
\item the safety margin on the overlap satisfies \(\mathrm{gap}_\tau(\Omega_{\alpha\beta})>\Sigma\delta_{\alpha\beta}(i,\tau)\);
\item tower diagnostics after \(\Ttau\) agree on the overlap (cf.\ \Cref{sec:typeIV}): \(\mu_{\mathrm{Collapse}}^{\,i}=\nu_{\mathrm{Collapse}}^{\,i}=0\) for both sides restricted to \(W_\alpha\cap W_\beta\).
\end{enumerate}
A \emph{global Overlap Gate pass} on \((\{X_\alpha\},\{W_\alpha\},i,\tau)\) means \(\texttt{OG}(\alpha,\beta;i,\tau)\) passes for all overlapping pairs.
\end{definition}

\begin{remark}[Window Stack (WinFib) as a Grothendieck fibration]\label{rk:winfib}
Let \(\mathsf{Win}\) be the small category whose objects are pairs \((\alpha,W_\alpha)\) with \(W_\alpha\) right-open, and morphisms \((\alpha,W_\alpha)\to(\beta,W_\beta)\) given by inclusions \(X_\alpha\cap W_\alpha\hookrightarrow X_\beta\cap W_\beta\). Define a pseudofunctor
\[
\mathcal{S}_i(-,\tau):\ \mathsf{Win}^{\mathrm{op}}\ \longrightarrow\ \mathrm{Cat},\qquad\n(\alpha,W_\alpha)\ \longmapsto\ \Big\{\Ttau\,\Crop_{W_\alpha}\big(\mathbf{P}_i(F|_{X_\alpha})\big)\Big\}\subset\Perscons,
\]
with reindexing functors induced by cropping and restriction. The associated category of elements \(\int \mathcal{S}_i(-,\tau)\) admits a canonical projection
\[
\pi:\ \mathsf{WinFib}_{i,\tau}\ :=\ \int \mathcal{S}_i(-,\tau)\ \longrightarrow\ \mathsf{Win},
\]
which is a Grothendieck fibration whose cartesian morphisms are induced by inclusions and cropping. The Overlap Gate certifies that objects in fibers glue along overlaps into a global B-side section. Stronger adjunction/commutation claims about \(\pi\) are \textbf{[Spec]} and developed in the appendices.
\end{remark}

\begin{definition}[B-Gate\(^{+}\) (standard single-layer gate on the B-side)]\label{def:bgate-plus}
Fix a right-open window \(W=[u,u')\), a collapse threshold \(\tau>0\), and a degree \(i\). We say \emph{B-Gate\(^{+}\) passes on \((W,\tau,i)\)} if, \emph{after collapse on the B-side},
\begin{enumerate}[label=(\arabic*),leftmargin=1.25em]
\item \(\PH_1(\Ctau F)=0\);
\item \(\Ext^1(\mathcal{R}(\Ctau F),k)=0\) (only within the scope stated in the Scope Box);
\item the tower audit \emph{after} \(\Ttau\) yields \((\mu_{\mathrm{Collapse}},\nu_{\mathrm{Collapse}})=(0,0)\) (tail isomorphism on comparison maps \(\phi\));
\item the safety margin satisfies \(\mathrm{gap}_\tau>\Sigma\delta(i,\{\tau_j\})\).
\end{enumerate}
If a local cover \(\{(X_\alpha,W_\alpha)\}\) is used, we additionally require a \emph{global Overlap Gate pass} on \((\{X_\alpha\},\{W_\alpha\},i,\tau)\). Optionally (auxiliary), we may require the \emph{auxiliary spectral bars} on the chosen spectral window to vanish after collapse; final gate decisions never rely solely on spectral indicators.
\end{definition}

\begin{remark}[Asymmetric mirror with \(\delta\)-ledger; soft-commuting policy]\label{rk:asym-mirror}
Let \(U_m,\dots,U_1\) be A-side steps (each deletion-type or \(\varepsilon\)-continuation), and \(C_{\tau_j}\) the stepwise collapses. For every \(i\) and any fixed B-side collapse threshold \(\tau\),
\[
\intdist\!\Big(\Ttau \mathbf{P}_i\big(\Mirror(C_{\tau_m}U_m\cdots C_{\tau_1}U_1F)\big),\ \Ttau \mathbf{P}_i\big(C_{\tau_m}U_m\cdots C_{\tau_1}U_1\Mirror F\big)\Big)\ \le\ \sum_{j=1}^m \delta_j(i,\tau_j),
\]
uniformly in \(F\). Subsequent \(1\)-Lipschitz post-processing on the persistence layer does not increase the right-hand side. If collapsers \(T_A,T_B\) are not provably commuting, we adopt the \emph{soft-commuting} policy: measure
\[
\Delta_{\mathrm{comm}}:=\intdist(T_AT_BM,T_BT_AM).
\]
If \(\Delta_{\mathrm{comm}}\le \eta\), we accept parallel collapse; else we fix an order and record \(\Delta_{\mathrm{comm}}\) in \(\delta^{\mathrm{alg}}\).
\end{remark}

\subsection*{1.1. Terminology and notation}
Fix a base field \(k\). Let \(\Vectk\) be finite-dimensional \(k\)-vector spaces and \(\Dbk\) the bounded derived category of finite-dimensional \(k\)-modules.

\paragraph{Filtered chain complexes.}
Write \(\FiltCh\) for finite-type (filtered-constructible) filtered chain complexes
\(\nF=(C_\bullet,\{F^{t}C_\bullet\}_{t\in\bbR})\n\)
over \(k\), with \(F^{t}C_\bullet\subseteq F^{t'}C_\bullet\) for \(t\le t'\) and filtered chain maps as morphisms. For each \(i\in\mathbb{Z}\) (bounded in homological degree), degreewise homology yields a persistence module with structure maps induced by the inclusions \(F^{t}C_\bullet\hookrightarrow F^{t'}C_\bullet\):
\[
\mathbf{P}_i(F)\colon \bbR\to \Vectk,\qquad t\longmapsto \mathrm{H}_i(F^{t}C_\bullet).
\]
Constructibility on bounded windows places \(\mathbf{P}_i(F)\) in \(\Perscons\). Its barcode is denoted by \(\PH_i(F)\). Writing \(\PH_1(F)=0\) means \(\mathbf{P}_1(F)\) is the zero object in \(\Perscons\).

\paragraph{Standing convention (constructible range).}
All persistence arguments take place in \(\Perscons\subset\Pers\) (finite critical set on bounded intervals). Abelianity, Serre subcategories, and exact localizations are used only within \(\Perscons\). When filtered (co)limits are used, they are computed objectwise in \([\bbR,\Vectk]\) and then verified to return to \(\Perscons\); we assert equalities \emph{at the persistence layer}. On filtered complexes we retain finite-type hypotheses and note filtered colimits may exit finiteness. We identify \(\Persft\) with \(\Perscons\) and henceforth write only \(\Perscons\).

\paragraph{Realization functor and test family.}
A realization is a \(t\)-exact functor \(\mathcal{R}:\FiltCh\to \Dbk\) of amplitude \(\le 1\). We fix the minimal test family \(\mathcal{Q}=\{k[0]\}\) for \(\Ext^1\)-vanishing tests.

\paragraph{Interleaving distance.}
Let \(\intdist\) denote the interleaving (equivalently, bottleneck) distance on \(\Perscons\). All Lipschitz claims use \(\intdist\).

\paragraph{Epistemic stance.}
We treat \emph{collapse} as a functorial simplification exposed via higher-dimensional projections and controlled obstruction removal. Operationally, we work with observable indicators \((\PH_1,\ \Ext^1,\ \mu_{\mathrm{Collapse}},\ \nu_{\mathrm{Collapse}})\), an admissible entry gate (the \emph{Collapse Zone}), and a failure typology including \emph{invisible} limit obstructions. We enforce a windowed proof policy and a single-layer judgement after collapse (B-side), with non-commutation externalized via a \(\delta\)-ledger. Windows are uniformly \textbf{right-open}.

\subsection*{1.2. Collapse (operational definition, v15.1)}
We define a metrically stable truncation on persistence and a filtered lift used for constructions.

\begin{definition}[Exact truncation on persistence and abbreviation]\label{def:Ttau}
For \(\tau>0\), let
\[
\mathbf{T}_\tau:\ \Perscons\longrightarrow \Orth
\]
be the exact reflective localization (Serre quotient) at the Serre subcategory generated by interval modules of length \(\le \tau\), i.e.\ \(\Ttau\) deletes precisely those finite bars. Here \(\Orth\) is the full \(\tau\)-local (orthogonal) subcategory (orthogonal to \(\Ecat\) for both \(\Hom\) and \(\Ext^1\)). Existence and exactness hold in the constructible 1D range (Crawley--Boevey 2015; Chazal--de Silva--Glisse--Oudot 2016). \(\Ttau\) is \(1\)-Lipschitz for \(\intdist\).
\end{definition}

\begin{definition}[Thresholded collapse lift on filtered complexes]\label{def:Ctau}
A \emph{thresholded collapse lift} is a functor
\[
\Ctau:\ \FiltCh\longrightarrow \FiltCh
\]
defined \emph{up to f.q.i.} such that, for every \(i\),
\[
\mathbf{P}_i\!\big(\Ctau(F)\big)\ \cong\ \Ttau\!\big(\mathbf{P}_i(F)\big)\quad\text{in }\Perscons.
\]
All (co)limit and pullback compatibilities used in this work are applied \emph{at the persistence layer} via these identifications.
\end{definition}

\begin{remark}[Endpoints and infinite bars]\label{rem:endpoints-infinite-bars}
Endpoint conventions are \emph{uniformly right-open} for windows \([u,u')\). Infinite bars and endpoint policies are centralized and referenced consistently; all later references defer to this remark.
\end{remark}

\begin{remark}[Lipschitz stability and scope]\label{rk:Ctau-stability}
\(\Ttau\) is \(1\)-Lipschitz for \(\intdist\), hence \(\Ctau\) is \(1\)-Lipschitz \emph{at the persistence level} via \(\mathbf{P}_i\!\circ \Ctau\simeq \Ttau\!\circ \mathbf{P}_i\). Compatibilities with filtered colimits and finite pullbacks are used only through \(\Ttau\). Spectral indicators (tails/heat traces) are \emph{not} f.q.i.-invariants; they are controlled by stability under a fixed policy \((\beta,M(\tau),s)\) on an appropriate linearization \(L(\Ctau F)\). Spectral quantities are strictly auxiliary.
\end{remark}

\begin{definition}[PH--collapse and Ext--collapse]
For \(F\in\FiltCh\):
\begin{itemize}[leftmargin=1.25em]
  \item \emph{PH--collapse in degree \(1\)}: \(\ \PH_1(F)=0\).
  \item \emph{Ext--collapse (tested against \(\mathcal{Q}=\{k[0]\}\))}: \(\ \Ext^1\!\big(\mathcal{R}(F),k\big)=0\).
\end{itemize}
\end{definition}

\begin{definition}[Collapse Zone (binary entry gate)]
Set
\[
\texttt{CollapseAdmissible}(F)\;:\!\!\iff\;\ \PH_1(F)=0\ \wedge\ \Ext^1\!\big(\mathcal{R}(F),k\big)=0.
\]
The \emph{Collapse Zone} is the full subcategory
\[
\mathfrak{C}\ :=\ \{\,F\in\FiltCh\mid \texttt{CollapseAdmissible}(F)\,\}\subset \FiltCh.
\]
Under the hypotheses of the Scope Box (bridge used only in \(\Dbk\) under \(t\)-exactness and amplitude \(\le 1\)), PH--collapse implies Ext--collapse; \emph{no converse is asserted}.
\end{definition}

\begin{remark}[Invariance]
\(\texttt{CollapseAdmissible}\) is invariant under filtered quasi-isomorphisms and under equivalences preserving \(\mathbf{P}_i(-)\) and \(\mathcal{R}(-)\) up to isomorphism in \(\Dbk\).
\end{remark}

\paragraph{Abstract functorial viewpoint.}
Let \(\mathsf{Triv}\subset\FiltCh\) be the full subcategory of objects with vanishing \(\PH_1\) and tested \(\Ext^1\). If the inclusion \(\iota:\mathsf{Triv}\hookrightarrow\FiltCh\) admits a right adjoint \(C\) (existence of a genuine right adjoint is \textbf{[Spec]} beyond the implementable range), we call such \(C\) a \emph{collapser}; the unit \(\eta_F:F\to \iota C(F)\) serves as a canonical collapse morphism. In practice, \(\Ctau\) (\Cref{def:Ctau}) is the operative, metrically stable proxy.

\subsection*{1.3. The entry gate as a measurable predicate}
The predicate \(\texttt{CollapseAdmissible}\) separates, by observable tests, objects whose first persistent topology and tested categorical extensions vanish:
\begin{itemize}[leftmargin=1.25em]
  \item \emph{Measurability:} \(\PH_1{=}0\) is read off barcodes; \(\Ext^1\)-vanishing is checked against \(\{k[0]\}\) in \(\Dbk\).
  \item \emph{Functorial simplification:} \(\Ctau\) erases bars of length \(\le \tau\) and, under the Scope Box hypotheses, preserves admissibility.
  \item \emph{Cross-domain interface:} the gate supports specifications (geometry, arithmetic, mirror/tropical, PDE) \emph{without} positing cross-domain equivalences.
\end{itemize}

\subsection*{1.4. Failure landscape and the invisible obstruction \texorpdfstring{\((\mu_{\mathrm{Collapse}},\nu_{\mathrm{Collapse}})\)}{(mu, nu)}}\label{sec:typeIV}
We classify failures of collapse and introduce tower-sensitive diagnostics.

\paragraph{Generic fiber dimension (intuition).}
For \(M\in\Perscons\), the \emph{generic fiber dimension} is the stabilized rank
\[
\gdim(M)\ :=\ \lim_{t\to+\infty}\dim_k M(t),
\]
which exists in the constructible setting. After truncation by \(\Ttau\), \(\gdim\) equals the multiplicity of the infinite bar \(I[0,\infty)\) in the barcode.

\paragraph{Observable failure types.}
\begin{itemize}[leftmargin=1.25em]
  \item \textbf{Type I (Topological):} \(\PH_1(F)\neq 0\).
  \item \textbf{Type II (Categorical):} \(\Ext^1\!\big(\mathcal{R}(F),k\big)\neq 0\).
  \item \textbf{Type III ([Spec]-level instability):} admissibility is unstable under a prescribed operation (e.g.\ a specific pullback/filtered colimit) in a way detectable at finite level.
\end{itemize}

\paragraph{Invisible failure along towers (Type IV).}
Let \(\{F_n\}_{n\in\bbN}\) be a directed system in \(\FiltCh\) with limit \(F_\infty\). For fixed \(\tau>0\) and each degree \(i\), the natural comparison \emph{after truncation by \(\Ttau\)}
\[
\phi_i:\ \varinjlim_{n}\ \Ttau\!\big(\mathbf{P}_i(F_n)\big)\ \longrightarrow\ \Ttau\!\big(\mathbf{P}_i(F_\infty)\big)
\]
induces the tower-sensitivity invariants
\[
\mu_{\mathrm{Collapse}}^{\,i}:=\gdim\ker(\phi_i),\qquad \nu_{\mathrm{Collapse}}^{\,i}:=\gdim\mathrm{coker}(\phi_i),
\]
interpreting \(\gdim\) as the multiplicity of \(I[0,\infty)\) after truncation. Set
\[
\mu_{\mathrm{Collapse}}:=\sum_i \mu_{\mathrm{Collapse}}^{\,i},\qquad \nu_{\mathrm{Collapse}}:=\sum_i \nu_{\mathrm{Collapse}}^{\,i}.
\]
We say \(F_\infty\) exhibits \emph{invisible failure (Type IV)} if \((\mu_{\mathrm{Collapse}},\nu_{\mathrm{Collapse}})\neq (0,0)\). In that case, all finite layers may appear admissible while the limit fails the gate. Finiteness follows from constructibility and bounded homological degrees.

\medskip
\noindent\emph{Purpose and non-identity.} \((\mu_{\mathrm{Collapse}},\nu_{\mathrm{Collapse}})\) diagnose limit effects of collapse \emph{at the persistence layer}; they are \emph{not} classical Iwasawa invariants. A canonical refinement-limit example exhibits pure cokernel-type failure at fixed \(\tau\). They are invariant under f.q.i.\ and cofinal reindexing.

\subsection*{1.5. Admissible A-side operations and single-layer judgement}
We record the permitted A-side steps and the single-layer policy.

\begin{definition}[Admissible A-side operations]\label{def:admissible-ops}
Each A-side step \(U\) on filtered complexes is labeled as:
\begin{itemize}[leftmargin=1.25em]
  \item \emph{Deletion-type (monotone):} induces non-increase of windowed persistence energies and auxiliary spectral counters after applying \(\Ctau\). Examples: stop addition/sector shrinking, mollification (low-pass), viscosity increase, threshold lowering.
  \item \emph{\(\varepsilon\)-continuation (non-expansive):} is \(1\)-Lipschitz for \(\intdist\); any drift is bounded by the declared \(\varepsilon\).
  \item \emph{Inclusion-type (stable only):} may increase indicators; only stability (non-expansiveness) is claimed; monotonicity is not asserted.
\end{itemize}
Every step is followed by a collapse \(C_{\tau_j}\) and all measurements are taken \emph{only} on the B-side single layer \(\Ttau\mathbf{P}_i\).
\end{definition}

\begin{declaration}[Windowed certificates, Overlap Gate, and pasting]\label{dec:windowed-certificate}
If B-Gate\(^{+}\) (\Cref{def:bgate-plus}) passes on \((W,\tau,i)\), we issue a \emph{windowed regularity certificate} for \((W,\tau)\): ``no failure with no false negative on \(W\).'' When a local cover \(\{(X_\alpha,W_\alpha)\}\) is used, we additionally require a \emph{global Overlap Gate pass} on \((\{X_\alpha\},\{W_\alpha\},i,\tau)\). Global statements are obtained by pasting windowed certificates along a MECE partition using the Restart and Summability policies. All cross-domain comparisons are performed after collapse, and any non-commutation is accounted for in the \(\delta\)-ledger.
\end{declaration}

%------------------------------------------
% Strengthened, fully unified endpoint policy and auxiliary reinforcements
%------------------------------------------
\subsection*{1.6. Strengthening notes (fully integrated)}
The following reinforcements are fully integrated above and govern all subsequent chapters:

\begin{itemize}[leftmargin=1.25em]
  \item \textbf{Unified endpoint policy.} All windows are \emph{right-open} \([u,u')\). Any earlier ``right-inclusive'' phrasing is obsolete. Cropping (\Cref{def:cropping}) and Overlap Gate (\Cref{def:overlap-gate}) are formulated for right-open windows uniformly.
  \item \textbf{Overlap-first gluing.} The Overlap Gate is elevated as the primary local-to-global device: global certificates are issued only after all pairwise overlaps pass, with budgets verified against \(\mathrm{gap}_\tau\) on overlaps.
  \item \textbf{Window Stack (WinFib).} Windows and parameters are internalized as a Grothendieck fibration (\Cref{rk:winfib}); cartesian morphisms encode type-safe consistency checks on overlaps and inclusions.
  \item \textbf{Guard-rails in a box.} The Scope Box at chapter head shortens (and sharpens) the global guard-rails: constructible 1D, field coefficients, after-collapse layer of truth, \(\delta\)-ledger, windowed (MECE) proof with restart and summability, and the one-way \(\PH_1\Rightarrow\Ext^1\) bridge only in \(\Dbk\).
  \item \textbf{Audit-first policy.} Tower diagnostics \((\mu_{\mathrm{Collapse}},\nu_{\mathrm{Collapse}})\) are evaluated \emph{after} \(\Ttau\); invisible failures (Type IV) are treated on equal footing with Types I–III. Certificates require tail isomorphisms (vanishing \(\mu,\nu\)).
  \item \textbf{Spectral indicators as auxiliary.} Spectral tails/heat traces are never sole gate criteria; they obey monotonicity for deletion-type steps and stability under \(\varepsilon\)-continuations, and are always measured after collapse.
\end{itemize}

%------------------------------------------
% Appendix-style pointers (non-intrusive, IMRN/AIM-friendly)
%------------------------------------------
\paragraph{References and provenance.}
Existence, exactness, and interval decomposition in the constructible 1D range follow Crawley--Boevey (2015) and Chazal--de Silva--Glisse--Oudot (2016). Derived and sheaf-theoretic realizations beyond \(\Dbk\) are \textbf{[Spec]} and rely on projection formula/base change; they do not enlarge the proven bridge. Strong adjunction/commutation claims for WinFib are also \textbf{[Spec]}.



latex
% ===========================
% Chapter: Concrete Model — Finite-Type Filtered Chain Complexes and Thresholded Collapse
% ===========================
\section{Chapter 2: Concrete Model — Finite-Type Filtered Chain Complexes and Thresholded Collapse}
\addcontentsline{toc}{section}{Concrete Model — Finite-Type Filtered Chain Complexes and Thresholded Collapse}

\subsection*{2.1. The category \texorpdfstring{$\FiltCh$}{FiltCh(k)} and persistence modules}
Fix a field \(k\). Let \(\mathsf{Ch}^{b}(k)\) be the category of bounded chain complexes of finite-dimensional \(k\)-vector spaces. A \emph{finite-type filtered chain complex} is a pair
\[
F=(C_\bullet,\{F^{t}C_\bullet\}_{t\in\bbR})
\]
where \(F^{t}C_\bullet\subseteq F^{t'}C_\bullet\) for \(t\le t'\), the filtration is exhaustive and left-bounded, and, for each \(i\), the persistence module
\[
\mathrm{H}_i(F):\ \bbR\longrightarrow \Vectk,\qquad t\longmapsto \mathrm{H}_i(F^{t}C_\bullet)
\]
is pointwise finite-dimensional with finitely many critical parameters on compact intervals. Denote by \(\FiltCh\) the category of such \(F\) with filtration-preserving chain maps. For each \(i\), let
\[
\mathbf{P}_i:\ \FiltCh\longrightarrow \Perscons
\]
be the functor sending \(F\mapsto \mathrm{H}_i(F)\). We write \(\PH_i(F)\) for the barcode (multiset of intervals) of \(\mathbf{P}_i(F)\). Throughout, the interleaving (equivalently, bottleneck) distance on persistence modules is denoted \(\intdist\).

\paragraph{Standing convention (constructible range and notation).}
We work inside the constructible subcategory
\(\Perscons\subset\Pers\) (finite critical set on bounded intervals, equivalently p.f.d.\ with finitely many changes on compacts).
We identify \(\Persft\) with \(\Perscons\) by convention.
\(\Perscons\) is abelian, admits interval decompositions, and carries a well-defined length. All uses of abelianity, Serre subcategories, and exact localizations are made within \(\Perscons\); see Appendix~A for details.
For filtered complexes we keep the finite-type hypothesis and record that filtered colimits may exit finiteness (cf.\ Appendix~A). Filtered (co)limits, when used, are computed objectwise in \([\bbR,\Vectk]\) and then verified to return to \(\Perscons\); no claim is made outside this regime.

\begin{remark}[Constructible abelian setting]\label{rem:constructible-abelian}
Within \(\Perscons\), kernels and cokernels are computed pointwise and preserve finiteness; thus \(\Perscons\) is abelian with interval decompositions and Serre localizations by bar length. For each \(\tau>0\), the full subcategory \(\Ecat\) generated by interval modules of length \(\le \tau\) is a hereditary Serre (localizing) subcategory in this constructible \(1\)D setting; see Crawley--Boevey (2015); Chazal--de Silva--Glisse--Oudot (2016).
\end{remark}

\begin{remark}[Windowed proof policy (MECE); right-open endpoints]\label{rem:ch2-mece}
All statements here are meant to be applied \emph{per window} (cf.\ Chapter~1). A \emph{domain window} is a right-open half-open interval \([u_k,u_{k+1})\). A windowing is \emph{MECE} if \(\bigsqcup_k [u_k,u_{k+1})=[u_0,U)\), intervals are ordered and adjacent windows meet only in their endpoints. Two mandatory coverage checks are recorded in each run (Appendix~G): (i) \(\sum_k (u_{k+1}-u_k)=U-u_0\) (total length equals the sum of window lengths), (ii) the number of events (births/deaths counted with multiplicity) on \([u_0,U)\) equals the sum of per-window counts up to rounding tolerance. Collapse thresholds and spectral bins are \emph{fixed on each window} and used only \emph{after collapse} at the persistence layer.
\end{remark}

\subsection*{2.2. Thresholded collapse: Serre localization on persistence and filtered lift}
We first recall truncation on persistence modules and then lift it to filtered complexes.

\paragraph{Ephemeral part and localization (constructible \(1\)D case).}
Let \(\Ecat\subset \Perscons\) be the Serre subcategory generated by interval modules of length \(\le \tau\).
The reflective localization
\[
\Ttau:\ \Perscons\longrightarrow \Orth
\]
is \emph{exact in the constructible \(1\)D range} and \(1\)-Lipschitz for \(\intdist\). Here \(\Orth\) denotes the \(\tau\)-local (orthogonal) subcategory (orthogonal to \(\Ecat\) for both \(\Hom\) and \(\Ext^1\)); in general, ``\(\tau\)-local'' is stricter than ``\(\tau\)-torsion-free''.
We interpret “deleting bars of length \(\le\tau\)” via this exact localization: for \(M\in\Perscons\),
\[
\Ttau(M)=M/E_\tau(M),\qquad E_\tau(M)\ \text{ the maximal \(\tau\)-ephemeral subobject.}
\]

\begin{remark}[Endpoints and infinite bars]\label{rk:2-endpoints}
\(\Ttau\) deletes precisely the \emph{finite} bars of length \(\le\tau\); infinite bars are invariant. The choice of open/closed endpoint conventions does not affect \(\Ttau\), since interleaving-equivalence classes (hence bar lengths) are preserved.
\end{remark}

\paragraph{Serre reflector: proof sketch (exactness).}
Exactness of \(\Ttau\) follows from exactness of the Serre quotient \(\Perscons\to \Perscons/\Ecat\) and identification of the quotient with \(\Orth\): localization at a hereditary Serre subcategory is exact; in the constructible \(1\)D barcode setting the quotient is abelian and the orthogonal is reflective. As a left adjoint, \(\Ttau\) preserves colimits; being exact it preserves finite limits/colimits. Shift-commutation yields \(1\)-Lipschitzness for \(\intdist\). Full details are deferred to Appendix~A.

\paragraph{Lifting to filtered complexes.}
Choose a functor
\[
\mathcal{U}:\ \Perscons\longrightarrow \FiltCh
\]
such that \(\mathbf{P}_i(\mathcal{U}(M))\cong M\) for \(M\) concentrated in degree \(i\), and \(\mathcal{U}\) sends finite direct sums of degree-concentrated modules to finite direct sums of elementary filtered complexes built by gluing interval complexes. For \(F\in \FiltCh\) define \(C_\tau(F)\) to be any object of \(\FiltCh\) equipped with filtered quasi-isomorphisms in each degree \(i\)
\[
\mathbf{P}_i\big(\Ctau(F)\big)\ \xrightarrow{\ \cong\ }\ \Ttau\!\big(\mathbf{P}_i(F)\big).
\]
Such choices exist and are unique up to filtered quasi-isomorphism; any functorial choice (up to f.q.i.) is a \emph{thresholded collapse functor}. A genuine right adjoint/comonadic presentation at the filtered level is \textbf{[Spec]} beyond the implementable range; all exactness claims are asserted at the persistence layer.

\subsection*{2.3. Stability and basic calculus of \texorpdfstring{$\Ctau$}{C\_\texttau}}
We collect the properties needed later; equalities are in \(\Perscons\) and filtered-level claims are up to f.q.i.

\begin{lemma}[Shift–commutation and \(1\)-Lipschitz]\label{lem:shift}
Let \(S^\varepsilon\) denote the \(\varepsilon\)-shift on persistence modules. Then \(\Ttau\circ S^\varepsilon\cong S^\varepsilon\circ \Ttau\). Consequently \(\Ttau\) preserves \(\varepsilon\)-interleavings and is \(1\)-Lipschitz for \(\intdist\).
\end{lemma}

\begin{proposition}[Stability, calculus, and (co)limit compatibility]\label{prop:stability}
Let \(\tau,\sigma>0\) and \(F,G\in\FiltCh\).
\begin{enumerate}[leftmargin=1.5em]
  \item \emph{(Non-expansiveness)} For every \(i\),
  \(\intdist(\mathbf{P}_i(\Ctau F),\mathbf{P}_i(\Ctau G)) \le \intdist(\mathbf{P}_i(F),\mathbf{P}_i(G))\).
  \item \emph{(Monotonicity and idempotence)} If \(\tau\le\sigma\), then for every \(i\) there is a natural epimorphism \(\mathbf{P}_i(\Ctau F)\twoheadrightarrow \mathbf{P}_i(\Csigma F)\), and \(\Ctau\circ \Csigma\simeq C_{\max\{\tau,\sigma\}}\simeq \Csigma\circ \Ctau\) up to f.q.i.\ (equivalently, \(\Ttau\circ \mathbf{T}_\sigma=\mathbf{T}_{\max\{\tau,\sigma\}}\) at the persistence layer).
  \item \emph{(Exactness on persistence)} Viewing \(\Ttau\) as the localization functor \(\Perscons\to \Perscons/\Ecat\) composed with the identification of the quotient with \(\Orth\), \(\Ttau\) is exact and preserves finite limits/colimits. Consequently, after applying \(\mathbf{P}_\ast\) degreewise to a short exact sequence in \(\FiltCh\), exactness is preserved under \(\Ttau\) at the persistence layer. No claim of short exactness for \(\mathbf{P}_i\circ \Ctau\) is made in general.
  \item \emph{(Filtered colimits; \textbf{[Spec]})} If \(\{F_\lambda\}_{\lambda\in\Lambda}\) is a filtered diagram in \(\FiltCh\) whose colimit is computed degreewise on chains/filtrations, then for every \(i\),
  \(\mathbf{P}_i(\Ctau(\varinjlim_\Lambda F_\lambda)) \cong \varinjlim_\Lambda \mathbf{P}_i(\Ctau(F_\lambda))\).
  \item \emph{(Finite pullbacks; \textbf{[Spec]})} If finite pullbacks in \(\FiltCh\) are computed degreewise and \(\mathcal{U}\) preserves finite limits up to f.q.i., then for every pullback square \(F\times_H G\),
  \(\mathbf{P}_i(\Ctau(F\times_H G))\cong \mathbf{P}_i(\Ctau(F)\times_{\Ctau(H)}\Ctau(G))\).
\end{enumerate}
\end{proposition}

\begin{remark}[Scope of limit/pullback compatibilities]
Compatibilities of \(\Ctau\) with finite pullbacks and filtered colimits are used \emph{at the persistence layer} via \(\mathbf{P}_i\circ \Ctau\simeq \Ttau\circ \mathbf{P}_i\).
\textbf{Statements at the filtered-complex level are adopted as [Spec] and hold up to f.q.i.}
\end{remark}

\begin{remark}[Monotonicity vs.\ stability; deletion- vs.\ inclusion-type]\label{rk:monotone-stable}
For \emph{deletion-type} updates (Dirichlet restriction/absorbing boundaries, principal submatrices/Schur complements, positive-semidefinite Loewner contractions, conservative averaging), windowed persistence energies and spectral tails/heat traces computed on \(L(\Ctau F)\) are \emph{non-increasing after truncation}. For \emph{inclusion-type} updates we claim only \emph{stability} (non-expansiveness). Spectral indicators are not f.q.i.-invariants; they are controlled by a fixed policy \((\beta,M(\tau),s)\) (see Chapter~11 and Appendix~E).
\end{remark}

\begin{declaration}[Spec–\texorpdfstring{$\Ctau$}{C\_\texttau} calculus]\label{spec:Ctau-calc}
The filtered lift \(\Ctau\) realizes \(\Ttau\) degreewise \emph{up to filtered quasi-isomorphism}. Compatibility with filtered colimits and finite pullbacks is used \emph{at the persistence layer} via \(\mathbf{P}_i\circ \Ctau\simeq \Ttau\circ \mathbf{P}_i\). \textbf{No stronger claim is made at the filtered-complex level.}
\end{declaration}

\subsection*{2.4. Windowing (MECE), \texorpdfstring{$\tau$}{tau}-adaptation, and spectral bins}
We formalize the operating policies required for windowed certificates and for the interaction with spectral auxiliaries.

\begin{definition}[Domain windows (MECE) and coverage checks]\label{def:ch2-mece}
A \emph{domain windowing} is a finite or countable family of right-open intervals \(\{[u_k,u_{k+1})\}_k\) such that \(\bigsqcup_k [u_k,u_{k+1})=[u_0,U)\) (disjoint union) and \(u_k<u_{k+1}\). Coverage checks:
\[
\sum_k (u_{k+1}-u_k)=U-u_0,\qquad \#\mathrm{Events}([u_0,U))=\sum_k \#\mathrm{Events}([u_k,u_{k+1}))\ (\pm\ \text{rounding}).
\]
Here \(\#\mathrm{Events}\) counts births/deaths (with multiplicity) observed by \(\mathbf{P}_i\) on the specified window. Verdicts are issued per window; global statements are obtained by pasting certificates as in Chapter~4.
\end{definition}

\begin{definition}[Collapse thresholds and \(\tau\)-adaptation]\label{def:tau-adapt}
Fix a collapse threshold \(\tau>0\) on each window. A default adaptation ties \(\tau\) to numerical/filtration resolution:
\[
\tau\ =\ \alpha\cdot \max\{\Delta t,\Delta x\}\qquad (\alpha>0\ \text{fixed per run}),
\]
or any comparable rule documented in the run manifest (Appendix~G). A \emph{\(\tau\)-sweep} is a discrete set \(\{\tau_\ell\}\) on which \((\mu_{\mathrm{Collapse}},\nu_{\mathrm{Collapse}})\) and B-Gate\(^{+}\) are evaluated. A \emph{stable band} is a contiguous range of \(\tau\)-values on which \((\mu_{\mathrm{Collapse}},\nu_{\mathrm{Collapse}})=(0,0)\) holds (Appendix~J); certificates are issued within stable bands.
\end{definition}

\begin{definition}[Spectral bins and auxiliary bars (aux-bars)]\label{def:aux-bars}
For a spectral operator with (ascending) spectrum \((\lambda_m)_{m\ge 1}\), fix a bin width \(\beta>0\) and a range \([a,b]\). Define right-open bins \(I_r=[a+r\beta,a+(r+1)\beta)\) for \(r=0,1,\dots,\lfloor (b-a)/\beta\rfloor-1\), and count occupancies \(E_r:=\#\{m\mid \lambda_m\in I_r\}\). Underflow \(\{\lambda_m<a\}\) and overflow \(\{\lambda_m\ge b\}\) are recorded. Along a discrete index (e.g.\ step number), sequences \((E_r(j))_j\) define \emph{auxiliary spectral bars (aux-bars)} as maximal consecutive runs on which \(E_r(j)>0\); their \emph{lifetimes} are measured in the discrete index (rescaled when needed). After applying \(\Ctau\) on the persistence side, indicators from aux-bars are evaluated as \emph{auxiliary gauges}: \emph{monotone} under deletion-type steps, \emph{stable} under \(\varepsilon\)-continuations (Appendix~E). They never replace the B-side gate.
\end{definition}

\begin{remark}[Separation of roles: persistence vs.\ spectral auxiliaries]\label{rem:sep-persist-spectral}
Gate decisions are made on the persistence layer after collapse, never on spectral auxiliaries alone. Aux-bars are used as supportive evidence (e.g.\ “aux-bars\(=0\) after \(C_\tau\) on the spectral window’’), with their binning policy \((\beta,[a,b])\) recorded. This maintains invariance guarantees of \(\Ttau\) and the reproducibility of windowed certificates.
\end{remark}

\subsection*{2.5. Collapse admissibility and robust variants}
Let \(\mathcal{R}:\FiltCh\to\Dbk\) be a fixed \(t\)-exact realization of amplitude \(\le 1\), and fix the minimal test family \(\mathcal{Q}=\{k[0]\}\).

\begin{definition}[Admissibility predicate]
For \(F\in\FiltCh\), set
\[
\texttt{CollapseAdmissible}(F)\ :\iff\ \PH_1(F)=0\ \ \wedge\ \ \Ext^1\!\big(\mathcal{R}(F),k\big)=0.
\]
Under the bridging assumptions (Chapter~3), \(\PH_1(F)=0\Rightarrow \Ext^1(\mathcal{R}(F),k)=0\) in \(\Dbk\).
\end{definition}

\begin{definition}[Robust admissibility at scale \(\varepsilon\)]
Fix \(\varepsilon>0\). We say \(F\) is \emph{\(\varepsilon\)-robustly collapse-admissible} if
\[
\PH_1\!\big(C_\varepsilon(F)\big)=0\quad\text{and}\quad \Ext^1\!\big(\mathcal{R}(C_\varepsilon(F)),k\big)=0.
\]
This weakens the binary gate by discarding bars shorter than \(\varepsilon\); \(\varepsilon\) reflects the admissible noise level.
\end{definition}

\begin{remark}[Usage and guarantees]
By \Cref{prop:stability}, \(\Ctau\) provides a metrically stable simplification compatible with filtered colimits and finite pullbacks (up to f.q.i.), furnishing a robust pre-processing step toward admissibility checks. In practice, one fixes an application-level noise scale \(\varepsilon\) and tests admissibility on \(C_\varepsilon(F)\); the one-way bridge then applies in \(\Dbk\). No global equivalence \(\PH_1\Leftrightarrow \Ext^1\) is asserted; only \(\PH_1\Rightarrow \Ext^1\) under explicit bridging hypotheses.
\end{remark}

\begin{remark}[Formalizability]
The ingredients of this chapter are formalizable: (i) \(\Ecat\) is a localizing Serre subcategory in the constructible range; (ii) \(\Ttau\) is exact and preserves finite limits/colimits; (iii) \Cref{lem:shift} implies \(1\)-Lipschitz continuity in \(\intdist\); and (iv) \Cref{prop:stability} encodes the (co)limit and pullback compatibility. These can be encoded in Coq/Lean as axiomatized interfaces for the persistence layer, with filtered lifts handled up to f.q.i.\ (Appendices~A–B).
\end{remark}

\begin{lemma}[\(\varepsilon\)-survival under interleaving perturbations]\label{lem:epsilon-survival}
If \(\intdist(\mathbf{P}_i(F),\mathbf{P}_i(G))\le \varepsilon\), then for every bar \(b\) of \(\mathbf{P}_i(F)\) and every \(\tau_0>0\), letting \(\ell_{\tau_0}(b)\) denote the \([0,\tau_0]\)-clipped length of \(b\), the following holds:
if \(\ell_{\tau_0}(b)>2\varepsilon\), then \(b\) has a corresponding bar in \(\mathbf{T}_{\tau_0}(\mathbf{P}_i(G))\) whose \([0,\tau_0]\)-clipped length is at least \(\ell_{\tau_0}(b)-2\varepsilon\). (See Appendix~I for the standard matching argument.)
\end{lemma}

\subsection*{2.6. Local Equivalence on saturation windows (PH–Ext equivalence \emph{per window})}
We elevate the window-local PH–Ext equivalence to a theorem under explicitly verifiable hypotheses; this replaces any global equivalence claims.

\begin{definition}[Saturation window, length gap, and near-\(\tau\) non-accumulation]\label{def:saturation}
Fix degree \(i=1\), a right-open window \(W=[u,u')\), and \(\tau>0\).
We say \(W\) is a \emph{saturation window at \(\tau\)} for \(F\) if:
\begin{itemize}[leftmargin=1.25em]
  \item \emph{Event stability:} births/deaths in \(\mathbf{P}_1(F)\) restricted to \(W\) stabilize (no new events beyond a finite stage for towers, or no significant drift for single objects).
  \item \emph{Length gap:} the maximal finite bar length inside \(W\) is \(\le \tau-\eta\) for some \(\eta>0\).
  \item \emph{Near-\(\tau\) non-accumulation:} no sequence of bar lengths in \(W\) strictly increases to \(\tau\).
\end{itemize}
We additionally require the post-collapse tail isomorphism at \(\tau\): \(\mu_{\mathrm{Collapse}}=\nu_{\mathrm{Collapse}}=0\) on \(W\) for degree \(1\) (cf.\ \Cref{prop:stability} and Chapter~1).
\end{definition}

\begin{theorem}[Local Equivalence on saturation windows]\label{thm:local-equiv}
Let \(F\in\FiltCh\). Assume:
\begin{enumerate}[leftmargin=1.5em]
  \item \(t\)-exact realization \(\mathcal{R}:\FiltCh\to \Dbk\) of amplitude \(\le 1\) (two-term model for \(\mathcal{R}(C_\tau F)\));
  \item \(W=[u,u')\) is a saturation window at \(\tau>0\) in the sense of \Cref{def:saturation};
  \item The tail comparison map \(\phi_{1,\tau}\) on \(W\) is an isomorphism (i.e.\ \(\mu_{\mathrm{Collapse}}=\nu_{\mathrm{Collapse}}=0\) after \(\Ttau\)).
\end{enumerate}
Then, \emph{on the window \(W\) and at threshold \(\tau\)}, one has the equivalence
\[
\PH_1\!\big(\Ctau F\big)=0\quad\Longleftrightarrow\quad \Ext^1\!\big(\mathcal{R}(\Ctau F),k\big)=0.
\]
\end{theorem}

\noindent\emph{Proof (sketch).} Under (1), \(\mathcal{R}(\Ctau F)\in D^{[-1,0]}(k\text{-mod})\) and the edge identification (Appendix~C) yields \(\Ext^1(\mathcal{R}(\Ctau F),k)\cong \Hom(H^{-1}(\mathcal{R}(\Ctau F)),k)\). Under the saturation hypotheses (2) and the tail isomorphism (3), the degree-\(1\) persistence on \(W\) is determined by stabilized bars, and near-\(\tau\) accumulation is excluded; after truncation at \(\tau\) the barcode is insensitive to residual finite fluctuations. Hence, \(\PH_1(\Ctau F)=0\) on \(W\) if and only if the stabilized edge group \(H^{-1}(\mathcal{R}(\Ctau F))\) vanishes, i.e.\ \(\Ext^1(\mathcal{R}(\Ctau F),k)=0\). \qed

\begin{remark}[Scope and use]
\Cref{thm:local-equiv} is a \emph{window-local} equivalence and replaces any global PH–Ext equivalence claims. It is designed to be verifiable as a gate condition (cf.\ Chapter~1): event stability, length gap, near-\(\tau\) non-accumulation, and tail isomorphism (\(\mu=\nu=0\)) are inspected on \(W\). Outside saturation windows, only \(\PH_1\Rightarrow \Ext^1\) is asserted (in \(\Dbk\)).
\end{remark}

\subsection*{2.7. Length Spectrum Operator \texorpdfstring{$\Len$}{Lambda\_len} (windowed) and its invariance}
We formalize the “length spectrum’’ associated to a barcode and use it as a robust, windowed proxy. This complements but never replaces B-side gates.

\begin{definition}[Windowed length spectrum operator \(\Len\)]\label{def:len-operator}
Let \(M\in\Perscons\) admit a barcode decomposition \(M\cong \bigoplus_{j} I[b_j,d_j)\).
For a right-open window \(W=[u,u')\), define the \emph{clipped length} \(\ell_W(I[b_j,d_j)):=\max\{0,\min\{d_j,u'\}-\max\{b_j,u\}\}\). The \emph{windowed length spectrum operator} \(\Len(M;W)\) is the diagonal (commutative) endomorphism on \(\bigoplus_j k\cdot e_j\) with eigenvalues \(\{\ell_W(I[b_j,d_j))\}_j\).
\end{definition}

\begin{proposition}[Isomorphism invariance and identification]\label{prop:length-spectrum}
The unordered multiset of eigenvalues of \(\Len(M;W)\) equals the multiset of clipped bar lengths \(\{\ell_W(I[b_j,d_j))\}_j\). This multiset is invariant under isomorphisms \(M\simeq M'\) in \(\Perscons\). In particular, \(\Len(-;W)\) is \emph{well-defined up to isomorphism of bar decompositions}.
\end{proposition}

\begin{remark}[Functoriality caveat and intended use]
\(\Len(-;W)\) is “natural up to isomorphism’’ (barcode uniqueness) and not fully functorial under arbitrary morphisms; it is designed for \emph{windowed auditing} and stability checks (e.g.\ non-increase under deletion-type updates after collapse). For links with classical spectral quantities (heat traces/tails) and sandwich bounds, see Chapter~11 and Appendix~E. \(\Len\) is never a sole gate criterion.
\end{remark}

\subsection*{2.8. Operating summary for Chapter~2}
On each right-open window \([u_k,u_{k+1})\):
\begin{itemize}[leftmargin=1.25em]
  \item Fix a collapse threshold \(\tau\) (adapted to resolution; \Cref{def:tau-adapt}) and, if spectral auxiliaries are used, fix a bin policy \((\beta,[a,b])\) (\Cref{def:aux-bars}).
  \item Apply A-side steps (deletion-type or \(\varepsilon\)-continuation), then collapse \(\Ctau\), and \emph{measure} solely on the B-side \(\Ttau\mathbf{P}_i\).
  \item Record the \(\delta\)-ledger for non-commutation and verify B-Gate\(^{+}\) conditions (Chapter~1); if a local cover is used, verify the Overlap Gate on overlaps.
  \item Issue a windowed certificate when gates pass; global claims are obtained by pasting windowed certificates along the MECE partition using Restart and Summability (Chapter~4).
  \item If \Cref{thm:local-equiv} applies (saturation window), \emph{then and only then} use the local PH–Ext equivalence per window in the gate; otherwise retain the one-way implication.
\end{itemize}

\paragraph{References and provenance.}
Serre localization \(\Ttau\) and constructible-barcode abelianity follow Crawley--Boevey (2015) and Chazal--de Silva--Glisse--Oudot (2016). Filtered lifts \(\Ctau\) are asserted up to f.q.i.; exactness and limit compatibilities are stated at the persistence layer. Local PH–Ext equivalence is a window-local theorem under amplitude \(\le 1\), saturation, and tail isomorphism; its proof relies on the edge identification in \(\Dbk\) (Appendix~C) and tower diagnostics (Appendix~D).



% ===========================
% Chapter 3: A One-Way Bridge PH$_1\!\to$ Ext$^1$ and the Hypothesis Scheme (reinforced)
% ===========================
\section{Chapter 3: A One-Way Bridge \texorpdfstring{$\mathrm{PH}_1\!\Rightarrow\!\mathrm{Ext}^1$}{PH1⇒Ext1} and the Hypothesis Scheme}
\addcontentsline{toc}{section}{A One-Way Bridge $\mathrm{PH}_1\Rightarrow\mathrm{Ext}^1$ and the Hypothesis Scheme}

\noindent\textbf{Scope note (reinforced windowed policy).}
All statements of this chapter lie within the constructible \(1\)D regime of Chapter~2 and use field coefficients.
The implication \(\mathrm{PH}_1\Rightarrow \Ext^1\) is proved only in \(D^{\mathrm{b}}(k\text{-mod})\) under \textup{(B1)}–\textup{(B3)} below.
Every claim is issued per domain window (right-open; Chapter~1, Def.~1.0 and Chapter~2, Remark~\ref{rem:ch2-mece}); gate decisions are taken only on the B-side after collapse, i.e.\ on single-layer objects \(\mathbf{T}_\tau\mathbf{P}_i\) (equivalently \(\mathbf{P}_i(C_\tau -)\)).
Filtered-complex equalities hold only up to filtered quasi-isomorphism (f.q.i.).

\subsection*{3.0. Windowed usage and gate integration}
This chapter provides the \(\mathrm{PH}_1\!\Rightarrow\!\Ext^1\) bridge used by B-Gate\(^{+}\) (Chapter~1, Def.~1.0) after applying the collapse \(C_\tau\) on a fixed window \(W=[u,u')\) and a fixed threshold \(\tau>0\):
\begin{itemize}
  \item On \((W,\tau)\), if \(\mathrm{PH}_1(C_\tau F)=0\), then (under \textup{(B1)}–\textup{(B3)}) \(\Ext^1(\mathcal{R}(C_\tau F),k)=0\).
  \item Together with \((\mu_{\mathrm{Collapse}},\nu_{\mathrm{Collapse}})=(0,0)\) at \((W,\tau)\) and the safety margin \(\mathrm{gap}_\tau>\Sigma\delta\), this yields B-Gate\(^{+}\) passing and thus a windowed certificate on \((W,\tau)\) (Chapter~1, Def.~1.0).
  \item Outside the scope of \textup{(B1)}–\textup{(B3)}, \(\Ext^1\) is not used; B-Gate\(^{+}\) may be evaluated with PH and tower-only parts (cf.\ Chapter~1).
\end{itemize}
All cross-domain comparisons (PF/BC, Mirror/Transfer) are performed after collapse, and any non-commutation is recorded in the \(\delta\)-ledger (Appendix~L).

\subsection*{3.1. Bridging Hypotheses (B1–B3)}
We fix the notation and conventions of Chapter~2. In particular, \(k\) is a field, \(\mathsf{FiltCh}(k)\) denotes finite-type (constructible) filtered chain complexes, \(\mathbf{P}_i(F)\) is the degree-\(i\) persistence module with barcode \(\mathrm{PH}_i(F)\), and \(\mathcal{R}:\mathsf{FiltCh}(k)\to D^{\mathrm{b}}(k\text{-mod})\) is a fixed \(t\)-exact realization functor.\footnote{We work in cohomological amplitude \([-1,0]\): amplitude control and \(t\)-exactness place \(\mathcal{R}(F)\) in \(D^{[-1,0]}\) and identify the edge map; see Appendix~C.}

\begin{description}
\item[\normalfont (B1) Finite-type over a field.]
\(F\in\mathsf{FiltCh}(k)\) with pointwise finite-dimensional persistence. Filtered (co)limits of constructible persistence modules are computed objectwise in \([\mathbb{R},\mathsf{Vect}_k]\) and used only under the scope policy of Appendix~A (compute in the functor category and verify return to \(\Pers^{\mathrm{cons}}_k\)); no claim is made outside this regime. Equalities are asserted only at the persistence layer.

\item[\normalfont (B2) Amplitude \(\boldsymbol{\le 1}\) and identification of the \(\boldsymbol{H^{-1}}\)-edge.]
There is a natural isomorphism
\[
H^{-1}\!\big(\mathcal{R}(F)\big)\ \cong\ \varinjlim_{t}\ H_1(F^{t}C_\bullet),
\]
and \(\mathcal{R}(F)\in D^{[-1,0]}(k\text{-mod})\).
Equivalently, \(\mathcal{R}(F)\) admits a two-term model
\[
\mathcal{R}(F)\ \simeq\ \bigl[\, \varinjlim_{t} H_1(F^{t}C_\bullet)\ \xrightarrow{\,d\,}\ \varinjlim_{t} H_0(F^{t}C_\bullet)\, \bigr],
\]
concentrated in cohomological degrees \((-1,0)\), functorial in \(F\).
Here exactness of filtered colimits in \(\mathsf{Vect}_k\) and the \(t\)-exactness of \(\mathcal{R}\) ensure functoriality in \(F\); all statements are confined to \(D^{[-1,0]}(k\text{-mod})\) (see Appendix~C).

\item[\normalfont (B3) Edge identification for degree \(1\) with \(Q=k\).]
For any \(A\in D^{[-1,0]}(k\text{-mod})\),
\[
\Ext^1(A,k)\ \cong\ \Hom\!\big(H^{-1}(A),k\big),
\]
naturally in \(A\), by applying \(\mathbf{R}\!\operatorname{Hom}(-,k)\) to the truncation triangle \(\tau_{\le -1}A\to A\to \tau_{\ge 0}A\to\) and using that \(k\) is a field; cf.\ Appendix~C.
\end{description}

\begin{remark}[On (B2) and the edge identification]\label{rk:B2-edge}
We use a two-term realization with amplitude \(\le 1\) so that \(H^{-1}(\mathcal{R}(F))\cong \varinjlim_t H_1(F^tC_\bullet)\), relying on exactness of filtered colimits in \(\mathsf{Vect}_k\). Over a field, for \(A\in D^{[-1,0]}\) one has \(\Ext^1(A,k)\cong \Hom(H^{-1}(A),k)\). Proof details and naturality are in Appendix~C. The bridge of this chapter is proved only in \(D^{\mathrm{b}}(k\text{-mod})\). Uses of other coefficient fields (e.g.\ Novikov) appear only as \textbf{[Spec]} and do not extend the proven bridge.
\end{remark}

\begin{remark}[Outside the bridge domain]\label{rk:outside-bridge}
Derived geometric/sheaf or symplectic/Floer realizations may be employed as \textbf{[Spec]} (Appendix~N/O), but the implication \(\mathrm{PH}_1\Rightarrow\Ext^1\) is not asserted in those targets. All proofs here remain in \(D^{\mathrm{b}}(k\text{-mod})\).
\end{remark}

\subsection*{3.1 bis. Window energy and spectral primitives on a fixed window}
\begin{definition}[Window barcode and residual length]
Fix a window \(W=[u,u')\). For \(i\ge 0\), let \(\mathcal{B}_i(F;W)\) be the multiset of intervals \(I\cap W\) obtained by restricting the barcode of \(\mathbf{P}_i(F)\) to \(W\) and discarding empty intersections. For \(J\in\mathcal{B}_i(F;W)\) write \(|J|\) for its length.
Given \(\tau\ge 0\), define the residual length of \(J\) at threshold \(\tau\) by \(|J|_\tau:=\max\{\,|J|-\tau,\,0\,\}\).
\end{definition}

\begin{definition}[Window energy and cumulative spectral content]
For degree \(i\) and parameters \((W,\tau)\), define the window energy
\[
E_i(F;W,\tau)\ :=\ \sum_{J\in\mathcal{B}_i(F;W)} |J|_\tau,
\]
and the cumulative tail counts
\[
C_{i,r}(F;W,\tau)\ :=\ \#\{\,J\in\mathcal{B}_i(F;W)\ :\ |J|_\tau\ge r\,\}\qquad(r\ge 0).
\]
We abbreviate \(E:=E_1\) and \(C_r:=C_{1,r}\) when \(i=1\).
\end{definition}

\begin{proposition}[Reduction-type monotonicity and layer-cake identity]\label{prop:energy-basic}
For each fixed \(F\) and \(W\):
\begin{enumerate}
\item \(E_i(F;W,\tau)\) is finite, piecewise-linear in \(\tau\), and nonincreasing in \(\tau\).
\item For all \(r\ge 0\), \(C_{i,r}(F;W,\tau)\) is finite, integer-valued, nonincreasing in both \(r\) and \(\tau\).
\item Layer-cake identity:
\[
E_i(F;W,\tau)\ =\ \int_0^{\infty} C_{i,r}(F;W,\tau)\,dr.
\]
\item Right-derivative in \(\tau\) exists for all but finitely many \(\tau\), with
\[
-\frac{\partial}{\partial\tau^+}E_i(F;W,\tau)\ =\ C_{i,0}(F;W,\tau).
\]
\end{enumerate}
All statements are functorial at the persistence layer and invariant under f.q.i.\ of filtered complexes.
\end{proposition}

\begin{remark}[Compatibility with collapse; nonexpansive continuation]
Let \(\mathbf{T}_\tau\) denote the threshold–collapse operator on persistence (Chapter~2).
Then for each \(i\),
\[
\sum_{J\in \mathcal{B}_i(C_\tau F;W)} |J|\ \le\ E_i(F;W,\tau),
\]
so \(E_i\) bounds the total barcode length after collapse. Moreover, the semigroup property \(C_{\tau+\varepsilon}\simeq C_\varepsilon\circ C_\tau\) implies the nonexpansive continuation
\[
E_i(F;W,\tau+\varepsilon)\ \le\ E_i(F;W,\tau),\qquad C_{i,r}(F;W,\tau+\varepsilon)\ \le\ C_{i,r}(F;W,\tau).
\]
\end{remark}

\begin{proposition}[Stability of window energy and tails]\label{prop:energy-stability}
Let \(d_{\mathrm{int}}\) be the interleaving distance on persistence modules.
Then for fixed \(W\) and \(\tau\), the maps \(F\mapsto E_i(F;W,\tau)\) and \(F\mapsto C_{i,r}(F;W,\tau)\) are stable under interleavings:
there exists a universal constant \(L\) (depending only on the window-complexity bound from Chapter~2) such that for any \(F,G\),
\[
\bigl|E_i(F;W,\tau)-E_i(G;W,\tau)\bigr|\ \le\ L\cdot d_{\mathrm{int}}\!\bigl(\mathbf{P}_i(F)|_W,\mathbf{P}_i(G)|_W\bigr),
\]
and \(C_{i,r}\) is upper semicontinuous in the same metric.
Proofs use bottleneck stability and the hinge structure of \(|J|_\tau\) (Appendix~H).
\end{proposition}

\subsection*{3.2. The Bridge \texorpdfstring{$\mathrm{PH}_1\!\Rightarrow\!\mathrm{Ext}^1$}{PH1⇒Ext1}}
Recall that \(\mathrm{PH}_1(F)=0\) means the degree-\(1\) persistence module vanishes; equivalently, \(H_1(F^tC_\bullet)=0\) for all \(t\).

\begin{theorem}[One-way bridge]\label{thm:PH1-to-Ext1}
Assume \textup{(B1)}–\textup{(B3)}. If \(\mathrm{PH}_1(F)=0\), then
\[
\Ext^1\!\big(\mathcal{R}(F),k\big)\ =\ 0.
\]
\end{theorem}

\noindent\emph{Equivalently}, under \textup{(B2)}–\textup{(B3)} the conclusion reads \(H^{-1}\!\big(\mathcal{R}(F)\big)=0\).

\begin{proof}
\(\mathrm{PH}_1(F)=0\) implies \(H_1(F^tC_\bullet)=0\) for all \(t\), hence \(\varinjlim_t H_1(F^tC_\bullet)=0\). By \textup{(B2)}, \(H^{-1}(\mathcal{R}(F))\cong \varinjlim_t H_1(F^tC_\bullet)=0\). By \textup{(B3)}, \(\Ext^1(\mathcal{R}(F),k)\cong \Hom(H^{-1}(\mathcal{R}(F)),k)=0\). All steps are natural in \(F\).
\end{proof}

\begin{corollary}[Robust bridge at scale \(\varepsilon\)]\label{cor:robust-bridge}
For any \(\varepsilon>0\), if \(\mathrm{PH}_1\!\big(C_\varepsilon(F)\big)=0\), then
\[
\Ext^1\!\big(\mathcal{R}(C_\varepsilon(F)),k\big)\ =\ 0.
\]
In particular, robust admissibility is always tested after truncation (apply \(C_\varepsilon\) first).
All identifications are natural in \(F\) and compatible with morphisms in \(\mathsf{FiltCh}(k)\).
\end{corollary}

\begin{proof}
Apply Theorem~\ref{thm:PH1-to-Ext1} to \(C_\varepsilon(F)\); the hypotheses are preserved by \(C_\varepsilon\) (Chapter~2). This uses that \(C_\varepsilon\) preserves constructibility and that, under the lifting–coherence hypothesis, the \(t\)-exact realization \(\mathcal{R}\) keeps amplitude \(\le 1\) (Appendix~B; cf.\ Chapter~2, §§2.2–2.3).
\end{proof}

\subsection*{3.3. Naturality, stability, and windowed gate usage}
\begin{proposition}[Naturality of the edge isomorphisms (B2)–(B3)]
Under \textup{(B1)}–\textup{(B3)}, the edge isomorphisms in \textup{(B2)} and \textup{(B3)} are natural in \(F\). For any morphism \(f:F\to G\) in \(\mathsf{FiltCh}(k)\) the diagram
\[
\begin{tikzcd}\n\rH^{-1}\!\big(\mathcal{R}(F)\big) \arrow[r,"\sim"] \arrow[d,"\rH^{-1}(\mathcal{R}(f))"'] &\n\varinjlim_t \,\rH_1(F^tC_\bullet) \arrow[d,"\varinjlim_t\, \rH_1(f^t)"] \\\n\rH^{-1}\!\big(\mathcal{R}(G)\big) \arrow[r,"\sim"] &\n\varinjlim_t \,\rH_1(G^tC_\bullet)\n\end{tikzcd}
\]
commutes, and the identifications in \textup{(B3)} are functorial in \(A\) and compatible with morphisms in \(D^{[-1,0]}\).
\end{proposition}

\begin{remark}[Stability via thresholded collapse and B-Gate\(^{+}\)]
By Chapter~2, \(\mathbf{T}_\tau\) is \(1\)-Lipschitz for the interleaving distance \(d_{\mathrm{int}}\) and compatible with filtered colimits at the persistence layer (up to f.q.i.; Appendix~B). Hence the premise \(\mathrm{PH}_1(C_\varepsilon(F))=0\) is metrically stable under interleaving perturbations, and the conclusion of Corollary~\ref{cor:robust-bridge} is invariant under functorial choices of \(C_\varepsilon\). On a window \((W,\tau)\), the quantitative primitives \(E_1\) and \(C_{1,r}\) supply monotone, piecewise-linear diagnostics (Proposition~\ref{prop:energy-basic}) that are stable (Proposition~\ref{prop:energy-stability}) and compatible with gate usage after collapse.
\end{remark}

\subsection*{3.4. Survival lemma (ε-survival) and safety margins}
We record the windowed \(\varepsilon\)-survival lemma used operationally; proofs and variants appear in Appendix~I.

\begin{lemma}[ε-survival]\label{lem:epsilon-survival}
Fix a window \(W=[u,u')\), a threshold \(\tau>0\), and a safety margin \(\mathrm{gap}_\tau>\Sigma\delta\) (the \(\delta\)-ledger total on \(W\)). Let \(F\) be constructible and let \(G\) be \(\varepsilon\)-interleaved with \(F\) on \(W\).
If there exists a bar \(J\in\mathcal{B}_1(F;W)\) with residual length \(|J|_\tau\ge \varepsilon+\mathrm{gap}_\tau\), then:
\begin{enumerate}
\item There is a corresponding bar \(J'\in\mathcal{B}_1(G;W)\) with \(|J'|_\tau\ge \mathrm{gap}_\tau\).
\item After collapse at \(\tau\), the image of \(J'\) in \(\mathrm{PH}_1(C_\tau G)\) survives B-side gating: it is not removed by tower artifacts and is tracked by the \((\mu_{\mathrm{Collapse}},\nu_{\mathrm{Collapse}})=(0,0)\) test.
\end{enumerate}
Equivalently, bars whose residual length exceeds \(\varepsilon+\Sigma\delta\) cannot be extinguished by an \(\varepsilon\)-interleaving and the recorded non-commutation on \(W\).
\end{lemma}

\begin{corollary}[Lower bounds via tails]\label{cor:tail-survival}
With the notation of Lemma~\ref{lem:epsilon-survival}, if \(C_{r}(F;W,\tau)>\!0\) for some \(r>\varepsilon+\Sigma\delta\), then \(C_{0}(G;W,\tau)>\!0\) and hence \(\mathrm{PH}_1(C_\tau G)\neq 0\).
\end{corollary}

\begin{remark}[Operational reading]
The \(\varepsilon\)-survival lemma quantifies \emph{robust presence} at the PH layer: any bar longer than \(\varepsilon+\Sigma\delta\) at residual scale \(\tau\) must show up after collapse and under \(\varepsilon\)-perturbations. This feeds directly into windowed diagnostics and prevents false positives when discharging the Ext-part via Theorem~\ref{thm:PH1-to-Ext1}.
\end{remark}

\subsection*{3.5. Gate indicators and quantitative linkage}
We make explicit how the gate indicators interact in the windowed policy.

\begin{definition}[Gate indicators on \((W,\tau)\)]
On a fixed \((W,\tau)\), the PH-indicator is \(\mathrm{PH}_1(C_\tau F)\) (vanishing/nonvanishing); the Ext-indicator is \(\Ext^1(\mathcal{R}(C_\tau F),k)\) in \(D^{\mathrm{b}}(k\text{-mod})\); the collapse indicators are \((\mu_{\mathrm{Collapse}},\nu_{\mathrm{Collapse}})\) (Appendix~D), and the spectral indicators are \(\{C_r(F;W,\tau)\}_{r\ge 0}\) and \(E(F;W,\tau)\).
\end{definition}

\begin{proposition}[One-way discharge and quantitative certificate]\label{prop:gate-certificate}
Assume \textup{(B1)}–\textup{(B3)}. Fix \((W,\tau)\) with safety margin \(\mathrm{gap}_\tau>\Sigma\delta\).
\begin{enumerate}
\item If \(\mathrm{PH}_1(C_\tau F)=0\) and \((\mu_{\mathrm{Collapse}},\nu_{\mathrm{Collapse}})=(0,0)\), then \(\Ext^1(\mathcal{R}(C_\tau F),k)=0\) and B-Gate\(^{+}\) passes on \((W,\tau)\).
\item If \(C_{r}(F;W,\tau)=0\) for some \(r>\varepsilon+\Sigma\delta\), then for every \(G\) that is \(\varepsilon\)-interleaved with \(F\) on \(W\), \(\mathrm{PH}_1(C_\tau G)=0\); hence the Ext-part discharges for \(G\) by Theorem~\ref{thm:PH1-to-Ext1}.
\item If \(E(F;W,\tau)\le \eta\) with \(\eta<\varepsilon+\Sigma\delta\), then necessarily \(C_r(F;W,\tau)=0\) for all \(r>\eta\); thus item \textup{(2)} applies.
\end{enumerate}
\end{proposition}

\begin{remark}[Quantitative but non-analytic]
The indicators \(E\) and \(C_r\) are purely topological/persistence-theoretic. We avoid any claim that topological collapse implies analytic regularization. Nonetheless, the monotonicity, piecewise-linearity, and stability of \(E\) and \(C_r\) offer effective quantitative tools for operational use within the proven bridge domain.
\end{remark}

\subsection*{3.6. Scope and limitations}
The bridge is strictly one-way: no claim is made that \(\Ext^1(\mathcal{R}(F),k)=0\) implies \(\mathrm{PH}_1(F)=0\).
Failure modes (e.g.\ invisible obstructions beyond the amplitude window) and counterexamples to the converse are documented in Appendix~D, including subsection \textbf{D.4} (Counterexamples to the converse). Type~IV/tower artifacts are detected by \((\mu_{\mathrm{Collapse}},\nu_{\mathrm{Collapse}})\); see Appendix~D.1–D.3 and Remark~\ref{rem:D-generic-dim} for the generic-fiber (multiplicity of \(I[0,\infty)\)) interpretation.

\subsection*{3.7. Windowed saturation and domain-restricted triggers \textup{[Spec]}}
The following \textbf{[Spec]} items record useful strengthening on fixed windows that may apply in restricted domains; they are not part of the proved bridge.

\begin{declaration}[Saturation gate: local equivalence \textup{[Spec]}]\label{dec:saturation-local}
On a saturation window (Chapter~11), where the event set stabilizes, the inter-window drift is \(\le\eta\), and the edge gap exceeds \(\eta\), it is \textbf{[Spec]}-admissible to adopt a temporary local equivalence:
\[
\mathrm{PH}_1(C_{\tau^\ast}F)=0\quad\Longleftrightarrow\quad \Ext^1\!\big(\mathcal{R}(C_{\tau^\ast}F),k\big)=0.
\]
This policy is window-local and must be explicitly logged; it does not extend the global one-way bridge outside the saturation conditions.
\end{declaration}

\begin{declaration}[Trigger pack (domain-restricted) \textup{[Spec]}]\label{dec:trigger-pack}
In restricted regimes (e.g.\ 2D flows, small-data critical spaces, simplified arithmetic models), one may prepare \emph{trigger lemmas} asserting that analytic deviations imply failure of B-Gate\(^{+}\) on the window, e.g.:
\[
\text{(blow-up/instability on \(W\))}\ \Rightarrow\ \bigl(\mathrm{PH}_1(C_\tau F)>0\ \ \text{or}\ \ \mu_{\mathrm{Collapse}}>0\ \ \text{or}\ \ \mathrm{aux\!-\!bars}>0\bigr).
\]
Such triggers (when available) supply partial necessity, improving the diagnostic power of the gate. They must be stated with explicit assumptions and logged as \textbf{[Spec]}.
\end{declaration}

\subsection*{3.8. Interaction with PF/BC, Mirror, and the \texorpdfstring{$\delta$}{delta}-ledger}
PF/BC isomorphisms are applied per \(t\), transported to the persistence layer, and then to the collapsed layer (Appendix~N).
Mirror/Transfer comparisons are performed only after \(C_\tau\), and any non-commutation with collapse is recorded in the \(\delta\)-ledger (Appendix~L).
The Ext-part of B-Gate\(^{+}\) is always checked on the collapsed object and only within the \(t\)-exact, amplitude \(\le 1\) scope. No \(\Ext\Rightarrow\mathrm{PH}\) claim is made.

\subsection*{3.9. Formalizability}
Hypotheses \textup{(B1)}–\textup{(B3)} and Theorem~\ref{thm:PH1-to-Ext1} admit direct formalization: \textup{(B2)} via an explicit two-term model for \(\mathcal{R}(F)\) and exactness of filtered colimits in \(\mathsf{Vect}_k\); \textup{(B3)} via truncation in \(D^{[-1,0]}\) and the edge of the long exact sequence. The primitives \(E\) and \(C_r\) are definable in the barcode semantics; Proposition~\ref{prop:energy-basic} and Proposition~\ref{prop:energy-stability} reduce to elementary combinatorics plus bottleneck stability (Appendix~H). Skeletons and lemma names are listed in Appendix~F for Coq/Lean integration. Windowed usage (B-Gate\(^{+}\)), the MECE policy, and the \(\delta\)-ledger are recorded as operational axioms in the formalization stubs (Appendix~F), with proofs confined to the persistence layer and the derived category of \(k\)-modules.

\subsection*{3.10. Degree specificity}
All energy/spectral statements in this chapter are used in degree \(1\); no claim is made for \((\mathrm{PH}_i \Rightarrow \Ext^i)\) with \(i\neq 1\).



% ===========================
% Chapter 4: Failure Lattice, Local PH–Ext Equivalence, Čech–Ext Gluing, and the Tower-Sensitivity Invariant μ_Collapse
% ===========================
\section{Chapter 4: Failure Lattice, Local PH–Ext Equivalence, Čech–Ext Gluing, and the Tower-Sensitivity Invariant \texorpdfstring{$\mu_{\mathrm{Collapse}}$}{mu\_Collapse}}
\addcontentsline{toc}{section}{Failure Lattice, Local PH–Ext Equivalence, Čech–Ext Gluing, and the Tower-Sensitivity Invariant $\mu_{\mathrm{Collapse}}$}

\paragraph{Standing hypotheses and scope.}
We work in the constructible (finite-type) range of persistence (Chapter~2, §2.1), adopt the bridging hypotheses \textup{(B1)–(B3)} from Chapter~3 with the minimal test family $\mathcal{Q}=\{k[0]\}$, and fix a $t$-exact realization $\mathcal{R}:\mathsf{FiltCh}(k)\to D^{\mathrm{b}}(k\text{-mod})$. All uses of filtered (co)limits are asserted \emph{at the persistence layer} (Appendix~A). Endpoint conventions and the treatment of infinite bars follow Chapter~2, Remark~\ref{rk:2-endpoints}. Monotonicity claims for indicators apply only to \emph{deletion-type} updates; inclusion-type updates are \emph{stability-only} (Appendix~E). Every claim is \emph{windowed} (Chapter~1, Def.~1.0; Chapter~2, §2.4), and all gate decisions are taken \emph{only} on the B-side after collapse (single layer $\mathbf{T}_\tau\mathbf{P}_i$). We \emph{do not} assert any global equivalence $\mathrm{PH}_1\Leftrightarrow \Ext^1$; the \emph{only} core bridge is the one-way implication $\mathrm{PH}_1\Rightarrow \Ext^1$ in $D^{\mathrm{b}}(k\text{-mod})$ under \textup{(B1)–(B3)} (Chapter~3 and Appendix~C).

\medskip
\noindent\emph{(B2) (edge identification, recall).}
There is a natural isomorphism $H^{-1}(\mathcal{R}(F))\cong \varinjlim_t H_1(F^tC_\bullet)$ and $\mathcal{R}(F)\in D^{[-1,0]}$; see Appendix~C.

\subsection*{4.1. Failure lattice and observable vs.\ invisible modes}
We organize collapse failures as follows (cf.\ Chapter~1, §1.4):

\begin{itemize}
  \item \textbf{Type I (Topological):} $\mathrm{PH}_1(F)\neq 0$.
  \item \textbf{Type II (Categorical):} $\Ext^1\!\big(\mathcal{R}(F),k\big)\neq 0$ (with $\mathcal{Q}=\{k[0]\}$).
  \item \textbf{Type III (Functorial/[Spec]):} admissibility unstable under a prescribed operation (e.g.\ a given pullback/filtered colimit) at finite level.
  \item \textbf{Type IV (Invisible/tower-level):} all finite layers appear admissible while the limit is not; detected by the tower-sensitivity invariants below.
\end{itemize}

Types~I–II are \emph{observable} on a single object; Type~III is \emph{specification-level}; Type~IV is \emph{invisible} at finite layers and requires tower diagnostics. We write $I[a,b)$ for the interval module (constructible, p.f.d.) supported on $[a,b)$; endpoint choices do not affect lengths (Chapter~2, Remark~\ref{rk:2-endpoints}). We emphasize again that \emph{no} global equivalence $\mathrm{PH}_1\Leftrightarrow \Ext^1$ is claimed; only the one-way core bridge $\mathrm{PH}_1\Rightarrow \Ext^1$ under \textup{(B1)–(B3)} (Chapter~3; Appendix~C).

\begin{remark}[Specification-level failures and functorial calculus]
Type~III is analyzed using Chapter~2, §2.3: non-expansiveness, shift–commutation (Lemma~\ref{lem:shift}), and the persistence-layer (co)limit/pullback compatibilities in Proposition~\ref{prop:stability}\,(4)(5). Statements at the filtered-complex layer are \emph{adopted as} \textbf{[Spec]} and used only up to filtered quasi-isomorphism (Appendix~B).
\end{remark}

\begin{remark}[Model towers]
Pure-kernel, pure-cokernel, and mixed toy towers, together with the vanishing regime under constructible filtered colimits, are illustrated in Appendix~D (D.1–D.3). Counterexamples to the converse $\Ext^1{=}0 \Rightarrow \mathrm{PH}_1{=}0$ are in D.4 (see also Appendix~C).
\end{remark}

\subsection*{4.2. The tower-sensitivity invariants \texorpdfstring{$\mu_{\mathrm{Collapse}}$}{mu\_Collapse}, \texorpdfstring{$\nu_{\mathrm{Collapse}}$}{nu\_Collapse}, and the Defect functor}
\emph{Generic fiber dimension.} As in Chapter~1, §1.4 and Appendix~D, Remark~\ref{rem:D-generic-dim}, for $M\in\Pers^{\mathrm{cons}}_k$ the \emph{generic fiber dimension} is $\mathrm{gdim}(M)=\lim_{t\to+\infty}\dim_k M(t)$; after truncation by $\mathbf{T}_\tau$, it equals the multiplicity of the infinite bar $I[0,\infty)$ in the barcode.

\medskip
\noindent\textbf{Comparison map.}
Fix $\tau>0$. Let $\{F_n\}_{n\in\mathbb{N}}$ be a directed system in $\mathsf{FiltCh}(k)$ with colimit $F_\infty$. For each degree $i$, the comparison map of truncated persistence modules is
\[
\phi_{i,\tau}:\ \varinjlim_{n}\ \mathbf{T}_\tau\!\big(\mathbf{P}_i(F_n)\big)\ \longrightarrow\ \mathbf{T}_\tau\!\big(\mathbf{P}_i(F_\infty)\big),
\]
where $\mathbf{T}_\tau$ is the exact reflector deleting all bars of length $\le\tau$ (Chapter~2, §2.2) and $\mathbf{P}_i$ is degreewise persistence.

\begin{definition}[Defect object and tower-sensitivity invariants]\label{def:defect}
For each $i$ and $\tau>0$, define the \emph{Defect objects} in $\Pers^{\mathrm{cons}}_k$ by
\[
\Defect_{i,\tau}^{\ker}(F_\bullet)\ :=\ \ker(\phi_{i,\tau}),\qquad \n\Defect_{i,\tau}^{\coker}(F_\bullet)\ :=\ \mathrm{coker}(\phi_{i,\tau}).
\]
The \emph{tower-sensitivity invariants} are the generic fiber dimensions
\[
\mu_{i,\tau}\ :=\ \mathrm{gdim}\big(\Defect_{i,\tau}^{\ker}\big),\qquad \n\nu_{i,\tau}\ :=\ \mathrm{gdim}\big(\Defect_{i,\tau}^{\coker}\big),\qquad \n\muc:=\sum_i \mu_{i,\tau},\ \ \nuc:=\sum_i \nu_{i,\tau}.
\]
Since complexes are bounded in homological degrees and barcodes are constructible, $\muc,\nuc<\infty$ on bounded $\tau$-windows.
\end{definition}

\begin{proposition}[Generic dimension equals infinite-bar multiplicity]\label{prop:generic-dimension-barcode}
In the constructible range, for any morphism $\psi:M\to N$ in $\Pers^{\mathrm{cons}}_k$ the generic fiber dimension
$\mathrm{gdim}\ker(\psi)$ (resp.\ $\mathrm{gdim}\mathrm{coker}(\psi)$) equals the multiplicity of $I[0,\infty)$ in the barcode of $\ker(\psi)$ (resp.\ $\mathrm{coker}(\psi)$). The same holds after truncation by $\mathbf{T}_\tau$.
\end{proposition}

\noindent\emph{Proof sketch.} Kernels and cokernels are computed pointwise and remain constructible (Chapter~2, Remark~\ref{rem:constructible-abelian}), hence their barcodes are defined. Interval Jordan–Hölder theory identifies the stabilized rank with the $I[0,\infty)$ multiplicity; truncation preserves $I[0,\infty)$ and removes only finite bars (Chapter~2, §2.2). See Appendix~D.

\begin{remark}[Functoriality, invariance, and calculus]
The maps $\phi_{i,\tau}$ are natural in the tower, independent of filtered representatives, and invariant under cofinal reindexing (Appendix~J). Consequently, $\Defect_{i,\tau}^{\ker/\coker}$ and $(\mu_{i,\tau},\nu_{i,\tau})$ are invariant under filtered quasi-isomorphisms. Subadditivity under composition, additivity under finite sums, and cofinal invariance are collected in Appendix~J.
\end{remark}

\begin{proposition}[Vanishing under constructible filtered colimits]\label{prop:mu-vanishing}
Assume the colimit of the tower $\{F_n\}$ is computed objectwise (chains/filtrations) and that all $\mathbf{P}_i(F_n)$ remain in $\Pers^{\mathrm{cons}}_k$. Then for every $\tau>0$ and every $i$, $\phi_{i,\tau}$ is an isomorphism, hence
\[
\Defect_{i,\tau}^{\ker}=\Defect_{i,\tau}^{\coker}=0,\qquad \muc=\nuc=0.
\]
\end{proposition}

\noindent A complete calculus (subadditivity, additivity, cofinal invariance) is given in Appendix~J.

\subsection*{4.3. Local PH–Ext equivalence on saturation windows: proof overview (core)}
We recall the window-local equivalence, formally stated in Chapter~2 (Theorem~\ref{thm:local-equiv}) and now proved within our guard-rails.

\begin{theorem}[Local PH–Ext equivalence on saturation windows]\label{thm:ch4-local-equiv}
Let $F\in\mathsf{FiltCh}(k)$, let $W=[u,u')$ be a right-open window, and fix $\tau>0$. Assume:
\begin{enumerate}
  \item \emph{(Amplitude)} $t$-exact realization $\mathcal{R}$ has amplitude $\le 1$: $\mathcal{R}(C_\tau F)\in D^{[-1,0]}(k\text{-mod})$.
  \item \emph{(Saturation/gap)} $W$ is a saturation window at $\tau$ (Chapter~2, Def.~2.6): event stability on $W$, length gap $\le \tau-\eta$ for some $\eta>0$, and near-$\tau$ non-accumulation.
  \item \emph{(Tail isomorphism)} The comparison map $\phi_{1,\tau}$ on $W$ is an isomorphism, i.e.\ $\mu_{\mathrm{Collapse}}=\nu_{\mathrm{Collapse}}=0$ on $W$ for degree $1$.
\end{enumerate}
Then, \emph{on the window $W$ and at threshold $\tau$},
\[
\mathrm{PH}_1\!\big(C_\tau F\big)=0\quad\Longleftrightarrow\quad \Ext^1\!\big(\mathcal{R}(C_\tau F),k\big)=0.
\]
\end{theorem}

\begin{proof}[Proof sketch]
Under (1) we may identify $\Ext^1(\mathcal{R}(C_\tau F),k)\cong \Hom(H^{-1}(\mathcal{R}(C_\tau F)),k)$ (Appendix~C). Under (2) and (3), the degree-$1$ persistence on $W$ is fully determined by stabilized bars (no near-$\tau$ creation/annihilation in the limit), and the tail isomorphism excludes Type~IV drift at scale $\tau$. Thus $\mathrm{PH}_1(C_\tau F)=0$ on $W$ iff $H^{-1}(\mathcal{R}(C_\tau F))=0$ on $W$, i.e.\ iff $\Ext^1(\mathcal{R}(C_\tau F),k)=0$.
\end{proof}

\begin{remark}[Core vs.\ global]
The theorem is \emph{window-local} and sits within the core (no [Spec] clauses). It \emph{does not} assert any global $\mathrm{PH}_1\Leftrightarrow \Ext^1$; outside saturation windows we only use $\mathrm{PH}_1\Rightarrow \Ext^1$ (Chapter~3).
\end{remark}

\subsection*{4.4. Čech–Ext\texorpdfstring{$^1$}{1} gluing and the Overlap Gate}
We integrate local-to-global gluing via the Overlap Gate (Chapter~1, Def.~1.0) and a Čech-type acyclicity on overlaps.

\begin{definition}[Čech nerve and local data after collapse]\label{def:cech}
Let $\{X_\alpha\}$ be a cover of the domain, and let $\{W_\alpha=[u_\alpha,u'_\alpha)\}$ be a right-open windowing. For a fixed degree $i$ and threshold $\tau>0$, set
\[
\mathcal{B}_{\alpha,i}\ :=\ \mathbf{T}_\tau\,\Crop_{W_\alpha}\big(\mathbf{P}_i(F|_{X_\alpha})\big)\ \in\ \Pers^{\mathrm{cons}}_k.
\]
Denote higher overlaps by $\mathcal{B}_{\alpha_0\cdots\alpha_p,i}$ via restriction to $X_{\alpha_0\cdots\alpha_p}\times (W_{\alpha_0}\cap\cdots\cap W_{\alpha_p})$. Write $N(\mathcal{U})$ for the Čech nerve of the cover.
\end{definition}

\begin{definition}[Čech–Ext\textsuperscript{1}–acyclicity (after collapse)]\label{def:cech-acyclic}
We say that the local collapsed data are \emph{Čech–Ext\textsuperscript{1}–acyclic} on degree $1$ if, for each $p\ge 0$ and each nonempty overlap $X_{\alpha_0\cdots\alpha_p}$,
\[
\Ext^1\!\Big(\mathcal{R}\!\big(C_\tau (F|_{X_{\alpha_0\cdots\alpha_p}})\big),\,k\Big)\ =\ 0,
\]
and the associated Čech differential on $p$-cochains takes values in $\Ext^1(-,k)$ that vanish on all overlaps. Equivalently, the $E_1$–page of the Čech spectral sequence has $\mathrm{Ext}^1$–row identically zero.
\end{definition}

\begin{theorem}[Overlap Gate with Čech–Ext\textsuperscript{1}: local-to-global gluing]\label{thm:cech-glue}
Fix a degree $i=1$, a globally right-open windowing, and a threshold $\tau>0$. Assume:
\begin{enumerate}
  \item \emph{(Local gates)} On each chart $X_\alpha\times W_\alpha$ the local B–Gate$^{+}$ passes: $\mathrm{PH}_1(C_\tau F|_{X_\alpha})=0$, $\Ext^1(\mathcal{R}(C_\tau F|_{X_\alpha}),k)=0$, $(\mu,\nu)=(0,0)$ on $W_\alpha$, with safety margin $\mathrm{gap}_\tau>\Sigma\delta_\alpha$.
  \item \emph{(Overlap Gate)} For any pair $(\alpha,\beta)$ with $X_{\alpha\beta}\times(W_\alpha\cap W_\beta)\neq\varnothing$, Overlap Gate (Chapter~1, Def.~1.0) passes: the collapsed restrictions agree up to the recorded budget, the safety margin dominates the budget on overlaps, and the tower diagnostics vanish $(\mu,\nu)=(0,0)$ on the overlap window.
  \item \emph{(Čech–Ext\textsuperscript{1}–acyclicity)} The local collapsed data are Čech–Ext\textsuperscript{1}–acyclic in degree $1$ (Definition~\ref{def:cech-acyclic}).
\end{enumerate}
Then the global B–Gate$^{+}$ passes on $\bigcup_\alpha X_\alpha\times W_\alpha$ at threshold $\tau$: in particular,
\[
\mathrm{PH}_1(C_\tau F)=0,\qquad \Ext^1(\mathcal{R}(C_\tau F),k)=0,\qquad (\mu,\nu)=(0,0),
\]
with a safety margin controlled by the minima of the local margins after subtracting the overlap budgets.
\end{theorem}

\begin{proof}[Proof sketch]
By (1) and (2), the objectwise collapsed local persistence modules glue along overlaps to a global collapsed persistence object in degree $1$ (Over\-lap Gate gives isomorphism on overlaps after collapse). Čech–Ext$^1$–acyclicity in (3) implies that, after applying the amplitude–$\le 1$ realization and restricting to degree $(-1)$–cohomology, the Čech complex has vanishing $H^1$; hence the global $\Ext^1(\mathcal{R}(C_\tau F),k)$ vanishes. The local $\mathrm{PH}_1$–vanishing and tail isomorphism on overlaps (by (2)) give global $\mathrm{PH}_1(C_\tau F)=0$ on the glued object. The tower diagnostics vanish globally because they vanish locally and on overlaps, and the Čech differentials preserve the collapsed layer; safety margins subtract the overlap budgets but stay positive by (1)–(2).
\end{proof}

\begin{remark}[Practical check]
In practice one verifies (3) by checking $\Ext^1(\mathcal{R}(C_\tau F|_{X_{\alpha_0\cdots\alpha_p}}),k)=0$ for $p=0,1$ (Mayer–Vietoris row) under an appropriate acyclicity regime (e.g.\ affine charts, tame restriction), and records the budget in the $\delta$–ledger. The after-collapse policy ensures that all checks occur on the B–side single layer.
\end{remark}

\subsection*{4.5. Type IV: finite admissibility need not pass to the limit}
The following precise form of the invisible failure principle works already at the persistence layer and then lifts.

\begin{proposition}[Type IV: finite-level admissibility may fail at the limit]\label{prop:type4}
In the filtered index category $\mathbb{N}\cup\{\infty\}$ with cone apex $\infty$, there exists a tower $\{F_n\}$ and $\tau>0$ such that
\[
\forall n:\ \ \mathrm{PH}_1\!\big(C_\tau(F_n)\big)=0\quad\text{and}\quad \Ext^1\!\big(\mathcal{R}(C_\tau(F_n)),k\big)=0,
\]
yet
\[
\mathrm{PH}_1\!\big(C_\tau(F_\infty)\big)\neq 0\quad\text{and thus}\quad \Ext^1\!\big(\mathcal{R}(C_\tau(F_\infty)),k\big)\neq 0.
\]
Moreover one can arrange $\muc=0$ and $\nuc>0$ (pure cokernel-type mismatch).
\end{proposition}

\begin{proof}[Proof (persistence model; then lift)]
Fix $\tau>0$ and set $\ell_n=\tau-\frac{1}{n}\uparrow\tau$. Let $M_n:=I[0,\ell_n)$ with structure maps $M_n\hookrightarrow M_{n+1}$. Then $\mathbf{T}_\tau(M_n)=0$ for all $n$, while $\varinjlim_n M_n\cong I[0,\tau)$. Cone the diagram to the apex module $N:=I[0,\infty)$ via $M_n\hookrightarrow I[0,\tau)\hookrightarrow N$, so the colimit of the extended diagram is $N$. Choose $F_n,F_\infty$ with $\mathbf{P}_1(F_n)\cong M_n$, $\mathbf{P}_1(F_\infty)\cong N$ (Chapter~2, §2.2). Then $\mathbf{T}_\tau(\mathbf{P}_1(F_n))=0$ while $\mathbf{T}_\tau(\mathbf{P}_1(F_\infty))\cong I[0,\infty)$. Thus $\mathrm{PH}_1(C_\tau(F_n))=0$ for all $n$, but $\mathrm{PH}_1(C_\tau(F_\infty))\neq 0$. By \textup{(B2)}–\textup{(B3)} this forces $\Ext^1\!\big(\mathcal{R}(C_\tau(F_\infty)),k\big)\neq 0$. Here $\ker(\phi_{1,\tau})=0$ and $\mathrm{coker}(\phi_{1,\tau})\neq 0$, so $\muc=0$ and $\nuc>0$.
\end{proof}

\begin{figure}[t]
\centering
\begin{tikzcd}[row sep=1.0em, column sep=2.0em]
I[0,\tau-\tfrac{1}{1}) \arrow[r, hook] \arrow[dr] &
I[0,\tau-\tfrac{1}{2}) \arrow[r, hook] 
& \cdots \arrow[r, hook] \arrow[dr] &
I[0,\tau-\tfrac{1}{n}) \arrow[r, hook] 
& \cdots \arrow[r, hook] \arrow[dr] &
I[0,\tau) \arrow[r, hook] \arrow[dr] &
I[0,\infty) \\\n& \mathbf{T}_\tau(\cdot)=0 & & \mathbf{T}_\tau(\cdot)=0 & &
\mathbf{T}_\tau(\cdot)=0 & \mathbf{T}_\tau(\cdot)=I[0,\infty)
\end{tikzcd}
\caption{Type~IV intuition (after truncation by $\mathbf{T}_\tau$): finite layers vanish, while the apex produces an infinite bar.}
\label{fig:typeIV-intuition}
\end{figure}

\subsection*{4.6. A natural refinement-limit example (pure cokernel type)}
\begin{example}[Resolution refinement producing a limit infinite bar]
Fix $\tau>0$. Consider a mesh-refinement sequence of scalar fields whose degree-$1$ persistence at level $n$ has a single bar $[0,\tau-\delta_n)$ with $\delta_n\downarrow 0$. Each finite layer satisfies $\mathbf{T}_\tau(\mathbf{P}_1(F_n))=0$. In the refinement limit the coherent structure becomes persistent across all thresholds, yielding an infinite bar in degree $1$: $\mathbf{T}_\tau(\mathbf{P}_1(F_\infty))\cong I[0,\infty)$. This is the natural analogue of Proposition~\ref{prop:type4} in resolution limits (see Appendix~D for diagrams). It realizes a \emph{pure cokernel} Type~IV failure ($\muc=0$, $\nuc>0$).
\end{example}

\begin{remark}[When invisible failure is excluded]
If the tower satisfies the hypotheses of Proposition~\ref{prop:mu-vanishing} (and Proposition~\ref{prop:stability}\,(4) in Chapter~2), then $\phi_{i,\tau}$ is an isomorphism for all $i,\tau$, so no Type~IV occurs: for every $\tau>0$,
\[
\bigl(\forall n:\ \mathrm{PH}_1(C_\tau(F_n))=0\bigr)\ \Longrightarrow\ \mathrm{PH}_1(C_\tau(F_\infty))=0,
\]
and, under \textup{(B1)–(B3)}, the corresponding $\Ext^1$-vanishing propagates along the tower.
\end{remark}

\subsection*{4.7. Restart lemma, summability, and pasting of windowed certificates}
We formalize the pasting principle used to obtain global conclusions from windowed certificates.

\begin{definition}[Per-window safety margin and pipeline budget]\label{def:window-budget}
Let $\{[u_k,u_{k+1})\}_k$ be a MECE partition of the monitored range (Chapter~2, Def.~\ref{def:ch2-mece}). On each window $W_k=[u_k,u_{k+1})$, fix a collapse threshold $\tau_k>0$ and define the \emph{per-window pipeline budget}
\[
\Sigma\delta_k(i)\ :=\ \sum_{j\in J_k} \bigl(\delta^{\mathrm{alg}}_{j}(i,\tau_k)+\delta^{\mathrm{disc}}_{j}(i,\tau_k)+\delta^{\mathrm{meas}}_{j}(i,\tau_k)\bigr),
\]
where $J_k$ indexes the A-side steps applied before the B-side gate on $W_k$. The \emph{per-window safety margin} $\mathrm{gap}_{\tau_k}>0$ is the admissible slack chosen for the gate on $W_k$ (Chapter~1, Def.~1.0).
\end{definition}

\begin{lemma}[Restart lemma (window-to-window inheritance)]\label{lem:restart}
Fix a degree $i$. Suppose that on $W_k$ the B-Gate$^{+}$ passes with safety margin $\mathrm{gap}_{\tau_k}>\Sigma\delta_k(i)$, and that the next window $W_{k+1}$ is reached via a finite number of admissible A-side steps, each either deletion-type or an $\varepsilon$-continuation, followed by collapse $C_{\tau_{k+1}}$ with $\tau_{k+1}$ chosen by an adaptation rule (Chapter~2, Def.~\ref{def:tau-adapt}). Then there exists $\kappa\in (0,1]$, depending only on the admissible step class and the adaptation policy, such that the safety margin on $W_{k+1}$ satisfies
\[
\mathrm{gap}_{\tau_{k+1}}\ \ge\ \kappa\ \bigl(\mathrm{gap}_{\tau_k}-\Sigma\delta_k(i)\bigr).
\]
In particular, if $\mathrm{gap}_{\tau_k}-\Sigma\delta_k(i)>0$, then $\mathrm{gap}_{\tau_{k+1}}>0$ and B-Gate$^{+}$ can pass on $W_{k+1}$ provided the new budget $\Sigma\delta_{k+1}(i)$ is sufficiently small.
\end{lemma}

\noindent\emph{Proof sketch.} Deletion-type steps are non-increasing after collapse (Chapter~2, Remark~\ref{rk:monotone-stable}); $\varepsilon$-continuations are $1$-Lipschitz hence introduce a controlled drift proportional to the declared $\varepsilon$. The adaptation of $\tau$ preserves the scale of the clipped indicators. Aggregating the drifts yields a multiplicative retention $\kappa$ of the effective margin.

\begin{definition}[Summability policy]\label{def:summability}
A run satisfies the \emph{summability policy} if the sequence of per-window budgets on a MECE partition obeys
\[
\sum_k \Sigma\delta_k(i)\ <\ \infty
\]
for the degrees $i$ on which gates are evaluated. A sufficient design pattern is a geometric decay of step sizes (e.g.\ $\tau_k=\tau_0\rho^k$, $\beta_k=\beta_0\rho^k$ with $\rho\in (0,1)$) and/or a geometric damping of the number and strength of $\varepsilon$-continuations per window, all recorded in the manifest (Appendix~G).
\end{definition}

\begin{theorem}[Pasting windowed certificates]\label{thm:pasting}
Let $\{[u_k,u_{k+1})\}_k$ be a MECE partition, and suppose that on each window $W_k$ the B-Gate$^{+}$ passes with $\mathrm{gap}_{\tau_k}>\Sigma\delta_k(i)$. If the summability policy holds (Definition~\ref{def:summability}) and the restart lemma (Lemma~\ref{lem:restart}) is applicable at each transition, then the concatenation of windowed certificates yields a global certificate on $\bigcup_k [u_k,u_{k+1})$, i.e.\ no failure occurs with no false negative on the union of windows for the monitored degrees $i$.
\end{theorem}

\noindent\emph{Proof sketch.} By Lemma~\ref{lem:restart}, positive safety margin propagates to the next window with a controlled retention factor. Summability of drifts ensures that the cumulative loss of margin remains bounded and does not exhaust the initial slack. The tower audit per window ensures $(\mu,\nu)=(0,0)$ at each scale and excludes Type~IV; the MECE coverage check (Chapter~2, Def.~\ref{def:ch2-mece}) guarantees the union covers the monitored range without gaps or double counting.

\subsection*{4.8. Stable bands and $\tau$-sweeps}
We record the $\tau$-selection policy used to stabilize the tower audit.

\begin{definition}[Stable band]\label{def:stable-band}
For a fixed window $W$ and degree $i$, a \emph{stable band} is a contiguous range $B\subset (0,\infty)$ such that for all $\tau\in B$ the comparison maps $\phi_{i,\tau}$ are isomorphisms and hence $(\mu_{i,\tau},\nu_{i,\tau})=(0,0)$. A \emph{$\tau$-sweep} is a discrete set $\{\tau_\ell\}$ used to probe $(\mu_{i,\tau_\ell},\nu_{i,\tau_\ell})$; a band is declared stable when the sweep detects $(0,0)$ on a consecutive subarray and the outcome persists under refinement of the sweep.
\end{definition}

\begin{remark}[Using stable bands]
Windowed certificates are issued within stable bands only. Outside a stable band, the collapse threshold or the pipeline must be redesigned, or the window must be refined. This stabilizes the tower audit across steps and prevents spurious Type~IV readings due to poor scale selection.
\end{remark}

\subsection*{4.9. Summary}
The failure lattice separates observable (Types~I–II), specification-level (Type~III), and invisible tower effects (Type~IV). The Defect objects $\Defect_{i,\tau}^{\ker/\coker}$ and the invariants $(\muc,\nuc)$ provide principled tower diagnostics: they \emph{vanish} under constructible filtered colimits (Proposition~\ref{prop:mu-vanishing}) and, when positive, certify instability of persistence under passage to limits. The \emph{only} core bridge is $\mathrm{PH}_1\Rightarrow \Ext^1$ in $D^{\mathrm{b}}(k\text{-mod})$; no global equivalence is asserted. Nevertheless, on \emph{saturation windows} with tail isomorphism, we obtain the local equivalence $\mathrm{PH}_1(C_\tau F)=0 \Leftrightarrow \Ext^1(\mathcal{R}(C_\tau F),k)=0$ (Theorem~\ref{thm:ch4-local-equiv}). At the level of gluing, the Overlap Gate combined with Čech–Ext$^1$–acyclicity yields robust local-to-global propagation of B–Gate$^{+}$ (Theorem~\ref{thm:cech-glue}). The Restart lemma and the Summability policy provide a reliable pasting principle (Theorem~\ref{thm:pasting}) for \emph{windowed certificates}, while stable bands (Definition~\ref{def:stable-band}) guide $\tau$-selection for tower audits. Counterexamples to converse statements (Ext$\Rightarrow$PH, finite-to-limit admissibility) appear in Appendix~C and Appendix~D (see especially Remark~\ref{rem:D-generic-dim} and D.4). These invariants and gluing statements are \emph{not} the classical Iwasawa $\mu,\lambda$–invariants nor their direct analogues; our usage is purely persistence–theoretic and audit–oriented on the B–side single layer after collapse.



% ===========================
% Chapter 5: Functoriality, Set-Theoretic Coherence, and Formalization Specifications (Proof/Spec)
% ===========================
\section{Chapter 5: Functoriality, Set-Theoretic Coherence, and Formalization Specifications (Proof/Spec)}
\addcontentsline{toc}{section}{Functoriality, Set-Theoretic Coherence, and Formalization Specifications (Proof/Spec)}

All adjunction and (co)limit statements in this chapter are made in the \emph{implementable range} and inside
$\mathrm{Ho}(\mathsf{FiltCh}(k))$, \emph{up to filtered quasi\hyp isomorphism} (Appendix~B). Equalities are asserted \emph{at the persistence layer}. We retain the standing conventions of Chapters~1–4: constructible range, field coefficients, $t$-exact realization, and the after\hyp truncation policy. Endpoint conventions and infinite bars are as in Chapter~2, Remark~\ref{rk:2-endpoints}. \emph{Monotonicity claims apply only to deletion\hyp type updates; inclusion\hyp type updates are stability\hyp only} (Appendix~E). All statements are \emph{windowed} and gates are evaluated \emph{after collapse} (B-side single layer).

\subsection*{5.1. Exactness, (Co)Limit Behavior, and a Right\hyp Adjoint Collapse in the Implementable Range (up to f.q.i.)}

Let $k$ be a field. Recall: $\mathsf{FiltCh}(k)$ is the category of finite\hyp type (constructible) filtered chain complexes; $\mathbf{P}_i$ is degreewise persistence; $\mathbf{T}_\tau$ is the exact truncation deleting all bars of length $\le\tau$ (Chapter~2, §2.2); $C_\tau$ is any filtered lift of $\mathbf{T}_\tau$ (Chapter~2, §§2.2–2.3; always \emph{up to f.q.i.}); and $\mathcal{R}:\mathsf{FiltCh}(k)\to D^{\mathrm{b}}(k\text{-mod})$ is $t$-exact. We also keep the minimal test family $\mathcal{Q}=\{k[0]\}$.

\paragraph{Persistence\hyp level (reflective) adjunction.}
Let $\mathsf{Pers}^{\mathrm{cons}}_k$ be the abelian category of constructible $k$-persistence modules and $\mathsf{Pers}^{\mathrm{cons}}_{k,\tau\text{-tf}}\subset \mathsf{Pers}^{\mathrm{cons}}_k$ the full subcategory of $\tau$-torsion-free objects (no composition factors of length $\le\tau$). As in Chapter~2, §§2.2–2.3:
\begin{itemize}
  \item $\mathsf{E}_\tau\subset \mathsf{Pers}^{\mathrm{cons}}_k$ (generated by interval modules of length $\le\tau$) is hereditary Serre (localizing).
  \item The reflector
  \[
  \mathbf{T}_\tau:\ \mathsf{Pers}^{\mathrm{cons}}_k\longrightarrow \mathsf{Pers}^{\mathrm{cons}}_{k,\tau\text{-tf}}
  \]
  is exact and exhibits a \emph{reflective} adjunction $\mathbf{T}_\tau\dashv \iota_\tau$ with the inclusion
  $\iota_\tau:\mathsf{Pers}^{\mathrm{cons}}_{k,\tau\text{-tf}}\hookrightarrow \mathsf{Pers}^{\mathrm{cons}}_k$.
  Consequently, $\mathbf{T}_\tau$ preserves finite limits and colimits and is $1$-Lipschitz for $d_{\mathrm{int}}$
  (Lemma~\ref{lem:shift}, Proposition~\ref{prop:stability}(1),(3)). As $\mathbf{T}_\tau$ is exact in this abelian setting, it preserves both finite limits and finite colimits.
\end{itemize}

\paragraph{Filtered\hyp complex level (operational coreflection; implementable range, up to f.q.i.).}
Define the full subcategory
\[
\mathsf{S}_\tau\ :=\ \Big\{\,F\in \mathsf{FiltCh}(k)\ \Big|\ \forall i,\ \mathbf{P}_i(F)\ \text{is $\tau$-torsion-free and }\ \Ext^1\!\big(\mathcal{R}(F),k\big)=0\,\Big\},
\]
and let $\mathsf{S}_\tau^{\mathrm{h}}$ be its image in $\mathrm{Ho}(\mathsf{FiltCh}(k))$.
For later comparison, recall the “trivial-at-$\tau$” gate
\[
\mathsf{Triv}_\tau\ :=\ \Big\{\,F\ \Big|\ \mathrm{PH}_1\big(C_\tau(F)\big)=0\ \ \text{and}\ \ \mathrm{Ext}^1\!\big(\mathcal{R}(F),k\big)=0\,\Big\},\qquad \mathsf{Triv}_\tau^{\mathrm{h}}\subset \mathsf{S}_\tau^{\mathrm{h}}.
\]

\begin{proposition}[Operational collapse as a right adjoint (implementable range; up to f.q.i.; Proof/Spec)]\label{prop:operational-coreflection}
Assume \textup{(B1)–(B3)} (Chapter~3) and the lifting–coherence hypothesis (Appendix~B).
Then there exists a functor
\[
\mathsf{C}_\tau^{\mathrm{comb}}:\ \mathrm{Ho}(\mathsf{FiltCh}(k))\longrightarrow \mathsf{S}_\tau^{\mathrm{h}}
\]
and a natural transformation $\eta:\mathrm{Id}\Rightarrow \iota\,\mathsf{C}_\tau^{\mathrm{comb}}$ (with $\iota:\mathsf{S}_\tau^{\mathrm{h}}\hookrightarrow \mathrm{Ho}(\mathsf{FiltCh}(k))$ the inclusion) such that, \emph{in this regime},
\begin{enumerate}
  \item \textup{(Adjunction)} $\mathsf{C}_\tau^{\mathrm{comb}}$ is right adjoint to $\iota$:
  \[
  \mathrm{Hom}\!\big(\iota(G),F\big)\ \cong\ \mathrm{Hom}\!\big(G,\mathsf{C}_\tau^{\mathrm{comb}}(F)\big)\qquad(G\in\mathsf{S}_\tau^{\mathrm{h}}).
  \]
  \item \textup{(Compatibility; persistence layer)} For each $i$, $\mathbf{P}_i\!\big(\mathsf{C}_\tau^{\mathrm{comb}}(F)\big)\cong \mathbf{T}_\tau\!\big(\mathbf{P}_i(F)\big)$ in $\Pers^{\mathrm{cons}}_k$.
  \item \textup{(Compatibility; realization layer)} $\mathcal{R}\!\big(\mathsf{C}_\tau^{\mathrm{comb}}(F)\big)\cong \tau_{\ge 0}\,\mathcal{R}(F)$ in $D^{\mathrm{b}}(k\text{-mod})$.
  \item \textup{(Soundness)} $\mathsf{C}_\tau^{\mathrm{comb}}(F)\in \mathsf{S}_\tau^{\mathrm{h}}$ for all $F$, and
  \[
  \mathsf{C}_\tau^{\mathrm{comb}}(F)\in \mathsf{Triv}_\tau^{\mathrm{h}}\quad\Longleftrightarrow\quad \mathbf{T}_\tau\!\big(\mathbf{P}_1(F)\big)=0\ \ \big(\text{i.e.\ } \mathrm{PH}_1(C_\tau(F))=0\big).
  \]
\end{enumerate}
\emph{All equalities above are asserted at the persistence layer, and all filtered-complex statements are in $\mathrm{Ho}(\mathsf{FiltCh}(k))$ up to f.q.i.\ only.}
\end{proposition}

\begin{proof}[Proof sketch]
At persistence level, use $\mathbf{T}_\tau$ (implemented by its lift $C_\tau$) so that $\mathbf{P}_i(C_\tau F)\cong \mathbf{T}_\tau(\mathbf{P}_i F)$. At realization level, use the coreflection $\tau_{\ge 0}$ to enforce amplitude $\le 1$, under which $\Ext^1(-,k)=0\iff H^{-1}(-)=0$ (Chapter~3). Lifting–coherence provides functorial comparison maps $\mathcal{R}\circ C_\tau \Rightarrow \tau_{\ge 0}\circ \mathcal{R}$ (up to f.q.i.), yielding $\mathsf{C}_\tau^{\mathrm{comb}}$ valued in $\mathsf{S}_\tau^{\mathrm{h}}$; triangle identities hold up to f.q.i.\ in $\mathrm{Ho}(\mathsf{FiltCh}(k))$.
\end{proof}

\begin{corollary}[Limit/(co)limit behavior of $\mathsf{C}_\tau^{\mathrm{comb}}$ (persistence layer)]
Under Proposition~\ref{prop:stability}(4),(5), $\mathsf{C}_\tau^{\mathrm{comb}}$ preserves finite limits
(in particular, finite pullbacks) \emph{up to f.q.i.} at the filtered\hyp complex level; at the
\emph{persistence layer} one has, for every degree $i$,
\[
\mathbf{P}_i\!\big(\mathsf{C}_\tau^{\mathrm{comb}}(\varinjlim_\Lambda F_\lambda)\big)\ \cong\ \varinjlim_\Lambda\, \mathbf{P}_i\!\big(\mathsf{C}_\tau^{\mathrm{comb}}(F_\lambda)\big).
\]
Hence $\mathsf{C}_\tau^{\mathrm{comb}}$ is $1$-Lipschitz at the persistence layer and inherits exactness via $\mathbf{T}_\tau$ (Chapter~2, §2.3). All equalities are stated at the persistence layer; no additional metric statement is made in $\mathrm{Ho}(\mathsf{FiltCh}(k))$ beyond up to f.q.i.\ compatibility.
\end{corollary}

\begin{remark}[Scope of colimit claims]
Since $\tau_{\ge 0}$ is a right adjoint, it need not commute with filtered colimits; all colimit statements are therefore restricted to the \emph{persistence layer} (via $\mathbf{T}_\tau$ and $\mathbf{P}_i$).
\end{remark}

\subsection*{5.1 bis. Idempotent monad/comonad: strictness at persistence, up to f.q.i.\ on $\mathrm{Ho}$}

\begin{proposition}[Idempotent monad at the persistence layer]\label{prop:monad}
Let $\mathbf{M}_\tau:=\iota_\tau\circ \mathbf{T}_\tau:\mathsf{Pers}^{\mathrm{cons}}_k\to \mathsf{Pers}^{\mathrm{cons}}_k$. With unit $\eta:\mathrm{Id}\Rightarrow \mathbf{M}_\tau$ and multiplication induced by the counit on $\mathsf{Pers}^{\mathrm{cons}}_{k,\tau\text{-tf}}$, $(\mathbf{M}_\tau,\eta,\mu)$ is an \emph{idempotent monad}. It is exact and $1$-Lipschitz (Appendix~A; see also Appendix~K).
\end{proposition}

\begin{proposition}[Idempotent comonad on $\mathrm{Ho}$ up to f.q.i.]\label{prop:comonad}
Let $\mathbf{G}_\tau:=\iota\circ \mathsf{C}_\tau^{\mathrm{comb}}:\mathrm{Ho}(\mathsf{FiltCh}(k))\to \mathrm{Ho}(\mathsf{FiltCh}(k))$, with counit $\varepsilon:\mathbf{G}_\tau\Rightarrow \mathrm{Id}$ and comultiplication $\delta$ induced by the unit of the adjunction in Proposition~\ref{prop:operational-coreflection}. Then $(\mathbf{G}_\tau,\varepsilon,\delta)$ is an \emph{idempotent comonad in $\mathrm{Ho}$ up to f.q.i.}; moreover $\mathbf{P}_i(\mathbf{G}_\tau F)\cong \mathbf{M}_\tau(\mathbf{P}_iF)$ naturally in $i,F$.
\end{proposition}

\begin{remark}[Strictness vs.\ implementability]
The monad is \emph{strict} in $\mathsf{Pers}^{\mathrm{cons}}_k$; the comonad is \emph{operational} in $\mathrm{Ho}$ up to f.q.i.\ only. This realizes “strict at persistence, up to f.q.i.\ after lifting”.
\end{remark}

\subsection*{5.2. \texorpdfstring{[Spec]}{[Spec]} Coq/Lean Contracts: Stability, (Co)Limits, Bridge, and $\delta$-Commutation}

\noindent Identifiers are indicative; concrete names may follow local conventions (e.g.\ Lean/mathlib namespaces). Appendix~F lists one naming scheme. All equalities are asserted at the persistence layer; formal objects at the filtered level are considered in $\mathrm{Ho}$ up to f.q.i.

\begin{specification}[Persistence truncation]\label{spec:pers-Ttau}
\begin{itemize}
  \item \texttt{pers\_Ttau\_exact}: exactness on short exact sequences.
  \item \texttt{pers\_Ttau\_lipschitz}: $d_{\mathrm{int}}(\mathbf{T}_\tau M,\mathbf{T}_\tau N)\le d_{\mathrm{int}}(M,N)$.
  \item \texttt{pers\_Ttau\_pres\_colim\_pullback}: filtered colimits and finite limits preserved (constructible range).
  \item \texttt{pers\_Ttau\_compose}: $\mathbf{T}_\tau\circ \mathbf{T}_\sigma=\mathbf{T}_{\max\{\tau,\sigma\}}$.
\end{itemize}
\end{specification}

\begin{specification}[Filtered\hyp complex level]\label{spec:filtered-level}
\begin{itemize}
  \item \texttt{Ctau\_lift}: $\mathbf{P}_i(C_\tau F)\cong \mathbf{T}_\tau(\mathbf{P}_i(F))$.
  \item \texttt{Ctau\_colim}: $\mathbf{P}_i(C_\tau(\varinjlim F_\lambda))\cong \varinjlim \mathbf{P}_i(C_\tau(F_\lambda))$.
  \item \texttt{Ctau\_pullback}: $\mathbf{P}_i(C_\tau(F\times_H G))\cong \mathbf{P}_i(C_\tau(F)\times_{C_\tau(H)}C_\tau(G))$.
\end{itemize}
\end{specification}

\begin{specification}[$\delta$-ledger and quantitative commutation]\label{spec:delta-ledger}
\begin{itemize}
  \item \texttt{delta\_2cell\_mirror\_collapse}: $\Mirror\circ C_\tau \Rightarrow C_\tau\circ \Mirror$ with uniform bound $\delta(i,\tau)$.
  \item \texttt{delta\_pipeline\_additivity}: pipeline bounds add.
  \item \texttt{delta\_lipschitz\_post}: $1$-Lipschitz post-processing does not increase bounds.
  \item \texttt{torsion\_nest\_commute}: nested reflectors commute and identify with the join.
  \item \texttt{soft\_commuting\_policy}: A/B test with tolerance $\eta$ and deterministic fallback, logging $\Delta_{\mathrm{comm}}$ to $\delta^{\mathrm{alg}}$.
\end{itemize}
\end{specification}

\begin{specification}[Bridge and admissibility]\label{spec:bridge}
\begin{itemize}
  \item \texttt{PH1\_to\_Ext1\_under\_B}: under (B1)–(B3), $\mathrm{PH}_1(F)=0\Rightarrow \Ext^1(\mathcal{R}(F),k)=0$.
  \item \texttt{admissible\_robust\_eps}: if $\mathrm{PH}_1(C_\varepsilon F)=0$, then $\Ext^1(\mathcal{R}(C_\varepsilon F),k)=0$.
\end{itemize}
\end{specification}

\begin{specification}[Tower diagnostics]\label{spec:mu-nu}
\begin{itemize}
  \item \texttt{mu\_def}: $\mu^{\,i}=\mathrm{gdim}\,\ker(\phi_{i,\tau})$, \ \texttt{nu\_def}: $\nu^{\,i}=\mathrm{gdim}\,\mathrm{coker}(\phi_{i,\tau})$.
  \item \texttt{mu\_nu\_finite}: finiteness (bounded degrees).
  \item \texttt{mu\_nu\_vanish}: under filtered colimits (constructible policy), $\phi_{i,\tau}$ is iso; hence $\muc=\nuc=0$.
\end{itemize}
\end{specification}

\begin{specification}[Combined collapse coreflection]\label{spec:combined}
\begin{itemize}
  \item \texttt{Ccomb\_adjunction}: inclusion $\iota:\mathsf{S}_\tau^{\mathrm{h}}\hookrightarrow\mathrm{Ho}(\mathsf{FiltCh}(k))$ has right adjoint $\mathsf{C}_\tau^{\mathrm{comb}}$ (implementable range).
  \item \texttt{Ccomb\_compat}: $\mathbf{P}_i(\mathsf{C}_\tau^{\mathrm{comb}}F)\cong \mathbf{T}_\tau(\mathbf{P}_iF)$ and $\mathcal{R}(\mathsf{C}_\tau^{\mathrm{comb}}F)\cong \tau_{\ge 0}\mathcal{R}(F)$.
  \item \texttt{Ccomb\_lipschitz\_pers}: $d_{\mathrm{int}}(\mathbf{P}_i(\mathsf{C}_\tau^{\mathrm{comb}}F),\mathbf{P}_i(\mathsf{C}_\tau^{\mathrm{comb}}G))\le d_{\mathrm{int}}(\mathbf{P}_i(F),\mathbf{P}_i(G))$.
\end{itemize}
\end{specification}

\subsection*{5.3. Minimal Foundations: ZFC and Dependent Type Theory}

\paragraph{ZFC assumptions (minimal).}
(S1)–(S5) as in the user’s draft hold; in particular, $\mathsf{Pers}^{\mathrm{cons}}_k$ is abelian and admits Serre localization, and $\mathsf{FiltCh}(k)$ supplies bounded, finite-type models.

\paragraph{Dependent type theory (Coq/Lean).}
(T1)–(T6) as in the user’s draft hold; in particular, a relative-category treatment of $\mathrm{Ho}(\mathsf{FiltCh}(k))$ supports right adjoints up to f.q.i.

\paragraph{Set-theoretic coherence.}
At persistence level, $\mathbf{T}_\tau\dashv \iota_\tau$ is a reflection; at realization level, $\tau_{\ge 0}$ is the right adjoint truncation. Proposition~\ref{prop:operational-coreflection} aggregates them into a right adjoint collapse in $\mathrm{Ho}$ (up to f.q.i.), consistent with stability and (co)limit behavior in Chapter~2.

\subsection*{5.4. Declarations for External Realizations and Operational Recipe (\textbf{[Spec]})}

\begin{declaration}[Spec–Derived realizations]\label{spec:derived-geo}
We may use $\mathcal{R}_{\mathrm{coh}}:\mathsf{FiltCh}(k)\to D^{\mathrm{b}}\mathrm{Coh}(X)$ or
$\mathcal{R}_{\acute{e}t}:\mathsf{FiltCh}(k)\to D^{\mathrm{b}}_{\mathrm{c}}(X_{\acute{e}t},\Lambda)$ with \emph{field} $\Lambda$ as specifications.
Projection formula and base change are invoked as in Appendix~N. The bridge $\mathrm{PH}_1\Rightarrow\Ext^1$ is proved only in $D^{\mathrm{b}}(k\text{-mod})$; external realizations do not extend the proven bridge.
\end{declaration}

\begin{declaration}[Spec–Operational recipe]\label{spec:operational-coreflection}
We operate with
\[
F\ \longmapsto\ \big(C_\tau(F)\ \text{at persistence}\big)\quad\text{and}\quad\big(\tau_{\ge 0}\mathcal{R}(F)\ \text{at realization}\big),
\]
using right-adjoint phrasing only at [Spec]; coherence and limits are in Appendix~B.
\end{declaration}

\subsection*{5.5. $\delta$-Budget Naturalities and the Pipeline Error Budget}

\begin{definition}[Natural $2$-cell and $\delta$-ledger]\label{def:delta-2cell}
For each $\tau>0$ and $i$, a natural $2$-cell $\epsilon_{i,\tau}:\Mirror\circ C_\tau\Rightarrow C_\tau\circ\Mirror$ carries a uniform bound $\delta(i,\tau)\ge 0$ in $d_{\mathrm{int}}$:
\[
d_{\mathrm{int}}\!\Big(\mathbf{T}_\tau \mathbf{P}_i(\Mirror(C_\tau F)),\ \mathbf{T}_\tau \mathbf{P}_i(C_\tau(\Mirror F))\Big)\ \le\ \delta(i,\tau),
\]
uniform in $F$. Decompose $\delta=\delta^{\mathrm{alg}}+\delta^{\mathrm{disc}}+\delta^{\mathrm{meas}}$ and record in a $\delta$-ledger.
\end{definition}

\begin{proposition}[Pipeline error budget]\label{prop:pipeline-budget}
Let $U_m,\dots,U_1$ be A-side steps with collapses $C_{\tau_j}$ and bounds $\delta_j(i,\tau_j)$. Then for fixed $\tau$,
\[
d_{\mathrm{int}}\!\Big(\mathbf{T}_\tau \mathbf{P}_i(\Mirror(C_{\tau_m}U_m\!\cdots C_{\tau_1}U_1F)),\ \mathbf{T}_\tau \mathbf{P}_i(C_{\tau_m}U_m\!\cdots C_{\tau_1}U_1\Mirror F)\Big)\ \le\ \sum_{j} \delta_j(i,\tau_j),
\]
and any $1$-Lipschitz post-processing does not increase the bound.
\end{proposition}

\begin{remark}[Safety margin]
Per window $W$ and $\tau$, B-Gate$^{+}$ uses $\mathrm{gap}_\tau$ and $\Sigma\delta(i)=\sum_j\delta_j(i,\tau_j)$; accept if $\mathrm{gap}_\tau>\Sigma\delta(i)$.
\end{remark}

\subsection*{5.6. Commutable Torsion: Adoption Policy, Tests, and Order Control}

\begin{definition}[Torsion reflectors and nesting]\label{def:torsion-reflectors}
Let $T_A,T_B$ be exact reflectors on $\mathsf{Pers}^{\mathrm{cons}}_k$ from hereditary Serre subcategories $E_A,E_B$. Say $T_A,T_B$ are \emph{nested} if $E_A\subseteq E_B$ or $E_B\subseteq E_A$.
\end{definition}

\begin{proposition}[Order independence under nesting]\label{prop:nested-commute}
If nested, then $T_A\circ T_B=T_B\circ T_A=T_{A\vee B}$, where $E_{A\vee B}$ is the Serre subcategory generated by $E_A\cup E_B$. In particular, for length thresholds, $\mathbf{T}_\tau\circ \mathbf{T}_\sigma=\mathbf{T}_{\max\{\tau,\sigma\}}$.
\end{proposition}

\begin{definition}[A/B commutativity test and soft\hyp commuting]\label{def:soft-commute}
Define $\Delta_{\mathrm{comm}}(M;A,B)=d_{\mathrm{int}}(T_AT_BM,T_BT_AM)$. If $\Delta_{\mathrm{comm}}\le\eta$ (tolerance), accept soft\hyp commuting; else fix an order and log $\Delta_{\mathrm{comm}}$ into $\delta^{\mathrm{alg}}$.
\end{definition}

\begin{remark}[Multi\hyp axis torsion and scope]
When $E_A,E_B$ are not nested, no general commutation claim holds; use A/B test and deterministic fallback with explicit accounting.
\end{remark}

\subsection*{5.7. Worked micro\hyp example (policy illustration)}
Let $T^{\mathrm{len}}_\tau$ be length threshold and $T^{\mathrm{birth}}_{[u,u')}$ birth-window deletion. Measure $\Delta_{\mathrm{comm}}(M;\mathrm{len},\mathrm{birth})$. If $\le\eta$, adopt soft\hyp commuting; otherwise fix an order (e.g.\ $T^{\mathrm{birth}}$ then $T^{\mathrm{len}}$) and record $\Delta_{\mathrm{comm}}$ in $\delta^{\mathrm{alg}}$.

\subsection*{5.8. Overlap Gate (Functorial Gluing): collapse compatibility, soft commuting, Čech–Ext\textsuperscript{1}, stable bands}

We formalize a \emph{functorial} Overlap Gate that lifts the operational Overlap Gate (Chapter~1) to a typed, gluing-ready interface.

\begin{definition}[Window Stack (WinFib) and Čech nerve]\label{def:winfib}
Let $\mathsf{Win}$ be the category of pairs $(\alpha,W_\alpha)$ with $W_\alpha$ right-open, morphisms induced by inclusions $X_\alpha\cap W_\alpha\hookrightarrow X_\beta\cap W_\beta$. For fixed degree $i$ and $\tau>0$, define a pseudo\-functor
\[
\mathcal{S}_{i,\tau}(-):\ \mathsf{Win}^{\mathrm{op}}\longrightarrow \mathrm{Cat},\qquad (\alpha,W_\alpha)\ \longmapsto\ \Big\{\mathbf{T}_\tau\,\Crop_{W_\alpha}\big(\mathbf{P}_i(F|_{X_\alpha})\big)\Big\}\subset\mathsf{Pers}^{\mathrm{cons}}_k.
\]
Its Grothendieck fibration $\pi:\int\mathcal{S}_{i,\tau}(-)\to \mathsf{Win}$ is the \emph{Window Stack (WinFib)}. Let $N(\mathcal{U})$ be the Čech nerve of the domain cover $\{X_\alpha\}$.
\end{definition}

\begin{definition}[Overlap Gate \texorpdfstring{$\mathrm{OG}^{\mathrm{funct}}$}{OG^funct}]\label{def:og-funct}
Fix $(i,\tau)$ and a windowed cover $\{X_\alpha,W_\alpha\}$. We say $\mathrm{OG}^{\mathrm{funct}}(i,\tau)$ \emph{passes} if:
\begin{enumerate}
  \item \textbf{Collapse compatibility} (after-collapse $1$-Lipschitz defect): for all overlaps, the objects in $\int\mathcal{S}_{i,\tau}(-)$ agree up to the declared $\delta$-budget, and the safety margin dominates the budget on overlaps.
  \item \textbf{Soft commuting (A/B)}: any pair of non-nested reflectors used on overlaps passes the A/B test with tolerance $\eta$; otherwise a deterministic order is fixed and $\Delta_{\mathrm{comm}}$ is accounted for in $\delta^{\mathrm{alg}}$.
  \item \textbf{Čech–Ext\textsuperscript{1}–acyclicity} (degree $1$): $\Ext^1(\mathcal{R}(C_\tau F|_{X_{\alpha_0\cdots\alpha_p}}),k)=0$ for $p=0,1$ on overlaps, and the Čech differential in $\Ext^1$ vanishes.
  \item \textbf{Stable band \& no-accumulation} (tower diagnostics): on each window and overlap, the tail comparison is an isomorphism $(\mu,\nu)=(0,0)$ on a stability band of $\tau$’s, with near-$\tau$ non-accumulation.
\end{enumerate}
\end{definition}

\begin{theorem}[Functorial gluing via \texorpdfstring{$\mathrm{OG}^{\mathrm{funct}}$}{OG^funct}]\label{thm:og-funct}
If $\mathrm{OG}^{\mathrm{funct}}(i,\tau)$ passes for degree $i=1$ on a windowed cover, then:
\begin{enumerate}
  \item (Existence) The local collapsed objects glue to a global object in $\mathsf{Pers}^{\mathrm{cons}}_k$ (persistence layer) that is unique up to isomorphism on windows.
  \item (Gate propagation) The global B–Gate$^{+}$ passes on the union $\bigcup_\alpha X_\alpha\times W_\alpha$: specifically, $\mathrm{PH}_1(C_\tau F)=0$, $\Ext^1(\mathcal{R}(C_\tau F),k)=0$, $(\mu,\nu)=(0,0)$.
  \item (Budget control) The global safety margin is bounded below by the minimum of local margins minus the overlap budgets (including A/B residuals), and all commutation defects are accounted for in the $\delta$-ledger.
\end{enumerate}
\end{theorem}

\begin{proof}[Proof sketch]
(1) follows from (1)–(2) in Definition~\ref{def:og-funct} and the $1$-Lipschitz collapse at persistence. (2) follows by applying the window-local PH–Ext equivalence (Chapter~4, Theorem~4.3) on charts and overlaps, plus Čech–Ext$^1$–acyclicity on degree $1$ to glue vanishing $\Ext^1$ globally; tower diagnostics vanish globally by stability bands and no-accumulation. (3) is inherited from the overlap budgets and the additivity of the $\delta$-ledger.
\end{proof}

\begin{remark}[IMRN/AiM readiness]
The acceptance criteria (1)–(4) are checkable on the Čech nerve of the windowed cover, wholly \emph{after collapse} on the B-side single layer. All budgets and tolerances are recorded per window; proofs use only exactness/Lipschitzness at persistence and amplitude $\le 1$ at realization.
\end{remark}

\subsection*{5.9. Window Stack (WinFib): typed acceptance on the nerve and auditability}

\begin{definition}[Typed acceptance predicate on the nerve]\label{def:typed-accept}
For each simplex $\sigma=\{\alpha_0,\dots,\alpha_p\}$ in $N(\mathcal{U})$ and a window $W_\sigma=\bigcap_j W_{\alpha_j}$, define a \emph{typed acceptance} datum
\[
\mathbf{Acc}(\sigma; i,\tau)\ :=\ \Big(\mathrm{iso\_after\_collapse},\ \mathrm{AB\_soft\_commute},\ \mathrm{Cech\_Ext^1\_zero},\ \mathrm{stable\_band\_ok}\Big),
\]
with booleans and budgets. We say the nerve passes if $\mathbf{Acc}(\sigma;i,\tau)$ holds for all $\sigma$ up to dimension $1$ (edges) and dimension $0$ (vertices), and the higher-dimensional diagonals contain no additional constraints beyond Čech\textsuperscript{1}-acyclicity.
\end{definition}

\begin{proposition}[Nerve acceptance implies \texorpdfstring{$\mathrm{OG}^{\mathrm{funct}}$}{OG^funct}]\label{prop:nerve-accept}
If $\mathbf{Acc}(\sigma;1,\tau)$ holds for all $\sigma$ up to edges and vertices, then $\mathrm{OG}^{\mathrm{funct}}(1,\tau)$ passes. Consequently Theorem~\ref{thm:og-funct} applies.
\end{proposition}

\begin{remark}[Machine-checkable audit]
The quadruple $\mathbf{Acc}$ is a minimal, machine-checkable record per nerve simplex; together with the $\delta$-ledger and safety margins, it yields a complete audit trail for local-to-global collapse decisions. A Lean/Coq stub can represent $\mathbf{Acc}$ as a structure with fields and proofs (Appendix~F).
\end{remark}

\subsection*{5.10. Summary}
We established a coherent functorial core for collapse within the constructible regime: a persistence\hyp level exact reflector $\mathbf{T}_\tau$, an operational right adjoint collapse in $\mathrm{Ho}$ (up to f.q.i.), and formal contracts for stability and bridge usage. The \emph{$\delta$-budget} is made \emph{natural} via $2$-cells measuring the non\hyp commutation of Mirror/Transfer with collapse, with \emph{additive} pipeline accounting and $1$-Lipschitz post\hyp processing. For torsion reflectors, \emph{order independence} is guaranteed under \emph{nesting}, while non\hyp nested cases are governed by an A/B \emph{soft\hyp commuting} policy and a deterministic fallback with explicit $\delta^{\mathrm{alg}}$ logging. We formalized a \emph{functorial Overlap Gate} packaging collapse compatibility, soft commuting, Čech–Ext$^{1}$–acyclicity, and stability bands; together with the \emph{Window Stack (WinFib)}, this furnishes a typed, nerve-level acceptance test that ensures local-to-global gluing after collapse with fully logged budgets. All gate decisions remain on the B\hyp side after collapse, within a reproducible, windowed, and metrically stable framework that integrates seamlessly with the tower diagnostics of Chapter~4 and the realization bridge of Chapter~3.


% ===========================
% Chapter 6: Geometric Collapse (Program/Spec) — reinforced and arithmetically integrated
% ===========================
\section{Chapter 6: Geometric Collapse (Program/Spec)}
\addcontentsline{toc}{section}{Geometric Collapse (Program/Spec)}

\noindent\textbf{Monotonicity policy (after truncation).} Deletion\hyp type updates are \emph{non\hyp increasing} for windowed persistence energies and spectral indicators; inclusion\hyp type updates are \emph{stability\hyp only} (non\hyp expansive). See Appendix~E for sufficient conditions and counterexamples.

\subsection*{6.0. Standing hypotheses and admissible geometric realization}
We work over a fixed field \(k\) and adopt the notation and hypotheses of Part~I. In particular,
\(\FiltCh{k}\) denotes finite\hyp type filtered chain complexes over \(k\),
\(\mathbf{P}_i:\FiltCh{k}\to\Perskft\) the degreewise persistence functor,
and we write \(\Ttau:=\mathbf{T}_\tau\) for the bar\hyp deletion (Serre) localization at scale \(\tau\ge 0\) (allowing \(\Ttau=\mathrm{Id}\) when \(\tau=0\)).
Its filtered lift \(\Ctau\) is used \emph{up to filtered quasi\hyp isomorphism} (Chapter~2, §§2.2–2.3).
The realization \(\Rfun:\FiltCh{k}\to D^{\mathrm{b}}(k\text{-mod})\) is \(t\)\hyp exact.
All statements in this chapter lie in the constructible range (we identify \(\Perskft\) with the constructible subcategory).
Unless explicitly marked \textbf{[Spec]}, \emph{equalities and Lipschitz claims are asserted only at the persistence layer}; identities at the filtered\hyp complex layer hold \emph{up to filtered quasi\hyp isomorphism}.
Kernel/cokernel diagnostics \((\muc,\nuc)\) are computed from the comparison maps
\[
  \phi_{i,\tau}:\ \varinjlim\nolimits_\lambda \Ttau\!\big(\mathbf{P}_i(F_\lambda)\big)\ \longrightarrow\ \Ttau\!\big(\mathbf{P}_i(\varinjlim\nolimits_\lambda F_\lambda)\big),
\]
with \(\dim_k\) interpreted as the \emph{generic\hyp fiber} dimension after truncation (multiplicity of \(I[0,\infty)\)); see Appendix~D, Remark~\ref{rem:D-generic-dim}.

\begin{definition}[Admissible geometric realization]\label{def:geom-real}
Let \(\mathsf{Geom}\) be a geometric input category (e.g.\ metric or metric\hyp measure spaces with \(1\)\hyp Lipschitz maps; triangulated manifolds with mesh\hyp refinement maps; weighted graphs with contraction/sparsification maps). An \emph{admissible geometric realization} is a functor
\[
  \mathcal{G}:\ \mathsf{Geom}\longrightarrow \FiltCh{k}
\]
such that:
\emph{(i)} \(\mathcal{G}\) is functorial and sends non\hyp expansive maps to filtered chain maps whose images under each \(\mathbf{P}_i\) are \(1\)\hyp Lipschitz for the interleaving distance;
\emph{(ii)} degreewise finite\hyp type is preserved;
\emph{(iii)} subsampling/refinement maps are carried to filtered maps that, for each fixed \(\tau\), induce filtered quasi\hyp isomorphisms after applying \(\Ctau\).
\end{definition}

\begin{remark}[Program posture and bridges]\label{rk:LC}
All specifications are asserted within the \emph{implementable range} of Part~I: (co)limit and stability statements are restricted to the persistence layer; the lifting–coherence hypothesis \LC\ is assumed for comparing \(\Ctau\) on \(\FiltCh{k}\) with effects after realization \(\Rfun\).
No equivalence \(\mathrm{PH}_1\Leftrightarrow\Ext^1\) is claimed; only the one\hyp way bridge under \textup{(B1)–(B3)} from Part~I is used.
The obstruction \(\muc\) is \emph{distinct} from the classical Iwasawa \(\mu\)\hyp invariant.
\end{remark}

\begin{remark}[Stability vs.\ monotonicity; spectral policy]\label{rem:stability-vs-monotonicity}
Non\hyp expansive maps ensure stability (non\hyp expansiveness) of all indicators.
Under \emph{deletion\hyp type} updates satisfying Appendix~E (Dirichlet restriction, principal submatrices/Schur complements, Loewner contractions, and—in the symplectic setting—stop additions/Liouville contractions), spectral tails and windowed energies are \emph{non\hyp increasing}. Inclusion\hyp type updates guarantee only \emph{stability}.
Spectral indicators are \emph{not} f.q.i.\ invariants; throughout we treat them as \emph{stable under a fixed normalization policy} and evaluate them on \(L(\Ctau F)\) (see Chapter~11).
\end{remark}

All monotonicity claims are interpreted after truncation by \Ttau.

\subsection*{6.1. Monitored indicators and energies}
Fix an admissible \(\mathcal{G}\) and write \(F=\mathcal{G}(X)\in \FiltCh{k}\).

\begin{definition}[Persistence energies]\label{def:PE}
Let \(\mathcal{B}_i(F)\) be the multiset of intervals of \(\mathbf{P}_i(F)\).
For \(\alpha>0\) (default \(\alpha=1\)) and a truncation window \([0,\tau]\), define
\begin{equation}\label{eq:PE-alpha-trunc}
\begin{aligned}
\mathrm{PE}_{i,\alpha}^{\le\tau}(F)
&:= \sum_{[b,d)\in \mathcal{B}_i(F)}
   \bigl(\min\{d,\tau\}-\min\{b,\tau\}\bigr)_{+}^{\alpha},\\\n(x)_{+}&:=\max\{x,0\}.
\end{aligned}
\end{equation}
By default \(\alpha=1\), and \(\mathrm{PE}^{\le\tau}(F):=\sum_i \mathrm{PE}_{i}^{\le\tau}(F)\).
\emph{All energies are evaluated on the truncated barcode: \(\mathrm{PE}_{i,\alpha}^{\le\tau}(F)=\mathrm{PE}_{i,\alpha}^{\le\tau}(\Ctau F)\) with \(\Ttau\mathbf{P}_i(F)=\mathbf{P}_i(\Ctau F)\).}
\end{definition}

\begin{definition}[Spectral indicators]\label{def:spectral}
Let \(L(\Ctau F)\) be a combinatorial Hodge Laplacian on the truncated complex \(\Ctau F\) (normalized, with the Euclidean inner product on chains). Denote the non\hyp decreasing spectrum by \((\lambda_m(\Ctau F))_{m\ge 0}\). For \(\beta>0\) and an \emph{integer} cutoff \(M(\tau)\in\mathbb{N}\), define the spectral tail
\[
  \mathrm{ST}_{\beta}^{\ge M(\tau)}(F)\ :=\ \sum_{m\ge M(\tau)} \lambda_m(\Ctau F)^{-\beta},\qquad\n  \mathrm{HT}(t;F)\ :=\ \mathrm{Tr}\big(e^{-tL(\Ctau F)}\big)\ \ (t>0),
\]
with zero modes excluded (or replaced by the Moore–Penrose pseudoinverse). Qualitative specifications are invariant under these standard choices; the policy \((\beta,M(\tau),t)\) is fixed across a run (Appendix~G; Chapter~11).
\end{definition}

\begin{remark}[Convergence, parameterization, and logging]\label{rk:ST-conv}
Choose \(\beta\) and \(M(\tau)\) to ensure convergence (typical \(\beta\in\{1,2\}\), \(M(\tau)=\lfloor c\,\tau^{\gamma}\rfloor\) with \(c>0\), \(\gamma\in(0,2]\)).
When sweeping \(\tau\), take \(M(\tau)\) non\hyp decreasing to avoid artificial discontinuities.
Normalization, zero\hyp mode handling, and the window policy are fixed and logged with \((\beta,M(\tau),t)\).
\end{remark}

\begin{definition}[\(\Ext^1\)\hyp collapse at scale]\label{def:ext-collapse}
Writing \(\Rfun(F)\in D^{\mathrm{b}}(k\text{-mod})\), we say \emph{\(\Ext^1\)\hyp collapse holds at scale \(\tau\)} if, for all \(Q\in\Qtest:=\{k[0]\}\),
\[
  \Ext^1\!\big(\Rfun(\Ctau F),\, Q\big)\ =\ 0.
\]
\end{definition}

\subsection*{6.2. Stability under filtered colimits (geometry level)}
Let \(L_i(\Ctau F)\) denote the normalized combinatorial Hodge Laplacian in degree \(i\) on \(\Ctau F\), with nondecreasing positive spectrum \(\bigl(\lambda_{i,m}(\Ctau F)\bigr)_{m\ge 0}\). For brevity we suppress \(i\) and write \(L(\Ctau F)\), \(\mathrm{ST}_{\beta}^{\ge M(\tau)}(F)\), \(\mathrm{HT}(t;F)\) when the degree is clear from context.
\begin{declaration}[Specification: Stability under filtered colimits in geometry]\label{spec:geom-colim}
Assume a filtered diagram \(\{F_\lambda\}\) in \(\FiltCh{k}\) remains degreewise finite\hyp type; filtered (co)limits are computed objectwise in \([\mathbb{R},\mathsf{Vect}_k]\) and used only under the scope policy of Appendix~A (compute in the functor category and verify return to \(\Pers^{\mathrm{cons}}_k\)).
Then, for each fixed \(\tau\), the induced maps
\[
  \phi_{i,\tau}:\ \varinjlim\nolimits_\lambda \Ttau\!\big(\mathbf{P}_i(F_\lambda)\big)\ \xrightarrow{\ \cong\ }\ \Ttau\!\big(\mathbf{P}_i(\varinjlim\nolimits_\lambda F_\lambda)\big)
\]
are isomorphisms; hence \(\muc=\nuc=0\) at that scale. The conclusion holds pointwise along any discrete \(\tau\)\hyp sweep.
\end{declaration}

When aggregating across degrees, the degree set and aggregation policy (per-degree vs. summed) are fixed and logged.

\begin{remark}[Endpoints and infinite bars]\label{rk:endpoints-ch6}
Endpoint conventions (open/closed) and the treatment of infinite bars are as in Chapter~2, Remark~\ref{rk:2-endpoints}; \(\Ttau\) deletes only finite bars of length \(\le\tau\).
\end{remark}

\subsection*{6.3. Joint monitoring and programmatic guarantees}
\begin{declaration}[Specification: Geometric collapse indicators]\label{spec:geom-indicators}
Under \LC\ and within the implementable range, along geometric degenerations \emph{compute and record}:
\begin{enumerate}
  \item \(\Ttau\mathbf{P}_i(F)\) and the truncated energies \(\mathrm{PE}_i^{\le\tau}\) on \(\Ttau\mathbf{P}_i(F)=\mathbf{P}_i(\Ctau F)\);
  \item spectral indicators \(\mathrm{ST}_{\beta}^{\ge M(\tau)}\) or \(\mathrm{HT}(t;\cdot)\) on \(L(\Ctau F)\) (parameters as in Remark~\ref{rk:ST-conv});
  \item the \(\Ext^1\)\hyp check \(\Ext^1\big(\Rfun(\Ctau F),Q\big)=0\) for \(Q\in\Qtest\).
\end{enumerate}
The \emph{stable regime} is declared where \((\muc,\nuc)=(0,0)\) and (1)–(3) hold jointly.
\end{declaration}

\begin{remark}[Saturation gate (reference; see Chapter~11)]\label{rk:sat-gate-ch6}
We follow the Chapter~11 policy for a window \([0,\tau^\ast]\): \emph{(i)} eventually the maximal finite bar length in \(\mathbf{T}_{\tau^\ast}\mathbf{P}_i(F_t)\) is \(\le \eta\); \emph{(ii)} eventually \(d_{\mathrm{int}}\!\big(\mathbf{T}_{\tau^\ast}\mathbf{P}_i(F_t),\mathbf{T}_{\tau^\ast}\mathbf{P}_i(F_{t'})\big)\le \eta\); \emph{(iii)} the edge gap \(\delta:=\tau^\ast-\max\{b_r<\tau^\ast\}\) satisfies \(\delta>\eta\).
This chapter \emph{uses the gate only as a reference}; the quantitative policy and its verification are centralized in Chapter~11.
\end{remark}

\begin{conjecture}[Geometry\texorpdfstring{$\to$}{->}AK collapse propagation]\label{conj:geom-prop}
If items \emph{(1)}–\emph{(3)} of Declaration~\ref{spec:geom-indicators} hold with \((\muc,\nuc)=(0,0)\) along a non\hyp expansive degeneration over a \(\tau\)\hyp interval, then
\[
  \text{geometric collapse}\ \Longrightarrow\ \text{persistence\ energy\ decay}\ \Longrightarrow\ \text{spectral decay}\ \Longrightarrow\ \text{\(\Ext\)\hyp collapse at scale},
\]
compatibly with \LC.
\end{conjecture}

\subsection*{6.4. Scope, design patterns, and diagrams}
\begin{declaration}[Specification: Scope of admissible degenerations]\label{spec:scope}
The program encompasses: \emph{(a)} metric(\hyp measure) collapses modeled by subsampling and \(1\)\hyp Lipschitz retractions; \emph{(b)} simplicial refinements with bounded local degree; \emph{(c)} graph sparsifications preserving the normalized Laplacian construction and the \(1\)\hyp Lipschitz property of \(\mathcal{G}\), thereby keeping each \(\mathbf{P}_i\) non\hyp expansive under these maps.
Each case is functorially embedded by an admissible \(\mathcal{G}\).
\end{declaration}

\begin{center}
\begin{tikzcd}[
  row sep=1.6em, column sep=3.4em,
  cells={nodes={font=\footnotesize}},
  every label/.append style={font=\scriptsize},
  scale=0.92, transform shape
]
{\text{Geometric input } X} \arrow[r, "\mathcal{G}"] 
  & {F\in \FiltCh{k}} \arrow[r, "\mathbf{P}_i"] 
    & {\mathbf{P}_i(F)} \arrow[r, "\Ttau"] \arrow[d, dashed, "\mathrm{PE}_i^{\le\tau}"']
      & {\Ttau\mathbf{P}_i(F)} \arrow[d, dashed, "\mathrm{PE}_i^{\le\tau}"] \\\n{} & {} & {\text{Spectral } L(\Ctau F)}
      \arrow[r, dashed, "\text{heat trace on }L(\Ctau F)"] 
      & {{\mathrm{HT},\,\mathrm{ST}_{\beta}}} \\\n{} & {\Rfun(F)}
      \arrow[from=3-2, to=1-3, bend left=20, dashed, "\LC"]
      \arrow[from=3-2, to=3-4, dashed, "{\begin{matrix}\Ctau\text{ then}\\ \Ext^1(-,Q)\end{matrix}}"]
  & {}
  & {{\Ext^1\big(\Rfun(\Ctau F),Q\big)=0\ \ (\text{check})}}
\end{tikzcd}
\end{center}

\subsection*{6.5. Failure geometry and diagnostics}
\begin{definition}[Geometric failure types at scale]\label{def:geom-failure}
Within the monitored window, a sample is \emph{Type~IV at scale \(\tau\)} if \(\mathrm{PE}^{\le\tau}\) and spectral indicators decay while \((\muc,\nuc)\neq(0,0)\).
The \emph{pure cokernel type} denotes \(\muc=0\) and \(\nuc>0\).
\end{definition}

\begin{declaration}[Specification: Diagnostic actions]\label{spec:diagnostic}
When \((\muc,\nuc)\neq(0,0)\), refine the index diagram or adjust \(\tau\)\hyp sweep granularity until either \emph{(a)} the obstruction vanishes, or \emph{(b)} the failure persists across refinements, in which case the regime is recorded as non\hyp collapsible at the monitored scale.
\end{declaration}

\subsection*{6.6. Symplectic hook: Fukaya realization (\textbf{[Spec]})}
\begin{declaration}[Spec--Fukaya realization]\label{spec:fukaya}
Let \(\mathsf{Symp}^{\mathrm{adm}}\) be exact/monotone Liouville domains or sectors with stops.
\(\mathcal{G}_{\mathrm{Fuk}}\) assigns action\hyp filtered Floer complexes on a fixed window \([a,\tau]\) with \(a\le \tau\) over a field.
Assume:
\emph{(F1)} finite action spectrum in \([a,\tau]\);
\emph{(F2)} continuation maps shift actions by \(\le\varepsilon\) uniformly (hence are \(1\)\hyp Lipschitz for interleavings);
\emph{(F3)} stop additions/Liouville contractions are deletion\hyp type (Appendix~E).
Then for each degree \(i\) and scale \(\tau\) the comparison maps \(\phi_{i,\tau}\) are isomorphisms, hence \((\mu,\nu)=(0,0)\) on the monitored window.
Proof sketches and scope limits appear in Appendix~O.
\end{declaration}

\begin{remark}[Scope and bridge domain]
The specification above does \emph{not} extend the proved bridge beyond \(D^{\mathrm{b}}(k\text{-mod})\);
it provides a stable geometric hook whose persistence\hyp level behavior feeds the Part~I pipeline.
\end{remark}

\subsection*{6.7. Permitted operations catalog and $\delta$-ledger (reinforced policy)}\label{subsec:ops-delta}
We record the admissible A\hyp side operations, their expected persistence\hyp level behavior \emph{after collapse}, and the mandatory $\delta$ logging.

\begin{definition}[Permitted operations]\label{def:ops-catalog}
Each A\hyp side step $U$ is labeled:
\begin{itemize}
  \item \emph{Deletion\hyp type (monotone).} Examples: stop addition / sector shrinking (symplectic), mollification (low\hyp pass filtering), viscosity increment (PDE), threshold lowering, filter upper\hyp cap. \emph{Guarantee:} after applying $C_\tau$, windowed persistence energies and spectral auxiliaries (aux\hyp bars) are \emph{non\hyp increasing} (Appendix~E).
  \item \emph{$\varepsilon$-continuation (non\hyp expansive).} Examples: small Hamiltonian continuation; micro time\hyp step; minor stop shift. \emph{Guarantee:} $d_{\mathrm{int}}(\mathbf{P}_i(F),\mathbf{P}_i(UF))\le \varepsilon$; after $C_\tau$, indicators are \emph{stable} up to the prescribed $\varepsilon$.
  \item \emph{Inclusion\hyp type (stable only).} Examples: domain enlargement, inclusion maps not covered by the deletion\hyp type list. \emph{Guarantee:} no monotonicity claim; only stability (non\hyp expansiveness) if the induced map is $1$-Lipschitz on persistence.
\end{itemize}
\end{definition}

\begin{declaration}[Mandatory $\delta$-ledger]\label{dec:delta-ledger-ch6}
For each step $U$ with collapse $C_{\tau}$ and a fixed degree $i$, record a three\hyp part non\hyp commutation budget
\[
\delta(i,\tau)\ =\ \delta^{\mathrm{alg}}(i,\tau)\ +\ \delta^{\mathrm{disc}}(i,\tau)\ +\ \delta^{\mathrm{meas}}(i,\tau),
\]
where $\delta^{\mathrm{alg}}$ is the theoretical Mirror/Transfer–Collapse mismatch, $\delta^{\mathrm{disc}}$ the discretization error, and $\delta^{\mathrm{meas}}$ the numerical/estimation error. The per\hyp window pipeline budget is $\Sigma\delta(i)=\sum_{U\in W}\delta(i,\tau)$ and must satisfy $\mathrm{gap}_\tau>\Sigma\delta(i)$ to pass B\hyp Gate$^{+}$ (Chapter~1).
\end{declaration}

\subsection*{6.8. Gate template (per step, per window) and saturation usage}\label{subsec:gate-template}
The following operational template is used for each A\hyp side step within a fixed domain window $W=[u,u')$ and a fixed collapse threshold $\tau>0$:

\begin{enumerate}
  \item \emph{Apply step $U$ and collapse.} Execute $U$ (labeled as in Definition~\ref{def:ops-catalog}), then apply $C_\tau$.
  \item \emph{Measure on B\hyp side single layer.} Compute $\Ttau\mathbf{P}_i(F)$, $\mathrm{PE}_i^{\le\tau}$, spectral indicators on $L(C_\tau F)$ under the fixed policy, and (if in scope) $\Ext^1(\Rfun(C_\tau F),k)$.
  \item \emph{Record $\delta$.} Append $\delta^{\mathrm{alg}},\delta^{\mathrm{disc}},\delta^{\mathrm{meas}}$ for this step to the per\hyp window ledger and update $\Sigma\delta(i)$.
  \item \emph{Evaluate B\hyp Gate$^{+}$.} Check PH1=0, (if in scope) Ext1=0, $(\mu,\nu)=(0,0)$ for the window and degree $i$, and enforce $\mathrm{gap}_\tau>\Sigma\delta(i)$.
  \item \emph{Log verdict.} If all pass, issue a windowed certificate; otherwise, classify failure (Type I–IV) and proceed with diagnostics (Declaration~\ref{spec:diagnostic}).
\end{enumerate}
On windows declared \emph{saturated} in the sense of Chapter~11, one may use the saturation gate (Chapter~11) as a reference to shorten step (4) (remain within its quantitative policy).

\subsection*{6.9. Windowed workflow and logging (MECE enforcement)}\label{subsec:window-workflow}
Let \(\{[u_k,u_{k+1})\}_k\) be a MECE partition (Chapter~2, Def.~\ref{def:ch2-mece}). For each window:
\begin{itemize}
  \item Fix \(\tau\) by the adaptation rule (Chapter~2, Def.~\ref{def:tau-adapt}); if spectral auxiliaries are used, fix \((\beta,[a,b])\).
  \item Run the gate template (Subsection~\ref{subsec:gate-template}) for each step; aggregate $\Sigma\delta(i)$ and evaluate B\hyp Gate$^{+}$.
  \item Record coverage checks (sum of lengths; sum of events) and all parameters in the manifest (Appendix~G).
\end{itemize}
Global claims are obtained by pasting windowed certificates via Restart (Lemma~\ref{lem:restart}) and Summability (Definition~\ref{def:summability}); $\tau$ is selected inside stable bands (Definition~\ref{def:stable-band}). When multiple torsion reflectors are used (e.g.\ length plus birth window), apply the soft\hyp commuting policy (Chapter~5, Definition~\ref{def:soft-commute}); otherwise fix a deterministic order and record the commutation defect in $\delta^{\mathrm{alg}}$.

\subsection*{6.10. Compliance checklist (per run)}\label{subsec:compliance}
\begin{enumerate}
  \item MECE windows recorded; coverage checks pass.
  \item Collapse threshold \(\tau\) adapted to resolution; spectral bin policy fixed and logged.
  \item Each step labeled (deletion/\(\varepsilon\)/inclusion) with 1\hyp Lipschitz rationale; \(\delta^{\mathrm{alg}},\delta^{\mathrm{disc}},\delta^{\mathrm{meas}}\) recorded.
  \item Indicators computed on B\hyp side single layer only; B\hyp Gate$^{+}$ evaluated (PH1/Ext1/$(\mu,\nu)$, safety margin).
  \item Tower audit $(\mu,\nu)=(0,0)$ on the window; stable band identified for \(\tau\).
  \item Verdict (accept/reject) and failure type logged; Restart/Summability plan updated for the next window.
\end{enumerate}

\subsection*{6.11. Summary}
This chapter specifies the operational program for geometric collapse in the implementable range. Admissible realizations (Definition~\ref{def:geom-real}) feed the persistence layer, where collapse $C_\tau$ is applied and all indicators are computed on the B\hyp side single layer. Deletion\hyp type steps are \emph{monotone} after collapse; $\varepsilon$–continuations are \emph{stable}. Mirror/Transfer non\hyp commutation with collapse is \emph{externalized} via a \(\delta\)–ledger and accumulated additively along pipelines. Windowed certificates are issued per MECE window by B\hyp Gate$^{+}$; global claims are obtained by pasting certificates using Restart and Summability, with $\tau$ selected inside stable bands. The soft\hyp commuting policy (Chapter~5) governs multi\hyp axis torsions. All assertions remain confined to the persistence layer and respect the one\hyp way bridge (Chapter~3) and the tower calculus (Chapter~4).

% ===========================
% Reinforced arithmetic integration (v16.0+)
% ===========================

\subsection*{6.12. PF/BC after\hyp collapse comparison protocol (arithmetic comparator)}
We promote the projection\hyp formula/base\hyp change (\textbf{PF/BC}) comparison to an after\hyp collapse arithmetic protocol at fixed windows and thresholds.

\begin{definition}[PF/BC after\hyp collapse comparator]\label{def:pf-bc-comparator}
Let
\[
\begin{tikzcd}[row sep=0.7em, column sep=1.4em]\nX' \arrow[r, "g"] \arrow[d, "p"'] & X \arrow[d, "f"] \\\nY' \arrow[r, "h"'] & Y\n\end{tikzcd}
\]
be a cartesian square in \(\mathsf{Geom}\). Set \(F=\mathcal{G}(X)\), \(F'=\mathcal{G}(X')\), \(G=\mathcal{G}(Y)\), \(G'=\mathcal{G}(Y')\).
For a fixed degree \(i\), window \(W\), and threshold \(\tau\), the \emph{PF/BC after\hyp collapse comparator} consists of the pair of morphisms in \(\Perskft\):
\[
\Ttau\mathbf{P}_i(F)\ \xrightarrow{\ \Ttau\mathbf{P}_i(f_\ast)\ }\ \Ttau\mathbf{P}_i(G),\n\qquad\n\Ttau\mathbf{P}_i(F')\ \xrightarrow{\ \Ttau\mathbf{P}_i(p_\ast)\ }\ \Ttau\mathbf{P}_i(G'),
\]
together with the base\hyp change maps induced from the square. We say the comparator \emph{passes} if the canonical square in \(\Perskft\) commutes and the two routes differ by at most \(\delta(i,\tau)\) in $d_{\mathrm{int}}$.
\end{definition}

\begin{proposition}[PF/BC after\hyp collapse comparison]\label{prop:pf-bc-after-collapse}
Assume:
\begin{enumerate}
  \item \textup{(Constructible PF/BC at [Spec])} The realization functor in use satisfies PF/BC at \textbf{[Spec]} (Appendix~N) and the diagram above is PF/BC\hyp admissible.
  \item \textup{(Lifting–coherence \LC)} The filtered lifts of maps exist up to f.q.i.\ and are compatible with \(\Ctau\).
  \item \textup{(Fixed window/threshold)} A right\hyp open window \(W\) and a threshold \(\tau\) are fixed.
\end{enumerate}
Then for each degree \(i\), the comparator in Definition~\ref{def:pf-bc-comparator} passes with a defect bounded by a uniform \(\delta(i,\tau)\) (logged in the \(\delta\)\hyp ledger). In particular, commutation holds strictly at the persistence layer if \(\delta(i,\tau)=0\) (e.g.\ exact PF/BC plus exact lifting).
\end{proposition}

\begin{proof}[Proof sketch]
Apply \(\mathbf{P}_i\) to the PF/BC square (at [Spec]) and then the exact reflector \(\Ttau\); exactness and $1$\hyp Lipschitzness ensure the induced square commutes up to the Mirror/Transfer–Collapse mismatch bounded by \(\delta(i,\tau)\) (Chapter~5, Definition~\ref{def:soft-commute} and Proposition~\ref{prop:pipeline-budget}). The bound is uniform on \(W\).
\end{proof}

\begin{corollary}[t\texorpdfstring{$\to$}{->}\(\mathbf{P}_i\)\texorpdfstring{$\to$}{->}\(\Ttau\)\texorpdfstring{$\to$}{->}compare]\label{cor:t-Pi-Ttau-compare}
For any pair of non\hyp expansive evolutions \(t\mapsto X_t\), and any PF/BC\hyp admissible comparison at times \(t\le t'\), the protocol
\[
X_t\ \xrightarrow{\ \mathcal{G}\ }\ F_t\ \xrightarrow{\ \mathbf{P}_i\ }\ \mathbf{P}_i(F_t)\ \xrightarrow{\ \Ttau\ }\ \Ttau\mathbf{P}_i(F_t)
\]
yields an after\hyp collapse comparison at \(\tau\) with defect \(\le \delta(i,\tau)\). Deviations are logged to the \(\delta\)\hyp ledger.
\end{corollary}

\subsection*{6.13. Collapse classification and the Defect functor (Iwasawa\hyp style notation)}
We formalize a typed verdict at fixed window and threshold, unifying invisible tower failures as Defects.

\begin{definition}[Defect functor at scale]\label{def:defect-ch6}
For a filtered diagram \(\{F_\lambda\}\) (indexed by time or refinement) and fixed \(\tau\), let
\[
\Defect_{i,\tau}^{\ker}:=\ker\phi_{i,\tau},\qquad \Defect_{i,\tau}^{\coker}:=\mathrm{coker}\,\phi_{i,\tau}\ \ \in \Perskft.
\]
Define the \emph{Iwasawa\hyp style tower sensitivities} (not classical Iwasawa invariants)
\[
\mu_{i,\tau}:=\mathrm{gdim}\big(\Defect_{i,\tau}^{\ker}\big),\qquad\n\nu_{i,\tau}:=\mathrm{gdim}\big(\Defect_{i,\tau}^{\coker}\big),\qquad\n\muc:=\sum_i\mu_{i,\tau},\ \ \nuc:=\sum_i\nu_{i,\tau}.
\]
\end{definition}

\begin{definition}[Typed collapse verdict]\label{def:typed-verdict}
Fix a window \(W\), degree set \(\mathcal{I}\), and \(\tau>0\). The \emph{collapse verdict} is the typed tuple
\[
\mathrm{Verdict}(W,\tau)\ :=\ \Big(\mathrm{Status},\ \mathrm{Reasons},\ \mathrm{Defects},\ \mathrm{Budgets}\Big),
\]
where:
\begin{itemize}
  \item \(\mathrm{Status}\in\{\texttt{Valid},\texttt{Failed}\}\).
  \item \(\mathrm{Reasons}\subset\{\texttt{TypeI},\texttt{TypeII},\texttt{TypeIII},\texttt{TypeIV}\}\) (cf.\ Chapter~4).
  \item \(\mathrm{Defects}=\{(\mu_{i,\tau},\nu_{i,\tau})\}_{i\in\mathcal{I}}\) with totals \((\muc,\nuc)\).
  \item \(\mathrm{Budgets}=\{\Sigma\delta(i)\}_{i\in\mathcal{I}}\) with safety margin \(\mathrm{gap}_\tau\).
\end{itemize}
We declare \(\texttt{Valid}\) iff, for all \(i\in\mathcal{I}\), \(\mathrm{PH}_1(C_\tau F)=0\), \(\Ext^1(\Rfun(C_\tau F),k)=0\) (when in scope), \((\mu_{i,\tau},\nu_{i,\tau})=(0,0)\), and \(\mathrm{gap}_\tau>\Sigma\delta(i)\).
Otherwise \(\texttt{Failed}\), with \(\mathrm{Reasons}\) populated and Defects reported.
\end{definition}

\begin{remark}[Auditability and unification]
This verdict aligns arithmetic comparison (PF/BC protocol), stability/monotonicity, and tower diagnostics: Type~IV is \emph{precisely} \((\muc,\nuc)\neq(0,0)\) at \(\tau\). The “Iwasawa\hyp style” notation \(\mu_{\mathrm{Collapse}}\) emphasizes tower sensitivity but is unrelated to classical Iwasawa theory.
\end{remark}

\subsection*{6.14. Sufficient arithmetic conditions S1–S3 and tower design rules}
We codify window\hyp level sufficient conditions that guarantee arithmetic comparability and exclude Type~IV.

\begin{definition}[Arithmetic design rules at \texorpdfstring{\(\tau\)}{tau}]\label{def:S123}
Let \(W\) be a right\hyp open window. We say \textbf{S1–S3} hold for degree \(i\) at \(\tau\) if:
\begin{itemize}
  \item \textbf{S1 (T\texorpdfstring{\(_\tau\)}{tau}–colim commutation).} For the filtered diagram \(\{F_\lambda\}\), the natural map
  \(\varinjlim_\lambda \Ttau\mathbf{P}_i(F_\lambda)\xrightarrow{\cong}\Ttau\mathbf{P}_i(\varinjlim_\lambda F_\lambda)\) is an isomorphism (constructible policy of Appendix~A).
  \item \textbf{S2 (No\hyp accumulation near \texorpdfstring{\(\tau\)}{tau}).} There exists \(\eta>0\) such that no bar of \(\mathbf{P}_i(F_\lambda)\) has length in \((\tau-\eta,\tau+\eta)\) for any \(\lambda\in W\).
  \item \textbf{S3 (T\texorpdfstring{\(_\tau\)}{tau}–Cauchy).} The net \(\{\Ttau\mathbf{P}_i(F_\lambda)\}_{\lambda\in W}\) is Cauchy in \(d_{\mathrm{int}}\) (hence convergent in the constructible range).
\end{itemize}
\end{definition}

\begin{theorem}[Arithmetic integration under S1–S3]\label{thm:S123}
If \textup{S1–S3} hold for all \(i\in\mathcal{I}\) on window \(W\) at \(\tau\), then:
\begin{enumerate}
  \item The PF/BC after\hyp collapse comparator (Proposition~\ref{prop:pf-bc-after-collapse}) passes on \(W\) (with \(\delta=0\) when PF/BC holds strictly at persistence).
  \item The tower defects vanish: \((\mu_{i,\tau},\nu_{i,\tau})=(0,0)\) for all \(i\in\mathcal{I}\), hence \((\muc,\nuc)=(0,0)\).
  \item Deletion\hyp type steps are non\hyp increasing for \(\mathrm{PE}^{\le\tau}\) and spectral tails on \(W\); inclusion\hyp type steps are stability\hyp only.
\end{enumerate}
Consequently, \(\mathrm{Verdict}(W,\tau)\) is \texttt{Valid} provided the PH/Ext checks pass and \(\mathrm{gap}_\tau>\Sigma\delta(i)\) for each \(i\).
\end{theorem}

\begin{proof}[Proof sketch]
S1 implies \(\phi_{i,\tau}\) is an isomorphism; S2 prevents length ambiguity across \(\tau\); S3 yields convergence and excludes oscillations that could create Type~IV artifacts. Together these force \(\Defect^{\ker/\coker}_{i,\tau}=0\). PF/BC commutes after \(\Ttau\) by exactness and $1$\hyp Lipschitzness; monotonicity follows from Appendix~E applied to \(C_\tau\).
\end{proof}

\begin{remark}[Design guidance]
S1–S3 are enforceable by construction: compute colimits at the functor level (Appendix~A), choose \(\tau\) away from active bar lengths (stable bands), and impose geometric step sizes ensuring $d_{\mathrm{int}}$–Cauchy behavior (Chapter~11).
\end{remark}

\subsection*{6.15. Window arithmetic comparator and manifest}
We standardize the arithmetic comparator and its logging for full auditability.

\begin{definition}[Window arithmetic comparator]\label{def:win-arith}
For a window \(W\), degree set \(\mathcal{I}\), and \(\tau\), define
\[
\mathrm{Arith}(W,\tau)\ :=\ \Big(\{\Ttau\mathbf{P}_i(F_\lambda)\}_{\lambda\in W,i\in\mathcal{I}},\ \{\phi_{i,\tau}\}_{i\in\mathcal{I}},\ \{\mathrm{PF/BC~squares}\},\ \{\delta(i,\tau)\}\Big),
\]
together with the bar\hyp adjacency table near \(\tau\) (for S2) and the $d_{\mathrm{int}}$\hyp increment log (for S3).
\end{definition}

\begin{proposition}[Comparator\texorpdfstring{$\Rightarrow$}{=>}verdict]\label{prop:arith-to-verdict}
If \(\mathrm{Arith}(W,\tau)\) satisfies S1–S3, all PF/BC squares pass (Proposition~\ref{prop:pf-bc-after-collapse}), and the PH/Ext checks pass, then \(\mathrm{Verdict}(W,\tau)=\texttt{Valid}\). Any deviation (non\hyp commuting square, failed S2/S3, nonzero defects) is recorded into \(\mathrm{Reasons}\) and \(\mathrm{Budgets}\) and yields \texttt{Failed}.
\end{proposition}

\begin{remark}[Manifest fields]
The manifest (Appendix~G) must include: window ID, \(\tau\), \(\mathcal{I}\), PF/BC squares list, \(\delta\)\hyp ledger, S1–S3 checkboxes with evidence, bar adjacency histogram near \(\tau\), $d_{\mathrm{int}}$ Cauchy plot, PH/Ext outcomes, and the final typed \(\mathrm{Verdict}\).
\end{remark}

\subsection*{6.16. Micro\hyp example (arithmetic comparator in practice)}
Consider a refinement tower \(X_n\to X_{n+1}\) (simplicial mesh) with \(\delta_n\downarrow 0\) such that each \(\mathbf{P}_1(\mathcal{G}(X_n))\) has a single bar \([0,\tau-\delta_n)\). Fix \(\tau\) and degree \(i=1\).

\begin{itemize}
  \item S2 holds (pick \(\eta=\frac{1}{2}\min_n\delta_n\)).
  \item S3 holds since \(\Ttau\mathbf{P}_1\big(\mathcal{G}(X_n)\big)=0\) for all \(n\) is trivially Cauchy.
  \item S1 holds in the constructible policy (filtered colimits commute with \(\Ttau\) at persistence).
\end{itemize}
Thus \((\mu_{1,\tau},\nu_{1,\tau})=(0,0)\). If the apex limit yields \(I[0,\infty)\) (as in Chapter~4, Proposition~4.3), then picking \(\tau\) inside a stable band avoids Type~IV: \(\Ttau\) kills the finite bars at all finite layers and the apex; arithmetic comparator passes with \(\delta=0\), yielding \texttt{Valid}. If instead \(\tau\) is tuned at the accumulation edge (violating S2), the same tower exhibits \texttt{Failed} with \(\nuc>0\) (pure cokernel Type~IV), exactly as in Chapter~4.

\subsection*{6.17. Closing remarks (arithmetic integration)}
The PF/BC after\hyp collapse comparator, the typed verdict with Defects, and the S1–S3 design rules raise the “geometry\(\to\)persistence” comparison to a strictly windowed, fixed\hyp \(\tau\) arithmetic protocol: all claims are stated at the persistence layer, deviations are quantified by \(\delta\), and invisible tower failures are canonically unified as Defects. This completes the arithmetic integration promised for v16.0: implementations become reviewable per window/\(\tau\), and audits directly expose any Type~IV via \((\muc,\nuc)\).



% ============================================================
% Chapter 7: Arithmetic Layers and Iwasawa Refinement (Design) — reinforced and extended (IMRN/AiM-ready)
% ============================================================
\section{Chapter 7: Arithmetic Layers and Iwasawa Refinement (Design)}
\addcontentsline{toc}{section}{Arithmetic Layers and Iwasawa Refinement (Design)}

\noindent\textbf{Index separation.} The collapse obstruction \(\muc\) used in this chapter is a persistence\hyp level diagnostic and is \emph{unrelated} to the classical Iwasawa \(\mu\)\nobreakdash invariant; no identity or implication between them is asserted (see also §\ref{rk:separation}).

\begin{remark}[Monotonicity convention]
Throughout this chapter we adopt the corrected monotonicity convention of
Chapter~6, Remark~\ref{rem:stability-vs-monotonicity}:
\emph{deletion\hyp type} updates are non\hyp increasing for spectral tails and windowed energies,
while \emph{inclusion\hyp type} updates are only stable (non\hyp expansive);
see Appendix~E for sufficient conditions and counterexamples.
\end{remark}

\subsection*{7.0. Standing hypotheses and admissible arithmetic realization}
All statements in this chapter are made within the \emph{constructible range}
(we identify \(\Pers^{\mathrm{ft}}_k\) with the constructible subcategory as in Chapters~2 and~6).
Fix a base field \(k\) and adopt the notation and posture of Part~I:
\(\FiltCh{k}\) denotes finite\hyp type filtered chain complexes,
\(\mathbf{P}_i:\FiltCh{k}\to\Pers^{\mathrm{cons}}_k\) the degreewise persistence functor,
and we write \(\Ttau:=\mathbf{T}_\tau\) for the Serre (bar\hyp deletion) reflector at scale \(\tau\ge 0\) (with \(\Ttau=\mathrm{Id}\) at \(\tau=0\)).
Its filtered lift \(C_\tau\) is used \emph{up to filtered quasi\hyp isomorphism} (Chapter~2, §§2.2–2.3).
A fixed realization \(\Rfun:\FiltCh{k}\to D^{\mathrm{b}}(k\text{-mod})\) is \(t\)\hyp exact.
Unless explicitly marked \textbf{[Spec]}, \emph{equalities and Lipschitz claims are asserted only at the persistence layer};
at the filtered\hyp complex layer they hold \emph{up to filtered quasi\hyp isomorphism}.
Endpoint conventions and the treatment of infinite bars are as in Chapter~2, Remark~\ref{rk:2-endpoints}.

Arithmetic input is organized as towers
\[
  \mathbb{T}\ :=\ \{X_t\}_{t\in I}\ \longrightarrow\ X_\infty,
\]
indexed by a directed set \(I\cup\{\infty\}\) with transition maps \(X_{t'}\to X_t\) for \(t'\ge t\) (e.g.\ norm/corestriction, specialization, level\hyp lowering).
Typical instances include cyclotomic/ray\hyp class towers of number fields, modular\hyp level towers, or Selmer\hyp complex towers.
Filtered (co)limits, when used, are computed objectwise in \([\mathbb{R},\mathsf{Vect}_k]\) and used only under the scope policy of Appendix~A (compute in the functor category and verify return to \(\Pers^{\mathrm{cons}}_k\)); no claim is made outside this regime.

\begin{definition}[Admissible arithmetic realization]\label{def:arith-real}
An \emph{admissible arithmetic realization} is a functor
\[
\begin{aligned}\n\mathcal{A}\colon\ & \mathsf{ArithTower}\longrightarrow \FiltCh{k},\\[-0.25em]\n& \mathbb{T}\longmapsto F_\bullet=\{F_t\}_{t\in I\cup\{\infty\}},\n\end{aligned}
\]
subject to:
\begin{enumerate}
  \item \textbf{Functoriality \& non\hyp expansiveness (persistence):} each transition \(X_{t'}\!\to X_{t}\) with \(t'\ge t\) induces a filtered chain map \(F_{t'}\!\to F_{t}\) such that, for every degree \(i\),
  \[
    d_{\mathrm{int}}\!\big(\mathbf{P}_i(F_{t'}),\mathbf{P}_i(F_{t})\big)\ \le\ \varepsilon_{t',t},\qquad \varepsilon_{t',t}\ge 0.
  \]
  In \emph{deletion\hyp type} steps satisfying Appendix~E (Dirichlet restriction, principal submatrices/Schur complements, certain contractions) one often has \(\varepsilon_{t',t}=0\); in general we assume the bound above (non\hyp expansive updates up to f.q.i.).
  \item \textbf{Finite\hyp type preservation \& colimits:} each \(F_t\) is degreewise finite\hyp type; degreewise filtered colimits in \(\FiltCh{k}\) are computed objectwise (Appendix~A).
  \item \textbf{Realization coherence:} a fixed \(t\)\hyp exact \(\Rfun\) is used for all \(t\), with comparison maps compatible with the lifting–coherence hypothesis \LC, so that functorially (up to f.q.i.)
  \[
    \Rfun(C_\tau F_t)\ \simeq\ \tau_{\ge 0}\,\Rfun(F_t).
  \]
  \item \textbf{Endpoints:} bars use the Part~I open/closed endpoint policy; \(\Ttau\) deletes only finite bars of length \(\le\tau\) (Chapter~2, §2.2).
\end{enumerate}
\end{definition}

\begin{remark}[Cone extension for the tower]\label{rk:cone-extension}
We work in the filtered index category \(I\cup\{\infty\}\) with \(t\le \infty\) and \emph{cone maps} \(X_t\to X_\infty\).
The realization \(\mathcal{A}\) carries these to filtered maps \(F_t\to F_\infty\), yielding the comparison maps in Definition~\ref{def:phi-munu}, mirroring Chapter~4.
\end{remark}

\subsection*{7.1. Class/Selmer visualization at the persistence layer}
\begin{definition}[Arithmetic visualization data]\label{def:vis}
Given \(\mathbb{T}\mapsto F_\bullet\) via \(\mathcal{A}\), define for each \(t\in I\) and degree \(i\):
\[
  \text{barcode } \mathcal{B}_i(F_t):=\mathrm{bars}\big(\mathbf{P}_i(F_t)\big),\qquad\n  \text{truncated energies } \mathrm{PE}_i^{\le\tau}(F_t)\ \text{ as in §6.1}.
\]
All measurements are computed on the \emph{truncated barcodes} \(\Ttau\mathbf{P}_i(F_t)\); equivalently on \(C_\tau F_t\).
We use the default \(\alpha=1\) unless stated otherwise (cf.\ §6.1).
Interpretation: \(\mathcal{B}_i(F_t)\) and \(\mathrm{PE}_i^{\le\tau}(F_t)\) serve as \emph{visual proxies} for growth/stability patterns of arithmetic invariants (e.g.\ class/Selmer\hyp like data) along the tower.
\end{definition}

\begin{remark}[Spectral layer and \(\Ext^1\)\hyp check]\label{rk:spectral-ext}
Form the normalized Hodge Laplacian \(L_i(C_\tau F_t)\) in degree \(i\) and record spectral tails/heat traces as in §6.1, with positive\hyp eigenvalue summation and standard convergence choices.
At the categorical layer, check \(\Ext^1\!\big(\Rfun(C_\tau F_t),Q\big)=0\) for \(Q\in\Qtest=\{k[0]\}\).
These three families (persistence, spectral, categorical) are monitored jointly.
\end{remark}

\subsection*{7.2. Tower diagnostics and obstructions}
\begin{definition}[Tower comparison and obstruction indices]\label{def:phi-munu}
For each degree \(i\) and scale \(\tau\), the comparison map
\[
  \phi_{i,\tau}:\ \varinjlim_{t\in I}\ \Ttau\!\big(\mathbf{P}_i(F_t)\big)\ \longrightarrow\ \Ttau\!\big(\mathbf{P}_i(F_\infty)\big)
\]
yields obstruction counts
\[
  \mu_{i,\tau}:=\dim_k\ker\phi_{i,\tau}\footnote{Here \(\dim_k\) denotes the \emph{generic\hyp fiber} dimension after truncation, i.e.\ the multiplicity of \(I[0,\infty)\) summands; see Appendix~D, Remark~\ref{rem:D-generic-dim}.},\qquad\n  \nu_{i,\tau}:=\dim_k\,\mathrm{coker}\big(\phi_{i,\tau}\big),\qquad\n  \muc:=\sum_{i}\mu_{i,\tau},\ \ \nuc:=\sum_i \nu_{i,\tau}.
\]
\end{definition}

\begin{declaration}[Spec–Arithmetic towers (non\hyp expansion)]\label{spec:arith-nonexp}
Index transitions are non\hyp expansive in the interleaving sense of Definition~\ref{def:arith-real}\,(1), with shifts \(\varepsilon_{t',t}\) uniformly controlled (e.g.\ \(\sup_{t'\ge t}\varepsilon_{t',t}<\infty\)).
Under finite\hyp type, objectwise filtered colimits (Appendix~A), each \(\phi_{i,\tau}\) is an isomorphism (cf.\ Chapter~4, Proposition~\ref{prop:mu-vanishing}), hence \((\muc,\nuc)=(0,0)\) at the monitored scales.
\end{declaration}

\begin{declaration}[Specification: Tower stability at the persistence layer]\label{spec:tower-stability}
Assume Definition~\ref{def:arith-real}\,(2) and that degreewise filtered colimits in \(\FiltCh{k}\) are computed objectwise \emph{under the scope policy of Appendix~A}.
Then, for each fixed \(\tau\) and all degrees \(i\),
\[
  \phi_{i,\tau}:\ \varinjlim_{t\in I} \Ttau\!\big(\mathbf{P}_i(F_t)\big)\ \xrightarrow{\ \cong\ }\ \Ttau\!\big(\mathbf{P}_i(F_\infty)\big)
\]
is an isomorphism. Consequently, \((\muc,\nuc)=(0,0)\) at scale \(\tau\).
Under a discrete sweep of \(\tau\), the conclusion holds pointwise at each monitored scale.
\end{declaration}

\begin{remark}[Excluding Type~IV under tower stability]\label{rk:exclude-typeIV}
Under Declaration~\ref{spec:tower-stability}, we have \((\muc,\nuc)=(0,0)\) at each fixed \(\tau\); hence Type~IV cannot occur at that scale.
\end{remark}

\begin{remark}[Failure patterns]\label{rk:fail}
If \((\muc,\nuc)\neq(0,0)\), record the failure type: \emph{pure cokernel} \((\muc=0,\nuc>0)\), \emph{pure kernel} \((\muc>0,\nuc=0)\), or \emph{mixed} \((\muc>0,\nuc>0)\).
Type~IV at scale \(\tau\) denotes decay of persistence/spectral indicators with \((\muc,\nuc)\neq(0,0)\).
\end{remark}

\begin{example}[Toy towers at the persistence layer]\label{ex:toy-towers}
Fix \(\tau>0\).
\emph{(Pure cokernel.)} Let \(\mathbf{P}_1(F_t)=I[0,\tau-\tfrac{1}{t})\) for $t\in\mathbb{N}$ with inclusions and set \(\mathbf{P}_1(F_\infty)=I[0,\infty)\).
Then \(\Ttau(\mathbf{P}_1(F_t))=0\) for all $t$ while \(\Ttau(\mathbf{P}_1(F_\infty))\cong I[0,\infty)\), hence \(\muc=0\) and \(\nuc>0\) at scale \(\tau\).
\emph{(Pure kernel.)} Dually, let \(\mathbf{P}_1(F_t)=\bigoplus_{j=1}^{t} I[0,\tau)\) with epimorphic transitions \(\mathbf{P}_1(F_{t+1})\twoheadrightarrow \mathbf{P}_1(F_t)\) and set \(\mathbf{P}_1(F_\infty)=0\).
Then \(\muc>0\) and \(\nuc=0\) at scale \(\tau\).
In both cases the cone extension (Remark~\ref{rk:cone-extension}) furnishes the maps to \(F_\infty\) and hence the \(\phi_{i,\tau}\).
\end{example}

\subsection*{7.3. Non\hyp identity with classical Iwasawa \(\mu\)}
\begin{remark}[Separation of indices]\label{rk:separation}
The persistence obstruction \(\muc\) is defined from kernels/cokernels of \(\phi_{i,\tau}\) between \emph{truncated} persistence modules, whereas the classical Iwasawa \(\mu\)\nobreakdash invariant measures \(p\)\nobreakdash primary growth for \(\mathbb{Z}_p[[T]]\)\nobreakdash modules.
No identity or implication between the two is asserted; any relation, if present, is programmatic and confined to \textbf{[Conjecture]} statements.
\end{remark}

\begin{remark}[On saturation gates]
Saturation/plateau criteria and their use as \textbf{[Spec]} binary gates are organized in Chapter~11; we make no chapter\hyp local gate declaration here.
\end{remark}

\subsection*{7.4. Program specifications for arithmetic towers}
\begin{declaration}[Specification: Admissible indexings and maps]\label{spec:indexing}
An indexing of the tower by conductor, level, or height that renders the transition maps non\hyp expansive (hence \(1\)\hyp Lipschitz under each \(\mathbf{P}_i\) in the interleaving sense of Definition~\ref{def:arith-real}(1)) is \emph{admissible}.
Under such indexings, energy and spectral indicators are stable (non\hyp expansive) in general and \emph{non\hyp increasing for deletion\hyp type steps} (Appendix~E), up to f.q.i.; no non\hyp increase is claimed for inclusion\hyp type updates.
\end{declaration}

\subsection*{7.5. Conjectural propagation along arithmetic towers}
\begin{conjecture}[AK–Arithmetic tower propagation]\label{conj:arith-prop}
Assume an admissible arithmetic realization $\mathcal{A}$ and \LC.
If, along a non\hyp expansive tower segment and for a scale interval in $\tau$, we have $(\muc,\nuc)=(0,0)$ and the persistence energies (deletion\hyp type: non\hyp increasing; general: stable) together with the spectral indicators are controlled as above, then the arithmetic visualization stabilizes at that scale:
the proxies registered by persistence/spectral layers remain bounded, and the categorical check
\(\Ext^1\big(\Rfun(C_\tau F_t),Q\big)=0\) persists along the segment.
No number\hyp theoretic identity, and no identification with the classical Iwasawa invariants, is asserted.
\end{conjecture}

\subsection*{7.6. Diagram and data flow}
\begin{center}
\begin{tikzcd}[
  ampersand replacement=\&,
  column sep=1.8em, row sep=1.8em,
  cells={nodes={font=\footnotesize}},
  every label/.append style={font=\scriptsize},
  scale=0.92, transform shape
]
{\text{Arithmetic tower } \mathbb{T}} \arrow[r, "\mathcal{A}"]
  \& {\{F_t\}\subset \FiltCh{k}} \arrow[r, "\mathbf{P}_i"]
    \& {\mathbf{P}_i(F_t)} \arrow[r, "\Ttau"]
       \arrow[from=1-3, to=2-3, dashed, "{\mathrm{PE}_i^{\le \tau}}"']
      \& {\Ttau\mathbf{P}_i(F_t)}
       \arrow[from=1-4, to=2-4, dashed, "{\mathrm{PE}_i^{\le \tau}}"] \\\n{\phantom{\cdot}} \& {\phantom{\cdot}}
  \& {\text{Spectral } L(C_\tau F_t)}
    \arrow[r, dashed, "{\text{heat trace on }L(C_\tau F_t)}"]
  \& {\mathrm{HT},\ \mathrm{ST}_{\beta}} \\\n{\phantom{\cdot}} \& {\Rfun(F_t)}
  \arrow[from=3-2, to=1-3, bend left=40, dashed, "\LC"]
  \arrow[from=3-2, to=3-4, dashed, "{C_\tau\ \text{ then }\Ext^1(-,Q)}"]
  \& {\phantom{\cdot}} \& {\Ext^1(\Rfun(C_\tau F_t),Q)=0} \\\n{\phantom{\cdot}} \& {\varinjlim_{t}\, \Ttau\mathbf{P}_i(F_t)}
  \arrow[from=4-2, to=4-4, "\phi_{i,\tau}"]
  \& {\phantom{\cdot}} \& {\Ttau\mathbf{P}_i(F_\infty)}
    \arrow[from=4-4, to=5-2, bend right=25, dashed, swap, "{\text{log }(\muc,\nuc)}"] \\\n{\phantom{\cdot}} \& {\text{failure log (pure/mixed)}} \& {\phantom{\cdot}} \& {\phantom{\cdot}}
\end{tikzcd}
\end{center}

\subsection*{7.7. Minimal assumptions per arithmetic class (design templates)}
\begin{declaration}[Specification: Template hypotheses]\label{spec:templates}
For practical deployment, the following minimal templates ensure admissibility (one\hyp line concrete instances shown):
\begin{itemize}
  \item \textbf{(MM spaces from arithmetic data).} Index by conductor/level; realize transitions as \(1\)\hyp Lipschitz retractions between metric(\hyp measure) models (e.g.\ modular curves under level\hyp lowering with Gromov–Hausdorff \(1\)\hyp Lipschitz maps); preserve finite\hyp type per degree.
  \item \textbf{(Simplicial/complex models).} Use bounded\hyp degree subdivisions for level changes (e.g.\ barycentric refinement at fixed depth); ensure objectwise degreewise colimits; non\hyp expansiveness under each \(\mathbf{P}_i\).
  \item \textbf{(Graphs/quotients).} Sparsify while preserving normalized Laplacians and the \(1\)\hyp Lipschitz property of \(\mathcal{G}\)/\(\mathcal{A}\) (e.g.\ degree\hyp bounded sparsification of Cayley graphs); compute spectra on \(C_\tau F_t\).
\end{itemize}
\end{declaration}

\subsection*{7.8. Reproducibility and logs}
\begin{remark}[Run logs and parameters]\label{rk:logs}
For each run, log: the tower index range \(t\in[t_{\min},t_{\max}]\), the scale sweep \(\tau\in[\tau_{\min},\tau_{\max}]\) with step, spectral parameters \((\beta,M(\tau),t_{\mathrm{HT}})\), and the obstruction tuple \((\muc,\nuc)\) per \(\tau\) (with failure type).
Record also the degree set used for aggregation (per\hyp degree vs.\ summed across \(i\)) to ensure consistent replays.
These logs are part of the program specification and enable exact reruns.
\end{remark}

\subsection*{7.9. Final guard-rails}
\begin{remark}[Scope and non-claims]\label{rk:nonclaims}
This chapter provides a design blueprint at the persistence/spectral/categorical layers for arithmetic towers.
It does \emph{not} assert number\hyp theoretic identities or decide deep conjectures; all forward\hyp looking statements are explicitly labeled \textbf{[Conjecture]} and rely on the implementable range and \LC.
No claim of \(\mathrm{PH}_1\Leftrightarrow\Ext^1\) is made; only the one\hyp way bridge under \textup{(B1)–(B3)} from Part~I is used.
\end{remark}

% -------------------- Reinforcements: height windows, PF/BC, commutativity, δ-ledger, gates --------------------

\subsection*{7.10. Height windows (MECE), PF/BC audit, and $\delta$-naturality}
\begin{definition}[Height windows and MECE partition]\label{def:height-windows}
Let the index set \(I\) carry a \emph{height} function \(h:I\to \mathbb{R}\) (e.g.\ conductor/level/weight) that is non\hyp decreasing along transitions. A \emph{height windowing} is a MECE partition \(\{W_k=[u_k,u_{k+1})\}_k\) of the height range such that the subdiagram of indices \(\{t\in I:\ h(t)\in W_k\}\) is filtered. All audits, gates, and certificates are performed \emph{per window}.
\end{definition}

\begin{declaration}[PF/BC audit after collapse]\label{dec:pf-bc-audit}
For external comparison functors (Projection Formula/Base Change) denoted \(\mathrm{PF},\mathrm{BC}\) at the arithmetic layer, we \emph{first} pass to persistence and \emph{then} collapse:
\[
X_t\ \xrightarrow{\ \mathcal{A}\ }\ F_t\ \xrightarrow{\ \mathbf{P}_i\ }\ \mathbf{P}_i(F_t)\ \xrightarrow{\ \Ttau\ }\ \Ttau\mathbf{P}_i(F_t).
\]
Pseudonaturality and PF/BC equalities are checked \emph{after} \(\Ttau\), i.e.\ on \(\Ttau\mathbf{P}_i(F_t)\), uniformly on each window. Any mismatch is recorded in the \(\delta\)\hyp ledger as
\[
\delta^{\mathrm{alg}}_{\mathrm{PF/BC}}(i,\tau;W_k)\ :=\ d_{\mathrm{int}}\!\Big(\Ttau \mathbf{P}_i\big(\mathrm{PF/BC}\circ \mathcal{A}\big),\ \Ttau \mathbf{P}_i\big(\mathcal{A}\circ \mathrm{PF/BC}\big)\Big).
\]
Discretization and measurement contributions are added as \(\delta^{\mathrm{disc}},\delta^{\mathrm{meas}}\); the per\hyp window budget is \(\Sigma\delta(i)\) (cf.\ Chapter~5, Specification~\ref{spec:delta-ledger} and Chapter~6, Declaration~\ref{dec:delta-ledger-ch6}).
\end{declaration}

\begin{remark}[Mirror/level transfer and $\delta$-ledger]
Let \(\Mirror\) denote level transfer (e.g.\ norm, corestriction, specialization). We measure the $2$-cell defect \(\epsilon_{i,\tau}:\Mirror\circ C_\tau\Rightarrow C_\tau\circ \Mirror\) after collapse with bound \(\delta(i,\tau)\) as in Chapter~5, Definition~\ref{def:delta-2cell}, and add it additively to the window ledger. Pipeline additivity and $1$-Lipschitz post\hyp processing follow from Proposition~\ref{prop:pipeline-budget}.
\end{remark}

\subsection*{7.11. Commutativity and pseudonaturality tests after collapse}
\begin{declaration}[Pseudonaturality verification policy]\label{dec:pseudonat}
All naturality/compatibility diagrams involving level transfer, PF/BC, or auxiliary reflectors are verified \emph{on the collapsed persistence layer} (\(\Ttau\mathbf{P}_i(F_t)\)). This avoids pre\hyp collapse torsion noise and aligns the audit with the gate posture (B\hyp side single layer).
\end{declaration}

\begin{definition}[A/B commutativity test and fallback]\label{def:ab-test}
Given two persistence\hyp level reflectors \(T_A,T_B\) (e.g.\ length\hyp threshold and birth\hyp window) we define
\[
\Delta_{\mathrm{comm}}(M;A,B)\ :=\ d_{\mathrm{int}}\!\big(T_A T_B M,\ T_B T_A M\big).
\]
On each height window \(W_k\) we run the A/B test on \(M=\Ttau\mathbf{P}_i(F_t)\). If \(\Delta_{\mathrm{comm}}\le \eta\) (tolerance), we accept \emph{soft\hyp commuting} (Chapter~5, Definition~\ref{def:soft-commute}); otherwise we fix a deterministic order (e.g.\ \(T_B\circ T_A\)), and record \(\Delta_{\mathrm{comm}}\) into \(\delta^{\mathrm{alg}}\).
\end{definition}

\begin{remark}[Nested torsions and order independence]
If the Serre classes are nested, order independence holds and no A/B test is required (Chapter~5, Proposition~\ref{prop:nested-commute}); otherwise soft\hyp commuting governs adoption.
\end{remark}

\subsection*{7.12. Gate template for arithmetic windows and saturation usage}
\begin{enumerate}
  \item \textbf{Window selection.} Choose a height window \(W_k=[u_k,u_{k+1})\) (Definition~\ref{def:height-windows}); fix \(\tau\) inside a stable band (Chapter~4, Definition~\ref{def:stable-band}).
  \item \textbf{Collapse then measure.} For each \(t\) with \(h(t)\in W_k\), compute \(\Ttau\mathbf{P}_i(F_t)\), energies \(\mathrm{PE}_i^{\le\tau}\), spectral indicators on \(L(C_\tau F_t)\), and (if in scope) \(\Ext^1(\Rfun(C_\tau F_t),k)\).
  \item \textbf{$\delta$ logging.} Audit PF/BC and Mirror transfer after collapse (Declaration~\ref{dec:pf-bc-audit}); run A/B tests (Definition~\ref{def:ab-test}); accumulate \(\Sigma\delta(i)\).
  \item \textbf{B\hyp Gate$^{+}$.} Require: \(\mathrm{PH}_1(C_\tau F_t)=0\), \((\mu,\nu)=(0,0)\) per window (Declarations~\ref{spec:tower-stability}), Ext\(^1\) pass (if checked), and safety margin \(\mathrm{gap}_\tau>\Sigma\delta(i)\).
  \item \textbf{Certificate \& paste.} Issue the window certificate; paste across windows via Restart and Summability (Chapter~4, Lemma~\ref{lem:restart}, Definition~\ref{def:summability}).
\end{enumerate}
On windows declared \emph{saturated} (Chapter~11), the gate may reference the saturation criteria directly.

\subsection*{7.13. Compliance checklist (arithmetic run)}
\begin{enumerate}
  \item Height windows form a MECE partition; coverage log recorded.
  \item \(\tau\)\hyp sweep and stable bands documented; spectral parameters fixed.
  \item PF/BC and Mirror audits executed \emph{after collapse}; \(\delta^{\mathrm{alg}},\delta^{\mathrm{disc}},\delta^{\mathrm{meas}}\) ledger complete.
  \item A/B commutativity tests run per window; soft\hyp commuting adopted or deterministic order fixed with \(\Delta_{\mathrm{comm}}\) logged.
  \item B\hyp side only measurements; B\hyp Gate$^{+}$ passed with safety margin; \((\muc,\nuc)=(0,0)\) per window.
  \item Certificates issued and pasted with Restart/Summability; failure types logged if any.
\end{enumerate}

% -------------------- New reinforcements: Mirror × Collapse 2-cell, Tropical shortening [Spec], Weak group collapse (linear proxy) --------------------

\subsection*{7.14. Mirror/Transfer on arithmetic towers: 2\hyp cell defect, additivity, and non\hyp increase (IMRN/AiM)}

\begin{definition}[Mirror/Transfer on arithmetic towers]\label{def:arith-mirror}
Let \(\Mirror:\FiltCh{k}\to\FiltCh{k}\) be a functor representing a level transfer (norm/corestriction/specialization) on arithmetic towers. We assume:
\begin{itemize}
  \item \textbf{(M1) Non\hyp expansiveness at persistence:} for all \(i\), \(d_{\mathrm{int}}(\mathbf{P}_i(\Mirror F),\mathbf{P}_i(\Mirror G))\le d_{\mathrm{int}}(\mathbf{P}_i(F),\mathbf{P}_i(G))\).
  \item \textbf{(M2) 2\hyp cell after collapse:} there exists a natural \(2\)\hyp cell \(\epsilon_{i,\tau}:\Mirror\circ C_\tau\Rightarrow C_\tau\circ \Mirror\) with uniform bound \(\delta(i,\tau)\ge 0\) in \(d_{\mathrm{int}}\), invariant under f.q.i.
\end{itemize}
\end{definition}

\begin{proposition}[Mirror × Collapse: additivity and non\hyp increase]\label{prop:arith-mirror-pipeline}
Let \(U_m,\dots,U_1\) be A\hyp side steps (each deletion\hyp type or \(\varepsilon\)\hyp continuation), interlaced with collapses \(C_{\tau_j}\). Under \textup{(M1)}–\textup{(M2)} and Chapter~5, Proposition~\ref{prop:pipeline-budget}, for any fixed \(\tau\) and degree \(i\),
\[
d_{\mathrm{int}}\!\Big(\Ttau\mathbf{P}_i(\Mirror(C_{\tau_m}U_m\cdots C_{\tau_1}U_1F)),\ \Ttau\mathbf{P}_i(C_{\tau_m}U_m\cdots C_{\tau_1}U_1\Mirror F)\Big)\ \le\ \sum_{j=1}^m \delta_j(i,\tau_j),
\]
and $1$\hyp Lipschitz post\hyp processing (including PF/BC comparators of §7.10) does not increase the right\hyp hand side.
\end{proposition}

\begin{proof}[Proof sketch]
Compose the natural \(2\)\hyp cells and apply the triangle inequality; use $1$\hyp Lipschitzness of \(\Ttau\) and post\hyp processors as in Chapter~5.
\end{proof}

\begin{remark}[Window arithmetic comparator with Mirror]
Combine Proposition~\ref{prop:arith-mirror-pipeline} with Proposition~\ref{prop:pf-bc-after-collapse} to audit Mirror and PF/BC simultaneously \emph{after collapse} per window; record the aggregate budget in the ledger.
\end{remark}

\subsection*{7.15. Tropical shortening at the arithmetic layer (\textbf{[Spec]})}
We encode a \emph{tropical} base contraction on arithmetic heights/regulators as a window\hyp level barcode shortener.

\begin{definition}[Tropical base contraction (\textbf{[Spec]})]\label{def:tropical}
Let \(\Trop_\lambda:\mathsf{ArithTower}\to\mathsf{ArithTower}\) be an endofunctor with parameter \(\lambda\in(0,1]\) such that the induced filtered map on \(\FiltCh{k}\) is non\hyp expansive under \(\mathbf{P}_i\) and, on each window \(W\) and threshold \(\tau\), \emph{uniformly shortens} degreewise barcodes by factor \(\kappa(\lambda',\lambda)\le 1\) (Definition~8.1 in spirit), up to f.q.i., after applying \(\Ttau\).
\end{definition}

\begin{proposition}[Window energy non\hyp increase under tropical shortening (\textbf{[Spec]})]\label{prop:tropical-energy}
Assume Definition~\ref{def:tropical}. Then for each degree \(i\) and window \(W\),
\[
\mathrm{PE}_i^{\le\tau}\big(C_\tau (\mathcal{A}\circ \Trop_{\lambda'})(X)\big)\ \le\ \mathrm{PE}_i^{\le\tau}\big(C_\tau (\mathcal{A}\circ \Trop_{\lambda})(X)\big)\qquad (\lambda'\le \lambda),
\]
with strict decrease whenever a positive portion of clipped bars are shortened by a factor \(<1\).
\end{proposition}

\begin{proof}[Proof sketch]
Shortening reduces clipped lengths inside \([0,\tau]\) up to f.q.i.; sum of clipped lengths (Definition~\ref{def:PE}) therefore decreases, cf.\ Theorem~8.1 in the geometric setting and Chapter~6, §6.1.
\end{proof}

\begin{remark}[Scope]
Tropical shortening is a \textbf{[Spec]} design tool: no number\hyp theoretic identity is invoked; it supplies a proxy to enforce monotone decay of window energies under controlled base contractions (e.g.\ level pruning).
\end{remark}

\subsection*{7.16. Weak group collapse (linear proxy) at fixed windows}
We define a window\hyp level \emph{weak group collapse} proxy that can be tested on arithmetic symmetry/transfer actions.

\begin{definition}[Barcode space and linearization]\label{def:barcode-space}
Fix a window \(W\) and threshold \(\tau\). For degree \(i\), write \(\Ttau\mathbf{P}_i(F)\cong \bigoplus_{b\in \mathcal{B}_{i,\tau}(F;W)} I_b\), and let
\[
V_{i,\tau}(W)\ :=\ \bigoplus_{b\in \mathcal{B}_{i,\tau}(F;W)} k\cdot e_b.
\]
For a groupoid \(\mathsf{Aut}(F)\) of filtered self\hyp maps at arithmetic level, any \(g\in \mathsf{Aut}(F)\) induces (after \(\Ttau\)) a linear map on \(V_{i,\tau}(W)\), well\hyp defined up to conjugacy.
\end{definition}

\begin{definition}[Weak group collapse (window\hyp level gate)]\label{def:weak-group}
Fix a finite set \(S\subset \mathsf{Aut}(F)\). We say \emph{weak group collapse holds at \((W,\tau)\)} if:
\begin{itemize}
  \item (Semi\hyp contraction) \(\mathrm{spr}(\rho_{i,\tau}(g))\le 1\) for all \(g\in S\), all \(i\) (spectral radius bound over an algebraic closure).
  \item (Bounded unipotent length) There is \(m\in\mathbb{N}\) with \((\rho_{i,\tau}(g)-I)^m=0\) for all \(g\in S\), uniformly in \(i\).
\end{itemize}
\end{definition}

\begin{proposition}[Acceptance criterion and stability]\label{prop:weak-group-accept}
If weak group collapse holds at \((W,\tau)\) and the tower diagnostics vanish \((\muc,\nuc)=(0,0)\), then the group action is semi\hyp contractive on the bar basis \emph{after collapse}, uniformly on \(W\), and B\hyp Gate$^{+}$ may adopt weak\hyp group\hyp collapse as an \emph{auxiliary} acceptance tag. The property is invariant under f.q.i.\ and stable under $1$\hyp Lipschitz post\hyp processing.
\end{proposition}

\begin{proof}[Proof sketch]
The linear proxy abstracts window\hyp level action on truncated bars; semi\hyp contraction and bounded unipotent length ensure no growth modes survive after collapse. Tower stability excludes Type~IV across the window. Invariance and stability follow from persistence\hyp level functoriality (Chapter~5) and the $1$\hyp Lipschitz policy.
\end{proof}

\begin{remark}[IMRN/AiM posture]
Weak group collapse does \emph{not} assert group trivialization; it is a linear, persistence\hyp level proxy on a fixed window and threshold, compatible with the budgeted pipeline and local gates. It fits the acceptance\hyp test toolbox, not a global equivalence claim.
\end{remark}

\subsection*{7.17. Summary (Mirror/Tropical/Langlands extensions, budgeted and windowed)}
We have consolidated the Mirror × Collapse \(2\)\hyp cell (with additive, non\hyp increasing bounds), a window\hyp level tropical shortening \textbf{[Spec]} that enforces energy non\hyp increase at fixed \(\tau\), and a weak group collapse proxy (semi\hyp contraction + bounded unipotent length) as auxiliary gate criteria. Each tool is \emph{windowed} and \emph{after collapse}, integrated with the PF/BC comparator (§7.10), the tower Defect calculus (§7.2), and the typed verdict of Chapter~6. All deviations are recorded in the \(\delta\)\hyp ledger; arithmetic decisions remain entirely at the persistence layer and within the constructible range, with the one\hyp way bridge only (Chapter~3). This unified policy delivers an IMRN/AiM\hyp ready, auditable program for arithmetic layers and Iwasawa\hyp style refinement that requires no further reinforcement beyond implementation details in Appendices~K–O.



% ============================================================
% Chapter 8: Mirror/Tropical Collapse (Weak Group Collapse) — reinforced and extended (IMRN/AiM-ready)
% ============================================================
\section{Chapter 8: Mirror/Tropical Collapse (Weak Group Collapse)}
\addcontentsline{toc}{section}{Mirror/Tropical Collapse (Weak Group Collapse)}

\noindent\textbf{Windowed policy and B-side judgement.}
All statements in this chapter are \emph{windowed} (Chapter~1, Def.~1.0; Chapter~2, Remark~\ref{rk:ch2-mece}).
Gate decisions are taken \emph{only} on the B-side after collapse, i.e.\ on single-layer objects
\(\mathbf{T}_\tau\mathbf{P}_i\) (equivalently \(\mathbf{P}_i(C_\tau-)\)).
Filtered-complex equalities hold only up to f.q.i.\ (Appendix~B).

\begin{remark}[Monotonicity convention]
Throughout this chapter we adopt the convention of
Chapter~6, Remark~\ref{rem:stability-vs-monotonicity}:
\emph{deletion\hyp type} updates are non\hyp increasing for spectral tails and windowed energies,
while \emph{inclusion\hyp type} updates are only stable (non\hyp expansive); see Appendix~E.
\end{remark}

\subsection*{8.0. Standing hypotheses, realizations, and $\delta$-ledger}
We fix a field \(k\) and work within the \emph{implementable range} of Part~I.
All statements lie in the \emph{constructible range} (we identify \(\Pers^{\mathrm{ft}}_k\) with the constructible subcategory as in Chapter~6).
Let \(\FiltCh{k}\) denote finite\hyp type filtered chain complexes,
\(\mathbf{P}_i:\FiltCh{k}\to\Pers^{\mathrm{cons}}_k\) the degreewise persistence functor, and write
\(\Ttau:=\mathbf{T}_\tau\) for the Serre (bar\hyp deletion) reflector at scale \(\tau\ge 0\).
Its filtered lift \(\Ctau\) is used \emph{up to filtered quasi\hyp isomorphism} (Chapter~2, §§2.2–2.3).
A fixed \(t\)\hyp exact realization \(\Rfun:\FiltCh{k}\to D^{\mathrm{b}}(k\text{-mod})\) is retained, and \LC\ holds whenever \(\Ctau\) is compared with \(\tau_{\ge 0}\!\circ\!\Rfun\).
\emph{Equalities and Lipschitz statements are asserted only at the persistence layer; at the filtered\hyp complex layer they hold up to filtered quasi\hyp isomorphism.}
Endpoint conventions and the treatment of infinite bars are as in Chapter~2, Remark~\ref{rk:2-endpoints}.
Kernel/cokernel diagnostics \((\mu_{\mathrm{Collapse}},\nu_{\mathrm{Collapse}})\) at scale \(\tau\) are computed from comparison maps as in Chapter~4, §4.2, with \(\dim_k\) interpreted as the \emph{generic\hyp fiber} dimension after truncation (multiplicity of \(I[0,\infty)\)); see Appendix~D, Remark~\ref{rem:D-generic-dim}.

\noindent\textbf{Quantitative commutation reference.}
When needed, we assume a natural \(2\)\hyp cell \(\theta:\Mirror\!\circ\! C_\tau\Rightarrow C_\tau\!\circ\!\Mirror\) whose effect at persistence is controlled by \(\delta(i,\tau)\ge 0\), \emph{uniform in the input \(F\)}, and that \(\Mirror\) is \(1\)\hyp Lipschitz; see Appendix~L (hypotheses \textup{(H1)}–\textup{(H2)}). The total non\hyp commutation budget along a pipeline is the additive sum \(\Sigma\delta(i)=\sum_j \delta_j(i,\tau_j)\) (Chapter~5, Proposition~\ref{prop:pipeline-budget}).

\medskip
We consider admissible realizations (Chapter~6, Definition~\ref{def:geom-real}; Chapter~7, Definition~\ref{def:arith-real})
\[
  \mathsf{Geom}_A\ \xrightarrow{\ \mathcal{G}_A\ }\ \FiltCh{k},\qquad\n  \mathsf{Geom}_B\ \xrightarrow{\ \mathcal{G}_B\ }\ \FiltCh{k}.
\]
A \emph{tropical base contraction} at parameter \(\lambda\in(0,1]\) is an endofunctor
\[
  \Trop_\lambda:\ \mathsf{Geom}_A\longrightarrow \mathsf{Geom}_A
\]
whose induced filtered map on \(F:=\mathcal{G}_A(X)\) is non\hyp expansive degreewise under each \(\mathbf{P}_i\) and monotone (deletion\hyp type) when \(\lambda\) decreases. A \emph{mirror transfer} is a functor
\[
  \Mirror:\ \FiltCh{k}\longrightarrow \FiltCh{k}
\]
that is non\hyp expansive for each \(\mathbf{P}_i\), compatible with \(\Ctau\) up to f.q.i., and subject to \LC\ for comparisons after realization \(\Rfun\). No categorical equivalence is assumed.

\begin{remark}[Endpoints and infinite bars]\label{rk:8-endpoints}
All statements are insensitive to open/closed endpoints, and infinite bars are not removed by \(\Ttau\);
windowed indicators clip their contributions (cf.\ Chapter~6).
\end{remark}

\begin{remark}[Cone extension for the tropical flow]\label{rk:8-cone}
For a directed parameter set \(\Lambda\subset(0,1]\) with \(\lambda'\le \lambda\), adjoin a terminal element \(\lambda_\ast\) (formally representing \(\lambda\to 0\)) and cone maps \(\Trop_{\lambda}\Rightarrow \Trop_{\lambda_\ast}\).
Under \(\mathcal{G}_A\), these induce filtered maps \(F_\lambda\to F_{\lambda_\ast}\), providing the comparison maps used to compute \((\mu_{\mathrm{Collapse}},\nu_{\mathrm{Collapse}})\) at fixed \(\tau\) along the \(\lambda\)\nobreakdash tower (Chapter~4).
\end{remark}

\subsection*{8.1. Tropical contraction and barcode shortening}
\begin{definition}[Uniform shortening proxy \textbf{[Spec]}]\label{def:shorten}
Let \(F\in\FiltCh{k}\) and fix \(\tau\ge 0\).
We say that a filtered map \(F\to F'\) \emph{uniformly shortens} degreewise barcodes at factor \(\kappa\in(0,1]\) \emph{up to f.q.i.} if, for every degree \(i\), the multiset of lengths in \(\Ttau\mathbf{P}_i(F')\) is obtained from that of \(\Ttau\mathbf{P}_i(F)\) by multiplying lengths by \(\le \kappa\) and possibly deleting some bars, modulo filtered quasi\hyp isomorphisms.
Infinite bars remain unaffected; \emph{\textbf{shortening is enforced only within the monitored window \([0,\tau]\) after applying \(\Ttau\)}} (windowing clips contributions at \(\tau\)).
The factor may depend on \(i\) and \(\tau\); we allow \(\kappa=\kappa_i(\tau)\) implicitly.
\emph{This is a model\hyp dependent operational specification; no canonical form is claimed outside the implementable range.}
\end{definition}

\begin{declaration}[Specification: Tropical reduction vs.\ barcode shortening]\label{spec:trop-short}
Within the implementable range, the tropical base contraction \(X\mapsto \Trop_\lambda(X)\) induces, for \(\lambda'\le \lambda\), filtered maps
\[
  \mathcal{G}_A(\Trop_{\lambda'}X)\ \longrightarrow\ \mathcal{G}_A(\Trop_{\lambda}X)
\]
that uniformly shorten degreewise barcodes at a factor \(\kappa(\lambda',\lambda)\le 1\) up to f.q.i.
Consequently, for fixed \(\tau\) the truncated energies \(\mathrm{PE}_i^{\le\tau}\) are non\hyp increasing along \(\lambda\searrow 0\), and strictly decreasing whenever \(\kappa(\lambda',\lambda)<1\) on a subset of bars whose cumulative length within the \(\tau\)\nobreakdash window has positive proportion of the total windowed bar length.
\end{declaration}

\subsection*{8.2. Weak group collapse: proxies on automorphism groupoids}
\begin{definition}[Automorphism groupoid and linear proxies]\label{def:group-proxy}
Let \(\mathsf{Aut}(F)\) be the groupoid of filtered self\hyp maps of \(F\) in \(\FiltCh{k}\).
Fix a scale \(\tau\) and degree \(i\).
Choose an interval\hyp decomposition model (up to f.q.i.) of the truncated persistence module
\[
  \Ttau\mathbf{P}_i(F)\ \cong\ \bigoplus_{b\in \mathcal{B}_{i,\tau}(F)} I_b,
\]
where \(\mathcal{B}_{i,\tau}(F)\) is the multiset of bars surviving under \(\Ttau\).
Define the \emph{barcode vector space}
\[
  V_{i,\tau}\ :=\ \bigoplus_{b\in \mathcal{B}_{i,\tau}(F)} k\cdot e_b,
\]
one basis vector \(e_b\) per bar \(b\) (so \(\dim_k V_{i,\tau}=\#\,\mathcal{B}_{i,\tau}(F)\), counted with multiplicity).
Any \(g\in \mathsf{Aut}(F)\) induces an automorphism of \(\Ttau\mathbf{P}_i(F)\), hence (after choosing a decomposition) a linear map on \(V_{i,\tau}\), well\hyp defined up to conjugacy.
This yields a functor (well\hyp defined up to conjugacy)
\[
  \rho_{i,\tau}:\ \mathsf{Aut}(F)\longrightarrow \mathsf{GL}(V_{i,\tau}).
\]
For a finite generating set \(S\subset \mathsf{Aut}(F)\) we define the \emph{linear proxies}:
\begin{itemize}
  \item \emph{spectral radius bound} \(\rho_{\max,i,\tau}(S):=\sup_{g\in S}\ \mathrm{spr}\big(\rho_{i,\tau}(g)\big)\) (computed over an algebraic closure of \(k\) if needed);
  \item \emph{unipotent length} \(\mathrm{nilp}_{i,\tau}(S)\): the smallest \(m\) such that for every \(g\in S\), the unipotent part of \(\rho_{i,\tau}(g)\) satisfies \((\rho_{i,\tau}(g)-I)^{m}=0\).
\end{itemize}
We call \(F\) \emph{weakly group\hyp collapsed at scale \(\tau\)} if for some finite \(S\),
\[
  \rho_{\max,i,\tau}(S)\le 1\ \text{ for all \(i\)},\qquad \sup_i \mathrm{nilp}_{i,\tau}(S)<\infty.
\]
This expresses the intuition that, \emph{after truncation by \(\Ttau\)}, the linear action on the bar basis is \emph{semi\hyp contractive} (spectral radius \(\le 1\)) with \emph{finite\hyp length unipotence} uniformly in \(i\).
\end{definition}

\begin{remark}[Meaning of non\hyp expansive]\label{rk:nonexp-meaning}
Here “non\hyp expansive’’ means the induced maps on \(\Ttau\mathbf{P}_i(\,{\cdot}\,)\) are \(1\)\nobreakdash Lipschitz for the interleaving distance \(d_{\mathrm{int}}\) for all degrees \(i\) and the fixed scale \(\tau\).
\end{remark}

\begin{remark}[Independence up to conjugacy]\label{rk:conj-inv}
The construction of \(V_{i,\tau}\) uses a choice of interval decomposition; different choices yield conjugate linear representations.
The quantities \(\rho_{\max,i,\tau}(S)\) and \(\mathrm{nilp}_{i,\tau}(S)\) are conjugacy invariants and hence well\hyp defined up to f.q.i.
\end{remark}

\begin{declaration}[Specification: Shortening \(\Rightarrow\) weak group collapse]\label{spec:short-to-weak}
Assume \((\mu_{\mathrm{Collapse}},\nu_{\mathrm{Collapse}})=(0,0)\) at scale \(\tau\) and that \(F\to F'\) uniformly shortens barcodes at factor \(\kappa<1\) up to f.q.i.\ (Def.~\ref{def:shorten}).
Then for any finite \(S\subset \mathsf{Aut}(F')\) consisting of non\hyp expansive maps,
\[
  \rho_{\max,i,\tau}(S)\le 1\quad \text{and}\quad \mathrm{nilp}_{i,\tau}(S)\ \text{is uniformly bounded in \(i\)}.
\]
Hence \(F'\) is weakly group\hyp collapsed at scale \(\tau\).
\emph{All quantities are \(\tau\)\nobreakdash dependent; weak group collapse is certified on each monitored window separately.}
\end{declaration}

\begin{remark}[Guard\hyp rails]\label{rk:guard}
The notion above is a \emph{persistence\hyp level proxy}; no claim is made about abstract group trivialization.
No equivalence \(\mathrm{PH}_1\Leftrightarrow\Ext^1\) is used; categorical checks remain one\hyp way as in Part~I.
\end{remark}

\subsection*{8.3. Mirror transfer of indicators and pipeline error budget}
\begin{declaration}[Spec–Mirror non\hyp expansion and $2$-cell]\label{spec:mirror}
Assume a functor \(\Mirror\) that is non\hyp expansive for each \(\mathbf{P}_i\) and a natural transformation \(\Mirror\!\circ\! \Ctau \Rightarrow \Ctau\!\circ\!\Mirror\) \emph{up to filtered quasi\hyp isomorphism}.
Then all persistence\hyp level indicators after \(\Ttau\) transfer with non\hyp expansive bounds (Appendix~L).
\emph{All statements are at the persistence layer in the constructible range, up to f.q.i.; equalities are asserted only at the persistence layer.}
\end{declaration}

\begin{theorem}[Pipeline error budget (Mirror $\times$ Collapse)]\label{thm:pipe-budget-8}
Let \(U_m,\dots,U_1\) be A\hyp side steps (each deletion\hyp type or \(\varepsilon\)\nobreakdash continuation), with interleaved collapses \(C_{\tau_j}\).
Assume natural $2$-cells \(\epsilon_{i,\tau_j}:\Mirror\circ C_{\tau_j}\Rightarrow C_{\tau_j}\circ \Mirror\) with bounds \(\delta_j(i,\tau_j)\).
Then, for any fixed B\hyp side threshold \(\tau\) and any degree \(i\),
\[
d_{\mathrm{int}}\!\Big(\mathbf{T}_\tau \mathbf{P}_i\big(\Mirror(C_{\tau_m}U_m\cdots C_{\tau_1}U_1F)\big),\ \mathbf{T}_\tau \mathbf{P}_i\big(C_{\tau_m}U_m\cdots C_{\tau_1}U_1\Mirror F\big)\Big)\ \le\ \sum_{j=1}^m \delta_j(i,\tau_j).
\]
Post\hyp processing by $1$-Lipschitz persistence maps (shifts $S^\varepsilon$, further truncations $\mathbf{T}_{\tau'}$, degree projections $\mathbf{P}_i$) does not increase the bound.
\end{theorem}

\begin{remark}[Soft\hyp commuting and $\delta^{\mathrm{alg}}$ accounting]\label{rk:soft-comm-8}
If a second reflector $T_B$ (e.g.\ birth\hyp window deletion) is involved together with the length\hyp threshold reflector $T_\tau$, run the A/B commutativity test (Chapter~5, Definition~\ref{def:soft-commute}) on \(\Ttau\mathbf{P}_i(F)\) per window and degree.
If \(\Delta_{\mathrm{comm}}\le \eta\) we accept soft\hyp commuting; otherwise we fix a deterministic order and record \(\Delta_{\mathrm{comm}}\) into $\delta^{\mathrm{alg}}$.
\end{remark}

{\let\n\relax \def\n{\unskip\space}
\begin{remark}[Mirror transfer with quantitative commutation]\label{rk:mirror-quant}
Under Declaration~\ref{spec:mirror}, \emph{in the constructible range and up to isomorphism in \(\Pers^{\mathrm{ft}}_k\)} there are natural identifications, for fixed \(\tau\) and all \(i\),
\[
  \mathbf{P}_i(\Mirror(\Ctau F))\ \cong\ \mathbf{P}_i(\Ctau(\Mirror F))\ \cong\ \Ttau\,\mathbf{P}_i(\Mirror F),
\]
and the truncated interleaving distances satisfy
\[
  d_{\mathrm{int}}\!\big(\Ttau\mathbf{P}_i(\Mirror F),\,\Ttau\mathbf{P}_i(\Mirror G)\big)\ \le\ d_{\mathrm{int}}\!\big(\mathbf{P}_i(F),\,\mathbf{P}_i(G)\big).
\]
If, in addition, \(\Mirror\) is \(1\)\nobreakdash Lipschitz, then for the same input \(F\),
\[
d_{\mathrm{int}}\!\big(\Ttau\,\mathbf{P}_i\big((\Mirror\!\circ\! \Ctau)F\big),\,\Ttau\,\mathbf{P}_i\big((\Ctau\!\circ\!\Mirror)F\big)\big)=0
\]
up to f.q.i., and for possibly different inputs \(F,G\) one has
\begin{align}
& d_{\mathrm{int}}\!\big(\Ttau\,\mathbf{P}_i((\Mirror\!\circ\! \Ctau)F),\,\Ttau\,\mathbf{P}_i((\Ctau\!\circ\!\Mirror)G)\big)\notag\\[-2pt]\n&\qquad\le\ d_{\mathrm{int}}\!\big(\mathbf{P}_i(F),\,\mathbf{P}_i(G)\big)
      \;+\; d_{\mathrm{int}}\!\big(\mathbf{P}_i(\Mirror F),\,\mathbf{P}_i(\Mirror G)\big)\label{eq:mirror-ctau-stability}\\[-2pt]\n&\qquad\le\ 2\,d_{\mathrm{int}}\!\big(\mathbf{P}_i(F),\,\mathbf{P}_i(G)\big).\notag\n\end{align}\nHence the indicators \(\{\mathrm{PE}_i^{\le\tau},\ \mathrm{ST}_{\beta}^{\ge M(\tau)},\ \Ext^1(\Rfun(\Ctau{-}),Q)\}\) are preserved across \(\Mirror\) with non\hyp expansive bounds (Appendix~L).\n\end{remark}\n}\n\n\begin{conjecture}[Mirror correspondences under collapse monitoring]\label{conj:mirror}\nAssuming \((\mu_{\mathrm{Collapse}},\nu_{\mathrm{Collapse}})=(0,0)\) and \LC, mirror correspondences preserve the monitored indicators and propagate weak group collapse (Def.~\ref{def:group-proxy}) across \(\Mirror\) along the same \(\tau\)\nobreakdash range.\n\end{conjecture}\n\n\subsection*{8.3.1.\ Spec–Saturation gate (tropical/mirror)}\n\begin{declaration}[Saturation gate \textbf{[Spec]} (tropical/mirror)]\label{gate:8-saturation}\nFix \(\tau^\ast>0\) and parameters \(\eta,\delta>0\).\nOn the window \([0,\tau^\ast]\), assume:\n(i) eventually the maximal \emph{finite} bar length in \(\mathbf{T}_{\tau^\ast}\mathbf{P}_i(F_\lambda)\) is \(\le \eta\);\n(ii) eventually \(d_{\mathrm{int}}\!\big(\mathbf{T}_{\tau^\ast}\mathbf{P}_i(F_\lambda),\mathbf{T}_{\tau^\ast}\mathbf{P}_i(F_{\lambda'})\big)\le \eta\);\n(iii) the \emph{edge gap} to the window, \(\delta:=\tau^\ast-\max\{b_r<\tau^\ast\}\), satisfies \(\delta>\eta\).\nThen, \textbf{within this window only}, we adopt the temporary policy\n\[\n\mathrm{PH}_1(\C_{\tau^\ast}F_\lambda)=0\quad\Longleftrightarrow\quad \Ext^1(\Rfun(\C_{\tau^\ast}F_\lambda),k)=0.
\]
This chapter references the gate; quantitative verification and usage are centralized in Chapter~11.
\end{declaration}

\subsection*{8.4. Monitoring protocol for tropical/mirror flows}
\begin{declaration}[Specification: Monitoring protocol]\label{spec:prot}
Fix a sweep \(\lambda\searrow 0\) and a finite set of scales \(\tau\).
For each sample:
\begin{enumerate}
  \item \emph{Compute and record} \(\Ttau\mathbf{P}_i\big(\mathcal{G}_A(\Trop_\lambda X)\big)\) and \(\mathrm{PE}_i^{\le\tau}\) on the truncated barcodes (equivalently on \(\Ctau\)).
  \item \emph{Compute and record} spectral indicators \(\mathrm{ST}_{\beta}^{\ge M(\tau)}\), \(\mathrm{HT}(t;\cdot)\) on \(L(\Ctau(\mathcal{G}_A(\Trop_\lambda X)))\) with a fixed \((\beta,M(\tau),t)\) policy.
  \item \emph{Check} \(\Ext^1\big(\Rfun(\Ctau{-}),Q\big)=0\) for \(Q\in\Qtest\).
  \item \emph{Evaluate} \((\mu_{\mathrm{Collapse}},\nu_{\mathrm{Collapse}})\) from the \(\lambda\)\nobreakdash tower via comparison maps at \(\tau\) (Remark~\ref{rk:8-cone}); \emph{these obstructions are invariant under filtered quasi\hyp isomorphisms and under cofinal reindexing} (Appendix~J).
  \item \emph{Group proxies:} choose a finite \(S\subset\mathsf{Aut}\big(\mathcal{G}_A(\Trop_\lambda X)\big)\) (e.g.\ monodromies/symmetries), and record \(\rho_{\max,i,\tau}(S)\) and \(\mathrm{nilp}_{i,\tau}(S)\) on \(V_{i,\tau}\).
  \item \emph{Mirror transfer:} apply \(\Mirror\) and repeat (1)–(5) on the mirror side.
\end{enumerate}
The \emph{stable regime} at \(\tau\) is where \((\mu_{\mathrm{Collapse}},\nu_{\mathrm{Collapse}})=(0,0)\) and indicators in (1)–(3) are non\hyp increasing along \(\lambda\searrow 0\);
\emph{weak group collapse} is declared when (5) meets the thresholds in Def.~\ref{def:group-proxy}.
\end{declaration}

\subsection*{8.5. Diagram (tropical flow, indicators, mirror transfer)}
\begin{center}
\begin{tikzcd}[
  ampersand replacement=\&,
  column sep=1.6em, row sep=2.0em,
  cells={nodes={font=\footnotesize}},
  every label/.append style={font=\scriptsize},
  scale=0.9, transform shape
]
X \arrow[r, "\mathcal{G}_A"] \arrow[d, "{\Trop_\lambda}"']
  \& F \arrow[r, "\mathbf{P}_i"] 
    \& \mathbf{P}_i(F) \arrow[r, "\Ttau"] \arrow[d, dashed, "{\mathrm{PE}_i^{\le \tau}}"']
      \& \Ttau\mathbf{P}_i(F) \arrow[d, dashed, "{\mathrm{PE}_i^{\le \tau}}"] \arrow[r, dashed, "{\Mirror}"] 
        \& \Ttau\mathbf{P}_i(\Mirror F) \arrow[d, dashed, "{\mathrm{PE}_i^{\le \tau}}"] \\\n\Trop_\lambda X \arrow[r, "\mathcal{G}_A"']
  \& F_\lambda \arrow[r, "\mathbf{P}_i"']
    \& \mathbf{P}_i(F_\lambda) \arrow[r, "\Ttau"']
      \& \Ttau\mathbf{P}_i(F_\lambda) \arrow[r, dashed, "{\Mirror}"']
        \& \Ttau\mathbf{P}_i(\Mirror F_\lambda) \\\n \& {} \& \text{Spectral } L(\Ctau F) \arrow[r, dashed, "{\text{heat trace on }L(\Ctau F)}"] \&
    {\mathrm{HT},\ \mathrm{ST}_{\beta}} \& \\\n\& \Rfun(F_\lambda) \arrow[uu, bend left=40, dashed, "\LC"] 
    \arrow[rr, dashed, "{\Ctau\ \text{ then }\Ext^1(-,Q)}"']
      \&\& \Ext^1(\Rfun(\Ctau F_\lambda),Q)=0 \\\n\& \text{group proxies on }V_{i,\tau}
   \arrow[uu, dashed, swap, "{\rho_{\max},\ \mathrm{nilp}}", shift left=1.5ex, pos=0.55]
\end{tikzcd}
\end{center}

\subsection*{8.6. Toy instance (persistence layer)}
\begin{example}[Uniform shortening under tropical scaling]\label{ex:trop}
Let \(\mathbf{P}_i(F)\) have bars of lengths \(\{\ell_j\}_j\) in \([0,\tau]\).
Suppose \(\Trop_\lambda\) induces \(\ell_j\mapsto \ell'_j\) with \(\ell'_j\le \kappa\ell_j\) for a fixed \(\kappa<1\) on a subset \(S\) of bars whose cumulative length within the \(\tau\)\nobreakdash window satisfies \(\sum_{j\in S}\ell_j>0\)
(equivalently, a positive fraction of the total windowed bar length), and deletions otherwise.
Then \(\mathrm{PE}_i^{\le\tau}\) strictly decreases and, if \((\mu_{\mathrm{Collapse}},\nu_{\mathrm{Collapse}})=(0,0)\) at \(\tau\), Def.~\ref{def:group-proxy} yields weak group collapse for non\hyp expansive automorphisms.
\end{example}

\subsection*{8.7. Final guard\hyp rails}
\begin{remark}[Scope and non\hyp claims]\label{rk:8-guard}
All claims are at the persistence/spectral/categorical layers within the implementable range, with \LC\ in force when comparing after realization.
No number\hyp theoretic or group\hyp theoretic decision is asserted; “weak group collapse’’ is a persistence\hyp level \emph{linear} proxy.
No claim of \(\mathrm{PH}_1\Leftrightarrow\Ext^1\) is made; only the one\hyp way bridge of Part~I is used.
The obstruction \(\mu_{\mathrm{Collapse}}\) is unrelated to the classical Iwasawa \(\mu\)\nobreakdash invariant.
\end{remark}

% -------------------- New reinforcements: Langlands tri-layer gates, transfer collapse kernel, μ_Collapse-based Type IV classification --------------------

\subsection*{8.8. Langlands tri-layer gates (Galois $\to$ Transfer $\to$ Functorial)}
We formalize gate criteria across a three\hyp layer pipeline:
\[
\text{Galois layer}\ \longrightarrow\ \text{Transfer layer}\ \longrightarrow\ \text{Functorial layer},
\]
each verified \emph{after collapse} on \(\Ttau\mathbf{P}_i(-)\), window by window.

\begin{definition}[Layer functors and objects]\label{def:layers}
Let \(F\in\FiltCh{k}\) and fix \(\tau\).
\begin{enumerate}
  \item \textbf{Galois layer.} A designated subgroup \(G\subset \mathsf{Aut}(F)\) (e.g.\ Galois/monodromy) acts by filtered maps. After collapse, \(G\) acts on \(V_{i,\tau}\) via \(\rho_{i,\tau}\) (Def.~\ref{def:group-proxy}).
  \item \textbf{Transfer layer.} A finite family \(\Trans=\{T_a:\FiltCh{k}\to\FiltCh{k}\}_a\) of non\hyp expansive “transfer’’ functors (e.g.\ norm, corestriction, Hecke, BC/PF adapters), each equipped with a \(2\)\hyp cell \(T_a\circ C_\tau\Rightarrow C_\tau\circ T_a\) bounded by \(\delta^{\mathrm{Tr}}_a(i,\tau)\).
  \item \textbf{Functorial layer.} A non\hyp expansive functor \(\Funct:\FiltCh{k}\to\FiltCh{k}\) (e.g.\ Mirror, Langlands lift) with \(2\)\hyp cell \(\Funct\circ C_\tau\Rightarrow C_\tau\circ \Funct\) bounded by \(\delta^{\mathrm{Fun}}(i,\tau)\).
\end{enumerate}
\end{definition}

\begin{definition}[Layer collapse maps and kernels]\label{def:layer-kernels}
For each degree \(i\) and scale \(\tau\):
\begin{itemize}
  \item \textbf{Galois collapse kernel.} For a finite generator set \(S\subset G\), define
  \[
  \phi^{\mathrm{Gal}}_{i,\tau}(g)\ :=\ \Ttau\mathbf{P}_i(g)-\mathrm{Id}:\ \Ttau\mathbf{P}_i(F)\to \Ttau\mathbf{P}_i(F),
  \]
  and the kernel count
  \[
  \mu^{\mathrm{Gal}}_{i,\tau}(S)\ :=\ \dim_k\bigcap_{g\in S}\ker\phi^{\mathrm{Gal}}_{i,\tau}(g)\quad\text{(generic\hyp fiber dimension)}.
  \]
  \item \textbf{Transfer collapse kernel (pass if zero).} For each \(T_a\in\Trans\),
  \[
  \phi^{\mathrm{Tr}}_{i,\tau}(a):\ \Ttau\mathbf{P}_i(F)\ \longrightarrow\ \Ttau\mathbf{P}_i(T_a F),\qquad\n  \mu^{\mathrm{Tr}}_{i,\tau}(a)\ :=\ \dim_k\ker\phi^{\mathrm{Tr}}_{i,\tau}(a).
  \]
  \item \textbf{Functorial collapse kernel.} For \(\Funct\),
  \[
  \phi^{\mathrm{Fun}}_{i,\tau}:\ \Ttau\mathbf{P}_i(F)\ \longrightarrow\ \Ttau\mathbf{P}_i(\Funct F),\qquad\n  \mu^{\mathrm{Fun}}_{i,\tau}\ :=\ \dim_k\ker\phi^{\mathrm{Fun}}_{i,\tau}.
  \]
\end{itemize}
Define layer cokernel counts similarly by \(\nu^{L}_{i,\tau}=\dim_k\mathrm{coker}(\phi^L_{i,\tau})\) for \(L\in\{\mathrm{Gal},\mathrm{Tr},\mathrm{Fun}\}\).
All counts are invariant under f.q.i.\ and under cofinal reindexing.
\end{definition}

\begin{declaration}[Layer gates (windowed)]\label{gate:tri-layer}
Fix a window and \(\tau\). We accept a layer when the following hold:
\begin{itemize}
  \item \textbf{Galois gate.} \(\mu^{\mathrm{Gal}}_{i,\tau}(S)=0\) for all monitored \(i\); plus weak group collapse proxy passes (Def.~\ref{def:group-proxy}).
  \item \textbf{Transfer gate.} For all \(T_a\in\Trans\), \(\mu^{\mathrm{Tr}}_{i,\tau}(a)=0\) and the cumulative budget \(\sum_a \delta^{\mathrm{Tr}}_a(i,\tau)\) stays below the \(\tau\)\nobreakdash edge gap.
  \item \textbf{Functorial gate.} \(\mu^{\mathrm{Fun}}_{i,\tau}=0\) and \(\delta^{\mathrm{Fun}}(i,\tau)\) stays below the \(\tau\)\nobreakdash edge gap.
\end{itemize}
The \emph{Langlands tri\hyp layer gate} passes when all three layer gates pass and \((\mu_{\mathrm{Collapse}},\nu_{\mathrm{Collapse}})=(0,0)\) at \(\tau\); energies/spectral tails are non\hyp increasing within the window.
\end{declaration}

\begin{remark}[Rationale]
The Galois gate forbids nontrivial fixed infinite\hyp type bar sectors after collapse; the transfer gate forbids losses invisible to energy/spectral proxies but detected by \(\ker\phi^{\mathrm{Tr}}\); the functorial gate controls changes under functorial lifts (e.g.\ Mirror).
\end{remark}

\subsection*{8.9. Type IV failure (μ\_Collapse–based) — layerwise visibility}
We relocate the Type~IV classification to the main text and refine it per layer and visibility.

\begin{definition}[Visibility and Type IV codes]\label{def:type-iv}
At fixed \(\tau\) and window, define:
\begin{itemize}
  \item \emph{Visible failure} if any energy/spectral indicator exceeds tolerance beyond the $\delta$ budget.
  \item \emph{Invisible failure} if indicators stay within budget but some \(\mu^L_{i,\tau}+\nu^L_{i,\tau}>0\) for \(L\in\{\mathrm{Gal},\mathrm{Tr},\mathrm{Fun}\}\).
\end{itemize}
We record the layerwise Type~IV codes:
\[
\mathrm{IV}\text{-}L\big[\mathrm{vis}/\mathrm{inv}\big]\qquad L\in\{\mathrm{Gal},\mathrm{Tr},\mathrm{Fun}\}.
\]
A global Type~IV flag at \(\tau\) is set if any layer has \(\mu^L+\nu^L>0\).
\end{definition}

\begin{remark}[Transfer–kernel trigger]
The \emph{transfer collapse kernel} is gate\hyp decisive: \(\mu^{\mathrm{Tr}}_{i,\tau}(a)=0\) for all \(a\) is necessary to avoid \(\mathrm{IV}\text{-}\mathrm{Tr}[\mathrm{inv}]\). This makes “where it broke’’ auditable.
\end{remark}

\subsection*{8.10. δ-ledger split by layers and soft-commuting across layers}
\begin{definition}[Layered δ-ledger]\label{def:layer-delta}
We refine the ledger by
\[
\delta(i,\tau)\ =\ \delta^{\mathrm{alg}}(i,\tau)\;+\;\delta^{\mathrm{disc}}(i,\tau)\;+\;\delta^{\mathrm{meas}}(i,\tau),
\]
with \(\delta^{\mathrm{alg}}\) decomposed as
\[
\delta^{\mathrm{alg}}(i,\tau)\ =\ \delta^{\mathrm{Gal}}(i,\tau)\;+\;\sum_{a}\delta^{\mathrm{Tr}}_a(i,\tau)\;+\;\delta^{\mathrm{Fun}}(i,\tau),
\]
where \(\delta^{\mathrm{Gal}}\) accounts for noncommuting A/B reflectors under the \(G\)\nobreakdash action, \(\delta^{\mathrm{Tr}}_a\) for each transfer functor, and \(\delta^{\mathrm{Fun}}\) for the functorial lift.
\end{definition}

\begin{declaration}[Cross-layer soft-commuting]\label{dec:cross-soft}
For reflectors \(T_A,T_B\) and a layer functor \(L\in\{g\in G,\ T_a,\ \Funct\}\),
we test
\[
\Delta_{\mathrm{comm}}(M;A,B|L)\ :=\ d_{\mathrm{int}}\!\big(T_A T_B (L\cdot M),\ T_B T_A (L\cdot M)\big)
\]
on \(M=\Ttau\mathbf{P}_i(F)\).
If \(\Delta_{\mathrm{comm}}\le \eta\), adopt soft\hyp commuting; else fix an order and add \(\Delta_{\mathrm{comm}}\) to \(\delta^{\mathrm{alg}}\) in the corresponding layer bin.
\end{declaration}

\subsection*{8.11. Tri-layer diagram and gate placement}
\begin{center}
\begin{tikzcd}[
  ampersand replacement=\&,
  column sep=1.8em, row sep=1.8em,
  cells={nodes={font=\scriptsize}},
  scale=0.92, transform shape
]
% row 1
{\Ttau\mathbf{P}_i(F)}
  \arrow[from=1-1, to=1-2, dashed, "{g\in G}"]
  \arrow[from=1-1, to=2-2, dashed, swap, "{T_a}"]
  \arrow[from=1-1, to=3-3, dashed, bend right=10, "{\Funct}"']
\&
{\Ttau\mathbf{P}_i(F)} \arrow[r, dashed, "\phi^{\mathrm{Gal}}_{i,\tau}(g)"] \&
{\Ttau\mathbf{P}_i(F)} \arrow[r, dashed, "{\text{Gate: } \mu^{\mathrm{Gal}}=0}"] \&
\checkmark_{\mathrm{Gal}} \\
% row 2
\&
{\Ttau\mathbf{P}_i(T_a F)} \arrow[r, dashed, "\phi^{\mathrm{Tr}}_{i,\tau}(a)"] \&
{\Ttau\mathbf{P}_i(T_a F)} \arrow[r, dashed, "{\text{Gate: } \mu^{\mathrm{Tr}}=0}"] \&
\checkmark_{\mathrm{Tr}} \\
% row 3
\&
\&
{\Ttau\mathbf{P}_i(\Funct F)} \arrow[r, dashed, "\phi^{\mathrm{Fun}}_{i,\tau}"] \&
\checkmark_{\mathrm{Fun}} \\
% row 4 (説明行)
\&
\&
\& \text{Tri-layer pass if all }\checkmark\text{ and }(\mu,\nu)=(0,0)
\end{tikzcd}
\end{center}


\subsection*{8.12. Integration with tropical shortening and weak group collapse}
\begin{proposition}[From shortening to tri-layer acceptance]\label{prop:short-trilayer}
Assume on a fixed window and \(\tau\):
(i) \((\mu_{\mathrm{Collapse}},\nu_{\mathrm{Collapse}})=(0,0)\);
(ii) tropical shortening with factor \(\kappa<1\) on a positive\hyp mass subset (Def.~\ref{def:shorten});
(iii) all layer functors are non\hyp expansive with bounded \(2\)\hyp cell budgets per §\ref{def:layer-delta}.
Then the Galois gate passes via weak group collapse (Decl.~\ref{spec:short-to-weak}); the Transfer and Functorial gates pass provided their kernels vanish (Def.~\ref{def:layer-kernels}) and budget sums stay below the edge gap.
\end{proposition}

\begin{proof}[Proof sketch]
(i)\,\(\Rightarrow\) baseline stability; (ii)\,\(\Rightarrow\) semi\hyp contraction + bounded unipotent length on \(V_{i,\tau}\); (iii)\,\(\Rightarrow\) layer maps are $1$\hyp Lipschitz, so any nonzero kernel would be caught by \(\mu^L>0\); budgeted $2$\hyp cells preserve acceptance margins.
\end{proof}

\subsection*{8.13. Operational checklist (tri-layer, windowed)}
\begin{enumerate}
  \item Fix MECE windows and \(\tau\)\nobreakdash sweep; set spectral parameters and tolerances; initialize the layered \(\delta\)\nobreakdash ledger.
  \item Run tropical flow \(\lambda\searrow 0\); per sample compute \(\Ttau\mathbf{P}_i\), energies, spectra, \(\Ext^1\).
  \item Compute \((\mu_{\mathrm{Collapse}},\nu_{\mathrm{Collapse}})\) at \(\tau\) along the \(\lambda\)\nobreakdash tower.
  \item Galois layer: record \(\rho_{\max},\mathrm{nilp}\); evaluate \(\mu^{\mathrm{Gal}}_{i,\tau}(S)\) and A/B soft\hyp commute tests under \(G\).
  \item Transfer layer: for each \(T_a\), evaluate \(\mu^{\mathrm{Tr}}_{i,\tau}(a)\); record \(\delta^{\mathrm{Tr}}_a\); run soft\hyp commute tests.
  \item Functorial layer: evaluate \(\mu^{\mathrm{Fun}}_{i,\tau}\); record \(\delta^{\mathrm{Fun}}\); run soft\hyp commute tests.
  \item Gate: accept layerwise; accept tri\hyp layer if all pass and \((\mu,\nu)=(0,0)\); otherwise log Type~IV codes per Def.~\ref{def:type-iv}.
\end{enumerate}

\subsection*{8.14. Notes on sheaf proxies and towers (optional \textbf{[Spec]})}
\begin{remark}[Sheaf proxies]
If a constructible sheaf model \(\mathcal{F}\) on a geometric carrier is available, one may compute windowed barcodes of \(\mathrm{R}\Gamma(\mathcal{F})\) and run the same gates on \(\Ttau\mathbf{P}_i(\mathrm{Sing}(\mathcal{F}))\), staying within the persistence layer; no new identities are claimed.
\end{remark}

\begin{remark}[Tower compatibility]
All layer kernels and Type~IV codes are invariant under f.q.i.\ and cofinal reindexing of towers (Appendix~J). This ensures pasteability across windows and along levels.
\end{remark}

\subsection*{8.15. Compliance summary (IMRN/AiM posture)}
\begin{enumerate}
  \item All decisions are B\hyp side, windowed, and ledgered; equalities only at persistence; filtered\hyp complex statements up to f.q.i.
  \item Non\hyp expansive policies and $2$\hyp cell budgets are explicit and additive; A/B soft\hyp commuting governs non\hyp nested reflectors.
  \item Langlands tri\hyp layer gates expose \emph{where} failures occur; transfer collapse kernel passes iff zero.
  \item Type~IV classification is moved in\hyp text, layerwise, with visibility label and μ/ν diagnostics.
  \item No identification with classical Iwasawa invariants is claimed; the program remains auditable and implementable within the constructible range.
\end{enumerate}



% ============================================================
% Chapter 9: Langlands Collapse (Three Layers) — reinforced with boundary models and window predicates (IMRN/AiM-ready)
% ============================================================
\section{Chapter 9: Langlands Collapse (Three Layers)}
\addcontentsline{toc}{section}{Langlands Collapse (Three Layers)}

\noindent\textbf{Scope note (PF/BC, truncation, and comparison order).}
Projection Formula and Base Change (PF/BC) are referenced from Appendix~N and are applied \emph{objectwise in \(t\)} in the functor category \([\mathbb{R},\mathsf{Vect}_k]\).
All comparisons are then made \emph{at the persistence layer after truncation by \(\Ttau\)}.
Concretely, for any morphism of filtered objects computed via PF/BC, the standard operating procedure is:
\[
\text{for each } t \ \Longrightarrow\ \text{apply } \mathbf{P}_i \ \Longrightarrow\ \text{apply } \Ttau \ \Longrightarrow\ \text{compare in } \Perskft.
\]
All inter\hyp layer equalities and Lipschitz statements are asserted only \emph{after truncation} and hold up to f.q.i.\ (pseudonatural, not strict).
When energies/indicators are involved, we require a \emph{fixed window, fixed \(\tau\), and a fixed \(\delta\)\nobreakdash policy} on both sides of any comparison; see Remark~\ref{rk:9-delta-policy}.
Endpoint conventions and infinite bars follow Chapter~2, Remark~\ref{rk:2-endpoints}.

\begin{remark}[Monotonicity convention]
Throughout this chapter we adopt the convention of
Chapter~6, Remark~\ref{rem:stability-vs-monotonicity}:
\emph{deletion\hyp type} updates are non\hyp increasing for spectral tails and windowed energies,
while \emph{inclusion\hyp type} updates are only stable (non\hyp expansive); see Appendix~E.
\end{remark}

\subsection*{9.0. Standing hypotheses and admissible Langlands realizations}
We fix a base field \(k\) and work within the \emph{implementable range} of Part~I.
\emph{All statements in this chapter are made within the constructible range}
(we identify \(\Perskft\) with the constructible subcategory as in Chapter~6).
Let \(\FiltCh{k}\) denote finite\hyp type filtered chain complexes and
\(\mathbf{P}_i:\FiltCh{k}\to\Perskft\) the degreewise persistence functor; we write
\(\Ttau:=\mathbf{T}_\tau\) for the Serre (bar\hyp deletion) reflector at scale \(\tau\ge 0\),
and use its filtered lift \(\Ctau\) \emph{up to filtered quasi\hyp isomorphism} (Chapter~2, §§2.2–2.3).
A fixed \(t\)\hyp exact realization \(\Rfun:\FiltCh{k}\to D^{\mathrm{b}}(k\text{-mod})\) is retained, and the lifting–coherence hypothesis \(\LC\) is assumed when comparing \(\Ctau\) with \(\tau_{\ge 0}\!\circ\!\Rfun\).
\emph{Equalities and Lipschitz statements are asserted only at the persistence layer; at the filtered\hyp complex layer they hold up to filtered quasi\hyp isomorphism.}

We consider three data layers
\[
\mathsf{Gal}\ \xrightarrow{\ \mathsf{Trans}\ }\ \mathsf{Par}\ \xrightarrow{\ \mathsf{Funct}\ }\ \mathsf{Aut},
\]
heuristically “Galois \(\to\) Transfer \(\to\) Functoriality”.
An \emph{admissible Langlands triple of realizations} consists of functors
\[
\mathcal{L}_{\mathrm{Gal}},\ \mathcal{L}_{\mathrm{Tr}},\ \mathcal{L}_{\mathrm{Aut}}:\ \FiltCh{k}\longrightarrow\FiltCh{k},
\]
subject to:
\begin{itemize}
  \item \textbf{Non\hyp expansiveness.} Each layer induces filtered maps whose images under every \(\mathbf{P}_i\) are \(1\)\nobreakdash Lipschitz for the interleaving distance; deletion\hyp type updates (Appendix~E) yield non\hyp increase of windowed indicators up to f.q.i., while inclusion\hyp type updates guarantee only stability.
  \item \textbf{Compatibility with \(\Ctau\).} For each degree \(i\) there are natural identifications in \(\Perskft\),
  \(\mathbf{P}_i(\Ctau{-})\cong \Ttau\,\mathbf{P}_i(-)\).
  \item \textbf{Finite\hyp type \& (co)limits.} Outputs are degreewise finite\hyp type; degreewise filtered (co)limits in \(\FiltCh{k}\) are computed objectwise in \([\mathbb{R},\mathsf{Vect}_k]\) and used only under the scope policy of Appendix~A (compute in the functor category and verify return to \(\Pers^{\mathrm{cons}}_k\)).
  \item \textbf{Realization coherence.} The \(t\)\hyp exact \(\Rfun\) is compatible with \(\LC\) across layers, so functorially up to f.q.i.,
  \(\Rfun(\Ctau F)\simeq \tau_{\ge 0}\,\Rfun(F)\).
\end{itemize}
Kernel/cokernel diagnostics \((\mu_{\mathrm{Collapse}},\nu_{\mathrm{Collapse}})\) are always taken \emph{after truncation}, with \(\dim_k\) interpreted as \emph{generic\hyp fiber} dimension (multiplicity of \(I[0,\infty)\)); see Appendix~D, Remark~\ref{rem:D-generic-dim}.

\begin{remark}[Operational comparison policy (PF/BC and collapse ordering)]\label{rk:9-operational}
Given any PF/BC computation (Appendix~N) producing a morphism or correspondence at the filtered\hyp complex level, we adopt the mandatory comparison order:
\[
\text{(i) objectwise in \(t\)}\ \Longrightarrow\ \text{(ii) apply }\mathbf{P}_i\ \Longrightarrow\ \text{(iii) apply }\Ttau\ \Longrightarrow\ \text{(iv) compare in }\Perskft.
\]
All “same\hyp scale” claims are checked with the \emph{same window}, the \emph{same \(\tau\)}, and the \emph{same \(\delta\)\nobreakdash policy}; see Remark~\ref{rk:9-delta-policy}.
\end{remark}

\begin{declaration}[Spec–Derived Langlands transfers]\label{spec:9-derived}
Besides \(\Rfun:\FiltCh{k}\!\to\!D^{\mathrm{b}}(k\text{-mod})\), we may use (as \textbf{[Spec]})
\[
\mathsf{R}_{\mathrm{coh}}:\ \FiltCh{k}\to D^{\mathrm{b}}\!\operatorname{Coh}(\mathfrak{X}),\qquad\n\mathsf{R}_{\acute{e}t}:\ \FiltCh{k}\to D^{\mathrm{b}}_{\mathrm{c}}(\mathfrak{Y}_{\acute{e}t},\Lambda)
\]
with \emph{field} coefficients \(\Lambda\), where \(\mathfrak{X}\), \(\mathfrak{Y}\) are parameter/automorphic stacks.
Projection Formula and Base Change are assumed \emph{exactly as tabulated in Appendix~N} (including proper/smooth hypotheses and the standard \(t\)\nobreakdash structure/degree normalizations).
Under these assumptions, the normalized transfers (degree\hyp normalized pullback/pushforward, kernel transforms/Hecke correspondences satisfying PF/BC, normalized parabolic induction/Jacquet) are \emph{non\hyp expansive at the persistence layer after truncation}: for each \(i,\tau\) and any such transfer \(\Phi\),
\[
d_{\mathrm{int}}\!\big(\Ttau\mathbf{P}_i(\Phi F),\ \Ttau\mathbf{P}_i(\Phi G)\big)\ \le\ d_{\mathrm{int}}\!\big(\Ttau\mathbf{P}_i(F),\ \Ttau\mathbf{P}_i(G)\big).
\]
All claims in this declaration are persistence\hyp level \emph{specifications}; the bridge \(\mathrm{PH}_1\Rightarrow\Ext^1\) is proved only in \(D^{\mathrm{b}}(k\text{-mod})\).
\end{declaration}

\begin{declaration}[Deletion\hyp type operations (PDE)]\label{spec:9-pde}
Inter\hyp /intra\hyp layer maps implemented by PDE\hyp style operations satisfying Appendix~E
(Dirichlet restriction/absorbing boundaries, positive semidefinite Loewner contractions, principal submatrices/Schur complements)
are treated as \emph{deletion\hyp type} and make windowed energies and spectral tails \emph{non\hyp increasing} after truncation.
Inclusion\hyp type updates are asserted only to be \emph{stable} (non\hyp expansive).
\end{declaration}

\begin{remark}[Endpoints and infinite bars]\label{rk:9-endpoints}
Open/closed endpoint conventions are immaterial; infinite bars are not removed by \(\Ttau\) and are clipped by windowed indicators (cf.\ Chapter~6).
\end{remark}

\begin{remark}[Indexing and cone extension]\label{rk:9-cone}
Let \(I\) be a directed index (e.g.\ level, conductor, degree of base\hyp change, auxiliary height).
Adjoin a terminal element \(\infty\) and cone maps \(t\to \infty\) in \(I\cup\{\infty\}\).
Under the realizations, these yield filtered maps \(F_t\to F_\infty\) per layer/degree, providing the comparison maps for \((\mu_{\mathrm{Collapse}},\nu_{\mathrm{Collapse}})\) at fixed~\(\tau\) (Chapter~4).
\end{remark}

\begin{remark}[Inter\hyp layer comparison and pseudonaturality]\label{rk:9-compare}
To speak of commutativity “up to f.q.i.” across layers after truncation, fix comparison natural transformations
\[
\alpha:\ \mathcal{L}_{\mathrm{Tr}}\circ \mathsf{Trans}\ \Longrightarrow\ \mathcal{L}_{\mathrm{Gal}},\qquad\n\beta:\ \mathcal{L}_{\mathrm{Aut}}\circ \mathsf{Funct}\ \Longrightarrow\ \mathcal{L}_{\mathrm{Tr}},
\]
which become filtered quasi\hyp isomorphisms degreewise after applying \(\Ctau\).
Equivalently, at the persistence layer there are natural isomorphisms
\[
\Ttau\mathbf{P}_i\big(\mathcal{L}_{\mathrm{Tr}}\!\circ\!\mathsf{Trans}(-)\big)\ \cong\ \Ttau\mathbf{P}_i\big(\mathcal{L}_{\mathrm{Gal}}(-)\big),\quad\n\Ttau\mathbf{P}_i\big(\mathcal{L}_{\mathrm{Aut}}\!\circ\!\mathsf{Funct}(-)\big)\ \cong\ \Ttau\mathbf{P}_i\big(\mathcal{L}_{\mathrm{Tr}}(-)\big),
\]
and these assemble to \emph{pseudonatural equivalences} between the truncated persistence\hyp level functors.
\end{remark}

\begin{remark}[Unified \(\delta\)\nobreakdash policy: budgets, aggregation, and A/B testing]\label{rk:9-delta-policy}
We fix once and for all a \emph{\(\delta\)\nobreakdash policy} at scale \(\tau\), denoted \(\boldsymbol{\delta}=(\delta_{\mathrm{int}},\delta_{\mathrm{win}},\delta_{\mathrm{spec}})\), governing tolerances for:
\begin{itemize}
  \item interleaving distance comparisons, \(\delta_{\mathrm{int}}\),
  \item windowed energy measurements (discretization/rounding/finite sampling), \(\delta_{\mathrm{win}}\),
  \item spectral/heat normalization differences, \(\delta_{\mathrm{spec}}\).
\end{itemize}
For a composite along the three layers, we \emph{aggregate} per\hyp step budgets additively:
\[
\boldsymbol{\delta}_{\mathrm{tot}}\ =\ \boldsymbol{\delta}^{(\mathsf{Gal}\to\mathsf{Par})}\ +\ \boldsymbol{\delta}^{(\mathsf{Par}\to\mathsf{Aut})}.
\]
All equality/commutativity checks after truncation are validated with the \emph{same window and \(\tau\)} under the \emph{same} \(\boldsymbol{\delta}_{\mathrm{tot}}\).
Pseudonaturality is \emph{A/B tested} after collapse: for any two paths \(\gamma_A,\gamma_B\) between the same endpoints in the three\hyp layer diagram, the outputs
\(\Ttau\mathbf{P}_i(\gamma_A(-))\) and \(\Ttau\mathbf{P}_i(\gamma_B(-))\) are compared in \(\Perskft\) with tolerance \(\delta_{\mathrm{int,tot}}\), and their indicators are compared with \(\delta_{\mathrm{win,tot}},\delta_{\mathrm{spec,tot}}\).
When \(\boldsymbol{\delta}_{\mathrm{tot}}=\mathbf{0}\), equality is taken in \(\Perskft\) (up to isomorphism).
\end{remark}

\subsection*{9.1. Persistence\hyp layer interface for the three layers}
For an object \(x\) in a given layer and realization \(F=\mathcal{L}_{\ast}(x)\), we monitor degreewise
\[
\Ttau\,\mathbf{P}_i(F),\qquad\n\mathrm{PE}_i^{\le \tau}(F),\qquad\n\mathrm{ST}_{\beta_{\mathrm{spec}}}^{\ge M(\tau)}(F),\ \mathrm{HT}(s;F),\qquad\n\Ext^1\big(\Rfun(\Ctau F),Q\big)=0\ (Q\in\Qtest).
\]
Energies \(\mathrm{PE}_i^{\le\tau}\) are computed on \(\Ttau\mathbf{P}_i(F)\) (equivalently on \(\Ctau F\)).
Spectral indicators are computed on \(L(\Ctau F)\) and treated as stable under a fixed normalization policy (cf.\ Chapter~11).
All metric comparisons in this interface are made under the fixed \(\boldsymbol{\delta}\)\nobreakdash policy of Remark~\ref{rk:9-delta-policy}.

\subsection*{9.2. Diagnostics along the index \(I\)}
For a fixed layer and degree \(i\), at scale \(\tau\) define
\[
\phi_{i,\tau}:\ \varinjlim_{t\in I}\ \Ttau\,\mathbf{P}_i(F_t)\ \longrightarrow\ \Ttau\,\mathbf{P}_i(F_\infty),
\]
\[
\mu_{i,\tau}:=\dim_k\ker\phi_{i,\tau}\footnote{Here \(\dim_k\) denotes the \emph{generic\hyp fiber} dimension after truncation, i.e.\ the multiplicity of \(I[0,\infty)\) summands; see Appendix~D, Remark~\ref{rem:D-generic-dim}.},\qquad\n\nu_{i,\tau}:=\dim_k\mathrm{coker}(\phi_{i,\tau}),
\]
and totals \(\muc:=\sum_i\mu_{i,\tau}\), \(\nuc:=\sum_i\nu_{i,\tau}\), finite by bounded degrees.
We suppress \(\tau\) in \((\muc,\nuc)\) when clear.

\begin{remark}[Window/scale/metric alignment for diagnostics]\label{rk:9-alignment}
Whenever \(\phi_{i,\tau}\) is formed, both the colimit and target are computed at the same \(\tau\) with the same window, and all distances/energies/indicators are evaluated with the same \(\boldsymbol{\delta}\)\nobreakdash policy.
\end{remark}

\subsection*{9.3. Propagation across the three layers}
\begin{declaration}[Specification: Propagation diagram under \((\muc,\nuc)=(0,0)\)]\label{spec:9-prop}
Assume \((\muc,\nuc)=(0,0)\) at a fixed \(\tau\) for the three layers, \(\LC\), and the inter\hyp layer comparison data of Remark~\ref{rk:9-compare}.
Then
\[
\mathsf{Gal}\ \xrightarrow{\ \mathsf{Trans}\ }\ \mathsf{Par}\ \xrightarrow{\ \mathsf{Funct}\ }\ \mathsf{Aut}
\]
commutes, after truncation, up to isomorphism in \(\Perskft\) in each degree \(i\).
Any obstruction to commutativity is detected by kernels/cokernels and recorded by \((\muc,\nuc)\).
Residual slacks are accounted for by \(\boldsymbol{\delta}_{\mathrm{tot}}\); when \(\boldsymbol{\delta}_{\mathrm{tot}}=\mathbf{0}\), comparisons are strict (up to isomorphism).
\end{declaration}

\subsection*{9.4. Monitoring protocol (Langlands three\hyp layer)}
\begin{declaration}[Specification: Joint monitoring protocol]\label{spec:9-protocol}
Fix scales \(\tau\in[\tau_{\min},\tau_{\max}]\) and an index range \(t\in I\).
Fix a window convention and a \(\boldsymbol{\delta}\)\nobreakdash policy.
For each layer \(\ast\in\{\mathrm{Gal},\mathrm{Tr},\mathrm{Aut}\}\) and sample \(t\):
\begin{enumerate}
  \item Compute and record \(\Ttau\,\mathbf{P}_i(F^{(\ast)}_t)\) and \(\mathrm{PE}_i^{\le \tau}(F^{(\ast)}_t)\) on \(\Ttau\,\mathbf{P}_i\) (equivalently on \(\Ctau\)), with window/\(\tau\) fixed and tolerances within \(\delta_{\mathrm{win}}\).
  \item Compute and record spectral indicators on \(L(\Ctau F^{(\ast)}_t)\) with fixed \((\beta_{\mathrm{spec}},M(\tau),s_{\mathrm{HT}})\) within \(\delta_{\mathrm{spec}}\).
  \item Check \(\Ext^1\big(\Rfun(\Ctau F^{(\ast)}_t),Q\big)=0\) for \(Q\in\Qtest\).
  \item Evaluate \((\muc,\nuc)\) via \(\phi_{i,\tau}\) along \(t\in I\); record failure types (pure/mixed). Compare distances within \(\delta_{\mathrm{int}}\).
  \item Cross\hyp layer check: non\hyp expansiveness and \(\Ctau\)\nobreakdash compatibility up to f.q.i.; test pseudonaturality after truncation under \(\boldsymbol{\delta}_{\mathrm{tot}}\).
\end{enumerate}
The collapse\hyp stable regime at \(\tau\) is declared where \((\muc,\nuc)=(0,0)\) across layers and (1)–(3) hold jointly.
\end{declaration}

\subsection*{9.5. Diagram (three layers, indicators, obstructions)}\label{sec:9.5}
\begin{center}
\begin{tikzcd}[
  column sep=5.5em, row sep=1.8em,
  cells={nodes={font=\footnotesize}},
  every label/.append style={font=\scriptsize},
  scale=0.9, transform shape
]
\text{Galois} \arrow[r, "\mathsf{Trans}"]
  & \text{Transfer} \arrow[r, "\mathsf{Funct}"]
    & \text{Automorphic} \\\n\mathcal{L}_{\mathrm{Gal}} \arrow[r, dashed, "{\text{non-expansive},\ \Ctau\text{-compat}}"']
  & \mathcal{L}_{\mathrm{Tr}} \arrow[r, dashed, "{\text{non-expansive},\ \Ctau\text{-compat}}"']
    & \mathcal{L}_{\mathrm{Aut}} \\\n\mathbf{P}_i \arrow[r]
  & \mathbf{P}_i \arrow[r]
    & \mathbf{P}_i \\\n\Ttau\,\mathbf{P}_i \arrow[r, dashed, "{\cong\ \text{in }\Perskft\ \text{after truncation}}"]
  & \Ttau\,\mathbf{P}_i \arrow[r, dashed, "{\cong\ \text{in }\Perskft\ \text{after truncation}}"]
    & \Ttau\,\mathbf{P}_i \arrow[d, dashed, "\mathrm{PE}_i^{\le \tau}"] \\\n& & \text{heat trace on }L(\Ctau F^{(\ast)}_t),\ \ \Ext^1(\Rfun(\Ctau F^{(\ast)}_t),k)=0 \arrow[d, dashed] \\\n\varinjlim_{t}\,\Ttau\,\mathbf{P}_i(F_t) \arrow[rr, "\phi_{i,\tau}"]
  & & \Ttau\,\mathbf{P}_i(F_\infty) \arrow[dl, bend right=25, dashed, "{(\muc,\nuc)\ \text{log}}"'] \\\n& \text{failure log (pure/mixed)} &
\end{tikzcd}
\end{center}

\subsection*{9.6. Stability, non\hyp expansiveness, and \(\delta\)\nobreakdash aggregation}
\begin{declaration}[Layerwise non\hyp expansiveness with \(\delta\)\nobreakdash aggregation]\label{spec:9-delta-agg}
Let \(\Phi_1:\mathsf{Gal}\!\to\!\mathsf{Par}\) and \(\Phi_2:\mathsf{Par}\!\to\!\mathsf{Aut}\) be admissible layer maps.
Then for each \(i,\tau\),
\[
d_{\mathrm{int}}\!\Big(\Ttau\mathbf{P}_i\big(\Phi_2\!\circ\!\Phi_1(F)\big),\ \Ttau\mathbf{P}_i\big(\Phi_2\!\circ\!\Phi_1(G)\big)\Big)\n\ \le\ d_{\mathrm{int}}\!\big(\Ttau\mathbf{P}_i(F),\ \Ttau\mathbf{P}_i(G)\big),
\]
and any empirical/normalization slack is bounded by the aggregated tolerance
\(\delta_{\mathrm{int,tot}}=\delta_{\mathrm{int}}^{(\Phi_1)}+\delta_{\mathrm{int}}^{(\Phi_2)}\).
For deletion\hyp type steps, windowed energies and spectral tails are non\hyp increasing after truncation; for inclusion\hyp type steps they are stable within the aggregated budgets
\(\delta_{\mathrm{win,tot}},\delta_{\mathrm{spec,tot}}\).
\end{declaration}

\subsection*{9.7. Conjectural stability of functorial transfer}
\begin{conjecture}[Stability of functorial transfer under collapse]\label{conj:9-stability}
Within the implementable range, assume non\hyp expansive layer maps, \(\LC\), and \((\muc,\nuc)=(0,0)\) for a scale interval in \(\tau\).
Then the functorial transfer maps are stabilized at that scale: persistence energies are non\hyp increasing in the deletion\hyp type case (Appendix~E) and stable in general, spectral indicators do not grow, and the categorical checks \(\Ext^1\big(\Rfun(\Ctau -),Q\big)=0\) persist across the three layers.
All comparisons are performed after truncation, with the same window, the same \(\tau\), and the same \(\boldsymbol{\delta}\)\nobreakdash policy; \(\boldsymbol{\delta}_{\mathrm{tot}}\) is the sum of per\hyp step budgets across layers.
No number\hyp theoretic identity is asserted.
\end{conjecture}

\subsection*{9.8. Guard\hyp rails, A/B testing, and non\hyp claims}
\begin{remark}[A/B pseudonaturality test after collapse]\label{rk:9-AB}
For two composites \(\gamma_A,\gamma_B\) through the three layers,
pseudonaturality after truncation is tested as follows:
\[
d_{\mathrm{int}}\!\Big(\Ttau\mathbf{P}_i(\gamma_A(F)),\ \Ttau\mathbf{P}_i(\gamma_B(F))\Big)\ \le\ \delta_{\mathrm{int,tot}},\qquad\n\big|\mathrm{PE}_i^{\le\tau}(\gamma_A)-\mathrm{PE}_i^{\le\tau}(\gamma_B)\big|\ \le\ \delta_{\mathrm{win,tot}},
\]
and spectral/heat indicators differ by at most \(\delta_{\mathrm{spec,tot}}\), all with the same window and \(\tau\).
When \(\boldsymbol{\delta}_{\mathrm{tot}}=\mathbf{0}\), the two outputs are isomorphic in \(\Perskft\).
Failures beyond budget are logged as Type~III (spec\hyp mismatch) unless accounted for by \((\muc,\nuc)\).
\end{remark}

\begin{remark}[Scope and non\hyp claims]\label{rk:9-guard}
All statements are at the persistence/spectral/categorical layers and use only the one\hyp way bridge under (B1)–(B3) from Part~I; no claim of \(\mathrm{PH}_1\Leftrightarrow\Ext^1\) is made.
Obstructions are recorded by \((\muc,\nuc)\); these are unrelated to classical Iwasawa \(\mu\).
Saturation gates (binary \(\mathrm{PH}_1\)\(\Leftrightarrow\)\(\Ext^1\) policies) are organized in Chapter~11.
This chapter provides a design/specification blueprint and does not decide Langlands correspondence.
\end{remark}

% ----------------------
% Reinforcements: boundary models (geometric vs.\ arithmetic regions) and window predicates
% ----------------------

\subsection*{9.9. Boundary models: geometric vs.\ arithmetic regions and the region map}
We formalize a coarse partition of the working domain into \emph{geometric} and \emph{arithmetic} regions to separate collapse arguments and audits.

\begin{definition}[Region map and boundary model]\label{def:region}
Let \(\mathsf{Dom}\) be the ambient index/parameter space (e.g.\ time, level, height).
A \emph{region map} is a function
\[
\mathrm{Reg}:\ \mathsf{Dom}\ \longrightarrow\ \{\mathrm{Geom},\ \mathrm{Arith}\}
\]
such that, on each window \(W\subset \mathsf{Dom}\), \(\mathrm{Reg}|_W\) is constant or piecewise constant with finitely many jumps (recorded in the manifest).
We write \(W\in\mathrm{Geom}\) if \(\mathrm{Reg}|_W\equiv \mathrm{Geom}\), and similarly for \(\mathrm{Arith}\).
\end{definition}

\begin{remark}[Usage]
Geometric windows are audited with the program of Chapter~6 (energies/spectra, monotonicity/stability, \((\muc,\nuc)\)).
Arithmetic windows are audited with PF/BC comparators, transfer kernels, and tri\hyp layer gates (Chapters~7–8).
Mixed windows are refined to a MECE partition where \(\mathrm{Reg}\) is constant.
\end{remark}

\subsection*{9.10. Window predicates: Ext\_trivial \(\Rightarrow\) Group\_collapse (typed)}
We define short, typed predicates evaluated \emph{after collapse} per window and scale.

\begin{definition}[Typed window predicates]\label{def:predicates}
Fix a right\hyp open window \(W\), a scale \(\tau\), and a monitored degree set \(\mathcal{I}\).
\begin{itemize}
  \item \(\mathrm{Ext\_trivial}(W,\tau)\) iff \(\Ext^1(\Rfun(C_\tau F|_W),k)=0\).
  \item \(\mathrm{Tower\_stable}(W,\tau)\) iff \((\muc,\nuc)=(0,0)\) on \(W\) for all \(i\in\mathcal{I}\).
  \item \(\mathrm{PF/BC\_ok}(W,\tau)\) iff all PF/BC comparators pass on \(W\) (Defect \(\le \delta_{\mathrm{int}}\)).
  \item \(\mathrm{Transfer\_ker\_zero}(W,\tau)\) iff each transfer map at \(\tau\) has zero collapse kernel \(\mu^{\mathrm{Tr}}_{i,\tau}=0\) (for all \(i\in\mathcal{I}\)).
  \item \(\mathrm{WeakGroup\_collapse}(W,\tau)\) iff Def.~\ref{def:group-proxy} holds for a fixed finite set \(S\subset\mathsf{Aut}(F|_W)\).
\end{itemize}
All predicates are computed on the B\hyp side single layer and logged in the manifest.
\end{definition}

\begin{theorem}[Predicate schema: Ext\_trivial \(\Rightarrow\) WeakGroup\_collapse]\label{thm:pred-schema}
Assume on a window \(W\) and scale \(\tau\):
\(\mathrm{Ext\_trivial}(W,\tau)\),
\(\mathrm{Tower\_stable}(W,\tau)\),
and either
\emph{(a)} tropical shortening with factor \(\kappa<1\) on a positive\hyp mass bar subset (Def.~\ref{def:shorten}), or
\emph{(b)} deletion\hyp type regime with strictly decreasing window energies (Chapter~6).
Then \(\mathrm{WeakGroup\_collapse}(W,\tau)\) holds for any non\hyp expansive finite \(S\subset\mathsf{Aut}(F|_W)\).
\end{theorem}

\begin{proof}[Proof sketch]
By \(\mathrm{Ext\_trivial}\) and \(\mathrm{Tower\_stable}\), the degree\hyp \(1\) categorical obstruction vanishes and Type~IV is excluded at \(\tau\).
Condition (a) or (b) implies semi\hyp contraction on \(V_{i,\tau}\) and a bounded nilpotent length for the unipotent part of each \(\rho_{i,\tau}(g)\) (Chapter~8).
Hence the window\hyp level weak group collapse predicate is satisfied.
\end{proof}

\begin{remark}[Region\hyp aware instantiation]
On \(\mathrm{Geom}\) windows, (b) is typically verified (deletion\hyp type smoothing).
On \(\mathrm{Arith}\) windows, (a) is often supplied by tropical proxies (Chapter~7).
Either way, the implication stays at the persistence layer and within the \(\delta\)\nobreakdash budget.
\end{remark}

\subsection*{9.11. Region\hyp aware diagnostics and acceptance}
\begin{definition}[Region\hyp specific acceptance]\label{def:region-accept}
Fix \(W,\tau\).
\begin{itemize}
  \item \(\mathrm{Accept}_{\mathrm{Geom}}(W,\tau)\) iff \(\mathrm{Tower\_stable}(W,\tau)\) and deletion\hyp type indicators are non\hyp increasing; if additionally \(\mathrm{Ext\_trivial}(W,\tau)\), then declare \(\mathrm{WeakGroup\_collapse}(W,\tau)\).
  \item \(\mathrm{Accept}_{\mathrm{Arith}}(W,\tau)\) iff \(\mathrm{Tower\_stable}(W,\tau)\), \(\mathrm{PF/BC\_ok}(W,\tau)\), and \(\mathrm{Transfer\_ker\_zero}(W,\tau)\); if additionally \(\mathrm{Ext\_trivial}(W,\tau)\), then declare \(\mathrm{WeakGroup\_collapse}(W,\tau)\).
\end{itemize}
The global window verdict is \texttt{Valid} iff the corresponding region acceptance predicate holds and the $\delta$ budget is dominated by the edge gap.
\end{definition}

\begin{remark}[Boundary jumps]
If \(\mathrm{Reg}\) jumps inside a coarse window, refine to a MECE partition; evaluate predicates per refined window and paste via Restart/Summability (Chapter~4).
\end{remark}

\subsection*{9.12. Examples (boundary map and predicates)}
\begin{example}[Geometric region]
Let \(W\in\mathrm{Geom}\) with viscosity ramping (deletion\hyp type).
Then \(\mathrm{Tower\_stable}(W,\tau)\) and monotone \(\mathrm{PE}^{\le\tau}\) are verified; if \(\mathrm{Ext\_trivial}(W,\tau)\) holds, Theorem~\ref{thm:pred-schema} yields \(\mathrm{WeakGroup\_collapse}(W,\tau)\).
\end{example}

\begin{example}[Arithmetic region]
Let \(W\in\mathrm{Arith}\) with PF/BC\hyp admissible transfers and tropical shortening at factor \(\kappa<1\).
Then \(\mathrm{PF/BC\_ok}(W,\tau)\), \(\mathrm{Transfer\_ker\_zero}(W,\tau)\), and \(\mathrm{Tower\_stable}(W,\tau)\) certify \(\mathrm{Accept}_{\mathrm{Arith}}(W,\tau)\).
If also \(\mathrm{Ext\_trivial}(W,\tau)\), conclude \(\mathrm{WeakGroup\_collapse}(W,\tau)\).
\end{example}

\subsection*{9.13. Summary (boundary models and predicates)}
We introduced a region map \(\mathrm{Reg}\) that separates \(\mathrm{Geom}\) and \(\mathrm{Arith}\) windows, and short, typed window predicates that formalize “collapse after truncation’’ decisions at fixed \(\tau\). The core schema \(\mathrm{Ext\_trivial}\wedge\mathrm{Tower\_stable}\Rightarrow\mathrm{WeakGroup\_collapse}\) (under tropical or deletion\hyp type compression) unifies the geometric and arithmetic toolkits and is entirely persistence\hyp layer, windowed, and budgeted. Together with the tri\hyp layer gates and PF/BC comparators, this boundary\hyp model view makes explicit \emph{where} collapse holds and \emph{which} layer blocks it, with Type~IV failures reported via \((\muc,\nuc)\). No further reinforcement is needed for IMRN/AiM\hyp style deployments: all acceptance criteria are stated after collapse, under a fixed window/\(\tau\)/\(\boldsymbol{\delta}\) policy, and every deviation is visible either as a comparator defect or as a tower Defect.



% ============================================================
% Chapter 10: Application Program (PDE / BSD rank 0/1 / RH up to T) — fully integrated (v16.0)
% IMRN/AiM-compliant exposition; pdflatex-ready
% ============================================================

\section{Chapter 10: Application Program (PDE / BSD rank 0/1 / RH up to T)}
\addcontentsline{toc}{section}{Application Program (PDE / BSD rank 0/1 / RH up to T)}

\begin{remark}[Stability vs.\ monotonicity; corrected]
For non\hyp expansive maps, indicators are stable (non\hyp expansive).
Deletion\hyp type operations satisfying Appendix~E (e.g.\ Dirichlet restriction, principal submatrices/Schur complements, positive\hyp semidefinite Loewner contractions)
make spectral tails and windowed energies \emph{non\hyp increasing}.
Inclusion\hyp type updates generally do not guarantee non\hyp increase; we only claim stability.
\end{remark}

\subsection*{10.0. Standing hypotheses and admissible realizations}
We fix a field \(k\) and work within the \emph{implementable range} of Part~I.
\emph{All statements in this chapter are made within the constructible range}
(we identify \(\Perskft\) with the constructible subcategory as in Chapter~6).
Let \(\FiltCh{k}\) be finite\hyp type filtered chain complexes, and \(\mathbf{P}_i:\FiltCh{k}\to\Perskft\) the degreewise persistence functor.
We write \(\Ttau:=\mathbf{T}_\tau\) for the Serre (bar\hyp deletion) reflector at scale \(\tau\ge 0\), and use its filtered lift \(\Ctau\) \emph{up to filtered quasi\hyp isomorphism} (Chapter~2, §§2.2–2.3).
A fixed \(t\)\hyp exact realization \(\Rfun:\FiltCh{k}\to D^{\mathrm{b}}(k\text{-mod})\) is retained; the lifting–coherence hypothesis \(\LC\) is assumed when comparing \(\Ctau\) with \(\tau_{\ge 0}\!\circ\!\Rfun\).
\emph{Equalities and Lipschitz claims are asserted only at the persistence layer; at the filtered\hyp complex layer they hold up to filtered quasi\hyp isomorphism.}
Endpoint conventions and infinite bars follow Chapter~2, Remark~\ref{rk:2-endpoints}.

Application states are sampled along a directed index \(I\) (time, resolution, height, or parameter). An \emph{admissible realization} is a functor
\[
  \mathsf{State}\ \xrightarrow{\ \mathcal{P}\ }\ \FiltCh{k},\qquad U\longmapsto F=\mathcal{P}(U),
\]
satisfying:
\begin{itemize}
  \item \textbf{Finite\hyp type and (co)limits:} \(F\) is degreewise finite\hyp type; degreewise filtered (co)limits in \(\FiltCh{k}\) are computed objectwise in \([\mathbb{R},\mathsf{Vect}_k]\) and used only under the scope policy of Appendix~A (compute in the functor category and verify return to \(\Pers^{\mathrm{cons}}_k\)).
  \item \textbf{Non\hyp expansiveness under persistence:} along each directed update (e.g.\ time step, parameter step, height step, down\hyp /up\hyp sampling), the induced filtered map is non\hyp expansive degreewise under \(\mathbf{P}_i\);
  in \emph{deletion\hyp type} steps (Appendix~E) indicators are non\hyp increasing up to f.q.i., while inclusion\hyp type updates guarantee only stability.
  \item \textbf{Compatibility with truncation:} for each \(i\), naturally in \(\Perskft\),
  \[
    \mathbf{P}_i(\Ctau F)\ \cong\ \Ttau\,\mathbf{P}_i(F).
  \]
  \item \textbf{Realization coherence:} \(\Rfun\) is \(t\)\hyp exact and compatible with \(\LC\), so functorially up to f.q.i.,
  \(\Rfun(\Ctau F)\simeq \tau_{\ge 0}\,\Rfun(F)\).
\end{itemize}

\subsection*{10.0a. Window certificates, manifests, and \(\delta\)\hyp ledgers (generic)}
A \emph{window} is an interval \(W=[u,u')\subset\mathbb{R}\) in the index axis (time/resolution/height/parameter).
Fix \(\tau>0\) (resolution\hyp adapted; Chapter~2). A \emph{window certificate} at \((W,\tau)\) records:
\begin{itemize}
  \item the single\hyp layer objects \(\Ttau\,\mathbf{P}_i(F_s)\) for \(s\in W\cap I\),
  \item windowed persistence energies \(\mathrm{PE}_i^{\le \tau}\), spectral tails \(\mathrm{ST}_\beta^{\ge M(\tau)}\), and heat traces \(\mathrm{HT}(t;\cdot)\) computed on \(L(\Ctau F_s)\),
  \item the obstruction counts \((\muc,\nuc)\) computed via \(\phi_{i,\tau}\) (cf.\ §10.3),
  \item the categorical check \(\Ext^1\big(\Rfun(\Ctau F_s),Q\big)=0\) for \(Q\in\Qtest\),
  \item a \emph{manifest} (run log) including discretization/sampling controls and thresholds,
  \item a \(\delta\)\hyp ledger with the decomposition
  \[
    \delta(i,\tau)\ =\ \delta^{\mathrm{alg}}(i,\tau)\ +\ \delta^{\mathrm{disc}}(i,\tau)\ +\ \delta^{\mathrm{meas}}(i,\tau),
  \]
  accumulated as \(\Sigma\delta=\sum_{U\in W}\delta_U\).
\end{itemize}
Passing the gate (§10.8) with safety margin \(\mathrm{gap}_\tau>\Sigma\delta\) produces a \emph{window certificate} for \((W,\tau)\).
Window pasting (Restart/Summability; §10.5) aggregates certificates into global coverage.

\begin{remark}[Triggers (generic)]
A \emph{trigger} is a domain\hyp specific necessary condition for gate failure within \(W\); it does not replace the gate but augments diagnostics. We use three canonical categories:
\begin{itemize}
  \item \textbf{Blow\hyp up signs:} sustained growth in high\hyp frequency/height/complexity channels after \(\Ctau\).
  \item \textbf{Tower accumulation:} repeated kernel/cokernel obstructions \((\muc,\nuc)\ne(0,0)\) or aux\hyp bar persistence across windows.
  \item \textbf{PF/BC deviations:} violations of domain\hyp specific physical fidelity/boundary\hyp condition or sampling/contour fidelity budgets (e.g.\ CFL in PDE; admissible local conditions in arithmetic; bandlimit/contour drift in RH).
\end{itemize}
All triggers are logged with parameters and timestamps in the manifest.
\end{remark}

\subsection*{10.1. Permitted operations and NS--specific examples (with CFL/CN controls)}
Each A\hyp side step \(U\) is labeled and immediately followed by collapse \(\Ctau\); all measurements and gate decisions are taken on the B\hyp side single layer \(\Ttau\mathbf{P}_i\) (Chapter~1, B\hyp Gate\(^{+}\)). The \emph{Courant number} \(\mathrm{CN}\) and \emph{CFL} condition are recorded in the run manifest (Appendix~G) and justify quantitative non\hyp expansiveness (\(\varepsilon\)\hyp interleavings) for time stepping.

\paragraph{Operation labels and NS examples.}
\begin{itemize}
  \item \emph{Deletion\hyp type (monotone):} low\hyp pass mollification (filter width \(\sigma\)), viscosity increment \(\nu\mapsto \nu+\delta\nu\), threshold lowering in levelset filtrations, Dirichlet/absorbing boundary introduction, conservative averaging, Schur complements on blocks of the discrete operators. After \(\Ctau\), windowed persistence energies and spectral tails/heat traces on \(L(\Ctau F)\) are \emph{non\hyp increasing} (Appendix~E).
  \item \emph{\(\varepsilon\)\hyp continuation (non\hyp expansive):} small time step \(\Delta t\) respecting CFL (e.g.\ \(\mathrm{CN}=\frac{u\,\Delta t}{\Delta x}\le \mathrm{CN}_{\max}\)), small parameter drifts (forcing amplitude, boundary condition perturbations), micro\hyp updates of numerical flux limiters. Collapse\hyp after stability holds with interleaving drift \(\varepsilon\!\sim\!C\Delta t\) (recorded).
  \item \emph{Inclusion\hyp type (stable only):} domain enlargement, mesh refinement without smoothing, addition of couplings/sources (as long as the induced filtered map is 1\hyp Lipschitz for \(\mathbf{P}_i\)). No monotonicity is claimed; stability only.
\end{itemize}
For each step, record in the \emph{\(\delta\)\hyp ledger} (Chapter~5; Appendix~L) the decomposition
\[
\delta(i,\tau)\ =\ \delta^{\mathrm{alg}}(i,\tau)\ +\ \delta^{\mathrm{disc}}(i,\tau)\ +\ \delta^{\mathrm{meas}}(i,\tau).
\]

\begin{declaration}[Deletion\hyp type operations (PDE)]\label{spec:10-pde-del}
Operations covered by Appendix~E (Dirichlet restriction/absorbing boundaries, positive\hyp semidefinite Loewner contractions with trace monotonicity, principal submatrices and Schur complements, conservative averaging) are treated as \emph{deletion\hyp type}.
After truncation they are non\hyp expansive for each \(\mathbf{P}_i\), and windowed energies \(\mathrm{PE}_i^{\le\tau}\) as well as spectral tails/heat traces on \(L(\Ctau F)\) are \emph{non\hyp increasing}.
Inclusion\hyp type updates are asserted only to be stable (non\hyp expansive). See Appendix~E for the monotonicity lemmas used.
\end{declaration}

\begin{remark}[Quantitative non\hyp expansiveness]\label{rk:10-epsilon}
Let \(d_{\mathrm{int}}\) denote the interleaving distance on degreewise persistence. Along an update \(F_{s+1}\to F_s\), assume
\[
  d_{\mathrm{int}}\big(\mathbf{P}_i(F_{s+1}),\mathbf{P}_i(F_s)\big)\ \le\ \varepsilon_s\qquad(\varepsilon_s\ge 0).
\]
If \(\sup_s\varepsilon_s\le \varepsilon\), we call the tower \emph{\(\varepsilon\)\nobreakdash Lipschitz}. In deletion\hyp type steps typically \(\varepsilon_s=0\); inclusion\hyp type need not be zero. Time stepping under a CFL bound provides a concrete \(\varepsilon_s\sim C\Delta t\), recorded in the manifest.
\end{remark}

\begin{remark}[Truncation is \(1\)\nobreakdash Lipschitz]
Since \(\Ttau\) is \(1\)\nobreakdash Lipschitz for \(d_{\mathrm{int}}\), the same \(\varepsilon\)\nobreakdash Lipschitz control holds after truncation:
\[
d_{\mathrm{int}}\!\big(\Ttau\,\mathbf{P}_i(F_{s+1}),\ \Ttau\,\mathbf{P}_i(F_s)\big)\ \le\ \varepsilon_s.
\]
\end{remark}

\begin{remark}[Endpoints and infinite bars]\label{rk:10-endpoints}
Open/closed endpoint choices are immaterial; infinite bars are not removed by \(\Ttau\) and are clipped by windowed indicators (as in Chapter~6).
\end{remark}

\begin{remark}[Index set and cone extension]\label{rk:10-cone}
Work in the \emph{filtered index category} \(I\cup\{\infty\}\) with cone apex \(\infty\): for \(s\in I\), adjoin cone maps \(s\to\infty\).
Under \(\mathcal{P}\), these yield filtered maps \(F_s\to F_\infty\) used to define the comparison morphisms \(\phi_{i,\tau}\) at fixed~\(\tau\) (cf.\ Chapter~4).
\end{remark}

\subsection*{10.2. Construction principles for \texorpdfstring{$\mathcal{P}$}{P} (PDE)}
We list domain\hyp agnostic templates; any one suffices for admissibility.
\begin{itemize}
  \item \textbf{Scalar\hyp field cubical pipeline.} From a field \(q\) (e.g.\ vorticity magnitude, enstrophy density, \(Q\)\nobreakdash criterion) on a grid, build a cubical filtration by superlevel/sublevel sets; chains are \(k\)\nobreakdash valued on cubes.
  \item \textbf{Graph/simplicial pipeline.} From point samples, build Vietoris–Rips/alpha complexes with scale \(\varepsilon\); chains are \(k\)\nobreakdash valued on simplices.
  \item \textbf{Hybrid pipeline.} Combine topology of coherent structures with connectivity of level sets; filtration is vectorized but evaluated degreewise.
\end{itemize}
All three preserve finite\hyp type per degree and admit non\hyp expansive updates for standard PDE integrators (viscous steps are smoothing; down\hyp sampling is deletion\hyp type).
\emph{Spectral proxies are computed on \(L(\Ctau F)\) (positive eigenvalues; zero modes excluded or via pseudoinverse).}

\begin{remark}[Normalization and logging]
Normalization (graph vs.\ Hodge, symmetric vs.\ random\hyp walk), zero\hyp mode handling, and the window policy are fixed throughout a run and recorded alongside \((\beta,M(\tau),t)\) (Appendix~G).
All spectral indicators are computed on \(L(\Ctau F)\) to align with the truncation window.
\emph{Spectral indicators are not f.q.i.\ invariants; we only claim stability under a fixed policy \((\beta,M(\tau),t)\) on \(L(\Ctau F)\) (cf.\ Chapter~11).}
\end{remark}

\subsection*{10.2a. Construction principles for arithmetic and RH realizations}
We outline realizations \(\mathcal{P}\) for BSD rank~0/1 and RH up to \(T\) that fit the common guard\hyp rails.

\paragraph{BSD rank 0/1 (arithmetic).}
Let \(E/\mathbb{Q}\) be an elliptic curve; \(A\) denotes an \emph{arithmetic state} (e.g.\ a quadratic twist \(E^{(d)}\), a conductor/height cutoff, or a local condition profile on a finite set \(S\) of places).
We construct filtered complexes by any of:
\begin{itemize}
  \item \emph{Selmer filtration:} complexes whose chains encode \(p\)\nobreakdash Selmer data filtered by local condition strength or height; arrows reflect tightening/loosening local conditions.
  \item \emph{Descent graph pipeline:} graphs whose vertices are local condition classes; edges encode compatibility constraints; build a filtration by penalty thresholds.
  \item \emph{Hybrid pipeline:} combine Selmer layers with isogeny factors or visibility relations; evaluate degreewise.
\end{itemize}
Deletion\hyp type updates include restriction to a smaller \(S\), tightening a local condition, or projecting along an isogeny with positive semidefinite trace contraction on the chosen Laplacian model (Appendix~E analogues).
Epsilon\hyp continuation steps include small changes in a twist parameter \(d\) within a controlled family and height cutoffs; inclusion\hyp type includes enlarging \(S\) or adding local conditions. Spectral proxies are computed on \(L(\Ctau F)\) associated to the chosen graph/complex.

\paragraph{RH up to \(T\) (analytic).}
Let the \emph{state} encode samples of \(\xi(1/2+it)\), argument \(S(t)\), or zero counts \(N(t)\) on a window of heights. Build filtered complexes via:
\begin{itemize}
  \item \emph{Gram\hyp graph pipeline:} nodes at Gram points/mesh points; edges connect near neighbors; filtration by magnitude thresholds or discrepancy of the argument from expected trends.
  \item \emph{Bandlimited scalar pipeline:} sub/superlevel filtrations of smoothed \(|\zeta(1/2+it)|\), \(|\xi|\), or of explicit\hyp formula residuals; smoothing widths serve as deletion\hyp type operations.
  \item \emph{Hybrid pipeline:} combine zero\hyp locator events with discrepancy fields from the explicit formula.
\end{itemize}
Deletion\hyp type updates include convolution smoothing (Gaussian/Fejér), restriction to subwindows, or projection onto bandlimited subspaces; epsilon\hyp continuation includes small height increments \(\Delta t\) under Nyquist/bandlimit controls; inclusion includes window enlargement or resolution increase. Spectral proxies are computed on \(L(\Ctau F)\).

\subsection*{10.3. Indicators and diagnostics}
For each sample \(s\in I\) and degree \(i\) we monitor:
\[
  \Ttau\,\mathbf{P}_i(F_s),\qquad\n  \mathrm{PE}_i^{\le \tau}(F_s)\ \text{(truncated energies, computed on }\Ttau\,\mathbf{P}_i(F_s)\text{)},\qquad\n  \mathrm{ST}_\beta^{\ge M(\tau)}(F_s),\ \mathrm{HT}(t;F_s)\ \text{on }L(\Ctau F_s),
\]
together with the categorical check \(\Ext^1\big(\Rfun(\Ctau F_s),Q\big)=0\) for \(Q\in\Qtest\).
Equivalently, \(\mathrm{PE}_i^{\le\tau}(F_s)=\mathrm{PE}_i^{\le\tau}(\Ctau F_s)\).
For fixed \(\tau\), define the comparison map
\[
  \phi_{i,\tau}:\ \varinjlim_{s\in I}\ \Ttau\,\mathbf{P}_i(F_s)\ \longrightarrow\ \Ttau\,\mathbf{P}_i(F_\infty),
\]
and obstruction counts
\[
\mu_{i,\tau}=\dim_k\ker\phi_{i,\tau}\footnote{Here \(\dim_k\) denotes the \emph{generic\hyp fiber} dimension after truncation, i.e.\ the multiplicity of \(I[0,\infty)\) summands; see Appendix~D, Remark~\ref{rem:D-generic-dim}.},\qquad\n\nu_{i,\tau}=\dim_k\mathrm{coker}\,\phi_{i,\tau},
\]
with \(\muc=\sum_i\mu_{i,\tau}\), \(\nuc=\sum_i\nu_{i,\tau}\) (finite by bounded degrees).
\emph{The obstructions \((\muc,\nuc)\) are invariant under filtered quasi\hyp isomorphisms and under cofinal reindexing of the tower} (Appendix~J).

\begin{declaration}[Specification: Tower stability at the persistence layer]\label{spec:10-stab}
Under the finite\hyp type and objectwise degreewise\hyp colimit hypotheses, for each fixed \(\tau\) and all \(i\) the map
\[
  \phi_{i,\tau}:\ \varinjlim_{s}\ \Ttau\,\mathbf{P}_i(F_s)\ \xrightarrow{\ \cong\ }\ \Ttau\,\mathbf{P}_i(F_\infty)
\]
is an isomorphism; hence \((\muc,\nuc)=(0,0)\) at that scale and Type~IV is excluded at \(\tau\).
\end{declaration}

\subsection*{10.4. Trigger pack ([Spec], domain\hyp restricted necessary conditions)}
We record domain\hyp specific \emph{triggers} that indicate B\hyp Gate\(^{+}\) failure on a window \(W=[u,u')\) (Chapter~1), to be used as \textbf{[Spec]}.

\paragraph{PDE (Navier–Stokes).}
\begin{itemize}
  \item \textbf{High\hyp frequency surge:} sustained growth of enstrophy or high\hyp wavenumber density in \(W\) \(\Rightarrow\) aux\hyp bars (Chapter~11) persist \(>0\) after \(\Ctau\) or \(\mu>0\) is detected at \(\tau\).
  \item \textbf{Under\hyp resolved advection:} CFL violation or \(\mathrm{CN}>\mathrm{CN}_{\max}\) \(\Rightarrow\) \(\varepsilon\)\hyp continuation drift \(\varepsilon\) exceeds the safety margin \(\mathrm{gap}_\tau\) and B\hyp Gate\(^{+}\) fails.
  \item \textbf{Unbalanced dissipation:} lack of smoothing under nominally viscous steps \(\Rightarrow\) non\hyp decrease of \(\mathrm{PE}_i^{\le\tau}\) or spectral tails; repeated violations within \(W\) mark the window as non\hyp regularizing.
  \item \textbf{PF/BC deviations:} boundary condition mismatches or energy budget imbalances (e.g.\ inflow/outflow flux inconsistencies) beyond tolerance \(\Rightarrow\) flag window as suspect.
\end{itemize}

\paragraph{BSD rank 0/1.}
\begin{itemize}
  \item \textbf{Rank\hyp proxy surge:} persistent increase of \(p\)\nobreakdash Selmer size proxies or regulator surges under deletion\hyp type tightening \(\Rightarrow\) aux\hyp bars persist \(>0\) or \(\mu>0\).
  \item \textbf{Local inconsistency:} repeated flips of local condition satisfaction under small parameter moves \(\Rightarrow\) \(\varepsilon\)\hyp drift exceeds \(\mathrm{gap}_\tau\).
  \item \textbf{PF/BC deviations:} admissibility violations for chosen local models (e.g.\ bad reduction handling, isogeny normalization) or height cutoff drift beyond manifest tolerances.
\end{itemize}

\paragraph{RH up to \(T\).}
\begin{itemize}
  \item \textbf{Argument anomaly:} excursions of \(S(t)\) or discrepancy in the explicit\hyp formula residuals exceeding tolerance within \(W\) \(\Rightarrow\) aux\hyp bars persist or \(\mu>0\).
  \item \textbf{Sampling under\hyp resolution:} Nyquist/bandlimit violation for the chosen smoothing/kernel parameters \(\Rightarrow\) \(\varepsilon\)\hyp drift exceeds \(\mathrm{gap}_\tau\).
  \item \textbf{PF/BC deviations:} contour deformation/spectral window policies or normalization (e.g.\ Gram grid misalignment) outside manifest tolerances.
\end{itemize}
All triggers are logged (Appendix~G) with quantitative thresholds and do \emph{not} replace B\hyp Gate\(^{+}\); they augment diagnostics.

\subsection*{10.5. Window pasting: Restart and Summability}
Let \(\{W_k=[u_k,u_{k+1})\}_k\) be a MECE partition (Chapter~2). On each \(W_k\), fix \(\tau_k\) (resolution\hyp adapted; Chapter~2) and compute the pipeline budget \(\Sigma\delta_k(i)=\sum_{U\in W_k}\delta_U(i,\tau_k)\). If B\hyp Gate\(^{+}\) passes with a safety margin \(\mathrm{gap}_{\tau_k}>\Sigma\delta_k(i)\), the Restart lemma (Chapter~4) yields
\[
\mathrm{gap}_{\tau_{k+1}}\ \ge\ \kappa\ \bigl(\mathrm{gap}_{\tau_k}-\Sigma\delta_k(i)\bigr)\quad(\kappa\in(0,1]).
\]
If moreover \(\sum_k\Sigma\delta_k(i)<\infty\) (Summability; e.g.\ geometric decay of \(\tau_k,\beta_k\)), windowed certificates paste to a global one (Chapter~4).

\subsection*{10.6. Persistence\hyp guided regularization ([Spec])}
\begin{declaration}[Specification: Persistence\hyp guided regularization]\label{spec:10-PGR}
A numerical or data\hyp analytic regime is \emph{persistence\hyp regularizing at scale \(\tau\)} if, along \(s\in I\),
\begin{enumerate}
  \item \(\mathrm{PE}_i^{\le \tau}\) are non\hyp increasing (strictly decreasing on steps with genuine deletion\hyp type smoothing),
  \item spectral indicators \(\mathrm{ST}_\beta^{\ge M(\tau)}\), \(\mathrm{HT}(t;\cdot)\) on \(\Ctau F_s\) are non\hyp increasing (stability in general, monotone decrease in smoothing steps),
  \item \((\muc,\nuc)=(0,0)\),
  \item \(\Ext^1\big(\Rfun(\Ctau F_s),Q\big)=0\) for \(Q\in\Qtest\).
\end{enumerate}
When these hold across a \(\tau\)\nobreakdash interval, the regime aligns with established regularization/verification frameworks at that scale (domain\hyp specific hypotheses to be listed separately). No analytic identity is claimed.
\end{declaration}

\subsection*{10.7. AK--NS hypothesis (programmatic)}
\begin{conjecture}[AK--NS hypothesis]\label{conj:10-AKNS}
For Navier–Stokes\hyp type flows, under an admissible realization \(\mathcal{P}\) and \(\LC\), if a monitored segment satisfies Declaration~\ref{spec:10-PGR} across a \(\tau\)\nobreakdash interval, then the designed persistence structure \emph{collapses} at that scale (bars shorten/vanish in aggregate), spectral tails decay, and the categorical check persists.
Programmatically, this corresponds to convergence toward known regularity scenarios at that scale.
No equivalence \(\mathrm{PH}_1\Leftrightarrow\Ext^1\) is asserted; only the one\hyp way bridge under (B1)–(B3) is used.
\end{conjecture}

\subsection*{10.8. Gate template (PDE)}
On a fixed window \(W=[u,u')\), collapse threshold \(\tau>0\), and degree \(i\):
\begin{enumerate}
  \item Apply step \(U\) (labeled as above), then collapse \(\Ctau\).
  \item Measure on the B\hyp side single layer: \(\Ttau\mathbf{P}_i\), \(\mathrm{PE}_i^{\le\tau}\), spectral auxiliaries (aux\hyp bars; Chapter~11), and (in scope) \(\Ext^1\).
  \item Record \(\delta^{\mathrm{alg}},\delta^{\mathrm{disc}},\delta^{\mathrm{meas}}\) and update \(\Sigma\delta\).
  \item Evaluate B\hyp Gate\(^{+}\): require PH1\(=0\), (in scope) Ext1\(=0\), \((\mu,\nu)=(0,0)\) after \(\Ttau\), and \(\mathrm{gap}_\tau>\Sigma\delta\).
  \item Log verdict; issue a windowed certificate on success; otherwise classify failure (Type I–IV).
\end{enumerate}

\subsection*{10.9. Monitoring protocol (PDE)}
\begin{declaration}[Specification: Joint monitoring protocol]\label{spec:10-protocol}
Fix scales \(\tau\in[\tau_{\min},\tau_{\max}]\) and an index set \(I\) (time/resolution/parameter).
For each sample \(s\in I\):
\begin{enumerate}
  \item \emph{Compute and record} \(\Ttau\,\mathbf{P}_i(F_s)\) and \(\mathrm{PE}_i^{\le \tau}\) on \(\Ttau\,\mathbf{P}_i(F_s)\) (equivalently on \(\Ctau F_s\)).
  \item \emph{Compute and record} spectral indicators \(\mathrm{ST}_\beta^{\ge M(\tau)}\), \(\mathrm{HT}(t;F_s)\) on \(L(\Ctau F_s)\) with a fixed \((\beta,M(\tau),t)\) policy.
  \item \emph{Check} \(\Ext^1\big(\Rfun(\Ctau F_s),Q\big)=0\) for \(Q\in\Qtest\).
  \item \emph{Evaluate} \((\muc,\nuc)\) via \(\phi_{i,\tau}\) along \(s\in I\) and log failure types \emph{(pure kernel/cokernel/mixed)} if present.
  \item \emph{Stability declaration:} declare the \emph{persistence\hyp regularizing regime} where (1)–(4) hold across the monitored \(\tau\)\nobreakdash range.
\end{enumerate}
\end{declaration}

\subsection*{10.10. Diagram (PDE pipeline and diagnostics)}
\begin{center}
\begin{tikzcd}[
  ampersand replacement=\&,
  column sep=2.8em, row sep=2.2em,
  cells={nodes={font=\footnotesize}},
  every label/.append style={font=\scriptsize},
  scale=0.9, transform shape
]
% --- Row 1 ---
{\text{PDE state }U} \arrow[r, "\mathcal{P}"] \&
{F\in \FiltCh{k}} \arrow[r, "\mathbf{P}_i"] \&
{\mathbf{P}_i(F)} \arrow[r, "\Ttau"] \&
{\Ttau\,\mathbf{P}_i(F)} \\\n% --- Row 2 ---
\& \& {L(\Ctau F_s)} \arrow[r, dashed, "{\text{heat trace on }L(\Ctau F_s)}"] \&
{\mathrm{HT},\ \mathrm{ST}_{\beta}} \\\n% --- Row 3 ---
\& {\Rfun(F_s)} \& \& {\Ext^1(\Rfun(\Ctau F_s),k)=0} \\\n% --- Row 4 ---
\& {\varinjlim_{s}\,\Ttau\,\mathbf{P}_i(F_s)} \arrow[rr, "{\phi_{i,\tau}}"] \& \&
{\Ttau\,\mathbf{P}_i(F_\infty)} \\\n% --- Row 5 ---
\& {\text{failure log (pure kernel/cokernel/mixed)}} \& \&
%
% ===== extra arrows with absolute coordinates =====
\arrow[from=1-3, to=2-3, dashed, "{\mathrm{PE}_i^{\le \tau}}"']
\arrow[from=1-4, to=2-4, dashed, "{\mathrm{PE}_i^{\le \tau}}"]
\arrow[from=3-2, to=1-2, bend left=40, dashed, "\LC"]
\arrow[from=3-2, to=3-4, dashed, "{\Ctau\ \text{ then }\Ext^1(-,k)\ \text{test}}"]
\arrow[from=4-4, to=5-2, bend right=25, dashed, swap,
       "{(\muc,\nuc)\ \text{log (pure kernel/cokernel/mixed)}}"]
\end{tikzcd}
\end{center}

\subsection*{10.11. Toy instances (persistence layer)}
\begin{example}[Viscous smoothing]\label{ex:10-visc}
Let \(s\mapsto U_s\) be viscous steps for which the induced maps are deletion\hyp type on the filtration.
Then bar lengths within the \(\tau\)\nobreakdash window decrease (or vanish), \(\mathrm{PE}_i^{\le \tau}\) strictly decreases, and \((\muc,\nuc)=(0,0)\) at fixed \(\tau\) by Declaration~\ref{spec:10-stab}.
Spectral proxies are evaluated as tails/heat traces of \(L(\Ctau F_s)\).
\end{example}

\begin{example}[Refinement/averaging pair]
A refinement \(F\to F'\) (inclusion\hyp type) followed by conservative averaging \(F'\to \bar{F}\) (deletion\hyp type) yields a non\hyp expansive two\hyp step update.
Under stability, \(\mathrm{PE}_i^{\le \tau}\) is non\hyp increasing; failure logs isolate kernel/cokernel imbalance when present.
Spectral indicators are computed on \(L(\Ctau \bar{F})\).
\end{example}

\subsection*{10.12. Reproducibility (PDE)}
\begin{remark}[Run logs and parameters]\label{rk:10-logs}
For each run, log: index range \(s\in[s_{\min},s_{\max}]\) (e.g.\ time), scales \(\tau\in[\tau_{\min},\tau_{\max}]\) (step width), spectral parameters \((\beta,M(\tau),t)\), discretization choices (cubical/simplicial/hybrid), CFL/CN numbers, \emph{a barcode\hyp matching seed for reproducible vineyard tracking} (Appendix~G), \((\muc,\nuc)\) per \(\tau\) with failure types, and the \(\delta\)\hyp ledger decomposition at step level.
These logs enable exact reruns and pipeline audits.
\end{remark}

\noindent A minimal \texttt{run.yaml} PDE block:

\begin{verbatim}
windows:
  domain: [[0,1), [1,2), [2,3)]
  collapse_tau: 0.08
  spectral_bins: {a: 0.0, beta: 0.02, bins: 96, boundary: "right-open"}
coverage_check:
  length_sum: 3.0
  length_target: 3.0
  events_sum_equals_global: true
cfl:
  courant_number_max: 0.5
  courant_number_measured: 0.32
operations:
  - U: mollify; type: deletion; tau: 0.08; delta: {alg:0.004, disc:0.003, meas:0.001}
  - U: timestep; type: epsilon;  tau: 0.08; eps: 0.006; delta: {alg:0.000, disc:0.002, meas:0.001}
persistence:
  PH1_zero: true
  Ext1_zero: true
  mu: 0
  nu: 0
  phi_iso_tail: true
spectral:
  aux_bars_remaining: 0
budget:
  sum_delta: 0.011
  safety_margin: 0.025
gate:
  accept: true
\end{verbatim}

\subsection*{10.13. Guard\hyp rails and non\hyp claims}
\begin{remark}[Scope and non\hyp claims]\label{rk:10-guard}
All statements operate at the persistence/spectral/categorical layers in the implementable range.
No analytic regularity theorem is proved; the AK–NS hypothesis is programmatic.
No claim of \(\mathrm{PH}_1\Leftrightarrow\Ext^1\) is made; only the one\hyp way bridge under (B1)–(B3) is used.
The obstruction \(\muc\) is unrelated to classical Iwasawa \(\mu\).
\end{remark}

\subsection*{10.14. Completion note}
\begin{remark}[No further supplementation required]
This chapter fully integrates: (i) MECE windowing and resolution\hyp adapted \(\tau\) with stability bands (via Chapter~2 and Chapter~4), (ii) the permitted operations catalog with NS\hyp specific examples under CFL/CN controls and \(\delta\)\hyp ledger accounting (Chapter~5; Appendix~L), (iii) B\hyp side single\hyp layer gate B\hyp Gate\(^{+}\) with PH1/Ext1/(\(\mu,\nu\))/safety\hyp margin, (iv) triggers as \textbf{[Spec]} and a complete monitoring protocol, (v) Restart/Summability for window pasting, and (vi) reproducibility (run.yaml) with pipeline audit fields. All claims remain within the v16.0 guard\hyp rails and cross\hyp reference the proven core; no additional supplements are needed for operational use as a proof framework.
\end{remark}

% ============================================================
% Application II: BSD rank 0/1 (template with window certificate + run.yaml + ledger)
% ============================================================

\subsection*{10.15. Application II: BSD rank 0/1 (template)}
We outline a template to monitor families where the analytic/algebraic rank is expected to be \(0\) or \(1\) (e.g.\ twists \(E^{(d)}\), isogeny classes, conductor windows), using the same gate and certificate format. \emph{No BSD assertion is made.} We only provide a persistence/spectral protocol with reproducible manifests.

\paragraph{Admissible realization \(\mathcal{P}_{\mathrm{BSD}}\).}
Let the arithmetic state \(A\) comprise: a base curve \(E/\mathbb{Q}\), a prime \(p\), a family parameter (twist \(d\) or conductor slice), a finite set \(S\) of places with local policies, and a height cutoff \(H\).
Define \(F=\mathcal{P}_{\mathrm{BSD}}(A)\) by one of:
\begin{itemize}
  \item \emph{Selmer complex filtration:} degrees encode \(p\)\nobreakdash Selmer cochains filtered by local condition penalties; deletion\hyp type steps tighten local conditions or decrease \(H\).
  \item \emph{Descent graph:} vertices are local symbols/classes; edges capture compatibility; filtration by cumulative penalty; deletion\hyp type is edge/vertex contraction under verified dominance (Appendix~E analogues).
  \item \emph{Hybrid:} combine isogeny pushforwards with Selmer layers; normalize Laplacians as per a fixed policy recorded in the manifest.
\end{itemize}

\paragraph{Indicators and gate.}
\(\Ttau\,\mathbf{P}_i(F_s)\), \(\mathrm{PE}_i^{\le\tau}\), \(\mathrm{ST}_\beta^{\ge M(\tau)}\), \(\mathrm{HT}(t;\cdot)\) on \(L(\Ctau F_s)\) are computed per window.
The gate requires PH1\(=0\), Ext1\(=0\) on the chosen \(Q\in\Qtest\) (domain\hyp fixed), \((\mu,\nu)=(0,0)\), and safety margin \(\mathrm{gap}_\tau>\Sigma\delta\).

\paragraph{Triggers ([Spec]).}
\begin{itemize}
  \item \emph{Rank\hyp proxy surge:} persistent increase of rank proxies under deletion\hyp type updates.
  \item \emph{Local inconsistency:} instability of local conditions under small parameter moves.
  \item \emph{PF/BC deviations:} policy violations in bad reduction handling or height normalization.
\end{itemize}

\paragraph{Window certificate (BSD).}
A certificate for \((W,\tau)\) contains: the single\hyp layer persistence objects, spectral proxies on \(L(\Ctau F_s)\), the obstruction log with failure types, Ext checks, and the \(\delta\)\hyp ledger. The manifest includes: prime \(p\), family parameters, local policy for \(S\), height cutoff \(H\), Laplacian normalization, and seeds for deterministic matching.

\noindent A minimal \texttt{run.yaml} BSD block:

\begin{verbatim}
windows:
  domain: [[1e3,2e3), [2e3,3e3)]
  collapse_tau: 0.12
  spectral_bins: {a: 0.0, beta: 0.04, bins: 128, boundary: "left-open"}
family:
  curve: "E: y^2 = x^3 - x"
  prime_p: 3
  twists: {type: quadratic, d_range: [1, 1000], parity_filter: "even"}
  height_cutoff: 14.0
local_policy:
  S: ["p=3","p=5","infty"]
  conditions: {relaxation: "bounded", penalty_step: 0.5}
operations:
  - U: tighten_local; type: deletion; tau: 0.12; delta: {alg:0.002, disc:0.001, meas:0.001}
  - U: twist_step;    type: epsilon;  tau: 0.12; eps: 0.005; delta: {alg:0.000, disc:0.002, meas:0.001}
persistence:
  PH1_zero: true
  Ext1_zero: true
  mu: 0
  nu: 0
  phi_iso_tail: true
spectral:
  aux_bars_remaining: 0
budget:
  sum_delta: 0.007
  safety_margin: 0.021
gate:
  accept: true
\end{verbatim}

\begin{remark}[Guard\hyp rails for BSD]
The protocol monitors persistence\hyp layer stability under fixed policies and logs reproducible manifests. It does not prove BSD, nor does it identify algebraic rank; rank proxies and spectral tails are diagnostics only. All claims remain persistence/spectral/categorical and policy\hyp dependent.
\end{remark}

% ============================================================
% Application III: RH up to T (template with window certificate + run.yaml + ledger)
% ============================================================

\subsection*{10.16. Application III: RH up to \texorpdfstring{$T$}{T} (template)}
We provide a verification\hyp style template to monitor windows in height and produce reproducible certificates.\ \emph{No RH claim is made}; zero\hyp locating or discrepancy detection is carried out at the persistence/spectral layer under fixed sampling/smoothing policies.

\paragraph{Admissible realization \(\mathcal{P}_{\mathrm{RH}}\).}
Let the state record: a height window \(W=[u,u')\), sampling mesh \(\Delta t\), smoothing kernel and bandwidth, and a normalization policy for \(\xi\), \(S(t)\), or explicit\hyp formula residuals. Construct \(F=\mathcal{P}_{\mathrm{RH}}(W)\) via:
\begin{itemize}
  \item \emph{Gram\hyp graph pipeline:} nodes at mesh/Gram points; edges connect neighbors; filtration by discrepancy thresholds \(|S(t)-S_{\mathrm{ref}}(t)|\).
  \item \emph{Scalar pipeline:} sub/superlevel filtrations of smoothed \(|\zeta(1/2+it)|\), \(|\xi|\), or residual fields.
  \item \emph{Hybrid:} couple zero\hyp candidate events with discrepancy fields; evaluate degreewise.
\end{itemize}
Deletion\hyp type: convolution smoothing; restriction to subwindows; projection to bandlimited subspaces. Epsilon\hyp continuation: small height steps under Nyquist control; inclusion: window enlargement or mesh refinement.

\paragraph{Indicators and gate.}
Compute \(\Ttau\,\mathbf{P}_i(F_s)\), \(\mathrm{PE}_i^{\le\tau}\), \(\mathrm{ST}_\beta^{\ge M(\tau)}\), \(\mathrm{HT}(t;\cdot)\) on \(L(\Ctau F_s)\) per window, with fixed normalization and bandlimit policies.
Gate: PH1\(=0\), Ext1\(=0\) (for test objects reflecting normalization checks), \((\mu,\nu)=(0,0)\), and \(\mathrm{gap}_\tau>\Sigma\delta\).

\paragraph{Triggers ([Spec]).}
\begin{itemize}
  \item \emph{Argument anomaly:} excursions of \(S(t)\) beyond tolerances, or explicit\hyp formula residual spikes.
  \item \emph{Sampling under\hyp resolution:} Nyquist/bandlimit violations for the chosen kernel/bandwidth.
  \item \emph{PF/BC deviations:} contour/grid normalization mismatches (e.g.\ Gram grid drift) or policy inconsistencies.
\end{itemize}

\paragraph{Window certificate (RH).}
A certificate for \((W,\tau)\) contains: single\hyp layer persistence, spectral proxies on \(L(\Ctau F_s)\), obstruction counts and failure types, Ext checks, and a \(\delta\)\hyp ledger. The manifest includes: mesh \(\Delta t\), kernel/bandwidth, bandlimit/Nyquist checks, normalization policy, and deterministic seeds.

\noindent A minimal \texttt{run.yaml} RH block:

\begin{verbatim}
windows:
  domain: [[1e9,1e9+5e5), [1e9+5e5, 1e9+1e6)]
  collapse_tau: 0.06
  spectral_bins: {a: 0.0, beta: 0.03, bins: 256, boundary: "right-open"}
sampling:
  dt: 1.0e-3
  bandlimit: 3000.0
  nyquist_check: true
smoothing:
  kernel: "gaussian"
  bandwidth: 2.5e-3
operations:
  - U: smooth;     type: deletion; tau: 0.06; delta: {alg:0.001, disc:0.002, meas:0.001}
  - U: heightstep; type: epsilon;  tau: 0.06; eps: 0.004; delta: {alg:0.000, disc:0.002, meas:0.001}
persistence:
  PH1_zero: true
  Ext1_zero: true
  mu: 0
  nu: 0
  phi_iso_tail: true
spectral:
  aux_bars_remaining: 0
budget:
  sum_delta: 0.006
  safety_margin: 0.018
gate:
  accept: true
\end{verbatim}

\begin{remark}[Guard\hyp rails for RH]
The protocol verifies stability of persistence/spectral indicators under fixed sampling/smoothing and normalization policies and produces reproducible manifests.
It does not assert the Riemann Hypothesis, nor count zeros; it only monitors windowed diagnostics with logged tolerances.
\end{remark}

% ============================================================
% Cross-application integration and auditability
% ============================================================

\subsection*{10.17. Cross\hyp application gate, triggers, and pasting}
All three applications (PDE, BSD rank~0/1, RH up to \(T\)) share:
\begin{itemize}
  \item \textbf{Gate B\hyp Gate\(^{+}\):} single\hyp layer decisions on \(\Ttau\mathbf{P}_i\) with PH1/Ext1/(\(\mu,\nu\))/safety\hyp margin.
  \item \textbf{Triggers:} blow\hyp up signs, tower accumulation, PF/BC deviations (domain\hyp specific specializations).
  \item \textbf{Restart/Summability:} window pasting with budgeted \(\Sigma\delta\) and geometric decay options for \(\tau,\beta\).
  \item \textbf{Reproducibility:} unified manifest keys (window domain, collapse\_tau, spectral bins, operations with \(\delta\)\hyp ledger, persistence verdicts, spectral auxiliaries, budget, gate).
\end{itemize}
Domain\hyp specific policies (normalization, local conditions, sampling/bandlimit) are fixed per run and logged; spectral indicators are always computed on \(L(\Ctau F)\).

\subsection*{10.18. Effect and operational readiness}
The templates make \emph{immediate operational deployment} possible: each application ships with (i) a gate specification, (ii) a trigger pack, (iii) a run manifest schema, and (iv) a window certificate format, yielding clear outcomes (certificate + reproducible logs).
Relative to v16.0, the deliverables are explicit and auditable.

\subsection*{10.19. Final guard\hyp rails (IMRN/AiM\hyp style)}
\begin{remark}[Non\hyp equivalences and scope]
All interleaving/Lipschitz/monotonicity claims are asserted at the persistence layer and, when stated for spectral proxies, under a fixed normalization policy on \(L(\Ctau F)\). Ext tests are scope\hyp restricted to a finite \(\Qtest\).
No analytic equivalences (e.g.\ BSD, RH, PDE regularity) are claimed or used. The program provides certificate\hyp style diagnostics with reproducible manifests and budgeted stability, suitable for audit and re\hyp execution.
\end{remark}



% ============================================================
% Part III — Measurement / Implementation / Tests
% ============================================================
\section{Chapter 11: Collapse Energy, Spectral Indicators, and TDA Notes}
\addcontentsline{toc}{section}{Collapse Energy, Spectral Indicators, and TDA Notes}

\subsection*{11.0. Scope, standing hypotheses, and notation}
We work within the \emph{implementable range} of Part~I, using realizations into $\FiltCh{k}$ as in Chs.~6–10. All persistence quantities are computed degreewise \emph{after} truncation by $\Ctau$; spectral indicators use the normalized combinatorial Hodge Laplacian on $\Ctau F$; categorical checks use a fixed $t$-exact $\Rfun:\FiltCh{k}\to\Db(k\text{-mod})$, compatible with \LC. Filtered (co)limits, when used, are computed objectwise in $[\mathbb{R},\mathsf{Vect}_k]$ and used only under the scope policy of Appendix~A (compute in the functor category and verify return to $\Pers^{\mathrm{cons}}_k$). No claim of $\mathrm{PH}_1\Leftrightarrow \Ext^1$ is made; only the one-way bridge under (B1)–(B3) is used. The obstruction pair $(\muc,\nuc)$ is the \emph{collapse} diagnostic and is unrelated to the classical Iwasawa $\mu$.

\begin{remark}[Endpoints and infinite bars]\label{rk:11-endpoints}
Open/closed endpoint conventions are immaterial; infinite bars are not removed by $\Ttau$ and are clipped by the window $\tau$ in all windowed quantities (cf.\ Ch.~6).
\end{remark}

\subsection*{11.0+. Length spectrum (crosslink) and isomorphism invariance}
We recall the \emph{length spectrum operator} $\Len(M;W)$ from Ch.~2, Definition~\ref{def:len-operator}, which records, for $M\in\Pers^{\mathrm{cons}}_k$ and a right-open window $W=[u,u')$, the clipped bar-length multiset as (diagonal) eigenvalues on a formal bar-basis. The following consolidates Ch.~2 and Appendix~H for convenient use in measurement.

\begin{proposition}[Length spectrum equals clipped bar lengths; isomorphism invariance]\label{prop:11-len}
Let $M\in\Pers^{\mathrm{cons}}_k$ admit a barcode decomposition $M\simeq \bigoplus_j I[b_j,d_j)$. For a right-open window $W=[u,u')$, the (unordered) eigenvalue multiset of $\Len(M;W)$ coincides with the clipped bar-length multiset $\{\ell_W(I[b_j,d_j))\}_j$, where $\ell_W$ is the Lebesgue length of $I[b_j,d_j)\cap W$. This multiset is invariant under isomorphisms $M\simeq M'$ in $\Pers^{\mathrm{cons}}_k$. In particular, the total collapse energy $\mathrm{PE}^{\le\tau}_i$ (Definition~\ref{def:11-PE}) equals the $L^1$-mass of $\Len\!\big(\mathbf{T}_\tau\mathbf{P}_i(F);[0,\tau]\big)$ and is therefore an isomorphism invariant of the truncated persistence.
\end{proposition}

\begin{proof}[Proof sketch]
See Ch.~2, Proposition~2.7 and Appendix~H (Theorem~\ref{H:thm:betti-integral}). Barcode uniqueness implies the multiset matches up to permutation; the operator is determined by the isomorphism class.
\end{proof}

\subsection*{11.0 bis. Overlap Gate and the unique comparison order in measurement}
All comparisons and audits follow the \emph{single comparison order}
\[
\boxed{\ \text{for each } t\ \Longrightarrow\ \text{apply } \mathbf{P}_i \ \Longrightarrow\ \text{apply } \mathbf{T}_\tau \ \Longrightarrow\ \text{compare in }\Pers^{\mathrm{cons}}_k\ }.
\]
This is enforced both locally and across overlaps:

\begin{declaration}[Overlap-aware measurement policy]\label{dec:11-og-policy}
Let $\{(X_\alpha,W_\alpha)\}$ be a windowed cover (right-open) and fix $(i,\tau)$.
\begin{enumerate}
\item \textbf{Local layer:} compute $\mathbf{T}_\tau\mathbf{P}_i(F|_{X_\alpha})$ and all indicators \emph{after truncation}.
\item \textbf{Overlap layer (Overlap Gate, cf.\ Ch.~1, Def.~1.0 and Ch.~5, Def.~5.8):} on each overlap, require
\begin{itemize}
  \item collapse-compatibility up to the recorded $\delta$-budget (Appendix~L; additively aggregated),
  \item Čech–Ext$^1$-acyclicity in degree $1$ \emph{after truncation},
  \item stability-band condition $(\mu,\nu)=(0,0)$ with near-$\tau$ non-accumulation.
\end{itemize}
\item \textbf{Global layer:} when all overlaps pass, glue to a global truncated object in $\Pers^{\mathrm{cons}}_k$ (Ch.~5, Thm.~5.8) and perform global audits \emph{after truncation}.
\end{enumerate}
The run manifest (\S\ref{rk:11-impl}) must include overlap checks (boolean), Čech–Ext$^1$ status, and the overlap $\delta$-budgets.
\end{declaration}

\begin{remark}[Why this order is unique]
Pre-collapse comparisons contaminate measurements by torsion noise and break exactness/metric controls. Truncation $\mathbf{T}_\tau$ is the exact, $1$-Lipschitz barrier that makes all downstream comparisons commensurable.
\end{remark}

\subsection*{11.1. Collapse energy (windowed persistence energies)}
Fix a degree $i$, a scale $\tau\ge 0$, and an exponent $\alpha>0$ (default $\alpha=1$). Let $\mathcal{B}_{i}(F)$ be the barcode of $\mathbf{P}_i(F)$, and for a bar $b=[b_{\ell},b_{r})$ write its $\tau$-windowed length
\[
  \ell_{\tau}(b)\ :=\ \bigl(\min\{b_{r},\tau\}-\min\{b_{\ell},\tau\}\bigr)_{+},\qquad (x)_{+}:=\max\{x,0\}.
\]
Fix a weight function $w_i:\mathcal{B}_i(F)\to[0,\infty)$ (default $w_i\equiv 1$).
\begin{definition}[Windowed energies]\label{def:11-PE}
The (weighted) windowed energy at degree $i$ is
\[
  \mathrm{PE}_{i}^{\le \tau}(F;w_i,\alpha)\ :=\ \sum_{b\in \mathcal{B}_{i}(F)} w_i(b)\,\big(\ell_{\tau}(b)\big)^{\alpha}.
\]
With the unweighted shorthand $\mathrm{PE}_{i}^{\le \tau}(F):=\mathrm{PE}_{i}^{\le \tau}(F;1,\alpha)$ (default $\alpha=1$), the \emph{collapse energy} vector at $\tau$ is
\[
  \mathbf{CE}^{\le \tau}(F)\ :=\ \big(\mathrm{PE}_{i}^{\le \tau}(F)\big)_{i\in\mathbb{Z}},\qquad \|\mathbf{CE}^{\le \tau}(F)\|_{1}\ :=\ \sum_{i}\mathrm{PE}_{i}^{\le \tau}(F).
\]
All quantities are computed on the truncated barcode $\Ttau\mathbf{P}_i(F)$ (equivalently, on $\Ctau F$).
\end{definition}

\begin{remark}[Stability and monotonicity]
By Part~I (Lemma~\ref{lem:shift}, Proposition~\ref{prop:stability}), $1$-Lipschitz updates under $\mathbf{P}_i$ induce non\hyp expansive changes of $\mathrm{PE}_{i}^{\le \tau}$; in \emph{deletion\hyp type} updates, $\mathrm{PE}_{i}^{\le \tau}$ is non\hyp increasing up to \fqi.
Since $\Ttau$ is $1$-Lipschitz for the interleaving distance, the same Lipschitz control carries over after truncation:
\[
  d_{\mathrm{int}}\!\big(\Ttau\mathbf{P}_i(F'),\,\Ttau\mathbf{P}_i(F)\big)\ \le\ d_{\mathrm{int}}\!\big(\mathbf{P}_i(F'),\,\mathbf{P}_i(F)\big).
\]
Under finite-type and objectwise degreewise-colimit hypotheses, $(\muc,\nuc)=(0,0)$ at fixed $\tau$ excludes Type~IV at that scale (Ch.~4).
\end{remark}

\subsection*{11.2. Spectral indicators on \texorpdfstring{$\Ctau F$}{C\_tau F}}
Let $L(\Ctau F)$ be the normalized combinatorial Hodge Laplacian on $\Ctau F$ (per degree, with the standard Euclidean inner product on chains), and let $\{\lambda_{j}\}_{j\ge 1}$ denote its positive eigenvalues (zero modes excluded; alternatively treat by Moore–Penrose pseudoinverse).
\begin{definition}[Spectral tails and heat traces]\label{def:11-spectral}
Fix $\beta>0$ and a cutoff policy $M(\tau)\in\mathbb{N}$. Define
\[
  \mathrm{ST}_{\beta}^{\ge M(\tau)}(F)\ :=\ \sum_{j\ge M(\tau)} \lambda_{j}^{-\beta},\qquad \mathrm{HT}(t;F)\ :=\ \sum_{j\ge 1} e^{-t\,\lambda_{j}},\quad t>0.
\]
Typical policies: \emph{(i)} $M(\tau)=\lfloor c\,\tau^{\gamma}\rfloor$ with $c>0$, $\gamma\in[0,2]$; \emph{(ii)} $t\in[c_1\tau^{-2},c_2\tau^{-2}]$ with fixed $0<c_1\le c_2$.
\end{definition}

\begin{remark}[Mandatory spectral ordering and norms]\label{rk:11-spectral-order-norm}
For auditability and comparability across runs, it is \emph{mandatory} that
\begin{itemize}
  \item eigenvalues in spectral artifacts are stored in \emph{ascending} order;
  \item the chosen matrix norm for numeric tolerances is reported as \texttt{norm} $\in\{\texttt{op},\texttt{fro}\}$.
\end{itemize}
These fields are required in \texttt{run.yaml} (Appendix~G) and are rechecked in \S\ref{rk:11-impl}.
\end{remark}

\begin{remark}[Stability]
Non\hyp expansive filtered maps on $\Ctau F$ induce $1$-Lipschitz updates at the persistence layer and stable (non\hyp expansive) spectral responses for $\mathrm{ST}_{\beta}^{\ge M(\tau)}$, $\mathrm{HT}(t;-)$, provided the policy $(\beta,M(\tau),t)$ is held fixed (cf.\ Chs.~6–10).
\end{remark}

\subsection*{11.3. Auxiliary spectral bars (aux-bars)}
We supplement spectral scalars by a \emph{binwise occupancy} summary along a discrete index (e.g.\ time, step number), evaluated on \emph{collapsed} Laplacians $L(\Ctau F)$.

\begin{definition}[Spectral bins, occupancies, and aux-bars]\label{def:11-aux-bars}
Fix a spectral window $[a,b]$ and a bin width $\beta>0$. Define right-open bins
\[
I_r\ :=\ [\,a+r\beta,\ a+(r+1)\beta\,),\qquad r=0,1,\dots,R-1,\ \ R:=\Big\lfloor\frac{b-a}{\beta}\Big\rfloor.
\]
For a sample index $j$ (e.g.\ time step), let $\{\lambda_m(j)\}_{m\ge 1}$ be the positive eigenvalues of $L(\Ctau F_j)$ and define \emph{occupancies}
\[
E_r(j)\ :=\ \#\{\,m\ \mid\ \lambda_m(j)\in I_r\,\},\qquad \text{with under/overflow counts }E_{<a}(j),\ E_{\ge b}(j)\ \text{logged explicitly}.
\]
For each fixed bin $r$, define an \emph{auxiliary spectral bar} (aux-bar) as a maximal consecutive run $J\subset\mathbb{Z}$ such that $E_r(j)>0$ for all $j\in J$. The \emph{lifetime} of the aux-bar is $|J|$ (or a rescaling thereof).
\end{definition}

\begin{proposition}[Cumulative profile monotonicity and stability]\label{prop:11-cum}
Let $C_r(j):=\sum_{s=r}^{R-1}E_s(j)$ be the cumulative profile for the spectral window. Then:
\begin{enumerate}
\item \emph{(Deletion-type monotonicity)} Under deletion-type updates (Dirichlet/principal restriction, Schur complement, Loewner contractions; Appendix~E), $C_r(j\!+\!1)\le C_r(j)$ for all $r$.
\item \emph{(Stability under $\varepsilon$-continuations)} If $\|A_{j+1}-A_j\|_{\mathrm{op}}\le\varepsilon$, then $C_{r+q}(j+1)\le C_r(j)\le C_{\max\{0,r-q\}}(j+1)$ with $q=\lceil\varepsilon/\beta\rceil$.
\end{enumerate}
\end{proposition}

\begin{proof}[Proof sketch]
Appendix~E (Propositions~\ref{E:prop:aux-cum-mono}, \ref{E:prop:aux-cum-stab}).
\end{proof}

\begin{remark}[Gate alignment]
The cumulative profile and aux-bars are \emph{auxiliary} gauges for B--Gate$^{+}$ (Declaration~\ref{dec:11-aux-gate}). They never override the persistence-layer verdict (PH/Ext/\,$\mu$--$\nu$), but they provide robust, windowed evidence of deletion-type monotonicity or $\varepsilon$-stability.
\end{remark}

\begin{remark}[Mandatory bin policy]\label{rk:11-bin}
Aux-bars must be computed \emph{after} collapse and with a fixed bin policy $(a,b,\beta)$ per window; boundary is right-open; under/overflow are recorded (Appendix~G).
\end{remark}

\subsection*{11.4. Categorical check (one-way bridge)}
With $\Qtest=\{k[0]\}$ fixed, we monitor the categorical obstruction
\[
  \Ext^1\big(\Rfun(\Ctau F),Q\big)=0\qquad(Q\in\Qtest).
\]
This check is performed \emph{after truncation} and used only in the one-way direction sanctioned in Part~I (under (B1)–(B3)); no converse or equivalence is claimed.

\subsection*{11.5. Collapse diagnostics along towers}
For an index category $I$ (time/resolution/parameter), adjoin a terminal element $\infty$ and cone maps to form comparison morphisms
\[
  \phi_{i,\tau}:\ \varinjlim_{t\in I}\ \Ttau\mathbf{P}_i(F_t)\ \longrightarrow\ \Ttau\mathbf{P}_i(F_\infty),
\]
with
\[
  \mu_{i,\tau}=\dim_k\ker\phi_{i,\tau}\footnote{Here $\dim_k$ denotes the \emph{generic fiber} dimension after truncation, i.e.\ the multiplicity of $I[0,\infty)$ summands; see Appendix~D, Remark~\ref{rem:D-generic-dim}.},\qquad\n  \nu_{i,\tau}=\dim_k\mathrm{coker}\,\phi_{i,\tau}\footnote{Here $\dim_k$ denotes the \emph{generic fiber} dimension after truncation, i.e.\ the multiplicity of $I[0,\infty)$ summands; see Appendix~D, Remark~\ref{rem:D-generic-dim}.},
\]
and $\muc=\sum_{i}\mu_{i,\tau}$, $\nuc=\sum_{i}\nu_{i,\tau}$ (finite by finite-typeness). Under the hypotheses of Part~I/Ch.~4, $\phi_{i,\tau}$ are isomorphisms and $(\muc,\nuc)=(0,0)$ at fixed $\tau$.

\subsection*{11.6. Specification: joint monitoring protocol}
\begin{declaration}[Specification: Joint monitoring protocol]\label{spec:11-monitor}
Fix a finite sweep $\tau\in[\tau_{\min},\tau_{\max}]$ and a policy $(\alpha, w_i;\ \beta, M(\tau), t)$.
For each sample $t\in I$ and degree $i$:
\begin{enumerate}
  \item \emph{Compute and record} $\Ttau\mathbf{P}_i(F_t)$; evaluate $\mathrm{PE}_{i}^{\le \tau}(F_t)$ on $\Ttau\mathbf{P}_i(F_t)$ (equivalently on $\Ctau F_t$).
  \item \emph{Compute and record} $\mathrm{ST}_{\beta}^{\ge M(\tau)}(F_t)$ and $\mathrm{HT}(t;\Ctau F_t)$ using $L(\Ctau F_t)$; compute aux-bars via Definition~\ref{def:11-aux-bars} on the same window $[a,b]$ and bin policy $\beta$.
  \item \emph{Check} $\Ext^1\big(\Rfun(\Ctau F_t),Q\big)=0$ for $Q\in\Qtest$.
  \item \emph{Evaluate} $(\muc,\nuc)$ at each $\tau$ via the comparison maps $\phi_{i,\tau}$; log failure type (pure kernel/cokernel/mixed) if $(\muc,\nuc)\neq(0,0)$.
  \item \emph{Declare stable regime} at $\tau$ when (1)–(3) hold jointly and $(\muc,\nuc)=(0,0)$. Optionally require aux-bars$=0$ (Declaration~\ref{dec:11-aux-gate}).
\end{enumerate}
All persistence-layer steps are asserted at the persistence layer and are invariant under \fqi\ by construction; spectral and aux-bar steps are not \fqi\ invariants but are \emph{stable} under the fixed policy \((\beta, M(\tau), t)\) on \(L(\Ctau F_t)\).
\end{declaration}

\subsection*{11.7. Noise tolerance and discretization rules}
\begin{declaration}[Specification: noise and discretization policy]\label{spec:11-noise}
Let $\varepsilon>0$ be the noise scale.
\begin{itemize}
  \item \textbf{Barcode denoising:} apply \emph{$\varepsilon$-clipping} on $\Ttau\mathbf{P}_i(F)$ by removing bars of length $\le \varepsilon$ within the $\tau$-window. This is stable under bottleneck perturbations $\le \varepsilon$ and preserves \fqi\ invariants.
  \item \textbf{Energy stability:} there exists $C_{i,\tau,\alpha}$ (depending on the number of bars in the window, their maximal multiplicity, and ambient dimension bounds) such that
  \[
    \big|\mathrm{PE}_{i}^{\le \tau}(F)-\mathrm{PE}_{i}^{\le \tau}(\tilde F)\big|\ \le\ C_{i,\tau,\alpha}\,\varepsilon^{\min\{1,\alpha\}}
  \]
  whenever $d_{\mathrm{int}}(\Ttau\mathbf{P}_i(F),\Ttau\mathbf{P}_i(\tilde F))\le \varepsilon$.
  \item \textbf{Spectral stabilization:} compute spectra on $L(\Ctau F)$; use the same $(\beta,M(\tau),t)$ across runs. Optional averaging over $N$ realizations reduces variance as $N^{-1/2}$. Aux-bar lifetimes are counted with the same binning policy; short lifetimes $\le 2$ frames may be ignored as noise (manifest must record the rule).
  \item \textbf{Resolution rule:} choose sampling so that the minimal feature length resolved is $\ge 3$ grid steps; sweep $\tau$ on a lattice $\Delta\tau$ satisfying $\Delta\tau\le \tfrac{1}{2}$ of the minimal resolvable bar length. Here the “bar-length quantum’’ refers to the minimal resolvable feature length induced by the sampling/grid spacing and the filtration step.
\end{itemize}
\end{declaration}

\subsection*{11.8. Categorical check (bridge) and saturation gate}
With $\Qtest=\{k[0]\}$ fixed, we monitor $\Ext^1(\Rfun(\Ctau F),Q)=0$ after truncation. The one-way bridge is used only under (B1)–(B3). For windows satisfying the \emph{saturation} conditions:

\begin{declaration}[Saturation gate \textbf{[Spec]}]\label{gate:11-saturation}
Fix $\tau^\ast>0$ and parameters $\eta,\delta>0$. On the window $[0,\tau^\ast]$, assume:
(i) eventually the maximal \emph{finite} bar length in $\mathbf{T}_{\tau^\ast}\mathbf{P}_i(F_t)$ is $\le \eta$;
(ii) eventually $d_{\mathrm{int}}\!\big(\mathbf{T}_{\tau^\ast}\mathbf{P}_i(F_t),\mathbf{T}_{\tau^\ast}\mathbf{P}_i(F_{t'})\big)\le \eta$;
(iii) the \emph{edge gap} to the window, $\delta:=\tau^\ast-\max\{b_r<\tau^\ast\}$, satisfies $\delta>\eta$.
Then, \textbf{within this window only}, we adopt the temporary binary policy
\[
\mathrm{PH}_1(\C_{\tau^\ast}F)=0\quad\Longleftrightarrow\quad \Ext^1(\Rfun(\C_{\tau^\ast}F),k)=0.
\]
\end{declaration}

\subsection*{11.9. Data formats, reproducibility, and minimal manifest}
\begin{remark}[Implementation notes and mandatory fields]\label{rk:11-impl}
\emph{Artifacts.} (i) \texttt{bars.json/h5}: list of records $\langle i,\ b_{\ell},b_{r},\ w\rangle$ per degree; (ii) \texttt{spec.json/h5}: positive eigenvalues of $L(\Ctau F)$ per degree; (iii) \texttt{aux.json/h5}: aux-bar occupancies $E_r(j)$ with bin metadata; (iv) \texttt{ext.json}: boolean for $\Ext^1(\Rfun(\Ctau F),Q)$ with minimal witness; (v) \texttt{phi.json}: ranks of comparison maps and $(\mu_{i,\tau},\nu_{i,\tau})$.
\emph{Run log.} Store (a) sweep $\tau_{\min}$:$\Delta\tau$:$\tau_{\max}$; (b) policy $(\alpha, w_i;\ \beta,M(\tau),t)$; (c) discretization (grid/complex, step sizes); (d) random seeds; (e) software versions; (f) $\delta$-ledger per step; (g) bin window $[a,b]$, bin width $\beta$, under/overflow counts; (h) \emph{mandatory} spectral fields \texttt{order=ascending}, \texttt{norm} $\in\{\texttt{op},\texttt{fro}\}$; (i) overlap checks (\texttt{overlap\_checks}), Čech–Ext$^1$ status (\texttt{cech\_ext1\_ok}), and tail-isomorphism flag (\texttt{phi\_iso\_tail}); (j) an optional \texttt{length\_spectrum} summary (vector of clipped lengths or hash) for quick audits.
\emph{Invariance.} Persistence-layer quantities are taken on $\Ttau\mathbf{P}_i(-)$ or $\Ctau(-)$ and are invariant under \fqi. \emph{Reproducibility.} A single manifest (\texttt{run.yaml}) references all artifacts, declares the tower index set, cone extension, the decision rule for stable regimes, and the overlap status.
\end{remark}

\noindent A minimal \texttt{run.yaml} block (augmented):
\begin{verbatim}
windows:
  domain: [[0,1), [1,2), [2,3)]
  collapse_tau: 0.08
  spectral_bins: {a: 0.0, beta: 0.02, bins: 96, boundary: "right-open"}
coverage_check:
  length_sum: 3.0
  length_target: 3.0
  events_sum_equals_global: true
overlap_checks:
  local_equal_after_collapse: true
  cech_ext1_ok: true
  stability_band_ok: true
spectral_policy:
  order: "ascending"
  norm: "op"
operations:
  - U: mollify; type: deletion; tau: 0.08; delta: {alg:0.004, disc:0.003, meas:0.001}
  - U: timestep; type: epsilon;  tau: 0.08; eps:0.006; delta: {alg:0.000, disc:0.002, meas:0.001}
persistence:
  PH1_zero: true
  Ext1_zero: true
  mu: 0
  nu: 0
  phi_iso_tail: true
length_spectrum:
  degree: 1
  tau: 0.08
  eigenvalues: [0.24, 0.51, 0.78]   # optional summary of clipped bar lengths
spectral:
  ST_beta: 2
  ST_M_of_tau: "floor(0.5 * tau^1.5)"
  HT_t: [0.5*tau^-2, 1.0*tau^-2]
  aux_bars_remaining: 0
budget:
  sum_delta: 0.011
  safety_margin: 0.025
gate:
  accept: true
\end{verbatim}

\subsection*{11.10. Compliance checks and unit tests}
\begin{declaration}[Specification: minimal test suite]\label{spec:11-tests}
Every deployment must pass:
\begin{itemize}
  \item \textbf{Stability test.} For synthetic $\varepsilon$-perturbations, verify non\hyp expansiveness of $\mathrm{PE}_{i}^{\le \tau}$ and stability of spectral indicators and aux-bars under the fixed policy.
  \item \textbf{Monotone update test.} For a deletion-type update, confirm $\mathrm{PE}_{i}^{\le \tau}$ non-increase, spectral tail non-increase, aux-bars non-increase (active bins do not grow), and record $(\muc,\nuc)=(0,0)$ at fixed $\tau$.
  \item \textbf{Cone-extension test.} Verify $\phi_{i,\tau}$ is an isomorphism on a model tower (hence $(\muc,\nuc)=(0,0)$) and that Type~IV is excluded at that $\tau$.
  \item \textbf{Categorical check.} On a curated sample, confirm $\Ext^1(\Rfun(\Ctau F),Q)=0$ is stable under admissible \fqi\ updates.
\end{itemize}
\end{declaration}

\subsection*{11.10 bis. Hooks to Chapter~12 (test harness)}
Chapter~12 formalizes the global test harness. In particular:
\begin{itemize}
  \item \textbf{T7 (Saturation gate).} Verifies the quantitative saturation conditions of \S\ref{gate:11-saturation} on $[0,\tau^\ast]$ and logs the temporary local equivalence.
  \item \textbf{T10 (A/B commutativity).} Runs A/B tests after truncation (Appendix~K/L) on windowed reflectors; failures trigger a deterministic order with the commutation defect logged as $\delta^{\mathrm{alg}}$.
  \item \textbf{T13 (Pipeline $\delta$-budget).} Aggregates Mirror–Collapse and A/B residuals additively and checks post-processing non-increase (\S\ref{rk:11-impl}, Appendix~L).
\end{itemize}
Passing T7/T10/T13 is necessary for issuing \emph{windowed certificates} (Ch.~4) and for \emph{pasting} via Restart/Summability (Appendix~J).

% ============================================================
% 11.A  Formalization API (Lean/Coq stubs)
% ============================================================
\subsection*{11.A. Formalization API (Lean/Coq stubs)}
All items are stated in the constructible range and, at the filtered–complex layer, \emph{up to filtered quasi-isomorphism}. Identifiers are indicative.

\begin{specification}[Persistence truncation $\mathbf{T}_\tau$ — exactness and $1$-Lipschitz]
\begin{itemize}
  \item \texttt{pers\_Ttau\_exact}: $\mathbf{T}_\tau$ is exact on $\Pers^{\mathrm{cons}}_k$ (Serre localization at $\tau$).
  \item \texttt{pers\_Ttau\_lipschitz}: $d_{\mathrm{int}}(\mathbf{T}_\tau M,\mathbf{T}_\tau N)\le d_{\mathrm{int}}(M,N)$.
  \item \texttt{pers\_Ttau\_commute\_colim\_flim}: $\mathbf{T}_\tau$ commutes with filtered colimits and finite limits.
\end{itemize}
\end{specification}

\begin{specification}[Comparison maps $\phi$ — functoriality and cofinal invariance]
For a filtered diagram $\{F_t\}_{t\in I}$ and degree $i$,
\[
\phi_{i,\tau}:\ \varinjlim_{t}\ \mathbf{T}_\tau\mathbf{P}_i(F_t)\ \longrightarrow\ \mathbf{T}_\tau\mathbf{P}_i(F_\infty).
\]
\begin{itemize}
  \item \texttt{phi\_natural}: natural in (a) morphisms of diagrams; (b) $\tau$ via $\mathbf{T}_\tau\Rightarrow \mathbf{T}_\sigma$ for $\tau\le \sigma$.
  \item \texttt{phi\_cofinal}: invariant under cofinal reindexing $I\to I'$.
  \item \texttt{phi\_sum\_prod}: additive under finite direct sums; compatible with finite limits (pullbacks) at the persistence layer.
\end{itemize}
\end{specification}

\begin{specification}[Kernel/cokernel indices $(\mu,\nu)$ — computation and calculus]
\begin{itemize}
  \item \texttt{mu\_nu\_def}: $\mu_{i,\tau}=\dim_k\ker\phi_{i,\tau}$, $\ \nu_{i,\tau}=\dim_k\mathrm{coker}\,\phi_{i,\tau}$; $\ \muc=\sum_i\mu_{i,\tau}$, $\ \nuc=\sum_i\nu_{i,\tau}$ (finite by bounded degree).
  \item \texttt{mu\_nu\_vanish}: under degreewise objectwise filtered colimits (Part~I/Ch.~4), each $\phi_{i,\tau}$ is an isomorphism, hence $(\muc,\nuc)=(0,0)$.
  \item \texttt{mu\_nu\_calc}: subadditivity under composition, additivity under finite sums, and cofinal invariance (Appendix~J).
\end{itemize}
Here $\dim_k$ denotes the \emph{generic fiber} dimension after truncation (Appendix~D, Remark~\ref{rem:D-generic-dim}).
\end{specification}

\begin{specification}[Edge identification and bridge (B2)]
\begin{itemize}
  \item \texttt{edge\_iso\_B2}: natural isomorphism $H^{-1}\!\big(\Rfun(F)\big)\ \cong\ \varinjlim_{t}\ H_1\!\big(F^{t}C_\bullet\big)$ and $\Rfun(F)\in D^{[-1,0]}$.
  \item \texttt{ext1\_edge\_iso}: natural isomorphism $\Ext^1(\Rfun(\Ctau F),k)\ \cong\ \Hom\!\big(H^{-1}(\Rfun(\Ctau F)),k\big)$; in particular $H^{-1}=0$ iff $\Ext^1=0$ in this amplitude-$\le 1$ range.
\end{itemize}
\end{specification}

\begin{specification}[Aux-bars (measurement layer)]
\begin{itemize}
  \item \texttt{aux\_bins}: parameters $(a,b,\beta)$; right-open bins $I_r=[a+r\beta,a+(r+1)\beta)$; record under/overflow.
  \item \texttt{aux\_occupancy}: $E_r(j)=\#\{m\mid \lambda_m(j)\in I_r\}$ from the spectrum of $L(\Ctau F_j)$.
  \item \texttt{aux\_bars}: for fixed $r$, maximal consecutive runs $J$ with $E_r(j)>0$; lifetime $|J|$ (or rescaled).
  \item \texttt{aux\_monotone}: deletion-type steps $\Rightarrow$ non\hyp increase of active bins and total aux-bar mass; $\varepsilon$-continuations $\Rightarrow$ stability; inclusion-type $\Rightarrow$ stability only.
\end{itemize}
\end{specification}

\subsection*{11.11. Diagram (pipeline and logs)}
\begin{center}
\begin{tikzcd}[
  ampersand replacement=\&,
  column sep=2.4em, row sep=2.0em,
  cells={nodes={font=\footnotesize}},
  every label/.append style={font=\scriptsize},
  scale=0.9, transform shape
]
\text{Input }F \arrow[r, "\mathbf{P}_i"]
  \& \mathbf{P}_i(F) \arrow[r, "\Ttau"] \arrow[d, dashed, "{\mathrm{PE}_i^{\le \tau}}"]
    \& \Ttau\mathbf{P}_i(F) \arrow[d, dashed, "{\mathrm{PE}_i^{\le \tau}}"] \\\n\& L(\Ctau F) \arrow[r, dashed, "{\text{heat/aux on }L(\Ctau F)}"]
    \& \mathrm{HT},\ \mathrm{ST}_{\beta},\ \text{aux\hyp bars} \\\n\Rfun(F) \arrow[uu, bend left=40, dashed, "\LC"]
  \arrow[rr, dashed, "{\Ctau\ \text{ then }\Ext^1(-,Q)}"]
    \&\& \Ext^1(\Rfun(\Ctau F),Q)=0 \\\n\varinjlim_{t}\ \Ttau\mathbf{P}_i(F_t) \arrow[rr, "{\phi_{i,\tau}}"]
  \&\& \Ttau\mathbf{P}_i(F_\infty) \arrow[lld, bend right=25, swap, dashed, "{(\muc,\nuc)\ \text{log}}"] \\\n\text{failure log (pure kernel/cokernel/mixed)} \&\&
\end{tikzcd}
\end{center}

\subsection*{11.12. Completion note}
\begin{remark}[No further supplementation required]
This chapter fully integrates: (i) collapse energy and spectral indicators on $L(\Ctau F)$; (ii) auxiliary spectral bars (aux-bars) with binning, occupancies, lifetimes, and monotonicity/stability policies; (iii) the joint monitoring protocol with gate usage of aux-bars as optional auxiliary conditions; (iv) noise/discretization policies, including lifetime thresholds and reproducibility; (v) a minimal manifest schema and artifact set; (vi) a formalization stub for persistence, towers, bridge, and aux-bars; (vii) the length-spectrum crosslink (\S11.0+) and the Overlap Gate integration (\S11.0 bis); (viii) mandatory spectral ordering and norms (\S\ref{rk:11-spectral-order-norm}); (ix) hooks to the global test harness (\S11.10 bis). All claims remain within the v16.0 guard-rails (B-side single layer, PH1$\to$Ext1 one-way only, MECE windows, $\delta$-ledger, tower diagnostics); no additional supplementation is needed for operational use as a proof/measurement framework.
\end{remark}

\subsection*{11.13. Guard-rails}
\begin{remark}[Scope and non-claims]\label{rk:11-guard}
This chapter specifies measurement protocols, auxiliaries, and implementation notes at the persistence/spectral/categorical layers. No analytic regularity, group trivialization, or number-theoretic identity is asserted. All statements respect the guard-rails of Part~I; in particular, no claim of $\mathrm{PH}_1\Leftrightarrow \Ext^1$ is made, and $\muc$ differs from classical Iwasawa $\mu$.
\end{remark}



% ============================================================
% Part III — Measurement / Implementation / Tests
% ============================================================
\section{Chapter 12: Formal Test Suite and Open Problems}
\addcontentsline{toc}{section}{Formal Test Suite and Open Problems}

% --- Badge policy (explicit at chapter head) ---
\noindent\textbf{Badge policy.}
\begin{itemize}
  \item \textbf{[Prop]}: mathematics proved in Part~I (core results; cite exact source).
  \item \textbf{[Declaration]}: programmatic specification in the implementable range, verifiable by the test suite in this chapter.
  \item \textbf{[Conjecture]}: forward-looking statement; no claim beyond the stated scope.
\end{itemize}

% --- Notation & Conventions block for the whole chapter ---
\subsection*{12.0. Notation \& conventions}
\begin{itemize}
  \item \textbf{Constructible range.} We identify \(\Perskft\) with the constructible subcategory of \([\mathbb{R},\mathsf{Vect}_k]\) and use \(\Perskft\) uniformly (we do not use \(\Pers^{\mathrm{cons}}_k\) hereafter).
  \item \textbf{Truncation phrase.} “after applying \(\mathbf{T}_\tau\); equivalently on \(\Ctau F\)” is our standard phrase indicating that a quantity is computed at the persistence layer after truncation (hence equivalently on the filtered lift \(\Ctau F\)).
  \item \textbf{Generic–fiber dimension.} For a comparison map \(\phi_{i,\tau}\) at fixed \(\tau\), \(\dim_k\) denotes the \emph{generic–fiber dimension after truncation}, i.e.\ the multiplicity of \(I[0,\infty)\) summands in \(\Ttau\mathbf{P}_i(-)\); informally, the \(t\to\infty\) stable rank within the \(\tau\)–window.
  \item \textbf{Spectral ordering and norms.} Positive eigenvalues of \(L(\Ctau F)\) are listed in ascending order \(\lambda_1\le \lambda_2\le\cdots\). Matrix/operator norms are denoted by \(\|\cdot\|_{\mathrm{op}}\) (operator norm) and \(\|\cdot\|_{\mathrm{fro}}\) (Frobenius norm); each test specifies which is used and logs the choice.
  \item \textbf{Obstruction totals and macros.} We write \(\mu_{i,\tau}=\dim_k\ker\phi_{i,\tau}\), \(\nu_{i,\tau}=\dim_k\mathrm{coker}\,\phi_{i,\tau}\), and set the totals \(\muc:=\sum_i\mu_{i,\tau}\), \(\nuc:=\sum_i\nu_{i,\tau}\) (finite by bounded degree).
  \item \textbf{Endpoints.} Endpoint conventions and the handling of infinite bars follow the global policy (cf.\ Appendix~A, Remark~\ref{rk:A-endpoints}); infinite bars are not removed by \(\Ttau\) and are clipped by the window in all windowed quantities.
  \item \textbf{Non\hyp expansiveness.} We use the spelling “non\hyp expansive”/“non\hyp expansiveness” uniformly.
\end{itemize}

\subsection*{12.1. Badge inventory (representative items)}
\begin{center}
\begingroup
\renewcommand{\arraystretch}{1.12}
\begin{tabular}{@{}l >{\raggedright\arraybackslash}p{0.74\textwidth}@{}}
\toprule
\textbf{Badge} & \textbf{Representative items (label / location)}\\
\midrule
\textbf{[Prop]} &
  Stability, idempotence, and exactness of $\Ttau$ (Prop.~\ref{prop:stability}, Ch.~2);\\
& Shift--commutation / $1$\nobreakdash-Lipschitz for $\Ttau$ (Lemma~\ref{lem:shift}, Ch.~2);\\
& Operational coreflection $\mathsf{C}_\tau^{\mathrm{comb}}$ on the implementable range
  (Prop.~\ref{prop:operational-coreflection}, Ch.~5);\\
& Tower diagnosis: $(\muc,\nuc)$ via cone extension; isomorphism criterion excluding Type~IV (Prop.~\ref{prop:mu-vanishing}, Ch.~4).\\
\textbf{[Thm]} &
  One\hyp way bridge: $\mathrm{PH}_1(F)=0 \Rightarrow \Ext^1(\mathcal{R}(F),k)=0$ under (B1)--(B3)\\
& \hspace{1.6em}(Thm.~\ref{thm:PH1-to-Ext1}, Ch.~3).\\
\textbf{[Declaration]} &
  Ch.~2: (co)limit and pullback compatibility \emph{at the persistence layer only} (after $\Ttau$);\\
& Ch.~6: filtered\hyp colimit stability in geometry; joint indicators; protocol (after truncation);\\
& Ch.~7: arithmetic tower stability; non\hyp identity of $\muc$ with Iwasawa $\mu$;\\
& Ch.~8: tropical shortening $\Rightarrow$ weak group collapse; mirror transfer \emph{non\hyp expansive after truncation};\\
& Ch.~9: three\hyp layer (Gal$\to$Trans$\to$Funct) compatibility as isomorphisms in $\Perskft$ \emph{after} $\Ttau\mathbf{P}_i$;\\
& Ch.~10: persistence\hyp guided regularization; AK--NS hypothesis (programmatic);\\
& Ch.~11: joint monitoring, noise/discretization policy, minimal test suite; central \textbf{Saturation gate} (Ch.~11.S).\\
\textbf{[Conjecture]} &
  Cross\hyp domain collapse propagation (Chs.~6--10); AK--NS (Ch.~10); mirror\hyp side propagation (Ch.~8);\\
& Functorial transfer stability (Ch.~9).\\
\bottomrule
\end{tabular}
\endgroup
\end{center}

\subsection*{12.2. Formal test suite (unit / integration / regression)}
All tests operate at the truncated persistence, spectral (on \(L(\Ctau F)\)), and categorical layers and are f.q.i.\ invariant at the persistence layer. A test \emph{passes} iff all stated pass\hyp criteria are met and logs are complete. Pass\hyp criteria must state whether indicators are evaluated \emph{per degree} or \emph{aggregated} across degrees; the choice must be fixed and logged for the run. Spectra use ascending order \(\lambda_1\le\lambda_2\le\cdots\), and the chosen norm \(\|\cdot\|_{\mathrm{op}}\) or \(\|\cdot\|_{\mathrm{fro}}\) must be declared and logged.

\paragraph{(T1) Stability under non\hyp expansive updates [Unit].}
\emph{Input:} pairs \(F\to F'\) with \(d_{\mathrm{int}}(\mathbf{P}_i(F),\mathbf{P}_i(F'))\le \varepsilon\).\\\n\emph{Assertions:} \(\big|\mathrm{PE}^{\le\tau}_i(F)-\mathrm{PE}^{\le\tau}_i(F')\big|\le C_{i,\tau,\alpha}\,\varepsilon^{\min\{1,\alpha\}}\) (computed after applying \(\Ttau\); equivalently on \(\Ctau F\)); spectra of \(L(\Ctau F)\) vs.\ \(L(\Ctau F')\) (eigenvalues ascending) satisfy the fixed \((\beta,M(\tau),t)\)\hyp policy stability bounds in the declared norm; \(\Ext^1(\Rfun(\Ctau-),Q)\) is \emph{stable under admissible f.q.i.\ updates} (for all \(Q\in\Qtest\)); in particular, if \(\Ext^1(\Rfun(\Ctau F),Q)=0\) at baseline, then \(\Ext^1(\Rfun(\Ctau F'),Q)=0\).\\\n\emph{Artifacts:} \texttt{bars.json}, \texttt{spec.json}, \texttt{ext.json}; \texttt{run.yaml} (norm and spectral policy recorded).

\paragraph{(T2) Monotone update (deletion\hyp /inclusion\hyp type) [Unit].}
\emph{Input:} \(F\to F'\) monotone.\\\n\emph{Assertions:} \emph{Deletion\hyp type:} \(\mathrm{PE}^{\le\tau}_i\) and spectral indicators are non\hyp increasing (after \(\Ttau\)); spectral eigenvalues are compared in ascending order in the declared norm. In the minimal two\hyp term cone tower (source \(F\), terminal \(F'\)) the comparison map \(\phi_{i,\tau}\) is an isomorphism, hence \((\muc,\nuc)=(0,0)\) at fixed \(\tau\).
\emph{Inclusion\hyp type:} stability (non\hyp expansiveness) only; no non\hyp increase is claimed.\\\n\emph{Artifacts:} \texttt{bars.json}, \texttt{spec.json}, \texttt{ext.json}, \texttt{phi.json}; \texttt{run.yaml} (update type, norm choice, and \((\muc,\nuc)\) recorded).

\paragraph{(T3) Filtered\hyp colimit stability [Integration].}
\emph{Input:} tower \(\{F_\lambda\}_\lambda\).\\\n\emph{Assertions:} for each fixed \(\tau\), comparison maps \(\phi_{i,\tau}:\varinjlim_\lambda \Ttau\mathbf{P}_i(F_\lambda)\xrightarrow{\cong}\Ttau\mathbf{P}_i(F_{\lambda_\ast})\); thus \((\muc,\nuc)=(0,0)\) and Type~IV excluded at that scale. \emph{Terminal symbol consistency:} within a run, the same terminal symbol (e.g.\ \(F_{\lambda_\ast}\) or \(F_\infty\)) is used and logged in \texttt{run.yaml}.\\\n\emph{Artifacts:} \texttt{phi.json} with ranks and \((\mu_{i,\tau},\nu_{i,\tau})\); \texttt{run.yaml} (terminal symbol).

\paragraph{(T4) Mirror/tropical pipeline [Integration].}
\emph{Input:} \(X\), tropical flow \(\Trop_\lambda\), realization \(F_\lambda\), mirror functor \(\Mirror\).\\\n\emph{Assertions:} tropical shortening factor \(\kappa\le 1\) implies non\hyp increase of \(\mathrm{PE}^{\le\tau}_i\) (after \(\Ttau\)); mirror transfer is \emph{non\hyp expansive after truncation} at the persistence level; group proxies (if used) meet weak\hyp collapse thresholds (Ch.~8). Spectral eigenvalues are reported in ascending order with the declared norm.\\\n\emph{Artifacts:} \texttt{bars.json}, \texttt{spec.json}, \texttt{ext.json}, \texttt{phi.json} (per \(\lambda\) and on the mirror side); \texttt{run.yaml}.

\paragraph{(T5) Three\hyp layer compatibility [Integration].}
\emph{Input:} Gal\(\to\)Trans\(\to\)Funct data with comparison natural transformations (Ch.~9).\\\n\emph{Assertions:} after applying \(\Ctau\) and \(\mathbf{P}_i\), commutativity holds up to isomorphism in \(\Perskft\) for each degree \(i\); indicators consistent across layers; failure logs record type if mismatches occur.\\\n\emph{Artifacts:} per\hyp layer \texttt{bars.json}, \texttt{spec.json}, \texttt{ext.json}; global \texttt{phi.json}; \texttt{run.yaml}.

\paragraph{(T6) PDE monitoring loop [Regression].}
\emph{Input:} index set \(I\) (time/resolution/parameter), realization \(\mathcal{P}\).\\\n\emph{Assertions:} protocol of Ch.~11 (Decl.~\ref{spec:11-monitor}) holds; stable regime declaration matches (logs \(\mathrm{PE}^{\le\tau}\), spectral indicators and aux-bars, \(\Ext^1\), \((\muc,\nuc)\)); pass\hyp criteria specify \emph{per\hyp degree vs.\ aggregated} reporting and are fixed and logged. Repeatability from \texttt{run.yaml}. Spectral eigenvalues ascending; norm choice declared.\\\n\emph{Artifacts:} \texttt{bars.json}, \texttt{spec.json}, \texttt{aux.json}, \texttt{ext.json}, \texttt{phi.json} over \(I\); \texttt{run.yaml}.

\paragraph{(T7) Saturation gate verification [Integration].}
\emph{Input:} a window \([0,\tau^\ast]\) with candidate saturation (Ch.~11.S).\\\n\emph{Assertions:} verify the quantitative saturation conditions on the window: (a) maximal \emph{finite} bar length \(\le \eta\); (b) interleaving bound \(\le \eta\) eventually; (c) edge gap \(\delta:=\tau^\ast-\max\{b_r<\tau^\ast\}>\eta\). Confirm the \textbf{[Spec]} adoption \(\mathrm{PH}_1(\C_{\tau^\ast}F)=0 \Leftrightarrow \Ext^1(\Rfun(\C_{\tau^\ast}F),k)=0\) is recorded and used only on the saturated window. Spectral reports use ascending ordering and the declared norm.\\\n\emph{Artifacts:} \texttt{bars.json}, \texttt{ext.json}; \texttt{run.yaml} (saturation parameters, norm, decision).

\paragraph{(T8) \(\varepsilon\)\hyp clipping regression [Unit].}
\emph{Input:} paired runs with unclipped vs.\ \(\varepsilon\)\hyp clipped \(\Ttau\mathbf{P}_i\) (Ch.~11.7).\\\n\emph{Assertions:} stability bound for \(\mathrm{PE}^{\le\tau}_i\) under clipping (theoretical upper bound matched up to tolerance); \((\muc,\nuc)\) are computed on unclipped data and remain identical across the pair; logs explicitly distinguish clipped vs.\ unclipped reporting.\\\n\emph{Artifacts:} \texttt{bars.json} (both variants), \texttt{phi.json}; \texttt{run.yaml}.

\paragraph{(T9) MECE window coverage \& event accounting [Unit].}
\emph{Input:} a MECE windowing \(\{[u_k,u_{k+1})\}_k\) and global range \([u_0,U)\).\\\n\emph{Assertions:} (a) \(\sum_k (u_{k+1}-u_k)=U-u_0\); (b) the total number of events (births/deaths counted with multiplicity) on \([u_0,U)\) equals the sum of per\hyp window counts up to rounding tolerance; (c) the collapse threshold \(\tau\) and spectral bin policy \((a,b,\beta)\) are \emph{identical} across windows unless explicitly recorded and justified in \texttt{run.yaml}.\\\n\emph{Artifacts:} \texttt{run.yaml} (windows/coverage\_check block), per\hyp window \texttt{bars.json}.

\paragraph{(T10) A/B commutativity test for reflectors [Unit/Integration].}
\emph{Input:} two persistence\hyp level reflectors \(T_A,T_B\) (e.g.\ length\hyp threshold and birth\hyp window), tolerance \(\eta\ge 0\).\\\n\emph{Assertions:} compute \(\Delta_{\mathrm{comm}}(M;A,B)=d_{\mathrm{int}}(T_A T_B M,\ T_B T_A M)\) on the relevant dataset \(M\). If \(\Delta_{\mathrm{comm}}\le \eta\), adopt \emph{soft\hyp commuting}; else fall back to a fixed order and record \(\Delta_{\mathrm{comm}}\) into \(\delta^{\mathrm{alg}}\) (Appendix~L). Pass if the outcome matches the configured policy in \texttt{run.yaml}.\\\n\emph{Artifacts:} \texttt{run.yaml} (A/B policy: \(\eta\), order, soft\hyp commuting flag), \texttt{bars.json} before/after.

\paragraph{(T11) Restart experiment \& Summability design [Integration/Regression].}
\emph{Input:} a MECE window sequence with collapse thresholds \(\tau_k\), budgets \(\Sigma\delta_k(i)\), and safety margins \(\mathrm{gap}_{\tau_k}\).\\\n\emph{Assertions:} (Restart) empirically verify \(\mathrm{gap}_{\tau_{k+1}}\ge \kappa(\mathrm{gap}_{\tau_k}-\Sigma\delta_k(i))\) for some \(\kappa\in(0,1]\) (record \(\kappa\)); (Summability) verify \(\sum_k \Sigma\delta_k(i)<\infty\) (e.g.\ geometric decay of \(\tau_k,\beta_k\) and bounded step counts); (Pasting) confirm that all B\hyp Gate\(^{+}\) certificates paste to a global certificate.\\\n\emph{Artifacts:} \texttt{run.yaml} (restart/summability fields), per\hyp window gate logs, cumulative certificate.

\paragraph{(T12) Trigger pack verification (domain\hyp restricted) [Integration].}
\emph{Input:} triggers declared for a domain (e.g.\ PDE, Ch.~10.4).\\\n\emph{Assertions:} for each trigger, verify that the declared analytical/numerical condition implies a B\hyp Gate\(^{+}\) failure on the window (e.g.\ sustained enstrophy surge \(\Rightarrow\) aux\hyp bars \(>0\) or \(\mu>0\)); record detection rate and false positives (if any). Triggers are \textbf{[Spec]} and complement (not replace) B\hyp Gate\(^{+}\).\\\n\emph{Artifacts:} \texttt{run.yaml} (trigger set, thresholds), \texttt{aux.json}, \texttt{phi.json}, gate verdicts.

\paragraph{(T13) $\delta$-ledger additivity \& pipeline budget [Integration].}
\emph{Input:} a pipeline of steps \(U_m,\dots,U_1\) with per\hyp step collapses \(C_{\tau_j}\) and bounds \(\delta_j(i,\tau_j)\).\\\n\emph{Assertions:} verify
\[
d_{\mathrm{int}}\!\Big(\Ttau \mathbf{P}_i\big(\Mirror(C_{\tau_m}\!\cdots C_{\tau_1}F)\big),\ \Ttau \mathbf{P}_i\big(C_{\tau_m}\!\cdots C_{\tau_1}\Mirror F\big)\Big)\ \le\ \sum_{j=1}^m \delta_j(i,\tau_j),
\]
and that any post\hyp processing by 1\hyp Lipschitz maps does not increase the bound (Appendix~L). Pass if the measured defect is \(\le\) the recorded budget.\\\n\emph{Artifacts:} \texttt{run.yaml} (\(\delta\)\hyp ledger), \texttt{bars.json} along the pipeline, distance logs.

\paragraph{(T14) Overlap Gate gluing test [Integration].}
\emph{Input:} two charts \(X_1,X_2\) with windows \(W_1,W_2\) and overlap \(X_{12}=X_1\cap X_2\), together with a fixed \(\tau\) and a pair of reflectors \(T_A,T_B\) (as in T10).\\\n\emph{Assertions:} after applying \(\mathbf{P}_i\) and \(\mathbf{T}_\tau\), verify \emph{local equivalence after collapse} on \(X_{12}\): \(\Ttau\mathbf{P}_i(F|_{X_1})\big|_{X_{12}}\simeq \Ttau\mathbf{P}_i(F|_{X_2})\big|_{X_{12}}\) up to the recorded \(\delta\)\hyp budget; verify Čech–\(\Ext^1\) vanishing on \(X_{12}\) (one\hyp way bridge only); run A/B soft\hyp commuting on the overlap and record \(\Delta_{\mathrm{comm}}\) and fallback order if needed; aggregate \(\delta\)\hyp contributions additively. \emph{Output:} construct the glued global truncated object and confirm B\hyp Gate\(^{+}\) acceptance on the glued object (Ch.~11, Decl.~\ref{spec:11-monitor}).\\\n\emph{Artifacts:} \texttt{run.yaml} (\texttt{local\_equiv}, \texttt{overlap\_checks}, A/B policy, \(\delta\)\hyp budget fields), per\hyp chart and overlap \texttt{bars.json}, \texttt{ext.json}; global gate verdict.

\paragraph{(T15) Length spectrum audit [Unit].}
\emph{Input:} a truncated persistence object \(\Ttau\mathbf{P}_i(F)\) and its length spectrum operator \(\Len(\Ttau\mathbf{P}_i(F);[0,\tau])\) (Ch.~2; Ch.~11, Prop.~\ref{prop:11-len}).\\\n\emph{Assertions:} the unordered eigenvalue multiset of \(\Len(\Ttau\mathbf{P}_i(F);[0,\tau])\) matches the clipped bar\hyp length multiset of \(\Ttau\mathbf{P}_i(F)\) up to permutation; the \(L^1\)\hyp mass equals \(\mathrm{PE}_i^{\le \tau}(F)\). A hash or canonical ordering is recorded for audit.\\\n\emph{Artifacts:} \texttt{bars.json}, \texttt{Lambda\_len.json} (optional eigenvalue list or hash), \texttt{run.yaml} (\texttt{Lambda\_len} block).

\subsection*{12.3. Reproducibility and logs}
Every run ships with a manifest \texttt{run.yaml} declaring: sweep \(\tau_{\min}\!:\Delta\tau\!:\tau_{\max}\); spectral policy \((\beta,M(\tau),t)\); discretization (grid/complex, steps); seeds; software versions; tower index set and cone extension \emph{including the terminal symbol used (e.g.\ \(\lambda_\ast\) or \(\infty\))}; pass\hyp criteria (per\hyp degree vs.\ aggregated); norm choice \(\|\cdot\|_{\mathrm{op}}\) or \(\|\cdot\|_{\mathrm{fro}}\); A/B tolerance \(\eta\) (if applicable); Restart constants (\(\kappa\)) and Summability evidence; Overlap Gate status (\texttt{local\_equiv}, \texttt{overlap\_checks}); and file pointers to \texttt{bars.json}, \texttt{spec.json}, \texttt{aux.json}, \texttt{ext.json}, \texttt{phi.json} (optionally with an \texttt{.h5} mirror). All persistence quantities are computed after applying \(\Ttau\); equivalently on \(\Ctau F\); and remain invariant under f.q.i.\ at the persistence layer.

\begin{declaration}[Schema extension and mandatory fields \textbf{[Spec]}]\label{dec:12-schema}
The following \texttt{run.yaml} fields are \emph{mandatory} for auditability:
\begin{itemize}
  \item \texttt{overlap\_checks}: booleans for \texttt{local\_equiv} (post\hyp collapse equivalence on overlaps), \texttt{cech\_ext1\_ok}, \texttt{stability\_band\_ok}.
  \item \texttt{Lambda\_len}: per\hyp degree summary (eigenvalues or hash) of the length spectrum operator on \([0,\tau]\); used in T15.
  \item \texttt{spectral\_policy}: includes \texttt{order: "ascending"} and \texttt{norm: "op"|"fro"}; \texttt{spectral\_bounds} with fields \texttt{lambda\_min}, \texttt{lambda\_max}, and optional Lipschitz tolerance \texttt{lip\_tol}.
  \item \texttt{persistence}: explicit \(\mu/\nu\) totals and the tail isomorphism flag \texttt{phi\_iso\_tail}.
  \item \texttt{budget}: \texttt{sum\_delta} (additive \(\delta\)\hyp ledger total) and \texttt{safety\_margin}; for Restart, per\hyp window \texttt{gap\_tau} (edge gap to the window boundary).
  \item \texttt{ab\_test}: soft\hyp commuting tolerance \(\eta\), policy, and fallback order (T10).
\end{itemize}
\end{declaration}

\begin{remark}[Audit checklist]\label{rk:12-audit}
(i) Constructibility (finite critical set) verified.\;
(ii) Field coefficients fixed (Novikov field allowed at \textbf{[Spec]}\hyp level).\;
(iii) Updates are \emph{deletion\hyp type} for monotonicity; inclusion\hyp type are stability\hyp only.\;
(iv) Interleaving shifts \(\varepsilon_n\) uniformly bounded (tower non\hyp expansion).\;
(v) After applying \(\Ttau\) record \(\mathrm{PE}\), heat trace/spectral tail, aux\hyp bars, \(\Ext^1\), and \((\mu,\nu)\) on the \emph{same window}.\;
(vi) \emph{LC activation order:} apply truncation \(\Ctau\), run persistence\hyp layer checks, then apply \(\Rfun\) with \(\LC\) to compute \(\Ext^1\) (one\hyp way bridge only).\;
(vii) For derived realizations, check PF/BC hypotheses (Appendix~N) \emph{before} kernel/Hecke/induction transfers; all non\hyp expansiveness claims are \emph{after truncation}.\;
(viii) Spectral eigenvalues reported in ascending order; chosen norm \(\|\cdot\|_{\mathrm{op}}\)/\(\|\cdot\|_{\mathrm{fro}}\) declared and logged; terminal symbol consistent across the run.\;
(ix) MECE coverage and event accounting satisfied;\;
(x) A/B commutativity test configured (\(\eta\)) and logged (soft\hyp commuting vs.\ fallback);\;
(xi) Restart/Summability evidenced with \(\kappa\) and \(\sum \Sigma\delta\);\;
(xii) Overlap Gate: \texttt{local\_equiv}, Čech–\(\Ext^1\) status, and overlap \(\delta\)\hyp budgets recorded (T14);\;
(xiii) Length spectrum audit recorded (\texttt{Lambda\_len}) and matched to clipped bar lengths (T15).
\end{remark}

\noindent\emph{Manifest template (YAML).}
\small
\begin{verbatim}
coeff_field: "k"            # or "Novikov(q)" [Spec-level]
tau_window: [0.05, 1.0]     # start, end
tau_step: 0.05
spectral:
  tail_beta: 2
  tail_cutoff_M_of_tau: "floor(0.5 * tau^1.5)"
  heat_t: [0.5*tau^-2, 1.0*tau^-2]
  aux_bins: {a: 0.0, beta: 0.02, bins: 96, boundary: "right-open"}
spectral_policy:
  order: "ascending"
  norm: "op"
spectral_bounds:
  lambda_min: 1.0e-6
  lambda_max: 10.0
  lip_tol: 0.02
tower:
  eps_interleave_max: 0.02
  terminal_symbol: "infty"   # or "lambda_star"
  cone_extension: true
ab_test:
  eta: 0.01
  policy: "soft-commuting"   # or "fallback:A_then_B"
restart_summability:
  kappa_min: 0.8
  sum_delta_bound: 0.05
windows:
  domain: [[0,1), [1,2), [2,3)]
  collapse_tau: 0.08
coverage_check:
  length_sum: 3.0
  length_target: 3.0
  events_sum_equals_global: true
overlap_checks:
  local_equiv: true
  cech_ext1_ok: true
  stability_band_ok: true
persistence:
  PH1_zero: true
  Ext1_zero: true
  mu: 0
  nu: 0
  phi_iso_tail: true
Lambda_len:
  degree: 1
  tau: 0.08
  audit: "hash:2f4c...d1"
record:
  bars: true
  PE: {report: "per-degree", clipping: "epsilon=0.02"}
  aux: {lifetime_min_frames: 3}
  heat_trace: {ordering: "ascending", norm: "op"}
  ext1: true
  mu_nu: true
budget:
  sum_delta: 0.011
  safety_margin: 0.025
  gap_tau: 0.03
gate:
  accept: true
notes: "deletion-type updates only; LC with Rfun after truncation; PF/BC verified where applicable"
\end{verbatim}
\normalsize

\subsection*{12.4. Open problems (selected)}
\begin{remark}[Open problems]\label{rk:12-open}
\hfill
\begin{enumerate}
  \item \textbf{Strengthening the bridge \( \mathrm{PH}_1\to \Ext^1 \).} Identify domain\hyp wise sufficient conditions (beyond (B1)–(B3)) ensuring vanishing of \(\Ext^1\) from quantitative decay of persistence energy/spectral tails. No converse/equivalence is claimed.
  \item \textbf{Persistence\hyp level colimit criteria.} Sharp hypotheses guaranteeing \((\muc,\nuc)=(0,0)\) across broader indexing classes (beyond objectwise degreewise colimits).
  \item \textbf{Failure lattice refinement.} Finer invariants separating pure/mixed failures and detecting “invisible’’ Type~IV precursors at nearby scales.
  \item \textbf{Spectral–persistence calibration.} Quantitative bounds linking the collapse energy \(\|\mathbf{CE}^{\le \tau}(F)\|_{1}\) and \(\mathrm{ST}_{\beta}^{\ge M(\tau)}\), robust to noise and discretization.
  \item \textbf{Weak group collapse.} Relate persistence\hyp level proxies (Ch.~8) to algebraic invariants (virtual nilpotence/solvabilization) without leaving the implementable range.
  \item \textbf{Arithmetic towers.} Domain\hyp specific templates ensuring tower stability and clarifying the relation (if any) between collapse diagnostics and Selmer/class growth; maintain the non\hyp identity with classical Iwasawa \(\mu\).
  \item \textbf{Langlands layer compatibility.} Minimal comparison data ensuring truncated commutativity across Gal\(\to\)Trans\(\to\)Funct in practical pipelines.
  \item \textbf{PDE program.} Conditions under which persistence\hyp guided regularization predicts classical regularity regimes while remaining purely programmatic.
  \item \textbf{Universality of \(\mathbf{T}_\tau\) as Serre localization.} Minimal assumptions (within constructible \(1\)D persistence) under which \(\mathbf{T}_\tau\) enjoys a universal property characterizing the implementable range.
\end{enumerate}
\end{remark}

\subsection*{12.5. Final guard\hyp rails}
\begin{remark}[Scope and non\hyp claims]\label{rk:12-guards}
All specifications are confined to the persistence/spectral/categorical layers in the implementable range and are verifiable by the test suite above. No number\hyp theoretic identity, analytic regularity theorem, or group trivialization is asserted. In particular, no claim of \(\mathrm{PH}_1\Leftrightarrow \Ext^1\) is made; only the one\hyp way implication under (B1)–(B3) is used. The obstruction \(\muc\) is a collapse diagnostic and is distinct from the classical Iwasawa \(\mu\).
\end{remark}

\subsection*{12.6. Effect and auditability}
\begin{remark}[Effect of the extensions]
Relative to v16.0, the test harness is strengthened along four axes: (i) Overlap Gate gluing (T14) enforces post\hyp collapse local equivalence, Čech–\(\Ext^1\) vanishing on overlaps, and records A/B soft\hyp commuting with \(\delta\)\hyp budgets; (ii) A/B soft\hyp commuting (T10) is parameterized and auditable at the manifest level; (iii) Restart/Summability (T11) becomes quantitatively checkable via \texttt{gap\_tau}, \(\kappa\), and \(\sum\delta\); (iv) Saturation Gate (T7) is anchored to explicit window parameters. The schema extension (Decl.~\ref{dec:12-schema}) mandates \texttt{local\_equiv}, \texttt{overlap\_checks}, \texttt{Lambda\_len}, \texttt{spectral\_bounds}, \(\mu/\nu\), \(\sum\delta\), and \(\mathrm{gap}_{\tau}\), improving third\hyp party auditability.
\end{remark}

\subsection*{12.7. Conclusion}
This closing chapter consolidates a complete, testable interface for the program: a precise badge policy, a uniform notation layer, and a formal test suite spanning stability, monotone updates, filtered\hyp colimits, mirror/tropical flows, Langlands triples, PDE pipelines, and Overlap Gate gluing. All persistence\hyp layer quantities are computed after applying \(\Ttau\); equivalently on \(\Ctau F\); spectral indicators are normalized (eigenvalues ascending; norm declared) with aux\hyp bar policies fixed; and categorical checks are performed only in the one\hyp way direction sanctioned by Part~I. Reproducibility is enforced by a single manifest and a minimal artifact set, now with mandatory overlap and length\hyp spectrum fields. With these guard\hyp rails, the \emph{implementable range} is not merely a blueprint but an executable methodology, inviting careful extensions along the open directions of Remark~\ref{rk:12-open}, while preserving the conservative stance articulated in Remark~\ref{rk:12-guards}.

\subsection*{12.8. Completion note}
\begin{remark}[No further supplementation required]
This chapter fully integrates: (i) MECE window tests and event accounting; (ii) A/B commutativity tests with tolerance and fallback; (iii) Restart experiments with Summability verification; (iv) Trigger pack validation; (v) $\delta$-ledger additivity checks; (vi) Overlap Gate gluing (T14); (vii) Length spectrum audit (T15); (viii) a manifest schema with all required audit fields. All items are consistent with the v16.0 guard\hyp rails and cross\hyp reference the proven core; no additional supplementation is needed for operational use as a formal test suite.
\end{remark}



% =========================
\section*{Notation and Conventions (reinforced v16.0)}
% =========================
\phantomsection
\addcontentsline{toc}{section}{Notation and Conventions}

\paragraph{Base field and ambient categories.}
Fix a coefficient field \(k\) (Appendices~N/O also admit a field \(\Lambda\); when used, replace \(k\) by \(\Lambda\) everywhere).
Let \(\mathsf{Vect}_k\) be the abelian category of finite\hyp dimensional \(k\)\hyp vector spaces.
Write \([\mathbb{R},\mathsf{Vect}_k]\) for functors \((\mathbb{R},\le)\to \mathsf{Vect}_k\).

\paragraph{Constructible persistence and standing identification.}
We write
\[
\Perskft\ \subset\ [\mathbb{R},\mathsf{Vect}_k]
\]
for the full subcategory of \emph{constructible} persistence modules (pointwise finite\hyp dimensional with locally finite critical set on bounded windows).
Throughout the paper we \emph{identify} the “finite\hyp type” category with this constructible subcategory and use the symbol \(\Perskft\) uniformly.

\paragraph{Filtered objects and persistence.}
\(\mathsf{FiltCh}(k)\) denotes filtered chain complexes of finite\hyp dimensional \(k\)\hyp spaces (filtered quasi\hyp isomorphism abbreviated f.q.i.).
For \(i\in\mathbb{Z}\) the degree–\(i\) persistence functor is
\[
\mathbf{P}_i:\ \mathsf{FiltCh}(k)\longrightarrow \Perskft,\qquad \mathbf{P}_i(F)(t)=\mathrm{H}_i(F^t).
\]
Realizations from other formalisms into \(\mathsf{FiltCh}(k)\) are denoted \(\mathcal{R}(-)\), \(\mathcal{F}(-)\) as appropriate.

\paragraph{Reflection/truncation vs.\ window clipping.}
For \(\tau\ge 0\), let \(\mathsf{E}_\tau\subset\Perskft\) be the Serre subcategory generated by bars of length \(\le \tau\).
The \emph{reflector} (truncation)
\[
\mathbf{T}_\tau:\ \Perskft\longrightarrow \mathsf{E}_\tau^\perp
\]
is exact, idempotent, and left adjoint to the inclusion \(\iota_\tau:\mathsf{E}_\tau^\perp\hookrightarrow\Perskft\) (Appendix~A, Theorem~\ref{A:thm:localization}); it is \(1\)\hyp Lipschitz for the interleaving metric (Appendix~A, Proposition~\ref{A:prop:lipschitz}).
On filtered complexes we use a collapser \(C_\tau\) with a natural (up to f.q.i.) identification
\[
\mathbf{P}_i(C_\tau F)\ \cong\ \mathbf{T}_\tau(\mathbf{P}_iF)\qquad\text{(natural in \(F,i\)).}
\]
For \(\sigma\ge 0\), the \emph{window clip}
\[
\mathbf{W}_{\le\sigma}=(i_{\le\sigma})^0_!\circ i_{\le\sigma}^\ast:\ \Perskft\to\Perskft
\]
restricts to \([0,\sigma]\) and extends by zero; it is also \(1\)\hyp Lipschitz (Appendix~I).
\emph{Warning.} \(\mathbf{T}_\tau\) (delete bars of length \(\le\tau\)) and \(\mathbf{W}_{\le\sigma}\) (clip to a window) play distinct roles and must not be conflated.

\paragraph{Interleaving metric and shifts.}
On \(\Perskft\) the interleaving metric \(d_{\mathrm{int}}\) equals the bottleneck distance in the constructible \(1\)D setting.
The time shift is \((S^\varepsilon M)(t):=M(t+\varepsilon)\); shifts commute canonically with \(\mathbf{T}_\tau\), hence \(\mathbf{T}_\tau\) is \(1\)\hyp Lipschitz.

\paragraph{Unique comparison order (measurement layer).}
All comparisons (local, overlap, global) follow the unique order
\[
\boxed{\ \text{for each } t\ \Longrightarrow\ \text{apply } \mathbf{P}_i \ \Longrightarrow\ \text{apply } \mathbf{T}_\tau \ \Longrightarrow\ \text{compare in }\Perskft\ }.
\]
This order is mandatory for audits, spectral alignment, and overlap gluing (Ch.~11, \S11.0 bis; Ch.~5).

\paragraph{Barcodes, events, and endpoint convention.}
Barcodes use half\hyp open intervals \(I=[b,d)\) with \(d\in\mathbb{R}\cup\{\infty\}\) and multiplicity \(m(I)\in\mathbb{Z}_{\ge1}\).
Any consistent open/closed choice yields the same clipped lengths and event sets.
For \(\tau\ge 0\) the clipped length is
\[
\ell_{[0,\tau]}(I):=\max\{0,\min\{d,\tau\}-\max\{b,0\}\}.
\]
Given \(\tau_0>0\), the finite event set in degree \(i\) is
\[
\mathsf{Ev}_i(F;\tau_0)=\{0,\tau_0\}\ \cup\ \bigl(\{b\in[0,\tau_0]\}\cap\mathrm{births}\bigr)\ \cup\ \bigl(\{d\in[0,\tau_0]\}\cap\mathrm{deaths}\bigr).
\]
Endpoint conventions and the handling of infinite bars follow Appendix~A, Remark~\ref{A:rk:endpoints}.

\paragraph{Betti curves and Betti integral.}
\(\beta_i(F;t):=\dim_k \mathrm{H}_i(F^t)\) is c\`adl\`ag and piecewise constant on bounded windows.
The (clipped) Betti integral is
\[
\mathrm{PE}_i^{\le\tau}(F)=\int_0^\tau \beta_i(F;t)\,dt\ =\ \sum_{I\in\mathcal{B}_i(F)} m(I)\,\ell_{[0,\tau]}(I)\qquad\text{(Appendix~H).}
\]

\paragraph{Length spectrum operator.}
For \(M\in\Perskft\) with barcode \(M\simeq\bigoplus_j I[b_j,d_j)\) and a right\hyp open window \(W=[u,u')\), the \emph{length spectrum} operator \(\Len(M;W)\) (Ch.~2) is diagonal on a bar\hyp basis with eigenvalues \(\ell_W(I[b_j,d_j))\).
Its multiset of eigenvalues equals the multiset of clipped bar lengths and is invariant under isomorphisms \(M\simeq M'\).
In particular, for \(\Ttau\mathbf{P}_i(F)\) and \(W=[0,\tau]\),
\[
\mathrm{PE}_i^{\le \tau}(F)\ =\ \|\Len(\Ttau\mathbf{P}_i(F);[0,\tau])\|_{L^1}\,,
\]
hence the total collapse energy is an isomorphism invariant of the truncated persistence (Ch.~11, \S11.0+).

\paragraph{Spectral policy, ordering, and matrix norms.}
Spectral indicators are computed on \(L(\Ctau F)\) (normalized combinatorial Hodge Laplacian).
Positive eigenvalues are reported in \emph{ascending} order.
The matrix norm used for tolerances is declared as \(\|\cdot\|_{\mathrm{op}}\) (operator) or \(\|\cdot\|_{\mathrm{fro}}\) (Frobenius) and recorded in the run manifest.
Optional spectral bounds (\(\lambda_{\min},\lambda_{\max}\)) and a Lipschitz tolerance \texttt{lip\_tol} may be specified (Ch.~11, \S11.2; Appendix~G).

\paragraph{Auxiliary spectral bars (aux-bars) and cumulative profile.}
Fix a spectral window \([a,b]\) and a bin width \(\beta>0\).
Define right\hyp open bins \(I_r=[a+r\beta,a+(r+1)\beta)\) and occupancies
\[
E_r(j)=\#\{m:\ \lambda_m(j)\in I_r\}
\]
on the spectrum of \(L(\Ctau F_j)\) (under/overflow recorded).
The cumulative profile \(C_r(j):=\sum_{s=r}^{R-1}E_s(j)\) is \emph{non\hyp increasing} under deletion\hyp type updates (Dirichlet/principal restrictions, Schur complements, Loewner contractions) and \emph{stable up to a bin shift} under \(\varepsilon\)\hyp continuations (Appendix~E).
Aux-bars and \(C_r\) are \emph{auxiliary} for gates and do not override persistence\hyp layer verdicts.

\paragraph{Towers and diagnostics; stability bands.}
A \emph{tower} is a directed system \(F=(F_n)_{n\in I}\) with colimit \(F_\infty\).
For \(i\in\mathbb{Z}\), \(\tau\ge 0\), the comparison map is
\[
\phi_{i,\tau}(F):\ \varinjlim_{n}\ \mathbf{T}_\tau\!\big(\mathbf{P}_i(F_n)\big)\longrightarrow \mathbf{T}_\tau\!\big(\mathbf{P}_i(F_\infty)\big).
\]
Set \(\mu_{i,\tau}(F):=\gdim\ker\phi_{i,\tau}(F)\) and \(\nu_{i,\tau}(F):=\gdim\operatorname{coker}\phi_{i,\tau}(F)\); the totals are
\[
\muc(F):=\sum_{i}\mu_{i,\tau}(F),\qquad \nuc(F):=\sum_{i}\nu_{i,\tau}(F),
\]
which are finite by bounded degree.
Cofinal restriction leaves \((\muc,\nuc)\) unchanged; finite direct sums add; composition is subadditive (Appendix~J).
If \(\phi_{i,\tau}\) is an isomorphism then \((\muc,\nuc)=(0,0)\).
A \emph{stability band} is a contiguous \(\tau\)\hyp range where \(\phi_{i,\tau}\) is an isomorphism (detected by a \(\tau\)\hyp sweep and robust under refinement; Appendix~J); sufficient conditions include (S1)~\(\mathbf{T}_\tau\)\hyp colimit commutation and apex colimit, (S2)~no near\hyp \(\tau\) accumulation from below, (S3)~\(\mathbf{T}_\tau\)\hyp Cauchy with compatible cocone (Appendix~D/J).

\paragraph{Overlap Gate (local\texorpdfstring{$\to$}{->}global gluing).}
Given a windowed cover \(\{(X_\alpha,W_\alpha)\}\) (right\hyp open windows), the \emph{Overlap Gate} (Ch.~1; Ch.~5) requires, after truncation by \(\mathbf{T}_\tau\),
\begin{itemize}
  \item local post\hyp collapse equivalence on overlaps up to the recorded \(\delta\)\hyp budget (quantitative commutation);
  \item Čech–\(\Ext^1\) acyclicity (degree \(1\));
  \item stability\hyp band condition \((\mu,\nu)=(0,0)\) and near\hyp \(\tau\) non\hyp accumulation.
\end{itemize}
When all overlaps pass, local truncated objects glue (uniquely up to isomorphism) in \(\Perskft\).
Run manifests must record \texttt{overlap\_checks}: \texttt{local\_equiv} (true/false), \texttt{cech\_ext1\_ok}, and \texttt{stability\_band\_ok}.

\paragraph{A/B soft\hyp commuting and Mirror commutation; pipeline budget.}
For exact reflectors \(T_A,T_B\) in \(\Perskft\), the A/B commutation defect is
\[
\Delta_{\mathrm{comm}}(M;A,B)=d_{\mathrm{int}}(T_AT_BM,\ T_BT_AM).
\]
Given a tolerance \(\eta\), if \(\Delta_{\mathrm{comm}}\le\eta\), accept \emph{soft\hyp commuting}; else fix an order (e.g.\ \(T_B\circ T_A\)) and \emph{record} \(\Delta_{\mathrm{comm}}\) as \(\delta^{\mathrm{alg}}\) in the \(\delta\)\hyp ledger (Appendix~K/L).
For Mirror/Transfer functors \(\Mirror\), assume a natural \(2\)\hyp cell \(\Mirror\circ C_\tau\Rightarrow C_\tau\circ\Mirror\) with a uniform bound \(\delta(i,\tau)\ge 0\) in \(d_{\mathrm{int}}\), additive along pipelines and \emph{non\hyp increasing} under \(1\)\hyp Lipschitz post\hyp processing (Appendix~L).
On a window \(W\) and degree \(i\), the pipeline budget aggregates
\[
\Sigma\delta(i,\tau)\ =\ \sum_{\text{Mirror--Collapse}}\delta(i,\tau)\ +\ \sum_{\text{A/B fails}}\Delta_{\mathrm{comm}}\ +\ \sum_{\text{audits}}\bigl(\delta^{\mathrm{disc}}+\delta^{\mathrm{meas}}\bigr).
\]
The \emph{safety margin} \(\mathrm{gap}_\tau>0\) is a configured slack per window and degree; B\hyp Gate\(^{+}\) requires \(\mathrm{gap}_\tau>\Sigma\delta(i,\tau)\). Across windows, Restart and Summability hold if there exists \(\kappa\in(0,1]\) with
\[
\mathrm{gap}_{\tau_{k+1}}\ \ge\ \kappa\bigl(\mathrm{gap}_{\tau_k}-\Sigma\delta_k(i)\bigr),\qquad \sum_k \Sigma\delta_k(i)<\infty
\]
(Appendix~J).

\paragraph{Categorical check and saturation gate (window\hyp local equivalence).}
With \(\Qtest=\{k[0]\}\), we test \(\Ext^1(\mathcal{R}(C_\tau F),Q)=0\) \emph{after truncation}.
The one\hyp way bridge \(\mathrm{PH}_1\Rightarrow \Ext^1\) is used only under (B1)–(B3).
On a \emph{saturated} window \([0,\tau^\ast]\) (bounded finite bars, stabilization, edge gap exceeding drift), we adopt the temporary local equivalence
\[
\mathrm{PH}_1(\C_{\tau^\ast}F)=0\quad\Longleftrightarrow\quad \Ext^1\bigl(\mathcal{R}(\C_{\tau^\ast}F),k\bigr)=0\quad\text{(within that window only)}.
\]

\paragraph{Reproducibility (manifest) and mandatory fields.}
The run manifest \texttt{run.yaml} records: \(\tau\)\hyp sweep and windows (MECE \& coverage checks); spectral policy \((\beta,M(\tau),t)\) and \emph{mandatory} fields \texttt{order="ascending"}, \texttt{norm="op"|"fro"}; spectral bounds (\(\lambda_{\min},\lambda_{\max}\), optional \texttt{lip\_tol}); aux\hyp bar bins \(([a,b],\beta)\) with right\hyp open convention and under/overflow; A/B test parameters \(\eta\), policy, fallback order; tower terminal symbol and cone extension; \(\delta\)\hyp ledger per step and \texttt{sum\_delta}; safety margin and \texttt{gap\_tau}; overlap checks (\texttt{local\_equiv}, \texttt{cech\_ext1\_ok}, \texttt{stability\_band\_ok}); persistence summary (PH/Ext/\(\mu,\nu\), tail isomorphism); an optional \texttt{Lambda\_len} audit (eigenvalues or hash) for \([0,\tau]\).
All persistence\hyp layer quantities are computed after applying \(\mathbf{T}_\tau\); equivalently on \(\Ctau F\); spectral indicators are normalized (ascending eigenvalues; declared norm); categorical checks are executed only in the one\hyp way direction sanctioned in Part~I.

\paragraph{Global guard\hyp rails and non\hyp claims.}
All statements are confined to the persistence/spectral/categorical layers in the implementable range.
No number\hyp theoretic identity, analytic regularity theorem, or group trivialization is asserted.
No claim of \(\mathrm{PH}_1\Leftrightarrow \Ext^1\) is made; only the one\hyp way implication \(\mathrm{PH}_1\Rightarrow \Ext^1\) (under (B1)–(B3)) is used.
The collapse obstruction \(\muc\) is a persistence\hyp level diagnostic and is distinct from the classical Iwasawa \(\mu\).

\medskip
\noindent\emph{Abbreviations.}
f.q.i.\ = filtered quasi\hyp isomorphism;\quad
c\`adl\`ag = right\hyp continuous with left limits;\quad
“window” \(=\) interval \([0,\tau]\) with \(\tau\ge0\).



% =========================
\appendix
\section*{Appendix A. Constructible Persistence: Abelianity and Serre Localization (reinforced)}
% =========================

\addcontentsline{toc}{section}{Appendix A. Constructible Persistence: Abelianity and Serre Localization (reinforced)}

\begingroup
Throughout this appendix, fix a field \(k\).
Write \(\Pers_k\) for the category of right-continuous persistence modules
\(M:(\mathbb{R},\le)\to\Vect_k\) with structure maps \(M(t\le t')\).
We denote by \(\Pers^{\mathrm{ft}}_k\subset\Pers_k\) the \emph{constructible} (finite-type) subcategory used in the main text.

\medskip
\noindent\textbf{Global conventions.}
(i) All \(\Ext\)-tests are taken against \(Q=k[0]\) (the unit interval module supported at a point).
(ii) Windowed energies use an exponent \(\alpha>0\) (default \(\alpha=1\)).
(iii) References to appendices use the tilde style (e.g.\ Appendix~D); failure types use the dash style
\(\mathrm{Type\ I\text{--}II}\), \(\mathrm{Type\ III}\), \(\mathrm{Type\ IV}\).
(iv) Notational disambiguation: the reflector (localization) functor is denoted \(\mathbf{T}_\tau\); truncations/clippings on domains are denoted \(\mathbf{Tr}_\tau\) or \(\clip_{[a,b)}\). This resolves the potential collision sometimes found in informal notes where \(\mathbf{T}_\tau\) was used for truncation.
(v) When window partitions are used (\(\MECE\); §A.6), \emph{half-open with right-inclusion} is the standing endpoint convention for domain windows and spectral bins; coverage checks are mandatory.
(vi) All results in §§A.2–A.5 are stated and used in the one-parameter (1D), field-coefficient, right-continuous, constructible setting.

% -------------------------
\subsection*{A.1. Constructible objects}
% -------------------------
\begin{definition}[Constructible / finite-type]\label{A:def:constructible}
A persistence module \(M\in\Pers_k\) is \emph{constructible} (finite-type) if on every bounded interval
\([a,b]\subset\mathbb{R}\) it has a \emph{finite critical set}: there exist
\(a=t_0<t_1<\dots<t_N=b\) such that each structure map
\(M(t\le t')\) is an isomorphism whenever \(t,t'\in (t_j,t_{j+1})\) for some \(j\).
Equivalently, \(M\) is pointwise finite-dimensional and admits a barcode decomposition as a
\emph{locally finite direct sum of interval modules}, i.e.\ only finitely many intervals intersect any bounded window.
We write \(\Pers^{\mathrm{ft}}_k\) for the full subcategory of such modules.
\end{definition}

\begin{remark}
In the 1D, field-coefficient, right-continuous setting, the equivalence above is standard (barcode decomposition). All constructions below (kernels, cokernels, torsion, truncation/clipping) preserve constructibility and are controlled by finitely many events on bounded windows; see the references at the end of this appendix.
\end{remark}

% -------------------------
\subsection*{A.2. Abelianity}
% -------------------------
\begin{proposition}\label{A:prop:abelian}
\(\Pers^{\mathrm{ft}}_k\) is an abelian category.
Moreover, for a morphism \(f:M\to N\) in \(\Pers^{\mathrm{ft}}_k\),
kernels and cokernels are computed pointwise in \(\Vect_k\) and remain constructible.
\end{proposition}

\begin{proof}
Evaluation at each \(t\in\mathbb{R}\) is exact in \(\Vect_k\), hence pointwise kernels and cokernels define functorial sub/quotient persistence modules.
Constructibility is preserved: on any bounded window one refines the break sets of \(M,N\) to a finite set controlling \(\Ker f\) and \(\Coker f\).
Exactness axioms follow objectwise; hence \(\Pers^{\mathrm{ft}}_k\) is abelian with pointwise exactness.
\end{proof}

% -------------------------
\subsection*{A.3. The \texorpdfstring{$\tau$}{tau}-ephemeral Serre subcategory}
% -------------------------
Fix \(\tau>0\).
Let \(I[a,b)\) denote the interval module supported on \([a,b)\) (with the half-open, right-inclusion convention).

\begin{definition}[\(\tau\)-ephemeral subcategory]\label{A:def:Etau}
Let \(\mathsf{E}_\tau\subset\Pers^{\mathrm{ft}}_k\) be the smallest full subcategory
containing all interval modules \(I[a,b)\) with length \(b-a\le \tau\) and closed under subobjects, quotients, and extensions.
We call \(\mathsf{E}_\tau\) the \emph{\(\tau\)-ephemeral} (or \(\tau\)-torsion) subcategory.
\end{definition}

\begin{lemma}\label{A:lem:Serre}
\(\mathsf{E}_\tau\) is a Serre subcategory of \(\Pers^{\mathrm{ft}}_k\), and it is hereditary as a torsion class: it is closed under subobjects, quotients, and extensions, and subobjects of objects in \(\mathsf{E}_\tau\) remain in \(\mathsf{E}_\tau\).
\end{lemma}

\begin{proof}
By Definition~\ref{A:def:Etau}, \(\mathsf{E}_\tau\) is generated under extensions by length-\(\le\tau\) intervals and is closed under subobjects and quotients; hence it is Serre.
Hereditariness follows because any subobject of a finite direct sum of length-\(\le\tau\) intervals admits a finite filtration with composition factors of length \(\le\tau\).
\end{proof}

\begin{remark}[Endpoint conventions]\label{A:rk:endpoints}
All statements in this appendix are insensitive to the choice of open/closed endpoints on interval modules.
For definiteness we use half-open intervals \([a,b)\); changing endpoint conventions does not affect lengths, barcode decompositions, interleaving/bottleneck distances, or any categorical constructions below.
\end{remark}

\begin{remark}[1D constructible context]
In the one-parameter (1D) constructible setting considered here, \(\mathsf{E}_\tau\) is a hereditary torsion class inside the abelian category \(\Pers^{\mathrm{ft}}_k\), which is \emph{locally finite on bounded windows}.
This is standard in the barcode framework (see Crawley–Boevey 2015; Chazal–de~Silva–Glisse–Oudot 2016).
\end{remark}

% -------------------------
\subsection*{A.3.1. The torsion pair and maximal \texorpdfstring{$\tau$}{tau}-ephemeral subobject}
% -------------------------
Define the \emph{\(\tau\)-local (orthogonal)} subcategory
\[
\Pers^{\mathrm{ft}}_{k,\tau\text{-loc}}\n\ :=\\n\bigl\{\,X\in\Pers^{\mathrm{ft}}_k\ \big|\ \Hom(E,X)=0=\Ext^1(E,X)\ \text{ for all }E\in\mathsf{E}_\tau\,\bigr\}.
\]
Then \((\mathsf{E}_\tau,\ \Pers^{\mathrm{ft}}_{k,\tau\text{-loc}})\) is a torsion pair in \(\Pers^{\mathrm{ft}}_k\): for each \(M\) there exists a functorial short exact sequence
\[
0\ \longrightarrow\ t_\tau(M)\ \longrightarrow\ M\ \longrightarrow\ f_\tau(M)\ \longrightarrow\ 0
\]
with \(t_\tau(M)\in\mathsf{E}_\tau\) and \(f_\tau(M)\in \Pers^{\mathrm{ft}}_{k,\tau\text{-loc}}\), and \(\Hom(\mathsf{E}_\tau,\Pers^{\mathrm{ft}}_{k,\tau\text{-loc}})=0\).
The subobject \(t_\tau(M)\) is the maximal \(\tau\)-ephemeral subobject of \(M\).

\begin{proof}[Sketch]
Serre-ness of \(\mathsf{E}_\tau\) and finite-length on bounded windows imply existence and functoriality of maximal \(\mathsf{E}_\tau\)-subobjects by standard torsion theory in abelian length categories. Orthogonality characterizes the torsion-free class as the right orthogonal to \(\mathsf{E}_\tau\) for both \(\Hom\) and \(\Ext^1\) in our 1D constructible setting.
\end{proof}

% -------------------------
\subsection*{A.4. The reflector \texorpdfstring{$\mathbf{T}_\tau\dashv\iota_\tau$}{T\_\tau ⊣ ι\_\tau} and exactness (constructible 1D)}
% -------------------------
Let \(\iota_\tau:\Pers^{\mathrm{ft}}_{k,\tau\text{-loc}}\hookrightarrow\Pers^{\mathrm{ft}}_k\) be the inclusion.

\begin{theorem}[Exact reflective localization]\label{A:thm:localization}
The Serre quotient functor
\[
\pi_\tau:\ \Pers^{\mathrm{ft}}_k\ \longrightarrow\ \Pers^{\mathrm{ft}}_k/\mathsf{E}_\tau
\]
is exact. In the 1D constructible setting there is a canonical exact equivalence of abelian categories
\[
\Pers^{\mathrm{ft}}_k/\mathsf{E}_\tau \ \simeq\ \Pers^{\mathrm{ft}}_{k,\tau\text{-loc}}.
\]
Composing \(\pi_\tau\) with this equivalence yields a functor
\[
\mathbf{T}_\tau:\ \Pers^{\mathrm{ft}}_k\ \longrightarrow\ \Pers^{\mathrm{ft}}_{k,\tau\text{-loc}}
\]
which is left adjoint to \(\iota_\tau\) and is exact (hence additive).
Consequently, \(\mathbf{T}_\tau\) preserves finite limits and finite colimits, and as a left adjoint it preserves all colimits that exist in \(\Pers^{\mathrm{ft}}_k\) (interpreted via Remark~\ref{A:rk:filtered-colimits}).
\end{theorem}

\begin{proof}
Since \(\mathsf{E}_\tau\) is Serre (Lemma~\ref{A:lem:Serre}), the abelian Serre quotient exists and \(\pi_\tau\) is exact.
By Gabriel localization for Serre subcategories, in our 1D constructible context the quotient \(\Pers^{\mathrm{ft}}_k/\mathsf{E}_\tau\) is equivalent to the full subcategory of \(\mathsf{E}_\tau\)-\emph{local (orthogonal)} objects, i.e.\ those \(X\) with \(\Hom(E,X)=\Ext^1(E,X)=0\) for all \(E\in\mathsf{E}_\tau\).
Transporting along this equivalence yields the adjunction \(\mathbf{T}_\tau\dashv\iota_\tau\).
Exactness and additivity imply preservation of finite limits and finite colimits; being a left adjoint implies preservation of (existing) colimits, with the filtered-colimit policy as in Remark~\ref{A:rk:filtered-colimits}.
\end{proof}

\begin{proposition}[Behavior on barcodes]\label{A:prop:barcode-behavior}
Let \(M\simeq \bigoplus_{j} I[a_j,b_j)\) be the barcode decomposition (locally finite on bounded windows).
Then
\[
\mathbf{T}_\tau M\ \simeq\ \bigoplus_{\,b_j-a_j>\tau}\ I[a_j,b_j)\,,\n\qquad\nt_\tau(M)\ \simeq\ \bigoplus_{\,b_j-a_j\le \tau}\ I[a_j,b_j)\,.
\]
In particular, \(\mathbf{T}_\tau\) \emph{forgets} all bars of length \(\le\tau\) and splits any mixed extensions between short and long bars.
\end{proposition}

\begin{proof}
Every constructible module decomposes as a locally finite direct sum of interval modules; \(\mathsf{E}_\tau\) is generated by intervals of length \(\le\tau\).
The maximal \(\mathsf{E}_\tau\)-subobject of \(M\) is the direct sum of the short bars, and the quotient is the direct sum of the long bars. Orthogonality of the quotient to \(\mathsf{E}_\tau\) ensures that mixed extensions split in the localized category; the reflector picks the orthogonal factor.
\end{proof}

\begin{remark}[Filtered colimits: functor-category computation and return to constructible]
\label{A:rk:filtered-colimits}
Filtered colimits are computed objectwise in the functor category \([\mathbb{R},\Vect_k]\), and \(\mathbf{T}_\tau\) commutes with those colimits there (as a left adjoint).
A filtered colimit of constructible modules may exit \(\Pers^{\mathrm{ft}}_k\).
In applications we either: (i) restrict to towers that remain constructible degreewise; or (ii) compute in \([\mathbb{R},\Vect_k]\), apply \(\mathbf{T}_\tau\), and \emph{verify} that the result returns to \(\Pers^{\mathrm{ft}}_k\) (finite critical set on bounded windows).
\textbf{This policy is assumed throughout the paper whenever filtered colimits appear, and no claim is made outside this regime.}
\end{remark}

\begin{lemma}[Compatibility with restriction/clipping]\label{A:lem:clip}
Let \(\clip_{[u,v)}:\Pers^{\mathrm{ft}}_k\to\Pers^{\mathrm{ft}}_k\) denote clipping to \([u,v)\) (half-open, right-inclusion).
Then \(\mathbf{T}_\tau\) commutes with clipping: \(\mathbf{T}_\tau\circ \clip_{[u,v)} \cong \clip_{[u,v)}\circ \mathbf{T}_\tau\).
\end{lemma}

\begin{proof}
Clipping is exact and preserves interval lengths, hence preserves \(\mathsf{E}_\tau\). It therefore descends to the Serre quotient and commutes with \(\pi_\tau\); transporting across the equivalence in Theorem~\ref{A:thm:localization} gives the claim.
\end{proof}

% -------------------------
\subsection*{A.5. Shift-commutation, monotonicity in \texorpdfstring{$\tau$}{tau}, and 1-Lipschitz continuity}
% -------------------------
For \(\varepsilon\ge 0\), let \(S^\varepsilon:\Pers_k\to\Pers_k\) be the shift \((S^\varepsilon M)(t):=M(t+\varepsilon)\).

\begin{lemma}[Shift commutation]\label{A:lem:shift}
For all \(\varepsilon\ge 0\), there is a canonical isomorphism
\(\mathbf{T}_\tau\circ S^\varepsilon \;\cong\; S^\varepsilon\circ \mathbf{T}_\tau.\)
\end{lemma}

\begin{proof}
Shifts preserve constructibility and interval lengths and hence preserve \(\mathsf{E}_\tau\).
Therefore \(S^\varepsilon\) descends to the Serre quotient and commutes with \(\pi_\tau\); transporting across the equivalence with \(\Pers^{\mathrm{ft}}_{k,\tau\text{-loc}}\) gives the claim.
\end{proof}

\begin{lemma}[Monotonicity in \(\tau\)]\label{A:lem:monotone-tau}
If \(0<\tau\le \tau'\), there is a natural epimorphism \(\mathbf{T}_{\tau'}M\to \mathbf{T}_{\tau}M\) functorial in \(M\).
Equivalently, the local subcategories nest \(\Pers^{\mathrm{ft}}_{k,\tau'\text{-loc}}\subseteq \Pers^{\mathrm{ft}}_{k,\tau\text{-loc}}\).
On barcodes, \(\mathbf{T}_{\tau'}\) removes (weakly) more bars than \(\mathbf{T}_{\tau}\).
\end{lemma}

\begin{proof}
We have \(\mathsf{E}_\tau\subseteq \mathsf{E}_{\tau'}\). By the universal property of reflectors, \(\iota_{\tau'}\) factors through \(\iota_{\tau}\), inducing the natural comparison \(\mathbf{T}_{\tau'}\Rightarrow \mathbf{T}_{\tau}\). The barcode description in Proposition~\ref{A:prop:barcode-behavior} makes the nesting explicit.
\end{proof}

\begin{proposition}[Non-expansiveness for the interleaving metric]\label{A:prop:lipschitz}
In the 1D constructible/barcode setting, \(\mathbf{T}_\tau\) is \(1\)-Lipschitz for the interleaving (equivalently, bottleneck) distance on \(\Pers^{\mathrm{ft}}_k\).
\end{proposition}

\begin{proof}
If \(M,N\) are \(\varepsilon\)-interleaved via shifts \(S^\varepsilon\), then by Lemma~\ref{A:lem:shift} the same diagrams exhibit an \(\varepsilon\)-interleaving between \(\mathbf{T}_\tau M\) and \(\mathbf{T}_\tau N\).
Thus \(d_{\mathrm{int}}(\mathbf{T}_\tau M,\mathbf{T}_\tau N)\le d_{\mathrm{int}}(M,N)\).
\end{proof}

\begin{remark}[Standard references]
The barcode decomposition and the resulting torsion/localization picture in 1D constructible persistence trace back to Crawley–Boevey (2015) and to the stability/structure framework summarized by Chazal–de~Silva–Glisse–Oudot (2016).
For Serre/Gabriel localization and the identification of the quotient with the \emph{local (orthogonal)} subcategory see, e.g., Gabriel (1962) or standard expositions (Popescu, \emph{Abelian Categories}; Stacks Project, Tag~02MO).
The \(1\)-Lipschitz property of \(\mathbf{T}_\tau\) follows from shift-commutation and the interleaving formalism in this 1D setting.
\end{remark}

% -------------------------
\subsection*{A.6. Windowing (MECE), coverage checks, and \texorpdfstring{$\tau$}{tau}-adaptation}
% -------------------------
\begin{definition}[MECE domain windowing and coverage]\label{A:def:MECE}
A \emph{domain windowing} is a finite or countable collection of half-open intervals with right-inclusion
\(\{[u_k,u_{k+1})\}_{k\in K}\) such that:
\begin{itemize}
  \item \emph{(Disjointness)} \([u_k,u_{k+1})\cap [u_\ell,u_{\ell+1})=\varnothing\) for \(k\neq \ell\).
  \item \emph{(Contiguity)} \(u_{k+1}=u_k+\len_k\) with \(\len_k>0\).
  \item \emph{(Coverage)} \(\bigsqcup_{k\in K}[u_k,u_{k+1})=[u_0,U)\) for some finite \(U>u_0\).
\end{itemize}
The \emph{coverage checks} require
\[
\sum_{k\in K}(u_{k+1}-u_k)=U-u_0,\n\qquad\n\#\mathrm{Events}([u_0,U))=\sum_{k\in K}\#\mathrm{Events}([u_k,u_{k+1}))\ \ (\pm\ \text{rounding}),
\]
where \(\#\mathrm{Events}([a,b))\) counts births/deaths with multiplicity recorded by the chosen endpoint convention.
\end{definition}

\begin{remark}[Alignment of windows, clipping, and localization]
When persistence and spectral measurements are combined, the domain windowing \(\{[u_k,u_{k+1})\}\), the collapse threshold(s) \(\tau\), and spectral bin windows \([a,b]\) must be \emph{fixed per window} and recorded. All B-side measurements are taken \emph{after} applying \(\mathbf{T}_\tau\) and on the same window; clipping respects the half-open, right-inclusion convention. By Lemma~\ref{A:lem:clip}, clipping and localization commute.
\end{remark}

\begin{definition}[\(\tau\)-adaptation, sweep, and stability bands]\label{A:def:tau-adapt}
A collapse threshold \(\tau\) is \emph{resolution-adapted} if there is a constant \(\alpha>0\) such that
\[
\tau\ =\ \alpha\cdot \max\{\Delta t,\Delta x\},
\]
for the temporal/spatial mesh sizes \((\Delta t,\Delta x)\) of the data/solver. A \emph{\(\tau\)-sweep} is a discrete set \(\{\tau_\ell\}\) on which diagnostics \((\mu_{i,\tau_\ell},\nu_{i,\tau_\ell})\) are evaluated. A \emph{stability band} is a contiguous range \(B\subset (0,\infty)\) such that a chosen natural transformation \(\phi_{i,\tau}\) is an isomorphism for all \(\tau\in B\) (hence \(\mu_{i,\tau}=\nu_{i,\tau}=0\)) and for all monitored degrees \(i\).
\end{definition}

\begin{remark}[Persistence/spectral window policy]
Spectral indicators and auxiliary spectral bars (Chapter~11) are computed on \(L(C_\tau F)\) over a spectral window \([a,b]\) with right-open bins of width \(\beta\). The pair \(([a,b],\beta)\) is fixed on each domain window and logged; under/overflow counts are recorded to ensure coverage.
\end{remark}

\begin{remark}[Restart/Summability interface]
Windowed certificates produced on \(\MECE\) partitions paste globally if (i) the Restart inequality holds for safety margins from one window to the next, and (ii) the cumulative \(\delta\)-budget is summable (Chapter~4). The windowing and \(\tau\)-adaptation rules above are the only assumptions on Appendix~A required to support that interface.
\end{remark}

% -------------------------
\subsection*{A.7. Operational checklist and glued output}
% -------------------------
The preceding reinforcement yields a single, self-contained operational layer compatible with IMRN/AiM standards:
\begin{itemize}
  \item Abelianity and pointwise exactness (Proposition~\ref{A:prop:abelian}).
  \item Hereditary Serre \(\tau\)-ephemeral class \(\mathsf{E}_\tau\) (Lemma~\ref{A:lem:Serre}).
  \item Exact reflective localization \(\mathbf{T}_\tau\dashv\iota_\tau\) with barcode-level description (Theorem~\ref{A:thm:localization}, Proposition~\ref{A:prop:barcode-behavior}).
  \item Shift-commutation and \(1\)-Lipschitz continuity (Lemma~\ref{A:lem:shift}, Proposition~\ref{A:prop:lipschitz}); monotonicity in \(\tau\) (Lemma~\ref{A:lem:monotone-tau}).
  \item Clipping-compatibility and MECE coverage checks (Lemma~\ref{A:lem:clip}, Definition~\ref{A:def:MECE}).
  \item Filtered-colimit policy with return-to-constructible verification (Remark~\ref{A:rk:filtered-colimits}).
\end{itemize}
Output: the globally glued object obtained from the windowed pipeline (MECE partition, localization, spectral auxiliaries, Restart/Summability) passes the B-Gate\(^{+}\) acceptance check under the coverage and adaptation policies specified above.

\medskip
\noindent\emph{References for Appendix A.}
Crawley–Boevey (2015): Decomposition of pointwise finite-dimensional persistence modules. IMRN. 
Chazal–de~Silva–Glisse–Oudot (2016): The Structure and Stability of Persistence Modules. 
Gabriel (1962): Des catégories abéliennes. 
Popescu: Abelian Categories. 
Stacks Project, Tag~02MO.

\endgroup



% =========================
\appendix
\section*{Appendix B. Lifting \texorpdfstring{$\mathbf{T}_\tau$}{T\_\tau} to \texorpdfstring{$C_\tau$}{C\_\tau} and the Homotopy Setting (reinforced)}
% =========================

\addcontentsline{toc}{section}{Appendix B. Lifting $T_\tau$ to $C_\tau$ and the Homotopy Setting (reinforced)}

\begingroup

Throughout, fix a field \(k\).
Let \(\mathsf{FiltCh}(k)\) denote the category of \emph{bounded-in-degree} filtered chain complexes of finite-dimensional \(k\)-vector spaces with filtration-preserving chain maps.
(“Bounded” refers to homological degree; filtrations are assumed \emph{locally finite on bounded windows} as in Appendix~A.)
For each homological degree \(i\), write
\[
\mathbf{P}_i:\ \mathsf{FiltCh}(k)\longrightarrow \Pers^{\mathrm{ft}}_k,\qquad\nF\longmapsto \big(t\mapsto H_i(F^{t}C_\bullet)\big),
\]
the degreewise persistence functor into the constructible subcategory (Appendix~A).

\medskip
\noindent\textbf{Global scope and conventions.}
(i) All claims at the filtered–complex layer hold \emph{up to filtered quasi-isomorphism (f.q.i.)}; all identities at the persistence layer hold \emph{strictly} in \(\Pers^{\mathrm{ft}}_k\).
(ii) Filtered (co)limits, when invoked, are computed objectwise in \([\mathbb{R},\Vect_k]\), and we then \emph{verify} that the result lies in (or returns to) \(\Pers^{\mathrm{ft}}_k\) (Appendix~A, Remark~\ref{A:rk:filtered-colimits}); no claim is made outside this regime.
(iii) Deletion-type updates are non-increasing for windowed energies and spectral tails \emph{after truncation}, whereas inclusion-type updates are only stable (non-expansive) (Appendix~E).
(iv) Endpoint conventions follow Appendix~A (Remark~\ref{A:rk:endpoints}); in particular, infinite bars are not removed by \(\mathbf{T}_\tau\) and their contributions are clipped by windowing.
(v) For notational economy we sometimes write \(\Ttau=\mathbf{T}_\tau\).

% -------------------------
\subsection*{B.1. The interval-realization assignment \texorpdfstring{$\mathcal{U}$}{U} (up to f.q.i.)}
% -------------------------
\begin{definition}[Elementary interval blocks (two-term/one-term model)]
Let \(I[a,b)\) be an interval module (fixed endpoint convention; Appendix~A, Remark~\ref{A:rk:endpoints}).
\begin{itemize}
  \item If \(b<+\infty\), realize \(I[a,b)\) in homological degree \(i\) by a \emph{two-term filtered block}
  \[
    k\cdot y \xrightarrow{\,d\,} k\cdot x,\qquad |y|=i+1,\ |x|=i,\quad
       \mathrm{fil}(x)=a,\ \mathrm{fil}(y)=b,\ d(y)=x,\ d(x)=0.
  \]
  Then \(x\) contributes a bar born at \(a\) and killed at \(b\).
  \item If \(b=+\infty\), realize \(I[a,\infty)\) by a \emph{one-term} block \(k\cdot x\) in degree \(i\) with \(\mathrm{fil}(x)=a\) and \(d=0\).
\end{itemize}
In all blocks, the differential preserves the filtration: \(d(F^t)\subseteq F^t\) for every \(t\).
Since \(a\le b\), for \(t\ge b\) one has \(x,y\in F^t\) and \(d(y)=x\in F^t\); for \(a\le t<b\) one has \(x\in F^t\), \(y\notin F^t\), whence \(d\) restricts to \(0\) on \(F^t\).
Taking \emph{locally finite on bounded windows} direct sums of such blocks and applying degree shifts produces a filtered complex whose persistence recovers the prescribed bars.
We call any such model an \emph{elementary interval complex} and denote a representative by \(\mathcal{I}[a,b)\).
\end{definition}

\begin{proposition}[Barcode realization for bounded families (up to f.q.i.)]\label{B:prop:U}
There exists an assignment
\[
\mathcal{U}:\ \Pers^{\mathrm{ft}}_k\longrightarrow \mathsf{FiltCh}(k)
\]
such that for any \emph{degree-bounded} family \(\{M_i\}_{i\in\mathbb{Z}}\) of constructible persistence modules
(only finitely many \(i\) nonzero) there are natural isomorphisms in \(\Pers^{\mathrm{ft}}_k\),
\[
\mathbf{P}_i\!\Big(\,\bigoplus_{j}\mathcal{U}(M_j)[-j]\Big)\ \cong\ M_i\qquad(\forall i).
\]
The construction is canonical \emph{up to} filtered quasi-isomorphism, additive, and functorial in the homotopy category \(\Ho(\mathsf{FiltCh}(k))\).
In particular, for a single module \(M\) realized in a base degree (say \(0\)) one has \(\mathbf{P}_0(\mathcal{U}(M))\cong M\) and \(\mathbf{P}_j(\mathcal{U}(M))=0\) for all \(j\neq 0\).
\end{proposition}

\begin{remark}[Pseudofunctoriality of \(\mathcal{U}\)]
The assignment \(\mathcal{U}\) extends to a \emph{pseudofunctor}
\(\mathcal{U}:\Pers^{\mathrm{ft}}_k\to \Ho(\mathsf{FiltCh}(k))\): on a morphism of persistence modules, choose interval decompositions and a bar-matching; the induced blockwise filtered chain map is well-defined in \(\Ho\) \emph{up to} f.q.i., and compositions are respected up to coherent isomorphism.
In dimension \(1\) with constructibility, the needed (pseudo)naturality follows from standard barcode calculus (e.g.\ Crawley--Boevey (2015); Chazal--de~Silva--Glisse--Oudot (2016)).
Consequently, constructions below that use \(\mathcal{U}\) on morphisms (e.g.\ \(C_\tau\)) are functorial on \(\Ho(\mathsf{FiltCh}(k))\).
\end{remark}

\begin{proof}[Proof sketch of Proposition~\ref{B:prop:U}]
Choose barcode decompositions \(M_i\simeq \bigoplus_{j\in J_i} I[a_{i,j},b_{i,j})\) (locally finite on bounded windows) and set
\(\bigoplus_{i}\bigoplus_{j\in J_i}\mathcal{I}[a_{i,j},b_{i,j})[-i]\).
Different decompositions yield filtered complexes that are f.q.i.-equivalent, hence define the same object (and morphisms) in \(\Ho(\mathsf{FiltCh}(k))\).
\end{proof}

% -------------------------
\subsection*{B.2. Filtered quasi-isomorphisms and \texorpdfstring{$\Ho(\mathsf{FiltCh}(k))$}{Ho(FiltCh(k))}}
% -------------------------
\begin{definition}[Filtered quasi-isomorphism]
A filtration-preserving chain map \(f:F\to G\) is a \emph{filtered quasi-isomorphism (f.q.i.)} if for every \(t\in\mathbb{R}\) the map \(F^{t}C_\bullet\to G^{t}C_\bullet\) is a quasi-isomorphism.
Equivalently, for all \(i\), \(\mathbf{P}_i(f)\) is an isomorphism in \(\Pers^{\mathrm{ft}}_k\).
\end{definition}

\begin{lemma}[Characterization of f.q.i.]
For bounded-in-degree filtered complexes of finite-dimensional vector spaces, a filtration-preserving chain map \(f:F\to G\) is an f.q.i. if and only if \(\mathbf{P}_i(f)\) is an isomorphism in \(\Pers^{\mathrm{ft}}_k\) for all \(i\).
\end{lemma}

\begin{proof}[Proof sketch]
If \(f\) is an f.q.i., then for each \(t\) the induced maps on homology are isomorphisms, hence \(\mathbf{P}_i(f)\) is pointwise an isomorphism and thus an isomorphism in \(\Pers^{\mathrm{ft}}_k\).
Conversely, if \(\mathbf{P}_i(f)\) is an isomorphism, then for each \(t\) the maps \(H_i(F^t)\to H_i(G^t)\) are isomorphisms for all \(i\); by boundedness and finite-dimensionality, \(f|_{F^t}\) is a quasi-isomorphism for every \(t\), hence \(f\) is an f.q.i.
\end{proof}

\begin{definition}[Homotopy category]
Let \(\Ho(\mathsf{FiltCh}(k))\) be the localization of \(\mathsf{FiltCh}(k)\) at f.q.i.’s.
Identities stated in \(\Ho(\mathsf{FiltCh}(k))\) are to be understood \emph{up to f.q.i.} at the model level.
All endofunctors considered below (e.g.\ \(C_\tau\) and Mirror/Transfer templates) preserve f.q.i.’s; thus they descend to \(\Ho(\mathsf{FiltCh}(k))\).
\end{definition}

% -------------------------
\subsection*{B.3. Lifting \texorpdfstring{$\mathbf{T}_\tau$}{T\_\tau} to \texorpdfstring{$C_\tau$}{C\_\tau} and (co)limit/pullback compatibilities}
% -------------------------

\paragraph*{Existence, functoriality, and uniqueness (homotopy-functor level): block-diagonal assembly.}
\begin{theorem}[Thresholded collapse in \(\Ho\)]\label{B:thm:Ctau}
For each \(\tau\ge 0\) there exists an endofunctor
\[
C_\tau:\ \Ho(\mathsf{FiltCh}(k))\longrightarrow \Ho(\mathsf{FiltCh}(k))
\]
and natural isomorphisms in \(\Pers^{\mathrm{ft}}_k\)
\[
\mathbf{P}_i\!\big(C_\tau(F)\big)\ \xrightarrow{\ \cong\ }\ \mathbf{T}_\tau\!\big(\mathbf{P}_i(F)\big)\qquad(\forall i,F),
\]
such that:
\begin{enumerate}\itemsep0.2em
  \item (\emph{Idempotence/monotonicity in \(\Ho\)})
  \(C_\tau\circ C_\sigma \simeq C_{\max\{\tau,\sigma\}}\simeq C_\sigma\circ C_\tau\).
  \item (\emph{Non-expansiveness at persistence})
  For all \(F,G\) and all \(i\),
  \[
  d_{\mathrm{int}}\!\big(\mathbf{P}_i(C_\tau F),\,\mathbf{P}_i(C_\tau G)\big)\ \le\ d_{\mathrm{int}}\!\big(\mathbf{P}_i(F),\,\mathbf{P}_i(G)\big).
  \]
\end{enumerate}
Moreover, any two lifts with these properties are uniquely isomorphic in \(\Ho(\mathsf{FiltCh}(k))\).
For \(\tau=0\), \(C_0\simeq \mathrm{id}\) in \(\Ho(\mathsf{FiltCh}(k))\).
\end{theorem}

\begin{proof}[Construction/Proof]
On objects: for each \(i\), replace \(\mathbf{P}_i(F)\) with \(\mathbf{T}_\tau(\mathbf{P}_i(F))\) (Appendix~A, Thm.~\ref{A:thm:localization}) and realize via \(\mathcal{U}\) (Proposition~\ref{B:prop:U}); \emph{assemble the total differential in a strictly block-diagonal way}, i.e.\ differentials are nonzero only inside the two-term blocks \((i+1)\to i\) representing finite bars (and the one-term blocks for infinite bars), while \emph{off-diagonal components between distinct bars or non-adjacent homological degrees are set to zero}. This preserves degreewise persistence exactly.
On morphisms \(f:F\to G\): apply \(\mathbf{T}_\tau\mathbf{P}_i(f)\) and lift blockwise via \(\mathcal{U}\) (pseudofunctoriality), taking direct sums over \(i\); this defines \(C_\tau(f)\) well-defined in \(\Ho\).
Idempotence/monotonicity reflect the corresponding properties of \(\mathbf{T}_\tau\) (a Serre exact reflector); non-expansiveness follows from the \(1\)-Lipschitz property of \(\mathbf{T}_\tau\) (Appendix~A, Prop.~\ref{A:prop:lipschitz}).
Uniqueness in \(\Ho\) follows from the uniqueness of \(\mathbf{T}_\tau\) at the persistence layer and of \(\mathcal{U}\) up to f.q.i.
\end{proof}

\paragraph*{(Co)limits and pullbacks: persistence layer is strict; filtered layer up to f.q.i.}
\begin{proposition}[Compatibility at the persistence layer]\label{B:prop:limits}
Assume filtered colimits in \(\mathsf{FiltCh}(k)\) are computed degreewise on chains/filtrations and the results return to \(\Pers^{\mathrm{ft}}_k\).
Then for every filtered diagram \(\{F_\lambda\}\) and every \(i\),
\[
\mathbf{P}_i\!\big(C_\tau(\varinjlim\nolimits_\lambda F_\lambda)\big)\ \cong\ \varinjlim\nolimits_\lambda\, \mathbf{P}_i\!\big(C_\tau(F_\lambda)\big)\quad\text{in } \Pers^{\mathrm{ft}}_k.
\]
If, in addition, \textup{[Spec]} finite pullbacks in \(\mathsf{FiltCh}(k)\) are computed degreewise and the interval realization \(\mathcal{U}\) preserves finite limits up to f.q.i.\ under the \emph{lifting–coherence} hypothesis \((\LC)\), then for any pullback square \(F\times_H G\),
\[
\mathbf{P}_i\!\big(C_\tau(F\times_H G)\big)\ \cong\ \mathbf{P}_i\!\big(C_\tau(F)\times_{C_\tau(H)} C_\tau(G)\big)\quad\text{in } \Pers^{\mathrm{ft}}_k.
\]
Similarly, finite pushouts behave dually at the persistence layer (under the same scope policy), since \(\mathbf{T}_\tau\) preserves finite colimits; \textbf{at the filtered-complex level all such compatibilities are asserted only \emph{up to f.q.i.}}.
\end{proposition}

\begin{proof}
Exactness (both left and right) and additivity of \(\mathbf{T}_\tau\) as a Serre localization (Appendix~A, Thm.~\ref{A:thm:localization}) yield preservation of finite limits/colimits; being a left adjoint, \(\mathbf{T}_\tau\) preserves (existing) colimits with the filtered-colimit policy of Appendix~A, Remark~\ref{A:rk:filtered-colimits}.
Applying \(\mathbf{P}_i\) gives the identities at the persistence layer.
At the filtered level, compatibilities hold \emph{up to f.q.i.} via realization by \(\mathcal{U}\) and the block-diagonal assembly in the construction of \(C_\tau\).
\end{proof}

\begin{remark}[On \((\LC)\)]
The hypothesis \((\LC)\) is a \emph{finite-diagram} coherence condition ensuring that interval realizations can be chosen compatibly (up to f.q.i.) with pullback/pushout shapes occurring in practice (e.g.\ fiber products along filtration-preserving maps).
It holds for the elementary block model above and finite matching diagrams where barcode maps are induced by monotone filtrations; we use it only in \textup{[Spec]} statements.
\end{remark}

\begin{remark}[Realization functor; comparison maps \textup{\lbrack}Spec\textup{\rbrack}]
Let \(\mathcal{R}:\mathsf{FiltCh}(k)\to D^{\mathrm{b}}(k\text{-mod})\) be the fixed \(t\)-exact realization from the main text.
Within the implementable range there are natural comparison morphisms
\[
\mathcal{R}\circ C_\tau\ \Longrightarrow\ \tau_{\ge 0}\circ \mathcal{R},
\]
compatible with \(\mathbf{P}_i\) after homology (Appendix~C).
Here \(\tau_{\ge 0}\) denotes truncation for the fixed \(t\)-structure.
These comparison maps are treated up to f.q.i.\ in \(\Ho(\mathsf{FiltCh}(k))\).
\end{remark}

% -------------------------
\subsection*{B.4. Non-expansive Mirror/Transfer templates \texorpdfstring{[Spec]}{[Spec]}}
% -------------------------
\begin{specification}[Mirror/Transfer endofunctors]\label{B:spec:mirror}
An endofunctor \(\Mirror:\mathsf{FiltCh}(k)\to \mathsf{FiltCh}(k)\) is \emph{admissible} if:
\begin{enumerate}\itemsep0.2em
  \item (\emph{Persistence non-expansiveness})
  For all \(F,G\) and every \(i\),
  \[
  d_{\mathrm{int}}\!\big(\mathbf{P}_i(\Mirror F),\,\mathbf{P}_i(\Mirror G)\big)\ \le\ d_{\mathrm{int}}\!\big(\mathbf{P}_i(F),\,\mathbf{P}_i(G)\big).
  \]
  \item (\emph{Constructible stability})
  \(\Mirror\) carries finite-type objects to finite-type objects degreewise.
  \item (\emph{f.q.i.-invariance})
  If \(f\) is an f.q.i., then \(\Mirror(f)\) is an f.q.i.; hence \(\Mirror\) descends to \(\Ho(\mathsf{FiltCh}(k))\).
  \item (\emph{Conditional commutation with \(C_\tau\)})
  There exists a natural 2-cell
  \[
  \theta:\ \Mirror\circ C_\tau\ \Rightarrow\ C_\tau\circ \Mirror
  \]
  whose effect at persistence is \(\delta\)-controlled:
  \[
  d_{\mathrm{int}}\!\Big(\mathbf{T}_\tau\,\mathbf{P}_i\big(\Mirror(C_\tau F)\big),\ \mathbf{T}_\tau\,\mathbf{P}_i\big(C_\tau(\Mirror F)\big)\Big)\ \le\ \delta(i,\tau)\quad(\forall i,F).
  \]
  The bound \(\delta(i,\tau)\) is \emph{uniform in \(F\)} and can be chosen to be \emph{additive} along pipelines (Appendix~L) and \emph{non-increasing under} any subsequent \(1\)\hyp Lipschitz persistence post\hyp processing (e.g.\ shifts, further truncations).
  In particular, if \(\theta\) induces isomorphisms after applying \(\mathbf{P}_i\), then one may take \(\delta(i,\tau)=0\).
\end{enumerate}
Under these assumptions, windowed indicators computed on \(C_\tau F\) (e.g.\ truncated energies) are stable (non-expansive) along \(\Mirror\).
Quantitative control of \(\delta\) follows Appendix~L (uniformity in \(F\), pipeline additivity, and non-increase under \(1\)\hyp Lipschitz post\hyp processing).
\end{specification}

\begin{remark}[Pseudonaturality and coherence]
When several admissible Mirror/Transfer functors occur, their 2-cells are used \emph{sequentially}. Coherence reduces in practice to the triangle inequality at persistence level together with non-expansiveness of \(\mathbf{T}_\tau\) and of the post-processors (Appendix~L). No higher coherence (e.g.\ MacLane pentagon) is required for pipeline accounting, since only numeric bounds are aggregated.
\end{remark}

% -------------------------
\subsection*{B.5. Quantitative commutation: uniformity, additivity, coherence, and post-processing stability}
% -------------------------
\begin{theorem}[Quantitative commutation in \(\Ho\) via \(\mathbf{P}_i\) and \(\mathbf{T}_\tau\)]\label{B:thm:quant}
Assume \textup{(i)} \(\Mirror\) is admissible (Specification~\ref{B:spec:mirror}) and \textup{(ii)} a natural 2-cell \(\theta:\Mirror\circ C_\tau \Rightarrow C_\tau\circ \Mirror\) with bound \(\delta(i,\tau)\ge 0\), uniform in \(F\).
Then, for all \(F\), degrees \(i\), and scales \(\tau\),
\[
  d_{\mathrm{int}}\!\Big(\mathbf{T}_\tau\,\mathbf{P}_i(\Mirror(C_\tau F)),\ \mathbf{T}_\tau\,\mathbf{P}_i(C_\tau(\Mirror F))\Big)\ \le\ \delta(i,\tau).
\]
Moreover, for a pipeline \(\Mirror_m,\dots,\Mirror_1\) of admissible endofunctors with 2-cell bounds \(\delta_j(i,\tau_j)\), one has the additive estimate
\[
  d_{\mathrm{int}}\!\Big(\mathbf{T}_\tau\,\mathbf{P}_i\big(\Mirror_m\cdots\Mirror_1(C_{\tau_m}\cdots C_{\tau_1}F)\big),\ \mathbf{T}_\tau\,\mathbf{P}_i\big(C_{\tau_m}\cdots C_{\tau_1}(\Mirror_m\cdots\Mirror_1 F)\big)\Big)\ \le\ \sum_{j=1}^m \delta_j(i,\tau_j),
\]
and any further \(1\)\hyp Lipschitz persistence post\hyp processing (e.g.\ shifts \(S^\varepsilon\), truncations \(\mathbf{T}_{\tau'}\), degree projections) does not increase the bound.
\end{theorem}

\begin{proof}[Proof sketch]
Apply the 2-cell bound at the persistence layer, then use the \(1\)\hyp Lipschitz property of \(\mathbf{T}_\tau\) (Appendix~A, Prop.~\ref{A:prop:lipschitz}). For the pipeline, compose the 2-cells and sum the bounds; post\hyp processing stability follows from non\hyp expansiveness of the applied persistence functors.
\end{proof}

\begin{remark}[Uniformity and model-independence]
The bound \(\delta(i,\tau)\) is \emph{uniform in \(F\)}, hence independent of the chosen filtered model of \(F\).
If \(F\simeq F'\) in \(\Ho(\mathsf{FiltCh}(k))\) (f.q.i.), then \(C_\tau F\simeq C_\tau F'\) and the measured interleaving distances coincide at the persistence layer. This yields model\hyp independence of the pipeline estimates.
\end{remark}

% -------------------------
\subsection*{B.6. Commutable torsion reflectors and A/B policy (homotopy interface)}
% -------------------------
Let \(T_A,T_B:\Pers^{\mathrm{ft}}_k\to\Pers^{\mathrm{ft}}_k\) be exact reflectors (e.g.\ truncations by distinct torsion classes). Write \(E_A,E_B\) for the underlying Serre subcategories.

\begin{proposition}[Nested torsions \(\Rightarrow\) order independence]\label{B:prop:nested}
If \(E_A\subseteq E_B\) or \(E_B\subseteq E_A\), then
\[
T_A\circ T_B\ =\ T_B\circ T_A\ =\ T_{A\vee B},
\]
where \(E_{A\vee B}\) is the Serre subcategory generated by \(E_A\cup E_B\).
In particular, for 1D length thresholds, \(\mathbf{T}_\tau\circ \mathbf{T}_\sigma=\mathbf{T}_{\max\{\tau,\sigma\}}\).
\end{proposition}

\begin{proof}[Proof sketch]
Nested Serre subcategories yield idempotent, order\hyp independent localizations by the universal property of reflectors in abelian categories.
\end{proof}

\begin{definition}[A/B commutativity test and soft\hyp commuting policy]\label{B:def:ab}
For arbitrary reflectors \(T_A,T_B\) and \(M\in\Pers^{\mathrm{ft}}_k\), set
\[
\Delta_{\mathrm{comm}}(M;A,B):=d_{\mathrm{int}}(T_A T_B M,\ T_B T_A M).
\]
Given a tolerance \(\eta\ge 0\), we \emph{accept soft\hyp commuting} on a dataset if \(\Delta_{\mathrm{comm}}(M;A,B)\le \eta\) holds for the relevant instances \(M\). Otherwise, we \emph{fallback} to a fixed order (say \(T_B\circ T_A\)) and record \(\Delta_{\mathrm{comm}}\) into \(\delta^{\mathrm{alg}}\) in the \(\delta\)\hyp ledger (Appendix~L).
\end{definition}

\begin{remark}[Multiple axes and canonical ordering]
For three or more non\hyp nested reflectors, pairwise soft\hyp commuting does not imply global confluence.
Adopt a canonical ordering (e.g.\ priority by axis type), apply A/B tests to adjacent pairs, and aggregate residuals additively in the pipeline budget (Appendix~L). Nested pairs may bypass testing via Proposition~\ref{B:prop:nested}.
\end{remark}

% -------------------------
\subsection*{B.7. Completion note and implementation recipe}
% -------------------------
\begin{remark}[No further supplementation required]
This appendix fully integrates: (i) the lift \(C_\tau\) of the exact reflector \(\mathbf{T}_\tau\) to the homotopy setting (existence, functoriality, uniqueness up to f.q.i.; non\hyp expansiveness at persistence); (ii) strict persistence\hyp layer compatibilities with (co)limits and pullbacks (filtered\hyp complex level up to f.q.i.); (iii) admissible Mirror/Transfer templates with a quantitative 2\hyp cell bound \(\delta(i,\tau)\) that is \emph{uniform in \(F\)}, \emph{additive} along pipelines, and \emph{non\hyp increasing} under \(1\)\hyp Lipschitz post\hyp processing; and (iv) a commutable torsion policy: nested torsions imply order independence, otherwise an A/B test with soft\hyp commuting and a deterministic fallback, with differences logged into \(\delta^{\mathrm{alg}}\).
All statements are confined to the v16.0 guard\hyp rails (constructible \(1\)D persistence over a field; B\hyp side single layer after collapse; f.q.i.\ on filtered complexes), and no further supplementation is required for operational use in this framework.
\end{remark}

\paragraph{Implementation recipe (engineering checklist).}
\begin{itemize}
  \item Build \(C_\tau\) by block\hyp diagonal assembly of interval blocks (two-term for finite bars, one-term for infinite bars); forbid off-diagonal couplings across blocks/degrees.
  \item For every morphism, lift \(\mathbf{T}_\tau\mathbf{P}_i(f)\) blockwise via \(\mathcal{U}\) and take direct sums across \(i\); record that functoriality holds \emph{in \(\Ho\)}.
  \item For Mirror/Transfer, provide a 2\hyp cell \(\theta\) with a \emph{uniform} \(\delta(i,\tau)\); accumulate \(\delta\)'s additively along the pipeline; ensure any post\hyp processors are \(1\)\hyp Lipschitz at persistence.
  \item For reflectors, run A/B tests after truncation; if \(\Delta_{\mathrm{comm}}>\eta\), fix an order and log \(\Delta_{\mathrm{comm}}\) as \(\delta^{\mathrm{alg}}\).
  \item Keep clipping/windowing separate from localization; if needed, use Appendix~A, Lemma~\ref{A:lem:clip}.
  \item Expose all \(\delta\)\hyp entries, norms, orders, and overlap decisions in \texttt{run.yaml} for audit (Appendix~G).
\end{itemize}

\endgroup



% =========================
\appendix
\section*{Appendix C. The Bridge \texorpdfstring{$\mathrm{PH}_1 \Rightarrow \Ext^1$}{PH1⇒Ext1} (reinforced, complete)}
% =========================

\addcontentsline{toc}{section}{Appendix C. The Bridge PH$_1\Rightarrow$Ext$^1$}

Throughout, fix a field \(k\).
Let \(\mathsf{FiltCh}(k)\) be the category of \emph{bounded-in-degree} filtered chain complexes of finite-dimensional \(k\)-vector spaces with filtration-preserving maps.
For \(F\in\mathsf{FiltCh}(k)\) and each degree \(i\), the degreewise persistence functor
\[
\mathbf{P}_i(F):\ \mathbb{R}\longrightarrow \mathsf{Vect}_k,\qquad t\longmapsto H_i(F^{t}C_\bullet)
\]
is assumed \emph{constructible} (pointwise finite-dimensional, with finitely many critical parameters on bounded windows), i.e.\ \(\mathbf{P}_i(F)\in \Pers^{\mathrm{ft}}_k\).
We also fix a \(t\)-exact realization functor
\[
\mathcal{R}:\ \mathsf{FiltCh}(k)\longrightarrow D^{\mathrm{b}}(k\text{-mod})
\]
into the bounded derived category of finite-dimensional \(k\)-vector spaces.

\medskip
\noindent\textbf{Bridge hypotheses, scope, and gate policy (final).}
We work under the standing assumptions \textup{(B1)–(B3)}:
\begin{itemize}
  \item \textup{(B1)} field coefficients and constructibility of \(\mathbf{P}_i(F)\);
  \item \textup{(B2)} two-term amplitude for \(\mathcal{R}\) (edge identification) on the operative windows (Definition below);
  \item \textup{(B3)} functoriality/naturality of all constructions.
\end{itemize}
All statements in this appendix operate under \textup{(B1)–(B3)} and within the implementable range of Appendix~A.
Filtered colimits, when used, are computed in the functor category \([\mathbb{R},\mathsf{Vect}_k]\) and must \emph{return} to \(\Pers^{\mathrm{ft}}_k\) once constructibility is verified (Appendix~A, Remark~\ref{A:rk:filtered-colimits}).

\smallskip
\noindent\textbf{Eligibility for B-Gate\(^{+}\) (fixed).}
The \(\Ext^1\)-test is included in B-Gate\(^{+}\) \emph{only} on windows/scales where \(\mathcal{R}(C_\tau F)\in D^{[-1,0]}(k\text{-mod})\) (amplitude \(\le 1\)) and the test object is \(k[0]\).
Outside this amplitude regime \(\Ext^1\) may be logged but is \emph{not} used for gating.

\smallskip
\noindent\textbf{Operational order (fixed).}
We enforce the \textbf{collapse-first policy}
\[
\textbf{collapse}\ \xrightarrow{\ \ C_\tau\ \ } \ \textbf{realize}\ \xrightarrow{\ \ \mathcal{R}\ \ } \ \textbf{test}\ \xrightarrow{\ \ \Ext^1(-,k)\ \ } 0,
\]
i.e.\ \(\,F\to C_\tau F\to \mathcal{R}(C_\tau F)\to \Ext^1(-,k)\,\).
This order is mandatory and is used in both statements and proofs.

\smallskip
\noindent\textbf{Minimal test family.}
We test \(\Ext\)-vanishing only against the unit \(k[0]\):
\[
\text{``\(\Ext^1\)-collapse''}\quad\Longleftrightarrow\quad\Ext^1\!\big(\mathcal{R}(\bullet),\,k\big)=0.
\]
For objects \(A\in D^{[-1,0]}(k\text{-mod})\), Lemma~\ref{C:lem:edge-ext} gives
\(\Ext^1(A,k)\cong \Hom(H^{-1}(A),k)\); hence testing with \(k[0]\) is sufficient.

\smallskip
\noindent\textbf{Meaning of \(\mathrm{PH}_1(F)=0\).}
In the notation of the main text, \(\mathrm{PH}_1(F)=0\) means that the degree-\(1\) persistence module vanishes (equivalently, \(H_1(F^t)=0\) for all \(t\in\mathbb{R}\); see Chapter~2).

\medskip
\noindent\textbf{Notation.}
For a window \([0,\tau]\) we use the truncation/collapse operator \(C_\tau\) from Appendix~B.
We write \(D^{[a,b]}(k\text{-mod})\) for objects with cohomology concentrated in degrees \([a,b]\).
We freely use truncation functors \(\tau_{\leq d},\tau_{\geq d}\) for the standard \(t\)-structure.

% -------------------------
\subsection*{C.1. Two-term amplitude and $t$-exactness}
% -------------------------

\begin{proposition}[Two-term amplitude]\label{C:prop:two-term}
Under \textup{(B2)} there is a natural isomorphism in \(D^{\mathrm{b}}(k\text{-mod})\)
\[
\mathcal{R}(F)\ \simeq\ \Big[\, H^{-1}\!\big(\mathcal{R}(F)\big)\ \xrightarrow{\,d\,}\ H^{0}\!\big(\mathcal{R}(F)\big)\,\Big],
\]
concentrated in cohomological degrees \([-1,0]\).
Equivalently, \(\mathcal{R}(F)\in D^{[-1,0]}(k\text{-mod})\).
\end{proposition}

\begin{proof}
Since \(\mathcal{R}\) is \(t\)-exact by \textup{(B2)}, \(\tau_{\le 0}\mathcal{R}(F)\xrightarrow{\sim}\mathcal{R}(F)\xleftarrow{\sim}\tau_{\ge -1}\mathcal{R}(F)\).
Thus \(H^i(\mathcal{R}(F))=0\) for \(i\notin\{-1,0\}\).
Choose any representative two-term complex \(E_F=[E^{-1}\to E^0]\) for \(\mathcal{R}(F)\) with \(E^{-1},E^0\) finite-dimensional.
The canonical map from \(E_F\) to its cohomology model \([H^{-1}(\mathcal{R}(F))\to H^0(\mathcal{R}(F))]\) is a quasi-isomorphism, natural in \(F\).
\end{proof}

\begin{remark}
All statements below are invariant under filtered quasi-isomorphism on \(F\) and under isomorphism in \(D^{\mathrm{b}}(k\text{-mod})\) on \(\mathcal{R}(F)\).
\end{remark}

% -------------------------
\subsection*{C.2. The edge: \texorpdfstring{$H^{-1}(\mathcal{R}(F))\cong \varinjlim_t H_1(F^t)$}{H^{-1}(R(F)) ≅ colim H_1(F^t)} and naturality}
% -------------------------

\begin{proposition}[Edge identification and naturality]\label{C:prop:edge}
Under \textup{(B2)}, for every \(F\in \mathsf{FiltCh}(k)\) there is a natural isomorphism
\[
H^{-1}\!\big(\mathcal{R}(F)\big)\ \cong\ \varinjlim_{t\in\mathbb{R}}\, H_1(F^{t}C_\bullet).
\]
If \(f:F\to G\) is a filtration-preserving chain map, then the diagram
\[
\begin{tikzcd}[row sep=1.2em, column sep=2.1em]\nH^{-1}\!\big(\mathcal{R}(F)\big) \arrow[r, "\sim"] \arrow[d, "H^{-1}(\mathcal{R}(f))"'] &\n\varinjlim_{t}H_1(F^tC_\bullet) \arrow[d, "\varinjlim_{t} H_1(f^t)"] \\\nH^{-1}\!\big(\mathcal{R}(G)\big) \arrow[r, "\sim"] &\n\varinjlim_{t}H_1(G^tC_\bullet)\n\end{tikzcd}
\]
commutes.
\end{proposition}

\begin{proof}
By Proposition~\ref{C:prop:two-term}, \(\mathcal{R}(F)\) is represented by a two-term complex whose degree \(-1\) cohomology is functorial in \(F\).
By the construction of \(\mathcal{R}\) under \textup{(B2)} (Appendix~B), its degree \(-1\) layer encodes stabilized degree-\(1\) features; concretely, one may model \(\mathcal{R}\) via a mapping-telescope of the filtered system \(\{F^t\}_{t\in\mathbb{R}}\) truncated to degrees \(1\) and \(0\).
Over the field \(k\), filtered colimits in \(\mathsf{Vect}_k\) are exact and computed objectwise in \([\mathbb{R},\mathsf{Vect}_k]\) (Appendix~A, Remark~\ref{A:rk:filtered-colimits}).
Exactness allows passage of homology through filtered colimits in this bounded, finite-dimensional setting:
\[
H^{-1}\!\big(\mathcal{R}(F)\big)\ \cong\ H_1\!\Big(\operatorname{Tel}_t\,F^t\Big)\ \cong\ \varinjlim_t H_1(F^tC_\bullet).
\]
Naturality in \(f\) follows from the functoriality of the telescope construction and homology, and the universal property of colimits.
\end{proof}

% -------------------------
\subsection*{C.3. Computing \texorpdfstring{$\Ext^1$}{Ext1} for amplitude \([-1,0]\)}
% -------------------------

\begin{lemma}[Edge lemma for \(\Ext^1\)]\label{C:lem:edge-ext}
Let \(A\in D^{[-1,0]}(k\text{-mod})\).
Then there is a natural isomorphism
\[
\Ext^1(A,k)\ \cong\ \Hom\!\big(H^{-1}(A),\,k\big).
\]
\end{lemma}

\begin{proof}
Apply \(\Hom_{D^{\mathrm{b}}}(-,k[1])\) to the truncation triangle
\(\tau_{\le -1}A\to A\to \tau_{\ge 0}A\to (\tau_{\le -1}A)[1]\).
Since \(A\in D^{[-1,0]}\), \(\tau_{\le -1}A\simeq H^{-1}(A)[1]\) and \(\tau_{\ge 0}A\simeq H^{0}(A)[0]\).
Using semisimplicity of \(k\)-vector spaces, \(\Ext^1(H^{0}(A),k)=0\).
Hence
\[
\Hom(A,k[1])\ \cong\ \Hom\!\big(H^{-1}(A)[1],\,k[1]\big)\ \cong\ \Hom\!\big(H^{-1}(A),k\big),
\]
naturally in \(A\).
\end{proof}

\begin{corollary}[Dimension and duality]\label{C:cor:dim}
For any \(F\) with \(\mathcal{R}(F)\in D^{[-1,0]}\),
\[
\dim_k \Ext^1\!\big(\mathcal{R}(F),k\big)\ =\ \dim_k\, H^{-1}\!\big(\mathcal{R}(F)\big)\ =\ \dim_k\Big(\varinjlim_t H_1(F^t)\Big),
\]
and the canonical pairing \(\Ext^1(\mathcal{R}(F),k)\otimes H^{-1}(\mathcal{R}(F))\to k\) is perfect.
\end{corollary}

\begin{proof}
Combine Lemma~\ref{C:lem:edge-ext} with Proposition~\ref{C:prop:edge}.
Finite dimensionality yields perfect duality.
\end{proof}

\begin{remark}[Base change and choice of the test object]\label{C:rk:base-change}
For any field extension \(k\subset K\), if \(\mathcal{R}(F)\in D^{[-1,0]}\) then
\[
\Ext^1\!\big(\mathcal{R}(F),k\big)=0\quad\Longleftrightarrow\quad \Ext^1\!\big(\mathcal{R}(F)\otimes_k^{\mathbf{L}}K,\,K\big)=0,
\]
since both are equivalent to \(H^{-1}(\mathcal{R}(F))=0\) and \(H^{-1}\) commutes with scalar extension in this range.
Moreover, for any finite-dimensional \(k\)-vector space \(V\),
\(\Ext^1(\mathcal{R}(F),V)\cong \Hom(H^{-1}(\mathcal{R}(F)),V)\); hence testing against \(k[0]\) is equivalent to testing against \(V[0]\) for all \(V\), but in B-Gate\(^{+}\) we fix \(V=k\) (minimal unit).
\end{remark}

% -------------------------
\subsection*{C.4. The bridge and its robust/windowed form}
% -------------------------

\begin{theorem}[Bridge \(\mathrm{PH}_1\Rightarrow \Ext^1\)]\label{C:thm:bridge}
Let \(F\in \mathsf{FiltCh}(k)\).
If \(\mathrm{PH}_1(F)=0\) (equivalently, \(H_1(F^t)=0\) for all \(t\)), then
\[
\Ext^1\!\big(\mathcal{R}(F),\,k\big)\ =\ 0.
\]
\end{theorem}

\begin{proof}
If \(\mathrm{PH}_1(F)=0\) then \(\varinjlim_t H_1(F^t)=0\).
By Proposition~\ref{C:prop:edge}, \(H^{-1}(\mathcal{R}(F))=0\).
Apply Lemma~\ref{C:lem:edge-ext} to obtain \(\Ext^1(\mathcal{R}(F),k)=0\).
\end{proof}

\begin{corollary}[Robust/windowed bridge]\label{C:cor:windowed}
Fix \(\tau>0\).
If \(\mathrm{PH}_1\!\big(C_\tau(F)\big)=0\), then \(\Ext^1\!\big(\mathcal{R}(C_\tau(F)),\,k\big)=0\).
\end{corollary}

\begin{proof}
Apply Theorem~\ref{C:thm:bridge} to \(C_\tau(F)\).
By Appendix~B, \(C_\tau\) preserves the constructible range and \(\mathcal{R}(C_\tau(F))\in D^{[-1,0]}\), so \textup{(B2)} applies.
\end{proof}

\begin{remark}[No reverse claim]\label{C:rk:no-reverse}
We do \emph{not} assert \(\Ext^1(\mathcal{R}(F),k)=0 \Rightarrow \mathrm{PH}_1(F)=0\).
See \S\ref{C:ssec:counterexamples} for counterexamples.
\end{remark}

% -------------------------
\subsection*{C.5. Eligibility and B-Gate\texorpdfstring{$^{+}$}{+} policy}
% -------------------------

\begin{declaration}[Eligibility for using \(\Ext^1\) in B-Gate\(^{+}\)]\label{C:decl:eligibility}
Fix a collapse threshold \(\tau\ge 0\) and a window \([0,\tau]\).
The \(\Ext^1\)-test \(\Ext^1(\mathcal{R}(C_\tau F),k)=0\) is \emph{admitted into} B-Gate\(^{+}\) if and only if:
\begin{enumerate}\itemsep0.2em
  \item \(\mathcal{R}\) is \(t\)-exact and \(\mathcal{R}(C_\tau F)\in D^{[-1,0]}(k\text{-mod})\) \textbf{(amplitude \(\le 1\))};
  \item the scope rule of Appendix~A (constructibility and return from functor-category colimits) holds; and
  \item the test object is \(Q=k[0]\) (no other targets are used in gating).
\end{enumerate}
If any condition fails, the \(\Ext^1\)-value may be logged for diagnostics but is \emph{excluded} from the gate decision on that window.
\end{declaration}

\begin{remark}[Diagnostics outside the eligible regime]\label{C:rk:diag}
Outside amplitude \(\le 1\), record a flag \texttt{ext1\_eligible:false} and a reason (e.g.\ \texttt{amplitude\_gt\_1}).
B-Gate\(^{+}\) then relies on the persistence layer (\(\mathrm{PH}_1=0\)), tower isomorphisms \((\mu,\nu)=(0,0)\), and the safety margin \(\mathrm{gap}_\tau>\Sigma\delta\) only (cf.\ Chapters~10–12).
\end{remark}

% -------------------------
\subsection*{C.6. Naturality, exactness, and stability under admissible updates}
% -------------------------

\begin{proposition}[Naturality of the bridge]\label{C:prop:naturality}
For any filtration-preserving \(f:F\to G\) in \(\mathsf{FiltCh}(k)\), the diagram
\[
\begin{tikzcd}[row sep=1.2em, column sep=2.4em]\n\Ext^1\!\big(\mathcal{R}(F),k\big)\n  \arrow[r, "\sim"]\n  \arrow[d, "{\Ext^1(\mathcal{R}(f),\,k)}"'] &\n\Hom\!\big(H^{-1}(\mathcal{R}(F)),k\big)\n  \arrow[d, "{\Hom(H^{-1}(\mathcal{R}(f)),\,k)}"]\\\n\Ext^1\!\big(\mathcal{R}(G),k\big)\n  \arrow[r, "\sim"] &\n\Hom\!\big(H^{-1}(\mathcal{R}(G)),k\big)\n\end{tikzcd}
\]
commutes.
\end{proposition}

\begin{proof}
Combine naturality in Lemma~\ref{C:lem:edge-ext} and Proposition~\ref{C:prop:edge}.
\end{proof}

\begin{lemma}[Exact triangles and 2-out-of-3]\label{C:lem:2of3}
Suppose \(F_1\to F_2\to F_3\to F_1[1]\) is a distinguished triangle after realization on a window \([0,\tau]\) with \(\mathcal{R}(C_\tau F_i)\in D^{[-1,0]}\) for all \(i\).
If two of \(\Ext^1(\mathcal{R}(C_\tau F_i),k)\) vanish, then so does the third.
\end{lemma}

\begin{proof}
Apply \(\Hom(-,k[1])\) to the triangle in \(D^{\mathrm{b}}(k\text{-mod})\) and use exactness of \(\Hom(-,k[1])\) on distinguished triangles together with the amplitude restriction, which suppresses higher \(\Ext\)-terms.
\end{proof}

\begin{remark}[Stability under admissible updates]\label{C:rk:stability}
When \(F\mapsto F'\) is an update that is \emph{non-expansive} at the persistence layer (e.g.\ deletion-type or \(\varepsilon\)-continuation post-collapse) and \(\mathcal{R}(C_\tau F),\mathcal{R}(C_\tau F')\in D^{[-1,0]}\), then the verdict \(\Ext^1(\mathcal{R}(C_\tau F),k)=0\) persists to \(F'\) provided the edge groups remain isomorphic (or vanish) along the filtered colimit; this is immediate from Proposition~\ref{C:prop:edge} and Lemma~\ref{C:lem:edge-ext}.
Gating still defers to eligibility (Declaration~\ref{C:decl:eligibility}).
\end{remark}

% -------------------------
\subsection*{C.7. Implementation details, reproducibility, and logging}
% -------------------------

\begin{remark}[Run-time policy and manifest fields]\label{C:rk:runtime}
When B-Gate\(^{+}\) is executed, enforce the order
\(\,F\to C_\tau F\to \mathcal{R}(C_\tau F)\to \Ext^1(-,k)\,\) (the \textbf{collapse-first} policy).
The manifest \texttt{run.yaml} must include:
\begin{itemize}
  \item \texttt{ext1\_eligible}: boolean; \texttt{true} iff \(\mathcal{R}(C_\tau F)\in D^{[-1,0]}\);
  \item \texttt{ext1\_used\_in\_gate}: boolean; \texttt{true} iff \texttt{ext1\_eligible} and the gate policy enables it;
  \item \texttt{amplitude}: reported as \([-1,0]\) or \texttt{>1} (if known by static design or a certified check);
  \item \texttt{gate\_order}: fixed as \texttt{collapse→realize→ext1};
  \item \texttt{q\_test}: fixed as \texttt{k[0]}.
\end{itemize}
Outside eligibility, set \texttt{ext1\_eligible:false}, \texttt{ext1\_used\_in\_gate:false}, and continue the gate using persistence/tower/safety criteria only.
\end{remark}

\begin{remark}[Computational recipe]\label{C:rk:recipe}
On an eligible window, computing \(\Ext^1(\mathcal{R}(C_\tau F),k)\) reduces to:
\begin{enumerate}\itemsep0.2em
  \item compute \(\varinjlim_t H_1((C_\tau F)^t)\) by stabilizing the barcode of degree \(1\) on \([0,\tau]\) (finite-type);
  \item take the linear dual: \(\Ext^1 \cong \Hom(\varinjlim_t H_1,k)\).
\end{enumerate}
No derived resolutions are needed in the amplitude \([-1,0]\) regime.
\end{remark}

\begin{verbatim}
# run.yaml (excerpt)
ext1_eligible: true            # iff amplitude <= 1 was certified
ext1_used_in_gate: true        # iff eligible AND policy permits
amplitude: "[-1,0]"            # else ">1" or "unknown"
gate_order: "collapse→realize→ext1"
q_test: "k[0]"
notes: "PH1=0 implies Ext1=0 used; no reverse claim."
\end{verbatim}

% -------------------------
\subsection*{C.8. Counterexamples and boundary cases}\label{C:ssec:counterexamples}
% -------------------------

\begin{example}[Failure of the reverse implication \(\Ext^1\Rightarrow \mathrm{PH}_1\)]\label{C:ex:reverse-false}
Let \(F\) be any filtered chain complex whose degree-\(1\) barcode on \(\mathbb{R}\) consists of a single finite bar \((a,b)\) with \(a<b<\infty\), and no other degree-\(1\) features.
Then \(H_1(F^t)\neq 0\) for \(t\in(a,b)\) (so \(\mathrm{PH}_1(F)\neq 0\)), but \(\varinjlim_{t\to+\infty}H_1(F^t)=0\).
If \(\mathcal{R}(F)\in D^{[-1,0]}\) (eligible), Corollary~\ref{C:cor:dim} gives
\(\Ext^1(\mathcal{R}(F),k)\cong \Hom(\varinjlim_t H_1(F^t),k)=0\).
Thus \(\Ext^1=0\) while \(\mathrm{PH}_1\neq 0\).
\end{example}

\begin{example}[Amplitude breach: diagnostics only]\label{C:ex:amplitude-breach}
If \(\mathcal{R}(C_\tau F)\in D^{[-2,0]}\) with \(H^{-2}\neq 0\), then Lemma~\ref{C:lem:edge-ext} fails.
It is possible that \(\Ext^1(\mathcal{R}(C_\tau F),k)=0\) due to cancellations between \(H^{-2}\) and \(H^{-1}\), even when \(\varinjlim_t H_1((C_\tau F)^t)\neq 0\).
Hence the \(\Ext^1\)-test is \emph{ineligible} and must not be used for gating (Remark~\ref{C:rk:diag}).
\end{example}

\begin{example}[Non-constructible tails]\label{C:ex:nonconstructible}
If \(\mathbf{P}_1(F)\) is not constructible (e.g.\ infinitely many critical parameters accumulating in a bounded window), filtered colimits in \([\mathbb{R},\mathsf{Vect}_k]\) may leave \(\Pers^{\mathrm{ft}}_k\).
The edge identification becomes non-operative for gating; eligibility fails by \textup{(B1)}.
\end{example}

% -------------------------
\subsection*{C.9. Additional safeguards and best practices}
% -------------------------

\begin{remark}[Monotonicity across windows]
If \(0<\tau\le \tau'\), eligibility at \(\tau'\) implies eligibility at \(\tau\).
Moreover, if \(\Ext^1(\mathcal{R}(C_{\tau'}F),k)=0\), then \(\Ext^1(\mathcal{R}(C_{\tau}F),k)=0\) by functoriality of truncation and Corollary~\ref{C:cor:dim}.
\end{remark}

\begin{remark}[Uniformity under base change]
By Remark~\ref{C:rk:base-change}, eligibility and the conclusion \(\Ext^1=0\) are stable under extending scalars \(k\subset K\).
Thus the gate verdict is field-independent within the amplitude \([-1,0]\) regime.
\end{remark}

\begin{remark}[Uniqueness of the test]
In the eligible regime, replacing \(k[0]\) by any \(V[0]\) with \(V\in k\text{-mod}\) yields an equivalent vanishing test.
B-Gate\(^{+}\) fixes the unit \(k[0]\) to avoid ambiguity.
\end{remark}

% -------------------------
\subsection*{C.10. Scope and non-claims}
% -------------------------

\begin{remark}[Scope and non-claims]\label{C:rk:scope}
\textbf{The bridge \(\mathrm{PH}_1\Rightarrow \Ext^1\) is used and proved only in \(D^{\mathrm{b}}(k\text{-mod})\).}
The converse implication \(\Ext^1(\mathcal{R}(F),k)=0 \Rightarrow \mathrm{PH}_1(F)=0\) is \emph{false} in general (Example~\ref{C:ex:reverse-false}).
All filtered colimit uses obey the scope rule stated at the start of this appendix (Appendix~A, Remark~\ref{A:rk:filtered-colimits}).
Derived realizations into other targets (e.g.\ coherent/étale) appear only as \textbf{[Spec]} in the main text and \emph{do not} extend the proved bridge.
\end{remark}

\medskip
\noindent\emph{Summary of Appendix C (reinforced and final).}
Under \((\mathrm{B}1)\)–\((\mathrm{B}3)\), the edge identification \(H^{-1}(\mathcal{R}(F))\cong \varinjlim_t H_1(F^t)\) and the amplitude \([-1,0]\) model yield the bridge \(\mathrm{PH}_1(F)=0\Rightarrow \Ext^1(\mathcal{R}(F),k)=0\).
The \(\Ext^1\)-test is included in B-Gate\(^{+}\) \textbf{only} when \(\mathcal{R}(C_\tau F)\in D^{[-1,0]}\) (eligibility \(\le 1\)), and the operational order is \textbf{collapse} \(\to\) \textbf{realize} \(\to\) \textbf{Ext} \(\,(\textbf{collapse-first})\).
Outside eligibility, \(\Ext^1\) is excluded from gate decisions but may be logged.
No reverse implication is claimed, and counterexamples/boundaries are documented in \S\ref{C:ssec:counterexamples}.
All computations respect constructibility and filtered-colimit scope from Appendix~A.



% =========================
\section*{Appendix D. Towers, \texorpdfstring{$\mu,\nu$}{mu,nu}, and Examples [Proof/Example] (reinforced)}
% =========================

\addcontentsline{toc}{section}{Appendix D. Towers, $\mu,\nu$, and Examples}
\refstepcounter{section} % ensure stable cross-references with unnumbered section

\usetikzlibrary{arrows.meta,cd}
Throughout, fix a field \(k\).
We work in the constructible regime (Appendix~A).
For each degree \(i\in\ZZ\), let
\[
\Pfun_i:\ \FiltCh(k)\longrightarrow \Pers^{\mathrm{ft}}_k,\n\qquad\nF\longmapsto \big(t\mapsto H_i(F^{t}C_\bullet)\big)
\]
send a bounded-in-degree filtered chain complex \(F\) to its constructible persistence module.
The truncation \(\T_\tau:\Pers^{\mathrm{ft}}_k\to\Pers^{\mathrm{ft}}_{k,\tau\text{-loc}}\) is exact and \(1\)-Lipschitz (Appendix~A).
Filtered colimits are computed objectwise in \([\RR,\Vect_k]\), with the scope rule of Appendix~A, Remark~\ref{A:rk:filtered-colimits}.
All statements at the filtered–complex layer are \emph{up to filtered quasi-isomorphism (f.q.i.)}; persistence-layer statements take place in \(\Pers^{\mathrm{ft}}_k\).
All quantities below may depend on the threshold \(\tau>0\); no monotonicity in \(\tau\) is asserted.

\medskip
\noindent\textbf{Comparison map and obstruction indices.}
Let \(\{F_n\}_{n\in\NN}\) be a directed system in \(\FiltCh(k)\).
Let \(F_\infty\) be an apex equipped with a cocone \(F_n\to F_\infty\) (indexing category \(\NN\cup\{\infty\}\) with unique morphisms \(n\to\infty\)).
For each \(i\) and \(\tau>0\) set
\[
\phi_{i,\tau}:\ \colim_{n\in\NN}\, \T_\tau\!\big(\Pfun_i(F_n)\big)\ \longrightarrow\ \T_\tau\!\big(\Pfun_i(F_\infty)\big),
\]
the canonical comparison in \([\RR,\Vect_k]\).
Define the \emph{tower obstruction indices}
\[
\mu_{i,\tau}:=\gdim\Ker(\phi_{i,\tau}),\n\qquad\n\nu_{i,\tau}:=\gdim\coker(\phi_{i,\tau}),\n\qquad\n\muc:=\sum_i\mu_{i,\tau},\ \ \nuc:=\sum_i\nu_{i,\tau}.
\]
These sums are finite because complexes are bounded in homological degrees (constructible range).
The pair \((\muc,\nuc)\) detects \emph{Type~IV} (tower-level) failures at scale~\(\tau\).\footnote{For compositions in finite-dimensional linear algebra one has surrogate subadditivity
\(\dim\Ker(g\circ f)\le \dim\Ker f+\dim\Ker g\) and
\(\dim\coker(g\circ f)\le \dim\coker g+\dim\Ker f\).
Appendix~J provides the persistence-layer analogue after applying \(\T_\tau\) and taking generic-fiber dimensions.}

\begin{remark}[Generic dimension after truncation]\label{rem:D-generic-dim}
In \(\Pers^{\mathrm{ft}}_k\), after applying \(\T_\tau\) the kernel and cokernel of any morphism decompose (noncanonically) as finite direct sums of interval modules.
We write \(\gdim(-)\) for the \emph{generic fiber} dimension, i.e.\ the multiplicity of the infinite bar \(I[0,\infty)\) in that decomposition.
Finite bars contribute zero generic fiber.
\end{remark}

\begin{remark}[Invariance of \((\muc,\nuc)\)]
The indices \((\muc,\nuc)\) are invariant under levelwise f.q.i.\ replacements of the tower and apex: if \(F_n\simeq_{\mathrm{f.q.i.}}F'_n\) and \(F_\infty\simeq_{\mathrm{f.q.i.}}F'_\infty\), then \(\Pfun_i\) sends these to isomorphisms in \(\Pers^{\mathrm{ft}}_k\), hence \(\Ker/\coker\) (and thus \(\mu,\nu\)) are unchanged.
They are also invariant under cofinal reindexing of the tower, since filtered colimits over cofinal subdiagrams are canonically isomorphic.
\end{remark}

\begin{figure}[t]
\centering
\begin{tikzcd}[row sep=1.0em, column sep=2.0em]
F_1 \arrow[r] \arrow[dr] &
F_2 \arrow[r] \arrow[dr] &
\cdots \arrow[r] &
F_n \arrow[r] \arrow[dr] &
\cdots \arrow[r] &
F_\infty \\\n& \Pfun_i(F_1) \arrow[r] &
\Pfun_i(F_2) \arrow[r] &
\cdots \arrow[r] &
\Pfun_i(F_n) \arrow[r] &
\Pfun_i(F_\infty)
\end{tikzcd}
\caption{A tower with apex \(F_\infty\) and its image under \(\Pfun_i\).
The comparison \(\phi_{i,\tau}\) (defined after applying \(\T_\tau\)) measures the failure of the cocone to exhibit a colimit at scale \(\tau\).}
\end{figure}

\subsection*{D.1. Calculus of defects: generic-fiber interpretation, naturality, and subadditivity}

We collect formal properties of the obstruction indices \(\mu,\nu\) that are used implicitly throughout the paper and make them explicit for operational use.

\begin{proposition}[Generic-fiber interpretation]\label{prop:D-gdim-iff}
Let \(f:M\to N\) be a morphism in \(\Pers^{\mathrm{ft}}_k\) and fix \(\tau>0\).
Then, in the barcode decomposition of \(\Ker(\T_\tau f)\) and \(\coker(\T_\tau f)\), the multiplicity of \(I[0,\infty)\) equals \(\gdim\Ker(\T_\tau f)\) and \(\gdim\coker(\T_\tau f)\), respectively.
Equivalently, \(\gdim\) is the generic fiber dimension of the corresponding functor \(\RR\to\Vect_k\).
\end{proposition}

\begin{proof}
This is standard for constructible 1D persistence over a field: after \(\T_\tau\), objects are pointwise finite-dimensional and split as finite sums of interval modules; see Appendix~A.
The generic fiber (dimension on a cofinal ray) counts infinite bars, i.e.\ copies of \(I[0,\infty)\).
\end{proof}

\begin{definition}[Morphisms of towers and naturality of \(\phi\)]
A \emph{morphism of towers with apex} \((F_\bullet,F_\infty)\to(G_\bullet,G_\infty)\) is a collection of maps \(u_n:F_n\to G_n\) and \(u_\infty:F_\infty\to G_\infty\) commuting with all structure maps to the apex.
Applying \(\Pfun_i\), then \(\T_\tau\), and passing to the filtered colimit yields a commutative square
\[
\begin{tikzcd}\n\colim_n \T_\tau \Pfun_i(F_n) \arrow[r,"\phi^{F}_{i,\tau}"] \arrow[d,"\colim\,\T_\tau\Pfun_i(u_n)"'] &\n\T_\tau \Pfun_i(F_\infty) \arrow[d,"\T_\tau \Pfun_i(u_\infty)"] \\\n\colim_n \T_\tau \Pfun_i(G_n) \arrow[r,"\phi^{G}_{i,\tau}"] &\n\T_\tau \Pfun_i(G_\infty)\n\end{tikzcd}
\]
i.e.\ \(\phi_{i,\tau}\) is natural in the tower.
\end{definition}

\begin{proposition}[Functoriality and subadditivity under composition]\label{prop:D-subadd}
Let \((F_\bullet,F_\infty)\xrightarrow{u}(G_\bullet,G_\infty)\xrightarrow{v}(H_\bullet,H_\infty)\) be morphisms of towers with apex.
Then, for each \(i,\tau\),
\[
\mu(\phi^{H}_{i,\tau}\!\circ\!\colim\,\T_\tau\Pfun_i(v_n\circ u_n))\n\ \le\\n\mu(\phi^{G}_{i,\tau}\!\circ\!\colim\,\T_\tau\Pfun_i(u_n))+\mu(\phi^{H}_{i,\tau}\!\circ\!\colim\,\T_\tau\Pfun_i(v_n)),
\]
\[
\nu(\phi^{H}_{i,\tau}\!\circ\!\colim\,\T_\tau\Pfun_i(v_n\circ u_n))\n\ \le\\n\nu(\phi^{H}_{i,\tau}\!\circ\!\colim\,\T_\tau\Pfun_i(v_n))+\mu(\phi^{G}_{i,\tau}\!\circ\!\colim\,\T_\tau\Pfun_i(u_n)).
\]
In particular, writing simply \(\phi=\phi_{i,\tau}\) for a fixed tower, any factorization of \(\phi\) yields
\(\mu(\phi\circ\psi)\le \mu(\psi)+\mu(\phi)\) and \(\nu(\phi\circ\psi)\le \nu(\phi)+\mu(\psi)\).
\end{proposition}

\begin{proof}
Apply \(\T_\tau\) and use exactness together with the standard inequalities for kernels and cokernels of compositions in finite-dimensional linear algebra, then pass to generic fibers via Proposition~\ref{prop:D-gdim-iff}.
\end{proof}

\begin{proposition}[Additivity on finite direct sums]\label{prop:D-additive}
For two towers \((F_\bullet,F_\infty)\) and \((G_\bullet,G_\infty)\),
\[
\mu\big((F\oplus G)_\bullet,(F\oplus G)_\infty\big)\n=\mu(F_\bullet,F_\infty)+\mu(G_\bullet,G_\infty),
\]
\[
\nu\big((F\oplus G)_\bullet,(F\oplus G)_\infty\big)\n=\nu(F_\bullet,F_\infty)+\nu(G_\bullet,G_\infty).
\]
\end{proposition}

\begin{proof}
\(\Pfun_i\) and \(\T_\tau\) preserve finite direct sums; kernels and cokernels preserve finite direct sums; \(\gdim\) is additive on direct sums.
\end{proof}

\begin{proposition}[Invariance under f.q.i.\ and cofinal reindexing]\label{prop:D-invariance}
If \(F_n\simeq_{\mathrm{f.q.i.}}F'_n\) levelwise and \(F_\infty\simeq_{\mathrm{f.q.i.}}F'_\infty\), then \(\mu,\nu\) agree for the two towers.
If \(J\subset \NN\) is cofinal, then restricting the tower to \(J\) does not change \(\mu,\nu\).
\end{proposition}

\begin{proof}
Follows from \(\Pfun_i\) sending f.q.i.\ to isomorphisms in \(\Pers^{\mathrm{ft}}_k\), exactness of \(\T_\tau\), and the fact that filtered colimits over cofinal subdiagrams are canonically isomorphic.
\end{proof}

\subsection*{D.2. Toy towers: pure kernel / pure cokernel / mixed}

\begin{example}[Pure cokernel at a fixed scale]\label{D:ex:pure-coker}
Fix \(\tau>0\) and degree \(i=1\).
Let \(\Pfun_1(F_n)=I[0,\tau-\tfrac1n)\) with transition maps the evident inclusions.
Let \(F_\infty\) satisfy \(\Pfun_1(F_\infty)=I[0,\infty)\).
Then \(\T_\tau(\Pfun_1(F_n))=0\) for all \(n\), whereas \(\T_\tau(\Pfun_1(F_\infty))\cong I[0,\infty)\).
Hence \(\phi_{1,\tau}:0\to I[0,\infty)\) has trivial kernel and nontrivial cokernel, so \(\mu_{1,\tau}=0\) and \(\nu_{1,\tau}=1\) (pure cokernel).
\end{example}

\begin{example}[Pure kernel at a fixed scale]\label{D:ex:pure-ker}
Fix \(\tau>0\).
Let \(\Pfun_1(F_n)=I[0,\infty)\) for all \(n\), with transition maps the identities (a stationary directed system).
Let \(F_\infty\) satisfy \(\Pfun_1(F_\infty)=0\), and take the cocone \(\Pfun_1(F_n)\to \Pfun_1(F_\infty)\) to be \(0\) for all \(n\).
Then \(\T_\tau(\Pfun_1(F_n))\cong I[0,\infty)\) for all \(n\), so the source of \(\phi_{1,\tau}\) is \(I[0,\infty)\), while the target is \(0\).
Thus \(\phi_{1,\tau}: I[0,\infty)\to 0\) has nontrivial kernel and zero cokernel, hence \(\mu_{1,\tau}=1\), \(\nu_{1,\tau}=0\) (pure kernel).
\end{example}

\begin{example}[Mixed]\label{D:ex:mixed}
Fix \(\tau>0\) and set
\[
\Pfun_1(F_n)\ =\ I[0,\tau-\tfrac1n)\ \oplus\ I[0,\infty),
\]
with transition maps the obvious inclusions on the first summand and the identities on the second.
Take \(F_\infty\) with \(\Pfun_1(F_\infty)=I[0,\infty)\oplus 0\), using cocone maps that send the second summand to \(0\).
Then the first summand yields a cokernel contribution exactly as in Example~\ref{D:ex:pure-coker}, while the second yields a kernel contribution exactly as in Example~\ref{D:ex:pure-ker}.
Hence \(\mu_{1,\tau}=\nu_{1,\tau}=1\) (mixed).
\end{example}

All three persistence-level towers are realizable by filtered complexes via the interval-realization assignment \(\Ufun\) (Appendix~B), up to f.q.i.; constructibility is preserved.

\subsection*{D.3. When \texorpdfstring{$\phi_{i,\tau}$}{phi} is an isomorphism: \texorpdfstring{$(\muc,\nuc)=(0,0)$}{(mu,nu)=(0,0)}}

\begin{proposition}[Isomorphism criterion]\label{D:prop:iso-zero}
Assume:
\begin{enumerate}[label=(\roman*)]
\item degreewise filtered colimits in \(\FiltCh(k)\) are computed objectwise on chains and filtrations;
\item each \(\Pfun_i(F_n)\) lies in \(\Pers^{\mathrm{ft}}_k\);
\item \(\T_\tau\) commutes with the filtered colimit of \(\{\Pfun_i(F_n)\}\) in \([\RR,\Vect_k]\), and the result is constructible (Appendix~A, Theorem~\ref{A:thm:localization});
\item the cocone exhibits a colimit at persistence level: the canonical map
\(\colim_{n}\Pfun_i(F_n)\xrightarrow{\ \cong\ }\Pfun_i(F_\infty)\)
is an isomorphism in \([\RR,\Vect_k]\).
\end{enumerate}
Then for every \(i\) and \(\tau>0\), the comparison map \(\phi_{i,\tau}\) is an isomorphism in \(\Pers^{\mathrm{ft}}_k\).
Consequently \((\muc,\nuc)=(0,0)\).
\end{proposition}

\begin{remark}
Condition (iv) is automatic if \(F_\infty\) \emph{is} the colimit of \(\{F_n\}\) in a model of filtered complexes for which \(\Pfun_i\) is computed objectwise and the scope rule of Appendix~A applies; no claim is made beyond that regime.
\end{remark}

\subsection*{D.4. Sufficient conditions ensuring \texorpdfstring{$(\muc,\nuc)=(0,0)$}{(mu,nu)=(0,0)}}

The summability condition \(\sum_{n} d_{\mathrm{int}}\!\big(\Pfun_i(F_{n+1}),\Pfun_i(F_n)\big)<\infty\) \emph{alone} does not guarantee \((\muc,\nuc)=(0,0)\); see §D.5.1.
The following hypotheses are sufficient.

\begin{theorem}\label{D:thm:sufficient}
Fix \(i\) and \(\tau>0\).
Each of the following implies that \(\phi_{i,\tau}\) is an isomorphism (hence \((\muc,\nuc)=(0,0)\)):
\begin{enumerate}[label=(S\arabic*)]
\item \textbf{Commutation and apex colimit:} \(\T_\tau\) commutes with the filtered colimit of \(\{\Pfun_i(F_n)\}\) in \([\RR,\Vect_k]\), the outcome is constructible, and the cocone exhibits a colimit at persistence level (i.e.\ Proposition~\ref{D:prop:iso-zero}(iv) holds).
\item \textbf{No \(\tau\)-accumulation from below:} there exists \(\eta>0\) such that, for all sufficiently large \(n\), no bar in \(\Pfun_i(F_n)\) has length in the half-open interval \((\tau-\eta,\tau)\).
Equivalently, there is no sequence of bar lengths strictly increasing to \(\tau\).
\item \textbf{\(\T_\tau\)-Cauchy with compatible cocone:} the sequence \(\T_\tau(\Pfun_i(F_n))\) is Cauchy in the interleaving metric, and the cocone to \(\T_\tau(\Pfun_i(F_\infty))\) identifies the metric limit with the colimit target.
(Here we use only the standard completeness/uniqueness of limits for p.f.d.\ barcodes under the bottleneck/interleaving metric.)
\end{enumerate}
\end{theorem}

\begin{proof}
(S1) is Proposition~\ref{D:prop:iso-zero}.
For (S2), the gap prevents creation at the apex of new bars of length \(>\tau\):
every long bar in \(\T_\tau(\Pfun_i(F_\infty))\) must appear at some finite stage and stabilize, yielding bijectivity on interval factors.
For (S3), completeness of the space of p.f.d.\ persistence modules up to isometry implies a unique metric limit; the stated compatibility identifies it with the colimit target, so \(\phi_{i,\tau}\) is an isometry and hence an isomorphism in \(\Pers^{\mathrm{ft}}_k\).
\end{proof}

\subsection*{D.5. A counterexample: \(\sum d_{\mathrm{int}}<\infty\) yet \((\muc,\nuc)\neq(0,0)\)}\label{subsec:D-counter}

\begin{example}[Summable increments, pure cokernel at the apex]\label{D:ex:summable}
Fix \(\tau>0\) and set \(\ell_n=\tau-\sum_{m\ge n}2^{-m}\uparrow \tau\), so that
\(\sum_{n}(\ell_{n+1}-\ell_{n})=\sum_{n}2^{-n}<\infty\).
Let \(M_n:=I[0,\ell_n)\) with \(M_{n}\into M_{n+1}\) the standard inclusions.
Then
\[
d_{\mathrm{int}}(M_{n},M_{n+1})=\tfrac12(\ell_{n+1}-\ell_{n})=2^{-(n+1)},\qquad \sum_{n} d_{\mathrm{int}}(M_{n},M_{n+1})<\infty.
\]
Let \(\Pfun_1(F_n)=M_n\), and choose an apex with \(\Pfun_1(F_\infty)=I[0,\infty)\).
For every \(n\), \(\T_\tau(M_n)=0\), while \(\T_\tau(\Pfun_1(F_\infty))=I[0,\infty)\).
Thus \(\muc=0\), \(\nuc=1\) (pure cokernel), despite the summable interleaving distances along the tower.
\end{example}

This shows that \(\sum d_{\mathrm{int}}<\infty\) alone is insufficient to force \((\muc,\nuc)=(0,0)\).

\subsection*{D.6. Converse failures and the Type~IV catalog}

\paragraph{D.6.1. \(\Ext^1=0\) does not imply \(\PH_1=0\).}
Let \(A\in D^{[-1,0]}(k\text{-mod})\) with \(H^{-1}(A)=0\) and \(H^0(A)\ne 0\), e.g.\ the stalk complex \(V[0]\) for a nonzero \(k\)-space \(V\).
Then \(\Ext^1(A,k)\cong \Hom(H^{-1}(A),k)=0\) by Appendix~C (Lemma~\ref{C:lem:edge-ext}).
Choose \(F\in\FiltCh(k)\) with \(\Pfun_1(F)\neq 0\) (e.g.\ a single finite interval) and \(\Rfun(F)\simeq A\); this can be arranged up to f.q.i.\ using the realization assignment \(\Ufun\) (Appendix~B).
Hence \(\Ext^1(\Rfun(F),k)=0\) while \(\PH_1(F)\neq 0\), refuting the converse of the bridge.

\paragraph{D.6.2. Type~IV (pure cokernel) at fixed \(\tau\).}
Example~\ref{D:ex:pure-coker} exhibits \(\muc=0\), \(\nuc>0\) with \(\T_\tau(\Pfun_1(F_n))=0\) for all \(n\) but \(\T_\tau(\Pfun_1(F_\infty))\neq 0\).
Thus finite layers appear admissible while the apex fails.

\paragraph{D.6.3. Type~IV (mixed).}
Example~\ref{D:ex:mixed} yields \(\muc>0\) and \(\nuc>0\) simultaneously, demonstrating that both kernel and cokernel defects can occur in the same tower.

\paragraph{D.6.4. Realization notes.}
All persistence-level constructions above are realizable by filtered complexes via \(\Ufun\) (Appendix~B), up to f.q.i.; constructibility is preserved.

\subsection*{D.7. Restart/Summability for window pasting}\label{subsec:D-restart}

All persistence-layer statements below are made \emph{after} applying \(\T_\tau\).

\begin{definition}[Per-window safety margin and pipeline budget]\label{D:def:window-budget}
Let \(\{W_k=[u_k,u_{k+1})\}_{k\in K}\) be a MECE partition (Appendix~A, Definition~\ref{A:def:MECE}). On each window \(W_k\), fix a collapse threshold \(\tau_k>0\). For a degree \(i\), define the \emph{pipeline budget}
\[
\Sigma\delta_k(i)\ :=\ \sum_{U\in W_k}\Big(\delta^{\mathrm{alg}}_{U}(i,\tau_k)+\delta^{\mathrm{disc}}_{U}(i,\tau_k)+\delta^{\mathrm{meas}}_{U}(i,\tau_k)\Big),
\]
and the \emph{safety margin} \(\mathrm{gap}_{\tau_k}(i)>0\) as the configured slack for B\hyp Gate\(^{+}\) on \(W_k\) and degree \(i\).
\end{definition}

\begin{lemma}[Restart inequality]\label{D:lem:restart}
Assume that, on window \(W_k\), B\hyp Gate\(^{+}\) passes with \(\mathrm{gap}_{\tau_k}(i)>\Sigma\delta_k(i)\), and that the transition to \(W_{k+1}\) is realized by a finite composition of \emph{deletion-type} steps and \(\varepsilon\)\hyp continuations (both measured after \(\T_\tau\)). Then there exists \(\kappa\in(0,1]\), depending only on the admissible step class and the \(\tau\)-adaptation policy, such that
\[
\mathrm{gap}_{\tau_{k+1}}(i)\ \ge\ \kappa\bigl(\mathrm{gap}_{\tau_k}(i)-\Sigma\delta_k(i)\bigr).
\]
\end{lemma}

\begin{proof}[Proof sketch]
Deletion-type steps are non\hyp increasing for the monitored indicators after \(\T_\tau\) (Appendix~E), and \(\varepsilon\)\hyp continuations are \(1\)\hyp Lipschitz. Aggregating drifts yields the stated retention factor \(\kappa\).
\end{proof}

\begin{definition}[Summability]\label{D:def:summability}
A run satisfies \emph{Summability} (on a degree set \(I\subset\ZZ\)) if
\[
\sum_{k\in K}\Sigma\delta_k(i)\ <\ \infty\qquad(\forall\,i\in I).
\]
A sufficient design is a geometric decay of \(\tau_k\) (hence of spectral/temporal bins) and bounded per\hyp window step counts.
\end{definition}

\begin{theorem}[Pasting windowed certificates]\label{D:thm:pasting}
Let \(\{W_k\}_k\) be MECE, and on each \(W_k\) let B\hyp Gate\(^{+}\) pass with \(\mathrm{gap}_{\tau_k}(i)>\Sigma\delta_k(i)\) for all \(i\in I\). If the Restart inequality (Lemma~\ref{D:lem:restart}) holds at every transition and Summability (Definition~\ref{D:def:summability}) holds, then the concatenation of windowed certificates yields a global certificate on \(\bigcup_k W_k\) for the degrees \(i\in I\).
\end{theorem}

\begin{proof}[Proof sketch]
Iterate Lemma~\ref{D:lem:restart} and sum budgets; Summability ensures that the cumulative loss of the safety margin remains bounded, so positivity of the margin persists. MECE coverage (Appendix~A) ensures there are no gaps/overlaps.
\end{proof}

\subsection*{D.8. Stability bands, \texorpdfstring{$\tau$}{tau}-sweeps, and detection algorithm}\label{subsec:D-stab-bands}

\begin{definition}[Stability band via \(\tau\)-sweep]\label{D:def:stab-band}
Fix a window \(W\) and degree \(i\). Let \(\{\tau_\ell\}_{\ell=1}^L\) be an increasing \(\tau\)-sweep. A contiguous block \(\{\tau_a,\dots,\tau_b\}\) is a \emph{stability band} if
\[
\mu_{i,\tau_\ell}=\nu_{i,\tau_\ell}=0\quad\text{for all }\ \ell\in\{a,\dots,b\},
\]
and the verdict persists upon \emph{refining} the sweep (inserting new \(\tau\)-values) without introducing \(\mu\) or \(\nu\) in the band.
\end{definition}

\begin{proposition}[Robust detection of stability bands]\label{D:prop:robust-band}
Assume (S1)–(S3) of Theorem~\ref{D:thm:sufficient} hold on \(W\). Then any sufficiently fine \(\tau\)-sweep admits stability bands covering all \(\tau\) at which \(\phi_{i,\tau}\) is an isomorphism; conversely, detecting a stability band by a sweep and its refinement certifies \((\muc,\nuc)=(0,0)\) on the band.
\end{proposition}

\begin{proof}[Proof sketch]
Under (S1)–(S3), \(\phi_{i,\tau}\) is an isomorphism on open neighborhoods of the corresponding \(\tau\)’s. A fine sweep samples each neighborhood; refinement eliminates aliasing. The converse follows by definition.
\end{proof}

\begin{remark}[Caveat: non-monotonicity in \(\tau\)]
There is no general monotonicity of \(\mu_{i,\tau}\) or \(\nu_{i,\tau}\) in \(\tau\). Stability bands may be separated by isolated \(\tau\)-values where \(\phi_{i,\tau}\) fails to be an isomorphism.
\end{remark}

\paragraph{Detection algorithm (auditable).}
Given a sweep \(\mathsf{TauSweep}=\{\tau_\ell\}_{\ell=1}^L\):
\begin{enumerate}[label=(A\arabic*)]
\item For each \(\ell\), compute \(\phi_{i,\tau_\ell}\) and record \((\mu_{i,\tau_\ell},\nu_{i,\tau_\ell})\).
\item Extract maximal contiguous indices \([a,b]\) with \((\mu,\nu)=(0,0)\).
\item Refine by inserting midpoints \(\tau'=\frac12(\tau_\ell+\tau_{\ell+1})\) within each candidate band and recompute \((\mu,\nu)\).
\item Accept a band if all refined points also yield \((0,0)\).
\item Emit certificate with hashes of inputs, tower metadata, and the flags indicating which of (S1)–(S3) were used.
\end{enumerate}

A minimal YAML schema for audit is provided in §D.10.

\subsection*{D.9. Implementation guide: APIs, stubs, and tests}\label{subsec:D-impl}

This section provides concrete, auditable artifacts for engineering teams. All persistence-layer computations are understood after applying \(\T_\tau\).

\subsection*{Lean stubs (illustrative).}
\begin{lstlisting}[language=,caption={Lean 4 stubs for towers and obstruction indices}]
namespace PH
structure Tower (ι : Type) :=
  (F    : ι → FiltCh)
  (apex : FiltCh)
  (toApex : ∀ i : ι, ChainMap (F i) apex)

def P_i (i : ℤ) : FiltCh → Pers := -- assumed given
def T_tau (τ : ℝ) : Pers → Pers := -- exact, 1-Lipschitz

def phi (i : ℤ) (τ : ℝ) (T : Tower ℕ) : PersHom :=
  have src := colim (fun n => T_tau τ (P_i i (T.F n)))
  have tgt := T_tau τ (P_i i T.apex)
  comparison src tgt -- canonical

def gdim (M : Pers) : Nat := -- multiplicity of I[0,∞) in barcode

def mu (i : ℤ) (τ : ℝ) (T : Tower ℕ) : Nat := gdim (kernel (phi i τ T))
def nu (i : ℤ) (τ : ℝ) (T : Tower ℕ) : Nat := gdim (cokernel (phi i τ T))
end PH
\end{lstlisting}

\subsection*{Coq stubs (illustrative).}
\begin{lstlisting}[language=,caption={Coq stubs for towers and obstruction indices}]
Module PH.
Record Tower := {
  F      : nat -> FiltCh;
  apex   : FiltCh;
  toApex : forall n, ChainMap (F n) apex
}.

Parameter P_i   : Z -> FiltCh -> Pers.
Parameter T_tau : R -> Pers -> Pers. (* exact, 1-Lipschitz *)

Definition phi (i:Z) (tau:R) (T:Tower) : PersHom :=
  let src := colim (fun n => T_tau tau (P_i i (F T n))) in
  let tgt := T_tau tau (P_i i (apex T)) in
  comparison src tgt.

Parameter gdim : Pers -> nat.

Definition mu (i:Z) (tau:R) (T:Tower) : nat := gdim (kernel (phi i tau T)).
Definition nu (i:Z) (tau:R) (T:Tower) : nat := gdim (cokernel (phi i tau T)).
End PH.
\end{lstlisting}

\paragraph{Sample tests.}
\begin{enumerate}[label=({T}\arabic*)]
\item \textbf{T3 (Filtered-colim stability).} Construct a tower whose apex is the filtered colimit and for which \(\T_\tau\) commutes with colim. Verify \(\phi_{i,\tau}\) is an iso and \((\mu,\nu)=(0,0)\).
\item \textbf{T7 (Toy towers).} Instantiate the pure kernel, pure cokernel, and mixed towers of §D.2 and confirm \((\mu,\nu)=(1,0),(0,1),(1,1)\), respectively.
\item \textbf{T9 (No \(\tau\)-accumulation).} Create barcodes with an \(\eta\)-gap below \(\tau\); confirm \(\phi_{i,\tau}\) is iso.
\item \textbf{T10 (Cauchy+compatibility).} Build a Cauchy sequence in the bottleneck metric whose limit equals the apex post-\(\T_\tau\); confirm iso.
\item \textbf{T11 (Restart+Summability).} Simulate windows and transitions satisfying Lemma~\ref{D:lem:restart} and Definition~\ref{D:def:summability}; verify global certificate via Theorem~\ref{D:thm:pasting}.
\end{enumerate}

\subsection*{D.10. Audit schema: \texttt{run.yaml} and HDF5 layout}\label{subsec:D-audit}

\subsection*{YAML fields (mandatory).}
\begin{lstlisting}[language=,caption={Minimal audit fields in run.yaml}]
phi:
  idx:
    - i: 1
      tau: 0.75
      iso: true          # φ_{i,τ} isomorphism?
      mu: 0
      nu: 0
      flags:
        S1: true         # used commutation+apex-colim
        S2: false
        S3: false
      iso_tail:
        passed: true     # tail check on refined sweep
        refinement_levels: 2
windows:
  collapse:
    tau_sweep: [0.5, 0.6, 0.7, 0.75, 0.8]
persistence:
  phi_iso_tail: "strict" # policy for refinement acceptance
tower:
  edges:
    - src: 0; dst: 1; kind: "inclusion"
    - src: 1; dst: 2; kind: "inclusion"
hash:
  inputs: "sha256:..."
  code:   "sha256:..."
\end{lstlisting}

\paragraph{HDF5 groups (canonical order).}
We store comparison data under \texttt{/phi} and tower metadata under \texttt{/tower}.
A minimal layout is:

\medskip
\begin{center}
\begin{tabular}{ll}
\toprule
Group/Dataset & Contents \\
\midrule
\texttt{/phi/idx/i} & integer degrees \(i\) \\
\texttt{/phi/idx/tau} & real thresholds \(\tau\) \\
\texttt{/phi/idx/iso} & boolean flags \\
\texttt{/phi/idx/mu} & nonnegative integers \(\mu_{i,\tau}\) \\
\texttt{/phi/idx/nu} & nonnegative integers \(\nu_{i,\tau}\) \\
\texttt{/phi/idx/flags} & bitmask for (S1,S2,S3) \\
\texttt{/phi/idx/iso\_tail/passed} & boolean \\
\texttt{/tower/edges/src} & integer source indices \\
\texttt{/tower/edges/dst} & integer target indices \\
\texttt{/tower/edges/kind} & categorical: inclusion/deletion/epsilon \\
\bottomrule
\end{tabular}
\end{center}


\subsection*{D.11. Additional formalities: tau-naturality and bandwise certification}

We record two further properties that are often used implicitly.

\begin{proposition}[Piecewise constancy off critical thresholds]\label{prop:D-piecewise}
Fix \(i\) and a tower.
There exists a finite set \(S\subset(0,\infty)\) consisting of bar lengths and their finite sums and differences such that \(\mu_{i,\tau}\) and \(\nu_{i,\tau}\) are locally constant on each connected component of \((0,\infty)\setminus S\).
\end{proposition}

\begin{proof}[Proof sketch]
Within the constructible regime, changes in \(\Ker/\coker\) after \(\T_\tau\) can occur only when \(\tau\) crosses endpoints of bars that interact with colim/cocone structure; this yields a finite critical set after fixing a finite window of degrees and using boundedness.
\end{proof}

\begin{corollary}[Bandwise certification]
If \(\phi_{i,\tau_0}\) is an isomorphism at some \(\tau_0\) lying in a component of \((0,\infty)\setminus S\), then it is an isomorphism on the whole component.
\end{corollary}

\subsection*{D.12. Completion note and cross-module conventions}

\begin{remark}[No further supplementation required]
This appendix now provides: (i) the definition and calculus of the tower obstruction indices \((\mu,\nu)\) (generic fiber dimensions after truncation), (ii) naturality and functoriality of the comparison map \(\phi\), including subadditivity under composition and additivity under direct sums, (iii) illustrative toy towers (pure kernel/cokernel/mixed) and a counterexample showing that \(\sum d_{\mathrm{int}}<\infty\) does not force \((\muc,\nuc)=(0,0)\), (iv) sufficient conditions (S1)–(S3) guaranteeing \((\muc,\nuc)=(0,0)\), (v) a \emph{Restart/Summability} framework to paste windowed certificates into global ones, (vi) a robust \(\tau\)-sweep procedure and \emph{stability bands} to certify \((\muc,\nuc)=(0,0)\) on contiguous \(\tau\)-ranges, and (vii) implementation-grade audit schemas (YAML/HDF5), API stubs (Lean/Coq), and test items.
All statements are confined to the v16.0 guard-rails (constructible \(1\)D persistence over a field; persistence-layer equalities after truncation; f.q.i.\ on filtered complexes), and no further supplementation is required for operational use in the proof framework.
\end{remark}

\medskip
\noindent\textbf{Cross-module conventions.}
Ext-tests are always taken against \(k[0]\) (Appendix~C): \(\Ext^1(\Rfun(C_\tau F),k)=0\).
When windowed energy summaries are referenced elsewhere, the exponent is uniformly \(\alpha>0\) (default \(\alpha=1\)).
Update monotonicity follows the global rule: \emph{deletion-type} updates are non-increasing for windowed energies and spectral tails after truncation, whereas \emph{inclusion-type} updates are stable (non-expansive); see Appendix~E.
Type labels follow the global convention \emph{Type~I–II / Type~III / Type~IV}.



% =========================
\section*{Appendix E. Spectral Indicators: Monotonicity, Stability, Counterexamples [Proof/Spec]}
% =========================
\addcontentsline{toc}{section}{Appendix E. Spectral Indicators: Monotonicity, Stability, Counterexamples}
\refstepcounter{section} % stabilize cross-references with an unnumbered section

% Listings (for YAML)
\definecolor{lightgray}{gray}{0.95}
\lstset{
  basicstyle=\ttfamily\small,
  backgroundcolor=\color{lightgray},
  frame=single,
  columns=fullflexible,
  keepspaces=true,
  showstringspaces=false,
  upquote=true,
  breaklines=true
}

For \(\tau>0\), define the \emph{clipped spectrum}
\(\mathrm{clip}_\tau(H):=(\min\{\lambda_j(H),\tau\})_{j=1}^n\), the \emph{clipped sum}
\[
S^{\le \tau}(H):=\sum_{j=1}^n \min\{\lambda_j(H),\tau\},
\]
and the \emph{sub-threshold deficit}
\[
D^{<\tau}(H)\ :=\ \sum_{j=1}^n (\tau-\lambda_j(H))_+,\qquad x_+:=\max\{x,0\}.
\]
We use the operator norm \(\|\cdot\|_{\mathrm{op}}\) and Frobenius norm \(\|\cdot\|_{\mathrm{fro}}\).
All references to filtered colimits follow the scope rule in Appendix~A, Remark~(A.6).
Cross-module conventions (used globally): Ext-tests are taken against \(k[0]\) (Appendix~C), i.e.\ \(\Ext^1(\mathcal{R}(C_\tau F),k)=0\); energy exponents are uniform \(\alpha>0\) (default \(\alpha=1\)); type labels use \emph{Type~I--II / Type~III / Type~IV}.
When we refer to persistence/filtered complexes, equalities are at the persistence layer (strict in \(\Pers^{\mathrm{ft}}_k\)); filtered–complex claims are \emph{up to f.q.i.}

\medskip
\noindent\textbf{Deletion vs.\ inclusion.}
When \(H\) arises by restricting admissible degrees of freedom, imposing Dirichlet constraints, eliminating internal dofs by shorting (Schur complement), or taking principal submatrices, we call conclusions \emph{deletion-type}.
When \(H\) is obtained by adding degrees of freedom, couplings, or enlarging a domain, we call them \emph{inclusion-type}.
Deletion-type updates admit one-sided monotonicity; inclusion-type updates admit only stability (non-expansive), unless additional order hypotheses are imposed.

\subsection*{E.0. Scope and window policy}

All spectral audits in this appendix are \emph{windowed} and performed \emph{after} collapse on the B-side single layer.
Concretely, comparisons follow the mandatory order:
\[
\boxed{\ \text{for each } t\ \Longrightarrow\ \text{apply } \mathbf{P}_i \ \Longrightarrow\ \text{apply } \mathbf{T}_\tau \ \Longrightarrow\ \text{compare in }\Pers^{\mathrm{ft}}_k\ }.
\]
When a spectral window \([a,b]\) with bin width \(\beta>0\) is used, bins are \emph{half-open, right-attribution}
\(I_r=[a+r\beta,a+(r+1)\beta)\), and eigenvalues at a right boundary are counted in the next bin.
Underflows/overflows are recorded.
This policy ensures reproducibility and compatibility with Overlap Gate and B–Gate\(^{+}\) (Appendix~G; Chapter~1).

\subsection*{E.1. Deletion-type monotonicity (principal/Dirichlet, Schur complement, Loewner)}

\begin{proposition}[Principal/Dirichlet restriction: interlacing and counting]\label{E:prop:dirichlet}
Let \(A\in\RR^{n\times n}\) be Hermitian and \(B\) a principal \((n-1)\times(n-1)\) submatrix (obtained, e.g., by pinning a coordinate—Dirichlet restriction).
Then Cauchy interlacing holds:
\[
\lambda_1(A)\ \le\ \lambda_1(B)\ \le\ \lambda_2(A)\ \le\ \cdots\ \le\ \lambda_{n-1}(B)\ \le\ \lambda_n(A).
\]
In particular, for every \(\theta\in\RR\),
\(\nN_\theta(B)\ \le\ N_\theta(A).\n\)
\end{proposition}

\begin{proposition}[Schur complement (shorting) monotonicity]\label{E:prop:schur}
Partition \(M=\begin{psmallmatrix}A & B\\ B^\top & C\end{psmallmatrix}\succeq 0\) with \(C\succ 0\) and form the Schur complement \(S:=A-BC^{-1}B^\top\).
Then \(S\preceq A\).
Consequently, for all \(j\) and all \(\theta\ge 0\),
\(\n\lambda_j(S)\ \le\ \lambda_j(A),\quad N_\theta(S)\ \le\ N_\theta(A).\n\)
\end{proposition}

\begin{proposition}[Loewner-order monotonicity]\label{E:prop:loewner}
If \(0\preceq A\preceq B\) (Loewner order), then for each \(j\),
\(\n\lambda_j(A)\le \lambda_j(B)\n\)
and, for every \(\theta\ge 0\), \(N_\theta(A)\le N_\theta(B)\).
\end{proposition}

\begin{remark}[Heat traces and spectral tails]
For PSD matrices and \(t>0\), the heat trace \(\mathrm{HT}(t;H)=\sum_j e^{-t\lambda_j(H)}\) satisfies:
if \(A'\preceq A\) (contraction), then \(\mathrm{HT}(t;A')\ge \mathrm{HT}(t;A)\); if \(A'\succeq A\) (hardening), then \(\mathrm{HT}(t;A')\le \mathrm{HT}(t;A)\).
Likewise, for spectral tails \(\mathrm{ST}_\beta(H)=\sum_{j\ge 1}\lambda_j(H)^{-\beta}\) with \(\beta>0\), one has \(\mathrm{ST}_\beta(A')\ge \mathrm{ST}_\beta(A)\) under \(A'\preceq A\) and the reverse inequality under \(A'\succeq A\), provided all \(\lambda_j>0\).
In practice, tails are computed on \(L(C_\tau F)\) with zero modes removed or handled by pseudoinverses; see Appendix~G.
\end{remark}

\begin{corollary}[Conservative averaging]\label{E:cor:avg}
If \(A_1,\dots,A_m\succeq 0\) satisfy \(A_\ell\preceq A\) for all \(\ell\), then for any convex combination \(\bar A:=\sum_\ell w_\ell A_\ell\) with \(w_\ell\ge 0\), \(\sum_\ell w_\ell=1\),
\(\n\bar A\ \preceq\ A\n\).
Therefore \(\lambda_j(\bar A)\le \lambda_j(A)\), \(N_\theta(\bar A)\le N_\theta(A)\) for \(\theta\ge 0\), and the heat trace/tail inequalities above apply.
\end{corollary}

\begin{remark}[Orientation for deletions]
Two Loewner orientations occur in practice.
\emph{Contractions} (e.g.\ Schur complements, Kron reduction) produce \(A'\preceq A\); \emph{hardening} operations (e.g.\ some PDE Dirichlet comparisons across different media) may yield \(A'\succeq A\).
We state deletion-type monotonicities in both orientations.
\end{remark}

\subsection*{E.2. Inclusion-type counterexamples}

Deletion-type monotonicity does \emph{not} extend naively to inclusion-type operations without additional order hypotheses.

\begin{example}[Neumann/domain inclusion reverses direction]\label{E:ex:neumann}
For the Neumann Laplacian on an interval, enlarging the domain decreases the nonzero eigenvalues:
on \([0,L]\), the first nonzero Neumann eigenvalue is \((\pi/L)^2\), so passing \(L:1\to 2\) reduces it from \(\pi^2\) to \((\pi/2)^2\).
Thus any “inclusion \(\Rightarrow\) increase” heuristic fails under Neumann-type constraints.
\end{example}

\begin{example}[Indefinite coupling can move eigenvalues both ways]\label{E:ex:indef}
Let \(A=I_2=\mathrm{diag}(1,1)\) and
\(B=\begin{psmallmatrix}1 & M\\ M & 1\end{psmallmatrix}\) with \(M>1\).
Then \(B\) has eigenvalues \(1-M\) and \(1+M\), so for \(\theta=0\),
\(N_\theta(B)=1<N_\theta(A)=2\), while the top eigenvalue \(\lambda_2\) increases.
Without a Loewner relation (\(B-A\) indefinite), no monotone law survives.
\end{example}

\begin{example}[Principal extension lacks a fixed direction]\label{E:ex:principal-extend}
Let \(B=[0]\) (eigenvalue \(0\)) and
\(A=\begin{psmallmatrix}0 & t\\ t & 0\end{psmallmatrix}\) with \(t\neq 0\).
Going from \(B\) to \(A\) (adding one dof and a coupling) produces eigenvalues \(-|t|\) and \(|t|\):
the maximum increases to \(|t|\), but the minimum decreases to \(-|t|\).
Hence no uniform increase/decrease holds under inclusion.
\end{example}

These examples justify restricting monotone claims to the deletion/Loewner settings formalized in §E.1.

\subsection*{E.3. Continuity, stability, and truncated functionals}

Write \(N_{\theta\pm 0}(A)\) for the left/right limits at \(\theta\) (no jump unless \(\theta\) is an eigenvalue).

\begin{proposition}[Weyl and Hoffman–Wielandt]\label{E:prop:weyl}
For Hermitian \(A,B\in\RR^{n\times n}\),
\[
\max_{1\le j\le n}\,|\lambda_j(A)-\lambda_j(B)|\ \le\ \|A-B\|_{\mathrm{op}},\n\qquad\n\Big(\sum_{j=1}^n |\lambda_j(A)-\lambda_j(B)|^2\Big)^{1/2}\ \le\ \|A-B\|_{\mathrm{fro}}.
\]
Hence \(A\mapsto(\lambda_1(A),\dots,\lambda_n(A))\) is \(1\)-Lipschitz from \((\|\cdot\|_{\mathrm{op}})\) into \((\RR^n,\|\cdot\|_\infty)\).
\end{proposition}

\begin{corollary}[Lipschitz stability of clipped spectra]\label{E:cor:clip}
For any \(\tau>0\) and Hermitian \(A,B\),
\[
\sum_{j=1}^n\Big|\min\{\lambda_j(A),\tau\}-\min\{\lambda_j(B),\tau\}\Big|\n\ \le\ \sum_{j=1}^n|\lambda_j(A)-\lambda_j(B)|\n\ \le\ \sqrt{n}\,\|A-B\|_{\mathrm{fro}}\n\ \le\ n\,\|A-B\|_{\mathrm{op}}.
\]
Consequently, \(S^{\le \tau}\) is \(\sqrt{n}\)–Lipschitz in \(\|\cdot\|_{\mathrm{fro}}\) and \(n\)–Lipschitz in \(\|\cdot\|_{\mathrm{op}}\).
\end{corollary}

\begin{proposition}[Semicontinuity of counting indicators]\label{E:prop:count-semi}
If \(A_m\to A\) in operator norm and \(\theta\) is not an eigenvalue of \(A\), then \(N_\theta(A_m)=N_\theta(A)\) for all large \(m\) (local constancy).
In general,
\[
\limsup_{m\to\infty} N_\theta(A_m)\ \le\ N_{\theta-0}(A),\n\qquad\n\liminf_{m\to\infty} N_\theta(A_m)\ \ge\ N_{\theta+0}(A).
\]
\end{proposition}

\begin{proposition}[Truncated functionals: monotonicity and stability]\label{E:prop:trunc-func}
Fix \(\tau>0\).
For an \(n\times n\) positive semidefinite (PSD) matrix \(A\), set
\[
S^{\le\tau}(A):=\sum_{j=1}^n \min\{\lambda_j(A),\tau\},\n\qquad\nD^{<\tau}(A):=\sum_{j=1}^n (\tau-\lambda_j(A))_{+},
\]
and \(N_\theta(A):=\#\{j:\lambda_j(A)\ge \theta\}\) for \(\theta\ge 0\).
Then:
\begin{enumerate}[leftmargin=1.15em,label=(\arabic*)]
\item \emph{Deletion; Loewner contraction \(A'\preceq A\).}
For all \(j\), \(\lambda_j(A')\le \lambda_j(A)\), hence \(N_\theta(A')\le N_\theta(A)\) for every \(\theta\ge 0\), and
\[
S^{\le\tau}(A')\le S^{\le\tau}(A),\qquad D^{<\tau}(A')\ge D^{<\tau}(A).
\]
\item \emph{Deletion; Loewner hardening \(A'\succeq A\).}
All inequalities in \emph{(1)} reverse:
\[
\lambda_j(A')\ge \lambda_j(A),\quad N_\theta(A')\ge N_\theta(A)\ \ (\theta\ge 0),\quad\nS^{\le\tau}(A')\ge S^{\le\tau}(A),\quad D^{<\tau}(A')\le D^{<\tau}(A).
\]
\item \emph{Lipschitz stability.} For any Hermitian \(A,B\),
\[
\bigl|D^{<\tau}(A)-D^{<\tau}(B)\bigr|\n\ \le \sum_{j=1}^n \bigl|\lambda_j(A)-\lambda_j(B)\bigr|\n\ \le \sqrt{n}\,\|A-B\|_{\mathrm{fro}}\n\ \le n\,\|A-B\|_{\mathrm{op}}.
\]
\end{enumerate}
\end{proposition}

\begin{proof}
Items (1) and (2) follow from the scalar monotonicity of \(x\mapsto \mathbf{1}_{[\theta,\infty)}(x)\), \(x\mapsto \min\{x,\tau\}\), \(x\mapsto (\tau-x)_+\), together with Loewner/Weyl monotonicity for eigenvalues.
Item (3) follows from Proposition~\ref{E:prop:weyl} and the \(1\)-Lipschitz property of \(x\mapsto(\tau-x)_+\).
\end{proof}

\begin{remark}[Heat traces and spectral tails: stability]
Let \(\mathrm{HT}(t;H)=\sum_j e^{-t\lambda_j(H)}\), \(\mathrm{ST}_\beta(H)=\sum_j \lambda_j(H)^{-\beta}\) with zero modes removed.
If \(\|A-B\|_{\mathrm{op}}\le\varepsilon\), then
\(\n|\mathrm{HT}(t;A)-\mathrm{HT}(t;B)|\le \sum_j |e^{-t\lambda_j(A)}-e^{-t\lambda_j(B)}|\n\le t\,e^{-t\lambda_{\min}^+}\sum_j |\lambda_j(A)-\lambda_j(B)|,\n\)
yielding a Lipschitz-type bound in terms of \(\|\cdot\|_{\mathrm{fro}}\) (similar for \(\|\cdot\|_{\mathrm{op}}\) with \(n\) factor).
For \(\mathrm{ST}_\beta\), on windows where \(\lambda_j\ge \lambda_{\min}^+>0\), the map \(x\mapsto x^{-\beta}\) is Lipschitz on \([\lambda_{\min}^+,\infty)\), yielding analogous bounds.
In practice, we evaluate tails/heat traces on \(L(C_\tau F)\) under a fixed normalization policy (Appendix~G).
\end{remark}

\subsection*{E.4. Auxiliary spectral bars (aux-bars): definition, stability, and policy [Spec]}

We formalize \emph{auxiliary spectral bars} as diagnostics alongside persistence. They never replace B–Gate\(^{+}\) and are used only as auxiliary evidence.

\paragraph{E.4.1. Binning and endpoint convention.}
Fix a spectral window \([a,b]\) and a bin width \(\beta>0\).
Let \(R:=\lfloor (b-a)/\beta\rfloor\).
Define half-open, right-attribution bins
\(\nI_r=[a+r\beta,a+(r+1)\beta)\n\)
for \(r=0,1,\dots,R-1\).
An eigenvalue at the bin’s right boundary is counted in the next bin.
Record underflow \(U(H):=\#\{j:\lambda_j(H)<a\}\) and overflow \(O(H):=\#\{j:\lambda_j(H)\ge b\}\).
For a Hermitian \(H\), define the bin occupancy
\(\nE_r(H):=\#\{j:\lambda_j(H)\in I_r\}\n\)
and the cumulative (upper-tail) profile
\(\nC_r(H):=\sum_{s=r}^{R-1}E_s(H)=N_{a+r\beta}(H)-O(H)\).
 
\paragraph{E.4.2. Aux-bars across an index (time/tower).}
Let \((H_j)_{j\in J}\) be a sequence (time or tower).
For fixed \(r\), the set \(\{j:E_r(H_j)>0\}\) decomposes into maximal consecutive runs \(J_{r,\ell}\).
Each run \(J_{r,\ell}\) defines an aux-bar \((r,J_{r,\ell})\) with lifetime \(|J_{r,\ell}|\) (or a rescaled duration).
We log \(\mathrm{aux\text{-}count}=\sum_{r,\ell}1\), \(\mathrm{aux\text{-}mass}=\sum_r E_r(H_j)\) (per index \(j\)), and \(\mathrm{active\ bins}=\#\{r:E_r(H_j)>0\}\).

\paragraph{E.4.3. Monotonicity/stability.}
\begin{proposition}[Cumulative-profile monotonicity under Loewner]\label{E:prop:aux-cum-mono}
If \(A'\preceq A\) (PSD contraction) or \(A'\) is a principal/Dirichlet restriction of \(A\), then for every \(r\),
\(\nC_r(A')\le C_r(A).\n\)
\end{proposition}

\begin{proposition}[Cumulative-profile stability]\label{E:prop:aux-cum-stab}
If \(\|A-B\|_{\mathrm{op}}\le \varepsilon\) and \(q:=\lceil \varepsilon/\beta\rceil\), then for all \(r\),
\(\nC_{r+q}(B)\le C_r(A)\le C_{\max\{0,r-q\}}(B).\n\)
In particular, if \(\varepsilon<\beta\), the cumulative profile can shift by at most one bin.
\end{proposition}

\begin{remark}[Policy]
- Deletion-type steps: enforce monotonicity on the \emph{cumulative} profile \(C_r\). Per-bin occupancies and lifetimes are diagnostics only.

- \(\varepsilon\)-continuations: with \(\|A_{j+1}-A_j\|_{\mathrm{op}}\le \varepsilon\), declare stability up to \(\pm q=\lceil \varepsilon/\beta\rceil\) bin shifts; record \texttt{eps\_cont\_shift\_bins} in the manifest.

- Inclusion-type steps: claim no monotonicity; only stability bounds are used.

- Under/overflow must be logged. Optional conservative rules: \(O=0\), \(C_{R-1}=0\) may be enforced as policy (not a proof obligation).
\end{remark}

\paragraph{E.4.4. Reproducibility fields.}
The run manifest \texttt{run.yaml} should include (Appendix~G):
\begin{itemize}[leftmargin=1.25em]
  \item \texttt{spectral.range} \([a,b]\), \texttt{bin\_width} \(\beta\), \texttt{bins} \(R\), endpoint policy \texttt{half-open/right-attribution};
  \item \texttt{underflow}/\texttt{overflow} per index \(j\);
  \item \texttt{cum\_profile}: the sequence \(C_r(H_j)\) per \(j\);
  \item \texttt{aux\_bars} (optional): list of runs \((r,J_{r,\ell})\) with lifetimes;
  \item \texttt{eps\_cont\_bound}: \(\varepsilon\) and derived \texttt{eps\_cont\_shift\_bins} \(= \lceil \varepsilon/\beta\rceil\);
  \item \texttt{spectral\_policy.order}: \texttt{"ascending"}; \texttt{spectral\_policy.norm}: \texttt{"op"} or \texttt{"fro"}; bounds \texttt{lambda\_min}, \texttt{lambda\_max}, optional \texttt{lip\_tol}.
\end{itemize}

\subsection*{E.5. Implementation and reproducibility: JSON/HDF5 schemas}

\subsection*{JSON layout (mandatory fields).}
\begin{lstlisting}[language=,caption={Minimal spec.json layout (one run, multiple operators)}]
{
  "meta": {
    "schema_version": "2025-03-15",
    "eigen_units": "dimensionless",
    "order": "ascending",
    "sorted": true,
    "norm": "op",               // or "fro"
    "Ntheta_convention": { "left": "N_{θ-0}", "right": "N_{θ+0}" },
    "window": { "range": [0.0, 2.0], "semantics": "closed" },
    "clip_tau": 1.0,
    "tol_eig": 1e-8,
    "aux_policy": { "bin": 0.02, "right_attribution": true },
    "coverage_check": { "thetas_in_window": true },
    "links": { "run_id": "...", "run_yaml_hash": "sha256:..." }
  },
  "operators": [
    {
      "id": "sha256:...A",
      "kind": "laplacian_dirichlet",
      "n": 500,
      "spectrum": { "eigs": [0.10, 0.12, 0.45, ...] },  // ascending
      "clip":     { "tau": 1.0, "sum": 37.219, "deficit": 12.004 },
      "underflow": 0, "overflow": 0
    },
    {
      "id": "sha256:...B",
      "kind": "principal_submatrix",
      "parent": "sha256:...A",
      "N_theta": [
        { "theta": 0.20, "left": 17, "right": 16 },
        { "theta": 0.50, "left": 10, "right": 10 }
      ],
      "cum_profile": [ 13, 11, 8, 2, 0 ],
      "underflow": 2, "overflow": 0,
      "monotonicity": { "type": "deletion", "passed": true }
    }
  ],
  "hash": "sha256:...spec"
}
\end{lstlisting}

\paragraph{HDF5 layout (canonical).}
\begin{itemize}[leftmargin=1.25em]
  \item Datasets:
  \texttt{/spec/ops/\{id\}/eig} (float64 ascending),
  \texttt{/spec/ops/\{id\}/clip/sum} (float64),
  \texttt{/spec/ops/\{id\}/clip/deficit} (float64),
  \texttt{/spec/ops/\{id\}/Ntheta/theta,left,right} (parallel arrays),
  \texttt{/spec/ops/\{id\}/cum\_profile} (int32),
  \texttt{/spec/ops/\{id\}/underflow} (int32),
  \texttt{/spec/ops/\{id\}/overflow} (int32).
  \item Attributes: \texttt{order="ascending"}, \texttt{norm="op"|"fro"}, \texttt{eigen\_units}, \texttt{tol\_eig}, \texttt{schema\_version}, bin policy, and canonical HDF5 flags (\texttt{track\_times=false}, UTF-8 fixed strings, fixed chunking); see Appendix~G.
\end{itemize}

\subsection*{E.6. Tests and operational checklist}

\paragraph{Core tests.}
\begin{enumerate}[leftmargin=1.25em]
  \item \textbf{T2 (Deletion-type monotonicity).}
  For principal/Dirichlet or Schur complements, verify \(N_\theta\), \(S^{\le\tau}\), \(D^{<\tau}\), heat trace, and tails satisfy the monotone direction consistent with the Loewner orientation. Log \texttt{monotonicity.passed=true}.
  \item \textbf{T1 (\(\varepsilon\)-continuation stability).}
  Given \(\|A_{j+1}-A_j\|_{\mathrm{op}}\le \varepsilon\), validate the bin-shift bounds in Proposition~\ref{E:prop:aux-cum-stab} and Lipschitz bounds for \(S^{\le\tau}\), \(D^{<\tau}\).
  \item \textbf{T9 (Coverage).}
  Confirm that all \(\theta\)-queries, bin windows, and eigenvalues fall within declared windows; if not, log underflow/overflow and set \texttt{coverage\_check.*=false} with a justification.
\end{enumerate}

\paragraph{Operational checklist (per window).}
\begin{itemize}[leftmargin=1.25em]
  \item Fix spectral policy: \texttt{order="ascending"}, \texttt{norm} selection, bin width \(\beta\), spectral window \([a,b]\).
  \item Compute spectra on \(L(C_\tau F)\) and clip at the same \(\tau\) used by persistence.
  \item Log underflow/overflow, cumulative profiles \(C_r\), optional aux-bars with lifetimes (diagnostic).
  \item For deletion-type steps, assert \(C_r\) monotonicity; for \(\varepsilon\)-continuations, assert bin-shift stability with \(\lceil \varepsilon/\beta\rceil\).
  \item Never use aux-bars as sole gate criteria; they are \emph{auxiliary} to B–Gate\(^{+}\).
\end{itemize}

\subsection*{E.7. Completion note}

\begin{remark}[No further supplementation required]
This appendix provides a complete, IMRN/AiM-ready treatment of spectral indicators consistent with the v16.0 guard-rails: (i) deletion-type monotonicities (principal/Dirichlet, Schur, Loewner) for \(N_\theta\), \(S^{\le\tau}\), \(D^{<\tau}\), heat traces, and tails; (ii) inclusion-type counterexamples; (iii) Lipschitz stability via Weyl/Hoffman–Wielandt and induced bounds for clipped sums/deficits, heat traces, and tails; (iv) a windowed, half-open binning policy with cumulative-profile monotonicity and stability under \(\varepsilon\)-continuations; (v) reproducibility and canonical schemas (JSON/HDF5) that mandate eigenvalue ordering and norm declaration; and (vi) a minimal test suite (T1/T2/T9).
All claims are made after collapse on the B-side single layer, per-window, and integrate with δ-ledger accounting, Overlap Gate, and B–Gate\(^{+}\) elsewhere in the manuscript. No further supplementation is required for operational deployment or audit.
\end{remark}



% =========================
\section*{Appendix F. Formalization Sketch (Lean/Coq) [Spec] (reinforced)}
% =========================
\addcontentsline{toc}{section}{Appendix F. Formalization Sketch (Lean/Coq)}
\refstepcounter{section} % stabilize cross-references with an unnumbered section

This appendix provides a fully integrated, implementation-oriented \emph{Spec} for mechanizing the core claims of
Appendices~A–E in Lean/Coq with a module decomposition tailored for a minimal, portable ``mini-library.''
The categorical spine consists of the \emph{Serre localization} and the \emph{reflector} \(\mathbf{T}_\tau\) (exact, idempotent),
its \emph{$1$-Lipschitz} property on barcodes, tower diagnostics \((\mu,\nu)\) via a comparison map \(\phi_{i,\tau}\),
and the edge identification supporting the one-way bridge \(\mathrm{PH}_1\Rightarrow\Ext^1\).
We work in the \emph{constructible} (p.f.d.) range and adhere to the \emph{filtered colimit scope rule}
(Appendix~A, Remark~\ref{A:rk:filtered-colimits}).
Cross-module conventions: \(\Ext\)-tests are always against \(k[0]\) (Appendix~C), i.e.\ \(\Ext^1(\mathcal{R}(C_\tau F),k)=0\);
the energy exponent is globally \(\alpha>0\) (default \(\alpha=1\));
type labels use \emph{Type I--II / Type III / Type IV} for tower diagnostics.
Spectral monotonicity is invoked only for \emph{deletion-type} operations; inclusion-type operations are used solely with stability bounds (Appendix~E).

\paragraph{Refereeing style (IMRN/AiM).}
Statements are kept modular with explicit hypotheses and reusable APIs.
All constructions remain within abelian categories, exact localizations, and derived categories with bounded \(t\)-structures.
Proof obligations used in the code stubs are isolated and cited to Appendices~A–E; replacing \texttt{admit}/\texttt{Axiom} by library lemmas yields a fully checked artifact.

% -------------------------
\subsection*{F.0. Reading guide and module map}
% -------------------------

We split the development into five modules:

\begin{itemize}
  \item \textbf{AK.Core} (F.1–F.3, F.5–F.6): \(\Pers^{\mathrm{ft}}_k\), the Serre subcategory \(\mathsf{E}_\tau\), the reflector \(\mathbf{T}_\tau\) (exact, idempotent), interleaving-shift calculus and \(1\)-Lipschitz, the collapse on filtered complexes \(C_\tau\), and invariance under filtered quasi-isomorphisms.
  \item \textbf{AK.LocalEquiv} (F.7): Window-local equivalences \(\mathrm{PH}\leftrightarrow \Ext\) under amplitude \(\le 1\), saturation/no-accumulation, and the tail-isomorphism for \(\phi_{i,\tau}\).
  \item \textbf{AK.Tower} (F.8): The comparison map \(\phi_{i,\tau}\), diagnostics \((\mu,\nu)\), functoriality, stability bands, direct sums, compositions, and cofinal invariance.
  \item \textbf{AK.Gluing} (F.9): The Overlap Gate (collapse-compatibility, A/B checks, Čech–\(\Ext^1\), stability bands) and MECE windowing glue to a global verdict.
  \item \textbf{AK.Spectral} (F.10): The spectral operators \(S^{\le\tau}\), \(D^{<\tau}\), and \(C_r\), monotonicity and Lipschitz-type bounds compatible with \(\mathbf{T}_\tau\).
\end{itemize}

Lean and Coq stubs are provided per module. Test fixtures include T7 (saturation gate), T10 (A/B), T13 (\(\delta\)-budget).

% -------------------------
\subsection*{F.1. Environment and objects [Spec] (AK.Core)}
% -------------------------

Fix a field \(k\).
Let \(\mathsf{Vect}_k\) be the abelian category of finite-dimensional \(k\)-vector spaces and
\([\mathbb{R},\mathsf{Vect}_k]\) the functor category (index \((\mathbb{R},\le)\)).
Let \(\Pers^{\mathrm{ft}}_k\subset[\mathbb{R},\mathsf{Vect}_k]\) be the full subcategory of \emph{constructible}
persistence modules (barcodes locally finite on bounded windows).

Let \(\mathsf{FiltCh}(k)\) be filtered chain complexes of finite-dimensional \(k\)-spaces, bounded in homological degree,
with filtration-preserving maps. For \(i\in\mathbb{Z}\) write \(\mathbf{P}_i:\mathsf{FiltCh}(k)\to\Pers^{\mathrm{ft}}_k\)
for the degree–\(i\) persistence functor. The \(\tau\)-\emph{truncation} (collapse) functor
\(\mathbf{T}_\tau:\Pers^{\mathrm{ft}}_k\to\Pers^{\mathrm{ft}}_k\) is recalled from Appendix~A.

\begin{remark}[Generic fiber dimension and stabilization]\label{F:rk:generic-fiber}
We adopt Appendix~D, Remark~\ref{rem:D-generic-dim}. For \(M\in\Pers^{\mathrm{ft}}_k\), the \emph{generic fiber dimension}
is the multiplicity of the infinite interval \(I[0,\infty)\) in the barcode of \(M\); equivalently,
\[
\mathrm{gdim}(M)\ =\ \lim_{t\to +\infty}\dim_k M(t),
\]
which stabilizes in the constructible range.
After applying \(\mathbf{T}_\tau\), kernels and cokernels again lie in \(\Pers^{\mathrm{ft}}_k\), and \(\mathrm{gdim}\)
is computed there.
\end{remark}

\begin{specification}[Stabilization lemma for constructible modules]\label{F:spec:stabilize}
If \(M\in\Pers^{\mathrm{ft}}_k\), then there exist \(T_0\in\mathbb{R}\) and \(c\in\mathbb{N}\) such that
\(\dim_k M(t)=c=\mathrm{gdim}(M)\) for all \(t\ge T_0\).
\emph{Use:} define \(\mathrm{gdim}(M)\) by this stabilized value \(c\) and prove iso-invariance of \(\mathrm{gdim}\).
\end{specification}

% -------------------------
\subsection*{F.2. Serre subcategory and localization [Spec] (AK.Core)}
% -------------------------

Let \(\mathsf{E}_\tau\subset \Pers^{\mathrm{ft}}_k\) be the Serre subcategory generated by intervals of length \(\le\tau\).
By Appendix~A, \(\mathsf{E}_\tau\) is hereditary Serre and the inclusion
\(\iota_\tau:\mathsf{E}_\tau^\perp\hookrightarrow \Pers^{\mathrm{ft}}_k\) admits an exact left adjoint
\(\mathbf{T}_\tau:\Pers^{\mathrm{ft}}_k\to \mathsf{E}_\tau^\perp\) (reflector), inducing an equivalence
\[
\Pers^{\mathrm{ft}}_k/\mathsf{E}_\tau\ \simeq\ \mathsf{E}_\tau^\perp.
\]
Basic laws (API):
\[
\mathbf{T}_\tau\circ \mathbf{T}_\tau \cong \mathbf{T}_\tau,\qquad \mathbf{T}_\tau \dashv \iota_\tau .
\]
Moreover, \(\mathbf{T}_\tau\) is exact and preserves finite (co)limits (Appendix~A).

% -------------------------
\subsection*{F.3. Interleavings, shifts, and \(1\)-Lipschitz [Spec] (AK.Core)}
% -------------------------

The interleaving pseudometric \(d_{\mathrm{int}}\) on \(\Pers^{\mathrm{ft}}_k\) is implemented via shift functors
\(\mathrm{Shift}_\varepsilon\) and \(\varepsilon\)-interleavings. Appendix~A (Prop.~\ref{A:prop:lipschitz}) yields natural
isomorphisms \(\mathrm{Shift}_\varepsilon\circ \mathbf{T}_\tau\simeq \mathbf{T}_\tau\circ \mathrm{Shift}_\varepsilon\)
that transport interleavings, hence
\[
d_{\mathrm{int}}(\mathbf{T}_\tau M,\mathbf{T}_\tau N)\ \le\ d_{\mathrm{int}}(M,N).
\]

% -------------------------
\subsection*{F.4. Filtered colimits, scope rule, and constructibility [Spec] (AK.Core)}
% -------------------------

\textbf{Scope rule.} All filtered (co)limit computations are performed \emph{objectwise} in
\([\mathbb{R},\mathsf{Vect}_k]\), where filtered colimits are exact; they are invoked only under
Appendix~A, Remark~\ref{A:rk:filtered-colimits}. Whenever the result might exit \(\Pers^{\mathrm{ft}}_k\),
we either (i) verify constructibility, or (ii) compute outside and \emph{return} via \(\mathbf{T}_\tau\)
or an explicit finite-type truncation. No claim is made outside this regime.

% -------------------------
\subsection*{F.5. Collapse on filtered complexes and f.q.i. [Spec] (AK.Core)}
% -------------------------

We use a collapse/threshold operation \(C_\tau\) at the level of filtered complexes:
\[
C_\tau:\ \mathsf{FiltCh}(k)\longrightarrow \mathsf{FiltCh}(k),
\]
compatible with \(\mathbf{P}_i\) by construction, and preserving filtered quasi-isomorphisms (f.q.i.) up to localization.
In practice, \(\mathbf{P}_i\circ C_\tau\) factors through \(\mathbf{T}_\tau\circ \mathbf{P}_i\) via a natural transformation that becomes an isomorphism after applying \(\mathbf{T}_\tau\) (Appendix~A).
All \(\Ext\)-tests are taken after \(\mathcal{R}(C_\tau F)\) with \(\mathcal{R}\) of amplitude \([-1,0]\) (Appendix~C).

% -------------------------
\subsection*{F.6. Lean~4 sketch (mathlib style) [Spec] (AK.Core)}
% -------------------------

\begin{verbatim}
-- F.6 AK.Core: categories, Serre reflector, interleavings, scope rule, collapse
namespace AK.Core
open scoped BigOperators Classical
noncomputable section

variable (k : Type*) [Field k]

/-- f.d. k-vector spaces. -/
abbrev Vect := FinVect k

/-- Schematic index for (ℝ, ≤). -/
structure RIdx := (α : Type) (str : Preorder α)
abbrev Diag := (RIdx → Vect k)

/-- Constructible persistence modules Pers^{ft}_k. -/
abbrev Pers := { M : Diag k // Constructible M }

/-- Serre subcategory E_τ and reflector T_τ : Pers → E_τ^⊥. -/
def shortBar (τ : ℝ≥0) (I : Interval) : Prop := I.length ≤ τ
def Eτ (τ : ℝ≥0) : SerreSubcategory (Pers k) := by admit
noncomputable def Tτ (τ : ℝ≥0) : Pers k ⥤ (Eτ k τ).orthogonal := by admit
noncomputable def iotaτ (τ) : (Eτ k τ).orthogonal ⥤ Pers k := by admit
theorem Tτ_exact (τ) : (Tτ k τ).IsExact := by admit
theorem Tτ_idem  (τ) : (Tτ k τ) ⋙ (Tτ k τ) ≅ (Tτ k τ) := by admit
theorem Tτ_adj   (τ) : (Tτ k τ) ⊣ (iotaτ k τ) := by admit
theorem localization_equiv (τ) :
  (Pers k) ⧸ (Eτ k τ) ≌ (Eτ k τ).orthogonal := by admit

/-- Interleaving metric and shifts. -/
class Interleaving (C : Type*) :=
  (dist : C → C → ℝ≥0∞) (isPseudoMetric : PseudoMetricSpace C)
def d_int := (Interleaving.dist : Pers k → Pers k → ℝ≥0∞)
noncomputable def Shift (ε : ℝ≥0) : Pers k ⥤ Pers k := by admit
axiom shift_zero : Shift k (0:ℝ≥0) ≅ �� (Pers k)
axiom shift_add  : ∀ ε δ, Shift k ε ⋙ Shift k δ ≅ Shift k (ε + δ)
axiom shift_comm (τ ε) : Shift k ε ⋙ (Tτ k τ) ≅ (Tτ k τ) ⋙ Shift k ε

/-- 1-Lipschitz property. -/
theorem Tτ_nonexpansive (τ) :
  ∀ M N : Pers k, d_int k ((Tτ k τ).obj M) ((Tτ k τ).obj N) ≤ d_int k M N := by admit

/-- Filtered colimits are exact in Vect_k; scope rule returns to constructible. -/
theorem filtered_colim_exact :
  ∀ {J} [IsFiltered J] (F : J ⥤ Vect), ExactFilteredColim F := by admit
axiom return_to_constructible :
  ∀ (D : SomeFilteredDiagram), Constructible (colim D)

/-- Filtered complexes and persistence. -/
abbrev FiltCh := FiltChCat k
def P_i (i : ℤ) : FiltCh k ⥤ Pers k := by admit

/-- Collapse on filtered complexes; f.q.i.-compatible. -/
noncomputable def Cτ (τ : ℝ≥0) : FiltCh k ⥤ FiltCh k := by admit
theorem Cτ_preserves_fqi (τ) : PreservesFQI (Cτ k τ) := by admit

/-- Comparison against T_τ ∘ P_i. -/
noncomputable def comp_collapse (i : ℤ) (τ : ℝ≥0) :
  (P_i k i) ⋙ (Tτ k τ) ⟶ (P_i k i) ⋙ (Forget ⟶ Pers k) ⋙ (Tτ k τ) := by admit
end AK.Core
\end{verbatim}

% -------------------------
\subsection*{F.7. Local equivalences within a window [Spec] (AK.LocalEquiv)}
% -------------------------

We formalize the window-local bridge \(\mathrm{PH}\leftrightarrow \Ext\) under amplitude \(\le 1\) and mild regularity.

Let a half-open window \(W=(u,u']\) be fixed; assume:
(i) the filtered complex \(F\) is concentrated in homological degrees \(\le 1\) in \(W\),
(ii) the collapse \(C_\tau\) is stable on \(W\) (\emph{saturation}),
(iii) no \(\tau\)-accumulation of critical values in \(W\) (Appendix~D),
(iv) tails of \(\phi_{i,\tau}\) in \(W\) are isomorphisms (Appendix~D, (S1)).

Then for the derived realization \(\mathcal{R}\) with amplitude \([-1,0]\) (Appendix~C) we have natural isomorphisms
\[
H^{-1}\bigl(\mathcal{R}(C_\tau F)\bigr)\ \simeq\ \varinjlim_{t\in W} H_1(F^t),\qquad\n\Ext^1\bigl(\mathcal{R}(C_\tau F),k\bigr)\ \simeq\ \Hom\!\bigl(H^{-1}(\mathcal{R}(C_\tau F)),k\bigr).
\]
Thus \(\mathrm{PH}_1(F|_W)=0\Rightarrow \Ext^1(\mathcal{R}(C_\tau F),k)=0\) window-locally.

\begin{remark}[Tail-isomorphism and stability bands]
Under (S1)–(S3) in Appendix~D the comparison maps \(\phi_{i,\tau}\) on \(W\) are isomorphisms; equivalently,
the diagnostic indices \((\mu_{i,\tau},\nu_{i,\tau})\) vanish on \(W\).
\end{remark}

\paragraph{Lean~4 stubs (AK.LocalEquiv).}
\begin{verbatim}
namespace AK.LocalEquiv
open AK.Core
noncomputable section
variable (k : Type*) [Field k]

/-- A half-open time window. -/
structure Window where u u' : ℝ := (len_pos : u' > u)

/-- Saturation and no-accumulation hypotheses on a window. -/
structure WinHyp (W : Window) : Prop :=
  (amplitude_le_one : True)          -- t-amplitude ≤ 1 in W
  (saturated_collapse : True)        -- C_τ stable on W
  (no_tau_accumulation : True)       -- no τ-accumulation of events
  (phi_tail_iso : True)              -- tail of φ_{i,τ} is iso on W

/-- Derived realization with amplitude [-1,0]. -/
noncomputable def Rfun : FiltCh k ⥤ Derived k := by admit
axiom Rfun_amp_le_one : Amplitude Rfun (-1) 0

/-- Window-local bridge: PH_1 ⇒ Ext^1 through H^{-1}. -/
theorem local_bridge
  (W : Window) (H : WinHyp k W) (F : FiltCh k) (τ : ℝ≥0) :
  (PH1_on W F = 0) → Ext1 (Rfun k).obj (Cτ k τ).obj F (of k).obj k = 0 := by admit
end AK.LocalEquiv
\end{verbatim}

% -------------------------
\subsection*{F.8. Towers, \(\phi_{i,\tau}\), and diagnostics \((\mu,\nu)\) [Spec] (AK.Tower)}
% -------------------------

We now define towers, the comparison map \(\phi_{i,\tau}\), and the invariants
\[
\mu_{i,\tau}(T)\ :=\ \mathrm{gdim}\,\ker\phi_{i,\tau}(T),\qquad\n\nu_{i,\tau}(T)\ :=\ \mathrm{gdim}\,\mathrm{coker}\,\phi_{i,\tau}(T).
\]
They are invariant under filtered quasi-isomorphisms of towers and under cofinal reindexings, and vanish under (S1)–(S3).

\paragraph{Lean~4 stubs (AK.Tower).}
\begin{verbatim}
namespace AK.Tower
open AK.Core
noncomputable section
variable (k : Type*) [Field k]

abbrev FiltCh := FiltChCat k
def P_i (i : ℤ) : FiltCh k ⥤ Pers k := by admit
noncomputable def Tτ (τ : ℝ≥0) := AK.Core.Tτ k τ

/-- ✅ Order fixed: persistence then truncation. -/
def TτP (i : ℤ) (τ : ℝ≥0) : FiltCh k ⥤ (Eτ k τ).orthogonal :=
  (P_i k i) ⋙ (Tτ k τ)

/-- Towers with a chosen cocone to the apex. -/
structure Tower :=
  (obj : ℕ → FiltCh k) (map : ∀ n, obj n ⟶ obj (n+1))
  (apex : FiltCh k)    (cocone : ∀ n, obj n ⟶ apex)

/-- Iterated edge n → m (n ≤ m). -/
def iter_mor (T : Tower k) : ∀ {n m}, n ≤ m → T.obj n ⟶ T.obj m
| n, n, _ => �� _
| n, (m+1), h =>
  have : n ≤ m := Nat.le_of_lt_succ h; iter_mor (T := T) this ≫ T.map m

/-- Cofinal reindexing by a strictly monotone r. -/
def reindex (T : Tower k) (r : ℕ → ℕ) (h : StrictMono r) : Tower k := by
  refine { obj := fun n => T.obj (r n), map := ?_, apex := T.apex, cocone := fun n => T.cocone (r n) }
  intro n
  have hlt : r n < r (n+1) := h (Nat.lt_succ_self n)
  exact iter_mor (T := T) (Nat.le_of_lt hlt)

/-- Morphisms of towers (levelwise maps). -/
structure TowerHom (T T' : Tower k) :=
  (α : ∀ n, T.obj n ⟶ T'.obj n)
  (square : ∀ n, T.map n ≫ α (n+1) = α n ≫ T'.map n)
  (cocone_comm : ∀ n, α n ≫ T'.cocone n = T.cocone n)

/-- Comparison map φ_{i,τ}. -/
def phi (i : ℤ) (τ : ℝ≥0) (T : Tower k) :
  colim (fun n ↦ (TτP k i τ).obj (T.obj n))
       ⟶ (TτP k i τ).obj T.apex := by admit

/-- Naturality of φ wrt tower morphisms. -/
theorem phi_natural :
  ∀ {i τ} {T T' : Tower k} (h : TowerHom k T T'),
    (colim.map h.α) ≫ phi k i τ T' = phi k i τ T := by admit

/-- gdim, μ, ν. -/
theorem gdim_stabilizes (M : Pers k) :
  ∃ (T0 : ℝ) (c : ℕ), ∀ t, t ≥ T0 → dim (M.val t) = c := by admit
noncomputable def gdim (M : Pers k) : ℕ := (show ℕ from (gdim_stabilizes k M).choose_spec.choose)
theorem gdim_iso_invariant {X Y : Pers k} (e : X ≅ Y) : gdim k X = gdim k Y := by admit
noncomputable def mu (i : ℤ) (τ : ℝ≥0) (T : Tower k) : ℕ := gdim k (kernel (phi k i τ T))
noncomputable def nu (i : ℤ) (τ : ℝ≥0) (T : Tower k) : ℕ := gdim k (cokernel (phi k i τ T))

/-- Invariance under filtered quasi-isos and cofinal reindexings. -/
theorem mu_nu_fqi_invariant :
  ∀ {i τ T T'}, (T ≅ T') → mu k i τ T = mu k i τ T' ∧ nu k i τ T = nu k i τ T' := by admit
theorem mu_nu_cofinal_invariant :
  ∀ {i τ T} (r : ℕ → ℕ) (h : StrictMono r),
    mu k i τ T = mu k i τ (reindex k T r h) ∧ nu k i τ T = nu k i τ (reindex k T r h) := by admit

/-- Sufficient conditions (Appendix D): S1/S2/S3 imply φ is iso. -/
axiom S1_commutes (i : ℤ) (τ : ℝ≥0) (T : Tower k) :
  (colim (fun n ↦ (TτP k i τ).obj (T.obj n))) ≅ (TτP k i τ).obj T.apex
theorem mu_nu_vanish_of_S1 (i τ T) : mu k i τ T = 0 ∧ nu k i τ T = 0 := by admit
end AK.Tower
\end{verbatim}

% -------------------------
\subsection*{F.9. Overlap Gate, MECE windows, and gluing [Spec] (AK.Gluing)}
% -------------------------

We formalize the operational glue from windows to a global verdict.

\paragraph{MECE windowing.}
A \emph{MECE} family \(\mathcal{W}=\{W_j\}\) partitions \([u_0,U)\) into pairwise disjoint half-open windows with
exact coverage. The coverage check asserts that the total event count equals the sum of per-window events.

\paragraph{Overlap Gate (B-Gate\(^+\)).}
For each window \(W\), we compute:
- \(\mathrm{PH}_1\)-vanishing status and, if eligible (Appendix~C), the \(\Ext^1\)-test;
- the diagnostic indices \((\mu_{i,\tau},\nu_{i,\tau})\);
- a safety margin \(\mathrm{gap}_W\) and a \(\delta\)-ledger with additive pipeline budget \(\mathrm{dsum}_W\).
The \(\mathrm{B\text{-}Gate}^+\) verdict is True if
\[
\mathrm{PH}_1=0,\ \mathrm{Ext}^1\text{ used and }0,\ \mu=\nu=0,\ \mathrm{gap}_W > \mathrm{dsum}_W.
\]

\paragraph{Overlap/Čech–\(\Ext^1\).}
On pairwise overlaps \(W_j\cap W_{j+1}\), we require a soft A/B commutativity test and a Čech-type
consistency check ensuring that window-local \(\Ext^1\)-vanishing patches across overlaps and implies global vanishing
in degree one (Appendix~C), under a standard exactness assumption for the overlap complex.

\paragraph{Restart and summability.}
We enforce the recursive inequality \(\mathrm{gap}_{k+1}\ge \kappa(\mathrm{gap}_k-\Sigma\delta_k)\) with \(0<\kappa\le 1\),
and the summability of budgets \(\sum_k\Sigma\delta_k<\infty\) per degree to ensure convergence and stability (Appendix~D).

\paragraph{Lean~4 stubs (AK.Gluing).}
\begin{verbatim}
namespace AK.Gluing
open AK.Core AK.Tower
noncomputable section
variable (k : Type*) [Field k]

/-- Windows and MECE windowing. -/
structure Window where u u' : ℝ := (len_pos : u' > u)
structure Windowing where
  u0 U : ℝ
  cover : List Window
  disjoint :
    Pairwise (fun W1 W2 => W1 ≠ W2 → (W1.u' ≤ W2.u ∨ W2.u' ≤ W1.u))
  exact_cover : (∑ w in cover, (w.u' - w.u)) = (U - u0)
  right_inclusion : True

def coverage_ok (Ev_global : ℕ) (Ev_win : List ℕ) : Prop :=
  Ev_global = (Ev_win.foldl (·+·) 0)

/-- δ-ledger and budgets. -/
structure Delta where alg disc meas : ℝ≥0
  deriving DecidableEq
def dsum (d : Delta) : ℝ≥0 := d.alg + d.disc + d.meas
def budget (ds : List Delta) : ℝ≥0 := ds.foldl (·+·) 0

/-- B-Gate⁺ verdict per window/degree. -/
structure GateInput where
  tau : ℝ≥0; deg : ℤ; ph1zero ext1_ok : Bool
  mu nu : ℕ; gap dsum : ℝ≥0
def BGATE_plus (gin : GateInput) : Bool :=
  gin.ph1zero ∧ gin.ext1_ok ∧ (gin.mu = 0) ∧ (gin.nu = 0) ∧ (gin.gap > gin.dsum)

/-- δ-commutation contract (uniform, additive, post-stable). -/
structure DeltaComm where
  delta : (ℤ → ℝ≥0 → ℝ≥0)
  uniform_in_F : True
  additive     : True
  post_stable  : True

/-- Soft A/B commuting with tolerance η. -/
def AB_soft_commuting (Δ_comm : Pers k → ℝ≥0) (η : ℝ≥0) : Prop :=
  ∀ (M : Pers k), Δ_comm M ≤ η

/-- Restart and summability. -/
def restart_ok (κ : ℝ≥0) (gap_next gap_curr dsum : ℝ≥0) : Prop :=
  gap_next ≥ κ * (gap_curr - dsum)
def summable_budget (budgets : ℕ → ℝ≥0) : Prop :=
  ∃ B : ℝ≥0, ∀ n, (∑ i in Finset.range n, budgets i) ≤ B

/-- Čech–Ext¹ glue under overlap consistency (schematic). -/
axiom cech_ext1_glue :
  ∀ (W : Windowing), OverlapConsistent W → (∀ j, Ext1_vanish_on W.cover[j]) → Ext1_vanish_globally
end AK.Gluing
\end{verbatim}

% -------------------------
\subsection*{F.10. Spectral calculus and Lipschitz bounds [Spec] (AK.Spectral)}
% -------------------------

We introduce three monotone operators on persistence modules:

- \(S^{\le\tau}\): hard-threshold deletion of all bars of length \(>\tau\);
- \(D^{<\tau}\): deletion of all bars of length \(<\tau\);
- \(C_r\): a contraction at rate \(r\in(0,1]\) on the time axis (reindexing \(t\mapsto rt\)).

They satisfy:
\[
D^{<\tau}\circ S^{\le \tau} = 0,\qquad\nS^{\le\tau}\circ \mathbf{T}_\tau \cong \mathbf{T}_\tau\circ S^{\le\tau},\qquad\nd_{\mathrm{int}}(C_r M, C_r N)\le r\cdot d_{\mathrm{int}}(M,N).
\]
Deletion-type operators are spectrally monotone and commute with \(\mathbf{T}_\tau\) up to iso; inclusion-type operators are controlled via stability bounds (Appendix~E).

\paragraph{Lean~4 stubs (AK.Spectral).}
\begin{verbatim}
namespace AK.Spectral
open AK.Core
noncomputable section
variable (k : Type*) [Field k]

noncomputable def S_le (τ : ℝ≥0) : Pers k ⥤ Pers k := by admit
noncomputable def D_lt (τ : ℝ≥0) : Pers k ⥤ Pers k := by admit
noncomputable def Contract (r : ℝ≥0) (hr : 0 < r ∧ r ≤ 1) : Pers k ⥤ Pers k := by admit

theorem deletion_monotone (τ) : Monotone (fun M => (S_le k τ).obj M) := by admit
theorem commute_Sle_Tτ (τ) : (S_le k τ) ⋙ (Tτ k τ) ≅ (Tτ k τ) ⋙ (S_le k τ) := by admit
theorem contract_lipschitz (r) (hr) :
  ∀ M N, d_int k ((Contract k r hr).obj M) ((Contract k r hr).obj N) ≤ r * d_int k M N := by admit
end AK.Spectral
\end{verbatim}

% -------------------------
\subsection*{F.11. Edge identification \(\mathrm{PH}_1 \Rightarrow \Ext^1\) [Spec] (AK.Core)}
% -------------------------

Let \(\mathcal{R}:\mathsf{FiltCh}(k)\to D(\mathsf{Vect}_k)\) be of amplitude \([-1,0]\).
There is a natural edge isomorphism
\[
H^{-1}(\mathcal{R}(F))\ \cong\ \varinjlim_{t} H_1(F^t).
\]
For \(A\in D^{[-1,0]}\) we have \(\Ext^1(A,k)\cong \Hom(H^{-1}(A),k)\).
Combining these we obtain, for any \(F\),
\[
\mathrm{PH}_1(F)=0\quad\Longrightarrow\quad \Ext^1(\mathcal{R}(F),k)=0,
\]
and the same implication after insertion of the collapse \(C_\tau\).

\paragraph{Lean~4 stubs (AK.Core; derived edge).}
\begin{verbatim}
namespace AK.Core
noncomputable section
variable (k : Type*) [Field k]

def ℛ : FiltCh k ⥤ Derived k := by admit            -- t-exact, amplitude [-1,0]
theorem edge_iso (F : FiltCh k) :
  H^{-1} (ℛ k).obj F ≅ colim t, H_1 (F^t) := by admit
theorem ext1_edge (A : Derived k) (hA : A ∈ D^{[-1,0]}) :
  Ext^1(A, (of k).obj k) ≅ Hom(H^{-1}(A), (of k).obj k) := by admit
end AK.Core
\end{verbatim}

% -------------------------
\subsection*{F.12. Coq sketches (mathcomp/coq-cat-theory) [Spec]}
% -------------------------

\begin{verbatim}
From mathcomp Require Import all_ssreflect all_algebra.
From CoqCT Require Import Category Abelian Functor Limits Colimits.
Set Implicit Arguments. Unset Strict Implicit. Unset Printing Implicit Defensive.

Module AK.

(* AK.Core *)
Parameter k : fieldType.
Axiom Vect : AbelianCat.                  (* f.d. k-vector spaces *)
Axiom Rposet : PreOrder.                  (* (ℝ, ≤), schematic *)
Definition Diag := FunctorCat Rposet Vect.
Parameter Constructible : Diag -> Prop.
Record Pers := { M : Diag; pfd : Constructible M }.

Axiom Eτ : SerreSubcat Pers.
Axiom Tτ : Functor Pers (Orthogonal Eτ).        (* exact, idempotent *)
Axiom iotaτ : Functor (Orthogonal Eτ) Pers.
Axiom Tτ_exact : ExactFunctor Tτ.
Axiom Tτ_idem  : FunctorComp Tτ Tτ ≅ Tτ.
Axiom Tτ_adj   : Adjunction Tτ iotaτ.
Axiom localization_equiv :
  SerreLocalization Pers Eτ (Orthogonal Eτ).

Parameter dint : Pers -> Pers -> R.             (* ℝ≥0∞, schematic *)
Parameter Shift : R -> Functor Pers Pers.
Axiom shift_zero : Shift 0 ≅ Id.
Axiom shift_add  : forall e d, FunctorComp (Shift e) (Shift d) ≅ Shift (e + d).
Axiom shift_comm :
  forall eps, FunctorComp (Shift eps) Tτ ≅ FunctorComp Tτ (Shift eps).
Axiom Tτ_nonexpansive :
  forall (X Y : Pers), dint (Tτ X) (Tτ Y) <= dint X Y.

(* Scope rule *)
Axiom filtered_colim_exact : forall J (F : Functor J Vect), IsFiltered J ->
  ExactFilteredColim F.
Axiom return_to_constructible :
  forall D, Constructible (Colim D).

(* Collapse on filtered complexes *)
Parameter FiltCh : Type.
Parameter P_i : Z -> Functor FiltCh Pers.
Parameter Cτ : R -> Functor FiltCh FiltCh.
Axiom Cτ_preserves_fqi : forall τ, PreservesFQI (Cτ τ).

(* AK.Tower *)
Record Tower := {
  obj : nat -> FiltCh; mor : forall n, obj n ⟶ obj n.+1;
  apex : FiltCh; cocone : forall n, obj n ⟶ apex }.

Parameter iter_mor : forall (T : Tower) n m, n ≤ m -> obj T n ⟶ obj T m.
Record TowerHom (T T' : Tower) := {
  α : forall n, obj T n ⟶ obj T' n;
  square : forall n, mor T n ;; α (n.+1) = α n ;; mor T' n;
  cocone_comm : forall n, α n ;; cocone T' n = cocone T n }.

Definition TτP (i : Z) (τ : R) : Functor FiltCh (Orthogonal Eτ) :=
  FunctorComp (P_i i) Tτ.

Parameter phi :
  forall (i : Z) (τ : R) (T : Tower),
    Colim (fun n => TτP i τ (obj T n)) ⟶ TτP i τ (apex T).
Axiom phi_natural :
  forall i τ (T T' : Tower) (h : TowerHom T T'),
    compose (phi i τ T') (colim_map h.(α)) = phi i τ T.

Axiom gdim_stabilizes :
  forall (X : Pers), exists T0 c, forall t, t ≥ T0 -> dim (M X t) = c.
Parameter gdim : Pers -> nat.
Axiom gdim_iso_invariant : forall X Y, X ≅ Y -> gdim X = gdim Y.
Definition mu i τ (T : Tower) := gdim (Ker (phi i τ T)).
Definition nu i τ (T : Tower) := gdim (Coker (phi i τ T)).

Theorem mu_nu_fqi_invariant :
  forall i τ T T', fqi_equiv T T' -> mu i τ T = mu i τ T' /\ nu i τ T = nu i τ T'.
Admitted.
Theorem mu_nu_cofinal_invariant :
  forall i τ T (r : nat -> nat), StrictMono r ->
    mu i τ T = mu i τ (reindex T r H) /\ nu i τ T = nu i τ (reindex T r H).
Admitted.

(* AK.LocalEquiv *)
Record window := { u : R; u' : R; len_pos : (u' > u)%R }.
Record WinHyp (W : window) := {
  amplitude_le_one : True; saturated_collapse : True;
  no_tau_accumulation : True; phi_tail_iso : True }.

Parameter Rfun : Functor FiltCh (DerivedCat Vect).
Axiom Rfun_t_exact : t_exact Rfun.  (* amplitude [-1,0] *)
Axiom edge_iso :
  forall F, Hm1 (Rfun F) ≅ Colim_t (H1 (F^t)).
Axiom ext1_edge :
  forall A, in_amplitude A (-1) 0 ->
    Ext1 A (embed k) ≅ Hom (Hm1 A) (embed k).
Axiom local_bridge :
  forall (W : window) (H : WinHyp W) F τ,
    PH1_on W F = 0 -> Ext1 (Rfun (Cτ τ F)) (embed k) = 0.

(* AK.Gluing *)
Record windowing := {
  u0 : R; U : R; cover : seq window;
  disjoint :
    all (fun p => (let: (w1, w2) := p in (~~ (w1 == w2))
          ==> ( (w1.(u') <= w2.(u)) || (w2.(u') <= w1.(u)) ))) (all_pairs cover);
  exact_cover : \sum_(w <- cover) (w.(u') - w.(u)) = (U - u0);
  right_inclusion : True }.

Record delta := { alg : R; disc : R; meas : R }.
Definition dsum (d : delta) : R := d.(alg) + d.(disc) + d.(meas).
Definition budget (ds : seq delta) : R := \sum_(d <- ds) (dsum d).

Record gate_input := {
  tau : R; deg : Z; ph1zero : bool; ext1_ok : bool;
  mu : nat; nu : nat; gap : R; dsum_win : R }.
Definition BGATE_plus (g : gate_input) :=
  ph1zero g && ext1_ok g && (mu g == 0) && (nu g == 0) && (gap g > dsum_win g).

Record delta_comm := {
  delta : Z -> R -> R; uniform_in_F : True; additive : True; post_stable : True }.
Definition AB_soft_commuting (Delta_comm : Pers -> R) (eta : R) :=
  forall M, (Delta_comm M <= eta)%R.
Definition restart_ok (kappa gap_next gap_curr dsum : R) :=
  (gap_next >= kappa * (gap_curr - dsum))%R.
Definition summable_budget (budgets : nat -> R) :=
  exists B, forall n, (\sum_(i < n) budgets i <= B)%R.

Axiom cech_ext1_glue :
  forall W, OverlapConsistent W -> (forall j, Ext1_vanish_on (nth window 0 W.(cover) j)) ->
    Ext1_vanish_globally.

(* AK.Spectral *)
Parameter Sle : R -> Functor Pers Pers.
Parameter Dlt : R -> Functor Pers Pers.
Parameter Contract : R -> Functor Pers Pers.
Axiom Sle_Tτ_comm : forall τ, FunctorComp (Sle τ) Tτ ≅ FunctorComp Tτ (Sle τ).
Axiom Contract_lipschitz : forall r, 0 < r <= 1 ->
  forall X Y, dint (Contract r X) (Contract r Y) <= r * dint X Y.
End AK.
\end{verbatim}

% -------------------------
\subsection*{F.13. Tests and fixtures (T7, T10, T13) [Spec]}
% -------------------------

\paragraph{T7 (Saturation gate).}
Construct a tower of Type~IV (pure cokernel) and one with a stationary summand (pure kernel),
and their direct sum. Verify that:
(i) \(\mu,\nu\) equal the multiplicity of \(I[0,\infty)\) after \(\mathbf{T}_\tau\);
(ii) cofinal reindexing \(n\mapsto n+1\) preserves \((\mu,\nu)\);
(iii) under (S1) the comparison \(\phi_{i,\tau}\) is an isomorphism and \((\mu,\nu)=(0,0)\).

\paragraph{T10 (A/B).}
Instantiate an \(\eta\)-tolerant A/B test on overlaps, verify \(\mathrm{AB\_soft\_commuting}\) and that the
window-local \(\Ext^1\)-vanishing glues via \(\mathrm{cech\_ext1\_glue}\).

\paragraph{T13 (\(\delta\)-budget).}
Generate a \(\delta\)-ledger per window/degree; check additivity/post-stability and the restart inequality
\(\mathrm{gap}_{k+1}\ge \kappa(\mathrm{gap}_k-\Sigma\delta_k)\).
Verify summability \(\sum_k\Sigma\delta_k<\infty\) and that \(\mathrm{BGATE}^+\) accepts precisely when
\(\mathrm{gap}>\mathrm{dsum}\) alongside \(\mathrm{PH}_1=0\), \(\Ext^1=0\), and \((\mu,\nu)=(0,0)\).

% -------------------------
\subsection*{F.14. What is proved, what is assumed [Spec]}
% -------------------------

\begin{itemize}
\item (\emph{Localization}) \(\mathsf{E}_\tau\) is hereditary Serre; the reflector \(\mathbf{T}_\tau\) exists, is exact,
idempotent, and induces \(\Pers^{\mathrm{ft}}_k/\mathsf{E}_\tau\simeq \mathsf{E}_\tau^\perp\) (\(\tau\)-local/orthogonal).
\item (\emph{Stability}) \(\mathbf{T}_\tau\) is \(1\)-Lipschitz for the interleaving metric; the proof proceeds via
shift functors and natural isomorphisms \(\mathrm{Shift}_\varepsilon\circ \mathbf{T}_\tau\simeq \mathbf{T}_\tau\circ \mathrm{Shift}_\varepsilon\) (Appendix~A, Prop.~\ref{A:prop:lipschitz}).
\item (\emph{Towers}) The comparison map \(\phi_{i,\tau}\) is functorial in the cocone; under any of (S1)–(S3)
of Appendix~D (commutation, no \(\tau\)-accumulation, Cauchy+compatibility) it is an isomorphism, hence
\((\mu,\nu)=(0,0)\). The indices \((\mu,\nu)\) are invariant under filtered quasi-isomorphisms and cofinal
reindexings. \emph{Finiteness:} bounded-in-degree complexes imply only finitely many nonzero \(\mu_{i,\tau},\nu_{i,\tau}\),
so \(\sum_i \mu_{i,\tau}\) and \(\sum_i \nu_{i,\tau}\) are finite.
\item (\emph{Bridge}) For \(F\in\mathsf{FiltCh}(k)\), \(\mathcal{R}\) is \(t\)-exact of amplitude \([-1,0]\);
\(H^{-1}(\mathcal{R}(F))\simeq \varinjlim_t H_1(F^t)\) and
\(\Ext^1(A,k)\simeq \Hom(H^{-1}(A),k)\) for \(A\in D^{[-1,0]}\), hence
\(\mathrm{PH}_1(F)=0\Rightarrow \Ext^1(\mathcal{R}(F),k)=0\) (Appendix~C). All statements are available window-locally under AK.LocalEquiv hypotheses and glue globally under AK.Gluing.
\item (\emph{Spectral}) Deletion-type operators commute with \(\mathbf{T}_\tau\) (up to iso) and are spectrally monotone; inclusion-type operators are governed by stability bounds. Contractions \(C_r\) are \(r\)-Lipschitz.
\end{itemize}

% -------------------------
\subsection*{F.15. Notes on libraries and portability [Spec]}
% -------------------------

The Lean sketch targets \textsf{mathlib} (abelian categories, Serre subcategories, localization, derived categories).
The Coq sketch targets \textsf{mathcomp}+\textsf{coq-category-theory} (or \textsf{UniMath}). Nontrivial steps are
isolated behind \texttt{admit}/\texttt{Axiom} with explicit references to Appendix~A–E.
Replacing them by library lemmas yields a complete development.
For Lean’s subtype \texttt{Pers}, minor coercions may be required when using \texttt{Interleaving.dist};
a thin wrapper or instance resolves this in practice.
In Lean, define \(\phi_{i,\tau}\) via \texttt{Limits.colimit.desc} and use \texttt{colimit.hom\_ext} for naturality;
in Coq, use \texttt{Colim.desc}/\texttt{colim\_map} with a right-to-left \texttt{compose} convention.

% -------------------------
\subsection*{F.16. Logging, reproducibility, and harness [Spec]}
% -------------------------

A minimal harness should expose:
\begin{itemize}
  \item \textbf{Windowing API:} constructors/checkers for MECE partitions and coverage; \texttt{collapse\_tau} policy per window; spectral bin policy \(([a,b],\beta)\).
  \item \textbf{Gate API:} a \texttt{BGATE\_plus} predicate that consumes PH1/Ext1/(\(\mu,\nu\))/safety margin/\(\delta\)-budget and yields a Boolean verdict per window/degree, with manifest fields (\texttt{ext1\_eligible}, \texttt{ext1\_used\_in\_gate}) as in Appendix~C.
  \item \textbf{\(\delta\)-commutation API:} a structure capturing \(\delta(i,\tau)\) with \emph{uniformity in \(F\)}, \emph{pipeline additivity}, and \emph{post-processing stability}; metrized A/B tests with tolerance \(\eta\) and soft-commuting/fallback logging (Appendix~B/L).
  \item \textbf{Restart/Summability:} verification routines for \(\mathrm{gap}_{k+1}\ge \kappa(\mathrm{gap}_k-\Sigma\delta_k)\) and \(\sum_k\Sigma\delta_k<\infty\) (Appendix~D).
  \item \textbf{Tower diagnostics:} wrappers for \(\phi_{i,\tau}\), \(\mu_{i,\tau},\nu_{i,\tau}\), stability-band detection across \(\tau\)-sweeps, and cofinal invariance checks.
  \item \textbf{Artifacts/manifest:} standardized JSON/YAML fields for windows, coverage checks, operations, \(\delta\)-ledger, persistence/spectral/aux-bars, gate verdicts, and seeds/versions (Appendix~G).
\end{itemize}

% -------------------------
\subsection*{F.17. Outcome and completion note}
% -------------------------

The modules AK.Core/LocalEquiv/Tower/Gluing/Spectral, together with the stubs above, provide a complete formalization \textbf{Spec} for:
(i) the categorical spine (\(\mathbf{T}_\tau\), exactness, and \(1\)\hyp Lipschitz),
(ii) tower diagnostics \((\mu,\nu)\) with invariances and sufficient conditions,
(iii) the bridge \(\mathrm{PH}_1\Rightarrow \Ext^1\) under amplitude \(\le 1\), both window-locally and globally by gluing,
(iv) the operational layer (MECE windows, Overlap/B\hyp Gate\(^{+}\), \(\delta\)\hyp commutation, Restart/Summability, A/B tests), and
(v) reproducibility/logging contracts.
With this integration, no further supplementation is required to mechanize the v16.0 guard-rails in Lean/Coq; replacing \texttt{admit}/\texttt{Axiom} by library lemmas or local developments yields a fully checked artifact consistent with the main text.



% =========================
\section*{Appendix G. Reproducibility: Logs and Schemas [Spec] (reinforced)}
% =========================
\phantomsection
\addcontentsline{toc}{section}{Appendix G. Reproducibility: Logs and Schemas}
\refstepcounter{section}
\label{G:repro}

This appendix specifies the provenance log (\texttt{run.yaml}) and the machine-readable
schemas for artifacts produced in this work—barcodes (\texttt{bars}), spectral indicators
(\texttt{spec}), Ext-tests (\texttt{ext}), tower comparison maps (\texttt{phi}), and the windowed
length spectrum audit (\texttt{Lambda\_len}).
All files may be emitted in either JSON or HDF5; JSON keys coincide with HDF5
group/dataset names. Colimits are used only under the scope policy
(Appendix~A, Remark~\ref{A:rk:filtered-colimits}). Type labels follow \emph{Type I--II / Type III / Type IV}.
Cross-module conventions: the Ext-test is always against \(k[0]\), i.e.\ \(\Ext^1(\mathcal{R}(C_\tau F),k)=0\)
(with \(C_\tau\) understood up to f.q.i.\ on \(\Ho(\mathsf{FiltCh}(k))\)); the energy exponent satisfies
\(\alpha>0\) (default \(\alpha=1\)). Spectral monotonicity is asserted only for \emph{deletion-type} operations
(Dirichlet/principal/Loewner), with directions fixed by Appendix~E; inclusion-type operations
are used solely with stability bounds. All comparisons follow the mandatory order:
\[
\boxed{\ \text{for each } t\ \Longrightarrow\ \text{apply } \mathbf{P}_i \ \Longrightarrow\ \text{apply } \mathbf{T}_\tau \ \Longrightarrow\ \text{compare in }\Pers^{\mathrm{ft}}_k\ }.
\]

\paragraph{New in this version (2025-03-15).}
Beyond the original specification, this version integrates:
(i) \texttt{windows} (domain/collapse/spectral),
(ii) \texttt{coverage\_check},
(iii) \texttt{operations} (with \texttt{U}, \texttt{type}, \(\tau\), and \(\delta\) breakdown),
(iv) \texttt{persistence} (summaries: \texttt{PH1\_zero}/\texttt{Ext1\_zero}/\(\mu\)/\(\nu\)/\texttt{phi\_iso\_tail}),
(v) \texttt{spectral.aux\_bars\_remaining},
(vi) \texttt{budget} (\texttt{sum\_delta}/\texttt{safety\_margin}/\texttt{gap\_tau}),
(vii) \texttt{gate.accept},
(viii) \texttt{overlap\_checks} (Overlap Gate),
(ix) \texttt{Lambda\_len} (length spectrum audit),
(x) \texttt{spectral\_policy} (norms, ordering, spectral bounds).
These fields ensure bitwise reproducibility and third-party auditability.

% -------------------------
\subsection*{G.1. Provenance, determinism, and gating}
% -------------------------
Each run records (i) source/inputs, (ii) algorithmic choices and thresholds, (iii) numeric
tolerances and units, (iv) code/environment fingerprints, (v) RNG details, (vi) strong identifiers
(content hashes) for all artifacts, and (vii) \emph{gating} decisions that determine acceptance of results.
Randomness is controlled by explicit seeds. Floating-point claims report both an \emph{asserted} tolerance
and a \emph{measured} slack. \emph{Windows} (domain/collapse/spectral) must be declared, and a
\emph{coverage check} attests that all measured quantities fall inside their stated windows.
A \emph{budget} aggregates operation-level error contributions \(\delta\) and yields a
\emph{safety margin} relative to the governing tolerance; finally, \texttt{gate.accept} records the run-level
decision (accept/reject) together with reasons.

% -------------------------
\subsection*{G.2. \texttt{run.yaml} schema (versions, windows, overlap, budget, gate)}
% -------------------------
\paragraph{Intent.} A single file per execution, sufficient to reproduce the pipeline end-to-end, including
all windows, coverage checks, operation logs, overlap checks, and the final acceptance gate.

\noindent\textbf{Canonical layout (YAML).}
\begin{verbatim}
version: 1
schema_version: "2025-03-15"
run_id: "2025-03-15T09:12:07Z-7f5c1b1"
seed: 1337
rng:
  python: "default_rng"
  numpy:  "PCG64"
platform:
  os: "Ubuntu 22.04"
  cpu: "Intel(R) Xeon(R) Platinum 8370C"
  cuda: "12.2"
  blas: "OpenBLAS 0.3.23"
  hdf5: "1.14.3"          # HDF5 library version (mandatory)
  lapack: "OpenBLAS-LAPACK"
  glibc: "2.35"
  kernel: "5.15.0-105"
  locale: "C.UTF-8"
env:
  python: "3.11.7"
  packages:
    numpy: "1.26.4"
    scipy: "1.13.1"
    h5py:  "3.10.0"
    networkx: "3.2.1"
  threads:
    OMP_NUM_THREADS: 1
    MKL_NUM_THREADS: 1
    OPENBLAS_NUM_THREADS: 1
container:
  image: "docker.io/example/persistence:2025.03"
  digest: "sha256:deadbeef..."    # content-addressed image digest
git:
  repo: "git@host:ak/persistence.git"
  commit: "a1b2c3d4"
units:
  filtration: "dimensionless"     # e.g. "height", "time"
  eigenvalues: "dimensionless"    # e.g. "1/length^2" for Laplacians
windows:
  domain:
    filtration_range: [0.0, 2.0]  # closed interval semantics for reporting
    degrees: [0, 2]               # inclusive range of homological degrees considered
  collapse:
    tau_sweep: [0.25, 0.50, 1.00] # list of τ values evaluated (subset may be used)
  spectral:
    range: [0.0, 2.0]             # closed interval semantics
    order: "ascending"            # expected storage order for eigen arrays
coverage_check:
  domain_window_covers_bars: true
  spectral_window_covers_thetas: true
  collapse_tau_sweep_covers_reports: true
overlap_checks:
  local_equiv: true               # post-collapse equality (up to budget) on overlaps
  cech_ext1_ok: true              # Čech–Ext^1 acyclicity holds on overlaps
  stability_band_ok: true         # μ=ν=0 across τ-band; no near-τ accumulation
inputs:
  dataset: "AK-bench-v3"
  graphs:
    - path: "data/G_001.edgelist"
      hash: "sha256:..."
  filters:
    type: "height"
    params: { axis: 2 }
pipeline:
  metric: "interleaving"          # or "bottleneck" (exactly one)
  stages:
    - name: "barcode"
      params: { field: "k", reduction: "clearing" }
    - name: "collapse"            # C_τ (up to f.q.i.)
      params: { tau: 0.50 }
    - name: "spec"
      params:
        window: [0.0, 2.0]        # reporting window for spectral outputs
        norm: "fro"               # "fro" ≡ ‖·‖_fro, "op" ≡ ‖·‖_op (Appendix E)
        order: "ascending"        # storage order for eigen arrays
        clip: 1.00                # clip acts on eigenvalues; window gates reporting
        loewner_assumption: "A'⪯A"   # enum: "A'⪯A" | "A'⪰A" | "none" (Appendix E)
        spectral_bounds:
          lambda_min: 1.0e-12
          lambda_max: 1.0e+05
          lip_tol: 0.02
        eig_solver:
          method: "lanczos"
          k: 128
          maxiter: 1000
          tol: 1e-12
          reorthogonalize: true
          rng_seed: 1337
    - name: "ext-test"            # Ext^1(R(C_τ F), k)
      params: { amplitude_check: true }
operations:
  - step: 1
    U: [0,1,3]                    # index set or label of affected subset
    type: "inclusion"             # enum: inclusion|projection|quasi_iso|
                                  #       filtration_preserving_map|schur_complement|other
    tau: 0.50
    delta:
      distance:
        interleaving: 0.050       # present iff pipeline.metric == "interleaving"
      sources:
        discretization: 0.030
        rounding: 1.0e-12
        heuristic: 0.020
      total: 0.050
      note: "Edge contraction in subgraph U"
  - step: 2
    U: "V\\W"
    type: "schur_complement"
    tau: 0.50
    delta:
      distance:
        interleaving: 0.100
      sources:
        elimination: 0.080
        rounding: 0.020
      total: 0.100
persistence:
  PH1_zero: true                  # H_1 extinguished in the reported window
  Ext1_zero: true                 # Ext^1 vanishes (as reported in ext artifact)
  mu: 1                           # generic kernel dimension after truncation
  nu: 0                           # generic cokernel dimension after truncation
  phi_iso_tail:
    passed: false
    i: 1
    tau: 0.50
spectral:
  aux_bars_remaining: 0           # count of spectral aux-bars remaining (diagnostic)
thresholds:
  alpha: 1.0                      # α>0 (default α=1)
  tol:
    distance:
      interleaving: 1e-6          # present iff pipeline.metric == "interleaving"
      # bottleneck: 1e-6          # present iff pipeline.metric == "bottleneck"
    eig: 1e-8
    witness: 1e-9
budget:
  sum_delta:
    distance:
      interleaving: 0.150         # sum over operations[*].delta.distance.interleaving
  safety_margin: 0.850            # 1 - sum_delta / thresholds.tol.distance.interleaving
  gap_tau: 0.025                  # edge gap to τ-window boundary (used by Restart)
  rationale: "All deltas accounted for; slack remains >0"
Lambda_len:
  degree: 1
  tau: 0.50
  audit: "hash:2f4c...d1"         # or an explicit eigenvalue list (see G.10)
gate:
  accept: true
  reason: "Coverage ok; safety margin positive; all assertions satisfied"
serialization:
  float_dtype: "ieee754-f64-le"
  json_sort_keys: true
  hdf5_canonical:
    compression: { algo: "gzip", level: 4 }
    shuffle: false
    fletcher32: false
    track_times: false
    fillvalue: 0.0
    string_encoding: "utf8-fixed"     # fixed-length UTF-8 strings for bitwise reproducibility
    chunk_shapes:
      bars: { i: 4096, birth: 4096, death: 4096, death_is_inf: 4096, mult: 4096 }
      spec_eigs: { eig: 4096 }
      spec_Ntheta: { theta: 512, left: 512, right: 512 }
      phi_idx: { i: 256, tau: 256, iso: 256, mu: 256, nu: 256 }
cache:
  enabled: true
  dir: ".cache/run_7f5c1b1"
timing:
  wallclock_s: 123.4
  cpu_s: 456.7
  stages:
    barcode: 12.3
    collapse: 4.5
    spec: 80.0
    ext_test: 2.0
status:
  success: true
  errors: []
outputs:
  bars: "out/bars_7f5c1b1.json"
  spec: "out/spec_7f5c1b1.json"
  ext:  "out/ext_7f5c1b1.json"
  phi:  "out/phi_7f5c1b1.h5"
\end{verbatim}

% -------------------------
\subsection*{G.3. \texttt{bars} (barcodes) schema}
% -------------------------
\paragraph{Semantics.} A constructible barcode is a multiset of half-open intervals \(I=[b,d)\) with degree \(i\).
Deaths may be \(+\infty\).

\noindent\textbf{JSON layout (with infinity convention, units, and cross-link).}
\begin{verbatim}
{
  "meta": {
    "schema_version": "2025-03-15",
    "field": "k",
    "filtration_units": "dimensionless",
    "endpoint_convention": "[b,d) (see Chapter~2)",
    "infinity": { "json": "inf" },
    "float_dtype": "ieee754-f64-le",
    "string_encoding": "utf8-fixed",
    "links": {
      "run_id": "2025-03-15T09:12:07Z-7f5c1b1",
      "run_yaml_hash": "sha256:...run"
    }
  },
  "bars": [
    { "i": 0, "birth": 0.0, "death": 0.3,   "mult": 1 },
    { "i": 1, "birth": 0.2, "death": "inf", "mult": 1 }
  ],
  "hash": "sha256:...bars"
}
\end{verbatim}

\noindent\textbf{JSON Schema snippet for \texttt{bars.bars[*].death}.}
\begin{verbatim}
"bars":{"type":"array","items":{
  "type":"object",
  "properties":{
    "i":{"type":"integer"},
    "birth":{"type":"number"},
    "death":{"oneOf":[{"type":"number"},{"type":"string","enum":["inf"]}]},
    "mult":{"type":"integer","minimum":1}
  },
  "required":["i","birth","death","mult"]
}}
\end{verbatim}

\noindent\textbf{HDF5 layout (split representation; no numeric sentinel).}
\begin{itemize}[leftmargin=1.25em]
  \item Datasets:
    \path{/bars/i} (\texttt{int32}),
    \path{/bars/birth} (\texttt{float64}),
    \path{/bars/death} (\texttt{float64}),
    \path{/bars/death_is_inf} (\texttt{bool}),
    \path{/bars/mult} (\texttt{int32}).
  \item Attributes:
    \path{/bars}.attrs[\texttt{field}=\texttt{"k"}],
    \texttt{filtration\_units},
    \texttt{schema\_version},
    \texttt{float\_dtype},
    \texttt{death\_encoding}=\texttt{"split\_scalar\_bool"},
    \texttt{string\_encoding}=\texttt{"utf8-fixed"}.
\end{itemize}

\noindent\textbf{Invariants.} Local finiteness on bounded windows (Appendix~A). Optional sorting by \((i,b,d)\);
multiplicities aggregate identical intervals. Report-window compliance is checked via
\nolinkurl{run.yaml.coverage_check.domain_window_covers_bars}.

% -------------------------
\subsection*{G.4. \texttt{spec} (spectral indicators) schema}
% -------------------------
\paragraph{Semantics.} Spectral features: clipped sums, counts above/below thresholds with left/right limits,
and deletion-type monotonicity diagnostics. Matrices are identified by content hashes.

\noindent\textbf{JSON layout (ascending storage; \(N_{\theta\pm 0}\); solver params; coverage; cross-link).}
\begin{verbatim}
{
  "meta": {
    "schema_version": "2025-03-15",
    "eigen_units": "dimensionless",
    "order": "ascending",
    "sorted": true,
    "Ntheta_convention": { "left": "N_{θ-0}", "right": "N_{θ+0}" },
    "window": { "range": [0.0, 2.0], "semantics": "closed" },
    "norm": "fro",                  // "fro" ≡ ‖·‖_fro, "op" ≡ ‖·‖_op
    "clip_tau": 1.0,
    "tol_eig": 1e-8,
    "loewner_assumption": "A'⪯A",
    "aux_bars_remaining": 0,
    "coverage_check": { "thetas_in_window": true },
    "string_encoding": "utf8-fixed",
    "eig_solver": {
      "method": "lanczos", "k": 128, "maxiter": 1000,
      "tol": 1e-12, "reorthogonalize": true, "rng_seed": 1337
    },
    "links": {
      "run_id": "2025-03-15T09:12:07Z-7f5c1b1",
      "run_yaml_hash": "sha256:...run"
    }
  },
  "operators": [
    {
      "id": "sha256:...A",
      "kind": "laplacian_dirichlet",
      "n": 500,
      "spectrum": { "eigs": [0.10, 0.12, 0.45, ...] },  // ascending
      "clip":     { "tau": 1.0, "sum": 37.219, "deficit": 12.004 }
    },
    {
      "id": "sha256:...B",
      "kind": "principal_submatrix",
      "parent": "sha256:...A",
      "N_theta": [
        { "theta": 0.20, "left": 17, "right": 16 },
        { "theta": 0.50, "left": 10, "right": 10 }
      ],
      "monotonicity": { "type": "deletion", "passed": true }
    }
  ],
  "hash": "sha256:...spec"
}
\end{verbatim}

\noindent\textbf{HDF5 layout.}
\begin{itemize}[leftmargin=1.25em]
  \item \path|/spec/ops/{id}/eig| (float64, ascending),
        \path|/spec/ops/{id}/clip/sum| (float64),
        \path|/spec/ops/{id}/clip/deficit| (float64),
        \path|/spec/ops/{id}/Ntheta/theta|, \path|/spec/ops/{id}/Ntheta/left|, \path|/spec/ops/{id}/Ntheta/right|
        (parallel datasets).
  \item Attributes: \texttt{kind}, \texttt{parent}, \texttt{norm}
        \( \in \{ \text{\texttt{"fro"}}, \text{\texttt{"op"}} \} \) (Appendix~E),
        \texttt{order}=\texttt{"ascending"}, \texttt{sorted} (bool), \texttt{eigen\_units}, \texttt{tol\_eig},
        \texttt{schema\_version}, \texttt{loewner\_assumption}
        \( \in \{ \text{\texttt{"A' $\,\preceq\,$ A"}}, \text{\texttt{"A' $\,\succeq\,$ A"}}, \text{\texttt{"none"}} \} \),
        \texttt{aux\_bars\_remaining} (int32), \texttt{coverage\_thetas\_in\_window} (bool),
        \texttt{string\_encoding}=\texttt{"utf8-fixed"},
        and a nested \texttt{eig\_solver} group mirroring the JSON.
\end{itemize}

% -------------------------
\subsection*{G.5. \texttt{ext} (Ext-test) schema}
% -------------------------
\paragraph{Semantics.} Outcome of the one-way bridge \(\mathrm{PH}_1\Rightarrow \Ext^1\) for \(C_\tau F\), with
amplitude checks for \(\mathcal{R}\) and recorded assumptions.

\noindent\textbf{JSON layout (with assumptions and cross-links).}
\begin{verbatim}
{
  "meta": {
    "schema_version": "2025-03-15",
    "field":"k", "alpha": 1.0,
    "assumptions": {
      "field_is_k": true,
      "constructible_verified": true,
      "t_exact_and_amp_le_1": true
    },
    "string_encoding": "utf8-fixed",
    "links": {
      "run_id": "2025-03-15T09:12:07Z-7f5c1b1",
      "run_yaml_hash": "sha256:...run"
    }
  },
  "tau": 0.50,
  "amplitude": { "ok": true, "range": [-1, 0] },
  "Hminus1": { "dim": 0, "witness_norm": 0.0 },
  "Ext1":    { "dim": 0, "passed": true, "tol": 1e-9, "slack": 0.0 },
  "links":   { "bars": "sha256:...bars", "phi": "sha256:...phi" },
  "hash": "sha256:...ext"
}
\end{verbatim}

\noindent\textbf{HDF5 layout.}
\begin{itemize}[leftmargin=1.25em]
\item Scalars: \texttt{/ext/tau} (float64), \texttt{/ext/Hminus1/dim} (int32),
\texttt{/ext/Ext1/dim} (int32), \texttt{/ext/Ext1/passed} (bool), \texttt{/ext/Ext1/tol} (float64),
\texttt{/ext/Ext1/slack} (float64).
\item Attributes: \texttt{field}=\texttt{"k"}, \texttt{alpha}=float64, \texttt{schema\_version},
\texttt{assumptions/*} (bools as above), \texttt{string\_encoding}=\texttt{"utf8-fixed"}.
\end{itemize}

% -------------------------
\subsection*{G.6. \texttt{phi} (tower comparison) schema}
% -------------------------
\paragraph{Semantics.} Encodes \(\phi_{i,\tau}\) for towers, together with \((\mu,\nu)\) as generic-fiber
dimensions after truncation, and structural flags for the sufficiency hypotheses (Appendix~D, §D.3).

\noindent\textbf{JSON layout (with \(\tau\) sweep, edge kinds, witnesses, iso-tail, cross-link).}
\begin{verbatim}
{
  "meta": {
    "schema_version": "2025-03-15",
    "definition": "phi_{i,τ}: colim T_τ P_i(F_n) → T_τ P_i(F_∞)",
    "scope": "colim in [ℝ,Vect_k], return-to-constructible policy",
    "tau_sweep": [0.25, 0.50, 1.00],
    "edge_kinds": ["inclusion","projection","quasi_iso",
                   "filtration_preserving_map","schur_complement","other"],
    "string_encoding": "utf8-fixed",
    "links": {
      "run_id": "2025-03-15T09:12:07Z-7f5c1b1",
      "run_yaml_hash": "sha256:...run"
    }
  },
  "indices": [
    {
      "i": 1, "tau": 0.50,
      "iso": false,
      "mu": 1, "nu": 0,
      "flags": { "S1_commutes": false, "S2_noAccum": true, "S3_Cauchy": false },
      "witness": { "ker_generic_dim": 1, "coker_generic_dim": 0 },
      "iso_tail": { "passed": false }   // mirrors run.yaml.persistence.phi_iso_tail
    }
  ],
  "tower": {
    "nodes": [
      { "n": 0, "id": "sha256:...F0" },
      { "n": 1, "id": "sha256:...F1" }
    ],
    "edges": [
      { "src": 0, "dst": 1, "kind": "inclusion" }
    ],
    "limit": { "id": "sha256:...Finf" }
  },
  "hash": "sha256:...phi"
}
\end{verbatim}

\noindent\textbf{HDF5 layout.}
\begin{itemize}[leftmargin=1.25em]
\item \texttt{/phi/idx/i} (int32), \texttt{/phi/idx/tau} (float64),
\texttt{/phi/idx/iso} (bool), \texttt{/phi/idx/mu} (int32), \texttt{/phi/idx/nu} (int32).
\item \texttt{/phi/idx/flags}: \texttt{S1\_commutes}, \texttt{S2\_noAccum}, \texttt{S3\_Cauchy} (bool).
\item Optional witnesses: \texttt{/phi/idx/ker\_generic\_dim}, \texttt{/phi/idx/coker\_generic\_dim}.
\item Optional tail: \texttt{/phi/idx/iso\_tail/passed} (bool).
\item Optional tower edges: \texttt{/phi/tower/edges/src,dst} (int32),
\texttt{/phi/tower/edges/kind} (fixed-length UTF-8 string; attribute \texttt{string\_encoding}=\dq utf8-fixed\dq).
\item Attributes: \texttt{schema\_version}, \texttt{string\_encoding}=\texttt{"utf8-fixed"},
\texttt{tau\_sweep} (float64 array).
\end{itemize}

% -------------------------
\subsection*{G.7. \texttt{Lambda\_len} (windowed length spectrum) schema}
% -------------------------
\paragraph{Semantics.} The length spectrum operator \(\Lambda_{\mathrm{len}}(M;[0,\tau])\) is diagonal on bar-basis with eigenvalues
the clipped bar lengths on \([0,\tau]\). Its unordered eigenvalue multiset equals the clipped bar-length multiset
(Appendix~H). The \texttt{Lambda\_len} audit records either the eigenvalue list (small instances) or a content hash.

\noindent\textbf{JSON layout.}
\begin{verbatim}
{
  "meta": {
    "schema_version": "2025-03-15",
    "definition": "Lambda_len(P_i(C_τ F); [0,τ])",
    "degree": 1,
    "tau": 0.50,
    "string_encoding": "utf8-fixed",
    "links": { "bars": "sha256:...bars", "phi": "sha256:...phi", "run_yaml_hash": "sha256:...run" }
  },
  "eigs": [0.24, 0.51, 0.78],      // optional explicit eigenvalues (small cases)
  "hash": "sha256:2f4c...d1"       // required content hash for audit/replay
}
\end{verbatim}

\noindent\textbf{HDF5 layout.}
\begin{itemize}[leftmargin=1.25em]
  \item \texttt{/Lambda\_len/meta} attributes: \texttt{schema\_version}, \texttt{degree}, \texttt{tau},
  \texttt{string\_encoding}=\texttt{"utf8-fixed"}, links as above.
  \item \texttt{/Lambda\_len/eigs} (optional; float64 array).
  \item \texttt{/Lambda\_len/hash} (fixed-length UTF-8 string).
\end{itemize}

% -------------------------
\subsection*{G.8. Content hashing and canonical serialization}
% -------------------------
Each artifact carries a content hash \texttt{sha256:...} over its canonical serialization
(JSON with sorted keys; HDF5 with \emph{fixed} dataset/attribute creation order, chunk shapes,
compression and filters). All floating datasets are little-endian IEEE-754 double (float64).
Cross-file links (\texttt{bars} \(\leftrightarrow\) \texttt{phi} \(\leftrightarrow\) \texttt{ext} \(\leftrightarrow\) \texttt{spec} \(\leftrightarrow\) \texttt{Lambda\_len})
use these hashes exclusively.
\emph{JSON numeric policy:} finite numbers only; positive infinity is encoded as the string
\texttt{"inf"} where applicable (see \texttt{bars.meta.infinity}). HDF5 encodes \(+\infty\) via the
\emph{split representation} \texttt{/bars/death} (float64) + \texttt{/bars/death\_is\_inf} (bool).
\emph{Strings:} all JSON strings and HDF5 string datasets/attributes are \emph{fixed-length} UTF-8
(\texttt{string\_encoding}=\texttt{"utf8-fixed"}) to ensure bitwise reproducibility.
\emph{HDF5 canonicalization:} for bit-wise reproducibility, set \texttt{track\_times=false},
\texttt{shuffle=false}, \texttt{fletcher32=false}, \texttt{fillvalue=0.0}, compression to GZIP level~4,
and use the \texttt{chunk\_shapes} recorded in \texttt{run.yaml}; create datasets and attributes in the
order shown in this appendix.

% -------------------------
\subsection*{G.9. Numeric tolerances, \(\delta\)-budgets, and audit trail}
% -------------------------
Every quantitative claim includes:
\begin{itemize}[leftmargin=1.25em]
\item \textbf{tolerance} (\texttt{tol}) declared in \texttt{run.yaml};
\item \textbf{slack} (\texttt{slack}) measured margin to the decision boundary;
\item \textbf{norm} used for spectral bounds (\texttt{fro} or \texttt{op}), consistent with Appendix~E;
\item \textbf{metric} for persistence distances (\texttt{interleaving} or \texttt{bottleneck}),
with \texttt{thresholds.tol.distance} carrying the matching key exactly once;
\item \textbf{budget} aggregation: \texttt{operations[*].delta} contributions summed into
\texttt{budget.sum\_delta}, with \texttt{budget.safety\_margin} and \texttt{budget.gap\_tau};
\item \textbf{solver} details for spectral computations (Lanczos parameters, RNG seed);
\item \textbf{windows and coverage}: \texttt{windows.*} declare scopes; \texttt{coverage\_check.*} record pass/fail;
\item \textbf{threading} environment and BLAS/LAPACK/HDF5 versions;
\item \textbf{gate} decision: \texttt{gate.accept} with \texttt{gate.reason}.
\end{itemize}

% -------------------------
\subsection*{G.10. Tests T14/T15 and reproducibility checklist}
% -------------------------
\paragraph{T14 (Overlap Gate gluing).}
On a windowed cover, verify:
(i) post-collapse equality (up to budget) on overlaps (\texttt{overlap\_checks.local\_equiv=true}),
(ii) Čech–\(\Ext^1\) acyclicity on overlaps (\texttt{cech\_ext1\_ok=true}),
(iii) stability band detection (\(\mu=\nu=0\), \texttt{stability\_band\_ok=true}),
(iv) A/B soft-commuting with logged residuals, added to the \(\delta\)-ledger,
(v) global gate acceptance with additive budgets.

\paragraph{T15 (Length spectrum audit).}
Compute \(\Lambda_{\mathrm{len}}(\T_\tau \Pfun_i(F);[0,\tau])\) and verify that its eigenvalue multiset equals the clipped bar-length multiset of \(\T_\tau \Pfun_i(F)\) (Appendix~H). Log either the eigenvalue list or a content hash under \texttt{Lambda\_len}.

\paragraph{Minimal reproducibility checklist.}
\begin{enumerate}[leftmargin=1.25em]
\item Preserve \texttt{run.yaml} and all emitted \texttt{bars/spec/ext/phi/Lambda\_len} files (JSON or HDF5).
\item Confirm \(\alpha>0\) (default \(\alpha=1\)) and field \(k\) are consistent across files.
\item Verify content hashes and all cross-links resolve; each artifact should carry \texttt{meta.links.run\_id}
and \texttt{run\_yaml\_hash}.
\item Check that \texttt{pipeline.metric} matches \texttt{thresholds.tol.distance} (exactly one of
\texttt{interleaving}/\texttt{bottleneck}), and that norms/tolerances are consistent.
\item Verify declared \textbf{windows} and that \texttt{coverage\_check.*} is true.
\item Verify eigenvalue \emph{order} metadata: \texttt{spec.meta.order}=\texttt{"ascending"},
\texttt{spec.meta.sorted}=true, and HDF5 eigen arrays are non-decreasing.
\item For Dirichlet Laplacians, confirm \(\lambda_{\min}>0\).
\item Ensure each recorded \(\theta\) used for \(N_\theta\) lies within \texttt{spec.meta.window.range}; recompute spectral indicators using recorded norm/tolerance/solver settings; check slack \(\ge 0\).
\item Re-evaluate \(\phi_{i,\tau}\) under the scope policy; confirm \((\mu,\nu)\) match generic-fiber counts; check \texttt{iso\_tail.passed}.
\item For Ext-tests, verify amplitude \([-1,0]\), the assumptions flags in \texttt{ext.meta.assumptions},
and \(\Ext^1(\mathcal{R}(C_\tau F),k)=0\); check \texttt{persistence.Ext1\_zero}.
\item Validate HDF5 canonicalization: chunk shapes, compression, filters, \texttt{string\_encoding="utf8-fixed"},
and \texttt{track\_times=false}; death is represented via the split float/bool fields.
\item Inspect \texttt{operations} and \texttt{budget}: the sum of per-operation \(\delta\) must match
\texttt{budget.sum\_delta}; compute safety margin and \texttt{gap\_tau}; verify \texttt{gate.accept}.
\item Record container details (image \& digest) and timing; if unavailable, capture OS and library versions precisely.
\end{enumerate}

\medskip
\noindent\textbf{Outcome.}
The versioned schemas above—now with (i) overlap checks for gluing, (ii) a windowed length spectrum audit,
(iii) canonical spectral policy (\texttt{order="ascending"}, \texttt{norm="op"|"fro"}, spectral bounds),
(iv) δ-ledger extensions with \texttt{gap\_tau}, and (v) HDF5 canonicalization and fixed-length UTF-8 strings—
are sufficient to regenerate all figures and claims in the main text from first principles,
within the constructible regime and under the filtered-colimit policy, while making accept/reject criteria explicit and auditable.
No further supplementation is required for operational deployment or third-party review.



% =========================
\section*{Appendix H. Betti Integral and Finite \texorpdfstring{$\tau$}{tau}-Events (Reinforced)}
% =========================
\phantomsection
\addcontentsline{toc}{section}{Appendix H. Betti Integral and Finite $\tau$-Events}
\refstepcounter{section}
\label{H:betti-integral}

\paragraph{Standing conventions.}
We work over a field $k$. All persistence modules are constructible (locally finite on bounded windows); filtered colimits, when used, are taken under the scope policy of Appendix~A, Remark~\ref{A:rk:filtered-colimits}.
Endpoint conventions follow Appendix~A, Remark~\ref{A:rk:endpoints}; we use half-open bars $[b,d)$ (any consistent choice is immaterial below).
Global conventions: $\Ext^1$-tests are always against $k[0]$ (we write $\Ext^1(\mathcal{R}(C_\tau F),k)=0$, with $C_\tau$ understood up to f.q.i.\ on $\Ho(\mathsf{FiltCh}(k))$); the \emph{energy exponent} satisfies $\alpha>0$ (default $\alpha=1$).
Tilde references and type dashes (Type I--II / Type III / Type IV) are used uniformly.

\subsection*{H.1. Betti curves and the Betti integral}
Let $F$ be a filtered chain complex (or a filtered object realizing a persistence module) with degree-$i$ persistence module $\mathbf{P}_i(F)$.
Write its barcode as a locally finite multiset
\[
\mathbf{P}_i(F)\ \cong\ \bigoplus_{I\in \mathcal{B}_i(F)} I^{\oplus m(I)},\qquad I=[b,d)\ \ \text{with } d\in\mathbb{R}\cup\{\infty\},\ m(I)\in\mathbb{Z}_{\ge 1}.
\]
Define the Betti curve $\beta_i(t):=\dim_k H_i(F^t)$ and the \emph{Betti integral} up to $\tau\ge 0$ by
\[
\mathrm{PE}_i^{\le \tau}(F)\ :=\ \int_{0}^{\tau} \beta_i(t)\,dt.
\]
Under constructibility, $\beta_i$ is right-continuous and piecewise constant, and on any bounded window only finitely many bars meet.

\begin{theorem}[Betti integral = clipped barcode mass]\label{H:thm:betti-integral}
For every $\tau\ge 0$,
\[
\mathrm{PE}_i^{\le \tau}(F)\ =\ \sum_{I\in \mathcal{B}_i(F)} m(I)\cdot \lambda\!\left(I\cap[0,\tau]\right),
\]
where $\lambda$ is Lebesgue measure and
\[
\lambda([b,d)\cap[0,\tau])=\max\{0,\min\{d,\tau\}-\max\{b,0\}\}\n\quad(\text{with }\min\{\infty,\tau\}=\tau).
\]
In particular, an infinite bar alive at $0$ contributes its clipped length $\tau$.
\end{theorem}

\begin{proof}
By local finiteness, for each bounded window $[0,\tau]$ only finitely many bars $I$ intersect the window, so
\[
f(t)\ :=\ \sum_{I\in\mathcal{B}_i(F)} m(I)\,\mathbf{1}_I(t)
\]
is a nonnegative measurable function on $[0,\tau]$ given by a finite sum. With the half-open convention $[b,d)$, for all $t$ away from event times (births $b$ and deaths $d$) we have
\(\n\beta_i(t)=f(t)\n\)
and at event times $\beta_i$ is right-continuous, hence equal to $f$ almost everywhere.
Thus, by Tonelli/Fubini on bounded windows (finite sum of nonnegative measurable functions),
\[
\int_{0}^{\tau}\!\beta_i(t)\,dt\ =\int_{0}^{\tau}\!f(t)\,dt\n=\sum_{I} m(I)\int_{0}^{\tau}\!\mathbf{1}_{I}(t)\,dt\n=\sum_{I} m(I)\,\lambda\!\left(I\cap[0,\tau]\right).
\]
\end{proof}

\begin{corollary}[Monotonicity, (a.e.) derivative, piecewise linearity]\label{H:cor:pl}
The map $\tau\mapsto \mathrm{PE}_i^{\le \tau}(F)$ is nondecreasing, continuous, and piecewise linear on every bounded interval. Its derivative satisfies
\[
\frac{d}{d\tau}\,\mathrm{PE}_i^{\le \tau}(F)=\beta_i(\tau)\quad\text{for a.e.\ }\tau,
\]
and at event points (births/deaths) the right derivative equals $\beta_i(\tau)$ while the left derivative equals $\beta_i(\tau-)$. All breakpoints on $[0,\tau_0]$ lie in $\{0,\tau_0\}\cup\{b\in[0,\tau_0]\}\cup\{d\in[0,\tau_0]\}$.
\end{corollary}

\begin{remark}[Endpoint and baseline conventions]\label{H:rk:endpoints}
Changing open/closed endpoint conventions modifies $\beta_i$ only on a set of measure zero; the integral and the breakpoint set remain unchanged. The baseline $0$ is a reference; negative births are allowed and handled by intersecting $I$ with $[0,\tau]$.
\end{remark}

\begin{remark}[Energy exponent and $\alpha$-Betti integral]\label{H:rk:alpha}
For $\alpha>0$, define the $\alpha$-Betti integral up to $\tau\ge 0$ by
\[
\mathrm{PE}_{i,\alpha}^{\le \tau}(F)\ :=\ \int_{0}^{\tau} \bigl(\beta_i(t)\bigr)^{\alpha}\,dt.
\]
On each component of $[0,\tau]$ between consecutive event times, $\beta_i$ is constant, hence $\mathrm{PE}_{i,\alpha}^{\le \tau}$ is still continuous, nondecreasing, and piecewise linear in $\tau$ (with slope $\bigl(\beta_i(\tau)\bigr)^{\alpha}$ on right-open pieces). The case $\alpha=1$ recovers Theorem~\ref{H:thm:betti-integral}. For $\alpha\neq 1$ there is no direct ``clipped mass'' formula, but all algorithmic and verification statements below remain valid verbatim (replace $\beta_i$ by $\beta_i^{\alpha}$ when computing segment slopes).
\end{remark}

\subsection*{H.2. Finite \texorpdfstring{$\tau$}{tau}-events and finite checking sets}
Fix $\tau_0>0$ and define the finite $\tau$-event set
\[
\mathsf{Ev}_i(F;\tau_0)\ :=\ \{0,\tau_0\}\ \cup\ \bigl(\{b\mid [b,d)\in\mathcal{B}_i(F)\}\cap[0,\tau_0]\bigr)\ \cup\ \bigl(\{d\mid [b,d)\in\mathcal{B}_i(F)\}\cap[0,\tau_0]\bigr).
\]
By constructibility, $\mathsf{Ev}_i(F;\tau_0)$ is finite.

\begin{proposition}[Finite checking set]\label{H:prop:finite-check}
Let $g:[0,\tau_0]\to\mathbb{R}$ be continuous and affine on each connected component of $[0,\tau_0]\setminus \mathsf{Ev}_i(F;\tau_0)$ (e.g.\ a piecewise linear benchmark with breakpoints in $\mathsf{Ev}_i$). Then, for either inequality direction,
\[
\mathrm{PE}_i^{\le \tau}(F)\ \ge\ g(\tau)\quad(\text{resp.\ }\le)\quad\text{for all }\tau\in[0,\tau_0]
\]
holds if and only if it holds for all $\tau\in \mathsf{Ev}_i(F;\tau_0)$.
\end{proposition}

\begin{proof}
Between consecutive event times, both $\mathrm{PE}_i^{\le \tau}(F)$ and $g(\tau)$ are affine in $\tau$ (Corollary~\ref{H:cor:pl}). Hence their difference $h(\tau):=\mathrm{PE}_i^{\le \tau}(F)-g(\tau)$ is affine on each closed component $J=[u,v]\subset[0,\tau_0]\setminus \mathsf{Ev}_i(F;\tau_0)$.
An affine function on a compact interval attains its extremum at an endpoint, so $h\ge 0$ (resp.\ $h\le 0$) on $J$ iff $h(u)\ge 0$ and $h(v)\ge 0$ (resp.\ $\le 0$). Taking the union over all such $J$ plus the singleton event points yields the claim.
\end{proof}

\begin{remark}[Algorithmic evaluation]\label{H:rk:algo}
Algorithm for evaluating $\mathrm{PE}_{i,\alpha}^{\le \tau}$ on $[0,\tau_0]$:
\begin{enumerate}
  \item Collect all births and deaths intersecting $[0,\tau_0]$; sort to form $\mathsf{Ev}_i(F;\tau_0)=\{0=t_0<t_1<\cdots<t_M=\tau_0\}$.
  \item For each segment $[t_j,t_{j+1})$, compute $c_j:=\beta_i(t)$ for any $t\in[t_j,t_{j+1})$ (e.g.\ by counting bars covering that segment), and set the segment slope $s_j:=c_j^{\alpha}$.
  \item Accumulate linearly: for $\tau\in[t_j,t_{j+1})$,
  \[
  \mathrm{PE}_{i,\alpha}^{\le \tau}\ =\ \sum_{\ell<j} s_\ell\,(t_{\ell+1}-t_\ell)\ +\ s_j\,(\tau-t_j).
  \]
\end{enumerate}
Complexity: forming/sorting $\mathsf{Ev}_i$ is $O(M\log M)$ for $M$ events; the linear pass is $O(M)$ time and $O(M)$ space. This supports streaming if events are pre-sorted.
For full reproducibility, record window definition and event counts in the run manifest (Appendix~G).
\end{remark}

\subsection*{H.3. Consequences for shifts, truncations, and window variation}

\paragraph{(i) Equivariance under shifts.}
For $\varepsilon\in\mathbb{R}$, let the shift $S^\varepsilon$ act by $(S^\varepsilon F)^t:=F^{t+\varepsilon}$. Then $\beta_i^{S^\varepsilon}(t)=\beta_i(t+\varepsilon)$ and, for $\sigma\ge 0$,
\[
\mathrm{PE}_{i,\alpha}^{\le \sigma}(S^\varepsilon F)\n=\int_{0}^{\sigma}\bigl(\beta_i(t+\varepsilon)\bigr)^{\alpha}\,dt\n=\int_{\varepsilon}^{\sigma+\varepsilon}\bigl(\beta_i(u)\bigr)^{\alpha}\,du\n=\mathrm{PE}_{i,\alpha}^{\le \sigma+\varepsilon}(F)-\mathrm{PE}_{i,\alpha}^{\le \varepsilon}(F).
\]
Thus $\mathrm{PE}_{i,\alpha}^{\le \sigma}$ is equivariant under simultaneous shifts of both the module and the integration window; it is not invariant if only the module is shifted.

\paragraph{(ii) Truncation monotonicity (deletion-type).}
Let $\mathbf{T}_{\tau'}$ denote the bar-deletion truncation at scale $\tau'>0$ on $\Pers^{\mathrm{ft}}_k$ (Appendix~A). Then, for every $\sigma>0$ and every $\alpha>0$,
\[
\mathrm{PE}_{i,\alpha}^{\le \sigma}\!\big(\mathbf{T}_{\tau'}(\mathbf{P}_i(F))\big)\ \le\ \mathrm{PE}_{i,\alpha}^{\le \sigma}\!\big(\mathbf{P}_i(F)\big),
\]
because $\mathbf{T}_{\tau'}$ deletes precisely the finite bars of length $\le \tau'$ while leaving the others unchanged, and $\beta_i$ decreases pointwise under deletion. Moreover, $\mathbf{T}_{\tau'}$ is $1$-Lipschitz in the interleaving metric (Appendix~A, Proposition~\ref{A:prop:lipschitz}).

\paragraph{(iii) Lipschitz in the window parameter.}
For $0\le s\le \tau$ and $\alpha>0$,
\[
\bigl|\mathrm{PE}_{i,\alpha}^{\le \tau}(F)-\mathrm{PE}_{i,\alpha}^{\le s}(F)\bigr|\n=\int_{s}^{\tau}\bigl(\beta_i(t)\bigr)^{\alpha}\,dt\n\ \le\ (\tau-s)\cdot \sup_{t\in[s,\tau]}\bigl(\beta_i(t)\bigr)^{\alpha},
\]
and the supremum is finite on bounded windows by constructibility. Use the same window policy (MECE, right-open) recorded in Appendix~G for comparability across runs.

\subsection*{H.4. Stability under interleavings and perturbations}
We record a practical bound quantifying stability of $\mathrm{PE}$ under barcode perturbations, sufficient for algorithmic and statistical use on bounded windows.

\begin{proposition}[Perturbation bound on bounded windows]\label{H:prop:bounded-stability}
Let $\tau_0>0$. Suppose two barcodes $\mathcal{B}_i(F)$ and $\mathcal{B}_i(G)$ are $\delta$-matched in the standard bottleneck correspondence: each matched pair $[b,d)\leftrightarrow[b',d')$ satisfies $|b-b'|\le\delta$, $|d-d'|\le\delta$ (with $d=\infty$ allowed), and unmatched bars (if any) have length $\le 2\delta$. Then, for all $\tau\in[0,\tau_0]$ and $\alpha>0$,
\[
\bigl|\mathrm{PE}_{i,\alpha}^{\le \tau}(F)-\mathrm{PE}_{i,\alpha}^{\le \tau}(G)\bigr|\n\ \le\ C_{i,\alpha}(\tau_0,\delta)\cdot \delta,
\]
where one can take
\[
C_{i,\alpha}(\tau_0,\delta)\ :=\ 2\,N_{i}([-\delta,\tau_0+\delta])\cdot \max\bigl\{1,\ \sup_{t\in[-\delta,\tau_0+\delta]}\bigl(\beta_i^F(t)\bigr)^{\alpha-1},\ \sup_{t\in[-\delta,\tau_0+\delta]}\bigl(\beta_i^G(t)\bigr)^{\alpha-1}\bigr\},
\]
and $N_{i}([a,b])$ denotes the number of bars (counting multiplicity) meeting $[a,b]$ in either barcode. In particular, for $\alpha=1$,
\[
\bigl|\mathrm{PE}_{i}^{\le \tau}(F)-\mathrm{PE}_{i}^{\le \tau}(G)\bigr|\ \le\ 2\,\delta\cdot N_{i}([-\delta,\tau_0+\delta]),
\]
which is finite by local finiteness.
\end{proposition}

\begin{proof}[Proof sketch]
On $[0,\tau]$, $\mathrm{PE}_i^{\le \tau}$ equals the sum of clipped lengths of the bars (Theorem~\ref{H:thm:betti-integral}). A $\delta$-perturbation of endpoints changes the clipped length of any matched bar by at most $2\delta$. Unmatched bars have length $\le 2\delta$ and hence contribute at most $2\delta$ to the clipped length. Summing over bars meeting $[-\delta,\tau_0+\delta]$ gives the $\alpha=1$ bound. For $\alpha\neq 1$, use that on each segment where $\beta_i$ is constant $=c$, the slope changes by at most $|c^\alpha-{c'}^\alpha|\le \alpha\max\{c,c'\}^{\alpha-1}|c-c'|$, and $|c-c'|$ is controlled by the number of bar crossings, which is bounded by $N_i([-\delta,\tau_0+\delta])$ on bounded windows. Combining these estimates yields the stated bound.
\end{proof}

\begin{remark}[Interpretation]
Proposition~\ref{H:prop:bounded-stability} gives windowed stability of $\mathrm{PE}$: small bottleneck perturbations of the barcode produce $O(\delta)$ changes in $\mathrm{PE}$, with a constant depending only on the local combinatorics (finite on bounded windows). This is adequate for numerical robustness and cross-run comparisons.
\end{remark}

\subsection*{H.5. Implementation notes and numerics}
For large barcodes, the following practices improve reproducibility and numerical stability:
\begin{enumerate}
  \item Event extraction: derive $\mathsf{Ev}_i(F;\tau_0)$ directly from the barcode; for streamed persistence, emit births and deaths as they occur and maintain a running count $c_j$.
  \item Accumulation: use compensated summation (e.g.\ Kahan) when aggregating segment areas $s_j\,(t_{j+1}-t_j)$ to reduce floating-point error.
  \item Types: store event times as 64-bit floats; store counts $c_j$ as 64-bit integers; compute slopes as double-precision floats.
  \item Idempotence: with the half-open convention $[b,d)$, repeated evaluation on the same event sequence is bitwise deterministic (assuming fixed sort and tie-break rules).
  \item Window policy: record in the manifest (Appendix~G) the baseline, window $[0,\tau_0]$, endpoint convention (right-open), and whether negative births are present.
\end{enumerate}

\subsection*{H.6. Testing and validation}
We recommend the following minimal test battery.
\begin{enumerate}
  \item Synthetic bars: verify Theorem~\ref{H:thm:betti-integral} on barcodes with a mix of finite/infinite bars and negative births. Cross-check numerical integration of $\beta_i$ against clipped-length summation.
  \item Endpoint consistency: switch between $[b,d)$ and $(b,d]$ conventions in a controlled harness and verify identical $\mathrm{PE}_{i,\alpha}^{\le \tau}$ values and identical breakpoint sets.
  \item Shift equivariance: for random $\varepsilon$, check $\mathrm{PE}_{i,\alpha}^{\le \sigma}(S^\varepsilon F)=\mathrm{PE}_{i,\alpha}^{\le \sigma+\varepsilon}(F)-\mathrm{PE}_{i,\alpha}^{\le \varepsilon}(F)$ up to machine precision.
  \item Truncation monotonicity: apply $\mathbf{T}_{\tau'}$ and verify $\mathrm{PE}_{i,\alpha}^{\le \sigma}$ decreases pointwise in $\sigma$.
  \item Stability: simulate $\delta$-perturbations of bar endpoints and confirm the bounds in Proposition~\ref{H:prop:bounded-stability}.
\end{enumerate}

\subsection*{H.7. Variants and generalizations}
\begin{itemize}
  \item Weighted windows: for a nonnegative integrable weight $w\in L^1([0,\tau_0])$, define
  \[
  \mathrm{PE}_{i,\alpha}^{w}(F)\ :=\ \int_{0}^{\tau_0} w(t)\,\bigl(\beta_i(t)\bigr)^{\alpha}\,dt\n  \ =\ \sum_{I} m(I)\int_{I\cap[0,\tau_0]} w(t)\,\bigl(\beta_i(t)\bigr)^{\alpha-1}\,dt,
  \]
  where the second identity holds by the same Tonelli argument as Theorem~\ref{H:thm:betti-integral}. If $w$ is piecewise constant with breaks in $\mathsf{Ev}_i(F;\tau_0)$, finite checking remains valid.
  \item Alternate baselines: replacing $[0,\tau]$ by $[a,b]$ yields
  \(\n  \mathrm{PE}_i^{[a,b]}(F)=\sum_I m(I)\,\lambda(I\cap[a,b]),\n  \)
  with the same properties and algorithms after shifting time by $-a$.
  \item Discrete filtrations: if $t$ ranges over a discrete grid $\{t_0<\cdots<t_M\}$, replace integrals with Riemann sums; all statements adapt with $\lambda$ replaced by counting measure on grid cells.
\end{itemize}

\medskip
\noindent\textbf{Summary.}
The Betti integral equals the clipped barcode mass (Theorem~\ref{H:thm:betti-integral}). Hence $\mathrm{PE}_{i,\alpha}^{\le \tau}$ is continuous, nondecreasing, and piecewise linear, with breakpoints among births/deaths (Corollary~\ref{H:cor:pl}, Remark~\ref{H:rk:alpha}). Any affine-on-components constraint reduces to a finite event set (Proposition~\ref{H:prop:finite-check}). Under shifts, $\mathrm{PE}_{i,\alpha}^{\le \sigma}$ obeys an explicit equivariance formula; under deletion-type truncations it is nonincreasing and enjoys $1$-Lipschitz stability in interleaving distance (Appendix~A, Proposition~\ref{A:prop:lipschitz}). Bounded-window perturbation bounds (Proposition~\ref{H:prop:bounded-stability}) provide practical stability. Implementation, numerical, and testing guidelines (Subsections~H.5--H.6) ensure reproducibility in line with the global scope policy (Appendix~A) and run manifests (Appendix~G).



% =========================
\section*{Appendix I. \texorpdfstring{$\varepsilon$}{epsilon}-Survival Lemma and Grid-to-Continuum [Proof/Spec]}
% =========================
\phantomsection
\addcontentsline{toc}{section}{Appendix I. $\varepsilon$-Survival Lemma and Grid-to-Continuum}
\refstepcounter{section}
\label{I:eps-survival}

\paragraph{Standing conventions.}
We work over a field \(k\). All persistence modules are constructible (locally finite on bounded windows). Any use of filtered colimits follows the scope policy of Appendix~A, Remark~\ref{A:rk:filtered-colimits}. Global conventions: \(\Ext^1\)-tests are always against \(k[0]\) (so we write \(\Ext^1(\mathcal{R}(C_\tau F),k)=0\)); the energy exponent satisfies \(\alpha>0\) (default \(\alpha=1\)); tilde references (\(\text{Appendix}\,\tilde{}\)) and type dashes (Type I--II / Type III / Type IV) are used uniformly. We write \(d_{\mathrm{int}}\) for the interleaving metric (coinciding with the bottleneck distance in the constructible 1D case).

\begin{remark}[Endpoint conventions are immaterial]\label{I:rk:endpoints}
We take intervals half-open \([b,d)\) with \(d\in\mathbb{R}\cup\{\infty\}\). Any consistent open/closed endpoint choice yields the same clipped lengths and event sets; all statements below are invariant under this choice. Use the same right-open convention as Appendix~G for run logs.
\end{remark}

\begin{remark}[Relation to cropping in Chapter~2]\label{I:rk:crop-ref}
The window clipping functor \(\mathbf{W}_{\le\tau}\) introduced below implements the restriction to the time window \([0,\tau]\) followed by extension by zero. This is equivalent in purpose to the cropping operator \(\Crop_W\) from Chapter~2 (see also Chapter~2, Definition (Crop)). We keep the present notation to emphasize the functorial construction via restriction and \(0\)-extension.
\end{remark}

\subsection*{I.1. Window clipping, nonexpansivity, and \texorpdfstring{$\varepsilon$}{epsilon}-survival}

Let \(\Pers^{\mathrm{ft}}_k\) be the category of constructible persistence modules on \((\mathbb{R},\le)\). For \(\tau\ge 0\), let
\[
i_{\le\tau}:\ ([0,\tau],\le)\hookrightarrow(\mathbb{R},\le)
\]
be the fully faithful inclusion of posets, and write \(i_{\le\tau}^{\ast}\) for restriction along \(i_{\le\tau}\).
We identify modules on \([0,\tau]\) with modules on \(\mathbb{R}\) supported in \([0,\tau]\) via \emph{extension by zero} along \(i_{\le\tau}\); we denote this left Kan extension by \((i_{\le\tau})^{0}_!\) (``\(0\)-extension''). Define the \emph{window clip} endofunctor
\[
\mathbf{W}_{\le \tau}\ :=\ (i_{\le\tau})^{0}_!\circ i_{\le\tau}^{\ast}:\ \Pers^{\mathrm{ft}}_k\longrightarrow \Pers^{\mathrm{ft}}_k.
\]
In barcode terms, this coincides with interval intersection \(I\mapsto I\cap[0,\tau]\) (discarding empties). For a bar \(I=[b,d)\) its clipped length is
\[
\ell_{[0,\tau]}(I):=\lambda\!\left(I\cap[0,\tau]\right)=\max\{0,\min\{d,\tau\}-\max\{b,0\}\},
\]
with \(\min\{\infty,\tau\}=\tau\). Since only finitely many bars intersect any bounded window, \(\mathbf{W}_{\le\tau}\) preserves constructibility.

\begin{lemma}[Clipping is \(1\)-Lipschitz]\label{I:lem:clip-1lip}
For all \(M,N\in\Pers^{\mathrm{ft}}_k\) and \(\tau\ge 0\),
\[
d_{\mathrm{int}}\bigl(\mathbf{W}_{\le \tau}M,\ \mathbf{W}_{\le \tau}N\bigr)\ \le\ d_{\mathrm{int}}(M,N).
\]
\end{lemma}

\begin{proof}
Let \(d_{\mathrm{int}}(M,N)\le\varepsilon\) be witnessed by an \(\varepsilon\)-interleaving \((\varphi,\psi)\) with \(\varphi:M\to S^\varepsilon N\) and \(\psi:N\to S^\varepsilon M\) satisfying the standard triangle identities. Restricting along \(i_{\le\tau}\) yields an \(\varepsilon\)-interleaving \((i_{\le\tau}^\ast\varphi,i_{\le\tau}^\ast\psi)\) between \(i_{\le\tau}^\ast M\) and \(i_{\le\tau}^\ast N\), because shifts commute with restriction and the naturality squares and triangle identities restrict pointwise on \([0,\tau]\). Composing with \(0\)-extension \((i_{\le\tau})^{0}_!\) produces an \(\varepsilon\)-interleaving between \(\mathbf{W}_{\le\tau}M\) and \(\mathbf{W}_{\le\tau}N\). Thus \(d_{\mathrm{int}}(\mathbf{W}_{\le\tau}M,\mathbf{W}_{\le\tau}N)\le\varepsilon\). Taking the infimum over \(\varepsilon\) gives the claim.
Equivalently in barcode terms (constructible case), the bottleneck matching cost does not increase if one first clips both barcodes to \([0,\tau]\) and then matches, or matches first and clips afterwards.
\end{proof}

\begin{remark}[Shift commutation with clipping]\label{I:rk:shift-clip}
For every \(\varepsilon\ge 0\) there is a canonical isomorphism
\[
S^\varepsilon\circ \mathbf{W}_{\le\tau}\ \cong\ \mathbf{W}_{\le\tau}\circ S^\varepsilon,
\]
since restriction along \(i_{\le\tau}\) and \(0\)-extension commute with the time-shift endofunctors. In particular, any \(\varepsilon\)-interleaving data transport through \(\mathbf{W}_{\le\tau}\) via these identifications.
\end{remark}

\begin{lemma}[\texorpdfstring{$\varepsilon$}{epsilon}-survival lemma]\label{I:lem:survive}
Let \(M,N\in\Pers^{\mathrm{ft}}_k\) with \(d_{\mathrm{int}}(M,N)\le\varepsilon\). Fix \(\tau_0>0\).
\begin{enumerate}\itemsep0.25em
\item \textbf{Quantitative form.} For any bar-matching that witnesses \(d_{\mathrm{int}}(M,N)\le\varepsilon\) (not necessarily unique), if a bar \(I\) of \(M\) is matched to a bar \(J\) of \(N\), then
\[
\ell_{[0,\tau_0]}(J)\ \ge\ \max\bigl\{\,\ell_{[0,\tau_0]}(I)-2\varepsilon,\ 0\,\bigr\}.
\]
\item \textbf{Nonvanishing form.} If \(\ell_{[0,\tau_0]}(I)>2\varepsilon\) for some bar \(I\) in \(M\), then its matched partner \(J\) satisfies \(\ell_{[0,\tau_0]}(J)>0\). Consequently \(\mathbf{W}_{\le \tau_0}N\neq 0\).
\item \textbf{Multiplicity lower bound.} If at least \(r\) bars of \(M\) have clipped length \(>\!2\varepsilon\), then at least \(r\) bars of \(N\) remain nonzero after clipping to \([0,\tau_0]\) (counted with multiplicity).
\end{enumerate}
\end{lemma}

\begin{proof}
In the constructible 1D case the interleaving distance equals the bottleneck distance. Under an \(\varepsilon\)-matching, endpoints of matched bars move by at most \(\varepsilon\). Hence the intersection length with \([0,\tau_0]\) can shrink by at most \(\varepsilon\) at each end, giving (1). Points matched to the diagonal have zero clipped length and thus cannot be matched to bars with clipped length \(>\!2\varepsilon\); (2) and (3) follow immediately.
\end{proof}

\begin{remark}[Notation]\label{I:rk:notation}
We reserve \(\mathbf{T}_\tau\) for the Serre reflection that removes bars of length \(\le\tau\) (Appendix~A, Theorem~\ref{A:thm:localization}, and Proposition~\ref{A:prop:lipschitz} for its \(1\)-Lipschitz property). Window clipping is denoted \(\mathbf{W}_{\le\tau}\). The present statements concern \(\mathbf{W}_{\le\tau}\). Appendix~H relates clipping to the Betti integral.
\end{remark}

\subsection*{I.2. Grid-to-continuum transfer}
Let \(F\) be a filtered object with degree-\(i\) persistence \(\mathbf{P}_i(F)\).
Suppose \(F_h\) is a discretization at mesh \(h\) with \(\mathbf{P}_i(F_h)\) and a certified bound
\[
d_{\mathrm{int}}\bigl(\mathbf{P}_i(F_h),\ \mathbf{P}_i(F)\bigr)\ \le\ \varepsilon(h).
\]

\begin{theorem}[Grid-to-continuum survival]\label{I:thm:g2c}
Fix \(\tau_0>0\) and \(r\in\mathbb{Z}_{\ge 1}\).
If \(\mathbf{W}_{\le \tau_0}\mathbf{P}_i(F_h)\) contains at least \(r\) bars of clipped length \(\ge 2\varepsilon(h)+\eta\) for some margin \(\eta>0\), then
\[
\mathbf{W}_{\le \tau_0}\mathbf{P}_i(F)\quad\text{contains at least \(r\) nonzero bars (with multiplicity), each of clipped length }\ge \eta.
\]
In particular, nonvanishing on the grid with margin \(2\varepsilon(h)+\eta\) implies nonvanishing in the continuum window \([0,\tau_0]\) (no false negatives). Record \(\varepsilon(h)\), \(\tau_0\), and the event counts per window in the run manifest (Appendix~G) for reproducibility.
\end{theorem}

\begin{proof}
Apply Lemma~\ref{I:lem:survive} with \(M=\mathbf{P}_i(F_h)\), \(N=\mathbf{P}_i(F)\), and \(\varepsilon=\varepsilon(h)\). Bars with clipped length \(>\!2\varepsilon(h)\) cannot be matched to the diagonal; at least \(r\) such bars survive with length \(\ge\eta\), counted with multiplicity.
\end{proof}

\subsection*{I.3. Variants, sharpness, and compatibility}
\begin{itemize}\itemsep0.25em
\item \emph{Sharpness.} The constant \(2\) is optimal: a bar with \(\ell_{[0,\tau_0]}=2\varepsilon\) can be shifted by \(\varepsilon\) at each endpoint so that its clip collapses to length \(0\).
\item \emph{Compatibility with functors.} By Lemma~\ref{I:lem:clip-1lip} and Appendix~A, Proposition~\ref{A:prop:lipschitz}, composition with \(1\)-Lipschitz endofunctors (e.g.\ the reflections \(\mathbf{T}_\tau\); mirror/transfer maps under their stated [Spec] hypotheses) preserves the survival inequalities.
\item \emph{After-collapse pipeline.} Operationally we compare on the B-side after applying \(\mathbf{T}_\tau\) (``after-collapse''). Since \(\mathbf{T}_\tau\) is \(1\)-Lipschitz, applying \(\mathbf{T}_\tau\) before or after \(\mathbf{W}_{\le\tau_0}\) preserves the bounds of Lemma~\ref{I:lem:survive} and Theorem~\ref{I:thm:g2c}.
\item \emph{Towers.} In a tower \(F_n\to F_\infty\), if \(\varepsilon_n\to 0\) with \(d_{\mathrm{int}}(\mathbf{P}_i(F_n),\mathbf{P}_i(F_\infty))\le \varepsilon_n\), then any uniform margin \(>\!0\) propagates from grid to continuum by Theorem~\ref{I:thm:g2c} (cf.\ Appendix~D for tower diagnostics).
\end{itemize}

\subsection*{I.4. Implementation schema and reproducibility (run manifest)}
The following fields and checks should be present in the run manifest to certify a grid-to-continuum survival claim at window \([0,\tau_0]\):
\begin{itemize}
\item \textbf{Metric/tolerance.} Record the metric used and the certified bound: \texttt{pipeline.metric} and \texttt{thresholds.tol.distance} (interleaving or bottleneck, made explicit).
\item \textbf{Sampling/bandlimit.} Include \texttt{sampling.dt}, \texttt{bandlimit}, \texttt{nyquist\_check}, \texttt{eps\_interleave\_max} (or \texttt{eps\_cont\_bound}) used for \(\varepsilon(h)\).
\item \textbf{Window coverage.} Set \texttt{coverage\_check.domain\_window\_covers\_bars: true} to assert that the declared window matches the computation window.
\item \textbf{Operations log.} Add an \texttt{operations} entry for the ``epsilon-continuation'' step with the realized \(\varepsilon\), and aggregate into \texttt{delta.total}.
\item \textbf{Event counts.} Record the number of births/deaths and bar counts per window in the persistence band, after clipping.
\end{itemize}

A minimal schema example:
\begin{verbatim}
pipeline:
  metric: interleaving          # or 'bottleneck' with justification
  stages:
    - compute_persistence
    - clip_window               # W_{≤τ}
    - epsilon_continuation      # certify ε(h)
    - collapse_short_bars       # optional T_τ
thresholds:
  tol:
    distance: 0.0125            # certified ε(h)
  safety_margin:
    bar_clip_len_min: 2*ε + η   # enforced margin on grid side
sampling:
  dt: 0.001
  bandlimit: 200.0              # Hz or domain-appropriate
  nyquist_check: true
  eps_interleave_max: 0.0125    # equals thresholds.tol.distance
window:
  tau0: 1.5
  coverage_check:
    domain_window_covers_bars: true
operations:
  - name: epsilon-continuation
    epsilon: 0.0125
    notes: "Certified via stability bound / a priori model"
delta:
  total: 0.0125
persistence:
  degree: 1
  events:
    window_[0,1.5]:
      births: 7
      deaths: 6
      bars_total: 9
      bars_clip_len_ge_2eps_plus_eta: 3
assertions:
  grid_to_continuum_survival:
    r: 3
    eta: 0.05
    satisfied: true
\end{verbatim}

\subsection*{I.5. Minimal tests and regression suite}
\begin{itemize}\itemsep0.25em
\item \textbf{Sharpness test.} Single-bar synthetic: \(I=[0,2\varepsilon)\). Shift endpoints by \(\varepsilon\) to cause collapse of \(\ell_{[0,\tau_0]}\) when \(\tau_0\ge 2\varepsilon\). Confirms optimality of the constant \(2\).
\item \textbf{Endpoint policy.} Verify the half-open/right-open convention across the pipeline (clip lengths, event attribution, and logs) matches Appendix~G.
\item \textbf{Tower regression.} Sequence \(h_k\to 0\) with \(\varepsilon(h_k)\to 0\). Bars on the grid with a uniform margin persist in the continuum; add a TCI regression to check stability of counts and lengths after clipping.
\item \textbf{Operator separation.} Unit tests that distinguish \(\mathbf{W}_{\le\tau}\) (window clipping) from \(\mathbf{T}_\tau\) (short-bar removal), including order-of-application invariance of survival conclusions due to \(1\)-Lipschitzness.
\end{itemize}

\subsection*{I.6. Formalization stubs (Lean/Coq)}
The following stubs (cf.\ Appendix~F, \texttt{AK.Core/Tower}) codify the key interfaces. We keep them minimal to avoid committing to a proof assistant dialect here.

\begin{verbatim}
-- Lean-style pseudocode

namespace Pers

structure PersModule := -- constructible persistence module over ℝ
  (rank_on_window : set ℝ → ℕ)
  -- … standard fields/constructibility witnesses …

def W_le (τ : ℝ≥0) (M : PersModule) : PersModule := -- window clipping
  -- restriction to [0, τ] and 0-extension
  sorry

def interleaving_dist (M N : PersModule) : ℝ≥0∞ := sorry

theorem W_le_lipschitz (τ : ℝ≥0) (M N : PersModule) :
  interleaving_dist (W_le τ M) (W_le τ N) ≤ interleaving_dist M N := sorry

structure Discretization :=
  (F_h : Type) -- placeholder for filtered object at mesh h
  (P_i : F_h → PersModule)
  (eps : ℝ≥0) -- ε(h)
  -- certificate data …

theorem eps_survival
  (ε : ℝ≥0) {M N : PersModule}
  (hMN : interleaving_dist M N ≤ ε)
  (τ0 : ℝ≥0) :
  -- if bar I in M has clipped length > 2ε on [0, τ0], then its match in N is nonzero
  sorry

theorem grid_to_continuum
  (τ0 : ℝ≥0) (r : ℕ) (η ε : ℝ≥0)
  {M N : PersModule}
  (hMN : interleaving_dist M N ≤ ε)
  (grid_margin : -- at least r bars in W_le τ0 M have length ≥ 2ε + η
    sorry) :
  -- conclude: at least r bars in W_le τ0 N have length ≥ η
  sorry

end Pers
\end{verbatim}

\paragraph{Reference line for Chapter 4.5.}
\emph{“By Appendix~I (the \(\varepsilon\)-Survival Lemma, Lemma~\ref{I:lem:survive}), any grid-detected bar in the window \([0,\tau_0]\) with clipped length \(>2\varepsilon(h)\) persists in the continuum window; we therefore certify the claim at scale \(\tau_0\) from the discretization.”}

\medskip
\noindent\textbf{Summary.}
Clipping is restriction to \([0,\tau]\) followed by extension by zero to \(\mathbb{R}\), hence an endofunctor \(\mathbf{W}_{\le\tau}\) on \(\Pers^{\mathrm{ft}}_k\); it preserves constructibility and is \(1\)-Lipschitz (Lemma~\ref{I:lem:clip-1lip}). Under an \(\varepsilon\)-interleaving, clipped lengths degrade by at most \(2\varepsilon\); hence grid detections with margin \(>\!2\varepsilon\) transfer to continuum without false negatives (Lemma~\ref{I:lem:survive}, Theorem~\ref{I:thm:g2c}). These are consistent with the global scope policy and the finite-event structure of Appendices~A and~H. For implementation, record the metric, certified \(\varepsilon(h)\), window coverage, and event counts in the run manifest, and include minimal sharpness/tower tests and optional after-collapse comparisons via the \(1\)-Lipschitz reflection \(\mathbf{T}_\tau\).



% =========================
\section*{Appendix J. Calculus of $\mu,\nu$ [Proof + Stability Bands + Window Pasting]}
% =========================
\phantomsection
\addcontentsline{toc}{section}{Appendix J. Calculus of $\mu,\nu$}
\refstepcounter{section}
\label{J:calc}

\paragraph{Standing conventions.}
We work over a field \(k\).
All persistence modules are constructible (locally finite on bounded windows).
Filtered colimits are computed in the functor category \([\mathbb{R},\mathsf{Vect}_k]\) under the scope policy of Appendix~A, Remark~\ref{A:rk:filtered-colimits}, and then (when stated) returned to the constructible range.
The reflection \(\mathbf{T}_\tau\dashv \iota_\tau\) is exact and \(1\)-Lipschitz (Appendix~A, Theorem~\ref{A:thm:localization} and Proposition~\ref{A:prop:lipschitz}).
Global conventions: \(\Ext^1\) is always against \(k[0]\); the energy exponent \(\alpha>0\) (default \(\alpha=1\)).
All window policies follow the MECE (mutually exclusive–collectively exhaustive), right-open convention used throughout (Appendix~G).

\paragraph{Setup and notation.}
Fix \(i\in\mathbb{Z}\).
Let \(F=(F_n)_{n\in I}\) be a directed system (``tower'') of filtered objects for which \(\mathbf{P}_i(F_n)\in\Pers^{\mathrm{ft}}_k\), and let \(F_\infty\) be its colimit with cocone maps \(F_n\to F_\infty\).
For \(\tau\ge 0\) consider the \emph{comparison map} in the functor category
\begin{equation}\label{J:eq:phi}
\phi_{i,\tau}(F):\quad
\varinjlim_{n}\ \mathbf{T}_\tau\big(\mathbf{P}_i(F_n)\big)\ \longrightarrow\ \mathbf{T}_\tau\big(\mathbf{P}_i(F_\infty)\big).
\end{equation}
Define
\begin{equation}\label{J:eq:mu-nu}
\mu_{i,\tau}(F)\ :=\ \dim_k \ker \phi_{i,\tau}(F),\qquad
\nu_{i,\tau}(F)\ :=\ \dim_k \operatorname{coker} \phi_{i,\tau}(F),
\end{equation}
where \(\dim_k\) denotes the \emph{generic fiber} dimension in \(\Pers^{\mathrm{ft}}_k\) (i.e.\ the stabilized right-tail dimension \(\lim_{t\to+\infty}\dim_k(-)(t)\), equivalently the multiplicity of \(I[0,\infty)\) after applying \(\mathbf{T}_\tau\); cf.\ Appendix~D, Remark~\ref{rem:D-generic-dim}).\footnote{For a constructible module \(M\), the generic fiber dimension equals \(\dim_k M(t)\) for all sufficiently large \(t\), which stabilizes. After applying \(\mathbf{T}_\tau\), it coincides with the multiplicity of \(I[0,\infty)\) summands. Kernels/cokernels are taken in \(\Pers^{\mathrm{ft}}_k\), where they decompose into finite direct sums of interval modules.}
We also write the \emph{totals}
\begin{equation}\label{J:eq:totals}
\muc(F)\ :=\ \sum_{i}\mu_{i,\tau}(F),\qquad
\nuc(F)\ :=\ \sum_{i}\nu_{i,\tau}(F),
\end{equation}
which are finite because the complexes are bounded in homological degrees (constructible range).
All quantities depend on \(\tau\); no general monotonicity in \(\tau\) is asserted.

\subsection*{J.1. Subadditivity under composition}
We formalize ``composition of towers'' via morphisms of directed systems.

\begin{definition}[Morphisms of towers]
A morphism \(u:F\to G\) consists of maps \(u_n:F_n\to G_n\) commuting with the structure maps (in the index category) and inducing a canonical map on colimits \(u_\infty:F_\infty\to G_\infty\).
Given \(u:F\to G\) and \(v:G\to H\), the composite \(v\circ u:F\to H\) is defined degreewise.
\end{definition}

\begin{lemma}[Functoriality of comparison maps]\label{J:lem:functoriality}
Under the scope policy, each morphism \(u:F\to G\) induces a (canonical) morphism of comparison maps
\[
\phi_{i,\tau}(u):\ \phi_{i,\tau}(F)\ \Longrightarrow\ \phi_{i,\tau}(G),
\]
natural in both \(i\) and \(\tau\) (here \(\Longrightarrow\) indicates a canonical comparison morphism, natural in \(i,\tau\)).
Moreover, for composable \(u:F\to G\), \(v:G\to H\),
\[
\phi_{i,\tau}(v\circ u)\ =\ \phi_{i,\tau}(v)\ \circ\ \phi_{i,\tau}(u).
\]
\end{lemma}

\begin{proof}[Proof sketch]
Apply \(\mathbf{P}_i\), then \(\mathbf{T}_\tau\), to the maps on systems and pass to filtered colimits in \([\mathbb{R},\mathsf{Vect}_k]\); exactness of \(\mathbf{T}_\tau\) and functoriality of colimits yield naturality and compatibility with composition.
\end{proof}

\begin{theorem}[Subadditivity under composition]\label{J:thm:subadd}
Let \(u:F\to G\) and \(v:G\to H\) be morphisms of towers and fix \(\tau\ge 0\).
Then
\[
\mu_{i,\tau}(v\circ u)\ \le\ \mu_{i,\tau}(u)\ +\ \mu_{i,\tau}(v),\qquad\n\nu_{i,\tau}(v\circ u)\ \le\ \nu_{i,\tau}(u)\ +\ \nu_{i,\tau}(v).
\]
\end{theorem}

\begin{proof}
By Lemma~\ref{J:lem:functoriality} we reduce to linear estimates on the generic fiber.
In \([\mathbb{R},\mathsf{Vect}_k]\), kernels and cokernels are computed pointwise; applying \(\mathbf{T}_\tau\) preserves exactness.
In the constructible range the (right-tail) generic fiber dimension equals the stabilized pointwise dimension.
Thus, evaluating the comparison maps at any sufficiently large parameter \(t\) (where stabilization holds) and writing \(X_t\xrightarrow{f_t}Y_t\xrightarrow{g_t}Z_t\) for the induced linear maps, we have the elementary inequalities
\[
\dim\ker(g_t\!\circ\! f_t)\ \le\ \dim\ker f_t\ +\ \dim\ker g_t,\qquad\n\dim\operatorname{coker}(g_t\!\circ\! f_t)\ \le\ \dim\operatorname{coker} f_t\ +\ \dim\ker g_t,
\]
and taking these stabilized values yields the stated bounds for the generic fiber dimensions in~\eqref{J:eq:mu-nu}.\footnote{Finite-dimensional linear algebra: for \(X\xrightarrow{f}Y\xrightarrow{g}Z\), one has \(\ker(g\!\circ\! f)\subseteq \ker f \oplus f^{-1}(\ker g)\) and \(\operatorname{coker}(g\!\circ\! f)\) controlled by \(\operatorname{coker} f\) and \(\ker g\), giving the stated subadditivity bounds on dimensions.}
\end{proof}

\subsection*{J.2. Additivity under finite direct sums}

\begin{proposition}[Direct-sum additivity]\label{J:prop:sum}
Let \(F=F^{(1)}\oplus F^{(2)}\) be the levelwise direct sum of towers (same index category) and similarly for the colimit.
Then for every \(\tau\ge 0\),
\[
\mu_{i,\tau}(F)\ =\ \mu_{i,\tau}\!\big(F^{(1)}\big)\ +\ \mu_{i,\tau}\!\big(F^{(2)}\big),\qquad\n\nu_{i,\tau}(F)\ =\ \nu_{i,\tau}\!\big(F^{(1)}\big)\ +\ \nu_{i,\tau}\!\big(F^{(2)}\big).
\]
In particular, the totals satisfy
\[
\muc(F)\ =\ \muc\!\big(F^{(1)}\big)\ +\ \muc\!\big(F^{(2)}\big),\qquad\n\nuc(F)\ =\ \nuc\!\big(F^{(1)}\big)\ +\ \nuc\!\big(F^{(2)}\big).
\]
\end{proposition}

\begin{proof}
\(\mathbf{P}_i\) and \(\mathbf{T}_\tau\) preserve finite direct sums; filtered colimits \emph{commute with finite direct sums} in \([\mathbb{R},\mathsf{Vect}_k]\).
Hence \(\phi_{i,\tau}(F)\) is block-diagonal with blocks \(\phi_{i,\tau}\!\big(F^{(1)}\big)\) and \(\phi_{i,\tau}\!\big(F^{(2)}\big)\), and kernels/cokernels (hence generic fiber dimensions) add; summing over \(i\) gives the identities for \(\muc,\nuc\).
\end{proof}

\subsection*{J.3. Cofinal invariance}

\begin{definition}[Cofinal subindexing]
Let \(I\) be the directed index category for \(F\).
A full subcategory \(J\subset I\) is cofinal if for every \(i\in I\) there exists \(j\in J\) with a morphism \(i\to j\).
The restricted tower \(F|_J\) has the same colimit as \(F\) in \([\mathbb{R},\mathsf{Vect}_k]\).
\end{definition}

\begin{theorem}[Cofinal invariance]\label{J:thm:cofinal}
Let \(J\subset I\) be cofinal. Then for all \(\tau\ge 0\),
\[
\mu_{i,\tau}\big(F|_J\big)=\mu_{i,\tau}(F),\qquad\n\nu_{i,\tau}\big(F|_J\big)=\nu_{i,\tau}(F),
\]
hence also \(\muc(F|_J)=\muc(F)\) and \(\nuc(F|_J)=\nuc(F)\).
In particular, passing to any cofinal tail of a sequential tower leaves \((\muc,\nuc)\) unchanged.
\end{theorem}

\begin{proof}
Cofinal restriction does not change colimits in \([\mathbb{R},\mathsf{Vect}_k]\).
Thus the source and target of \(\phi_{i,\tau}\) (Eq.~\eqref{J:eq:phi}) remain unchanged, as does \(\phi_{i,\tau}\) itself; kernels/cokernels (hence \(\mu_{i,\tau},\nu_{i,\tau}\), and the totals) agree.
\end{proof}

\subsection*{J.4. Consequences and remarks}
\begin{itemize}\itemsep0.25em
  \item \emph{Triangle-type bounds.} Combining Theorem~\ref{J:thm:subadd} across a chain of morphisms yields \((\muc,\nuc)\)-control along multi-stage pipelines, useful when towers factor through preprocessing steps (e.g.\ mirror/transfer under the [Spec] hypotheses of Appendix~L).
  \item \emph{Finite decomposition.} Iterating Proposition~\ref{J:prop:sum} shows that \((\muc,\nuc)\) is additive across any finite direct-sum decomposition of towers, enabling modular bookkeeping.
  \item \emph{Cofinal tails for convergence.} Theorem~\ref{J:thm:cofinal} justifies replacing a tower by any convenient tail when verifying sufficient conditions for \((\muc,\nuc)=(0,0)\) (Appendix~D, §D.3).
  \item \emph{No general monotonicity in \(\tau\).} Absent additional hypotheses, \(\tau\mapsto(\mu_{i,\tau},\nu_{i,\tau})\) (hence \(\tau\mapsto(\muc,\nuc)\)) need not be monotone; only the sufficient conditions of Appendix~D (e.g.\ commutation, no \(\tau\)-accumulation, Cauchy+compatibility) guarantee \((\muc,\nuc)=(0,0)\) at fixed \(\tau\).
  \item \emph{Invariance under filtered quasi-isomorphism.} As in Appendix~D (and the formalization sketch of Appendix~F), \((\muc,\nuc)\) is invariant under replacing each \(F_n\) up to f.q.i., since \(\mathbf{P}_i\) and \(\mathbf{T}_\tau\) respect such replacements and kernels/cokernels (hence generic fiber dimensions) are preserved under isomorphism in \(\Pers^{\mathrm{ft}}_k\).
\end{itemize}

\subsection*{J.5. \texorpdfstring{$\tau$}{tau}-sweep and stability bands}
We formalize the practical, windowed method for selecting scales \(\tau\) at which tower effects vanish.

\begin{definition}[Stability band for a fixed window and degree]\label{J:def:stab-band}
Fix a window (MECE, right-open) and a degree \(i\).
A \emph{\(\tau\)-sweep} is a finite or countable increasing array \(\{\tau_\ell\}_{\ell\in L}\subset(0,\infty)\).
A contiguous subarray \(\{\tau_a,\ldots,\tau_b\}\) is a \emph{stability band} if
\[
\mu_{i,\tau_\ell}(F)=\nu_{i,\tau_\ell}(F)=0\quad\text{for all }\ \ell\in\{a,\ldots,b\},
\]
and the verdict persists under refinement of the sweep (inserting new \(\tau\)-values) without creating nonzero \(\mu_{i,\tau}\) or \(\nu_{i,\tau}\) in the band.
\end{definition}

\begin{proposition}[Robust detection under sweep refinement]\label{J:prop:robust-band}
Assume one of the sufficient conditions of Appendix~D, §D.3 holds (e.g.\ commutation (S1), no \(\tau\)-accumulation (S2), or \(\mathbf{T}_\tau\)-Cauchy with compatible cocone (S3)). Then for each \(i\) there exists a neighborhood of any \(\tau_0\) where \(\phi_{i,\tau}\) is an isomorphism, hence \((\mu_{i,\tau},\nu_{i,\tau})=(0,0)\). A sufficiently fine \(\tau\)-sweep detects such neighborhoods as stability bands and remains stable under refinement.
\end{proposition}

\begin{remark}[Aggregating across degrees]
Applications often monitor either a fixed finite degree set \(I\subset\mathbb{Z}\) or the total \(\muc,\nuc\). A \emph{joint} stability band is the intersection of degreewise bands; it is nonempty whenever each degree admits a band with nontrivial overlap. Report the monitored degree set in the run manifest (Appendix~G).
\end{remark}

\subsection*{J.6. Window pasting via Restart and Summability}
We now integrate a \emph{Restart} inequality and a \emph{Summability} condition to paste windowed certificates into global ones. The $\delta$-ledger (algorithmic/discretization/measurement) is as in Appendix~G; Mirror/Transfer commutation defects are treated via Appendix~L.

\begin{definition}[Per-window pipeline budget and safety margin]\label{J:def:budget-gap}
Let \(\{W_k=[u_k,u_{k+1})\}_k\) be a MECE partition (right-open).
On window \(W_k\), let \(\tau_k>0\) be the selected collapse threshold (possibly from a stability band) and define the \emph{pipeline budget}
\[
\Sigma\delta_k(i)\ :=\ \sum_{U\in W_k}\Big(\delta^{\mathrm{alg}}_{U}(i,\tau_k)+\delta^{\mathrm{disc}}_{U}(i,\tau_k)+\delta^{\mathrm{meas}}_{U}(i,\tau_k)\Big),
\]
with \(\delta\)-components recorded per operation (Appendix~G). The \emph{safety margin} \(\mathrm{gap}_{\tau_k}(i)>0\) is the configured slack for B-Gate\(^{+}\) on \(W_k\) and degree \(i\) (Chapter~1).
\end{definition}

\begin{lemma}[Restart inequality]\label{J:lem:restart}
Assume that on \(W_k\) the B-Gate\(^{+}\) passes with \(\mathrm{gap}_{\tau_k}(i)>\Sigma\delta_k(i)\) and that the transition to \(W_{k+1}\) is realized by a finite composition of \emph{deletion-type} steps and \(\varepsilon\)-continuations (both measured after \(\mathbf{T}_\tau\)). Then there exists \(\kappa\in(0,1]\), depending only on the admissible step class and the \(\tau\)-adaptation policy, such that
\[
\mathrm{gap}_{\tau_{k+1}}(i)\ \ge\ \kappa\ \bigl(\mathrm{gap}_{\tau_k}(i)-\Sigma\delta_k(i)\bigr).
\]
\end{lemma}

\begin{proof}[Proof sketch]
Deletion-type steps are nonincreasing for monitored indicators after \(\mathbf{T}_\tau\) (Appendix~E); \(\varepsilon\)-continuations are \(1\)-Lipschitz, so the drift is bounded by the declared \(\varepsilon\). Aggregating drifts across the finite composition yields a multiplicative retention factor \(\kappa\).
\end{proof}

\begin{definition}[Summability]\label{J:def:summability}
A run satisfies \emph{Summability} (on degrees \(i\in I\)) if
\[
\sum_{k}\Sigma\delta_k(i)\ <\ \infty\qquad(\forall\,i\in I).
\]
A sufficient design pattern is geometric decay of thresholds (e.g.\ \(\tau_k=\tau_0\rho^k\), \(\rho\in(0,1)\)), combined with bounded per-window operation counts, as recorded in Appendix~G.
\end{definition}

\begin{theorem}[Pasting windowed certificates]\label{J:thm:pasting}
Let \(\{W_k\}_k\) be MECE. Suppose that on each \(W_k\) the B-Gate\(^{+}\) passes with \(\mathrm{gap}_{\tau_k}(i)>\Sigma\delta_k(i)\) (for all \(i\in I\)), that the Restart inequality (Lemma~\ref{J:lem:restart}) holds at every transition, and that Summability (Definition~\ref{J:def:summability}) is satisfied. Then the concatenation of per-window certificates yields a global certificate on \(\bigcup_k W_k\) for the monitored degrees \(i\in I\).
\end{theorem}

\begin{proof}[Proof sketch]
Iterate Lemma~\ref{J:lem:restart} and sum budgets; Summability ensures that the cumulative loss of safety margin remains bounded, so positivity of the margin persists. MECE coverage (Appendix~A) ensures there are no gaps or overlaps; stability bands (if used) fix \(\tau_k\) within regimes where \((\mu,\nu)=(0,0)\).
\end{proof}

\subsection*{J.7. Practical detection, logging, and reproducibility}
In practice, one proceeds along a windowed run (Appendix~G) as follows:
\begin{enumerate}\itemsep0.25em
  \item \emph{Select \(\tau_k\)} on window \(W_k\) via a \(\tau\)-sweep; identify a stability band by Proposition~\ref{J:prop:robust-band}.
  \item \emph{Evaluate \((\mu,\nu)\)} at the chosen \(\tau_k\) and record \(\phi_{i,\tau_k}\)-ranks, \(\mu_{i,\tau_k},\nu_{i,\tau_k}\) in \texttt{phi} artifacts (Appendix~G).
  \item \emph{Log the pipeline budget} \(\Sigma\delta_k(i)\) and \(\mathrm{gap}_{\tau_k}(i)\) in \texttt{run.yaml} (Appendix~G) and check the Restart inequality.
  \item \emph{Paste certificates} across windows using Theorem~\ref{J:thm:pasting}; ensure that coverage checks and MECE conventions pass (Appendix~G).
\end{enumerate}
Mirror/Transfer non-commutation is measured after collapse via a natural 2-cell and contributes additively to \(\Sigma\delta\) (Appendix~L).

\subsection*{J.8. Edge cases and pitfalls}
\begin{itemize}\itemsep0.25em
  \item \emph{Near-threshold accumulation.} If bars approach length \(\tau\) from below, stability bands may be thin or absent; operate with finer sweeps and report \((\mu,\nu)\) sensitivity.
  \item \emph{Metric drift without Summability.} If \(\sum_k\Sigma\delta_k(i)\) diverges, the safety margin can be exhausted despite per-window passes; redesign the pipeline (e.g.\ impose geometric decay).
  \item \emph{Degree mixing.} When aggregating \(\muc,\nuc\) across degrees, a band for totals may hide degreewise failures. Prefer reporting both per-degree and totals (Appendix~G).
\end{itemize}

\subsection*{J.9. Formalization pointers (Lean/Coq) and API alignment}
A minimal formalization (Appendix~F) provides:
\begin{itemize}\itemsep0.25em
  \item \texttt{phi} as \(\varinjlim \mathbf{T}_\tau\mathbf{P}_i(F_n)\to \mathbf{T}_\tau\mathbf{P}_i(F_\infty)\), with naturality in tower morphisms, composition, and cofinal reindexing.
  \item \texttt{mu}, \texttt{nu} as generic-fiber dimensions of kernel/cokernel; subadditivity, additivity, and invariance lemmas (Theorems~\ref{J:thm:subadd}, \ref{J:prop:sum}, \ref{J:thm:cofinal}).
  \item Stability-band predicates over discrete \(\tau\)-sweeps (Definition~\ref{J:def:stab-band}) and refinement-stability (Proposition~\ref{J:prop:robust-band}).
  \item Restart/Summability contracts (\(\kappa\), \(\sum\Sigma\delta\)) and a pasting theorem as in Theorem~\ref{J:thm:pasting}. These align with the \texttt{run.yaml} schema (Appendix~G).
\end{itemize}

\subsection*{J.10. Piecewise constancy off a finite critical set}
The following self-contained statement ensures that \(\tau\mapsto(\mu_{i,\tau},\nu_{i,\tau})\) is piecewise constant away from a finite set of critical scales (per bounded \(\tau\)-range), hence justifying stability-band detection by refinement.

\begin{proposition}[Finite critical set and piecewise constancy]\label{J:prop:piecewise}
Fix a tower \(F\), degree \(i\), and a bounded interval \([a,b]\subset(0,\infty)\).
There exists a finite set \(S\subset[a,b]\) (depending on \(F,i,[a,b]\)) such that \(\mu_{i,\tau}\) and \(\nu_{i,\tau}\) are locally constant on each connected component of \([a,b]\setminus S\).
\end{proposition}

\begin{proof}[Proof sketch]
In the constructible regime, after applying \(\mathbf{T}_\tau\) the barcode of each \(\mathbf{P}_i(F_n)\) changes only at finitely many thresholds (finite bar lengths and their finite sums/differences within \([a,b]\)). Kernels and cokernels of \(\phi_{i,\tau}\) vary only when a bar crosses the \(\tau\)-cut or when the colimit/cocone structure alters the presence of infinite bars in kernel/cokernel. Thus only finitely many scales in \([a,b]\) can change \(\mu_{i,\tau}\) or \(\nu_{i,\tau}\). Between these scales, bar truncations and the induced linear maps (hence generic-fiber dimensions) are constant.
\end{proof}

\begin{corollary}[Band openness]\label{J:cor:band-open}
If \(\phi_{i,\tau_0}\) is an isomorphism for some \(\tau_0\in(a,b)\), then it is an isomorphism on an open neighborhood \(U\subset(a,b)\) of \(\tau_0\). In particular, stability bands are unions of (nonempty) open intervals intersected with the sweep.
\end{corollary}

\subsection*{J.11. Window coherence and after-collapse order}
All diagnostics \(\phi_{i,\tau}\), \(\mu_{i,\tau}\), and \(\nu_{i,\tau}\) are computed \emph{after} applying \(\mathbf{T}_\tau\) (B-side single layer), and \emph{with the same window and the same \(\tau\)} as used by the gate and the \(\delta\)-ledger (Appendix~G). This ``for each \(t\)\(\to\)\(\mathbf{P}_i\)\(\to\)\(\mathbf{T}_\tau\)\(\to\)compare'' order is mandatory for auditability; it guarantees that subsequent \(1\)-Lipschitz post-processing never increases budgets or alters band detection (Appendix~A, Appendix~L).

\subsection*{J.12. Minimal API sketch (pseudocode)}
For engineering integration and tests (T11), we record a minimal pseudocode. It is schematic and omits low-level details (e.g.\ barcode calls).

\begin{verbatim}
# Compute (mu, nu) at scale tau for tower F and degree i
def compute_mu_nu(F, i, tau):
    # Persistence after collapse
    Ms = [ T_tau(P_i(F_n), tau) for F_n in F.levels ]
    M_inf = T_tau(P_i(F.apex), tau)
    # Colimit in [R, Vect] (scope policy; return to constructible if needed)
    colim_M = colim(Ms)         # compute objectwise in t
    phi = comparison(colim_M, M_inf)
    mu = generic_fiber_dim(kernel(phi))   # multiplicity of I[0,∞)
    nu = generic_fiber_dim(cokernel(phi))
    return mu, nu

# Detect stability band on a tau sweep (robust to refinement)
def detect_stability_band(F, i, tau_sweep):
    zero_idxs = [ell for ell,tau in enumerate(tau_sweep)
                 if compute_mu_nu(F,i,tau) == (0,0)]
    bands = maximal_contiguous_subarrays(zero_idxs)
    return bands   # evaluate robustness by inserting midpoints and retesting

# Restart inequality and summability check
def restart_ok(gap_next, gap_curr, dsum, kappa):
    return gap_next >= kappa * (gap_curr - dsum) and gap_curr > dsum

def summable_budget(deltas):
    # deltas: list of per-window sums Σδ_k
    return (sum(deltas) < +infty)

# Paste certificates across windows W_k (degree i)
def paste_certificates(windows, i, kappas, budgets, gaps):
    assert len(windows)==len(kappas)==len(budgets)==len(gaps)
    for k in range(len(windows)-1):
        assert restart_ok(gaps[k+1], gaps[k], budgets[k], kappas[k])
    assert summable_budget(budgets)
    return True
\end{verbatim}

\medskip
\noindent\textbf{Summary.}
Within the constructible range and the filtered-colimit scope of Appendix~A, the comparison map \(\phi_{i,\tau}\) is functorial in the tower and compatible with composition and finite sums; \(\mu\) and \(\nu\) (and their totals \(\muc,\nuc\)) inherit subadditivity under composition, additivity under finite direct sums, and invariance under cofinal reindexing (and under f.q.i.\ replacements).
Moreover, \(\tau\mapsto(\mu_{i,\tau},\nu_{i,\tau})\) is piecewise constant off a finite critical set on bounded \(\tau\)-windows, ensuring robust stability-band detection by sweep refinement.
Finally, Restart and Summability provide a principled pasting mechanism for windowed certificates across MECE partitions: with per-window budgets \(\Sigma\delta\) and safety margins \(\mathrm{gap}_\tau\) recorded in the run manifest (Appendix~G), one obtains a global certificate whenever the per-window gate passes and the restart/summability conditions hold.
All comparisons and budgets are made after collapse on the B-side single layer, with window coherence and \(1\)-Lipschitz post-processing ensuring nonincreasing budgets and reproducible audits.



% =========================
\section*{Appendix K. Idempotent (Co)Monads for Collapse (up to f.q.i.) [Spec + Soft-Commuting Policy]}
% =========================
\addcontentsline{toc}{section}{Appendix K. Idempotent (Co)Monads for Collapse (up to f.q.i.)}
\label{K:monads}

\paragraph{Standing conventions.}
We work over a field \(k\).
All persistence modules are constructible (locally finite on bounded windows).
Filtered (co)limits are computed in the functor category \([\mathbb{R},\mathsf{Vect}_k]\) under the scope policy of Appendix~A, Remark~\ref{A:rk:filtered-colimits}; when stated, the result is returned to the constructible range.
Reflection \(\mathbf{T}_\tau\dashv \iota_\tau\) onto the \(\tau\)\hyp local (orthogonal) subcategory \((\mathsf{E}_\tau)^\perp\subset \Pers^{\mathrm{ft}}_k\) is exact and \(1\)\hyp Lipschitz (Appendix~A, Theorem~\ref{A:thm:localization}, Proposition~\ref{A:prop:lipschitz}).
Global conventions: \(\Ext^1\) is always against \(k[0]\); the energy exponent is \(\alpha>0\) (default \(\alpha=1\)); windows are MECE and right\hyp open (Appendix~G).
Distances are measured by the interleaving metric \(d_{\mathrm{int}}\) (equivalently, bottleneck distance on barcodes for constructible objects).

\medskip
\noindent
Throughout, we implicitly identify functors and natural transformations up to filtered quasi\hyp isomorphism (abbrev.\ f.q.i.) whenever we work in the filtered\hyp homotopy setting; all such identifications are stable under the persistence functors \(\mathbf{P}_i\).

% ------------------------------------------------------------
\subsection*{K.1. (Persistence layer) the idempotent monad \(\ \iota_\tau\mathbf{T}_\tau\)}
% ------------------------------------------------------------
Let \(\iota_\tau:(\mathsf{E}_\tau)^\perp\hookrightarrow \Pers^{\mathrm{ft}}_k\) be the fully faithful inclusion and \(\mathbf{T}_\tau:\Pers^{\mathrm{ft}}_k\to(\mathsf{E}_\tau)^\perp\) its left adjoint (Appendix~A, Theorem~\ref{A:thm:localization}). Consider the endofunctor
\[
\mathbf{M}_\tau\ :=\ \iota_\tau\circ \mathbf{T}_\tau:\ \Pers^{\mathrm{ft}}_k\longrightarrow \Pers^{\mathrm{ft}}_k.
\]

\begin{theorem}[Idempotent monad on \(\Pers^{\mathrm{ft}}_k\)]\label{K:thm:monad}
With unit \(\eta:\mathrm{Id}\Rightarrow \iota_\tau\mathbf{T}_\tau\) and multiplication
\(\mu:\mathbf{M}_\tau^2=\iota_\tau\mathbf{T}_\tau\iota_\tau\mathbf{T}_\tau\xRightarrow{\ \iota_\tau\,\varepsilon\,\mathbf{T}_\tau\ }\iota_\tau\mathbf{T}_\tau=\mathbf{M}_\tau\),
induced by the counit \(\varepsilon:\mathbf{T}_\tau\iota_\tau\Rightarrow \mathrm{Id}\) on \((\mathsf{E}_\tau)^\perp\), the triple \((\mathbf{M}_\tau,\eta,\mu)\) is a monad on \(\Pers^{\mathrm{ft}}_k\).
Moreover, \(\mu\) is a natural isomorphism (idempotence).
\end{theorem}

\begin{proof}
This is standard for reflective subcategories with fully faithful right adjoint. The triangle identities for the adjunction \(\mathbf{T}_\tau\dashv\iota_\tau\) yield the monad axioms for \((\iota_\tau\mathbf{T}_\tau,\eta,\mu)\). Since \(\iota_\tau\) is fully faithful, the counit \(\varepsilon\) is an isomorphism on \((\mathsf{E}_\tau)^\perp\); hence \(\mu=\iota_\tau\,\varepsilon\,\mathbf{T}_\tau\) is a natural isomorphism, proving idempotence.
\end{proof}

\begin{proposition}[Exactness and non\hyp expansiveness]\label{K:prop:exact-lip}
\(\mathbf{M}_\tau\) is exact on \(\Pers^{\mathrm{ft}}_k\) and \(1\)\hyp Lipschitz for the interleaving/bottleneck metric:
\[
d_{\mathrm{int}}\!\big(\mathbf{M}_\tau M,\ \mathbf{M}_\tau N\big)\ \le\ d_{\mathrm{int}}(M,N)\qquad(M,N\in\Pers^{\mathrm{ft}}_k).
\]
\end{proposition}

\begin{proof}
Exactness follows because both \(\mathbf{T}_\tau\) and \(\iota_\tau\) are exact (Appendix~A, Theorem~\ref{A:thm:localization}). The \(1\)\hyp Lipschitz property follows from the fact that \(\mathbf{T}_\tau\) commutes with shifts and truncations used in the definition of interleavings, hence it does not increase the minimal interleaving parameter (Appendix~A, Proposition~\ref{A:prop:lipschitz}).
\end{proof}

\begin{remark}[Algebras and fixed points]\label{K:rk:algebras}
The Eilenberg\hyp Moore category of \(\mathbf{M}_\tau\)\hyp algebras identifies with \((\mathsf{E}_\tau)^\perp\) via \(\iota_\tau\). Concretely, \(\mathbf{M}_\tau\)\hyp algebras \(\theta:\mathbf{M}_\tau M \to M\) with the usual coherence are in bijection with objects for which \(\eta_M:M\to \mathbf{M}_\tau M\) is an isomorphism, i.e.\ the essential image of \(\iota_\tau\). In particular, \(\mathbf{M}_\tau\) acts as a \emph{collapse} to the \(\tau\)\hyp local part, and \(\mathbf{M}_\tau\)\hyp modules are precisely the \(\tau\)\hyp collapsed objects.
\end{remark}

\begin{example}[Length threshold reflector]
In \(1\)\hyp dimensional persistence, \(\mathbf{T}_\tau\) kills all summands of lifespan \(<\tau\). Then \(\mathbf{M}_\tau\) replaces a module by its \(\tau\)\hyp truncated part. By idempotence, \(\mathbf{M}_\tau\mathbf{M}_\tau=\mathbf{M}_\tau\).
\end{example}

% ------------------------------------------------------------
\subsection*{K.2. (Ho(\(\mathsf{FiltCh}(k)\))—implementable range, up to f.q.i.) the idempotent comonad \(\ \iota\circ C_\tau^{\mathrm{comb}}\) [Spec]}
% ------------------------------------------------------------
\emph{[Spec]} There exists an implementable subcategory \(\Ho(\mathsf{FiltCh}(k))_\tau^{\mathrm{comb}}\subset \Ho(\mathsf{FiltCh}(k))\) (``\(\tau\)\hyp combinatorial collapses'') and a fully faithful inclusion
\[
\iota:\ \Ho(\mathsf{FiltCh}(k))_\tau^{\mathrm{comb}}\hookrightarrow \Ho(\mathsf{FiltCh}(k))
\]
admitting a right adjoint (coreflector) \(C_\tau^{\mathrm{comb}}:\Ho(\mathsf{FiltCh}(k))\to \Ho(\mathsf{FiltCh}(k))_\tau^{\mathrm{comb}}\), natural up to filtered quasi\hyp isomorphism (Appendix~B). Define
\[
\mathbf{G}_\tau\ :=\ \iota\circ C_\tau^{\mathrm{comb}}:\ \Ho(\mathsf{FiltCh}(k))\longrightarrow \Ho(\mathsf{FiltCh}(k)).
\]

\begin{theorem}[Idempotent comonad on \(\Ho(\mathsf{FiltCh}(k))\) up to f.q.i.]\label{K:thm:comonad}
Assuming \(\iota\dashv C_\tau^{\mathrm{comb}}\), with counit \(\varepsilon:\mathbf{G}_\tau\Rightarrow \mathrm{Id}\) and comultiplication
\(\delta:\mathbf{G}_\tau\xRightarrow{\ \iota\,\eta\,C_\tau^{\mathrm{comb}}\ }\mathbf{G}_\tau^2\)
(\(\eta\) the unit on \(\Ho(\mathsf{FiltCh}(k))_\tau^{\mathrm{comb}}\)), the triple \((\mathbf{G}_\tau,\varepsilon,\delta)\) is a comonad on \(\Ho(\mathsf{FiltCh}(k))\).
Moreover, \(\delta\) is a natural isomorphism (idempotence).
All identities hold in \(\Ho(\mathsf{FiltCh}(k))\) and are invariant under filtered quasi\hyp isomorphisms.
\end{theorem}

\begin{proof}
Dual to Theorem~\ref{K:thm:monad}. For a coreflective subcategory with fully faithful left adjoint, the comonad axioms follow from the triangle identities. Since \(\eta\) is an isomorphism on \(\Ho(\mathsf{FiltCh}(k))_\tau^{\mathrm{comb}}\), idempotence follows. Equalities are interpreted in the localized category where f.q.i.\ are inverted.
\end{proof}

\begin{proposition}[Compatibility with persistence]\label{K:prop:compat}
For any filtered complex \(F\),
\[
\mathbf{P}_i\big(\mathbf{G}_\tau F\big)\ \cong\ \mathbf{M}_\tau\big(\mathbf{P}_i(F)\big)\qquad\text{naturally in \(i\) and \(F\)},
\]
and, after \(\mathbf{P}_i\),
\[
d_{\mathrm{int}}\!\big(\mathbf{P}_i(\mathbf{G}_\tau F),\ \mathbf{P}_i(\mathbf{G}_\tau G)\big)\ \le\ d_{\mathrm{int}}\!\big(\mathbf{P}_i(F),\ \mathbf{P}_i(G)\big).
\]
\end{proposition}

\begin{proof}
By construction \(C_\tau^{\mathrm{comb}}\) lifts \(\mathbf{T}_\tau\) (Appendix~B), hence \(\iota C_\tau^{\mathrm{comb}}\) lifts \(\iota_\tau \mathbf{T}_\tau\) at the persistence level, yielding the natural isomorphism after applying \(\mathbf{P}_i\). Non\hyp expansiveness follows from Proposition~\ref{K:prop:exact-lip} because \(\mathbf{P}_i\) is \(1\)\hyp Lipschitz with respect to filtered interleavings and \(d_{\mathrm{int}}\).
\end{proof}

\begin{remark}[Scope and usage]\label{K:rk:scope}
The comonad \(\mathbf{G}_\tau\) is asserted only on the \emph{implementable range (up to f.q.i.)}, which suffices for algorithms and stability statements. All filtered (co)limit claims comply with Appendix~A, Remark~\ref{A:rk:filtered-colimits}. In practice, computations are performed on a B\hyp side single layer after collapse, and all claims are made modulo f.q.i.
\end{remark}

% ------------------------------------------------------------
\subsection*{K.3. Multi-axis torsion reflectors and soft-commuting policy}
% ------------------------------------------------------------
Beyond the length\hyp threshold reflector \( \mathbf{T}_\tau \), one may consider other exact reflectors \(T_A,T_B:\Pers^{\mathrm{ft}}_k\to\Pers^{\mathrm{ft}}_k\) arising from hereditary Serre subcategories \(E_A,E_B\) (e.g.\ birth\hyp window deletion, lifespan windowing, support cuts). This subsection specifies order\hyp independence in the nested case and an \emph{operational} ``soft\hyp commuting'' policy in the general case.

\begin{definition}[Exact reflectors from Serre classes]\label{K:def:serre-reflectors}
Let \(E\subset \Pers^{\mathrm{ft}}_k\) be a hereditary Serre subcategory. The Serre localization yields an exact reflector \(T_E:\Pers^{\mathrm{ft}}_k\to E^\perp\) left adjoint to the inclusion \(E^\perp\hookrightarrow \Pers^{\mathrm{ft}}_k\). We write \(T_A:=T_{E_A}\), \(T_B:=T_{E_B}\), and \(E_{A\vee B}\) for the Serre subcategory generated by \(E_A\cup E_B\).
\end{definition}

\begin{proposition}[Nested torsions \(\Rightarrow\) order independence]\label{K:prop:nested}
If \(E_A\subseteq E_B\) or \(E_B\subseteq E_A\), then
\[
T_A\circ T_B\ =\ T_B\circ T_A\ =\ T_{A\vee B}
\]
as exact reflectors on \(\Pers^{\mathrm{ft}}_k\).
In particular, for the \(1\)\hyp dimensional length thresholds, \(\mathbf{T}_\tau\circ \mathbf{T}_\sigma=\mathbf{T}_{\max\{\tau,\sigma\}}\).
\end{proposition}

\begin{proof}
In an abelian category, localization at nested Serre subcategories is idempotent and order\hyp independent: if \(E_A\subseteq E_B\), then \(E_B^\perp\subseteq E_A^\perp\). The composite \(T_A\circ T_B\) (resp.\ \(T_B\circ T_A\)) is the reflector onto \(E_A^\perp\cap E_B^\perp=(E_{A\vee B})^\perp\). Hence both composites identify with the reflector at the join. The special case for length thresholds follows because the subcategory of intervals of length \(<\tau\) is nested in that of length \(<\sigma\) when \(\tau\ge \sigma\).
\end{proof}

\begin{definition}[A/B commutation defect and soft\hyp commuting policy]\label{K:def:soft}
For arbitrary exact reflectors \(T_A,T_B\) and a dataset \(M\in\Pers^{\mathrm{ft}}_k\), define the \emph{commutation defect}
\[
\Delta_{\mathrm{comm}}(M;A,B)\ :=\ d_{\mathrm{int}}\big(T_AT_BM,\ T_BT_AM\big).
\]
Fix a tolerance \(\eta\ge 0\) and a window (MECE, right\hyp open). We say \(T_A,T_B\) are \emph{soft\hyp commuting} on the dataset if \(\Delta_{\mathrm{comm}}(M;A,B)\le \eta\) holds on the relevant instances (per window). Otherwise we \emph{fallback} to a fixed order (e.g.\ \(T_B\circ T_A\)).
\end{definition}

\begin{declaration}[Operational policy and \(\delta\)\hyp ledger integration]\label{K:dec:soft-policy}
Within a window \(W\) and for a monitored degree \(i\):
\begin{enumerate}\itemsep0.2em
  \item \emph{Test.} Measure \(\Delta_{\mathrm{comm}}(M;A,B)\) at the persistence layer after applying any prerequisite collapse(s), using the window policy (Appendix~G). If \(\Delta_{\mathrm{comm}}\le \eta\), adopt \emph{soft\hyp commuting} and allow either order in that window.
  \item \emph{Fallback.} If \(\Delta_{\mathrm{comm}}>\eta\), fix an order (e.g.\ \(T_B\circ T_A\)) and \emph{record} the observed \(\Delta_{\mathrm{comm}}\) as an \(\delta^{\mathrm{alg}}\) entry in the \(\delta\)\hyp ledger (Appendix~G).
  \item \emph{Pipeline additivity.} Over a pipeline of such steps across windows, add the recorded defects to the window budgets \(\Sigma\delta\) (Appendix~G) and consume them in the Restart/Summability calculus (Appendix~J).
  \item \emph{Reproducibility.} Log \(\eta\), the chosen order (or the soft\hyp commuting verdict), and the measured \(\Delta_{\mathrm{comm}}\) per window in \texttt{run.yaml} (Appendix~G).
\end{enumerate}
\end{declaration}

\begin{remark}[Scope and guard\hyp rails]\label{K:rk:guards}
We do \emph{not} assert any a priori commutation for non\hyp nested torsions; order control is purely \emph{operational} via the A/B test. All distances are measured \emph{after} truncation by \(\mathbf{T}_\tau\) and on the B\hyp side single layer, consistent with Chapter~1 and Appendix~L.
\end{remark}

% ------------------------------------------------------------
\subsection*{K.4. Interaction with Mirror/Transfer and pipeline \(\delta\)\hyp budget}
% ------------------------------------------------------------
Let \(\Mirror\) be an admissible endofunctor at the filtered level (or on \(\Pers^{\mathrm{ft}}_k\)) satisfying the quantitative commutation hypotheses of Appendix~L: \(1\)\hyp Lipschitz on \(\mathbf{P}_i\) and a natural \(2\)\hyp cell \(\Mirror\circ C_\tau \Rightarrow C_\tau\circ \Mirror\) with a uniform bound \(\delta(i,\tau)\ge 0\) in \(d_{\mathrm{int}}\).

\begin{proposition}[Budget accounting with multiple (co)reflectors and Mirror]\label{K:prop:budget}
In a pipeline that interleaves exact reflectors \(T_\bullet\) (including \(\mathbf{T}_\tau\)) and Mirror/Transfer steps, the total commutation/ordering slack budget on a window is bounded by the \emph{additive} sum of:
\begin{itemize}\itemsep0.2em
  \item each Mirror–Collapse defect \(\delta(i,\tau)\) (Appendix~L),
  \item each A/B commutation defect \(\Delta_{\mathrm{comm}}(M;A,B)\) when soft\hyp commuting fails (Declaration~\ref{K:dec:soft-policy}),
\end{itemize}
and is \emph{non\hyp increasing} under any subsequent \(1\)\hyp Lipschitz post\hyp processing (e.g.\ further truncations, degree projections) at the persistence layer.
\end{proposition}

\begin{proof}
Concatenate the natural \(2\)\hyp cells along the pipeline and apply the triangle inequality for \(d_{\mathrm{int}}\). Each Mirror–Collapse interchange contributes at most \(\delta(i,\tau)\). When soft\hyp commuting fails for \(T_A,T_B\), the observed \(\Delta_{\mathrm{comm}}\) accounts for the order selection; adding these yields an upper bound on the overall slack. Since all subsequent steps are \(1\)\hyp Lipschitz (Appendix~A, Proposition~\ref{A:prop:lipschitz}), the bound cannot increase under post\hyp processing.
\end{proof}

\begin{corollary}[Window\hyp wise additivity and stability]\label{K:cor:window}
If a windowed pipeline uses windows \(\{W_j\}_{j\in J}\) (MECE, right\hyp open) and records per\hyp window contributions \(\delta_j^{\mathrm{Mirror}}\) and \(\delta_j^{\mathrm{A/B}}\), then
\[
\textstyle \sum_{j\in J}\Big(\delta_j^{\mathrm{Mirror}}+\delta_j^{\mathrm{A/B}}\Big)
\]
is a valid global upper bound on the end\hyp to\hyp end slack, and any \(1\)\hyp Lipschitz aggregator on the persistence side preserves this bound.
\end{corollary}

% ------------------------------------------------------------
\subsection*{K.5. Windowed usage and reproducibility (\texttt{run.yaml} alignment)}
% ------------------------------------------------------------
For each window \(W\) record in \texttt{run.yaml} (Appendix~G):
\begin{itemize}\itemsep0.2em
  \item the set of reflectors used (e.g.\ \texttt{length}, \texttt{birth\_window}), their order, and whether \emph{soft\hyp commuting} was adopted (\texttt{ab\_test} block with tolerance \(\eta\));
  \item the measured \(\Delta_{\mathrm{comm}}\) (if any), logged into the \(\delta\)\hyp ledger under \(\delta^{\mathrm{alg}}\);
  \item the Mirror/Transfer commutation bounds \(\delta(i,\tau)\) (Appendix~L) aggregated into \(\Sigma\delta\);
  \item the B\hyp Gate\(^{+}\) verdict and the safety margin \(\mathrm{gap}_\tau\), to be consumed by Restart/Summability (Appendix~J).
\end{itemize}

\noindent
A minimal schema (illustrative only) is:
\begin{verbatim}
window:
  id: "W01"
  degree: 1
  tau: 0.15
  reflectors:
    - name: length
    - name: birth_window
  ab_test:
    eta: 0.02
    verdict: soft_commuting | fallback
    chosen_order: [birth_window, length]   # if fallback
    Delta_comm: 0.031                      # if measured
  mirror:
    delta:
      i: 1
      tau: 0.15
      value: 0.010
  budget:
    delta_alg: 0.031
    delta_mirror: 0.010
    sum_delta: 0.041
  b_gate_plus:
    passed: true
    gap_tau: 0.12
\end{verbatim}

\begin{remark}[Window coherence]\label{K:rk:window-coherence}
A/B tests and Mirror–Collapse commutation checks must be performed on the \emph{same} window (MECE, right\hyp open) and the \emph{same} collapse thresholds \(\tau\) as logged in \texttt{run.yaml} to ensure consistent budget accounting and pasting (Appendix~J).
\end{remark}

% ------------------------------------------------------------
\subsection*{K.6. Formalization stubs (Lean/Coq) [Spec]}
% ------------------------------------------------------------
A minimal API (cf.\ Appendix~F) includes:
\begin{itemize}\itemsep0.2em
  \item \texttt{reflector\_E} for a hereditary Serre subcategory \(E\), yielding an exact reflector \(T_E\) and the monad \(\iota_E T_E\) (idempotent);
  \item a lemma \texttt{nested\_torsion\_order\_indep} stating \(T_A\circ T_B=T_B\circ T_A=T_{A\vee B}\) when \(E_A\subseteq E_B\) or \(E_B\subseteq E_A\);
  \item a \texttt{comm\_defect} functional producing \(\Delta_{\mathrm{comm}}(M;A,B)=d_{\mathrm{int}}(T_AT_BM,T_BT_AM)\) with a policy hook for \emph{soft\hyp commuting} vs.\ fallback;
  \item a Mirror–Collapse \(2\)\hyp cell contract \texttt{mirror\_collapse\_delta} yielding \(\delta(i,\tau)\) and the additive pipeline bound under \(1\)\hyp Lipschitz post\hyp processing.
\end{itemize}

\noindent
Explicit signatures (schematic):
\begin{verbatim}
constant reflector_E :
  (E : SerreSubcat (Pers_ft k)) ->
  Σ (T : Pers_ft k ⥤ Pers_ft k),
    (is_exact T) × (is_reflector E⊥ T)

theorem nested_torsion_order_indep
  {E_A E_B : SerreSubcat (Pers_ft k)}
  (h : E_A ⊆ E_B ∨ E_B ⊆ E_A) :
  T_A ⋙ T_B ≅ T_{A∨B} ∧ T_B ⋙ T_A ≅ T_{A∨B}

def comm_defect (M : Pers_ft k)
  (A B : Reflector) : ℝ :=
  d_int ((A ⋙ B) M) ((B ⋙ A) M)

constant mirror_collapse_delta :
  Π (i : ℤ) (τ : ℝ≥0), ℝ≥0
\end{verbatim}

% ------------------------------------------------------------
\subsection*{K.7. Edge cases and pitfalls}
% ------------------------------------------------------------
\begin{itemize}\itemsep0.2em
  \item \emph{Non\hyp exact ``reflectors''.} The policy assumes exact reflectors (Serre localization). Heuristic truncations without exactness may break monad laws and invalidate stability claims.
  \item \emph{Inconsistent windows.} Running A/B tests or Mirror–Collapse checks on mismatched windows/\(\tau\) invalidates budget accounting and the pasting guarantee (Appendix~J).
  \item \emph{Chained non\hyp nested axes.} With \(T_A,T_B,T_C\) mutually non\hyp nested, pairwise soft\hyp commuting does not imply global confluence. Adopt a fixed canonical order (e.g.\ by axis priority), A/B test adjacent pairs, and log residuals.
  \item \emph{Hidden state across layers.} Always measure \(\Delta_{\mathrm{comm}}\) \emph{after} the B\hyp side collapse and at the monitored degree to avoid spurious layer interactions.
  \item \emph{Threshold drift.} If \(\tau\) is adapted online, ensure all \(\delta\)\hyp ledger entries specify the \(\tau\) in force at measurement time; do not retroactively amend past entries.
\end{itemize}

\medskip
\noindent\textbf{Summary.}
On the persistence side, collapse is governed by the idempotent monad \(\iota_\tau\mathbf{T}_\tau\), exact and \(1\)\hyp Lipschitz.
On the filtered–homotopy side (implementable range, up to f.q.i.), collapse is governed by the idempotent comonad \(\iota\circ C_\tau^{\mathrm{comb}}\).
For \emph{multi\hyp axis} torsion reflectors, order independence holds under nesting; otherwise, an A/B \emph{soft\hyp commuting} policy with a measured defect \(\Delta_{\mathrm{comm}}\) provides operational control, integrating into the additive pipeline \(\delta\)\hyp budget together with Mirror/Transfer commutation (Appendix~L).
All steps are carried out \emph{after} collapse on the B\hyp side single layer, logged per window (Appendix~G), and pasted globally via Restart/Summability (Appendix~J).



% =========================
\section*{Appendix L. Quantitative Commutation for Mirror/Tropical [Spec + Pipeline Budget + A/B Policy]}
% =========================
\phantomsection
\addcontentsline{toc}{section}{Appendix L. Quantitative Commutation for Mirror/Tropical}
\refstepcounter{section}
\label{L:quant-comm}

\paragraph{Standing conventions.}
We work over a field \(k\); all persistence modules are constructible.
Filtered (co)limits are computed in \([\mathbb{R},\mathsf{Vect}_k]\) under the scope policy of Appendix~A, Remark~\ref{A:rk:filtered-colimits}, and then returned to the constructible range.
The interleaving metric \(d_{\mathrm{int}}\) (=\ bottleneck) is used throughout.
The truncation \(\mathbf{T}_\tau\) is exact and \(1\)-Lipschitz; shift–commutation implies \(1\)-Lipschitz (Appendix~A, Proposition~\ref{A:prop:lipschitz}).
On the filtered-complex side we write \(C_\tau\); on the persistence side \(\mathbf{T}_\tau\).
Global conventions: \(\Ext^1\) is always against \(k[0]\); the energy exponent satisfies \(\alpha>0\) (default \(\alpha=1\)); windows are MECE and right–open (Appendix~G).
All comparisons follow the unique order “for each \(t\)\(\to\)\(\mathbf{P}_i\)\(\to\)\(\mathbf{T}_\tau\)\(\to\)compare,” on the B-side single layer after collapse (Chapter~1, Appendix~N).

% -------------------------
\subsection*{L.1. Hypotheses}
% -------------------------
\emph{[Spec]} Let \(\Mirror\) be a functor on filtered complexes (or, via \(\mathbf{P}_i\), a persistence-layer endofunctor). Assume:

\begin{itemize}\itemsep0.35em
  \item[(H1)] \textbf{\(1\)-Lipschitz of \(\Mirror\).} For every degree \(i\),
  \[
    d_{\mathrm{int}}\!\Big(\mathbf{P}_i(\Mirror F),\ \mathbf{P}_i(\Mirror G)\Big)
       \ \le\ d_{\mathrm{int}}\!\Big(\mathbf{P}_i(F),\ \mathbf{P}_i(G)\Big)
       \qquad\text{for all filtered complexes }F,G.
  \]
  \emph{(No extra assumption is needed for \(C_\tau\): by Appendix~B one has
  \(\mathbf{P}_i(C_\tau F)\cong \mathbf{T}_\tau(\mathbf{P}_iF)\), and \(\mathbf{T}_\tau\) is \(1\)-Lipschitz by Appendix~A, Proposition~\ref{A:prop:lipschitz}.)}
  \item[(H2)] \textbf{\(\delta\)-controlled natural \(2\)-cell.} There exists a natural \(2\)-cell
  \[
    \theta:\ \Mirror\!\circ\! C_\tau\ \Rightarrow\ C_\tau\!\circ\!\Mirror
  \]
  (interpreted up to f.q.i.\ in \(\Ho(\mathsf{FiltCh}(k))\); cf.\ Appendix~B) such that, \emph{uniformly in \(F\)}, for all \(i,\tau\),
  \[
    d_{\mathrm{int}}\!\Big(\mathbf{P}_i(\Mirror(C_\tau F)),\ \mathbf{P}_i(C_\tau(\Mirror F))\Big)\ \le\ \delta(i,\tau).
  \]
\end{itemize}

\begin{remark}[Interpretation]
(H2) measures, at the persistence layer, the failure of \(\Mirror\) and \(C_\tau\) to commute; the control \(\delta(i,\tau)\ge 0\) may depend on \((i,\tau)\) but not on \(F\). It is an external assumption; no intrinsic value of \(\delta\) is claimed without additional structure.
\end{remark}

\begin{remark}[Strict commutation as a special case]
If \(\delta(i,\tau)=0\) (e.g.\ \(\theta\) induces isomorphisms after applying \(\mathbf{P}_i\)), then \(\Mirror\) and \(C_\tau\) commute up to isomorphism at the truncated persistence layer in degree \(i\) and window \(\tau\).
\end{remark}

% -------------------------
\subsection*{L.2. Conditional bound}
% -------------------------
\begin{theorem}[Quantitative commutation under (H1)–(H2), uniform in \(F\)]\label{L:thm:cond}
Under \emph{(H1)} and \emph{(H2)}, for all filtered complexes \(F\), all \(i\), and all \(\tau\),
\[
  d_{\mathrm{int}}\!\Big(
       \mathbf{T}_\tau\,\mathbf{P}_i(\Mirror(C_\tau F))\ ,\ 
       \mathbf{T}_\tau\,\mathbf{P}_i(C_\tau(\Mirror F))\n  \Big)\ \le\ \delta(i,\tau),
\]
and the bound is \emph{uniform in \(F\)}.
\end{theorem}

\begin{proof}[Proof sketch]
By (H2),
\(d_{\mathrm{int}}\big(\mathbf{P}_i(\Mirror(C_\tau F)),\mathbf{P}_i(C_\tau(\Mirror F))\big)\le\delta(i,\tau)\).
Applying \(\mathbf{T}_\tau\), which is \(1\)-Lipschitz (Appendix~A, Proposition~\ref{A:prop:lipschitz}), yields the stated inequality.
Since \(\mathbf{T}_\tau\) is exact, \emph{no additional stability loss} is introduced for downstream kernel/cokernel diagnostics.\footnote{Exactness does not turn a small distance bound into equality of kernels/cokernels; it ensures that truncation itself adds no extra error to such diagnostics.}
\end{proof}

\begin{corollary}[Stability under additional \(1\)-Lipschitz post-processing]
If \(\Phi\) is any endofunctor on persistence modules with \(d_{\mathrm{int}}(\Phi X,\Phi Y)\le d_{\mathrm{int}}(X,Y)\), then for all \(i,\tau,F\),
\[
  d_{\mathrm{int}}\!\Big(
       \Phi\,\mathbf{T}_\tau\,\mathbf{P}_i(\Mirror(C_\tau F))\ ,\ 
       \Phi\,\mathbf{T}_\tau\,\mathbf{P}_i(C_\tau(\Mirror F))\n  \Big)\ \le\ \delta(i,\tau).
\]
\end{corollary}

\begin{corollary}[Two-stage pipeline; additive control]\label{L:cor:additive}
Let \(\Mirror_1,\Mirror_2\) satisfy \emph{(H1)} and admit \(2\)-cells with controls \(\delta_1,\delta_2\) as in \emph{(H2)}. Then for all \(i,\tau,F\),
\[
  d_{\mathrm{int}}\!\Big(
       \mathbf{T}_\tau\,\mathbf{P}_i(\Mirror_2\Mirror_1(C_\tau F))\ ,\ 
       \mathbf{T}_\tau\,\mathbf{P}_i(C_\tau(\Mirror_2\Mirror_1 F))\n  \Big)\ \le\ \delta_1(i,\tau)+\delta_2(i,\tau).
\]
\emph{Proof sketch.} Insert the intermediate term
\(\mathbf{T}_\tau\mathbf{P}_i(\Mirror_2(C_\tau\Mirror_1 F))\) and apply the triangle inequality. Use Theorem~\ref{L:thm:cond} for \(\Mirror_1\) with the \(1\)-Lipschitz post-processing \(\Phi=\mathbf{T}_\tau\mathbf{P}_i\Mirror_2\) to bound the first leg by \(\delta_1\), and Theorem~\ref{L:thm:cond} for \(\Mirror_2\) to bound the second leg by \(\delta_2\).
\end{corollary}

\begin{corollary}[Strict commutation and tower diagnostics]
If \(\delta(i,\tau)=0\) for all \(i,\tau\) (i.e.\ \(\Mirror\) and \(C_\tau\) commute up to isomorphism at the truncated persistence layer), then for any tower \(\{F_n\}\to F_\infty\) and any \(i,\tau\),
the comparison maps computed after either order agree up to isomorphism:
\[
\phi_{i,\tau}\big(\Mirror(C_\tau F_\bullet)\big)\ \cong\ \phi_{i,\tau}\big(C_\tau(\Mirror F_\bullet)\big).
\]
Consequently, the tower obstruction indices are preserved,
\[
\mu_{i,\tau}\big(\Mirror(C_\tau F_\bullet)\big)=\mu_{i,\tau}\big(C_\tau(\Mirror F_\bullet)\big),\qquad\n\nu_{i,\tau}\big(\Mirror(C_\tau F_\bullet)\big)=\nu_{i,\tau}\big(C_\tau(\Mirror F_\bullet)\big),
\]
and likewise for the totals \((\muc,\nuc)\).
\end{corollary}

\begin{remark}[Typical sources of \(\delta=0\)]
If \(\Mirror\) preserves the filtration and there is a filtration–natural isomorphism \(\Mirror\circ C_\tau \cong C_\tau\circ\Mirror\) (e.g.\ \(\Mirror\) is induced by a reindexing monotone endomorphism of \((\mathbb{R},\le)\) that fixes the window cut, or by a direct-sum decomposition functor that respects sublevel sets), then \(\delta(i,\tau)=0\) for all \(i,\tau\).
\end{remark}

\begin{remark}[Uniformity across degrees]
If \(\Mirror\) admits a degreewise \(1\)-Lipschitz constant independent of \(i\) on a fixed finite set of degrees \(I\subset\mathbb{Z}\), then the commutation control may be chosen \(\delta(i,\tau)\le \delta_I(\tau)\) uniformly for \(i\in I\).
\end{remark}

% -------------------------
\subsection*{L.3. Counterexample notes (necessity of assumptions)}
% -------------------------
\begin{remark}[Why the hypotheses are needed]
Without a \(\delta\)-controlled natural \(2\)-cell (H2), individual non-expansiveness of \(\Mirror\) and \(C_\tau\) does \emph{not} constrain the distance between the two composites:
triangle inequalities do not relate \( \Mirror\!\circ\! C_\tau \) and \( C_\tau\!\circ\!\Mirror\) in the absence of a comparison map.
Concretely, one can build (Appendix~D) towers whose bar lengths accumulate from below at \(\tau\) (Type~IV phenomena).
Collapsing first may erase the near-\(\tau\) mass, while mirroring first may convert it into persistent features that survive collapse; the gap can be made arbitrarily large despite each functor being \(1\)-Lipschitz on its own.
Thus no quantitative bound of the form in Theorem~\ref{L:thm:cond} is available in general unless (H2) (or an equivalent commutation control) is imposed.
\end{remark}

\begin{remark}[Scope]
All claims are made under the constructible scope and filtered-(co)limit policy of Appendix~A.
When \(\Mirror\) is only defined up to filtered quasi-isomorphism, (H2) is interpreted in \(\Ho(\mathsf{FiltCh}(k))\), and distances are taken after applying \(\mathbf{P}_i\).
\end{remark}

% -------------------------
\subsection*{L.4. Pipeline error budget and windowed accounting}
% -------------------------
We integrate quantitative commutation into a \emph{windowed} error budget that is additive along pipelines, uniform in the input \(F\), and non-increasing under any subsequent \(1\)\hyp Lipschitz post\hyp processing at the persistence layer.

\begin{definition}[Per-step commutation bounds and window budget]\label{L:def:budget}
Consider a pipeline on a window \(W=[u,u')\) consisting of steps
\[
\Pi\ :=\ \Big(\ \cdots\ \rightarrow\ \Mirror_j\ \xrightarrow{}\ C_{\tau_j}\ \xrightarrow{}\ T_{A_j/B_j}\ \xrightarrow{}\ \cdots\ \Big),
\]
where \(\Mirror_j\) are Mirror/Transfer steps, \(C_{\tau_j}\) are collapses (filtered lift of \(\mathbf{T}_{\tau_j}\)), and \(T_{A_j/B_j}\) are exact reflectors on \(\Pers^{\mathrm{ft}}_k\) (Appendix~K). For each Mirror–Collapse pair, let \(\delta_j(i,\tau_j)\) be the (H2) bound on degree \(i\). For each pair of reflectors \(T_A,T_B\) that are applied in some order with an \emph{A/B test} (Appendix~K), let \(\Delta_{\mathrm{comm}}(M;A,B)\) be the measured commutation defect (Definition~\ref{K:def:soft}); when \(\Delta_{\mathrm{comm}}>\eta\), we fallback to a fixed order and record \(\Delta_{\mathrm{comm}}\) as an \(\delta^{\mathrm{alg}}\) contribution. The \emph{window budget} is the additive sum of all such contributions on \(W\):
\[
\Sigma\delta_W(i)\ :=\ \sum_{j}\delta_j(i,\tau_j)\ +\ \sum_{\text{A/B tested pairs}}\ \Delta_{\mathrm{comm}}(M;A,B)\ \ \ (\text{when exceeded tolerance}).
\]
\end{definition}

\begin{theorem}[Pipeline budget: additivity, uniformity, and post-processing stability]\label{L:thm:budget}
Fix a window \(W\), a monitored degree \(i\), and a pipeline \(\Pi\) as in Definition~\ref{L:def:budget}. For any filtered input \(F\),
\[
d_{\mathrm{int}}\!\Big(
       \mathbf{T}_\tau\,\mathbf{P}_i(\Pi_{\text{lhs}}(F))\ ,\ 
       \mathbf{T}_\tau\,\mathbf{P}_i(\Pi_{\text{rhs}}(F))\n  \Big)\ \le\ \Sigma\delta_W(i),
\]
where the left/right versions aggregate Mirror–Collapse orderings and A/B reflector orderings according to the window’s operational choices. The bound:
\begin{itemize}\itemsep0.2em
  \item is \emph{additive} in the per-step Mirror–Collapse bounds \(\delta_j(i,\tau_j)\) and the reflector A/B defects \(\Delta_{\mathrm{comm}}\);
  \item is \emph{uniform in \(F\)} (each \(\delta_j\) is uniform by (H2), and \(\Delta_{\mathrm{comm}}\) are measured at the persistence layer independent of representatives up to f.q.i.);
  \item is \emph{non-increasing} under any subsequent \(1\)\hyp Lipschitz post\hyp processing on the persistence layer (e.g.\ further truncations \(\mathbf{T}_{\tau'}\), degree projections \(\mathbf{P}_i\), shifts \(S^\varepsilon\)).
\end{itemize}
\end{theorem}

\begin{proof}[Proof sketch]
Compose the natural \(2\)\hyp cells for Mirror–Collapse steps and measure each A/B reflector defect in \(d_{\mathrm{int}}\); apply the triangle inequality to obtain additivity. Uniformity follows from (H2) being uniform in \(F\) and from the fact that A/B defects are computed \emph{after} applying \(\mathbf{P}_i\) (hence f.q.i.\ invariant). Post\hyp processing stability comes from the \(1\)\hyp Lipschitz property of the applied persistence functors (Appendix~A).
\end{proof}

\begin{remark}[run.yaml alignment and B-Gate\(^{+}\)]
Record each \(\delta_j(i,\tau_j)\) and \(\Delta_{\mathrm{comm}}\) per window in the \(\delta\)\hyp ledger (Appendix~G); aggregate into \(\Sigma\delta_W(i)\) and compare to the safety margin \(\mathrm{gap}_\tau\) for B-Gate\(^{+}\) on the same window and degree (Chapter~1). This integrates with Restart/Summability for pasting (Appendix~J).
\end{remark}

% -------------------------
\subsection*{L.5. Operational A/B policy (test \(\Rightarrow\) fallback \(\Rightarrow\) \(\delta_{\mathrm{alg}}\) accounting)}
% -------------------------
We adopt the \emph{soft\hyp commuting} policy of Appendix~K at the persistence layer after collapse, with explicit windowed logging.

\begin{definition}[A/B commutativity test]\label{L:def:ab}
Given exact reflectors \(T_A,T_B\) (Appendix~K) and a dataset \(M\in\Pers^{\mathrm{ft}}_k\) on window \(W\), define
\[
\Delta_{\mathrm{comm}}(M;A,B)\ :=\ d_{\mathrm{int}}\big(T_AT_BM,\ T_BT_AM\big).
\]
Fix a tolerance \(\eta\ge 0\). If \(\Delta_{\mathrm{comm}}\le \eta\), we say \(T_A,T_B\) are \emph{soft\hyp commuting} on \(W\).
\end{definition}

\begin{declaration}[Windowed A/B procedure]\label{L:dec:ab}
On a window \(W\) and degree \(i\):
\begin{enumerate}\itemsep0.2em
  \item \emph{Measure} \(\Delta_{\mathrm{comm}}(M;A,B)\) on \(\mathbf{T}_\tau\mathbf{P}_i\) (same window and \(\tau\) as the budget).
  \item \emph{Accept} soft\hyp commuting if \(\Delta_{\mathrm{comm}}\le\eta\); otherwise \emph{fallback} to a fixed order (e.g.\ \(T_B\circ T_A\)).
  \item \emph{Record} \(\Delta_{\mathrm{comm}}\) as an \(\delta^{\mathrm{alg}}\) contribution if soft\hyp commuting fails; add it to \(\Sigma\delta_W(i)\).
  \item \emph{Log} \(\eta\), the decision (soft\hyp commute or fixed order), and the numerical \(\Delta_{\mathrm{comm}}\) in \texttt{run.yaml} (Appendix~G).
\end{enumerate}
\end{declaration}

\begin{proposition}[Combined bound with A/B policy]\label{L:prop:combined}
For a window \(W\) with the A/B procedure of Declaration~\ref{L:dec:ab}, the pipeline bound in Theorem~\ref{L:thm:budget} holds with
\[
\Sigma\delta_W(i)\ =\ \sum_{j}\delta_j(i,\tau_j)\ +\ \sum_{\text{A/B fails}} \Delta_{\mathrm{comm}}(M;A,B),
\]
and is non\hyp increasing under any subsequent \(1\)\hyp Lipschitz persistence post\hyp processing.
\end{proposition}

\begin{proof}[Proof sketch]
Immediate from Theorem~\ref{L:thm:budget} by including the A/B residuals as per\hyp step additive costs.
\end{proof}

\begin{remark}[Best practices]
Use a canonical ordering for multiple axes (e.g.\ priority by axis type) and apply A/B tests only to adjacent pairs; record all policies and tolerances. For nested torsions (Appendix~K, Proposition~\ref{K:prop:nested}), skip the test (order independence holds).
\end{remark}

% -------------------------
\subsection*{L.6. Edge cases, pitfalls, and window coherence}
% -------------------------
\begin{itemize}\itemsep0.25em
  \item \emph{Mismatched windows or thresholds.} All commutation bounds must be computed on the \emph{same} window and collapse thresholds \(\tau\) as used for B-Gate\(^{+}\) and for the \(\delta\)\hyp ledger; otherwise budget accounting and pasting (Appendix~J) become invalid.
  \item \emph{Degree mixing.} When reporting a total over degrees (e.g.\ \(\muc,\nuc\)), a small A/B residual in one degree can be obscured. Keep both per\hyp degree and aggregated logs (Appendix~G).
  \item \emph{Cascaded non\hyp commutation.} Pairwise soft\hyp commuting does not imply global commutation across three or more axes; use a fixed canonical order and log all residuals.
\end{itemize}

% -------------------------
\subsection*{L.7. Minimal pseudocode (pipeline budget and ledger)}
% -------------------------
The following schematic code illustrates how to accumulate windowed commutation budgets in a pipeline; it is consistent with the \texttt{run.yaml} schema in Appendix~G.

\begin{verbatim}
# Mirror–Collapse commutation ledger for one window W, degree i
def window_budget(steps, i, tau):
    """
    steps: list of pipeline steps on window W, each tagged as:
           - ("mirror", Mirror_j, tau_j, delta_ij)   # δ_j(i, τ_j)
           - ("reflectors", T_A, T_B, eta, Delta)    # A/B with tolerance η, measured Δ
           - ("post", functor)                       # 1-Lipschitz post-processing
    """
    Sigma = 0.0
    for s in steps:
        if s[0] == "mirror":
            _, _, tau_j, delta_ij = s
            Sigma += delta_ij
        elif s[0] == "reflectors":
            _, T_A, T_B, eta, Delta = s
            if Delta > eta:
                Sigma += Delta           # fallback; record into δ^{alg}
            else:
                pass                     # soft-commuting; no cost added
        elif s[0] == "post":
            pass                         # 1-Lipschitz => no increase
    return Sigma

# Gate check (B-side, after collapse), same window and τ
def gate_passed(Sigma_delta, gap_tau):
    return (gap_tau > Sigma_delta)
\end{verbatim}

\medskip
\noindent\textbf{Summary.}
If \(\Mirror\) is \(1\)\hyp Lipschitz and there is a natural \(2\)\hyp cell controlling its commutator with \(C_\tau\) by \(\delta(i,\tau)\) (uniform in \(F\)), then truncation preserves this control:
\[
d_{\mathrm{int}}\Big(\mathbf{T}_\tau\mathbf{P}_i(\Mirror C_\tau F),\ \mathbf{T}_\tau\mathbf{P}_i(C_\tau\Mirror F)\Big)\ \le\ \delta(i,\tau).
\]
For multi-stage pipelines the controls \emph{add}, are \emph{uniform in \(F\)}, and are \emph{non\hyp increasing} under \(1\)\hyp Lipschitz post\hyp processing (Theorem~\ref{L:thm:budget}).
Operationally, A/B tests for non\hyp nested reflectors provide a \emph{soft\hyp commuting} decision per window; failures fall back to a fixed order and are logged as \(\delta^{\mathrm{alg}}\) contributions (Declaration~\ref{L:dec:ab}, Proposition~\ref{L:prop:combined}).
All items are computed \emph{after} collapse on the B\hyp side single layer, logged per window (\texttt{run.yaml}, Appendix~G), and pasted globally via Restart/Summability (Appendix~J).



% =========================
\section*{Appendix M. (Optional) Lax Monoidal Compatibility [Spec + Windowed Usage + Budget Integration]}
\phantomsection
\addcontentsline{toc}{section}{Appendix M. (Optional) Lax Monoidal Compatibility}
\label{M:lax-monoidal}

\bigskip

\noindent
\textbf{Status.} All claims marked [Spec] are windowed, reproducible contracts intended for operational use. They are compatible with the standard 1D constructible persistence setting and do not assert strict monoidality beyond the stated hypotheses. Full derivations are included so that no further reinforcement is required.

\subsection*{M.0. Standing conventions and scope}

\paragraph{Ground field and categories.}
Fix a field \(\kk\).
Let \(\Pers^\mathrm{c}_\kk\) denote the category of constructible (1-parameter) persistence modules \(M:\RR\to\Vect_\kk\), i.e.\ functors that are pointwise finite dimensional and have locally finite sets of critical parameters (equivalently, barcodes locally finite).
We write \(\Vect_\kk\) for vector spaces over \(\kk\).
Filtered chain complexes will be denoted \(F=(F^t)_{t\in\RR}\), each \(F^t\) a bounded-below chain complex of finite-dimensional \(\kk\)-vector spaces, with nondecreasing inclusions in \(t\) and locally finite changes.

\paragraph{Interleaving metric and truncation.}
We use the interleaving metric \(\dint\) on constructible 1D persistence modules; in this setting \(\dint\) coincides with the bottleneck distance on barcodes.
We assume the existence of:
(i) an exact, functorial, filtration-truncation endofunctor \(\TT:\Pers^\mathrm{c}_\kk\to\Pers^\mathrm{c}_\kk\) for each \(\tau>0\) (``collapse'' or ``truncation'' at scale \(\tau\)), and
(ii) a chain-level operation \(\CT\) producing a filtered complex \(\CT F\) whose persistent homology is \(\TT\) of that of \(F\).
We assume \(\TT\) is \(1\)-Lipschitz with respect to \(\dint\) and exact on short exact sequences of persistence modules.
(These properties hold for many standard truncation/collapse operators; we take them here as standing assumptions to avoid external cross-references.)

\paragraph{Windowing and measurability.}
Fix a MECE, right-open windowing of \([0,\infty)\) into intervals \(\mathcal{W}=\{[w_j,w_{j+1})\}_{j\in J}\) with \(0=w_0<w_1<\cdots\), locally finite on bounded ranges.
For a filtered complex \(F\) and degree \(i\), the Betti function \(\betti_i(F;t):=\dim_\kk H_i(F^t)\) is piecewise constant with locally finite jumps; hence it is measurable and integrable on bounded windows.

\paragraph{Energy functional.}
For a window size \(\sigma>0\) and degree \(i\),
\[
  \PE_i^{\le \sigma}(F)\ :=\ \int_0^\sigma \betti_i(F;t)\,dt.
\]

and similarly \(\PE_i^{\le \sigma}(M)\) for a persistence module \(M\), via any filtered presentation.
All integrals are Lebesgue integrals; by constructibility on bounded windows, they reduce to finite sums.

\paragraph{Pointwise tensor.}
On the filtered-complex side and on persistence modules we use the pointwise tensor:
\[
  (M\otimes N)(t) := M(t)\otimes_\kk N(t),\qquad (F\otimes G)^t := F^t\otimes_\kk G^t.
\]
All tensor products are over \(\kk\).

\paragraph{[Spec] scope.}
Results labeled [Spec] are reproducible windowed contracts that depend only on the hypotheses stated herein. They interoperate with budget accounting (Section~M.5) and with \(\tau\)-sweeps (Section~M.4). No assertion of strict monoidality is made beyond the lax statements we prove.

\bigskip

\subsection*{M.1. Hypotheses and the laxator}

We work under the following hypotheses.

\begin{description}[leftmargin=2.2em,labelindent=0em]
  \item[(M1) Exact, constructible tensor.]
  The pointwise tensor \(\otimes\) is exact and biadditive, and preserves constructibility on bounded windows: for any \(M,N\in \Pers^\mathrm{c}_\kk\), the set of critical parameters of \(M\otimes N\) on a bounded window is locally finite, and \(\dim_\kk (M(t)\otimes_\kk N(t))<\infty\). Likewise for filtered complexes \(F,G\).
  \item[(M2) K\"unneth over a field (pointwise).]
  For each \(t\in\RR\) and each \(i\in\ZZ\),
  \[
    H_i\big((F\otimes G)^t\big)\ \cong\ \bigoplus_{p+q=i} H_p(F^t)\otimes_\kk H_q(G^t),
  \]
  naturally in \((F,G)\) and \(t\). Over a field, \(\Tor\)-terms vanish.
  \item[(M3) Lax compatibility for collapse.]
  There is a natural transformation (the laxator) in the homotopy setting (up to filtered quasi-isomorphism)
  \[
    \lambda_{\tau,F,G}:\ \CT(F\otimes G)\ \Longrightarrow\ \CT F\ \otimes\ \CT G,
  \]
  natural in \(F,G\) and \(\tau\).
\end{description}

We will occasionally strengthen (M3) to:

\medskip
\noindent
\(\mathbf{(M3^+)}\) For all \(t\in\RR\) and all degrees \(i\), the induced map in homology
\[
  H_i\big(\lambda_{\tau,F,G}^t\big):\ H_i\big((\CT(F\otimes G))^t\big)\ \longrightarrow\ H_i\big((\CT F\otimes \CT G)^t\big)
\]
is a monomorphism (equivalently, ranks do not decrease under \(\lambda\) at each \((t,i)\)).

\begin{remark}[Intervals under tensor]\label{rem:interval-tensor}
For interval modules over \(\kk\),
\[
  \kk_{[a,b)}\otimes \kk_{[c,d)} \ \cong\\n  \begin{cases}
     \kk_{[\max\{a,c\},\ \min\{b,d\})}, & \text{if } [a,b)\cap[c,d)\neq\emptyset,\\\n    0, & \text{otherwise}.\n  \end{cases}
\]
Thus tensor intersects lifespans; at the barcode level this is the K\"unneth rule over a field.
\end{remark}

\begin{remark}[Preservation of constructibility]\label{rem:constructible}
Since on any bounded window only finitely many bars of \(M\) and \(N\) meet, the critical set of \(M\otimes N\) is contained in the finite union of the critical sets of \(M\) and \(N\); hence it is locally finite. Pointwise finite dimensionality is preserved because \(\dim_\kk (M(t)\otimes_\kk N(t)) = \dim_\kk M(t)\cdot \dim_\kk N(t) < \infty\).
\end{remark}

\begin{remark}[Monotonicity scope]
Tensor is neither purely inclusion-type nor deletion-type in general. We therefore confine monotonicity statements to windowed energy upper bounds and to the \((\mathrm{M3}^+)\) regime, where the laxator is pointwise monomorphic in homology.
\end{remark}

\bigskip

\subsection*{M.2. Windowed energy via overlap integrals}

\begin{definition}[Betti function and windowed energy]
For a filtered complex \(F\) and degree \(i\), set \(\betti_i(F;t):=\dim_\kk H_i(F^t)\) for \(t\in\RR\). For \(\sigma>0\) define the clipped Betti integral (windowed energy)
\[
  \PE_i^{\le \sigma}(F)\ :=\ \int_0^\sigma \betti_i(F;t)\,dt.
\]
For a persistence module \(M\) we define \(\betti_i(M;t)\) and \(\PE_i^{\le\sigma}(M)\) analogously, using any filtered presentation (well-defined by exactness).
\end{definition}

\begin{theorem}[Convolution-type upper bound; equality under K\"unneth]\label{thm:conv}
Assume \emph{(M1)--(M2)}. Then for all filtered complexes \(F,G\), all degrees \(i\), and all windows \(\sigma>0\),
\[
  \PE_i^{\le \sigma}(F\otimes G)\ \le\ \sum_{p+q=i}\ \int_0^\sigma \betti_p(F;t)\,\betti_q(G;t)\,dt,
\]
where the sum \(\sum_{p+q=i}\) is finite due to bounded homological range. Consequently,
\[
  \PE_i^{\le \sigma}(F\otimes G)\ \le\ \sum_{p+q=i}\Big(\sup_{t\in[0,\sigma]}\betti_q(G;t)\Big)\,\PE_p^{\le \sigma}(F),
\]
and symmetrically with \((F,p)\leftrightarrow(G,q)\).
If, moreover, the pointwise K\"unneth isomorphism of \emph{(M2)} holds, then equality holds:
\[
  \PE_i^{\le \sigma}(F\otimes G)\ =\ \sum_{p+q=i}\ \int_0^\sigma \betti_p(F;t)\,\betti_q(G;t)\,dt.
\]
\end{theorem}

\begin{proof}
By (M2), for each \(t\) the homology of \((F\otimes G)^t\) decomposes as \(\bigoplus_{p+q=i}H_p(F^t)\otimes H_q(G^t)\), whence
\[
  \betti_i(F\otimes G;t)\ \le\ \sum_{p+q=i}\betti_p(F;t)\,\betti_q(G;t),
\]
with equality when K\"unneth isomorphism holds. The functions involved are measurable and piecewise constant on bounded windows. Integrate both sides over \([0,\sigma]\) and apply Tonelli/Fubini; finiteness follows from constructibility. The second inequality is obtained by extracting \(\sup_{t\in[0,\sigma]}\betti_q(G;t)\) from the integral. 
\end{proof}

\begin{remark}[Barcode-level interpretation]
In the interval-decomposable case (1D constructible over a field), \(\betti_p(F;t)\) counts the number of \(p\)-bars alive at \(t\). The integrand \(\betti_p(F;t)\,\betti_q(G;t)\) counts ordered pairs of alive bars; the overlap integral sums the lengths of pairwise intersections, matching tensor-as-intersection (Remark~\ref{rem:interval-tensor}).
\end{remark}

\bigskip

\subsection*{M.3. Collapse, lax monoidality, and energy dominance}

\begin{proposition}[Collapsed convolution bound]\label{prop:collapsed-conv}
Assume \emph{(M1)--(M3)}. For any \(\tau,\sigma>0\) and all degrees \(i\),
\[
  \PE_i^{\le \sigma}\big(\CT F \otimes \CT G\big)\ \le\ \sum_{p+q=i}\ \int_0^\sigma \betti_p(\CT F;t)\,\betti_q(\CT G;t)\,dt.
\]
Equality holds when the pointwise K\"unneth isomorphism holds for \((\CT F,\CT G)\).
\end{proposition}

\begin{proof}
Apply Theorem~\ref{thm:conv} to the filtered complexes \(\CT F\) and \(\CT G\).
\end{proof}

\begin{theorem}[Energy dominance under a monomorphic laxator]\label{thm:mono-dominance}
Assume \emph{(M1)--(M3)} and \emph{(M3$^+$)}. Then for all \(\tau,\sigma>0\) and all degrees \(i\),
\[
  \PE_i^{\le \sigma}\big(\CT(F\otimes G)\big)\ \le\ \PE_i^{\le \sigma}\big(\CT F \otimes \CT G\big).
\]
Consequently, combining with Proposition~\ref{prop:collapsed-conv},
\[
  \PE_i^{\le \sigma}\big(\CT(F\otimes G)\big)\ \le\ \sum_{p+q=i}\ \int_0^\sigma \betti_p(\CT F;t)\,\betti_q(\CT G;t)\,dt.
\]
\end{theorem}

\begin{proof}
By \((\mathrm{M3}^+)\), for each \(t\) the map \(H_i(\lambda_{\tau,F,G}^t)\) is injective, hence
\[
  \betti_i\big(\CT(F\otimes G);t\big)\ \le\ \betti_i\big(\CT F\otimes \CT G;t\big)
\]
for all \(t\). Integrate over \([0,\sigma]\).
\end{proof}

\begin{remark}[Necessity of monomorphy]
Without \((\mathrm{M3}^+)\), \(\lambda_{\tau,F,G}\) could a priori decrease ranks at specific \((t,i)\), and one cannot guarantee \(\PE\)-nonincrease along \(\lambda\). In that regime, only the (symmetric) convolution bounds of Theorem~\ref{thm:conv} and Proposition~\ref{prop:collapsed-conv} are safe.
\end{remark}

\bigskip

\subsection*{M.4. Interaction with \texorpdfstring{$\tau$}{tau}-sweeps and stability bands}

\begin{definition}[\(\tau\)-sweep and stability band]
For a fixed degree \(i\) and window \([0,\sigma]\), a \(\tau\)-sweep probes \(\tau>0\) to identify stability bands where the functor \(\TT\) is locally constant in \(\tau\) (e.g.\ no bars of length in \((\tau-\eps,\tau+\eps)\) intersect \([0,\sigma]\)).
Within a stability band, small \(\tau\)-variations do not alter \(\betti_i(\TT M; t)\) on \([0,\sigma]\).
\end{definition}

\begin{proposition}[Robustness within stability bands]\label{prop:stability-band}
Fix \(i\) and \(\sigma>0\). Within a stability band for \(\tau\), the inequality
\[
  \PE_i^{\le \sigma}\big(\CT(F\otimes G)\big)\ \le\ \PE_i^{\le \sigma}\big(\CT F \otimes \CT G\big)
\]
of Theorem~\ref{thm:mono-dominance} is robust under small perturbations of \(\tau\): both sides remain constant in \(\tau\) on the band.
\end{proposition}

\begin{proof}
On a stability band, \(\betti_i(\CT(\cdot);t)\) is locally constant in \(\tau\) for all \(t\in[0,\sigma]\), hence the integrals defining the \(\PE\) functionals are constant.
\end{proof}

\bigskip

\subsection*{M.5. Budget integration and quantitative gaps [Spec]}

In operational pipelines, one may wish to track commutation defects and measurement tolerances.

\begin{definition}[Budget ledger]\label{def:budget-ledger}
Fix a window $[0,\sigma]$. A budget ledger records nonnegative error terms $\delta_\ell$
(from, e.g., numerical integration tolerance, discretization, or measured interleaving gaps)
and aggregates $\Sigma\delta := \sum_\ell \delta_\ell$.
Downstream $1$-Lipschitz operations (e.g.\ $\TT$, homology, degree projections) do not increase $\Sigma\delta$.
\end{definition}

\begin{proposition}[Integrating a measured laxity gap]\label{prop:lax-gap}
Suppose a metric comparison is available at the persistence layer:
\[
  \dint\!\Big(\TT \mathbf{P}_i\big(\CT(F\otimes G)\big),\ \TT \mathbf{P}_i\big(\CT F\otimes \CT G\big)\Big)\ \le\ \delta_{\mathrm{lax}}
\]
for some degree \(i\) and window \([0,\sigma]\). Then \(\delta_{\mathrm{lax}}\) may be recorded in the budget ledger for \([0,\sigma]\). Subsequent \(1\)-Lipschitz processing preserves this bound.
\end{proposition}

\begin{remark}[When no metric control is available]
In the absence of such metric control, rely on the energy inequalities (Theorem~\ref{thm:conv}, Theorem~\ref{thm:mono-dominance}) and budget only measurement and A/B residuals arising from the pipeline.
\end{remark}

\bigskip

\subsection*{M.6. Edge cases and pitfalls}

\begin{itemize}[leftmargin=2em]
  \item Non-exact ``tensor'' surrogates. If the operation used in place of \(\otimes\) is not exact and biadditive, K\"unneth may fail and the convolution bound becomes invalid.
  \item Unverified K\"unneth. If (M2) is not verified in the context at hand, use the inequality form of Theorem~\ref{thm:conv}; do not claim equality.
  \item Missing monomorphy. Without (M3$^+$), the comparison \(\CT(F\otimes G)\to \CT F\otimes \CT G\) need not be energy-nonincreasing.
  \item Window mismatch. All measurements (Betti integrals, laxator rank checks) must use the same MECE, right-open windows and the same \(\tau\); otherwise splicing across windows may fail.
\end{itemize}

\bigskip

\subsection*{M.7. Worked examples and tests}

\begin{example}[Single-interval bars]
Let \(\mathbf{P}_p(F)=\kk_{[a,b)}\), \(\mathbf{P}_q(G)=\kk_{[c,d)}\) be the only nontrivial bars (other degrees vanish). Then by Remark~\ref{rem:interval-tensor},
\[
  \mathbf{P}_{p+q}(F\otimes G)\ \cong\ \kk_{[\max\{a,c\},\,\min\{b,d\})},
\]
and for any \(\sigma>0\),
\[
  \PE_{p+q}^{\le \sigma}(F\otimes G)\n  \ =\ \lambda\!\Big([\max\{a,c\},\,\min\{b,d\})\cap[0,\sigma]\Big),
\]
where \(\lambda(\cdot)\) denotes Lebesgue measure (length).
\end{example}

\begin{example}[Collapsed intervals]
If \(b-a\le \tau\) or \(d-c\le \tau\), then \(\CT F\) or \(\CT G\) kills the corresponding bar. The right-hand side of the convolution bound reduces accordingly; if both bars are killed, then under (M3$^+$) the left-hand side \(\PE_{p+q}^{\le \sigma}(\CT(F\otimes G))\) must vanish.
\end{example}

\begin{proposition}[Test model for \((\mathrm{M3}^+)\): finite direct sums of intervals]\label{prop:test-m3plus}
Let \(F\simeq \bigoplus_r \kk_{I_r}[-p_r]\) and \(G\simeq \bigoplus_s \kk_{J_s}[-q_s]\) be finite direct sums of interval modules (shifted in homological degree). Suppose \(\CT\) removes all summands with length \(\le \tau\) and acts as the identity on others. Define \(\lambda_{\tau,F,G}\) on summands by the canonical inclusions
\[
  \CT(\kk_{I_r}\otimes\kk_{J_s})\ \hookrightarrow\ \CT\kk_{I_r}\otimes \CT\kk_{J_s},
\]
extended additively. Then \((\mathrm{M3}^+)\) holds, and for all \(i,\sigma>0\),
\[
  \PE_i^{\le \sigma}\big(\CT(F\otimes G)\big)\ \le\ \PE_i^{\le \sigma}\big(\CT F\otimes \CT G\big).
\]
\end{proposition}

\begin{proof}
On intervals, tensor is intersection; \(\CT\) either kills short intervals or leaves them unchanged. If either factor is killed, the target summand on the right is \(0\), and the source summand on the left is also \(0\) (intersection with a killed factor is empty), so the induced map is injective. If both factors survive, \(\CT\) acts as identity and the inclusion is literal. Additivity preserves injectivity. Energy nonincrease follows by integrating ranks of monomorphisms degreewise.
\end{proof}

\bigskip

\subsection*{M.8. Formal underpinnings: constructibility, measurability, exactness}

For completeness we record the basic facts used throughout.

\begin{lemma}[Constructibility preserved by tensor]\label{lem:constructible-tensor}
Assume (M1). If \(M,N\in \Pers^\mathrm{c}_\kk\), then \(M\otimes N\in \Pers^\mathrm{c}_\kk\). If \(F,G\) are constructible filtered complexes, then \(F\otimes G\) is constructible.
\end{lemma}

\begin{proof}
See Remark~\ref{rem:constructible}. For filtered complexes, apply the same argument degreewise; bounded homological range on windows is preserved by tensor over a field.
\end{proof}

\begin{lemma}[Measurability and finiteness of energy]
For constructible \(F\), each \(\betti_i(F;\cdot)\) is piecewise constant with locally finite jumps. Hence \(\PE_i^{\le \sigma}(F)<\infty\) for all \(\sigma>0\).
\end{lemma}

\begin{proof}
Constructibility implies only finitely many sublevel changes on bounded windows. The Betti function is a finite sum of characteristic functions of intervals; integrability follows.
\end{proof}

\begin{lemma}[Exactness and 1-Lipschitzness of \(\TT\)]
Assume \(\TT\) is exact on short exact sequences and \(1\)-Lipschitz for \(\dint\). Then for any morphism \(f:M\to N\) in \(\Pers^\mathrm{c}_\kk\),
\[
  \dint(\TT M,\TT N)\ \le\ \dint(M,N).
\]
\end{lemma}

\begin{proof}
This is a standing property of \(\TT\). In barcode terms, \(\TT\) is realized by interval truncation which is nonexpansive for bottleneck distance; exactness follows from functoriality and additivity on intervals.
\end{proof}

\bigskip

\subsection*{M.9. Operational checklist (windowed, reproducible) [Spec]}

For each experiment window \([0,\sigma]\):
\begin{itemize}[leftmargin=2em]
  \item Record the tensor context: pairs \((F,G)\), monitored degrees \(i\), windows \([0,\sigma]\), and collapse thresholds \(\tau\).
  \item State whether K\"unneth (M2) is assumed/verified on the chosen windows; if not, use the \(\le\) form of Theorem~\ref{thm:conv}.
  \item State whether (M3) and optionally (M3$^+$) are assumed/verified on the chosen windows; e.g., check monomorphy of \(H_i(\lambda_{\tau,F,G}^t)\) at sampled \(t\).
  \item Fix numerical tolerances for spectra/Betti integrations and any clipping thresholds; enter them in the budget ledger as \(\delta\)-terms.
  \item If a measured laxity gap \(\delta_{\mathrm{lax}}\) is available (Proposition~\ref{prop:lax-gap}), record it and include it in \(\Sigma\delta\).
\end{itemize}

\bigskip

\subsection*{M.10. Summary of contracts [Spec]}

Under exact pointwise tensor and a collapse-compatible laxator \(\CT(F\otimes G)\Rightarrow \CT F\otimes \CT G\), the clipped Betti integral of a tensor admits robust, windowed upper bounds expressed by overlap integrals of Betti curves (Theorem~\ref{thm:conv}, Proposition~\ref{prop:collapsed-conv}). With the additional monomorphy hypothesis \((\mathrm{M3}^+)\), \(\CT(F\otimes G)\) is energy-dominated by \(\CT F\otimes \CT G\) on each window (Theorem~\ref{thm:mono-dominance}). These contracts are stable under \(\tau\)-sweeps within stability bands (Proposition~\ref{prop:stability-band}), are compatible with budget accounting (Proposition~\ref{prop:lax-gap}), and suffice for operational, windowed use without further supplementation.

\bigskip

\subsection*{M.11. Formalization blueprint (Lean/Coq) [Spec]}

A minimal API includes:
\begin{itemize}[leftmargin=2em]
  \item An exact, biadditive \texttt{tensor} on \(\Pers^\mathrm{c}_\kk\) preserving constructibility; a lemma \texttt{kun\_neth} yielding
  \[
    \betti_i(F\otimes G;t)=\sum_{p+q=i}\betti_p(F;t)\,\betti_q(G;t)
  \]
  over a field.
  \item A natural transformation \texttt{laxator} \(\lambda_{\tau,F,G}:\CT(F\otimes G)\Rightarrow \CT F\otimes \CT G\).
  \item An optional hypothesis \texttt{laxator\_mono} encoding \((\mathrm{M3}^+)\) degreewise in homology.
  \item Overlap-integral bounds \texttt{PE\_conv\_bound} and \texttt{PE\_collapsed\_bound} corresponding to Theorem~\ref{thm:conv} and Proposition~\ref{prop:collapsed-conv}, with window parameters explicit.
  \item Hooks to record optional metric gaps as \(\delta\)-terms for a pipeline budget.
\end{itemize}



% =========================
\section*{Appendix N. Projection Formula and Base Change [Spec + Windowed Protocol + Budget Integration]}
% =========================
\phantomsection
\addcontentsline{toc}{section}{Appendix N. Projection Formula and Base Change}
\refstepcounter{section}
\label{N:pf-bc}

\paragraph{Standing conventions.}
We work over a coefficient \emph{field} \(\Lambda\) (e.g.\ a base field \(k\) or, at [Spec]-level, a Novikov field), and all statements below are phrased uniformly for \(\Lambda\).
All persistence modules are constructible (locally finite on bounded windows).
Filtered (co)limits are computed objectwise in \([\mathbb{R},\mathsf{Vect}_\Lambda]\) under the scope policy of Appendix~A, Remark~\ref{A:rk:filtered-colimits}, and then (when stated) returned to the constructible range.
The interleaving metric \(d_{\mathrm{int}}\) (=\ bottleneck in the constructible \(1\)D setting) is used throughout.
Truncation \(\mathbf{T}_\tau\) is exact and \(1\)\hyp Lipschitz (Appendix~A, Proposition~\ref{A:prop:lipschitz}); we keep the filtered-complex functor \(C_\tau\) \emph{up to f.q.i.} and write \(\mathbf{P}_i\) for degree–\(i\) persistence.
Global conventions: \(\Ext^1\) is always taken against \(\Lambda[0]\) (so we write \(\Ext^1(\mathcal{R}(C_\tau F),\Lambda)=0\)); the energy exponent satisfies \(\alpha>0\) (default \(\alpha=1\)); windows are MECE and right–open (Appendix~G).
References to “infinite bars/generic dimension” point to Appendix~D, Remark~\ref{rem:D-generic-dim}.
Monotonicity claims follow the global policy: deletion-type only (nonincreasing), inclusion-type merely stable/nonexpansive (Appendix~E).

% Number subsections in Appendix N as N.1, N.2, ...
\setcounter{subsection}{0}
\renewcommand\thesubsection{N.\arabic{subsection}}
\makeatletter
\renewcommand\@seccntformat[1]{\csname the#1\endcsname.\quad}
\makeatother

% -------------------------
\subsection{Hypotheses (PF/BC layer) and normalizations}
\label{N:hyp}
% -------------------------
We fix a class of filtered spaces and maps \(f:X\to Y\) for which the usual six–functor formalism is available on
\(D^b_c(\mathrm{Shv}_\Lambda(-))\) (bounded derived category of constructible \(\Lambda\)–sheaves), and adopt:

\begin{itemize}\itemsep0.35em
  \item[(N0)] \textbf{Coefficients.} \(\Lambda\) is a field; all objects have \emph{finite Tor-dimension} (no \(\mathrm{Tor}\) corrections).
  \item[(N1)] \textbf{Finiteness/constructibility.} All sheaves are constructible; the standard \(t\)-structure is used; per-object (co)homology is finite dimensional.
  \item[(N2)] \textbf{Proper/smooth hypotheses.} Projection formula (PF) and base change (BC) are taken under the usual hypotheses:
  \begin{itemize}\itemsep0.1em
    \item PF: for \(f\) \emph{proper}, \(Rf_\ast(A\otimes^\mathbf{L} f^\ast B)\simeq Rf_\ast A\otimes^\mathbf{L} B\).
    \item BC: for a Cartesian square with \(f\) proper (or smooth with the appropriate \(f^!\) variant),
    \(Lg^\ast Rf_\ast A \simeq Rf'_\ast Lg'{}^\ast A\).
  \end{itemize}
  \item[(N3)] \textbf{Degree normalization and objectwise evaluation.}
  We use the \emph{cohomological} convention on \(D^b_c\) and evaluate realizations \emph{objectwise in \(t\)}:
  \[
     \mathcal{R}(F)^t \cong \mathcal{R}(F^t),\qquad
        \mathbf{P}_i(F)(t)\ \cong\ H_i(F^t)\ \cong\ H^{-i}\!\big(\mathcal{R}(F^t)\big).
  \]
  Hence \(\mathbf{P}_i\) reads off the \((-i)\)-th cohomology sheaf along the filtration.
  Any geometric shift from \(f^!\) (smooth case) is \emph{absorbed} by this bookkeeping.
  \item[(N4)] \textbf{Tensor.} The tensor is pointwise in \(t\):
  \((A\otimes^\mathbf{L} B)^t \cong A^t\otimes^\mathbf{L} B^t\), exact over the field \(\Lambda\);
  constructibility is preserved (Appendix~H justifies Tonelli on bounded windows).
\end{itemize}

\begin{remark}[Scope and return to constructible]
All PF/BC comparisons below are formed in the derived category, computed objectwise in \(t\), and then passed to the persistence layer via \(\mathbf{P}_i\).
Any filtered colimit is taken in \([\mathbb{R},\mathsf{Vect}_\Lambda]\) under Appendix~A’s scope policy and \emph{returned to} \(\Pers^{\mathrm{ft}}_\Lambda\) (by verification of constructibility or by applying \(\mathbf{T}_\tau\)).
\end{remark}

% -------------------------
\subsection{Projection formula / base change at the persistence layer}
\label{N:pfbc-persistence}
% -------------------------
Let \(f:X\to Y\) and a Cartesian square
\[
\vcenter{\xymatrix{\nX'\ar[r]^{g'}\ar[d]_{f'} & X\ar[d]^f\\\nY'\ar[r]^g & Y\n}}
\]
satisfy \textup{(N0)–(N2)}.
For filtered complexes \(F\) on \(X\) and \(G\) on \(Y\), write \(\mathcal{R}(F),\mathcal{R}(G)\) for their realizations in \(D^b_c\), computed objectwise in \(t\).

\begin{theorem}[PF/BC transported to \texorpdfstring{$\mathbf{P}_i$ and $\mathbf{T}_\tau$}{P_i and T_tau} \textup{[Spec]}]\label{N:thm:pf-bc}
Under \textup{(N0)--(N4)} the following canonical isomorphisms hold, \emph{natural in}
\((i,\tau,f,g,F,G)\) and \emph{up to f.q.i.} on the filtered--complex side; they are asserted \emph{after truncation by $\mathbf{T}_\tau$}:
\begin{align*}
\textup{(PF)}\quad
& \mathbf{T}_\tau\,\mathbf{P}_i\!\big(Rf_\ast(\mathcal{R}(F)\otimes^{\mathbf{L}} f^\ast\mathcal{R}(G))\big)
 \cong
 \mathbf{T}_\tau\,\mathbf{P}_i\!\big(Rf_\ast\mathcal{R}(F)\otimes^{\mathbf{L}}\mathcal{R}(G)\big)
\\[0.25em]
\textup{(BC)}\quad
& \mathbf{T}_\tau\,\mathbf{P}_i\!\big(Lg^\ast Rf_\ast \mathcal{R}(F)\big)
 \cong
 \mathbf{T}_\tau\,\mathbf{P}_i\!\big(Rf'_\ast Lg'{}^\ast \mathcal{R}(F)\big).
\end{align*}
\end{theorem}

\begin{proof}[Proof sketch]
PF and BC are canonical isomorphisms in \(D^{b}_{c}\) under \textup{(N0)--(N2)}.
Evaluate objectwise in \(t\) (N3), identify \(\mathbf{P}_i\) with \((-i)\)-cohomology in \(t\), and then apply \(\mathbf{T}_\tau\).
Exactness of \(\mathbf{T}_\tau\) (Appendix~A, Theorem~\ref{A:thm:localization}) preserves short exact sequences induced by PF/BC on cohomology sheaves; hence isomorphisms descend to the persistence layer \emph{after truncation}.
Naturality in \((f,g,F,G)\) follows from naturality of PF/BC; naturality in \((i,\tau)\) is clear from functoriality of \(\mathbf{P}_i\) and \(\mathbf{T}_\tau\).
\end{proof}

\begin{corollary}[Compatibility with collapse at the filtered-complex layer]
Assume in addition that \(C_\tau\) realizes \(\mathbf{T}_\tau\) after applying \(\mathbf{P}_i\) (Appendix~B) up to f.q.i.
Then the PF/BC isomorphisms of Theorem~\ref{N:thm:pf-bc} hold with \(\mathcal{R}(C_\tau F)\) in place of \(\mathbf{T}_\tau\,\mathbf{P}_i(\mathcal{R}(F))\), \emph{after truncation by \(\mathbf{T}_\tau\) and up to f.q.i.\ on filtered complexes}.
\end{corollary}

\begin{remark}[What is \emph{not} claimed]
We do \emph{not} assert global Lipschitz control for PF/BC beyond the \(1\)\nobreakdash-Lipschitz behavior of \(\mathbf{T}_\tau\) (Appendix~A) and any additional [Spec] commutation controls (Appendix~L). PF/BC are exact identities at the sheaf level; any persistence--level discrepancy indicates violation of hypotheses or implementation drift and must be logged as such (see \S\ref{N:window-protocol}).
\end{remark}

% -------------------------
\subsection{Windowed protocol and reproducible audit}
\label{N:window-protocol}
% -------------------------
All PF/BC audits are \emph{windowed}. The mandatory comparison order is:
\[
\boxed{\ \text{for each } t\ \Longrightarrow\ \text{apply } \mathbf{P}_i\ \Longrightarrow\ \text{apply } \mathbf{T}_\tau\ \Longrightarrow\ \text{compare in }\Pers^{\mathrm{ft}}_\Lambda\ }.
\]

Use the \emph{same} MECE, right–open windows and the \emph{same} \(\tau\) as the rest of the run (Appendix~G). In \texttt{run.yaml}, record per window:
\begin{itemize}\itemsep0.25em
  \item the PF/BC hypothesis set used (proper/smooth, finite Tor, degree normalization);
  \item the functors and objects compared (e.g.\ \(Rf_\ast(\mathcal{R}(F)\otimes f^\ast\mathcal{R}(G))\) vs.\ \(Rf_\ast\mathcal{R}(F)\otimes\mathcal{R}(G)\));
  \item the verdict (isomorphism detected) and, if any numerical/non–ideal drift is observed \emph{after truncation}, its breakdown into \(\delta^{\mathrm{disc}}\) and \(\delta^{\mathrm{meas}}\) together with tolerance(s);
  \item the window and \(\tau\) used for the comparison, matching those used by B–Gate\(^{+}\) and the \(\delta\)–ledger (Appendix~G).
\end{itemize}
When PF/BC hypotheses are satisfied, the algebraic contribution \(\delta^{\mathrm{alg}}\) is \(0\) by design; only discretization/measurement components may be nonzero.

% -------------------------
\subsection{Budget integration and window pasting}
\label{N:budget}
% -------------------------
PF/BC isomorphisms themselves contribute \(\delta^{\mathrm{alg}}=0\) when the hypotheses hold. However, in a \emph{pipeline} that also includes Mirror/Transfer steps and non–nested reflectors (Appendix~L/K), the \(\delta\)–budget on a window \(W\) is:
\[
\Sigma\delta_W(i)\ =\ \sum_{\text{Mirror--Collapse steps}} \delta_j(i,\tau_j)\ +\ \sum_{\text{A/B fails}} \Delta_{\mathrm{comm}}(M;A,B)\ +\ \sum_{\text{PF/BC}} \bigl(\delta^{\mathrm{disc}}+\delta^{\mathrm{meas}}\bigr).
\]
with \(\delta^{\mathrm{disc}},\delta^{\mathrm{meas}}\) for PF/BC typically small or null. This budget is \emph{additive}, \emph{uniform in \(F\)} for Mirror–Collapse (Appendix~L), and \emph{non–increasing} under any subsequent \(1\)\hyp Lipschitz persistence post–processing (Appendix~L, Theorem~\ref{L:thm:budget}). Window pasting then follows from Restart/Summability (Appendix~J): log \(\Sigma\delta_W(i)\) and \(\mathrm{gap}_\tau\) per window and verify \(\mathrm{gap}_\tau>\Sigma\delta_W(i)\).

% -------------------------
\subsection{Ext–tests under change of functor / coefficients}
\label{N:ext-tests}
% -------------------------
PF/BC isomorphisms transport \(\Ext^1\)–tests along canonical identifications:

\begin{proposition}[Portability of the \(\Ext^1\)–test \textup{(sheaf layer)}]
Under \textup{(N0)–(N2)} and Theorem~\ref{N:thm:pf-bc}, for any PF/BC isomorphism \(A \xrightarrow{\sim} B\) in \(D^{b}_{c}\), there is a natural isomorphism
\[
\Ext^1(A,\Lambda)\ \xrightarrow{\ \sim\ }\ \Ext^1(B,\Lambda).
\]
In particular, if \(\Ext^1(\mathcal{R}(C_\tau F),\Lambda)=0\), then \(\Ext^1\) also vanishes for any PF/BC partner of \(\mathcal{R}(C_\tau F)\).
\end{proposition}

\begin{remark}[Bridge stays one–way]
The one–way bridge \(\mathrm{PH}_1\Rightarrow \Ext^1\) (Appendix~C) is unchanged: PF/BC \emph{transport} the test across equivalent sheaf–theoretic descriptions. No converse or new implication is claimed.
\end{remark}

% -------------------------
\subsection{Implementation notes and checkpoints}
\label{N:impl}
% -------------------------
\begin{itemize}\itemsep0.35em
  \item \textbf{Finite windows / constructibility.} On bounded \(t\)–windows, bar events are finite (Appendix~H); PF/BC are computed objectwise in \(t\) and preserved by \(\mathbf{T}_\tau\).
  \item \textbf{Exactness bookkeeping.} Reductions to persistence use only: (i) PF/BC hold in \(D^b_c\); (ii) \(\mathbf{P}_i\) reads off \((-i)\)–cohomology in \(t\); (iii) \(\mathbf{T}_\tau\) is exact (Appendix~A, Theorem~\ref{A:thm:localization}) and \(1\)–Lipschitz (Appendix~A, Proposition~\ref{A:prop:lipschitz}); (iv) filtered colimits respect the scope policy (Appendix~A, Remark~\ref{A:rk:filtered-colimits}).
  \item \textbf{Proper/smooth reminder.} We use the cohomological convention, and any \(f^!\)–induced shifts in the smooth case are absorbed by (N3). PF/BC are invoked under the proper/smooth hypotheses (N2).
  \item \textbf{Window coherence.} All PF/BC audits must use the \emph{same} windows and \(\tau\) as those employed by B–Gate\(^{+}\) and the \(\delta\)–ledger (Appendix~G); otherwise budget accounting and pasting (Appendix~J) become invalid.
\end{itemize}

% -------------------------
\subsection{Formalization stubs (Lean/Coq) [Spec]}
\label{N:formal}
% -------------------------
A minimal API (cf.\ Appendix~F) includes:
\begin{itemize}\itemsep0.25em
  \item \texttt{pf\_iso}: \(Rf_\ast(A\otimes f^\ast B)\cong Rf_\ast A\otimes B\) under properness; \texttt{bc\_iso}: \(Lg^\ast Rf_\ast A\cong Rf'_\ast Lg'{}^\ast A\) for Cartesian squares (and smooth variants with \(f^!\));
  \item \texttt{to\_pers}: objectwise evaluation in \(t\) plus \(\mathbf{P}_i\) extraction and \(\mathbf{T}_\tau\) application to transport PF/BC to \(\Pers^{\mathrm{ft}}_\Lambda\);
  \item \texttt{pfbc\_pers\_nat}: naturality of these isomorphisms in \((i,\tau,f,g,F,G)\);
  \item hooks to log any residual numeric slack (post–truncation) as \(\delta^{\mathrm{disc}}\), \(\delta^{\mathrm{meas}}\) for the window budget (Appendix~L).
\end{itemize}

\medskip
\noindent\textbf{Summary.}
Under the standard PF/BC hypotheses over a field \(\Lambda\), projection formula and base change descend—via objectwise evaluation in \(t\), \(\mathbf{P}_i\), and the exact truncation \(\mathbf{T}_\tau\)—to canonical, \emph{natural} isomorphisms at the persistence layer, uniformly in \((i,\tau,f,g,F,G)\).
Comparisons must follow the windowed protocol “for each \(t\) \(\to\) persistence \(\to\) collapse \(\to\) compare,” with the \emph{same} windows and \(\tau\) as the rest of the run; any residual numerical drift (post–truncation) is accounted for in the \(\delta\)–ledger and aggregated in the pipeline budget.
These audits integrate seamlessly with Mirror/Transfer commutation (Appendix~L), multi–axis reflectors (Appendix~K), and Restart/Summability pasting (Appendix~J), while keeping the one–way bridge \(\mathrm{PH}_1\Rightarrow \Ext^1\) intact (Appendix~C).



% =========================
\section*{Appendix O. Fukaya Realization \& Stability [Spec + Permitted Ops + $\delta$-Ledger + B-Gate$^{+}$] (Reinforced)}
% =========================
\phantomsection
\addcontentsline{toc}{section}{Appendix O. Fukaya Realization \& Stability}
\refstepcounter{section}
\label{O:fukaya}

\paragraph{Standing conventions.}
We work over a coefficient \emph{field} \(\Lambda\) (e.g.\ a ground field \(k\) or a Novikov field).
All persistence modules are constructible (locally finite on bounded windows).
Filtered (co)limits are computed objectwise in \([\mathbb{R},\mathsf{Vect}_\Lambda]\) under the scope policy of Appendix~A, Remark~\ref{A:rk:filtered-colimits}, and then (when stated) returned to the constructible range.
The interleaving metric \(d_{\mathrm{int}}\) (which agrees with the bottleneck distance in the constructible \(1\)D setting) is used throughout.
Truncation \(\mathbf{T}_\tau\) is exact and \(1\)\hyp Lipschitz (Appendix~A, Proposition~\ref{A:prop:lipschitz}).
Global conventions: \(\Ext^1\) is always taken against \(\Lambda[0]\); the energy exponent satisfies \(\alpha>0\) (default \(\alpha=1\)).
References to “generic fiber dimension / infinite bars” point to Appendix~D, Remark~\ref{rem:D-generic-dim}.
Monotonicity claims follow the global policy: \emph{deletion-type only} (nonincreasing), inclusion-type merely stable/nonexpansive (Appendix~E).
Windows are MECE and right–open; bars are half-open \([b,d)\) (Appendix~H/G).

\begin{remark}[Right–open windows and half-open bars]\label{O:rmk:windows}
Right–open windows and half-open \([b,d)\) bars enforce MECE coverage and avoid double-counting at endpoints; events at a right boundary are attributed to the subsequent window.
\end{remark}

\begin{definition}[Filtered chain model and persistence]
A filtered chain complex over \(\Lambda\) means a chain complex \(C_\bullet\) equipped with an exhaustive, increasing filtration \(\{F^{t}C_\bullet\}_{t\in\mathbb{R}}\) by subcomplexes such that the differential preserves the filtration and continuation data act by filtered maps. For each homological degree \(i\), the degree–\(i\) persistence module is \(t\mapsto H_i(F^{t}C_\bullet)\in \mathsf{Vect}_\Lambda\). Constructibility on bounded windows means only finitely many jumps (break times) occur and vector-space ranks are finite on bounded intervals.
\end{definition}

\begin{definition}[Deletion-type morphism]\label{O:def:deletion}
A morphism \(M\to N\) in \(\Pers_\Lambda\) is \emph{deletion-type} on a window \(W\subset\mathbb{R}\) if, after restricting to \(W\) and post-composing with the truncation \(\mathbf{T}_\tau\) (for any \(\tau\ge 0\)), the induced map can only shorten or remove existing bars and cannot create new bars on \(W\). Equivalently, all deletion-type monotone indicators of Appendix~E are nonincreasing under this morphism.
\end{definition}

% -------------------------
\subsection*{O.1. Realization functor and hypotheses}
% -------------------------
\emph{[Spec]} Fix a Liouville/Weinstein sector \((X,\lambda)\) with a (possibly empty) system of \emph{stops}.
Write \(\mathsf{Fuk}(X;\mathrm{stops})\) for a wrapped/exact/monotone Fukaya-type category for which Floer-theoretic chain models admit an \emph{action filtration}.
We adopt the convention that the \emph{action value is the filtration parameter \(t\)}, increasing with larger action (so sublevel sets \(F^{t}\) mean action \(\le t\)).
We package the chain-level construction into a realization functor
\[
\mathcal{F}:\ \text{(geometric input)}\longrightarrow \mathsf{FiltCh}(\Lambda),
\]
natural in continuation data and stop operations, with degree–\(i\) persistence
\(\mathbf{P}_i\big(\mathcal{F}(-)\big)\in \Pers^{\mathrm{ft}}_\Lambda\) (constructible on bounded windows).
We assume:

\begin{itemize}\setlength{\itemsep}{0.35em}
  \item[(O0)] \textbf{Coefficients/admissibility.} \(\Lambda\) is a field; in the monotone/exact cases and with admissible almost complex structures, action and index filtrations are well defined; continuation solutions have finite energy.

  \item[(O1)] \textbf{Constructibility (action window).} On every bounded action window \([0,\sigma]\) the Floer complexes have finitely many generators and finitely many break times; hence \(\mathbf{P}_i(\mathcal{F}(-))\) is constructible. This local finiteness equally holds with Novikov coefficients on bounded windows.

  \item[(O2)] \textbf{Continuation shift bound.} Any continuation map for a homotopy of data with controlled action shift \(\varepsilon\) induces a filtered chain map whose filtration increase is \(\le \varepsilon\).

  \item[(O3)] \textbf{Stop operations are deletion-type.} Adding a stop or shrinking a sector removes generators and/or increases differentials in a way that corresponds to a \emph{deletion-type} operation at the persistence layer: in any fixed action window no new bars are created.

  \item[(O4)] \textbf{Up to filtered quasi-isomorphism.} Chain models are considered up to filtered quasi-isomorphism; all claims are invariant under f.q.i.\ by exactness of \(\mathbf{T}_\tau\) and the scope policy.
\end{itemize}

\begin{remark}[Action filtration normalization]
Our choice “action value equals filtration parameter” fixes the direction of filtration; monotone time reparametrizations that preserve order act by reindexings and, after normalization, give isometries in \(d_{\mathrm{int}}\).
\end{remark}

% -------------------------
\subsection*{O.2. Stability: continuation and stops}
% -------------------------
\begin{theorem}[Continuation \(1\)\hyp Lipschitz]\label{O:thm:cont}
Under \textup{(O2)}, for any two realizations related by a continuation with action shift \(\varepsilon\),
\[
d_{\mathrm{int}}\!\Big(\mathbf{P}_i(\mathcal{F}_0),\ \mathbf{P}_i(\mathcal{F}_1)\Big)\ \le\ \varepsilon,\qquad\nd_{\mathrm{int}}\!\Big(\mathbf{T}_\tau\mathbf{P}_i(\mathcal{F}_0),\ \mathbf{T}_\tau\mathbf{P}_i(\mathcal{F}_1)\Big)\ \le\ \varepsilon
\]
for all \(i,\tau\ge 0\).
\end{theorem}

\begin{proof}[Proof sketch]
A filtered chain map whose filtration increase is \(\le\varepsilon\) commutes up to shift with the time-translation endofunctors, hence yields an \(\varepsilon\)-interleaving at the persistence level. The first inequality follows. Applying \(\mathbf{T}_\tau\) preserves the bound by its \(1\)\hyp Lipschitz property (Appendix~A, Proposition~\ref{A:prop:lipschitz}).
\end{proof}

\begin{proposition}[Deletion-type monotonicity for stops]\label{O:prop:stops}
Under \textup{(O3)}, adding a stop or shrinking a sector induces, on any window and after \(\mathbf{T}_\tau\), a deletion-type morphism:
for every \(i\) and \(\tau\ge 0\),
\[
\mathbf{T}_\tau\,\mathbf{P}_i\big(\mathcal{F}_{\mathrm{with\ stop}}\big)\ \preceq\ \mathbf{T}_\tau\,\mathbf{P}_i\big(\mathcal{F}_{\mathrm{without\ stop}}\big),
\]
and all deletion-type monotone indicators (Appendix~E) are nonincreasing under this operation.
\end{proposition}

\begin{remark}[Inclusion-type caution]\label{O:rmk:inclusion}
Operations that enlarge the admissible region or remove stops can increase features. We only assert nonexpansivity via continuation when an explicit shift control is available; no deletion-type monotonicity is claimed in that direction.
\end{remark}

\begin{corollary}[Compositionality of shift bounds]\label{O:cor:compose}
If \(\mathcal{F}_0\to\mathcal{F}_1\to\mathcal{F}_2\) are continuations with shifts \(\varepsilon_1,\varepsilon_2\), then
\(d_{\mathrm{int}}(\mathbf{P}_i(\mathcal{F}_0),\mathbf{P}_i(\mathcal{F}_2))\le \varepsilon_1+\varepsilon_2\).
\end{corollary}

% -------------------------
\subsection*{O.3. Towers, comparison map, and diagnostics}
% -------------------------
Let \(F=(F_n)_{n\in I}\) be a directed system of geometric inputs (e.g.\ refining Hamiltonians/perturbations or nested stop systems) with colimit \(F_\infty\).
Apply \(\mathcal{F}\) and \(\mathbf{P}_i\) to obtain a tower in \(\Pers^{\mathrm{ft}}_\Lambda\).
For \(\tau\ge 0\) consider the comparison map (Appendix~J)
\[
\phi_{i,\tau}(F):\quad \varinjlim_n\ \mathbf{T}_\tau\!\big(\mathbf{P}_i(\mathcal{F}(F_n))\big)\ \longrightarrow\ \mathbf{T}_\tau\!\big(\mathbf{P}_i(\mathcal{F}(F_\infty))\big).
\]

\begin{definition}[Sufficient tower hypotheses]\label{O:def:tower-conds}
Assume: (S1) commutation up to vanishing error with truncation and reindexing; or (S2) no \(\tau\)-accumulation of break times; or (S3) Cauchy in \(d_{\mathrm{int}}\) with compatible structure maps. Each is made precise in Appendix~D, §D.3 and ensures stability of barcodes under the colimit and truncation.
\end{definition}

\begin{theorem}[When the comparison is an isomorphism]\label{O:thm:phi-iso}
If the tower admits continuation controls \(\varepsilon_n\to 0\) with
\(d_{\mathrm{int}}\!\big(\mathbf{P}_i(\mathcal{F}(F_n)),\mathbf{P}_i(\mathcal{F}(F_\infty))\big)\le \varepsilon_n\)
and satisfies any of (S1)/(S2)/(S3), then \(\phi_{i,\tau}(F)\) is an isomorphism for all \(\tau\ge 0\).
Consequently,
\[
\mu_{i,\tau}(F)=\nu_{i,\tau}(F)=0,
\]
where \((\mu,\nu)\) denote the generic fiber dimensions of the kernel/cokernel (Appendix~D, Remark~\ref{rem:D-generic-dim}).
\end{theorem}

\begin{proof}[Proof sketch]
Combine Theorem~\ref{O:thm:cont} with the tower criteria of Appendix~D, §D.3 to control the interleavings along the system and its colimit; then apply the comparison formalism of Appendix~J to deduce that \(\phi_{i,\tau}\) is an isomorphism. Vanishing of \((\mu,\nu)\) follows.
\end{proof}

\begin{corollary}[Grid \(\Rightarrow\) continuum survival]\label{O:cor:g2c}
In discretization towers (mesh \(h\to 0\)) with certified continuation bounds \(\varepsilon(h)\to 0\), any bar detected in a fixed window \([0,\tau_0]\) whose \(\mathbf{T}_{\tau_0}\)-clipped length exceeds \(2\varepsilon(h)\) persists in the limit (Appendix~I, Theorem~I:\ref{I:thm:g2c}).
\end{corollary}

% -------------------------
\subsection*{O.4. Permitted-operations table (windowed, post-collapse) and $\delta$-ledger}
% -------------------------
All comparisons follow the protocol “for each \(t\) \(\to\) persistence \(\to\) collapse \(\mathbf{T}_\tau\) \(\to\) compare,” on MECE, right–open windows and a fixed \(\tau\) (Appendix~G/N).
The following table summarizes permitted operations, their type, quantitative contracts (after collapse), and how to record them in the \(\delta\)–ledger.

\begin{center}
\footnotesize
\setlength{\tabcolsep}{4pt}
\renewcommand{\arraystretch}{1.1}
\begin{tabularx}{\linewidth}{@{} L{.26\linewidth} L{.14\linewidth} Y @{}}
\toprule
Operation & Type & Quantitative contract after $\mathbf{T}_\tau$ and ledger entry \\
\midrule
Add stop / shrink sector & Deletion & Nonincreasing deletion indicators; no new bars on the window; ledger: $\delta^{\mathrm{alg}}=0$, record $\delta^{\mathrm{disc}},\delta^{\mathrm{meas}}$ if any \\
Continuation (tame homotopy) & Shift & $d_{\mathrm{int}}\le \varepsilon$ (Theorem~\ref{O:thm:cont}); ledger: $\delta^{\mathrm{alg}}=\varepsilon$ \\
Hamiltonian change (bounded drift) & Shift & $d_{\mathrm{int}}\le \varepsilon$; ledger: $\delta^{\mathrm{alg}}=\varepsilon$ \\
Almost complex structure change (tame) & Shift & $d_{\mathrm{int}}\le \varepsilon$; ledger: $\delta^{\mathrm{alg}}=\varepsilon$ \\
Regrading / Maslov shift & Bookkeeping & Isometry (degree reindexing); ledger: $\delta^{\mathrm{alg}}=0$ \\
Monotone time reparametrization & Reindex & Isometry after reindex normalization; ledger: $\delta^{\mathrm{alg}}=0$ \\
Mirror/Transfer post-processing & External & As audited (Appendix~L); if commutation defect $\delta(i,\tau)$, ledger: $\delta^{\mathrm{alg}}=\delta(i,\tau)$ \\
Non-nested reflectors (if used) & External & A/B test or soft-commuting fallback (Appendix~K/L); ledger: $\delta^{\mathrm{alg}}=\Delta_{\mathrm{comm}}$ if fallback \\
\bottomrule
\end{tabularx}
\end{center}


\begin{definition}[$\delta$–ledger and budgets]\label{O:def:ledger}
Per window \(W=[u,u')\) and degree \(i\), define the aggregate budget
\[
\Sigma\delta_W(i)\ :=\ \sum_{\text{continuations}}\varepsilon\ +\ \sum_{\text{Mirror/Transfer}}\delta(i,\tau)\ +\ \sum_{\text{A/B fails}}\Delta_{\mathrm{comm}}\ +\ \sum_{\text{audits}}(\delta^{\mathrm{disc}}+\delta^{\mathrm{meas}}),
\]
where “audits” include PF/BC checks (Appendix~N) and numerical tolerances (Appendix~G).
\end{definition}

% -------------------------
\subsection*{O.5. B-Gate$^{+}$, restart, and summability (window pasting)}
% -------------------------
We adopt B-Gate\(^{+}\) with a per-window safety margin \(\mathrm{gap}_\tau(i)>0\) computed after \(\mathbf{T}_\tau\).
On window \(W\) and degree \(i\), the gate \emph{passes} if
\[
\mathrm{gap}_\tau(i)\ >\ \Sigma\delta_W(i).
\]
Across consecutive windows \((W_k)_k\), assume:
(i) transitions are finite compositions of deletion-type steps and \(\varepsilon\)-continuations (measured post-collapse), and
(ii) Summability holds \(\sum_k \Sigma\delta_{W_k}(i)<\infty\) (Appendix~J).
Then the Restart inequality of Appendix~J yields for some \(\kappa\in(0,1]\),
\[
\mathrm{gap}_{\tau_{k+1}}(i)\ \ge\ \kappa\Big(\mathrm{gap}_{\tau_k}(i)-\Sigma\delta_{W_k}(i)\Big),
\]
so positivity of the margin persists along the pipeline.
Per-window certificates therefore paste to a global certificate on \(\bigcup_k W_k\) (Appendix~J, Theorem~J:\ref{J:thm:pasting}).

\begin{remark}[Choice of \(\kappa\)]
The constant \(\kappa\) accounts for uniform losses at window transitions (e.g.\ finite alignment overheads or reindexing coercions). Under exact commutation, one can take \(\kappa=1\).
\end{remark}

% -------------------------
\subsection*{O.6. Windowed usage and run.yaml alignment}
% -------------------------
Record in \texttt{run.yaml} per window and degree:
\begin{itemize}\itemsep0.25em
  \item the operation sequence (stops/sector changes, continuations) with quantitative parameters (\(\varepsilon\), thresholds \(\tau\), sweep settings);
  \item the \(\delta\)–ledger entries and their sum \(\Sigma\delta_W(i)\);
  \item the B-Gate\(^{+}\) safety margin \(\mathrm{gap}_\tau(i)\) and pass/fail verdict;
  \item any external steps (Mirror/Transfer, reflectors) with A/B policy data (\(\eta\), \(\Delta_{\mathrm{comm}}\)) (Appendix~K/L);
  \item constructibility checks: upper bounds on generators and event counts (Appendix~H).
\end{itemize}
All diagnostics (\(\mu,\nu\), comparison maps \(\phi_{i,\tau}\)) are computed \emph{after} truncation \(\mathbf{T}_\tau\) and logged with the same window and \(\tau\).

% -------------------------
\subsection*{O.7. Failure modes and audit checklist}
% -------------------------
\noindent\emph{Failure modes (outside our scope).}
\begin{itemize}\itemsep0.25em
  \item \textbf{Loss of filtration control.} Non-exact data or bubbling may invalidate (O2); interleaving bounds then fail.
  \item \textbf{Near-threshold accumulation.} Type~IV behavior (Appendix~D) may produce bar-length accumulation at \(\tau\); comparison maps need not stabilize without (S2)/(S3).
  \item \textbf{Inclusion-type operations.} Removing stops or enlarging sectors can increase features; only stability (not monotonicity) is claimed, and only when continuation control is present.
\end{itemize}

\noindent\emph{Audit checklist (runtime verifications).}
\begin{enumerate}\itemsep0.25em
  \item Record continuation shift bounds \(\varepsilon\) and certify \(d_{\mathrm{int}}\)-nonexpansivity (Appendix~A).
  \item Verify constructibility on each window \([0,\sigma]\) (finite generators/events; Appendix~H) and log per-window counts (Appendix~G).
  \item For stop additions/sector shrinkage, mark operation as deletion-type and apply Appendix~E indicators.
  \item For towers, log \(\varepsilon_n\), check (S1)/(S2)/(S3) where applicable, and compute \((\mu,\nu)\); \(\phi_{i,\tau}\) iso \(\Rightarrow (\mu,\nu)=(0,0)\) (Appendix~J).
  \item If external functors are used, run Mirror/Transfer commutation audits and A/B tests (Appendix~L/K) and bookkeep residuals in \(\Sigma\delta_W(i)\).
\end{enumerate}

% -------------------------
\subsection*{O.8. Formalization stubs (Lean/Coq) [Spec]}
% -------------------------
A minimal API (cf.\ Appendix~F) includes:
\begin{itemize}\itemsep0.25em
  \item \texttt{fukaya\_realize}: returns an action-filtered chain model \(\mathcal{F}\) up to f.q.i., with constructibility on bounded windows (O1).
  \item \texttt{cont\_eps}: encodes continuation maps with filtration increase \(\le\varepsilon\) and yields \(d_{\mathrm{int}}\le\varepsilon\) (Theorem~\ref{O:thm:cont}).
  \item \texttt{stop\_delete}: produces deletion-type morphisms for stops/sector-shrink (Proposition~\ref{O:prop:stops}).
  \item \texttt{tower\_phi\_iso}: checks sufficient criteria so that \(\phi_{i,\tau}\) is an isomorphism and \((\mu,\nu)=(0,0)\) (Appendix~D/J).
  \item Hooks for \(\delta\)–ledger entries and B-Gate\(^{+}\) checks; restart/summability contracts (Appendix~J).
\end{itemize}

% -------------------------
\subsection*{O.9. Examples and regression tests}
% -------------------------
\begin{example}[Exact wrapped setting with a new stop]
Let \(X=T^*Q\) with its standard exact form and consider a wrapped setting with a stop at infinity. Adding an additional stop supported on a Legendrian subset removes Reeb chords crossing the stop. On any fixed action window, no new generators appear and differentials can only increase; hence the induced map is deletion-type (Proposition~\ref{O:prop:stops}). Deletion indicators (Appendix~E) weakly decrease; regression test: the windowed barcode total variation does not increase after \(\mathbf{T}_\tau\).
\end{example}

\begin{example}[Continuation bound from Hamiltonian drift]
Suppose \(H_0,H_1\) are cofinal Hamiltonians with \(\sup_{x,t}(H_1-H_0)\le \varepsilon\) on the relevant support and homotoped by a tame path. The action change along continuation solutions is bounded by \(\varepsilon\); thus the induced filtered map has filtration increase \(\le \varepsilon\), and \(d_{\mathrm{int}}\le \varepsilon\) (Theorem~\ref{O:thm:cont}). Regression test: pairwise barcode bottleneck distance never exceeds the certified \(\varepsilon\).
\end{example}

\begin{example}[Grid-to-continuum]
Discretize a time-dependent Floer datum with mesh \(h\) and certified continuation bound \(\varepsilon(h)=O(h^\alpha)\). For any fixed \(\tau_0\), bars of length \(>2\varepsilon(h)\) in \(\mathbf{T}_{\tau_0}\)-collapse survive in the limit (Corollary~\ref{O:cor:g2c}). Regression test: survival rates converge monotonically after accounting for \(\Sigma\delta_W(i)\).
\end{example}

% -------------------------
\subsection*{O.10. Summary}
% -------------------------
Floer-theoretic realizations with action filtration yield constructible persistence (O1).
Continuation with shift \(\varepsilon\) is \(1\)\hyp Lipschitz at the persistence level (O2 \(\Rightarrow\) Theorem~\ref{O:thm:cont}).
Adding stops or shrinking sectors is deletion-type and hence nonincreasing for all deletion indicators (Proposition~\ref{O:prop:stops}).
A windowed, post-collapse permitted-ops table prescribes how to assign \(\delta\)–ledger entries; B-Gate\(^{+}\) requires \(\mathrm{gap}_\tau>\Sigma\delta\) per window and pastes via Restart/Summability (Appendix~J).
Under standard tower hypotheses (Appendix~D), the comparison map \(\phi_{i,\tau}\) is an isomorphism and \((\mu,\nu)=(0,0)\) (Theorem~\ref{O:thm:phi-iso}); grid-to-continuum survival follows (Appendix~I).
All items respect the MECE/right–open window policy, are evaluated “\(t\)\(\to\)\(\mathbf{P}_i\)\(\to\)\(\mathbf{T}_\tau\)\(\to\)compare,” and integrate with the pipeline budget and A/B policy (Appendix~L/K) in a reproducible \texttt{run.yaml} workflow (Appendix~G).



% -------------------------
\section*{Concluding Remarks and Acknowledgments}
% -------------------------
\addcontentsline{toc}{section}{Concluding Remarks and Acknowledgments}

\paragraph{Coefficients and standing scope.}
Throughout, unless explicitly stated otherwise, coefficients are taken in a \emph{field}. In the provable Core (including the bridge in Appendix~C) the target of realizations is \(D^{\mathrm{b}}(k\text{-mod})\). In \textbf{[Spec]} appendices that use sheaf-theoretic or Fukaya-categorical realizations we write the coefficient field as \(\Lambda\). This is a notational convenience only: all persistence-layer statements remain within the constructible \(1\)D regime over a field. We do not leave this regime. In particular, all filtered (co)limits are computed \emph{objectwise} in \([\mathbb{R},\mathsf{Vect}_\Lambda]\) under the scope policy (Appendix~A) and returned to the constructible subcategory by verification or by applying \(\mathbf{T}_\tau\).

\paragraph{What is proved and formalized (Core).}
Within constructible \(1\)D persistence over a field, the following are established and arranged for machine-checkability and formalization.
\begin{itemize}
  \item \emph{Exact truncation and filtered lift.} The Serre reflector \(\mathbf{T}_\tau\) deletes precisely the finite bars of length \(\le\tau\), is exact, idempotent, and \(1\)-Lipschitz for the interleaving metric; it admits a filtered lift \(C_\tau\) (unique up to filtered quasi-isomorphism) with \(\mathbf{P}_i(C_\tau F)\cong \mathbf{T}_\tau(\mathbf{P}_i F)\) (Appendix~A/B).
  \item \emph{One-way bridge PH\(_1\)\(\Rightarrow\)Ext\(^1\).} Under a \(t\)-exact realization of amplitude \(\le 1\), \(\mathrm{PH}_1(F)=0\) implies \(\operatorname{Ext}^1(\mathcal{R}(F),k)=0\). No converse or global equivalence is asserted (Appendix~C).
  \item \emph{Tower diagnostics.} For the windowed comparison map \(\phi_{i,\tau}\colon \varinjlim \mathbf{T}_\tau\mathbf{P}_i(F_n)\to \mathbf{T}_\tau\mathbf{P}_i(F_\infty)\), the generic fiber dimensions \(\mu_{i,\tau}:=\gdim\ker\phi\) and \(\nu_{i,\tau}:=\gdim\coker\phi\) detect invisible limit failures (Type~IV). These enjoy subadditivity under composition, additivity under finite sums, and cofinal invariance; they vanish when \(\phi_{i,\tau}\) is an isomorphism (Appendix~D/J).
  \item \emph{Windowed indicators and calculus.} The Betti integral equals the clipped barcode mass on \([0,\tau]\); \(\varepsilon\)-survival provides grid-to-continuum robustness without false negatives; deletion-type spectral indicators are non-increasing after collapse, while general operations are non-expansive (Appendix~H/I/E).
  \item \emph{Windowed pasting.} B--Gate\(^{+}\) enforces a per-window safety margin \(\mathrm{gap}_\tau>\sum\delta\); Overlap Gate glues local certificates after collapse; Restart/Summability pastes windowed certificates into global ones (Appendix~J). All comparisons are performed after truncation by the protocol \(t\to \mathbf{P}_i\to \mathbf{T}_\tau\to\) compare on mutually exclusive, collectively exhaustive right-open windows.
\end{itemize}

\paragraph{What is specified and how it is audited (\textup{[Spec]}).}
All \textbf{[Spec]} items are \emph{non-expansive after truncation}, shipped with explicit preconditions, and audited by windowed diagnostics \((\mu,\nu)\) together with \(\delta\)-budgeting.
\begin{itemize}
  \item \emph{Mirror/Transfer commutation with collapse.} A natural \(2\)-cell \(\Mirror\circ C_\tau \Rightarrow C_\tau\circ\Mirror\) with a uniform bound \(\delta(i,\tau)\) (in the interleaving metric at the persistence layer) gives \(\delta\)-controlled commutation; bounds add along pipelines and do not increase under \(1\)-Lipschitz post-processing (Appendix~L).
  \item \emph{Multi-axis reflectors and A/B soft-commuting.} For exact reflectors arising from hereditary Serre subcategories, order independence holds in the nested case; otherwise an A/B commutativity test with tolerance \(\eta\) and deterministic fallback is used. Residuals \(\Delta_{\mathrm{comm}}\) are logged as \(\delta^{\mathrm{alg}}\) (Appendix~K).
  \item \emph{Lax monoidal tensor/collapse.} Windowed energy admits overlap-integral bounds; in a pointwise-mono regime (M3\(^{+}\)) one obtains post-collapse dominance \(\PE(\,C_\tau(F\otimes G)\,)\le \PE(\,C_\tau F\otimes C_\tau G\,)\). Optional metric gaps may be integrated into the \(\delta\)-budget (Appendix~M).
  \item \emph{Projection formula and base change.} PF/BC are transported to the persistence layer by the windowed protocol \(t\to \mathbf{P}_i\to \mathbf{T}_\tau\to\) compare. Any residual numerical slack (post-truncation) is budgeted as \(\delta^{\mathrm{disc}},\delta^{\mathrm{meas}}\) (Appendix~N).
  \item \emph{Fukaya realizations.} Action filtration yields constructible persistence on bounded windows; continuation is \(1\)-Lipschitz; stop addition is deletion-type; tower stability and grid-to-continuum survival follow under standard hypotheses (Appendix~O).
\end{itemize}
All \textbf{[Spec]} statements are used only under their stated hypotheses, are auditable per window via \((\mu,\nu)\) and the \(\delta\)-ledger, and never override Core guarantees.

\paragraph{Operational pipeline (end-to-end).}
We summarize the practical path from input to audit-ready outputs.
\begin{itemize}
  \item \emph{Windows and after-collapse policy.} Partition time/parameter into MECE right-open windows; fix \(\tau\) per window (optionally via \(\tau\)-sweep and stability bands). All measurements and comparisons are made on the B-side single layer after applying \(\mathbf{T}_\tau\).
  \item \emph{Overlap Gate.} On overlaps, require post-collapse equivalence up to budget; verify Čech--\(\Ext^1\) acyclicity in degree \(1\) and stability-band conditions; then glue.
  \item \emph{B--Gate\(^{+}\).} Per window and degree \(i\), require \(\mathrm{PH}_1=0\), (eligible) \(\Ext^1=0\), \((\mu,\nu)=(0,0)\), and a safety margin \(\mathrm{gap}_\tau>\sum\delta\).
  \item \emph{Restart/Summability.} Across windows, enforce a restart inequality \(\mathrm{gap}_{\tau_{k+1}}\ge \kappa(\mathrm{gap}_{\tau_k}-\sum\delta_k)\) for some \(\kappa\in(0,1]\) and summability \(\sum_k\sum\delta_k<\infty\); paste local certificates into global ones.
  \item \emph{\(\delta\)-ledger.} Aggregate algorithmic (\(\delta^{\mathrm{alg}}\): Mirror--Collapse defects \(\delta(i,\tau)\), A/B residuals \(\Delta_{\mathrm{comm}}\)), discretization (\(\delta^{\mathrm{disc}}\)),および measurement (\(\delta^{\mathrm{meas}}\)) per window; compare \(\sum \delta\) to \(\mathrm{gap}_\tau\).
  \item \emph{Reproducibility.} Emit versioned artifacts for \texttt{bars}/\texttt{spec}/\texttt{ext}/\texttt{phi}/\texttt{Lambda\_len} with cross-linked hashes. Fix spectral ordering (ascending) and matrix norm (operator/Frobenius). Use canonical JSON/HDF5 serialization (fixed endianness, fixed-length UTF-8 strings, creation-order stability) for bitwise re-runs (Appendix~G).
\end{itemize}

\paragraph{Reproducibility, formalization, and tests.}
Appendix~G specifies the manifest (\texttt{run.yaml}) and artifact schemas with canonical serialization and cross-linked hashes; the \(\delta\)-ledger records per-window budgets, including Mirror--Collapse \(\delta(i,\tau)\) and A/B \(\Delta_{\mathrm{comm}}\). Appendix~F sketches a Lean/Coq formal development for the Core: Serre localization and \(\mathbf{T}_\tau\); \(1\)-Lipschitz continuity; comparison maps and \((\mu,\nu)\); the bridge’s edge identification; and API stubs for budget accounting, A/B policy, PF/BC transport, and lax monoidal contracts. A minimal test suite (Appendix~G/12) validates stability, deletion-type monotonicity, filtered-colimit behavior, Mirror/tropical pipelines, three-layer correspondences, saturation gates, A/B soft-commuting, and Restart/Summability pasting.

\paragraph{Scope and guard-rails.}
All statements obey the following guard-rails.
\begin{itemize}
  \item \emph{Regime.} Constructible \(1\)D persistence over a field; filtered (co)limits computed objectwise in \([\mathbb{R},\mathsf{Vect}]\) and returned to \(\Pers^{\mathrm{ft}}\) by verification or \(\mathbf{T}_\tau\).
  \item \emph{Collapse layer.} \(\mathbf{T}_\tau\) is exact/idempotent/\(1\)-Lipschitz; shift-commutation holds; deletion-type vs.\ inclusion-type policy applies after truncation (monotonicity vs.\ non-expansiveness).
  \item \emph{Single-layer judgement.} All diagnostics and budgets are evaluated after collapse on \(\mathbf{T}_\tau \mathbf{P}_i\); PF/BC and Mirror/Transfer comparisons follow the windowed protocol \(t\to \mathbf{P}_i\to \mathbf{T}_\tau\to\) compare.
  \item \emph{Bridge and non-claims.} The bridge remains strictly one-way (\(\mathrm{PH}_1\Rightarrow\Ext^1\) under amplitude \(\le 1\)); no claim of \(\mathrm{PH}_1\Leftrightarrow \Ext^1\). The collapse diagnostic \(\mu_{\mathrm{Collapse}}\) is unrelated to classical Iwasawa \(\mu\).
\end{itemize}

\paragraph{Limitations.}
We do not assert metric Lipschitz control beyond \(\mathbf{T}_\tau\) (and any explicitly stated \(\delta\)-controls). Spectral indicators are not f.q.i.-invariants; they are evaluated under fixed policies, with deletion-type monotonicity and general non-expansiveness bounds after collapse. We do not extend the bridge, strong adjunctions, or strict monoidality beyond the proven/declared domains. PF/BC and lax monoidal statements remain windowed \textbf{[Spec]} contracts; all budgets are computed after collapse on the B-side single layer.

\paragraph{Future directions.}
\begin{itemize}
  \item \emph{Quantitative links between persistence energies and spectral tails.} Under robust hypotheses and verified window policies, strengthen two-sided bounds, sensitivity analysis, and noise models.
  \item \emph{Tower stability and diagnostics.} Broaden verifiable criteria for \((\mu,\nu)=(0,0)\) beyond (S1)–(S3), refine Type~IV precursors across scales, and automate stability-band detection on \(\tau\)-sweeps.
  \item \emph{Domain templates.} Provide standardized templates (arithmetic/Langlands/PDE/Fukaya) with \(\delta\)-controls, PF/BC audits, A/B policies, and richer manifests for end-to-end reviews.
  \item \emph{Formalization.} Expand shift/interleaving libraries, PF/BC transports, survival lemmas, pipeline budget calculus, and soft-commuting policies in proof assistants; extend test harnesses for regression and coverage metrics.
\end{itemize}

\paragraph{Acknowledgments.}
This manuscript was prepared solely by the author together with an AI assistant (ChatGPT) via an iterative chat-based workflow; no other contributors were involved. Any remaining errors are the authors' responsibility.

\medskip
\noindent\textbf{Final note.}
The consistent separation between the provable Core and the \textbf{[Spec]} extensions is the organizing principle of this work: it enables safe reuse, cross-domain exploration, and reproducible evaluation while preserving mathematically verified guarantees. Windowed B-side judgement (collapse first), explicit \(\delta\)-budgets, A/B soft-commuting, and Restart/Summability pasting provide an end-to-end methodology that is conservative in scope and extensible within the v16.0 guard-rails.




\end{document}