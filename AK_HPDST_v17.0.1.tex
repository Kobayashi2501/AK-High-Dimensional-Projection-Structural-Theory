% ===========================
% AK High-Dimensional Projection Structural Theory v17.0 (XeLaTeX)
% Clean, de-duplicated, compile-stable preamble (DROP-IN REPLACEMENT)
% ===========================

\RequirePackage[2023-11-01]{latexrelease}
\documentclass[11pt]{article}

% --- Math alphabet reservation cut (must be BEFORE ams/bm) ---
\newcommand\hmmax{0}
\newcommand\bmmax{0}

% --- Encoding & Language ---
% XeLaTeX: inputenc は不要
\usepackage{fontspec}
\usepackage[english]{babel}

% --- Page layout ---
\usepackage{geometry}
\geometry{margin=1in}

% --- Color (needed by listings + hyperref) ---
\usepackage{xcolor}

% --- Core Math & Utilities ---
\usepackage{amsmath,amssymb,amsfonts,amsthm,mathtools}
\usepackage{mathrsfs}
\usepackage{bm}
\usepackage{stmaryrd}
\usepackage{amscd}

\usepackage{textcomp}
\usepackage{etoolbox} % for \forcsvlist
\usepackage{enumitem}
\usepackage{array}
\usepackage{tabularx}
\usepackage{booktabs}
\usepackage{changepage}

% --- Fonts (Times-like text & math; OK under XeLaTeX too) ---
\usepackage{newtxtext,newtxmath}
\usepackage{inconsolata} % Monospace
\DeclareTextFontCommand{\textttit}{\ttfamily\slshape} % monospace "italic-like"

% --- Diagrams ---
\usepackage{tikz}
\usepackage{tikz-cd}
\usetikzlibrary{
  matrix, arrows.meta, cd, calc, positioning, quotes,
  decorations.pathmorphing, decorations.markings,
  shapes.misc, shapes.geometric, arrows, fit, backgrounds, fadings
}
\usepackage{pgfplots}
\pgfplotsset{compat=1.18}
\usepackage[all,cmtip]{xy}

% --- Listings / Code ---
\usepackage{listings}
\usepackage{float}
\lstset{
  basicstyle=\ttfamily\small,
  keywordstyle=\color{blue},
  commentstyle=\color{gray},
  stringstyle=\color{orange},
  frame=single,
  breaklines=true,
  showstringspaces=false,
  captionpos=b,
  xleftmargin=1em,
  columns=fullflexible,
  upquote=true,
  mathescape=false,
  literate=*%
    {∀}{{$\forall$}}1 {∃}{{$\exists$}}1
    {→}{{$\to$}}1 {←}{{$\leftarrow$}}1 {⟶}{{$\longrightarrow$}}1 {⇒}{{$\Rightarrow$}}1 {⇔}{{$\Leftrightarrow$}}1 {↦}{{$\mapsto$}}1 {⥤}{{$\to$}}1
    {α}{{$\alpha$}}1 {β}{{$\beta$}}1 {γ}{{$\gamma$}}1 {δ}{{$\delta$}}1 {ε}{{$\varepsilon$}}1
    {λ}{{$\lambda$}}1 {μ}{{$\mu$}}1 {ν}{{$\nu$}}1 {π}{{$\pi$}}1 {τ}{{$\tau$}}1
    {κ}{{$\kappa$}}1 {θ}{{$\theta$}}1 {η}{{$\eta$}}1 {φ}{{$\varphi$}}1
    {ι}{{$\iota$}}1 {ρ}{{$\rho$}}1 {χ}{{$\chi$}}1 {ω}{{$\omega$}}1
    {Γ}{{$\Gamma$}}1 {Δ}{{$\Delta$}}1 {Π}{{$\Pi$}}1 {Σ}{{$\Sigma$}}1 {Ω}{{$\Omega$}}1 {Λ}{{$\Lambda$}}1 {Φ}{{$\Phi$}}1 {Ψ}{{$\Psi$}}1
    {ℕ}{{$\mathbb{N}$}}1 {ℤ}{{$\mathbb{Z}$}}1 {ℚ}{{$\mathbb{Q}$}}1 {ℝ}{{$\mathbb{R}$}}1
    {≤}{{$\le$}}1 {≥}{{$\ge$}}1 {≠}{{$\ne$}}1 {∈}{{$\in$}}1 {⊆}{{$\subseteq$}}1
    {∧}{{$\land$}}1 {∨}{{$\lor$}}1
    {⪯}{{$\preceq$}}1 {⪰}{{$\succeq$}}1 {≡}{{$\equiv$}}1 {≅}{{$\cong$}}1 {⋙}{{$\ggg$}}1
    {⊕}{{$\oplus$}}1 {⊗}{{$\otimes$}}1 {⊙}{{$\odot$}}1 {⊣}{{$\dashv$}}1 {⊢}{{$\vdash$}}1 {⊥}{{$\perp$}}1
    {·}{{$\cdot$}}1 {∙}{{$\cdot$}}1 {⋅}{{$\cdot$}}1 {•}{{$\bullet$}}1
    {・}{{$\cdot$}}1 {‧}{{$\cdot$}}1
    {×}{{$\times$}}1 {⨯}{{$\times$}}1
    {…}{{$\ldots$}}1 {–}{{--}}1 {—}{{---}}1 {∞}{{$\infty$}}1
    {₁}{{$_1$}}1 {⁺}{{${}^{+}$}}1
}

% --- Over/underfull mitigation ---
\usepackage{microtype}
\emergencystretch=2em

% --- Hyperlinks (load ONCE, near the end) ---
\usepackage[colorlinks=true, linkcolor=blue, citecolor=blue, urlcolor=blue]{hyperref}
\usepackage{xurl}
\urlstyle{tt}
\def\UrlBreaks{\do\/\do\_\do\.\do\-}

% --- Clever references (after hyperref) ---
\usepackage[capitalise,nameinlink,noabbrev]{cleveref}

% --- Float / page control ---
\usepackage{placeins}

% --- Equation numbering ---
\numberwithin{equation}{section}

% --- Tables: column helpers ---
\newcolumntype{L}[1]{>{\raggedright\arraybackslash}p{#1}}
\newcolumntype{C}[1]{>{\centering\arraybackslash}p{#1}}
\newcolumntype{R}[1]{>{\raggedleft\arraybackslash}p{#1}}
\newcolumntype{Y}{>{\raggedright\arraybackslash}X}

% --- Monospace token helper (safe in text+math) ---
\newcommand{\fname}[1]{\ifmmode\text{\texttt{#1}}\else\texttt{#1}\fi}
% === Paste-artifact guard: robust \n handling (HEAVY) ===
\providecommand{\n}{\unskip\space}

% --- Minimal hyphenation helpers (no package) ---
\DeclareRobustCommand{\hyp}{\nobreakdash-} % "right\hyp open"
\providecommand{\NoHyphens}[1]{\mbox{#1}}
\providecommand{\nohyphens}[1]{\mbox{#1}}
% --- Denef--Pas / valued-field macros (Spec appendix) ---
\makeatletter
\@ifundefined{val}{\DeclareMathOperator{\val}{val}}{}
\@ifundefined{ac}{\DeclareMathOperator{\ac}{ac}}{}
\makeatother

\providecommand{\VF}{\mathsf{VF}} % valued field sort
\providecommand{\RF}{\mathsf{RF}} % residue field sort
\providecommand{\VG}{\mathsf{VG}} % value group sort

% ======================================================
% === Math Operators (safe: define only if undefined) ===
% ======================================================
\makeatletter
\@ifundefined{Ext}{\DeclareMathOperator{\Ext}{Ext}}{}
\@ifundefined{Hom}{\DeclareMathOperator{\Hom}{Hom}}{}
\@ifundefined{Spec}{\DeclareMathOperator{\Spec}{Spec}}{}
\@ifundefined{colim}{\DeclareMathOperator*{\colim}{colim}}{}
\@ifundefined{PH}{\DeclareMathOperator{\PH}{PH}}{}
\@ifundefined{Tor}{\DeclareMathOperator{\Tor}{Tor}}{}
\@ifundefined{rank}{\DeclareMathOperator{\rank}{rank}}{}
\@ifundefined{im}{\DeclareMathOperator{\im}{im}}{}
\@ifundefined{id}{\DeclareMathOperator{\id}{id}}{}
\@ifundefined{Ker}{\DeclareMathOperator{\Ker}{Ker}}{}
\@ifundefined{Coker}{\DeclareMathOperator{\Coker}{Coker}}{}
\@ifundefined{Collapse}{\DeclareMathOperator{\Collapse}{Collapse}}{}
\@ifundefined{Mot}{\DeclareMathOperator{\Mot}{Mot}}{}
\@ifundefined{GL}{\DeclareMathOperator{\GL}{GL}}{}
\@ifundefined{RHom}{\DeclareMathOperator{\RHom}{RHom}}{}
\@ifundefined{HT}{\DeclareMathOperator{\HT}{HT}}{}
\@ifundefined{gdim}{\DeclareMathOperator{\gdim}{gdim}}{}
\@ifundefined{coker}{\DeclareMathOperator{\coker}{coker}}{}
\makeatother

% ==================
% === Main Macros ===
% ==================
\newcommand{\kk}{k}

% --- make \renewcommand safe even if \T,\C were not defined ---
\providecommand{\T}{} % 追加
\providecommand{\C}{} % 追加
\renewcommand{\T}{\mathsf{T}}
\renewcommand{\C}{\mathsf{C}}

\newcommand{\QQ}{\mathbb{Q}}
\newcommand{\RR}{\mathbb{R}}
\newcommand{\CC}{\mathbb{C}}
\newcommand{\ZZ}{\mathbb{Z}}
\newcommand{\TT}{\mathbb{T}}
\newcommand{\NN}{\mathbb{N}}
\newcommand{\EE}{\mathbb{E}}

\newcommand{\bbR}{\RR}
\newcommand{\bbN}{\NN}

\newcommand{\cF}{\mathcal{F}}
\newcommand{\cG}{\mathcal{G}}
\newcommand{\cE}{\mathcal{E}}
\newcommand{\cO}{\mathcal{O}}
\newcommand{\cD}{\mathcal{D}}
\newcommand{\cH}{\mathcal{H}}
\newcommand{\cC}{\mathcal{C}}
\newcommand{\DD}{\mathcal{D}}

\newcommand{\into}{\hookrightarrow}
\newcommand{\onto}{\twoheadrightarrow}
\newcommand{\eps}{\varepsilon}
\newcommand{\Sha}{\mathcal{X}}

\newcommand{\CollapseZone}{\mathfrak{C}}
\newcommand{\CollapseCompatible}{\mathsf{CollapseCompatible}}
\newcommand{\CollapseTypeOf}[1]{\operatorname{CollapseType}(#1)}
\newcommand{\CollapseImage}{\operatorname{CollapseImage}}
\newcommand{\CollapseEnergy}{\mathcal{E}_{\mathrm{Coll}}}

\newcommand{\ptorsionfree}[1]{\ensuremath{#1}\nobreakdash-torsion-free}

\newcommand{\Pers}{\mathsf{Pers}}
\newcommand{\Vect}{\mathsf{Vect}}
\newcommand{\Vectk}{\mathsf{Vect}_{\kk}}

\newcommand{\Ho}{\ensuremath{\mathrm{Ho}}}
\newcommand{\rH}{\ensuremath{\mathrm{H}}}

\newcommand{\timevar}{t}

% Preserve original accent commands before overwriting \T, \C (escape hatches)
\makeatletter
\@ifundefined{T}{}{\let\Taccent\T}
\@ifundefined{C}{}{\let\Caccent\C}
\makeatother
\renewcommand{\T}{\mathsf{T}}
\renewcommand{\C}{\mathsf{C}}
\newcommand{\Coll}{\mathsf{C}} % alias

\newcommand{\Rfun}{\mathcal{R}}

% --- common arrows ---
\newcommand{\xto}{\xrightarrow}
\newcommand{\mono}{\ensuremath{\mathrm{mono}}}
\newcommand{\epi}{\ensuremath{\mathrm{epi}}}
\newcommand{\xtoH}{\xRightarrow{\ \simeq\ }}
\newcommand{\xtoM}{\xRightarrow{\ \mono\ }}

% --- text helpers ---
\newcommand{\dq}{\textquotedbl}

% --- persistent / integral notation ---
\newcommand{\PE}{\ensuremath{\mathrm{PE}}}
\newcommand{\betti}{\beta}
\newcommand{\CT}{C_{\tau}}
\newcommand{\dint}{d_{\mathrm{int}}}
\newcommand{\intdist}{d_{\mathrm{int}}}

% --- short operators used in text sometimes: text+math safe ---
\newcommand{\lift}{\ensuremath{\mathrm{lift}}}
\newcommand{\MECE}{\texorpdfstring{\ensuremath{\mathrm{MECE}}}{MECE}}
\newcommand{\clip}{\ensuremath{\mathrm{clip}}}
\newcommand{\len}{\ensuremath{\mathrm{len}}}
\newcommand{\supp}{\ensuremath{\mathrm{supp}}}
\newcommand{\op}{\ensuremath{\mathrm{op}}}
\newcommand{\win}{\ensuremath{\mathrm{win}}}
\newcommand{\spec}{\ensuremath{\mathrm{spec}}}
\newcommand{\disc}{\ensuremath{\mathrm{disc}}}
\newcommand{\meas}{\ensuremath{\mathrm{meas}}}
\newcommand{\tot}{\ensuremath{\mathrm{tot}}}
\newcommand{\inter}{\ensuremath{\mathrm{int}}}

% --- “ledger/defect” family (you used these in body) ---
\newcommand{\Pfun}{\ensuremath{\mathbf{P}}}
\newcommand{\Ufun}{\ensuremath{\mathcal{U}}}
\newcommand{\Defect}{\ensuremath{\mathsf{Defect}}}
\newcommand{\Gal}{\texorpdfstring{\ensuremath{\mathrm{Gal}}}{Gal}}
\newcommand{\Tr}{\texorpdfstring{\ensuremath{\mathrm{Tr}}}{Tr}}
\newcommand{\Fun}{\texorpdfstring{\ensuremath{\mathrm{Fun}}}{Fun}}
\newcommand{\alg}{\texorpdfstring{\ensuremath{\mathrm{alg}}}{alg}}
\DeclarePairedDelimiter{\norm}{\lVert}{\rVert}
% ---- v17 canon (Collapse measures) ----
\DeclareRobustCommand{\muc}{\mu_{\mathrm{Collapse}}}
\DeclareRobustCommand{\nuc}{\nu_{\mathrm{Collapse}}}
\DeclareRobustCommand{\muit}{\mu_{i,\tau}}
\DeclareRobustCommand{\nuit}{\nu_{i,\tau}}

\newcommand{\DiagZero}{(\muc,\nuc)=(0,0)}
\newcommand{\DiagNonzero}{(\muc,\nuc)\neq(0,0)}
\newcommand{\Zsing}{Z_{\mathrm{sing}}}

% --- derived categories / persistence ---
\DeclareRobustCommand{\Dbplain}{D^{\mathrm{b}}}
\DeclareRobustCommand{\Dbkmod}{D^{\mathrm{b}}(k\text{-mod})}
\DeclareRobustCommand{\Db}{D^{\mathrm{b}}(k\text{-mod})}
\DeclareRobustCommand{\Dbk}{D^{\mathrm{b}}(k\text{-mod})}
\DeclareRobustCommand{\Perskft}{\Pers^{\mathrm{cons}}_{\kk}}
\newcommand{\Perscons}{\Pers^{\mathrm{cons}}_{\kk}}
\newcommand{\Persft}{\Pers^{\mathrm{ft}}_{\kk}}

% --- unified notation / PDF-safe symbols ---
\DeclareRobustCommand{\FiltCh}[1]{\mathsf{FiltCh}(#1)}
\DeclareRobustCommand{\Pk}{\mathbf{P}}

\providecommand{\Tfun}[1]{\mathbf{T}_{#1}}
\providecommand{\Cfun}[1]{\mathsf{C}_{#1}}
\DeclareRobustCommand{\Ttau}{\texorpdfstring{\ensuremath{\Tfun{\tau}}}{T_\tau}}
\DeclareRobustCommand{\Tsigma}{\texorpdfstring{\ensuremath{\Tfun{\sigma}}}{T_\sigma}}
\DeclareRobustCommand{\Ctau}{\texorpdfstring{\ensuremath{\Cfun{\tau}}}{C_\tau}}
\DeclareRobustCommand{\Csigma}{\texorpdfstring{\ensuremath{\Cfun{\sigma}}}{C_\sigma}}

\DeclareRobustCommand{\fqi}{\text{f.q.i.}}
\DeclareRobustCommand{\LC}{\texorpdfstring{\ensuremath{\mathrm{(LC)}}}{(LC)}}
\DeclareRobustCommand{\Qtest}{\{\,\kk[0]\,\}}
\DeclareRobustCommand{\pos}[1]{\left(#1\right)_{+}}

\DeclareRobustCommand{\Trop}{\texorpdfstring{\ensuremath{\operatorname{Trop}}}{Trop}}
\DeclareRobustCommand{\Mirror}{\texorpdfstring{\ensuremath{\operatorname{Mirror}}}{Mirror}}

\newcommand{\ProofBadge}{\textsf{\footnotesize[Proof]}}
\newcommand{\SpecBadge}{\textsf{\footnotesize[Spec]}}

% --- extra convenience macros (kept) ---
\newcommand{\Crop}{\mathbf{W}}
\newcommand{\Res}{\ensuremath{\mathrm{Res}}}
\newcommand{\Ecat}{\mathsf{E}_\tau}
\newcommand{\Orth}{(\mathsf{E}_\tau)^{\perp}}
\newcommand{\Len}{\Lambda_{\mathrm{len}}}

\newcommand{\qand}{\quad\text{and}\quad}
\newcommand{\qtext}[1]{\quad\text{#1}\quad}
\newcommand{\ang}[1]{\langle #1\rangle}
\newcommand{\ol}[1]{\overline{#1}}

\newcommand{\Trans}{\mathsf{Trans}}
\newcommand{\Funct}{\mathsf{Funct}}
\newcommand{\loc}{\ensuremath{\mathrm{loc}}}

% ============================
% === Theorem environments ===
% ============================
\theoremstyle{plain}
\newtheorem{theorem}{Theorem}[section]
\newtheorem{proposition}[theorem]{Proposition}
\newtheorem{lemma}[theorem]{Lemma}
\newtheorem{corollary}[theorem]{Corollary}
\newtheorem{axiom}{Axiom}[section]
\newtheorem{conjecture}{Conjecture}[section]
\newtheorem{assumption}[theorem]{Assumption}
\newtheorem{auditobligation}[theorem]{Audit Obligation}
\theoremstyle{definition}
\newtheorem{definition}[theorem]{Definition}
\newtheorem{example}[theorem]{Example}
\newtheorem{remark}[theorem]{Remark}
\newtheorem{specification}[theorem]{Specification}
\newtheorem{declaration}[theorem]{Declaration}

% ======================================================
% === Paste-artifact guard: robust \n handling (HEAVY) ===
% ======================================================
\makeatletter

% Base \n: text-only unskip+space, math uses \; (or plain space if you prefer)
\providecommand{\n}{}
\renewcommand{\n}{\ifmmode\;\else\unskip\space\fi}

% Define \nWord macros safely: math mode => \; + Word, text mode => unskip+space+Word
\newcommand{\patchn}[1]{%
  \expandafter\def\csname n#1\endcsname{%
    \ifmmode\;\else\unskip\space\fi #1%
  }%
}

% Word lists (same as yours)
\forcsvlist{\patchn}{Hence,Therefore,Thus,Then,Proof,Remark,Theorem,Proposition,Lemma,Corollary,
Definition,Example,Declaration,Specification,Spec,Appendix,Section,Subsection,We,In,Let,Under,
Given,Assuming,Assumption,Fix,If,When,Where,While,For,From,By,On,Off,Also,Moreover,Conversely,
Equivalently,Finally,Next,First,Second,Third,Note,Notation,Convention,Claim,Case,Step,Algorithm,
Figure,Table,Thm,Prop,Lem,Cor,Def,Ex,Rem,Eq,Eqn,Ch,Chap,PF,BC,AK,NS,HT,PH}

\forcsvlist{\patchn}{hence,therefore,thus,then,proof,remark,theorem,proposition,lemma,corollary,
definition,example,declaration,specification,spec,appendix,section,subsection,we,in,let,under,
given,assuming,assumption,fix,if,when,where,while,for,from,by,on,off,also,moreover,conversely,
equivalently,finally,next,first,second,third,note,notation,convention,claim,case,step,algorithm,
figure,table,thm,prop,lem,cor,def,ex,rem,eq,eqn,ch,chap,pf,bc,ak,ns,ht,ph}

% With closing parens variants
\patchn{Proof)}\patchn{Remark)}\patchn{Theorem)}\patchn{Lemma)}\patchn{Corollary)}

% Single-letter / digit variants: \nA, \na, \n1, etc.
\newcommand{\patchnletter}[1]{%
  \expandafter\def\csname n#1\endcsname{%
    \ifmmode\;\else\unskip\space\fi #1%
  }%
}
\@tfor\@tempa:=ABCDEFGHIJKLMNOPQRSTUVWXYZabcdefghijklmnopqrstuvwxyz0123456789\do{%
  \expandafter\patchnletter\@tempa
}

\makeatother

% ---------- Font fallback ----------
\makeatletter
\AtBeginDocument{%
  \@ifpackageloaded{stmaryrd}{\SetSymbolFont{stmry}{bold}{U}{stmry}{m}{n}}{}%
}
\makeatother

% ---------- Unicode → TeX mappings (bookmark-safe) ----------
% hyperref が先に読み込まれている必要がある(texorpdfstring のため)
\usepackage{newunicodechar}
\newunicodechar{Σ}{\texorpdfstring{\ensuremath{\Sigma}}{Sigma}}
\newunicodechar{Γ}{\texorpdfstring{\ensuremath{\Gamma}}{Gamma}}
\newunicodechar{Δ}{\texorpdfstring{\ensuremath{\Delta}}{Delta}}
\newunicodechar{Π}{\texorpdfstring{\ensuremath{\Pi}}{Pi}}
\newunicodechar{τ}{\texorpdfstring{\ensuremath{\tau}}{tau}}
\newunicodechar{ν}{\texorpdfstring{\ensuremath{\nu}}{nu}}
\newunicodechar{κ}{\texorpdfstring{\ensuremath{\kappa}}{kappa}}
\newunicodechar{θ}{\texorpdfstring{\ensuremath{\theta}}{theta}}
\newunicodechar{η}{\texorpdfstring{\ensuremath{\eta}}{eta}}
\newunicodechar{φ}{\texorpdfstring{\ensuremath{\varphi}}{phi}}
\newunicodechar{π}{\texorpdfstring{\ensuremath{\pi}}{pi}}
\newunicodechar{ε}{\texorpdfstring{\ensuremath{\varepsilon}}{epsilon}}
\newunicodechar{δ}{\texorpdfstring{\ensuremath{\delta}}{delta}}
\newunicodechar{ℝ}{\texorpdfstring{\ensuremath{\mathbb{R}}}{R}}
\newunicodechar{ℤ}{\texorpdfstring{\ensuremath{\mathbb{Z}}}{Z}}
\newunicodechar{ℕ}{\texorpdfstring{\ensuremath{\mathbb{N}}}{N}}
\newunicodechar{≤}{\texorpdfstring{\ensuremath{\le}}{<=}}
\newunicodechar{≥}{\texorpdfstring{\ensuremath{\ge}}{>=}}
\newunicodechar{≠}{\texorpdfstring{\ensuremath{\ne}}{!=}}
\newunicodechar{∈}{\texorpdfstring{\ensuremath{\in}}{in}}
\newunicodechar{⊆}{\texorpdfstring{\ensuremath{\subseteq}}{subseteq}}
\newunicodechar{∨}{\texorpdfstring{\ensuremath{\lor}}{or}}
\newunicodechar{∧}{\texorpdfstring{\ensuremath{\land}}{and}}
\newunicodechar{⪯}{\texorpdfstring{\ensuremath{\preceq}}{preceq}}
\newunicodechar{⪰}{\texorpdfstring{\ensuremath{\succeq}}{succeq}}
\newunicodechar{≡}{\texorpdfstring{\ensuremath{\equiv}}{equiv}}
\newunicodechar{≅}{\texorpdfstring{\ensuremath{\cong}}{cong}}
\newunicodechar{⊣}{\texorpdfstring{\ensuremath{\dashv}}{dashv}}
\newunicodechar{⊥}{\texorpdfstring{\ensuremath{\perp}}{perp}}
\newunicodechar{⊕}{\texorpdfstring{\ensuremath{\oplus}}{oplus}}
\newunicodechar{⊙}{\texorpdfstring{\ensuremath{\odot}}{odot}}
\newunicodechar{⋅}{\texorpdfstring{\ensuremath{\cdot}}{.}}
\newunicodechar{⟶}{\texorpdfstring{\ensuremath{\longrightarrow}}{->}}
\newunicodechar{⥤}{\texorpdfstring{\ensuremath{\to}}{->}}
\newunicodechar{⋙}{\texorpdfstring{\ensuremath{\ggg}}{>>>}}
\newunicodechar{∞}{\texorpdfstring{\ensuremath{\infty}}{infty}}
\newunicodechar{∀}{\texorpdfstring{\ensuremath{\forall}}{forall}}
\newunicodechar{∃}{\texorpdfstring{\ensuremath{\exists}}{exists}}

% text/math 両対応($_1$ は数式内で壊れる)
\newunicodechar{₁}{\texorpdfstring{\ensuremath{_{1}}}{_1}}
\newunicodechar{⁺}{\texorpdfstring{\ensuremath{^{+}}}{+}}

\newunicodechar{Č}{\v C}
\newunicodechar{č}{\v c}
\newunicodechar{Š}{\v S}
\newunicodechar{š}{\v s}
\newunicodechar{Ž}{\v Z}
\newunicodechar{ž}{\v z}
\newunicodechar{–}{--}
\newunicodechar{—}{---}
\newunicodechar{§}{\S}
\newunicodechar{’}{'}
\newunicodechar{“}{``}
\newunicodechar{”}{''}

% --- HOTFIX: make \n math-safe (avoid \unskip in math mode) ---
\makeatletter
\let\n\relax
\DeclareRobustCommand{\n}{\ifmmode\;\else\unskip\space\fi}
\makeatother

% === Document Metadata ===
\title{AK High-Dimensional Projection Structural Theory\\
\Large Version 17.0.1: Collapse Structures, Group Simplification, and Persistent Projection Geometry} 
\author{\textbf{Atsushi Kobayashi} \quad {\small (with ChatGPT Research Partner)}}
\date{December 2025}

% === Document Start ===
\begin{document}

\maketitle



\begin{abstract}
We present \textbf{AK--HDPST v17.0.1}, a comprehensive framework that unifies rigorous topological auditing with AI-driven \emph{High-Dimensional Projection Search (HDPS)}. By re-expressing the \(\delta\)-ledger of v16.5 as a scalar \textbf{Defect Potential} \(\Phi\), we transform passive diagnosis into an active navigation problem on the parameter space \(\mathcal{M}\).

\textbf{Part I: Core Theory (The Auditor).}
Retaining the \emph{Unified Collapse Contract (UCC)} of v16.5, we work in constructible one-parameter persistence over a field. The exact bar-deletion reflector \(\mathbf{T}_\tau\) (collapse at scale \(\tau\)) is \(1\)-Lipschitz and idempotent on persistence and provides the foundational lens for all diagnostics. In addition, we track \emph{tower diagnostics} \((\mu,\nu)\) on certified regions, where \(\DiagZero\) is the obstruction-free outcome and \(\DiagNonzero\) indicates an essential (Type~IV) obstruction. On \textbf{Denef--Pas definable} windows, we adopt a \textbf{Local Reverse-Bridge admissibility rule} at the search layer,
\[
\mathrm{Ext}^1 = 0 \ \wedge \ \text{Spectral-Gap Condition} \ \Longrightarrow \ \mathrm{PH}_1 = 0,
\]
authorizing a controlled translation from categorical vanishing to topological regularity under audited spectral separation, without asserting any unconditional global equivalence \(\mathrm{Ext}^1 \Leftrightarrow \mathrm{PH}_1\).

\textbf{Part II: HDPS Engine (The Navigator).}
We partition \(\mathcal{M}\) into \emph{Terrain Cells} and deploy autonomous agents:
\begin{itemize}
    \item The \textbf{Hunter} minimizes \(\Phi\) via regime-aware descent to locate valid regions (\(\Phi < \mathrm{gap}_\tau\)).
    \item The \textbf{Lifter} addresses \textbf{Type~IV} obstructions (nonzero diagnostics \(\DiagNonzero\)) by dimensional extension, subject to a strict \textbf{Lifting Penalty} recorded in the quantale ledger.
    \item The \textbf{Mapper} glues local certificates into a global \textbf{Map of Validity}.
\end{itemize}
This architecture replaces black-box AI predictions with reproducible, white-box computational certificates. The resulting \textbf{AK Structural Regularity Theorem} certifies that, under global boundedness of \(\Phi\), certified coverage of \(\mathcal{M}\), and \emph{absence of Type~IV obstructions} (i.e.\ \(\DiagZero\) on all certified cells), the object collapses to the trivial class after \(\mathbf{T}_\tau\) in the audited persistence/realization pipeline. \textbf{[Spec]} Any downstream interpretation for specific problem families (e.g.\ Navier--Stokes flows, families of elliptic curves) requires separate \emph{faithful realization} hypotheses (Appendix~NS / AK\_AP) and is not asserted unconditionally here. All protocols are machine-verifiable via the \texttt{run.yaml} Proof Object and the associated audit logs.
\end{abstract}




% =========================================================================
% Chapter 1 : Collapse — Operational Definition and Scope (v17.0 / UCC+HDPS)
% =========================================================================
\section{Chapter 1: Collapse — Operational Definition and Scope}
\addcontentsline{toc}{section}{Collapse — Operational Definition and Scope}

%------------------------------------------
% Scope Box (Global Guard-Rails; UCC-enhanced)
%------------------------------------------
\noindent\fbox{%
\parbox{\textwidth}{%
\textbf{Scope Box (UCC guard-rails; after-collapse \& search-ready).}
All statements in this chapter (and throughout) are made under the following guard-rails, collectively referred to as the \emph{UCC layer} (Index/Collapse/Audit/Search):
\begin{itemize}[leftmargin=1.25em]
  \item \textbf{Constructible 1D over a field.}
        We work in constructible one-parameter persistence over a \emph{field}.
        Filtered (co)limits are computed objectwise in \([\mathbb{R},\mathsf{Vect}]\) and then \emph{returned} to the constructible subcategory by verification or by applying \(\mathbf{T}_\tau\) (Appendix~A).

  \item \textbf{Index/Quantale enrichment; definable windows.}
        The time index \((\mathbb{R},\le)\) is enriched over a \emph{commutative quantale} \((\mathsf{V},\otimes,\mathbb{I},\le)\) (e.g.\ \([0,\infty],+\!,0\)).
        This quantale serves dual roles: as a metric for \emph{audit} and as a cost space for \emph{search}.
        Windows are \emph{right-open} intervals \([u,u')\) and definable (o-minimal or Denef--Pas),
ensuring finite event sets; moreover, the covers used for gluing are required to have finite \v{C}ech depth (Theorem~\ref{thm:dp-sum}).


  \item \textbf{After-collapse policy.}
        \emph{All equalities, exactness claims, monotonicity, comparisons, and gluing} are asserted only \emph{after collapse} at the persistence layer; concretely, we evaluate by the protocol
        \[
          \boxed{\ \text{for each } t\ \Longrightarrow\ \mathbf{P}_i\ \Longrightarrow\ \mathbf{T}_\tau\ \Longrightarrow\ \text{compare in }\Pers^{\mathrm{cons}}_k\ }.
        \]

  \item \textbf{Bridge policy (one-way globally; local reverse under Safety).}
        The forward bridge \(\mathrm{PH}_1\Rightarrow \Ext^1\) is established in Chapter~3 under hypotheses (B1)--(B3).
        A \emph{local reverse bridge} \(\Ext^1\Rightarrow\mathrm{PH}_1\) is \emph{only} authorized on windows satisfying the \emph{Spectral-Gap Condition} and the Collapse-Consistent Conditions (CCC) of Chapter~16.
        No global equivalence \(\mathrm{PH}_1\Leftrightarrow\Ext^1\) is claimed.

  \item \textbf{Dual Mode: Audit \& Navigation.}
  \begin{itemize}
      \item \textbf{Audit Mode:} \(\Sigma\delta < \mathrm{gap}_\tau\) certifies validity (proof) on a window via B-Gate\(^{+}\).
      \item \textbf{Navigation Mode:} \(\Sigma\delta\) is scalarized to a potential \(\Phi\) and minimized by AI agents (HDPS).
  \end{itemize}
  Non-commutation and implementation errors are externalized in the \(\delta\)-ledger, which aggregates in \(\mathsf{V}\) and supports both modes.
\end{itemize}
Appendices detail implementable ranges (PF/BC in Appendix~N; Mirror/Transfer in Appendix~L; AWFS in Appendix~K; reproducibility in Appendix~G; Definability in Appendix~Q; Iwasawa in Appendix~R; AI Agents in Appendix~U).%
}%
}

\medskip

%------------------------------------------
% Unified Collapse Contract (UCC)
%------------------------------------------
\begin{theorem}[Unified Collapse Contract (UCC)]\label{thm:UCC}
Fix a commutative quantale \(\mathsf{V}\) for distances/budgets, and right-open windows (optionally definable).
Then:
\begin{enumerate}[label=(\roman*),leftmargin=1.25em]
  \item \textbf{\(\mathbf{T}_\tau\) is a \(\mathsf{V}\)-nucleus.}
        For each \(\tau>0\), the Serre reflector \(\mathbf{T}_\tau\) is idempotent, \(\mathsf{V}\)-\(1\)-Lipschitz for \(d_{\mathrm{int}}\), and a closure/nucleus on the \(\tau\)-local subcategory (Appendix~A).

  \item \textbf{\(C_\tau\) is idempotent up to f.q.i.\ in \(\Ho(\mathsf{FiltCh})\).}
        Any filtered lift \(C_\tau\) is an idempotent comonad in \(\Ho(\mathsf{FiltCh})\) \emph{up to filtered quasi-isomorphism} (Appendix~B/K).

  \item \textbf{Deletion vs.\ inclusion after-collapse.}
        After applying \(\mathbf{T}_\tau\), deletion-type steps are \emph{monotone} (energy non-increasing), and inclusion-type steps are \emph{non-expansive} (Appendix~E).

  \item \textbf{Quantale \(\delta\)-ledger as Potential.}
        The \(\delta\)-ledger aggregates all residuals additively in \(\mathsf{V}\).
        It is \emph{subadditive} under composition and \emph{non-increasing} under after-collapse \(1\)-Lipschitz post-processing.
        This allows \(\Sigma\delta\) to function as a stable Defect Potential \(\Phi\) for high-dimensional search (Chapter~13).
\end{enumerate}
\end{theorem}

\noindent\emph{Proof sketch.}
Unchanged from v16.5, with (iv) extended to support the potential interpretation via Appendix~S.

\medskip

%------------------------------------------
% Windowed policy + UCC terminology
%------------------------------------------
\subsection*{1.0. Windowed proof policy, UCC-Contract, Overlap Gate, and AI Integration}

\begin{definition}[UCC-Contract]\label{def:UCC-Contract}
A \emph{UCC-Contract} on a run consists of:
\begin{itemize}[leftmargin=1.25em]
  \item \textbf{Index layer.}
        A quantale \(\mathsf{V}\) and a MECE family of definable windows \([u,u')\).

  \item \textbf{Collapse layer.}
        The exact reflector \(\mathbf{T}_\tau\) and a filtered lift \(C_\tau\); plus the after-collapse protocol
        \(\mathbf{P}_i \to \mathbf{T}_\tau \to \text{compare in }\Pers^{\mathrm{cons}}_k\).

  \item \textbf{Audit/Search layer.}
        A \(\delta\)-ledger with values in \(\mathsf{V}\).
        \begin{itemize}
            \item In \textbf{Audit Mode}, it enforces the inequality \(\Sigma\delta < \mathrm{gap}_\tau\).
            \item In \textbf{Search Mode}, its scalarization provides the potential \(\Phi\) and (discrete) gradients \(\nabla\Phi\) for Hunter agents (Chapter~14).
        \end{itemize}
\end{itemize}
The run manifest \texttt{run.yaml} (Appendix~G) records the mode and all budgets.
\end{definition}

\begin{definition}[Windows (MECE) and \(\delta\)-ledger]\label{def:windows-delta}
A \emph{domain window} is a right-open interval \(W=[u,u')\).
A windowing is \emph{MECE} (Mutually Exclusive, Collectively Exhaustive) if these windows partition the time axis.
For per-step budgets \(\delta_j\in\mathsf{V}\), the \emph{pipeline budget} is
\[
\Sigma\delta(i,\{\tau_j\})\ :=\ \bigoplus\nolimits_j \delta_j(i,\tau_j).
\]
In Search Mode this sum serves as the \textbf{Defect Potential} \(\Phi\).
\end{definition}

\begin{definition}[Window restriction (cropping) functor]\label{def:cropping}
For a window \(W=[u,u')\), define
\(\Crop_{W}:\Pers^{\mathrm{cons}}_k\to \Pers^{\mathrm{cons}}_k\) by restricting bars to \(W\).
The functor \(\Crop_{W}\) is exact and \(1\)-Lipschitz for the interleaving distance \(\intdist\).
\end{definition}

\begin{definition}[Overlap Gate]\label{def:overlap-gate}
For overlapping windows \(W_\alpha, W_\beta\), the \emph{Overlap Gate} passes if:
\begin{enumerate}[label=(\roman*),leftmargin=1.25em]
\item \(\intdist\big(\Crop_{W_\alpha\cap W_\beta}\mathcal{B}_{\alpha,i},\,\Crop_{W_\alpha\cap W_\beta}\mathcal{B}_{\beta,i}\big)\le \Sigma\delta_{\alpha\beta}\);
\item the safety margin satisfies \(\mathrm{gap}_\tau > \Sigma\delta_{\alpha\beta}\);
\item tower diagnostics satisfy \(\DiagZero\).
\end{enumerate}
Global passing pastes local certificates into a coherent Map of Validity (Mapper Protocol, Chapter~14).
\end{definition}

\begin{definition}[B-Gate\(^{+}\) (after-collapse gate)]\label{def:bgate-plus}
On a window \(W\) at threshold \(\tau\), \emph{B-Gate\(^{+}\) passes} if:
\begin{enumerate}[label=(\arabic*),leftmargin=1.25em]
\item \(\PH_1(C_\tau F|_W)=0\);
\item \(\Ext^1(\mathcal{R}(C_\tau F|_W),k)=0\) (eligibility checked);
\item \(\DiagZero\);
\item \(\mathrm{gap}_\tau > \Sigma\delta\) (Budget Check).
\end{enumerate}
In Search Mode, failure of (4) triggers the Hunter; failure of (3) triggers the Lifter (Chapter~14).
\end{definition}

%------------------------------------------
% Terminology and notation (succinct)
%------------------------------------------
\subsection*{1.1. Terminology and notation}
Fix a base field \(k\).
\(\Vect_k\) denotes finite-dimensional \(k\)-vector spaces.
\(\FiltCh\) denotes finite-type filtered chain complexes.
\(\mathbf{P}_i\) is degreewise persistence.
\(\intdist\) is the interleaving distance.
\textbf{Convention:} All filtered (co)limits are objectwise in \([\mathbb{R},\Vect_k]\) and validated for constructibility (Appendix~A).

%------------------------------------------
% Canon: Type IV diagnostics convention (v17.0.1)
%------------------------------------------
\begin{remark}[Canon: Type IV diagnostic convention]\label{rk:canon-typeiv}
Throughout v17.0, the tower diagnostics \((\mu_{\mathrm{Collapse}},\nu_{\mathrm{Collapse}})\) are interpreted as follows:
\[
\begin{aligned}
\DiagZero &\text{ means obstruction-free (normal),}\\
\DiagNonzero &\text{ means a Type~IV obstruction.}
\end{aligned}
\]
Accordingly, any “absence of Type~IV” condition is stated as \(\DiagZero\) on all certified cells/windows.
\end{remark}

%------------------------------------------
% Operational collapse
%------------------------------------------
\subsection*{1.2. Collapse (operational)}

\begin{definition}[Exact truncation \(\mathbf{T}_\tau\) and lift \(C_\tau\)]\label{def:Ttau-Ctau}
\(\mathbf{T}_\tau:\Pers^{\mathrm{cons}}_k\to\Pers^{\mathrm{cons}}_k\) is the exact reflective localization deleting bars of length \(\le \tau\).
A filtered lift \(C_\tau:\FiltCh\to\FiltCh\) is chosen so that
\(\mathbf{P}_i \circ C_\tau \cong \mathbf{T}_\tau \circ \mathbf{P}_i\) up to filtered quasi-isomorphism.
\end{definition}

\begin{definition}[Collapse Zone \(\mathfrak{C}\)]
\[
\mathfrak{C} := \{F \mid \PH_1(F)=0 \ \wedge\ \Ext^1(\mathcal{R}(F),k)=0\}.
\]
Under the UCC, PH-collapse implies Ext-collapse (Chapter~3).
A local reverse bridge \(\Ext^1\Rightarrow\PH_1\) is available only under the Spectral-Gap and CCC hypotheses of Chapter~16 (marked \emph{[Spec]}).
\end{definition}

%------------------------------------------
% Failure landscape and tower diagnostics
%------------------------------------------
\subsection*{1.3. Failure landscape and the invisible obstruction \texorpdfstring{\((\mu,\nu)\)}{(mu, nu)}}
We retain the failure classification from v16.5:
\begin{itemize}[leftmargin=1.25em]
  \item \textbf{Type I--III:} Observable defects (topological, categorical, instability).
        These correspond to non-vanishing \(\PH_1\), \(\Ext^1\), or violation of monotonicity, and can be targeted directly by Hunter agents via \(\Phi\)-minimization.

  \item \textbf{Type IV (Invisible):}
        Tower limits \(\phi_i\) fail to be isomorphisms, yielding \((\mu_{\mathrm{Collapse}},\nu_{\mathrm{Collapse}}) \neq (0,0)\).
        This indicates an \emph{essential singularity} in the parameter space, treated in the HDPS layer by Dimensional Lifting (Lifter Agent, Chapter~14).
\end{itemize}

%------------------------------------------
% Convergence manager (definable cover; summability)
%------------------------------------------
\subsection*{1.5. Convergence manager (definable cover; summability)}

\begin{theorem}[Definable countable cover \(\Rightarrow\ \Sigma\delta<\infty\)]\label{thm:dp-sum}
Let \(\{W_n\}\) be a countable family of right-open \emph{definable} windows with finite Čech depth \(K\).
If \(\sum_n \delta(W_n)<\infty\) in \(\mathsf{V}\), then the global overlap error is bounded, and gluing via Overlap Gates holds on the covered domain.
\end{theorem}

\medskip

\noindent\textit{References and provenance.}
Existence, exactness, and interval decomposition in the constructible \(1\)D range follow standard sources (e.g.\ Crawley--Boevey; Chazal--de Silva--Glisse--Oudot).
Derived/sheaf-theoretic and Fukaya-category realizations are \textbf{[Spec]} (Appendix~N/O); they do not enlarge the proven bridge.
AWFS/2-cell commutation and quantitative soft-commuting are \textbf{[Spec]} (Appendix~K/L).
Denef--Pas definability is detailed in Appendix~Q; Iwasawa alignment in Appendix~R; Restart/Summability in Appendix~J.

%------------------------------------------
% Manifest requirements and Badges
%------------------------------------------
\subsection*{1.7. Manifest requirements and Badges}

\paragraph{run.yaml (v17.0 schema).}
\[
\texttt{quantale/\ definable/\ awfs/\ diagnostics/\ gates/\ cover/\ delta\_budget/\ search\_strategy}
\]
New keys: \texttt{search\_strategy} (Hunter configuration), \texttt{lifting\_penalty} (Lifter cost policy).

\begin{remark}[Naming note: \texttt{nu} in \texttt{run.yaml}]\label{rk:naming-nu}
In \texttt{run.yaml}, the keys \texttt{mu}/\texttt{nu} under \texttt{diagnostics} refer to the
collapse diagnostics \((\mu_{\mathrm{Collapse}},\nu_{\mathrm{Collapse}})\), and are distinct from any PDE viscosity parameter.
\end{remark}

\paragraph{Badges.}
\emph{Proof}: UCC (Theorem~\ref{thm:UCC}), Convergence (Theorem~\ref{thm:dp-sum}).
\emph{Spec}: AI Agent Protocols (Chapter~14), Defect Potential \(\Phi\) (Chapter~13), Reverse Bridge Programs (Chapter~16).




% ===========================
% Chapter 2 : Concrete Model — Finite-Type Filtered Chain Complexes and Thresholded Collapse (v17 / UCC)
% IMRN/AiM style: concise, proof-first where canonical; [Spec] marks implementable-but-abstract claims.
% ===========================
\section{Chapter 2: Concrete Model — Finite-Type Filtered Chain Complexes and Thresholded Collapse}\label{sec:ch2}
\addcontentsline{toc}{section}{Concrete Model — Finite-Type Filtered Chain Complexes and Thresholded Collapse}

\subsection*{2.1. The category \texorpdfstring{$\FiltCh$}{FiltCh(k)} and persistence modules}\label{subsec:ch2-cat}
Fix a field \(k\). Let \(\mathsf{Ch}^{b}(k)\) be the category of bounded chain complexes of finite-dimensional \(k\)-vector spaces. A \emph{finite-type filtered chain complex} is a pair
\[
F=(C_\bullet,\{F^{t}C_\bullet\}_{t\in\bbR})
\]
where \(F^{t}C_\bullet\subseteq F^{t'}C_\bullet\) for \(t\le t'\), the filtration is exhaustive and left-bounded, and, for each \(i\), the persistence module
\[
\mathrm{H}_i(F):\ \bbR\longrightarrow \Vectk,\qquad t\longmapsto \mathrm{H}_i(F^{t}C_\bullet)
\]
is pointwise finite-dimensional with finitely many critical parameters on compact intervals. Denote by \(\FiltCh\) the category of such \(F\) with filtration-preserving chain maps. For each \(i\), let
\[
\mathbf{P}_i:\ \FiltCh\longrightarrow \Perscons
\]
be the functor sending \(F\mapsto \mathrm{H}_i(F)\). We write \(\PH_i(F)\) for the barcode (multiset of intervals) of \(\mathbf{P}_i(F)\). Throughout, the interleaving (equivalently, bottleneck) distance on persistence modules is denoted \(\intdist\).

\paragraph{Standing convention (constructible range and notation).}
We work inside the constructible subcategory
\(\Perscons\subset\Pers\) (finite critical set on bounded intervals, equivalently p.f.d.\ with finitely many changes on compacts).
We identify \(\Persft\) with \(\Perscons\) by convention.
\(\Perscons\) is abelian, admits interval decompositions, and carries a well-defined length. All uses of abelianity, Serre subcategories, and exact localizations are made within \(\Perscons\); see Appendix~A for details.
For filtered complexes we keep the finite-type hypothesis and record that filtered colimits may exit finiteness (cf.\ Appendix~A). Filtered (co)limits, when used, are computed objectwise in \([\bbR,\Vectk]\) and then verified to return to \(\Perscons\); no claim is made outside this regime.

\begin{remark}[Constructible abelian setting]\label{rem:constructible-abelian}
Within \(\Perscons\), kernels and cokernels are computed pointwise and preserve finiteness; thus \(\Perscons\) is abelian with interval decompositions and Serre localizations by bar length. For each \(\tau>0\), the full subcategory \(\Ecat\) generated by interval modules of length \(\le \tau\) is a hereditary Serre (localizing) subcategory in this constructible \(1\)D setting.
\end{remark}

\begin{remark}[Windowed proof policy; right-open endpoints]\label{rem:ch2-mece}
Statements are applied \emph{per window} (cf.\ Chapter~1). A \emph{domain window} is a right-open interval \([u_k,u_{k+1})\). A windowing is \emph{MECE} if \(\bigsqcup_k [u_k,u_{k+1})=[u_0,U)\) and adjacent windows meet only at endpoints. Coverage checks: (i) \(\sum_k (u_{k+1}-u_k)=U-u_0\); (ii) event counts (births/deaths, with multiplicity) add over windows up to rounding tolerance. Thresholds and spectral bins are \emph{fixed per window} and used only \emph{after collapse}.
\end{remark}

\subsection*{2.2. Thresholded collapse: Serre localization on persistence and filtered lift}\label{subsec:ch2-collapse}
We recall truncation on persistence, give a \(V\)-enriched interleaving view, then lift to filtered complexes.

\paragraph{Lawvere \(V\)-distance and \(V\)-shifts.}\label{def:Vlawvere}
Let \((V,\le,\otimes,\mathbb{I})\) be a commutative unital quantale. A \emph{\(V\)-Lawvere metric} on \(\Perscons\) is encoded by a system of endofunctors \(\{S^{v}\}_{v\in V}\) with coherences:
\begin{enumerate}[leftmargin=1.25em]
    \item \(S^{\mathbb{I}}\cong \mathrm{Id}\), \(S^{v}\circ S^{w}\cong S^{v\otimes w}\), and if \(v\le w\) then \(S^v\Rightarrow S^w\).
    \item For intervals \(I[a,b)\), \(S^v\) preserves bar lengths (classically \(S^\varepsilon\) is the \(\varepsilon\)-shift).
\end{enumerate}
Two objects \(M,N\) are \(v\)-interleaved if there are \(f:M\!\to\! S^vN\), \(g:N\!\to\! S^vM\) closing to the units; put \(d_V(M,N):=\inf\{v\mid M,N\ \text{are }v\text{-interleaved}\}\). For \(V=([0,\infty],\le,+,0)\), \(d_V=\intdist\).

\paragraph{Ephemeral part and localization (constructible \(1\)D).}
Let \(\Ecat\subset\Perscons\) be the Serre subcategory generated by intervals of length \(\le\tau\).
The reflector
\[
\Ttau:\ \Perscons\longrightarrow \Orth
\]
is the exact localization at \(\Ecat\) and is \(1\)-Lipschitz for \(\intdist\) (in fact \(V\)-\(1\)-Lipschitz for \(d_V\); Lemma~\ref{lem:Vshift}). Here \(\Orth\) is the \(\tau\)-local (orthogonal) subcategory. Concretely,
\(\Ttau(M)=M/E_\tau(M)\) with \(E_\tau\) the maximal \(\tau\)-ephemeral subobject.

\begin{remark}[Endpoint independence]\label{rk:2-endpoints}
\(\Ttau\) deletes precisely finite bars of length \(\le\tau\); infinite bars are invariant. Open/closed endpoint conventions do not affect \(\Ttau\) (bar lengths are interleaving invariants).
\end{remark}

\begin{lemma}[\(V\)-shift commutation and \(V\)-\(1\)-Lipschitz]\label{lem:Vshift}
For any \(v\in V\), \(\Ttau\circ S^{v}\cong S^{v}\circ \Ttau\). Hence \(\Ttau\) preserves \(v\)-interleavings and is \(V\)-\(1\)-Lipschitz.
\emph{Sketch.} \(S^v\) preserves \(\Ecat\) and descends to the Serre quotient; use the universal property of localization. \qed
\end{lemma}

\paragraph{Lifting to filtered complexes.}
Fix a functor \(\mathcal{U}:\Perscons\to\FiltCh\) realizing interval modules by elementary filtered complexes. For \(F\in\FiltCh\) define \(C_\tau(F)\) by filtered quasi-isomorphisms
\[
\mathbf{P}_i\big(\Ctau(F)\big)\ \xrightarrow{\ \cong\ }\ \Ttau\!\big(\mathbf{P}_i(F)\big)\qquad\text{for all }i.
\]
Any functorial choice (up to f.q.i.) is a \emph{thresholded collapse}.

\begin{proposition}[Stability and calculus (persistence layer; after collapse)]\label{prop:stability}
Fix a threshold \(\tau\ge 0\). Let \(\mathbf{T}_\tau:\Perscons\to\Orth\) be the Serre localization (bar\hyp deletion),
and let \(\Ctau\) be any filtered lift defined up to filtered quasi\hyp isomorphism as above.
Then, for all \(F,G\in\mathsf{FiltCh}(k)\) and every degree \(i\), the following hold at the persistence layer:
\begin{enumerate}[leftmargin=1.25em]
  \item \emph{Exactness.} \(\mathbf{T}_\tau\) is exact (hence preserves finite limits and colimits) and fits into a reflective adjunction
  \(\mathbf{T}_\tau\dashv \iota_\tau\) with the inclusion of the \(\tau\)\hyp torsion\hyp free subcategory.
  \item \emph{Monotonicity and idempotence.} If \(\tau\le\sigma\) then there is a natural transformation \(\mathbf{T}_\sigma\Rightarrow \mathbf{T}_\tau\) and
  \(\mathbf{T}_\tau\circ \mathbf{T}_\sigma\cong \mathbf{T}_{\max\{\tau,\sigma\}}\); in particular \(\mathbf{T}_\tau^2\cong \mathbf{T}_\tau\).
  \item \emph{Non\hyp expansiveness.} For all \(F,G\) and all \(i\),
  \[
  \intdist\big(\mathbf{T}_\tau\mathbf{P}_i(F),\,\mathbf{T}_\tau\mathbf{P}_i(G)\big)\ \le\ \intdist\big(\mathbf{P}_i(F),\,\mathbf{P}_i(G)\big).
  \]
  Equivalently, any lift \(\Ctau\) satisfies \(\intdist(\mathbf{P}_i(\Ctau F),\mathbf{P}_i(\Ctau G))\le \intdist(\mathbf{P}_i(F),\mathbf{P}_i(G))\).
  \item \emph{Window commutation.} For any right\hyp open window \(W\), \(\Crop_W\circ \mathbf{T}_\tau\cong \mathbf{T}_\tau\circ \Crop_W\) (Lemma~\ref{lem:CropT}).
  \item \emph{\(V\)\hyp shift commutation and \(V\)\hyp \(1\)\hyp Lipschitz.} For any \(v\in V\), \(\mathbf{T}_\tau\circ S^{v}\cong S^{v}\circ \mathbf{T}_\tau\),
  hence \(\mathbf{T}_\tau\) preserves \(v\)\hyp interleavings and is \(V\)\hyp \(1\)\hyp Lipschitz (Lemma~\ref{lem:Vshift}).
\end{enumerate}
\end{proposition}


% --- NEW: CNF (P1) and Ext-identification (P2) ---
\subsection*{2.2.1. Collapse Normal Form and the Ext–Hom edge}\label{subsec:CNF}
\begin{theorem}[Collapse Normal Form (CNF)]\label{thm:CNF}
In \(D^{\mathrm b}(k\text{-mod})\), every object \(X\) is isomorphic (in general, non-canonically) to the direct sum of its cohomology objects placed in degrees:
\[
X\ \cong\ \bigoplus_{i\in\mathbb{Z}} H^i(X)[-i].
\]
Moreover, the isomorphism class of \(X\) is determined by the graded object \(\{H^i(X)\}_i\).
\emph{Proof sketch.} Over a field, the abelian category \(k\text{-mod}\) is semisimple, hence every bounded complex is quasi-isomorphic to its cohomology with zero differential. The resulting decomposition yields the stated isomorphism in the derived category. \qed
\end{theorem}


\begin{corollary}[Ext--Hom edge (degree \(1\))]\label{cor:ExtHom}
For any \(X\in D^{\mathrm b}(k\text{-mod})\),
\[
\Ext^1(X,k)\ \cong\ \Hom\!\big(H^{-1}(X),k\big).
\]
\emph{Proof.} Using the (non-canonical) splitting \(X\simeq \bigoplus_i H^i(X)[-i]\) over a field,
\(\Ext^1(X,k)=\Hom(X,k[1])\) receives a nonzero contribution only from the summand \(H^{-1}(X)[1]\), yielding \(\Hom(H^{-1}(X),k)\). \qed
\end{corollary}


\begin{remark}[Use in gates]
Applied to \(X=\mathcal{R}(C_\tau F|_W)\) (amplitude \(\le 1\)), \Cref{cor:ExtHom} identifies \(\Ext^1\) with the edge \(H^1\); on \(E_1\)-degenerate windows \(H^1\cong \PH_1\), yielding a \textnormal{\emph{[Spec]}} local reverse-bridge under CCC/Spectral-Gap hypotheses (Chapter~16).
\end{remark}

\subsection*{2.3. Operator toolkit at window scale}\label{subsec:toolkit}
We collect the window-level operators and their basic interactions.

\paragraph{Cropping on persistence.}
For a right-open window \(W=[u,u')\), let
\[
\Crop_W:\ \Perscons\to\Perscons
\]
restrict bars to \(W\) (precompose with \(W\hookrightarrow\RR\), extend by \(0\)). \(\Crop_W\) is exact and \(1\)-Lipschitz for \(\intdist\).

\begin{lemma}[Commutation with truncation]\label{lem:CropT}
\(\Crop_W\circ \mathbf{T}_\tau\ \cong\ \mathbf{T}_\tau\circ \Crop_W\) for all \(W,\tau\).
\emph{Proof.} \(\Crop_W\) preserves bar lengths and the Serre subcategory \(\Ecat\); argue as in Lemma~\ref{lem:Vshift}. \qed
\end{lemma}

\paragraph{Window lift on filtered complexes.}
Write \(W_{\mathrm{clip}}:\FiltCh\to\FiltCh\) for any filtered functor whose persistence equals \(\Crop_W\) degreewise; identities below are asserted at the persistence layer via \(\mathbf{P}_i\circ W_{\mathrm{clip}}\simeq \Crop_W\circ \mathbf{P}_i\).

\subsection*{2.3.1. Safe low-pass (optional; P4)}\label{subsec:lowpass}
Let \(L_\tau\) be a linear smoothing operator on the measured side (signal/filtration axis) with kernel \(\varphi_\tau\) satisfying:
\[
\text{(LP1) even},\qquad \text{(LP2) mass }1,\qquad \text{(LP3) bandwidth }\asymp \sqrt{\tau}.
\]
We only measure \emph{after} applying \(\Ctau\) (persistence layer).

\begin{proposition}[\textnormal{\textbf{[Spec]}} Safe low-pass: non-expansive after collapse]\label{prop:lowpass}
Under \textnormal{(LP1)–(LP3)}, there exists a budget \(\delta^{\mathrm{alg}}_{\mathrm{LP}}(i,\tau)\) such that
\[
\mathbf{T}_\tau\circ \mathbf{P}_i\circ L_\tau\ \cong\ \mathbf{T}_\tau\circ \mathbf{P}_i\quad\text{up to }\delta^{\mathrm{alg}}_{\mathrm{LP}}(i,\tau),
\]
and \(\mathbf{T}_\tau\circ \mathbf{P}_i\circ L_\tau\) is \(1\)-Lipschitz for \(\intdist\).
\emph{Sketch.} (LP1)–(LP3) preserve bar lengths up to sub-threshold deformation; commutation with \(\Ttau\) follows by the same Serre-localization argument, with deviation accounted for in \(\delta^{\mathrm{alg}}_{\mathrm{LP}}\). \qed
\end{proposition}

\begin{definition}[Test \texttt{T-Lipschitz-AfterCollapse}]
Accept \(L_\tau\) on a run iff the empirical Lipschitz constant of \(M\mapsto \mathbf{T}_\tau\mathbf{P}_i(L_\tau M)\) is \(\le 1\) within tolerance and the commutation defect with \(\Ttau\) stays \(\le \delta^{\mathrm{alg}}_{\mathrm{LP}}\) on the declared sample. Logs are recorded in the \(\delta\)-ledger.
\end{definition}

\subsection*{2.3.2. Orthogonal operator form (triad) and \texorpdfstring{$\delta$}{delta}-commutation}\label{subsec:triad}
\begin{definition}[Operator triad and normal form]\label{def:triad}
On each window \(W\) and threshold \(\tau\), we use the triad
\[
\boxed{\ C_\tau\quad\longrightarrow\quad W_{\mathrm{clip}}\quad\longrightarrow\quad L_\tau\ (\text{optional})\ }
\]
with all measurements made on \(\mathbf{T}_\tau\mathbf{P}_i\). We call this the \emph{orthogonal form}.
\end{definition}

\begin{proposition}[Pairwise commutation up to \texorpdfstring{$\delta$}{delta}]\label{prop:pairwise-delta}
For each \(i\), there exist budgets \(\delta^{\mathrm{alg}}_{\mathrm{CW}}(i,\tau;W)\), \(\delta^{\mathrm{alg}}_{\mathrm{WL}}(i,\tau;W)\), \(\delta^{\mathrm{alg}}_{\mathrm{CL}}(i,\tau)\) in the chosen quantale such that
\[
\begin{aligned}
\mathbf{P}_i(C_\tau W_{\mathrm{clip}}F)&\ \cong\ \mathbf{P}_i(W_{\mathrm{clip}} C_\tau F)&&\text{up to }\delta^{\mathrm{alg}}_{\mathrm{CW}},\\
\mathbf{T}_\tau\mathbf{P}_i(W_{\mathrm{clip}} L_\tau F)&\ \cong\ \mathbf{T}_\tau\mathbf{P}_i(L_\tau W_{\mathrm{clip}} F)&&\text{up to }\delta^{\mathrm{alg}}_{\mathrm{WL}},\\
\mathbf{T}_\tau\mathbf{P}_i(C_\tau L_\tau F)&\ \cong\ \mathbf{T}_\tau\mathbf{P}_i(L_\tau C_\tau F)&&\text{up to }\delta^{\mathrm{alg}}_{\mathrm{CL}}.
\end{aligned}
\]
All three maps are \(1\)-Lipschitz after applying \(\mathbf{T}_\tau\).
\emph{Sketch.} Use \Cref{lem:CropT}, \Cref{prop:lowpass}, and the fact that all claims are asserted \emph{after collapse}. \qed
\end{proposition}

\begin{remark}[Recording and acceptance]
Budgets in \Cref{prop:pairwise-delta} are recorded in the \(\delta\)-ledger; if \(\sum\delta<\infty\) on a definable cover of finite Čech depth, Overlap Glue holds globally (Chapter~1; Appendix~J).
\end{remark}

\subsection*{2.4. Windowing (MECE), \texorpdfstring{$\tau$}{tau}-adaptation, and spectral bins}\label{subsec:ch2-window}
\begin{definition}[Domain windows (MECE) and coverage]\label{def:ch2-mece}
A \emph{domain windowing} is a finite or countable family \(\{[u_k,u_{k+1})\}_k\) with \(\bigsqcup_k [u_k,u_{k+1})=[u_0,U)\) and \(u_k<u_{k+1}\). Coverage checks:
\[
\sum_k (u_{k+1}-u_k)=U-u_0,\qquad \#\mathrm{Events}([u_0,U))=\sum_k \#\mathrm{Events}([u_k,u_{k+1}))\ (\pm\ \text{rounding}).
\]
\end{definition}

\begin{definition}[Collapse thresholds and \(\tau\)-sweep]\label{def:tau-adapt}
Fix \(\tau>0\) per window, e.g.\ \(\tau=\alpha\cdot \max\{\Delta t,\Delta x\}\) (\(\alpha>0\) fixed per run). A \emph{\(\tau\)-sweep} is a discrete set \(\{\tau_\ell\}\) on which \((\mu_{\mathrm{Collapse}},\nu_{\mathrm{Collapse}})\) and B-Gate\(^{+}\) are evaluated. A \emph{stable band} is a contiguous range of \(\tau\) with \(\DiagZero\).
\end{definition}

\begin{definition}[Spectral bins and aux-bars]\label{def:aux-bars}
For spectrum \((\lambda_m)_{m\ge 1}\), fix \(\beta>0\) and \([a,b]\). Bins \(I_r=[a+r\beta,a+(r+1)\beta)\) collect counts \(E_r\).
Along a discrete index, runs where \(E_r(j)>0\) define \emph{auxiliary spectral bars} (lifetimes measured in that index). After applying \(\Ctau\), these are monotone under deletion-type steps and stable under \(\varepsilon\)-continuations. They never replace the B-side gate.
\end{definition}

\subsection*{2.5. Collapse admissibility and robust variants}\label{subsec:ch2-admissibility}
Let \(\mathcal{R}:\FiltCh\to\Dbk\) be \(t\)-exact of amplitude \(\le 1\); fix \(\mathcal{Q}=\{k[0]\}\).

\begin{definition}[Admissibility]
\[
\texttt{CollapseAdmissible}(F)\ :\iff\ \PH_1(F)=0\ \ \wedge\ \ \Ext^1\!\big(\mathcal{R}(F),k\big)=0.
\]
Under the bridge (Chapter~3), \(\PH_1(F)=0\Rightarrow \Ext^1(\mathcal{R}(F),k)=0\) in \(\Dbk\).
\end{definition}

\begin{definition}[Robust admissibility at scale \(\varepsilon\)]
\(F\) is \emph{\(\varepsilon\)-robustly collapse-admissible} if
\(\PH_1\!\big(C_\varepsilon(F)\big)=0\) and \(\Ext^1\!\big(\mathcal{R}(C_\varepsilon(F)),k\big)=0\).
\end{definition}

\subsection*{2.6. Local equivalence on saturation windows}\label{subsec:ch2-local-equiv}
\begin{definition}[Saturation window]\label{def:saturation}
Fix \(i=1\), a window \(W=[u,u')\), and \(\tau>0\). \(W\) is a \emph{saturation window} for \(F\) if: (i) event stability holds on \(W\); (ii) the maximal finite bar length in \(W\)\(\le \tau-\eta\) for some \(\eta>0\); (iii) no bar lengths in \(W\) increase to \(\tau\). Also require tail isomorphism: \(\mu_{\mathrm{Collapse}}=\nu_{\mathrm{Collapse}}=0\) on \(W\).
\end{definition}

\begin{theorem}[\textnormal{\textbf{[Spec]}} Local PH--Ext equivalence on CCC saturation windows]\label{thm:local-equiv}
Assume: (1) \(t\)-exact \(\mathcal{R}\) of amplitude \(\le 1\);
(2) \(W\) is a saturation window at \(\tau\);
(3) the CCC and Spectral-Gap Condition of Chapter~16 hold on \(W\);
(4) tail isomorphism at \(\tau\) on \(W\) (i.e.\ \(\mu_{\mathrm{Collapse}}=\nu_{\mathrm{Collapse}}=0\)).
Then, on \(W\) at threshold \(\tau\),
\[
\PH_1\!\big(\Ctau F\big)=0\quad\Longleftrightarrow\quad \Ext^1\!\big(\mathcal{R}(\Ctau F),k\big)=0.
\]
\emph{Sketch.} Use \Cref{cor:ExtHom} on \(\mathcal{R}(C_\tau F|_W)\).
Under CCC/Spectral-Gap, the relevant \(H^1\)-edge detects the window obstruction stably and aligns with the \(\PH_1\)-gate after \(\Ctau\).
Saturation excludes accumulation at \(\tau\). \qed
\end{theorem}


\subsection*{2.7. Length spectrum operator \texorpdfstring{$\Len$}{Lambda\_len} (windowed) and its invariance}\label{subsec:ch2-len}
\begin{definition}[Windowed length spectrum]\label{def:len-operator}
If \(M\cong \bigoplus_{j} I[b_j,d_j)\), define for \(W=[u,u')\) the clipped length \(\ell_W(I[b_j,d_j)):=\max\{0,\min\{d_j,u'\}-\max\{b_j,u\}\}\).
Let \(\Len(M;W)\) be the diagonal endomorphism on \(\bigoplus_j k\cdot e_j\) with eigenvalues \(\{\ell_W(I[b_j,d_j))\}_j\).
\end{definition}

\begin{proposition}[Invariance]\label{prop:length-spectrum}
The multiset of eigenvalues of \(\Len(M;W)\) equals \(\{\ell_W(I[b_j,d_j))\}_j\) and is invariant under isomorphisms \(M\simeq M'\).
\end{proposition}

\begin{remark}[First-length functional and Chapter~11]\label{rem:E1-crosslink}
\(E_1(M;W)=\mathrm{tr}(\Len(M;W))=\sum_j \ell_W(I[b_j,d_j))\). Stability for \(E_1\) follows from \(\Ttau\)'s \(1\)-Lipschitzness.
\end{remark}

\subsection*{2.8. \texorpdfstring{$\varepsilon$}{epsilon}-survival and robustness}\label{subsec:epsilon-survival}
\begin{lemma}[\(\varepsilon\)-survival under interleavings]\label{lem:epsilon-survival}
If \(\intdist(\mathbf{P}_i(F),\mathbf{P}_i(G))\le \varepsilon\), then any bar \(b\) of \(\mathbf{P}_i(F)\) with \([0,\tau_0]\)-clipped length \(\ell_{\tau_0}(b)>2\varepsilon\) has a counterpart in \(\mathbf{T}_{\tau_0}(\mathbf{P}_i(G))\) with clipped length at least \(\ell_{\tau_0}(b)-2\varepsilon\).
\end{lemma}

\subsection*{2.9. Operating summary (Chapter~2)}\label{subsec:ch2-summary}
Per right-open window \([u_k,u_{k+1})\):
\begin{itemize}[leftmargin=1.25em]
  \item Fix \(\tau\) (resolution-adapted) and, if used, spectral binning \((\beta,[a,b])\).
  \item Apply A-side steps (deletion-type or \(\varepsilon\)-continuation), then \(\Ctau\); \emph{measure only on} \(\mathbf{T}_\tau\mathbf{P}_i\).
  \item Use the operator triad \(C_\tau\to W_{\mathrm{clip}}\to L_\tau\) (optional) in orthogonal form; record \(\delta\)-commutation budgets from \Cref{prop:pairwise-delta}.
  \item Verify B-Gate\(^{+}\) and, for covers, the Overlap Gate; paste certificates using Restart/Summability (Chapter~4).
  \item If \Cref{thm:local-equiv} applies, use the window-local PH–Ext equivalence; otherwise keep the forward bridge only.
\end{itemize}

\paragraph{Core labels met in this chapter (cf.\ Chapter~1).}
\emph{P1 (CNF):} \Cref{thm:CNF}. \emph{P2 (Ext–Hom):} \Cref{cor:ExtHom}. \emph{P4 (Low-pass safety):} \Cref{prop:lowpass} with \texttt{T-Lipschitz-AfterCollapse}. Operator normal form \(C_\tau/W_{\mathrm{clip}}/L_\tau\) and \(\delta\)-commutation: \Cref{prop:pairwise-delta}.

\medskip
\noindent\textit{References and provenance.}
Serre localization, barcode abelianity, and interleaving stability are standard in the constructible \(1\)D setting. The CNF for \(D^{\mathrm b}(k\text{-mod})\) and the Ext–Hom edge are classical over fields. Cropping and truncation commute by preservation of the Serre subcategory. Safe low-pass is adopted with explicit acceptance tests; all smoothing is audited \emph{after collapse} and never used as a sole gate.



% ===========================
% Chapter 3 : A One-Way Bridge PH1⇒Ext1 and the Hypothesis Scheme (v17 / UCC)
% IMRN/AiM style: proof-first; [Spec] marks implementable-but-abstract items.
% ===========================
\section*{Chapter 3: A One-Way Bridge \texorpdfstring{$\mathrm{PH}_1\!\Rightarrow\!\mathrm{Ext}^1$}{PH1⇒Ext1} and the Hypothesis Scheme}

\noindent\textbf{Scope note (reinforced windowed policy).}
All statements lie in the constructible \(1\)D regime of Chapter~2 with field coefficients.
The implication \(\mathrm{PH}_1\Rightarrow \Ext^1\) is proved only in \(D^{\mathrm b}(k\text{-mod})\) under \textup{(B1)}–\textup{(B3)}.
Every claim is issued per right-open domain window \(W=[u,u')\) and fixed threshold \(\tau>0\); gate decisions are taken only on the B-side \emph{after collapse}, i.e.\ on single-layer objects \(\mathbf{T}_\tau\mathbf{P}_i\) (equivalently \(\mathbf{P}_i(C_\tau-)\)).
Equalities for filtered complexes hold up to filtered quasi-isomorphism (f.q.i.).

\subsection*{3.0. Windowed usage, \texorpdfstring{$E_1$}{E1}-first policy, and gate integration}
This chapter supplies the \(\mathrm{PH}_1\!\Rightarrow\!\Ext^1\) bridge used by B-Gate\(^{+}\) (Chapter~1) after applying \(C_\tau\) on \((W,\tau)\).
We adopt an \emph{\(E_1\)-first policy}: evaluate the windowed first-length functional \(\,E_1\) \emph{after collapse} as the primary determinant, then discharge Ext via the bridge.
Formally:
\begin{itemize}[leftmargin=1.25em]
  \item If \(E_1(C_\tau F;W)=0\), then \(\mathrm{PH}_1(C_\tau F|_W)=0\) (by definition of \(E_1\); cf.\ Chapter~2, \S2.7); if moreover \(\DiagZero\) and the safety margin satisfies \(\mathrm{gap}_\tau>\Sigma\delta\), B-Gate\(^{+}\) passes.
  \item If \(E_1(C_\tau F;W)>0\), the Ext-part cannot discharge the PH-part; B-Gate\(^{+}\) may still fail due to \(\mathrm{PH}_1>0\) or \(\DiagNonzero\).
  \item Independently, under \textup{(B1)}–\textup{(B3)} we always have the one-way implication \(\mathrm{PH}_1\!\Rightarrow\!\Ext^1\) on \((W,\tau)\).
\end{itemize}
Cross-domain comparisons (PF/BC, Mirror/Transfer) are performed \emph{after} \(C_\tau\); all non-commutations are recorded in the \(\delta\)-ledger.

\subsection*{3.1. Bridging Hypotheses (B1–B3)}
Fix the notation of Chapter~2. In particular, \(k\) is a field, \(\FiltCh(k)\) denotes finite-type filtered chain complexes, \(\mathbf{P}_i(F)\) is the degree-\(i\) persistence with barcode \(\mathrm{PH}_i(F)\), and \(\mathcal{R}:\FiltCh(k)\to D^{\mathrm b}(k\text{-mod})\) is a \(t\)-exact realization.

\begin{description}
\item[\normalfont (B1) Finite-type over a field.]
\(F\in\FiltCh(k)\) with pointwise finite-dimensional persistence. Filtered (co)limits are computed objectwise in \([\RR,\Vect_k]\) and used only within the constructible scope (Appendix~A).

\item[\normalfont (B2) Amplitude \(\boldsymbol{\le 1}\) and identification of the \(\boldsymbol{H^{-1}}\)-edge.]
There is a two-term model
\[
\mathcal{R}(F)\ \simeq\ \bigl[\ \varinjlim_{t} H_1(F^{t}C_\bullet)\ \xrightarrow{\,d\,}\ \varinjlim_{t} H_0(F^{t}C_\bullet)\ \bigr]\in D^{[-1,0]}(k\text{-mod}),
\]
natural in \(F\).

\item[\normalfont (B3) Edge identification for degree \(1\) with \(Q=k\).]
For any \(A\in D^{[-1,0]}(k\text{-mod})\),
\[
\Ext^1(A,k)\ \cong\ \Hom\!\big(H^{-1}(A),k\big),
\]
naturally in \(A\).
\end{description}

\begin{remark}[On (B2) and the edge identification]\label{rk:B2-edge}
By (B2), \(H^{-1}(\mathcal{R}(F))\cong \varinjlim_t H_1(F^tC_\bullet)\); over a field, (B3) gives \(\Ext^1(\mathcal{R}(F),k)\cong \Hom(H^{-1}(\mathcal{R}(F)),k)\). All uses remain in \(D^{\mathrm b}(k\text{-mod})\).
\end{remark}

\subsection*{3.2. One-way bridge and \texorpdfstring{$E_1$}{E1}-local strengthening}
\begin{theorem}[One-way bridge]\label{thm:PH1-to-Ext1}
Assume \textup{(B1)}–\textup{(B3)}. If \(\mathrm{PH}_1(F)=0\), then
\(\Ext^1\!\big(\mathcal{R}(F),k\big)=0\).
\end{theorem}

\begin{proof}
\(\mathrm{PH}_1(F)=0\) implies \(\varinjlim_t H_1(F^tC_\bullet)=0\); apply (B2) and (B3).
\end{proof}

\begin{theorem}[Local Reverse under \(\boldsymbol{E_1{=}0}\) (P3)]\label{thm:local-reverse}
Let \(W\) be a right-open window and \(\tau>0\). Assume after-collapse amplitude \(\le 1\) and tail isomorphism \(\DiagZero\) on \(W\).
If \(E_1(C_\tau F;W)=0\), then for any \(F\),
\[
\Ext^1\!\big(\mathcal{R}(C_\tau F|_W),k\big)=0\ \Longrightarrow\ \mathrm{PH}_1(C_\tau F|_W)=0.
\]
\end{theorem}

\begin{proof}
By Chapter~2, \S2.7, \(E_1(C_\tau F;W)=0\) iff the clipped-length multiset on \(W\) is identically \(0\), hence \(\mathrm{PH}_1(C_\tau F|_W)=0\).
On the other hand, by (B2)--(B3) (equivalently, by the Ext--Hom edge), we have
\(\Ext^1(X,k)\cong \Hom(H^{-1}(X),k)\) for \(X\in D^{[-1,0]}(k\text{-mod})\).
Amplitude \(\le 1\) and tail isomorphism ensure the \(\,H^{-1}\)-edge matches the stabilized degree-\(1\) persistence on \(W\).
Thus the stated implication holds (indeed, the premise on \(\Ext^1\) is superfluous once \(E_1=0\) is known; we keep it to match gate logging).
\end{proof}

\begin{remark}[Position relative to Theorem~\ref{thm:PH1-to-Ext1} and local equivalence]
Theorem~\ref{thm:local-reverse} is a reverse fragment valid on \emph{\(E_1\)-degenerate} windows.
A stronger window-local equivalence (PH\(\Leftrightarrow\)Ext) under CCC/Spectral-Gap authorization appears as a \textnormal{\emph{[Spec]}} statement in Theorem~\ref{thm:E1-local}.below; globally we retain only the one-way bridge Theorem~\ref{thm:PH1-to-Ext1}.
\end{remark}

\subsection*{3.2 bis. E\(_1\)-local equivalence on definable windows}
\begin{theorem}[\textnormal{\textbf{[Spec]}} E\(_1\)-local equivalence on CCC definable windows]\label{thm:E1-local}
Let \(W\) be right-open and o-minimal definable, and fix \(\tau>0\). Assume:
(i) all quantities are evaluated on \(\mathbf{T}_\tau\mathbf{P}_1(F|_W)\) (equivalently \(\mathbf{P}_1(C_\tau F|_W)\));
(ii) \(\mathcal{R}(C_\tau F)\in D^{[-1,0]}(k\text{-mod})\);
(iii) tail isomorphism \(\DiagZero\) on \(W\);
(iv) the CCC and Spectral-Gap Condition of Chapter~16 hold on \(W\) (reverse authorization).
Then
\[
E_1(F;W,\tau)=0\quad\Longleftrightarrow\quad \mathrm{PH}_1(C_\tau F|_{W})=0\quad\Longleftrightarrow\quad
\Ext^1\!\big(\mathcal{R}(C_\tau F|_{W}),k\big)=0.
\]
\end{theorem}


\begin{proof}[Proof sketch]
The first equivalence is by definition of \(E_1\) after collapse (Chapter~2, \S2.7).
The second follows from the two-term realization and Leray descent on a finite Čech nerve (definability), together with tail isomorphism.
\end{proof}

\subsection*{3.3. Quantitative primitives on a fixed window}
For a window \(W\) and threshold \(\tau\), define the residual-length energy
\(
E_i(F;W,\tau)=\sum_{J\in\mathcal{B}_i(F;W)} \max\{|J|-\tau,0\}
\)
and tail counts \(C_{i,r}(F;W,\tau)\) as in Chapter~3, \S3.2 bis; they are finite, piecewise-linear in \(\tau\), monotone, and stable for \(\intdist\).
After collapse,
\(
\sum_{J\in \mathcal{B}_i(C_\tau F;W)} |J|\le E_i(F;W,\tau)
\),
and \(E_i\), \(C_{i,r}\) are non-expansive along \(\tau\)-continuations.

\subsection*{3.4. Survival lemma and safety margins}
\begin{lemma}[ε-survival]\label{lem:epsilon-survival-CH3}
Fix \(W\), \(\tau>0\), and \(\mathrm{gap}_\tau>\Sigma\delta\). If \(F,G\) are \(\varepsilon\)-interleaved on \(W\) and some \(J\in\mathcal{B}_1(F;W)\) satisfies \(|J|_\tau\ge \varepsilon+\mathrm{gap}_\tau\), then a corresponding \(J'\in\mathcal{B}_1(G;W)\) has \(|J'|_\tau\ge \mathrm{gap}_\tau\) and survives collapse on \(G\).
\end{lemma}

\subsection*{3.5. Gate indicators and quantitative linkage}
On \((W,\tau)\), the PH-indicator is \(\mathrm{PH}_1(C_\tau F)\); the Ext-indicator is \(\Ext^1(\mathcal{R}(C_\tau F),k)\); collapse indicators are \((\mu,\nu)\); spectral indicators are \(\{C_r\}_{r\ge 0}\) and \(E_1\).
If \(\mathrm{PH}_1(C_\tau F)=0\), \(\DiagZero\), and \(\mathrm{gap}_\tau>\Sigma\delta\), then by Theorem~\ref{thm:PH1-to-Ext1} the Ext-part discharges and B-Gate\(^{+}\) passes.
If \(C_r(F;W,\tau)=0\) for some \(r>\varepsilon+\Sigma\delta\), then \(\mathrm{PH}_1(C_\tau G)=0\) for every \(G\) \(\varepsilon\)-interleaved with \(F\) on \(W\) (Lemma~\ref{lem:epsilon-survival-CH3}), hence Ext discharges.

\subsection*{3.6. Test \texttt{T-ExtZero-implies-PHZero} (window-local; audit-ready)}
\begin{definition}[Test specification]\label{def:T-ExtZero-PHZero}
On a right-open window \(W\) and threshold \(\tau\), the test \texttt{T-ExtZero-implies-PHZero} \emph{passes} for \(F\) if
\[
\Bigl(\ E_1(C_\tau F;W)=0\ \wedge\ \DiagZero\ \wedge\ \Ext^1(\mathcal{R}(C_\tau F|_W),k)=0\ \Bigr)\ \Longrightarrow\ \PH_1(C_\tau F|_W)=0.
\]
\emph{Run manifest.} The outcome, premises, and any violation are logged with keys
\noindent
\path{tests.T-ExtZero-implies-PHZero.{window, \tau, E1, mu, nu, Ext, PH, pass}} (Appendix~G).
Violations are tagged \path{counterexample.local_reverse}.
\end{definition}


\begin{remark}[Minimality and redundancy]
By Theorem~\ref{thm:local-reverse}, once \(E_1=0\) and \(\DiagZero\) hold, the conclusion \(\PH_1=0\) is forced; the explicit \(\Ext^1=0\) premise is recorded to align with audit trails and to surface any unexpected Ext anomalies under numerical or modeling noise.
\end{remark}

\subsection*{3.7. Naturality, stability, and windowed gate usage}
The edge identifications in (B2)–(B3) are natural in \(F\); \(\mathbf{T}_\tau\) is \(1\)-Lipschitz for \(d_{\mathrm{int}}\); thus \(\PH_1(C_\varepsilon(F))=0\) is metrically stable, and gate outcomes are invariant under functorial choices of \(C_\varepsilon\).
Quantitative primitives \(E_1\) and \(C_{1,r}\) provide monotone, stable diagnostics compatible with the gate.

\subsection*{3.8. Interaction with PF/BC, Mirror, and the \texorpdfstring{$\delta$}{delta}-ledger}
PF/BC transport is applied per \(t\), then collapsed; Mirror/Transfer comparisons are performed only after \(C_\tau\).
All non-commutations are externalized in the \(\delta\)-ledger; Ext checks are confined to \(D^{\mathrm b}(k\text{-mod})\).

\subsection*{3.9. Scope and limitations}
No claim is made that \(\Ext^1(\mathcal{R}(F),k)=0\Rightarrow \mathrm{PH}_1(F)=0\) globally.
Failure modes (including Type~IV/tower artifacts) are detected by \((\mu,\nu)\) (Appendix~D). Window-local equivalence requires definability, amplitude, and tail isomorphism (Theorem~\ref{thm:E1-local}).

\subsection*{3.10. Formalizability}
(B1)–(B3) and Theorem~\ref{thm:PH1-to-Ext1} are formalizable: (B2) via a two-term interface for \(\mathcal{R}\), (B3) via truncation and the long exact sequence.
\(E_1\), \(C_r\) are barcode-level primitives; their stability reduces to bottleneck stability (Appendix~H).
Windowed usage (B-Gate\(^{+}\)), MECE policy, and \(\delta\)-ledger appear as operational axioms in Appendix~F; proofs remain on the persistence layer and in \(D^{\mathrm b}(k\text{-mod})\).

\medskip
\noindent\textit{References and provenance.}
Serre localization, barcode abelianity, interleaving stability, and CNF/Ext–Hom (Chapter~2) are classical over fields.
The \(E_1\)-first policy is operational (Chapter~2, \S2.7; Chapter~11) and never replaces PH/Ext gates; it prioritizes a measurable determinant that is exact after collapse.



\section{Chapter 4: Failure Lattice, Local PH–Ext Equivalence, Čech–Ext Gluing, and the Tower-Sensitivity Invariant $\mu_{\mathrm{Collapse}}$}
\addcontentsline{toc}{section}{Failure Lattice, Local PH–Ext Equivalence, Čech–Ext Gluing, and the Tower-Sensitivity Invariant $\mu_{\mathrm{Collapse}}$}

\paragraph{Standing hypotheses and scope.}
We work in the constructible (finite-type) persistence range (Chapter~2, §2.1), adopt the bridging hypotheses \textup{(B1)–(B3)} from Chapter~3 with the minimal test family $\mathcal{Q}=\{k[0]\}$, and fix a $t$-exact realization $\mathcal{R}:\mathsf{FiltCh}(k)\to D^{\mathrm{b}}(k\text{-mod})$ of amplitude $\le1$.
All filtered (co)limit statements are asserted \emph{at the persistence layer} (Appendix~A).
Endpoints and infinite bars follow Chapter~2, Remark~\ref{rk:2-endpoints}.
Monotonicity for indicators applies only to \emph{deletion-type} updates; inclusion-type updates are \emph{stability-only} (Appendix~E).
Every claim is \emph{windowed} (Chapter~1, Def.~1.0; Chapter~2, §2.4), and all gate decisions are taken \emph{only} on the B-side after collapse (single layer $\mathbf{T}_\tau\mathbf{P}_i$).
We assert \emph{only} the one-way core bridge $\mathrm{PH}_1\Rightarrow \Ext^1$ in $D^{\mathrm{b}}(k\text{-mod})$ under \textup{(B1)–(B3)} (Chapter~3; Appendix~C).

\medskip
\noindent\emph{(B2) (edge identification, recall).}
There is a natural isomorphism $H^{-1}(\mathcal{R}(F))\cong \varinjlim_t H_1(F^tC_\bullet)$ and $\mathcal{R}(F)\in D^{[-1,0]}$; see Appendix~C.

\subsection*{4.1. Failure lattice and observable vs.\ invisible modes}
We organize collapse failures (cf.\ Chapter~1, §1.4):
\begin{itemize}[leftmargin=1.25em]
  \item \textbf{Type I (Topological):} $\mathrm{PH}_1(F)\neq 0$.
  \item \textbf{Type II (Categorical):} $\Ext^1(\mathcal{R}(F),k)\neq 0$ (tested against $\mathcal{Q}=\{k[0]\}$).
  \item \textbf{Type III (Functorial/[Spec]):} admissibility unstable under a prescribed operation (e.g.\ a given pullback/filtered colimit) at finite level.
  \item \textbf{Type IV (Invisible/tower-level):} all finite layers appear admissible while the limit is not; detected by the tower-sensitivity invariants below.
\end{itemize}
Types~I–II are \emph{observable}; Type~III is \emph{specification-level}; Type~IV is \emph{invisible} at finite layers and requires tower diagnostics.
We emphasize again: \emph{no} global equivalence $\mathrm{PH}_1\Leftrightarrow \Ext^1$ is claimed; only $\mathrm{PH}_1\Rightarrow \Ext^1$ under \textup{(B1)–(B3)} (Chapter~3; Appendix~C).

\begin{remark}[Specification-level failures and functorial calculus]
Type~III uses Chapter~2, §2.3: non-expansiveness, shift–commutation (Lemma~\ref{lem:Vshift}), and persistence-layer (co)limit/pullback compatibilities in Proposition~\ref{prop:stability}\,(4)(5). Filtered-level statements are \textbf{[Spec]} and used only up to f.q.i. (Appendix~B).
\end{remark}

\begin{remark}[Model towers]
Pure-kernel, pure-cokernel, and mixed toy towers, together with vanishing regimes under constructible filtered colimits, appear in Appendix~D (D.1–D.3). Counterexamples to the converse $\Ext^1{=}0\Rightarrow\mathrm{PH}_1{=}0$ are in D.4 (see also Appendix~C).
\end{remark}

\subsection*{4.2. The tower-sensitivity invariants $\mu_{\mathrm{Collapse}}$, $\nu_{\mathrm{Collapse}}$, and the Defect functor}
\emph{Generic fiber dimension.} As in Chapter~1, §1.4 and Appendix~D, Remark~\ref{D:rem:generic-dim}, for $M\in\Pers^{\mathrm{cons}}_k$ the \emph{generic fiber dimension} is $\mathrm{gdim}(M)=\lim_{t\to+\infty}\dim_k M(t)$; after $\mathbf{T}_\tau$, it equals the multiplicity of the infinite bar $I[0,\infty)$.

\medskip
\noindent\textbf{Comparison map.}
Fix $\tau>0$. Let $\{F_n\}_{n\in\bbN}$ be a directed system in $\mathsf{FiltCh}(k)$ with colimit $F_\infty$. For each degree $i$, define
\[
\phi_{i,\tau}:\ \varinjlim_{n}\ \mathbf{T}_\tau\!\big(\mathbf{P}_i(F_n)\big)\ \longrightarrow\ \mathbf{T}_\tau\!\big(\mathbf{P}_i(F_\infty)\big).
\]

\begin{definition}[Defect objects and tower-sensitivity invariants]\label{def:defect}
In $\Pers^{\mathrm{cons}}_k$ set
\[
\Defect_{i,\tau}^{\ker}:=\ker(\phi_{i,\tau}),\qquad \Defect_{i,\tau}^{\coker}:=\mathrm{coker}(\phi_{i,\tau}).
\]
Then
\[
\mu_{i,\tau}:=\mathrm{gdim}\big(\Defect_{i,\tau}^{\ker}\big),\quad \nu_{i,\tau}:=\mathrm{gdim}\big(\Defect_{i,\tau}^{\coker}\big),\quad
\mu_{\mathrm{Collapse}}:=\sum_i\mu_{i,\tau},\ \nu_{\mathrm{Collapse}}:=\sum_i\nu_{i,\tau}.
\]
Finite homological range and constructibility give $\mu_{\mathrm{Collapse}},\nu_{\mathrm{Collapse}}<\infty$ on bounded $\tau$-windows.
\end{definition}

\begin{proposition}[Generic dimension equals infinite-bar multiplicity]\label{prop:generic-dimension-barcode}
For any morphism $\psi:M\to N$ in $\Pers^{\mathrm{cons}}_k$, $\mathrm{gdim}\ker(\psi)$ (resp.\ $\mathrm{gdim}\mathrm{coker}(\psi)$) equals the multiplicity of $I[0,\infty)$ in the barcode of $\ker(\psi)$ (resp.\ $\mathrm{coker}(\psi)$). The same holds after $\mathbf{T}_\tau$.
\end{proposition}

\begin{remark}[Functoriality, invariance, and calculus]
The maps $\phi_{i,\tau}$ are natural in the tower, independent of filtered representatives, and invariant under cofinal reindexing (Appendix~J). Hence $\Defect_{i,\tau}^{\ker/\coker}$ and $(\mu_{i,\tau},\nu_{i,\tau})$ are invariant under f.q.i. Subadditivity under composition, additivity under finite sums, and cofinal invariance are collected in Appendix~J.
\end{remark}

\begin{definition}[V-distance and control]\label{def:V-metric}
A \emph{V-distance} on towers assigns to each pair of towers a value in $[0,\infty]$ and satisfies: (V1) compatibility with cropping and $\mathbf{T}_\tau$; (V2) non-expansiveness under filtered colimits and cofinal reindexing; (V3) triangle inequality for composable controlled morphisms; (V4) stability under finite sums. We write $V\big((F_\bullet,\phi_{i,\tau}),(\tilde F_\bullet,\tilde\phi_{i,\tau})\big)\le \varepsilon$ to mean the towers and their comparison maps are $\varepsilon$-controlled on the window.
\end{definition}

\begin{proposition}[V-subadditivity/additivity/cofinal invariance and stability]\label{prop:V-subadd}
Fix $i$ and $\tau>0$.
\begin{enumerate}[leftmargin=1.25em]
  \item \emph{Subadditivity.} For $A\xrightarrow{\phi}B\xrightarrow{\psi}C$ (arising from towers via $\mathbf{T}_\tau\mathbf{P}_i$),
  \[
  \mu(\psi\!\circ\!\phi)\le \mu(\phi)+\mu(\psi),\qquad \nu(\psi\!\circ\!\phi)\le \nu(\phi)+\nu(\psi).
  \]
  \item \emph{Additivity on finite sums.} $\mu(\phi\oplus\phi')=\mu(\phi)+\mu(\phi')$ and $\nu(\phi\oplus\phi')=\nu(\phi)+\nu(\phi')$.
  \item \emph{Cofinal invariance.} Cofinal reindexing preserves $\mu_{i,\tau},\nu_{i,\tau}$.
  \item \emph{V-stability on stable bands.} If $B$ is a stable band on a window (Def.~\ref{def:stable-band}) and $V\le\varepsilon$ uniformly on $\tau\in B$ with band margin $>\varepsilon$, then $\mu_{i,\tau},\nu_{i,\tau}$ agree for the two towers on $B$. In particular, $\mu_{i,\tau},\nu_{i,\tau}$ are upper semicontinuous in $V$.
\end{enumerate}
\end{proposition}

\begin{proposition}[Deletion-type monotonicity]\label{prop:deletion-mono}
Along any pipeline consisting only of deletion-type updates and $\varepsilon$-continuations, $(\mu_{i,\tau},\nu_{i,\tau})$ after collapse are nonincreasing in deletion steps and $1$-Lipschitz in $\varepsilon$-continuations (on stable bands). If $(\mu_{i,\tau},\nu_{i,\tau})=(0,0)$ at some stage on a stable band, it remains $(0,0)$ under further deletion-type updates within the same band.
\end{proposition}

\subsection*{4.3. Local PH–Ext equivalence on saturation windows \textnormal{\textbf{[Spec]}}}
\begin{theorem}[\textnormal{\textbf{[Spec]}} Local PH--Ext equivalence on CCC saturation windows]\label{thm:ch4-local-equiv}
Let $F\in\mathsf{FiltCh}(k)$, $W=[u,u')$ right-open, and $\tau>0$. Assume:
\begin{enumerate}[leftmargin=1.25em]
  \item \emph{(Amplitude)} $\mathcal{R}(C_\tau F)\in D^{[-1,0]}(k\text{-mod})$.
  \item \emph{(Saturation/gap)} $W$ is a saturation window at $\tau$ (Chapter~2, Def.~2.6).
  \item \emph{(Reverse authorization)} the CCC and Spectral-Gap Condition of Chapter~16 hold on $W$.
  \item \emph{(Tail isomorphism)} $\phi_{1,\tau}$ is an isomorphism on $W$, i.e.\ $\DiagZero$ in degree $1$.
\end{enumerate}
Then, on $W$ at threshold $\tau$,
\[
\mathrm{PH}_1\!\big(C_\tau F\big)=0\quad\Longleftrightarrow\quad \Ext^1\!\big(\mathcal{R}(C_\tau F),k\big)=0.
\]
\end{theorem}

\begin{remark}[Local vs.\ global]
The equivalence is \textnormal{\emph{[Spec]}} and \emph{window-local}. Globally, we keep the one-way core policy $\mathrm{PH}_1\Rightarrow \Ext^1$ (Chapter~3).
\end{remark}


\paragraph{Definable covers and finite Čech depth.}
\begin{lemma}[Definable Čech finiteness]\label{lem:definable-Cech-finite}
Let $\{X_\alpha\}$ be a definably locally finite cover in a fixed o-minimal expansion with finite overlap multiplicity $\le r$. Then all $(r{+}1)$–fold intersections are empty, the Čech nerve $N(\mathcal{U})$ has dimension $\le r{-}1$, and the Čech complex stops in degree $\le r{-}1$. Consequently, Overlap Glue (after collapse) reduces to finitely many overlap checks of order $\le r{-}1$.
\end{lemma}

\subsection*{4.4. Čech–Ext\texorpdfstring{$^1$}{1} gluing and the Overlap Gate \textnormal{\textbf{[Spec]}}}
\begin{definition}[Čech nerve and local data after collapse]\label{def:cech}
Given a cover $\{X_\alpha\}$ and right-open windows $\{W_\alpha\}$, set for $i$ and $\tau>0$
\[
\mathcal{B}_{\alpha,i}\ :=\ \mathbf{T}_\tau\,\Crop_{W_\alpha}\big(\mathbf{P}_i(F|_{X_\alpha})\big)\in \Pers^{\mathrm{cons}}_k.
\]
On overlaps use restrictions $\mathcal{B}_{\alpha_0\cdots\alpha_p,i}$.
\end{definition}

\begin{definition}[Čech–Ext\textsuperscript{1}–acyclicity (after collapse)]\label{def:cech-acyclic}
The collapsed local data are \emph{Čech–Ext\textsuperscript{1}–acyclic} in degree $1$ if $\Ext^1(\mathcal{R}(C_\tau(F|_{X_{\alpha_0\cdots\alpha_p}})),k)=0$ for all nonempty overlaps and the Čech differentials land in zero $\Ext^1$ on each overlap (equivalently, the $\Ext^1$–row of the Čech $E_1$–page vanishes).
\end{definition}

\begin{theorem}[Overlap Gate with Čech–Ext\textsuperscript{1}: local-to-global gluing]\textnormal{\textbf{[Spec]}}\label{thm:cech-glue}
Fix degree $i=1$, a right-open windowing, and $\tau>0$. Assume:
\begin{enumerate}[leftmargin=1.25em]
  \item \emph{(Local gates)} On each $X_\alpha\times W_\alpha$, B–Gate$^{+}$ passes: $\mathrm{PH}_1(C_\tau F|_{X_\alpha})=0$, $\Ext^1(\mathcal{R}(C_\tau F|_{X_\alpha}),k)=0$, $\DiagZero$, with safety margin $\mathrm{gap}_\tau>\Sigma\delta_\alpha$.
  \item \emph{(Overlap Gate)} For each $(\alpha,\beta)$ with nonempty overlap, the collapsed restrictions agree up to the recorded budget; the safety margin dominates the overlap budget; $\DiagZero$ on overlaps. \emph{All discrepancies are recorded as $\delta_{\alpha\beta}^{\mathrm{alg}}$ in the ledger.}
  \item \emph{(Čech–Ext\textsuperscript{1}–acyclicity)} As in Definition~\ref{def:cech-acyclic}.
\end{enumerate}
Then B–Gate$^{+}$ passes globally on $\bigcup_\alpha X_\alpha\times W_\alpha$ at threshold $\tau$:
\[
\mathrm{PH}_1(C_\tau F)=0,\qquad \Ext^1(\mathcal{R}(C_\tau F),k)=0,\qquad \DiagZero,
\]
with a global safety margin equal to the minimum of local margins minus recorded overlap budgets.
\end{theorem}

\begin{remark}[Practical check and logging]
In practice one checks $\Ext^1(\mathcal{R}(C_\tau F|_{X_\alpha}),k)=0$ and the Mayer–Vietoris row ($p=1$), records $\delta^{\mathrm{alg}}$ on overlaps in the $\delta$-ledger, and audits margins against \texttt{run.yaml} (Appendix~G). All checks are B-side (after collapse).
\end{remark}

\subsection*{4.4 bis. Convergence Manager on definable covers (quantale summability)}
\begin{theorem}[Countable DP-cover $\Rightarrow$ global Overlap Glue under $\Sigma\delta<\infty$]\label{thm:dp-sum-ch4}
Let $\{W_n\}_{n\ge1}$ be a countable family of right-open \emph{Denef–Pas definable} windows covering a bounded range, with Čech depth $\le K$. Let $\mathsf{V}$ be the fixed commutative quantale for budgets (Chapter~1). If
\[
\sum_{n=1}^{\infty}\ \delta(W_n)\ <\ \infty\qquad\text{in }\mathsf{V},
\]
then, for every point, the total overlap error is bounded by $K\cdot\sum_n\delta(W_n)$ in $\mathsf{V}$, and the Overlap Glue from Theorem~\ref{thm:cech-glue} holds globally. In particular, B–Gate$^{+}$ certificates paste to a global certificate on $\bigcup_n W_n$ whenever per-window margins dominate the local budgets and the above sum is finite.
\end{theorem}

\begin{proof}[Proof sketch]
Denef--Pas definability together with the assumed \v{C}ech depth bound $K$ on the declared cover controls the overlap levels on bounded subranges; quantale subadditivity and right-open MECE refinement control the cumulative budget on each $p$–fold overlap by $\sum_n\delta(W_n)$. Summing over $p\le K{-}1$ gives the bound $K\cdot\sum_n\delta(W_n)$. With margins dominating local budgets, Overlap Gate constraints are satisfied at each overlap level, hence gluing holds globally.
\end{proof}

\subsection*{4.5. Type IV: finite admissibility need not pass to the limit}
\begin{proposition}[Type IV: finite-level admissibility may fail at the limit]\label{prop:type4}
There exists a tower $\{F_n\}$ and $\tau>0$ such that
\[
\forall n:\ \ \mathrm{PH}_1\!\big(C_\tau(F_n)\big)=0\ \text{ and }\ \Ext^1\!\big(\mathcal{R}(C_\tau(F_n)),k\big)=0,\qquad
\mathrm{but}\quad \mathrm{PH}_1\!\big(C_\tau(F_\infty)\big)\neq 0,
\]
hence $\Ext^1\!\big(\mathcal{R}(C_\tau(F_\infty)),k\big)\neq 0$. One can arrange $\mu_{\mathrm{Collapse}}=0$ and $\nu_{\mathrm{Collapse}}>0$ (pure cokernel type).
\end{proposition}

\begin{figure}[t]
\centering
\begin{tikzcd}[row sep=1.0em, column sep=2.0em]
I[0,\tau-\tfrac{1}{1}) \arrow[r, hook] \arrow[dr] &
I[0,\tau-\tfrac{1}{2}) \arrow[r, hook] 
& \cdots \arrow[r, hook] \arrow[dr] &
I[0,\tau-\tfrac{1}{n}) \arrow[r, hook] 
& \cdots \arrow[r, hook] \arrow[dr] &
I[0,\tau) \arrow[r, hook] \arrow[dr] &
I[0,\infty) \\\n& \mathbf{T}_\tau(\cdot)=0 & & \mathbf{T}_\tau(\cdot)=0 & &
\mathbf{T}_\tau(\cdot)=0 & \mathbf{T}_\tau(\cdot)=I[0,\infty)
\end{tikzcd}
\caption{Type~IV intuition (after $\mathbf{T}_\tau$): all finite layers vanish, while the apex produces an infinite bar.}
\label{fig:typeIV-intuition}
\end{figure}

\subsection*{4.6. A natural refinement-limit example (pure cokernel type)}
\begin{example}[Resolution refinement producing a limit infinite bar]
Fix $\tau>0$. Let degree-$1$ persistence at level $n$ have one bar $[0,\tau-\delta_n)$ with $\delta_n\downarrow 0$. Then $\mathbf{T}_\tau(\mathbf{P}_1(F_n))=0$ for all $n$, while $\mathbf{T}_\tau(\mathbf{P}_1(F_\infty))\cong I[0,\infty)$. This realizes a \emph{pure cokernel} Type~IV failure ($\mu_{\mathrm{Collapse}}=0$, $\nu_{\mathrm{Collapse}}>0$).
\end{example}

\begin{remark}[When invisible failure is excluded]
Under the hypotheses of Proposition~\ref{prop:stability}\,(4) (Chapter~2) and the tower conditions of Proposition~\ref{J:prop:diagzero} (Appendix~J), each $\phi_{i,\tau}$ is an isomorphism, so no Type~IV occurs and finite-level admissibility propagates to the limit.
\end{remark}

\subsection*{4.7. Restart lemma, summability, and pasting of windowed certificates}
\begin{definition}[Per-window safety margin and pipeline budget]\label{def:window-budget}
For a MECE partition $\{[u_k,u_{k+1})\}_k$, threshold $\tau_k>0$, and degree $i$, set
\[
\Sigma\delta_k(i)\ :=\ \sum_{j\in J_k}\!\bigl(\delta^{\mathrm{alg}}_{j}(i,\tau_k)+\delta^{\mathrm{disc}}_{j}(i,\tau_k)+\delta^{\mathrm{meas}}_{j}(i,\tau_k)\bigr),
\]
where $J_k$ indexes the A-side steps before the B-side gate on $W_k$. The \emph{safety margin} $\mathrm{gap}_{\tau_k}>0$ is the admissible slack for the gate on $W_k$ (Chapter~1).
\end{definition}

\begin{lemma}[Restart lemma (window-to-window inheritance)]\label{lem:restart}
If B–Gate$^{+}$ passes on $W_k$ with $\mathrm{gap}_{\tau_k}>\Sigma\delta_k(i)$ and $W_{k+1}$ is reached via deletion-type steps and/or $\varepsilon$-continuations followed by $C_{\tau_{k+1}}$, then there exists $\kappa\in(0,1]$ (depending only on the admissible step class and the $\tau$-adaptation policy) with
\[
\mathrm{gap}_{\tau_{k+1}}\ \ge\ \kappa\ \bigl(\mathrm{gap}_{\tau_k}-\Sigma\delta_k(i)\bigr).
\]
Thus positive margin propagates provided the new budget $\Sigma\delta_{k+1}(i)$ is small enough.
\end{lemma}

\begin{definition}[Summability policy]\label{def:summability}
A run satisfies \emph{summability} if
\[
\sum_k \Sigma\delta_k(i)\ <\ \infty
\]
for the monitored degrees $i$ on a MECE partition. A sufficient design pattern is geometric damping of step sizes and/or continuation strengths, recorded in \texttt{run.yaml} (Appendix~G).
\end{definition}

\begin{theorem}[Restart–Summability (Convergence Manager, v17)]\label{thm:pasting}
If B–Gate$^{+}$ passes on each $W_k$ with $\mathrm{gap}_{\tau_k}>\Sigma\delta_k(i)$, the summability policy holds (Def.~\ref{def:summability}), and Lemma~\ref{lem:restart} applies at each transition, then windowed certificates paste to a global certificate on $\bigcup_k [u_k,u_{k+1})$ for the monitored degrees $i$.
\end{theorem}

\begin{remark}[Manifest integration]
Threshold adaptation, ledger aggregation law (quantale), Čech depth bound, and restart/summability parameters \emph{must} be recorded in \texttt{run.yaml} (Appendix~G). Pipelines read these values to audit Overlap Gate and Convergence Manager decisions.
\end{remark}

\subsection*{4.8. Stable bands and $\tau$-sweeps}
\begin{definition}[Stable band]\label{def:stable-band}
For a fixed window $W$ and degree $i$, a \emph{stable band} $B\subset (0,\infty)$ is a contiguous range such that for all $\tau\in B$ the comparison maps $\phi_{i,\tau}$ are isomorphisms; hence $(\mu_{i,\tau},\nu_{i,\tau})=(0,0)$. A \emph{$\tau$-sweep} is a discrete set $\{\tau_\ell\}$ used to probe $(\mu_{i,\tau_\ell},\nu_{i,\tau_\ell})$; a band is declared stable when a consecutive subarray reports $(0,0)$ and persists under refinement.
\end{definition}

\begin{proposition}[Sparse sweep on gap-protected bands]\label{prop:sparse-sweep}
Let $B\subset (0,\infty)$ be compact and \emph{gap-protected}: there exists $\eta>0$ such that no bar endpoint lies within distance $\eta$ of any $\tau\in B$ (on $W$). If a sweep $\{\tau_\ell\}\subset B$ with mesh $<\eta/3$ yields $(\mu_{i,\tau_\ell},\nu_{i,\tau_\ell})=(0,0)$ for all $\ell$, then $(\mu_{i,\tau},\nu_{i,\tau})=(0,0)$ for all $\tau\in B$; hence $B$ is stable.
\end{proposition}

\begin{theorem}[Stability-band detection via drift threshold]\label{thm:band-detect}
Fix $W$, degree $i$, and constants $\eta>0$ (gap) and $\Delta>0$ (drift threshold). Suppose a sweep $\{\tau_\ell\}$ in an open interval $I$ satisfies:
\begin{enumerate}[leftmargin=1.25em]
  \item $(\mu_{i,\tau_\ell},\nu_{i,\tau_\ell})=(0,0)$ for all sampled $\tau_\ell$;
  \item for every consecutive pair $\tau_\ell<\tau_{\ell+1}$ and any admissible continuation between the two gates, the interleaving drift of $\mathbf{T}_{\tau_\ell}\mathbf{P}_i$ to $\mathbf{T}_{\tau_{\ell+1}}\mathbf{P}_i$ is $<\Delta$ (measured in $d_{\mathrm{int}}$);
  \item $\min\{\tau_{\ell+1}-\tau_\ell\}\ge 3\eta$ and the band is $\eta$-gap-protected.
\end{enumerate}
Then there exists a nonempty open subinterval $B\subset I$ on which $\phi_{i,\tau}$ is an isomorphism and $(\mu_{i,\tau},\nu_{i,\tau})=(0,0)$ for all $\tau\in B$.
\end{theorem}

\begin{proof}[Proof sketch]
Gap protection implies local constancy of $\mathbf{T}_\tau$ on subintervals; small drift ensures no creation of $I[0,\infty)$ in kernels/cokernels between samples. Compactness of $I$ and mesh $\ge3\eta$ yield an open subinterval $B$ where $\mathbf{T}_\tau$ and the comparison maps are constant, hence $(\mu_{i,\tau},\nu_{i,\tau})=(0,0)$.
\end{proof}

\subsection*{4.9. Summary}
The failure lattice separates observable (Types~I–II), specification-level (Type~III), and invisible tower effects (Type~IV). The Defect objects $\Defect_{i,\tau}^{\ker/\coker}$ and the invariants $(\mu_{\mathrm{Collapse}},\nu_{\mathrm{Collapse}})$ provide principled tower diagnostics with subadditivity, additivity, and cofinal invariance. The \emph{only} core bridge is $\mathrm{PH}_1\Rightarrow \Ext^1$ in $D^{\mathrm{b}}(k\text{-mod})$; nevertheless, on \emph{saturation windows} with tail isomorphism we obtain a window-local equivalence $\mathrm{PH}_1(C_\tau F)=0 \Leftrightarrow \Ext^1(\mathcal{R}(C_\tau F),k)=0$ (Theorem~\ref{thm:ch4-local-equiv}). For gluing, the Overlap Gate with Čech–Ext$^1$–acyclicity gives local-to-global propagation (Theorem~\ref{thm:cech-glue}); the Convergence Manager (Theorems~\ref{thm:dp-sum-ch4} and \ref{thm:pasting}) ensures pasting under quantale-summable budgets on definable covers. Stable bands (Definition~\ref{def:stable-band}), sparse sweeps (Proposition~\ref{prop:sparse-sweep}), and the drift-threshold detector (Theorem~\ref{thm:band-detect}) guide $\tau$-selection for tower audits. Our $\mu_{\mathrm{Collapse}}$ is a persistence–theoretic audit invariant (Remark~\ref{rem:Iwasawa-mu-vs-collapse-mu}); it is distinct from classical Iwasawa $\mu$.

\begin{remark}[Iwasawa $\mu$ vs.\ $\mu_{\mathrm{Collapse}}$]\label{rem:Iwasawa-mu-vs-collapse-mu}
Both quantify \emph{defect accumulation along towers}, but in different categories and with different laws: $\mu_{\mathrm{Collapse}}$ depends on degree and threshold, is window-local, and is additive on sums/subadditive under composition; Iwasawa $\mu$ is prime- and tower-global via characteristic ideals. No direct implication is intended.
\end{remark}

\subsection*{4.10. Mandatory tests (operational)}
The following tests are part of the core suite; all are evaluated \emph{after collapse} and logged with budgets/margins.
\begin{description}[leftmargin=1.25em]
\item[\textbf{T-Lipschitz-AfterCollapse}.] Verify that for each monitored degree $i$ and adjacent pipeline steps, $\mathbf{T}_\tau$ preserves interleavings: $d_{\mathrm{int}}(\mathbf{T}_\tau\mathbf{P}_i(F),\mathbf{T}_\tau\mathbf{P}_i(G))\le d_{\mathrm{int}}(\mathbf{P}_i(F),\mathbf{P}_i(G))$ (Chapter~2, Lemma~\ref{lem:Vshift}).
\item[\textbf{T-Countable-Cover}.] On a Denef–Pas definable right-open cover, record the Čech depth bound $K$ (Lemma~\ref{lem:definable-Cech-finite}) and verify finiteness of overlap indices used by the Overlap Gate.
\item[\textbf{T-Delta-Sum-Converges}.] Check quantale-summability $\sum_n\delta(W_n)<\infty$ for the run’s cover and thresholds; then apply Theorem~\ref{thm:dp-sum-ch4} to assert global Overlap Glue under recorded margins.
\end{description}
All three tests are parameterized by \texttt{run.yaml} keys \texttt{quantale/*}, \texttt{diagnostics/*}, \texttt{gates/*}, \texttt{cover/*}, and \texttt{delta\_budget/*} (Appendix~G).



% NOTE (out-of-band): Some previously uploaded files appear to have expired. If you want me to load them again, please re-upload them. — This line is a comment and harmless for pdflatex.

\section{Chapter 5: Functoriality, Set-Theoretic Coherence, and Formalization Specifications (Proof/Spec)}
\addcontentsline{toc}{section}{Functoriality, Set-Theoretic Coherence, and Formalization Specifications (Proof/Spec)}

All adjunction and (co)limit statements in this chapter are made in the \emph{implementable range} and inside
\(\mathrm{Ho}(\mathsf{FiltCh}(k))\), \emph{up to filtered quasi\hyp isomorphism} (Appendix~B). Equalities are asserted \emph{at the persistence layer}. We retain the standing conventions of Chapters~1–4: constructible range, field coefficients, \(t\)\hyp exact realization, and the after\hyp truncation policy. Endpoint conventions and infinite bars are as in Chapter~2, Remark~\ref{rk:2-endpoints}. \emph{Monotonicity claims apply only to deletion\hyp type updates; inclusion\hyp type updates are stability\hyp only} (Appendix~E). All statements are \emph{windowed} and gates are evaluated \emph{after collapse} (B\hyp side single layer).

\subsection*{5.1. Exactness, (Co)Limit Behavior, and a Right\hyp Adjoint Collapse in the Implementable Range (up to f.q.i.)}

Let \(k\) be a field. Recall: \(\mathsf{FiltCh}(k)\) is the category of finite\hyp type (constructible) filtered chain complexes; \(\mathbf{P}_i\) is degreewise persistence; \(\mathbf{T}_\tau\) is the exact truncation deleting all bars of length \(\le\tau\) (Chapter~2, §2.2); \(C_\tau\) is any filtered lift of \(\mathbf{T}_\tau\) (Chapter~2, §§2.2–2.3; always \emph{up to f.q.i.}); and \(\mathcal{R}:\mathsf{FiltCh}(k)\to D^{\mathrm{b}}(k\text{-mod})\) is \(t\)\hyp exact. We also keep the minimal test family \(\mathcal{Q}=\{k[0]\}\).

\paragraph{Persistence\hyp level (reflective) adjunction.}
Let \(\mathsf{Pers}^{\mathrm{cons}}_k\) be the abelian category of constructible \(k\)\hyp persistence modules and \(\mathsf{Pers}^{\mathrm{cons}}_{k,\tau\text{-tf}}\subset \mathsf{Pers}^{\mathrm{cons}}_k\) the full subcategory of \(\tau\)\hyp torsion\hyp free objects (no composition factors of length \(\le\tau\)). As in Chapter~2, §§2.2–2.3:
\begin{itemize}
  \item \(\mathsf{E}_\tau\subset \mathsf{Pers}^{\mathrm{cons}}_k\) (generated by interval modules of length \(\le\tau\)) is hereditary Serre (localizing).
  \item The reflector
  \[
  \mathbf{T}_\tau:\ \mathsf{Pers}^{\mathrm{cons}}_k\longrightarrow \mathsf{Pers}^{\mathrm{cons}}_{k,\tau\text{-tf}}
  \]
  is exact and exhibits a \emph{reflective} adjunction \(\mathbf{T}_\tau\dashv \iota_\tau\) with the inclusion
  \(\iota_\tau:\mathsf{Pers}^{\mathrm{cons}}_{k,\tau\text{-tf}}\hookrightarrow \mathsf{Pers}^{\mathrm{cons}}_k\).
  Consequently, \(\mathbf{T}_\tau\) preserves finite limits and colimits and is \(1\)\hyp Lipschitz for \(d_{\mathrm{int}}\)
  (Lemma~\ref{lem:Vshift}, Proposition~\ref{prop:stability}(1),(3)).
\end{itemize}

\paragraph{Filtered-complex level (operational coreflection; implementable range, up to f.q.i.).}
Define the full subcategory
\[
\mathsf{S}_\tau\ :=\ \Big\{\,F\in \mathsf{FiltCh}(k)\ \Big|\ \forall i,\ \mathbf{P}_i(F)\ \text{is \(\tau\)-torsion-free}\,\Big\},
\]
and let \(\mathsf{S}_\tau^{\mathrm{h}}\) be its image in \(\mathrm{Ho}(\mathsf{FiltCh}(k))\).

For later comparison, define the “trivial-at-\(\tau\)” class (after collapse) as
\[
\mathsf{Triv}_\tau\ :=\ \Big\{\,G\in \mathsf{S}_\tau\ \Big|\ \mathrm{PH}_1(G)=0\,\Big\},\qquad \mathsf{Triv}_\tau^{\mathrm{h}}\subset \mathsf{S}_\tau^{\mathrm{h}}.
\]
\begin{remark}
Under \textup{(B1)--(B3)}, \(\mathrm{PH}_1(G)=0\Rightarrow \Ext^1(\mathcal{R}(G),k)=0\) (Theorem~3.2), so the Ext-check is redundant on \(\mathsf{Triv}_\tau\); it may still be logged for audit.
\end{remark}


\begin{proposition}[\textnormal{\textbf{[Spec]}} Operational collapse as a right adjoint (implementable range; up to f.q.i.)]\label{prop:operational-coreflection}
Assume \textup{(B1)–(B3)} (Chapter~3) and the lifting–coherence hypothesis (Appendix~B).
Then there exists a functor
\[
\mathsf{C}_\tau^{\mathrm{comb}}:\ \mathrm{Ho}(\mathsf{FiltCh}(k))\longrightarrow \mathsf{S}_\tau^{\mathrm{h}}
\]
and a natural transformation \(\eta:\mathrm{Id}\Rightarrow \iota\,\mathsf{C}_\tau^{\mathrm{comb}}\) (with \(\iota:\mathsf{S}_\tau^{\mathrm{h}}\hookrightarrow \mathrm{Ho}(\mathsf{FiltCh}(k))\) the inclusion) such that, \emph{in this regime},
\begin{enumerate}
  \item \textup{(Adjunction)} \(\mathsf{C}_\tau^{\mathrm{comb}}\) is right adjoint to \(\iota\):
  \[
  \mathrm{Hom}\!\big(\iota(G),F\big)\ \cong\ \mathrm{Hom}\!\big(G,\mathsf{C}_\tau^{\mathrm{comb}}(F)\big)\qquad(G\in\mathsf{S}_\tau^{\mathrm{h}}).
  \]
  \item \textup{(Compatibility; persistence layer)} For each \(i\), \(\mathbf{P}_i\!\big(\mathsf{C}_\tau^{\mathrm{comb}}(F)\big)\cong \mathbf{T}_\tau\!\big(\mathbf{P}_i(F)\big)\) in \(\mathsf{Pers}^{\mathrm{cons}}_k\).
  \item \textup{(Compatibility; realization layer)} \(\mathcal{R}\!\big(\mathsf{C}_\tau^{\mathrm{comb}}(F)\big)\cong \tau_{\ge 0}\,\mathcal{R}(F)\) in \(D^{\mathrm{b}}(k\text{-mod})\).
  \item \textup{(Soundness at the persistence layer)} \(\mathsf{C}_\tau^{\mathrm{comb}}(F)\in \mathsf{S}_\tau^{\mathrm{h}}\) for all \(F\), and
  \[
  \mathsf{C}_\tau^{\mathrm{comb}}(F)\in \mathsf{Triv}_\tau^{\mathrm{h}}
  \quad\Longleftrightarrow\quad
  \mathbf{T}_\tau\!\big(\mathbf{P}_1(F)\big)=0
  \quad(\text{equivalently } \mathrm{PH}_1(C_\tau(F))=0).
  \]
\end{enumerate}
\emph{All equalities above are asserted at the persistence layer, and all filtered\hyp complex statements are in \(\mathrm{Ho}(\mathsf{FiltCh}(k))\) up to f.q.i.\ only.}
\end{proposition}

\begin{declaration}[AWFS for collapse (operational, up to f.q.i.)]\label{dec:awfs}
Fix \(\tau>0\). There exist endofunctors
\[
L_\tau,\ R_\tau:\ \mathrm{Ho}(\mathsf{FiltCh}(k))\longrightarrow \mathrm{Ho}(\mathsf{FiltCh}(k))
\]
together with structure maps
\[
\varepsilon:\ L_\tau\Rightarrow \mathrm{Id},\quad \delta:\ L_\tau\Rightarrow L_\tau L_\tau,\qquad\eta:\ \mathrm{Id}\Rightarrow R_\tau,\quad \mu:\ R_\tau R_\tau\Rightarrow R_\tau,
\]
and a natural \(2\)\hyp cell (distributive law)
\[
\lambda:\ L_\tau R_\tau\ \Rightarrow\ R_\tau L_\tau
\]
such that:
\begin{enumerate}
  \item \(R_\tau=\iota\circ \mathsf{C}_\tau^{\mathrm{comb}}\) is an idempotent comonad up to f.q.i.\ (Proposition~\ref{prop:comonad}); dually, \(L_\tau\) is an idempotent monad up to f.q.i.\ modeling admissible \emph{pre\hyp processing} (e.g.\ normalization, calibration) that is \(1\)\hyp Lipschitz at persistence and preserves constructibility.
  \item \((L_\tau,R_\tau,\lambda)\) forms an \emph{operational algebraic weak factorization system} (AWFS) up to f.q.i.: on the pipeline class of maps (Appendix~E), every morphism functorially factors as an \(L_\tau\)\hyp step followed by an \(R_\tau\)\hyp step; the AWFS triangles and coherence hold up to f.q.i.\ and are quantitatively controlled in the \(\delta\)\hyp ledger (Specification~\ref{spec:delta-ledger}).
\end{enumerate}
At the persistence layer, the left/right structures are strict: \(L_\tau\) is exact and \(1\)\hyp Lipschitz; \(R_\tau\) realizes \(\mathbf{M}_\tau:=\iota_\tau\circ \mathbf{T}_\tau\).
\end{declaration}

\begin{corollary}[Limit/(co)limit behavior of \(\mathsf{C}_\tau^{\mathrm{comb}}\) (persistence layer)]
Under Proposition~\ref{prop:stability}(4),(5), \(\mathsf{C}_\tau^{\mathrm{comb}}\) preserves finite limits
(in particular, finite pullbacks) \emph{up to f.q.i.} at the filtered\hyp complex level; at the
\emph{persistence layer} one has, for every degree \(i\),
\[
\mathbf{P}_i\!\big(\mathsf{C}_\tau^{\mathrm{comb}}(\varinjlim_\Lambda F_\lambda)\big)\ \cong\ \varinjlim_\Lambda\, \mathbf{P}_i\!\big(\mathsf{C}_\tau^{\mathrm{comb}}(F_\lambda)\big).
\]
Hence \(\mathsf{C}_\tau^{\mathrm{comb}}\) is \(1\)\hyp Lipschitz at the persistence layer and inherits exactness via \(\mathbf{T}_\tau\) (Chapter~2, §2.3). All equalities are stated at the persistence layer; no additional metric statement is made in \(\mathrm{Ho}(\mathsf{FiltCh}(k))\) beyond up to f.q.i.\ compatibility.
\end{corollary}

\begin{remark}[Scope of colimit claims]
Since \(\tau_{\ge 0}\) is a right adjoint, it need not commute with filtered colimits; all colimit statements are therefore restricted to the \emph{persistence layer} (via \(\mathbf{T}_\tau\) and \(\mathbf{P}_i\)).
\end{remark}

\subsection*{5.1 bis. Idempotent monad/comonad and AWFS triangles: strictness at persistence, up to f.q.i.\ on \(\mathrm{Ho}\)}

\begin{proposition}[Idempotent monad at the persistence layer]\label{prop:monad}
Let \(\mathbf{M}_\tau:=\iota_\tau\circ \mathbf{T}_\tau:\mathsf{Pers}^{\mathrm{cons}}_k\to \mathsf{Pers}^{\mathrm{cons}}_k\). With unit \(\eta:\mathrm{Id}\Rightarrow \mathbf{M}_\tau\) and multiplication induced by the counit on \(\mathsf{Pers}^{\mathrm{cons}}_{k,\tau\text{-tf}}\), \((\mathbf{M}_\tau,\eta,\mu)\) is an \emph{idempotent monad}. It is exact and \(1\)\hyp Lipschitz (Appendix~A; see also Appendix~K).
\end{proposition}

\begin{proposition}[Idempotent comonad on Ho up to f.q.i.]\label{prop:comonad}
Let \(\mathbf{G}_{\tau}:=\iota\circ \mathsf{C}_\tau^{\mathrm{comb}}:\mathrm{Ho}(\mathsf{FiltCh}(k))\to \mathrm{Ho}(\mathsf{FiltCh}(k))\), 
with counit \(\varepsilon:\mathbf{G}_{\tau}\Rightarrow \mathrm{Id}\) and comultiplication \(\delta\) induced by the unit of the adjunction in Proposition~\ref{prop:operational-coreflection}. 
Then \((\mathbf{G}_\tau,\varepsilon,\delta)\) is an idempotent comonad in \(\mathrm{Ho}\) up to f.q.i.; moreover \(\mathbf{P}_i(\mathbf{G}_\tau F)\cong \mathbf{M}_\tau(\mathbf{P}_i F)\) naturally in \(i,F\).
\end{proposition}

\begin{theorem}[AWFS triangles and order control]\label{thm:awfs-triangle}
In the setting of Declaration~\ref{dec:awfs}, the following hold:
\begin{enumerate}
  \item \textup{(Idempotence)} \(C_\tau\circ C_\tau\ \simeq\ C_\tau\) in \(\mathrm{Ho}\) up to f.q.i., and \(\mathbf{T}_\tau\circ \mathbf{T}_\tau=\mathbf{T}_\tau\) strictly at persistence.
  \item \textup{(Distributive \(2\)\hyp cell)} There is a natural \(2\)\hyp cell \(\lambda:L_\tau R_\tau\Rightarrow R_\tau L_\tau\) whose persistence\hyp level image is an isomorphism; any non\hyp commutation on \(\mathrm{Ho}\) is bounded by \(\delta^{\mathrm{alg}}(\tau)\) and recorded in the \(\delta\)\hyp ledger (Definition~\ref{def:delta-2cell}, Specification~\ref{spec:delta-ledger}).
  \item \textup{(Triangle identities)} The AWFS triangles for \((L_\tau,R_\tau)\) hold up to f.q.i.; defects compose according to the Quantale structure of the \(\delta\)\hyp ledger (Specification~\ref{spec:delta-ledger}).
\end{enumerate}
\end{theorem}

\begin{remark}[Strictness vs.\ implementability]
The monad is \emph{strict} in \(\mathsf{Pers}^{\mathrm{cons}}_k\); the comonad and the AWFS are \emph{operational} in \(\mathrm{Ho}\) up to f.q.i.\ only. This realizes “strict at persistence, up to f.q.i.\ after lifting”.
\end{remark}

\subsection*{5.1 ter. Product–ledger Quantale and tolerance profile \texorpdfstring{\(\boldsymbol{\eta}\)}{η} (P7)}

We standardize the \emph{product–ledger} for multi\hyp axis budgeting.

\begin{definition}[Product–ledger quantale and tolerance]\label{def:product-ledger}
Fix axes \(\mathcal{A}=\{\text{alg},\text{disc},\text{meas}\}\). For each \(a\in\mathcal{A}\), let \((\mathsf{Q}_a,\otimes_a,\mathbf{1}_a,\le_a)\) be a commutative unital quantale (e.g.\ \(\overline{\mathbb{R}}_{\ge0},+\!,0,\le\)). Define the \emph{product–ledger quantale}
\[
\mathsf{Q}\ :=\ \prod_{a\in\mathcal{A}}\mathsf{Q}_a,\qquad 
(\delta_a)_{a}\ \otimes\ (\delta'_a)_{a}\ :=\ (\delta_a\otimes_a \delta'_a)_{a},\qquad
(\delta_a)_a\ \le\ (\delta'_a)_a\iff \forall a:\ \delta_a\le_a \delta'_a.
\]
A \emph{tolerance profile} is a vector \(\eta=(\eta_a)_{a\in\mathcal{A}}\in\mathsf{Q}\).
Acceptance on a window uses the \emph{componentwise} test \((\Sigma\delta)_a\le_a \eta_a\) for all \(a\).
When a scalar guard is needed (e.g.\ for logging), use any fixed \emph{monotone scalarization} \(\norm{-}:\mathsf{Q}\to \overline{\mathbb{R}}_{\ge0}\) (e.g.\ \(\ell^\infty\) or \(\ell^1\)) recorded in \texttt{run.yaml}.
\end{definition}

\begin{theorem}[AWFS \(2\)\hyp cell additivity on the product–ledger (P7)]\label{thm:2cell-additivity}
Let \(\epsilon_{i,\tau}:\Mirror\!\circ C_\tau \Rightarrow C_\tau\!\circ\Mirror\) be the natural \(2\)\hyp cell with bound \(\delta_{i,\tau}\in\mathsf{Q}\) (Definition~\ref{def:delta-2cell}). Then:
\begin{enumerate}
  \item \textbf{Vertical composition (series).} For composable \(2\)\hyp cells with bounds \(\delta,\delta'\in\mathsf{Q}\), the composite has bound \(\delta\otimes \delta'\) (componentwise aggregation).
  \item \textbf{Horizontal composition (parallel).} For independent branches with bounds \(\delta,\delta'\in\mathsf{Q}\), the product diagram has bound \(\delta\otimes \delta'\).
   \item \textbf{Tolerance check.} For any finite composite with total bound
  \(\delta_{\mathrm{tot}}:=\bigotimes_j \delta^{(j)}\),
  acceptance holds iff \(\delta_{\mathrm{tot}}\le \eta\) componentwise
  (and, if used for logging, \(\norm{\delta_{\mathrm{tot}}}\le \norm{\eta}\) under the recorded monotone scalarization).
\end{enumerate}
\emph{Proof sketch.} (1)–(2) are the monoidal laws in \(\mathsf{Q}=\prod_a \mathsf{Q}_a\). (3) follows by componentwise monotonicity and associativity of \(\otimes_a\).
\end{theorem}

\begin{remark}[Manifest fields (required)]
\texttt{run.yaml} must record: \texttt{ledger.axes}, \texttt{quantale.op} (per axis), \texttt{tolerance.eta} (vector), and \texttt{aggregation.scalarization}. This ensures reproducible acceptance with explicit \(\eta\) and scalar guard.
\end{remark}

\subsection*{5.2. [Spec] Coq/Lean Contracts: Stability, (Co)Limits, Bridge, AWFS, and Product–ledger \(\delta\)\hyp Commutation}

\noindent Identifiers are indicative; concrete names may follow local conventions (e.g.\ Lean/mathlib namespaces). Appendix~F lists one naming scheme. All equalities are asserted at the persistence layer; filtered\hyp level objects are considered in \(\mathrm{Ho}\) up to f.q.i.

\begin{specification}[Persistence truncation]\label{spec:pers-Ttau}
\begin{itemize}
  \item \texttt{pers\_Ttau\_exact}: exactness on short exact sequences.
  \item \texttt{pers\_Ttau\_lipschitz}: \(d_{\mathrm{int}}(\mathbf{T}_\tau M,\mathbf{T}_\tau N)\le d_{\mathrm{int}}(M,N)\).
  \item \texttt{pers\_Ttau\_pres\_colim\_pullback}: filtered colimits and finite limits preserved (constructible range).
  \item \texttt{pers\_Ttau\_compose}: \(\mathbf{T}_\tau\circ \mathbf{T}_\sigma=\mathbf{T}_{\max\{\tau,\sigma\}}\).
\end{itemize}
\end{specification}

\begin{specification}[Filtered\hyp complex level]\label{spec:filtered-level}
\begin{itemize}
  \item \texttt{Ctau\_lift}: \(\mathbf{P}_i(C_\tau F)\cong \mathbf{T}_\tau(\mathbf{P}_i(F))\).
  \item \texttt{Ctau\_colim}: \(\mathbf{P}_i(C_\tau(\varinjlim F_\lambda))\cong \varinjlim \mathbf{P}_i(C_\tau(F_\lambda))\).
  \item \texttt{Ctau\_pullback}: \(\mathbf{P}_i(C_\tau(F\times_H G))\cong \mathbf{P}_i(C_\tau(F)\times_{C_\tau(H)}C_\tau(G))\).
\end{itemize}
\end{specification}

\begin{specification}[AWFS contracts]\label{spec:awfs}
\begin{itemize}
  \item \texttt{awfs\_R\_comonad}: \(R_\tau\) is an idempotent comonad up to f.q.i.; persistence\hyp level image equals \(\mathbf{M}_\tau\).
  \item \texttt{awfs\_L\_monad}: \(L_\tau\) is an idempotent monad up to f.q.i.; exact and \(1\)\hyp Lipschitz at persistence.
  \item \texttt{awfs\_dist\_law}: a natural \(2\)\hyp cell \(\lambda:L_\tau R_\tau\Rightarrow R_\tau L_\tau\) with \(\delta\)\hyp bound.
  \item \texttt{awfs\_triangle}: AWFS triangle identities hold up to f.q.i.; defects aggregate in the product–ledger \(\mathsf{Q}\).
  \item \texttt{awfs\_factorization}: pipeline morphisms factor functorially as \(L_\tau\)\hyp step then \(R_\tau\)\hyp step.
\end{itemize}
\end{specification}

\begin{specification}[\(\delta\)\hyp ledger: product–ledger and tolerance]\label{spec:delta-ledger}
Fix \(\mathsf{Q}=\prod_{a\in\mathcal{A}}\mathsf{Q}_a\) as in Definition~\ref{def:product-ledger} and a tolerance profile \(\eta\in\mathsf{Q}\).
\begin{itemize}
  \item \texttt{delta\_quantale\_product}: bounds live in \(\mathsf{Q}\); aggregation is componentwise \(\otimes\).
  \item \texttt{delta\_2cell\_mirror\_collapse}: natural \(2\)\hyp cell \(\epsilon_{i,\tau}\) with bound \(\delta_{i,\tau}\in\mathsf{Q}\).
  \item \texttt{delta\_compose\_vertical/horizontal}: vertical and horizontal compositions aggregate via \(\otimes\) (Theorem~\ref{thm:2cell-additivity}).
  \item \texttt{delta\_tolerance}: acceptance if \((\Sigma\delta)_a\le_a \eta_a\) for all axes; optional scalarization \(\norm{-}\) is monotone.
  \item \texttt{delta\_lipschitz\_post}: any \(1\)\hyp Lipschitz post\hyp processing is \(\mathsf{Q}\)\hyp monotone (non\hyp increasing).
\end{itemize}
\end{specification}

\begin{specification}[Bridge and admissibility]\label{spec:bridge}
\begin{itemize}
  \item \texttt{PH1\_to\_Ext1\_under\_B}: under (B1)–(B3), \(\mathrm{PH}_1(F)=0\Rightarrow \Ext^1(\mathcal{R}(F),k)=0\).
  \item \texttt{admissible\_robust\_eps}: if \(\mathrm{PH}_1(C_\varepsilon F)=0\), then \(\Ext^1(\mathcal{R}(C_\varepsilon F),k)=0\).
\end{itemize}
\end{specification}

\begin{specification}[Tower diagnostics]\label{spec:mu-nu}
\begin{itemize}
  \item \texttt{mu\_def}: \(\mu^{\,i}=\mathrm{gdim}\,\ker(\phi_{i,\tau})\), \ \texttt{nu\_def}: \(\nu^{\,i}=\mathrm{gdim}\,\mathrm{coker}(\phi_{i,\tau})\).
  \item \texttt{mu\_nu\_finite}: finiteness (bounded degrees).
  \item \texttt{mu\_nu\_vanish}: under constructible filtered colimits, \(\phi_{i,\tau}\) is an isomorphism; hence \(\muc=\nuc=0\).
\end{itemize}
\end{specification}

\begin{specification}[Combined collapse coreflection]\label{spec:combined}
\begin{itemize}
  \item \texttt{Ccomb\_adjunction}: inclusion \(\iota:\mathsf{S}_\tau^{\mathrm{h}}\hookrightarrow\mathrm{Ho}(\mathsf{FiltCh}(k))\) has right adjoint \(\mathsf{C}_\tau^{\mathrm{comb}}\) (implementable range).
  \item \texttt{Ccomb\_compat}: \(\mathbf{P}_i(\mathsf{C}_\tau^{\mathrm{comb}}F)\cong \mathbf{T}_\tau(\mathbf{P}_iF)\) and \(\mathcal{R}(\mathsf{C}_\tau^{\mathrm{comb}}F)\cong \tau_{\ge 0}\mathcal{R}(F)\).
  \item \texttt{Ccomb\_lipschitz\_pers}: \(d_{\mathrm{int}}(\mathbf{P}_i(\mathsf{C}_\tau^{\mathrm{comb}}F),\mathbf{P}_i(\mathsf{C}_\tau^{\mathrm{comb}}G))\le d_{\mathrm{int}}(\mathbf{P}_i(F),\mathbf{P}_i(G))\).
\end{itemize}
\end{specification}

\subsection*{5.3. Minimal Foundations: ZFC and Dependent Type Theory}

\paragraph{ZFC assumptions (minimal).}
(S1)–(S5) as in the draft hold; in particular, \(\mathsf{Pers}^{\mathrm{cons}}_k\) is abelian and admits Serre localization, and \(\mathsf{FiltCh}(k)\) supplies bounded, finite\hyp type models.

\paragraph{Dependent type theory (Coq/Lean).}
(T1)–(T6) as in the draft hold; in particular, a relative\hyp category treatment of \(\mathrm{Ho}(\mathsf{FiltCh}(k))\) supports right adjoints up to f.q.i.

\paragraph{Set\hyp theoretic coherence.}
At persistence level, \(\mathbf{T}_\tau\dashv \iota_\tau\) is a reflection; at realization level, \(\tau_{\ge 0}\) is the right adjoint truncation. Proposition~\ref{prop:operational-coreflection} aggregates them into a right adjoint collapse in \(\mathrm{Ho}\) (up to f.q.i.), consistent with stability and (co)limit behavior in Chapter~2.

\subsection*{5.4. Declarations for External Realizations and Operational Recipe (\textbf{[Spec]})}

\begin{declaration}[Spec–Derived realizations]\label{spec:derived-geo}
We may use \(\mathcal{R}_{\mathrm{coh}}:\mathsf{FiltCh}(k)\to D^{\mathrm{b}}\mathrm{Coh}(X)\) or
\(\mathcal{R}_{\acute{e}t}:\mathsf{FiltCh}(k)\to D^{\mathrm{b}}_{\mathrm{c}}(X_{\acute{e}t},\Lambda)\) with \emph{field} \(\Lambda\) as specifications.
Projection formula and base change are invoked as in Appendix~N. The bridge \(\mathrm{PH}_1\Rightarrow\Ext^1\) is proved only in \(D^{\mathrm{b}}(k\text{-mod})\); external realizations do not extend the proven bridge.
\end{declaration}

\begin{declaration}[Spec–Operational recipe]\label{spec:operational-coreflection}
We operate with
\[
F\ \longmapsto\ \big(C_\tau(F)\ \text{at persistence}\big)\quad\text{and}\quad\big(\tau_{\ge 0}\mathcal{R}(F)\ \text{at realization}\big),
\]
using right\hyp adjoint phrasing only at [Spec]; coherence and limits are in Appendix~B. When present, admissible \(L_\tau\)\hyp preprocessing precedes collapse; the \(\delta\)\hyp ledger aggregates both sides.
\end{declaration}

\subsection*{5.5. \(\delta\)\hyp Budget Naturalities and the Pipeline Error Budget}

\begin{definition}[Natural \(2\)\hyp cell and \(\delta\)\hyp ledger]\label{def:delta-2cell}
For each \(\tau>0\) and \(i\), a natural \(2\)\hyp cell \(\epsilon_{i,\tau}:\Mirror\circ C_\tau\Rightarrow C_\tau\circ\Mirror\) carries a bound \(\delta_{i,\tau}\in\mathsf{Q}\) (Specification~\ref{spec:delta-ledger}):
\[
d_{\mathrm{int}}\!\Big(\mathbf{T}_\tau \mathbf{P}_i(\Mirror(C_\tau F)),\ \mathbf{T}_\tau \mathbf{P}_i(C_\tau(\Mirror F))\Big)\ \le\ \norm{\delta_{i,\tau}},
\]
monotone in any chosen scalarization \(\norm{-}\). Decompose \(\delta=(\delta_a)_{a\in\mathcal{A}}\) and record in the product–ledger.
\end{definition}

\begin{proposition}[Pipeline error budget]\label{prop:pipeline-budget}
Let \(U_m,\dots,U_1\) be A\hyp side steps with collapses \(C_{\tau_j}\) and bounds \(\delta_j(i,\tau_j)\in\mathsf{Q}\). Then for fixed \(\tau\),
\[
d_{\mathrm{int}}\!\Big(\mathbf{T}_\tau \mathbf{P}_i(\Mirror(C_{\tau_m}U_m\!\cdots C_{\tau_1}U_1F)),\ \mathbf{T}_\tau \mathbf{P}_i(C_{\tau_m}U_m\!\cdots C_{\tau_1}U_1\Mirror F)\Big)\ \le\ \norm{\bigotimes_{j=1}^m \delta_j(i,\tau_j)},
\]
and any \(1\)\hyp Lipschitz post\hyp processing is non\hyp increasing for the bound.
\end{proposition}

\begin{remark}[Safety margin with tolerance]
Per window \(W\) and \(\tau\), B\hyp Gate\({}^{+}\) uses a tolerance profile \(\eta\) and the product–ledger sum \(\Sigma\delta\); accept if \((\Sigma\delta)_a\le_a \eta_a\) on all axes (and, optionally, \(\norm{\Sigma\delta}\le \norm{\eta}\)).
\end{remark}

\subsection*{5.6. Commutable Torsion: Adoption Policy, A/B Soft\hyp Commuting Priorities, and Accounting}

\begin{definition}[Torsion reflectors and nesting]\label{def:torsion-reflectors}
Let \(T_A,T_B\) be exact reflectors on \(\mathsf{Pers}^{\mathrm{cons}}_k\) from hereditary Serre subcategories \(E_A,E_B\). Say \(T_A,T_B\) are \emph{nested} if \(E_A\subseteq E_B\) or \(E_B\subseteq E_A\).
\end{definition}

\begin{proposition}[Order independence under nesting]\label{prop:nested-commute}
If nested, then \(T_A\circ T_B=T_B\circ T_A=T_{A\vee B}\), where \(E_{A\vee B}\) is the Serre subcategory generated by \(E_A\cup E_B\). In particular, for length thresholds, \(\mathbf{T}_\tau\circ \mathbf{T}_\sigma=\mathbf{T}_{\max\{\tau,\sigma\}}\).
\end{proposition}

\begin{definition}[A/B soft\hyp commuting: priorities, fallback, and \(\Delta_{\mathrm{comm}}\) accounting]\label{def:soft-commute}
Given \(T_A,T_B\) not known to be nested:
\begin{enumerate}
  \item \textbf{Priority rule.} Prefer \emph{index\hyp local} reflectors first (e.g.\ cropping/birth\hyp window) then \emph{length\hyp type} reflectors; if incomparable, choose the order minimizing a pilot bound of \(\Delta_{\mathrm{comm}}:=d_{\mathrm{int}}(T_AT_BM,T_BT_AM)\) on a calibration set.
  \item \textbf{Tolerance test.} If \(\Delta_{\mathrm{comm}}\le \norm{\eta}\) (or componentwise \(\Delta_{\mathrm{comm}}\le \eta\)), accept soft\hyp commuting and do not reorder; else \emph{fix} the priority order and proceed deterministically.
  \item \textbf{Ledger rule.} Log \(\Delta_{\mathrm{comm}}\) as \(\delta^{\mathrm{alg}}\) in the product–ledger and include it in \(\Sigma\delta\); subsequent \(1\)\hyp Lipschitz post\hyp processing cannot increase it.
\end{enumerate}
\end{definition}

\begin{remark}[Micro example]
For \(T^{\mathrm{len}}_\tau\) (length) and \(T^{\mathrm{birth}}_{[u,u')}\) (birth\hyp window), apply \(T^{\mathrm{birth}}\) then \(T^{\mathrm{len}}\); if \(\Delta_{\mathrm{comm}}>\norm{\eta}\), keep this order and record \(\Delta_{\mathrm{comm}}\) into \(\delta^{\mathrm{alg}}\).
\end{remark}

\subsection*{5.7. Worked micro\hyp example (policy illustration)}
Let \(T^{\mathrm{len}}_\tau\) be length threshold and \(T^{\mathrm{birth}}_{[u,u')}\) birth\hyp window deletion. Measure \(\Delta_{\mathrm{comm}}(M;\mathrm{len},\mathrm{birth})\). If \(\le\norm{\eta}\), adopt soft\hyp commuting; otherwise fix the order \(T^{\mathrm{birth}}\) then \(T^{\mathrm{len}}\) and record \(\Delta_{\mathrm{comm}}\) in \(\delta^{\mathrm{alg}}\).

\subsection*{5.8. Overlap Gate (Functorial Gluing): collapse compatibility, soft commuting, Čech–Ext\textsuperscript{1}, stable bands \textnormal{\textbf{[Spec]}}}

We formalize a \emph{functorial} Overlap Gate that lifts the operational Overlap Gate (Chapter~1) to a typed, gluing\hyp ready interface.

\begin{definition}[Window Stack (WinFib) and Čech nerve]\label{def:winfib}
Let \(\mathsf{Win}\) be the category of pairs \((\alpha,W_\alpha)\) with \(W_\alpha\) right\hyp open, morphisms induced by inclusions \(X_\alpha\cap W_\alpha\hookrightarrow X_\beta\cap W_\beta\). For fixed degree \(i\) and \(\tau>0\), define a pseudo\hyp functor
\[
\mathcal{S}_{i,\tau}(-):\ \mathsf{Win}^{\mathrm{op}}\longrightarrow \mathrm{Cat},\qquad (\alpha,W_\alpha)\ \longmapsto\ \Big\{\mathbf{T}_\tau\,\Crop_{W_\alpha}\big(\mathbf{P}_i(F|_{X_\alpha})\big)\Big\}\subset\mathsf{Pers}^{\mathrm{cons}}_k.
\]
Its Grothendieck fibration \(\pi:\int\mathcal{S}_{i,\tau}(-)\to \mathsf{Win}\) is the \emph{Window Stack (WinFib)}. Let \(N(\mathcal{U})\) be the Čech nerve of the domain cover \(\{X_\alpha\}\).
\end{definition}

\begin{definition}[Overlap Gate \(\mathrm{OG}^{\mathrm{funct}}\)]\label{def:og-funct}
Fix \((i,\tau)\) and a windowed cover \(\{X_\alpha,W_\alpha\}\). We say \(\mathrm{OG}^{\mathrm{funct}}(i,\tau)\) \emph{passes} if:
\begin{enumerate}
  \item \textbf{Collapse compatibility} (after\hyp collapse \(1\)\hyp Lipschitz defect): for all overlaps, the objects in \(\int\mathcal{S}_{i,\tau}(-)\) agree up to the declared \(\delta\)\hyp budget, and the safety margin dominates the budget on overlaps.
  \item \textbf{Soft commuting (A/B)}: any pair of non\hyp nested reflectors used on overlaps passes the tolerance test with profile \(\eta\) (Definition~\ref{def:soft-commute}); otherwise a deterministic order is fixed and \(\Delta_{\mathrm{comm}}\) is accounted in \(\delta^{\mathrm{alg}}\).
  \item \textbf{Čech–Ext\textsuperscript{1}–acyclicity} (degree \(1\)): \(\Ext^1(\mathcal{R}(C_\tau F|_{X_{\alpha_0\cdots\alpha_p}}),k)=0\) for \(p=0,1\) on overlaps, and the Čech differential in \(\Ext^1\) vanishes.
  \item \textbf{Stable band \& no\hyp accumulation} (tower diagnostics): on each window and overlap, the tail comparison is an isomorphism \(\DiagZero\) on a stability band of \(\tau\)’s, with near\hyp \(\tau\) non\hyp accumulation.
\end{enumerate}
\end{definition}

\begin{theorem}[Functorial gluing via \(\mathrm{OG}^{\mathrm{funct}}\)]\label{thm:og-funct}
If \(\mathrm{OG}^{\mathrm{funct}}(i,\tau)\) passes for degree \(i=1\) on a windowed cover, then:
\begin{enumerate}
  \item (Existence) The local collapsed objects glue to a global object in \(\mathsf{Pers}^{\mathrm{cons}}_k\) (persistence layer) that is unique up to isomorphism on windows.
  \item (Gate propagation) The global B–Gate\({}^{+}\) passes on the union \(\bigcup_\alpha X_\alpha\times W_\alpha\): specifically, \(\mathrm{PH}_1(C_\tau F)=0\), \(\Ext^1(\mathcal{R}(C_\tau F),k)=0\), \(\DiagZero\).
  \item (Budget control) The global safety margin is bounded below by the minimum of local margins minus the overlap budgets (including A/B residuals), with product–ledger accounting.
\end{enumerate}
\end{theorem}

\begin{remark}[IMRN/AiM readiness]
The acceptance criteria (1)–(4) are checkable on the Čech nerve of the windowed cover, wholly \emph{after collapse} on the B\hyp side single layer. All budgets and tolerances are recorded per window; proofs use only exactness/Lipschitzness at persistence and amplitude \(\le 1\) at realization. The AWFS view (Declaration~\ref{dec:awfs}, Theorem~\ref{thm:awfs-triangle}) packages pre\hyp processing/collapse coherently and quantifies residual non\hyp commutation via the product–ledger \(\mathsf{Q}\).
\end{remark}

\subsection*{5.9. Window Stack (WinFib): typed acceptance on the nerve and auditability \textnormal{\textbf{[Spec]}}}

\begin{definition}[Typed acceptance predicate on the nerve]\label{def:typed-accept}
For each simplex \(\sigma=\{\alpha_0,\dots,\alpha_p\}\) in \(N(\mathcal{U})\) and a window \(W_\sigma=\bigcap_j W_{\alpha_j}\), define a \emph{typed acceptance} datum
\[
\mathbf{Acc}(\sigma; i,\tau)\ :=\ \Big(\mathrm{iso\_after\_collapse},\ \mathrm{AB\_soft\_commute},\ \mathrm{Cech\_Ext^1\_zero},\ \mathrm{stable\_band\_ok}\Big),
\]
with booleans and product–ledger budgets. We say the nerve passes if \(\mathbf{Acc}(\sigma;i,\tau)\) holds for all \(\sigma\) up to edges and vertices, and higher\hyp dimensional diagonals impose no extra constraints beyond Čech\(^1\)\hyp acyclicity.
\end{definition}

\begin{proposition}[Nerve acceptance implies \(\mathrm{OG}^{\mathrm{funct}}\)]\label{prop:nerve-accept}
If \(\mathbf{Acc}(\sigma;1,\tau)\) holds for all \(\sigma\) up to edges and vertices, then \(\mathrm{OG}^{\mathrm{funct}}(1,\tau)\) passes. Consequently Theorem~\ref{thm:og-funct} applies.
\end{proposition}

\begin{remark}[Machine\hyp checkable audit]
The quadruple \(\mathbf{Acc}\) is a minimal, machine\hyp checkable record per nerve simplex; together with the product–ledger and tolerance profile \(\eta\), it yields a complete audit trail for local\hyp to\hyp global collapse decisions. A Lean/Coq stub can represent \(\mathbf{Acc}\) as a structure with fields and proofs (Appendix~F).
\end{remark}

\subsection*{5.10. Summary}
We established a coherent functorial core for collapse within the constructible regime: a persistence\hyp level exact reflector \(\mathbf{T}_\tau\), an operational right adjoint collapse in \(\mathrm{Ho}\) (up to f.q.i.), and formal contracts for stability and bridge usage. An \emph{operational AWFS} \((L_\tau,R_\tau,\lambda)\) captures pre\hyp processing and collapse with idempotence and triangle laws up to f.q.i., while the \emph{product–ledger} makes the \(\delta\)\hyp budget \emph{natural} and \emph{additive} for both vertical and horizontal compositions with a \emph{tolerance profile} \(\eta\). For torsion reflectors, \emph{order independence} is guaranteed under \emph{nesting}; otherwise an A/B \emph{soft\hyp commuting} policy specifies priorities, a tolerance test, and a deterministic fallback with explicit \(\Delta_{\mathrm{comm}}\) logging. We formalized a \emph{functorial Overlap Gate} packaging collapse compatibility, soft commuting, Čech–Ext\({}^{1}\)–acyclicity, and stability bands; together with the \emph{Window Stack (WinFib)}, this yields a typed, nerve\hyp level acceptance test ensuring local\hyp to\hyp global gluing after collapse with fully logged budgets. All gate decisions remain on the B\hyp side after collapse, within a reproducible, windowed, and metrically stable framework that aligns with the tower diagnostics of Chapter~4 and the realization bridge of Chapter~3.


% NOTE TO READER: some previously uploaded auxiliary files appear to have expired on the platform; if you need them reloaded for cross-checks, please re-upload. This comment does not affect compilation.
\section{Chapter 6: Geometric Collapse (Program/Spec)}
\addcontentsline{toc}{section}{Geometric Collapse (Program/Spec)}
\label{sec:ch6}

\noindent\textbf{Monotonicity policy (after truncation).}
Deletion\hyp type updates are \emph{non\hyp increasing} for windowed persistence energies and spectral indicators; inclusion\hyp type updates are \emph{stability\hyp only} (non\hyp expansive). See Appendix~E for sufficient conditions and counterexamples. All comparisons, equalities, and gate decisions are made only after applying the truncation \(\mathbf{T}_\tau\) at the persistence layer:
\[
\boxed{\ \text{for each } t\ \Longrightarrow\ \mathbf{P}_i\ \Longrightarrow\ \mathbf{T}_\tau\ \Longrightarrow\ \text{compare in }\Pers^{\mathrm{ft}}\ }.
\]

\subsection*{6.0. Standing hypotheses and admissible geometric realization}
We work over a fixed field \(k\) and adopt the notation and hypotheses of Part~I. In particular,
\(\mathsf{FiltCh}(k)\) denotes finite\hyp type filtered chain complexes over \(k\),
\(\mathbf{P}_i:\mathsf{FiltCh}(k)\to\Pers^{\mathrm{ft}}_k\) the degreewise persistence functor,
and we write \(\mathbf{T}_\tau\) for the exact bar\hyp deletion (Serre) localization at scale \(\tau\ge 0\) (allowing \(\mathbf{T}_\tau=\mathrm{Id}\) when \(\tau=0\)).
Its filtered lift \(C_\tau\) is used \emph{up to filtered quasi\hyp isomorphism} (Chapter~2, §§2.2–2.3; Appendix~B).
The realization \(\mathcal{R}:\mathsf{FiltCh}(k)\to D^{\mathrm{b}}(k\text{-mod})\) is \(t\)\hyp exact.
All statements in this chapter lie in the constructible range (we identify \(\Pers^{\mathrm{ft}}_k\) with the constructible subcategory).
Unless explicitly marked \textbf{[Spec]}, \emph{equalities and Lipschitz claims are asserted only at the persistence layer}; identities at the filtered\hyp complex layer hold \emph{up to filtered quasi\hyp isomorphism}.
Kernel/cokernel diagnostics \((\DiagZero / \DiagNonzero)\) are computed from the comparison maps
\[
  \phi_{i,\tau}:\ \varinjlim\nolimits_\lambda \mathbf{T}_\tau\!\big(\mathbf{P}_i(F_\lambda)\big)\ \longrightarrow\ \mathbf{T}_\tau\!\big(\mathbf{P}_i(\varinjlim\nolimits_\lambda F_\lambda)\big),
\]
with \(\dim_k\) interpreted as the \emph{generic\hyp fiber} dimension after truncation (multiplicity of \(I[0,\infty)\)); see Appendix~D, Remark~\ref{D:rem:generic-dim}.
Windows are MECE and right\hyp open by default. When stated, windows are \emph{definable} in a fixed o\hyp minimal expansion to guarantee finite event sets and finite Čech depth (Appendix~H/J; used below).

\begin{definition}[Admissible geometric realization]\label{def:geom-real}
Let \(\mathsf{Geom}\) be a geometric input category (e.g.\ metric or metric\hyp measure spaces with \(1\)\hyp Lipschitz maps; triangulated manifolds with mesh\hyp refinement maps; weighted graphs with contraction/sparsification maps).
An \emph{admissible geometric realization} is a functor
\[
  \mathcal{G}:\ \mathsf{Geom}\longrightarrow \mathsf{FiltCh}(k)
\]
such that:
\emph{(i)} \(\mathcal{G}\) is functorial and sends non\hyp expansive maps to filtered chain maps whose images under each \(\mathbf{P}_i\) are \(1\)\hyp Lipschitz for the interleaving distance;
\emph{(ii)} degreewise finite\hyp type is preserved;
\emph{(iii)} subsampling/refinement maps are carried to filtered maps that, for each fixed \(\tau\), induce filtered quasi\hyp isomorphisms after applying \(C_\tau\).
\end{definition}

\begin{remark}[Program posture and bridges]\label{rk:LC}
All specifications are asserted within the \emph{implementable range} of Part~I: (co)limit and stability statements are restricted to the persistence layer; the lifting–coherence hypothesis \(\mathrm{(LC)}\) is assumed for comparing \(C_\tau\) on \(\mathsf{FiltCh}(k)\) with effects after realization \(\mathcal{R}\).
No equivalence \(\mathrm{PH}_1\Leftrightarrow\Ext^1\) is claimed; only the one\hyp way bridge under \textup{(B1)–(B3)} from Part~I is used.
The obstruction \(\muc\) is \emph{distinct} from the classical Iwasawa \(\mu\)\hyp invariant.
\end{remark}

\begin{remark}[Stability vs.\ monotonicity; spectral policy]\label{rem:stability-vs-monotonicity}
Non\hyp expansive maps ensure stability (non\hyp expansiveness) of all indicators.
Under \emph{deletion\hyp type} updates satisfying Appendix~E (Dirichlet restriction, principal submatrices/Schur complements, Loewner contractions, and—in the symplectic setting—stop additions/Liouville contractions), spectral tails and windowed energies are \emph{non\hyp increasing}.
Inclusion\hyp type updates guarantee only \emph{stability}.
Spectral indicators are \emph{not} f.q.i.\ invariants; throughout we treat them as \emph{stable under a fixed normalization policy} and evaluate them on \(L(C_\tau F)\) (see Chapter~11).
\end{remark}

All monotonicity claims are interpreted after truncation by \(\mathbf{T}_\tau\).

\subsection*{6.0bis. Pipeline normal form and safe low\hyp pass (C\texorpdfstring{\(_\tau\)}{tau}\(\rightarrow\)W\(_{\mathrm{clip}}\)\(\rightarrow\)LP\(_\tau\))}
\begin{definition}[Window clipping]\label{def:Wclip}
For a right\hyp open window \(W=[u,u')\), the \emph{window clip} \(W_{\mathrm{clip}}\) acts on a persistence module \(M\) by restriction and extension by zero: \(M\mapsto M|_W\) viewed inside \(\Pers^{\mathrm{ft}}_k\). At the filtered level we implement \(W_{\mathrm{clip}}\) by cropping the filtration (up to f.q.i.).
\end{definition}

\begin{definition}[Safe low-pass at scale $\tau$]\label{def:LP}
A safe low-pass operator $\mathrm{LP}_\tau$ is an \emph{optional} post-processing
acting only on spectral auxiliaries computed from $L(C_\tau F|_W)$ (e.g. heat trace/tails).
It does \emph{not} modify the persistence-layer objects $\mathbf{T}_\tau\mathbf{P}_i(F|_W)$ used for gates.
\end{definition}


\begin{theorem}[T\hyp Lipschitz\hyp After\hyp Collapse adoption test]\label{thm:LP-adopt}
Adopt the pipeline
\[
\boxed{\,C_\tau\ \longrightarrow\ W_{\mathrm{clip}}\ \longrightarrow\ \mathrm{LP}_\tau\,}
\]
\emph{only if} the after\hyp collapse Lipschitz condition of Definition~\ref{def:LP}\,(2) is verified on the window \(W\) within the declared tolerance. Otherwise, set \(\mathrm{LP}_\tau=\mathrm{Id}\). Under adoption, deletion\hyp type monotonicity (Remark~\ref{rem:stability-vs-monotonicity}) is preserved.
\end{theorem}

\subsection*{6.1. Monitored indicators and energies}
Fix an admissible \(\mathcal{G}\) and write \(F=\mathcal{G}(X)\in \mathsf{FiltCh}(k)\).

\begin{definition}[Persistence energies]\label{def:PE}
Fix a deletion threshold $\tau>0$ and (optionally) a cap $\rho>0$ (default $\rho=\tau$).
Let $\mathcal{B}_i^\tau(F)$ be the barcode multiset of $\mathbf{T}_\tau\mathbf{P}_i(F)$.
For $\alpha>0$ define
\[
\mathrm{PE}_{i,\alpha}^{\le\rho}(F;\tau)
:= \sum_{[b,d)\in \mathcal{B}_i^\tau(F)}
   \bigl(\min\{d,\rho\}-\min\{b,\rho\}\bigr)_{+}^{\alpha}.
\]
By definition this is evaluated \emph{after truncation} (at the persistence layer).
\end{definition}


\begin{definition}[Spectral indicators]\label{def:spectral}
Let \(L(C_\tau F)\) be a combinatorial Hodge Laplacian on the truncated complex \(C_\tau F\) (normalized, with the Euclidean inner product on chains).
Denote the non\hyp decreasing spectrum by \((\lambda_m(C_\tau F))_{m\ge 0}\).
For \(\beta>0\) and an \emph{integer} cutoff \(M(\tau)\in\mathbb{N}\), define the spectral tail
\[
  \mathrm{ST}_{\beta}^{\ge M(\tau)}(F)\ :=\ \sum_{m\ge M(\tau)} \lambda_m(C_\tau F)^{-\beta},\qquad
  \mathrm{HT}(t;F)\ :=\ \mathrm{Tr}\big(e^{-tL(C_\tau F)}\big)\quad (t>0),
\]
with zero modes excluded (or replaced by the Moore–Penrose pseudoinverse). Qualitative specifications are invariant under these standard choices; the policy \((\beta,M(\tau),t)\) is fixed across a run (Appendix~G; Chapter~11).
\end{definition}

\begin{remark}[Convergence, parameterization, and logging]\label{rk:ST-conv}
Choose \(\beta\) and \(M(\tau)\) to ensure convergence (typical \(\beta\in\{1,2\}\), \(M(\tau)=\lfloor c\,\tau^{\gamma}\rfloor\) with \(c>0\), \(\gamma\in(0,2]\)).
When sweeping \(\tau\), take \(M(\tau)\) non\hyp decreasing to avoid artificial discontinuities.
Normalization, zero\hyp mode handling, and the window policy are fixed and logged with \((\beta,M(\tau),t)\).
\end{remark}

\begin{definition}[\(\Ext^1\)\hyp collapse at scale]\label{def:ext-collapse}
Writing \(\mathcal{R}(F)\in D^{\mathrm{b}}(k\text{-mod})\), we say \emph{\(\Ext^1\)\hyp collapse holds at scale \(\tau\)} if, for all \(Q\in\mathcal{Q}:=\{k[0]\}\),
\[
  \Ext^1\!\big(\mathcal{R}(C_\tau F),\, Q\big)\ =\ 0.
\]
\end{definition}

\subsection*{6.2. Stability under filtered colimits (geometry level)}
Let \(L_i(C_\tau F)\) denote the normalized combinatorial Hodge Laplacian in degree \(i\) on \(C_\tau F\), with nondecreasing positive spectrum \(\bigl(\lambda_{i,m}(C_\tau F)\bigr)_{m\ge 0}\). For brevity we suppress \(i\) and write \(L(C_\tau F)\), \(\mathrm{ST}_{\beta}^{\ge M(\tau)}(F)\), \(\mathrm{HT}(t;F)\) when the degree is clear from context.
\begin{declaration}[Specification: Stability under filtered colimits in geometry]\label{spec:geom-colim}
Assume a filtered diagram \(\{F_\lambda\}\) in \(\mathsf{FiltCh}(k)\) remains degreewise finite\hyp type; filtered (co)limits are computed objectwise in \([\mathbb{R},\mathsf{Vect}_k]\) and used only under the scope policy of Appendix~A (compute in the functor category and verify return to \(\Pers^{\mathrm{cons}}_k\)).
Then, for each fixed \(\tau\), the induced maps
\[
  \phi_{i,\tau}:\ \varinjlim\nolimits_\lambda \mathbf{T}_\tau\!\big(\mathbf{P}_i(F_\lambda)\big)\ \xrightarrow{\ \cong\ }\ \mathbf{T}_\tau\!\big(\mathbf{P}_i(\varinjlim\nolimits_\lambda F_\lambda)\big)
\]
are isomorphisms; hence \(\muc=\nuc=0\) at that scale. The conclusion holds pointwise along any discrete \(\tau\)\hyp sweep.
\end{declaration}

\begin{remark}[Endpoints and infinite bars]\label{rk:endpoints-ch6}
Endpoint conventions (open/closed) and the treatment of infinite bars are as in Chapter~2, Remark~\ref{rk:2-endpoints}; \(\mathbf{T}_\tau\) deletes only finite bars of length \(\le\tau\).
\end{remark}

\subsection*{6.3. Joint monitoring and programmatic guarantees}
\begin{declaration}[Specification: Geometric collapse indicators]\label{spec:geom-indicators}
Under \(\mathrm{(LC)}\) and within the implementable range, along geometric degenerations \emph{compute and record}:
\begin{enumerate}
  \item \(\mathbf{T}_\tau\mathbf{P}_i(F)\) and the truncated energies \(\mathrm{PE}_i^{\le\tau}\) on \(\mathbf{T}_\tau\mathbf{P}_i(F)=\mathbf{P}_i(C_\tau F)\);
  \item spectral indicators \(\mathrm{ST}_{\beta}^{\ge M(\tau)}\) or \(\mathrm{HT}(t;\cdot)\) on \(L(C_\tau F)\) (parameters as in Remark~\ref{rk:ST-conv});
  \item the \(\Ext^1\)\hyp check \(\Ext^1\big(\mathcal{R}(C_\tau F),Q\big)=0\) for \(Q\in\mathcal{Q}\).
\end{enumerate}
The \emph{stable regime} is declared where \(\DiagZero\) and (1)–(3) hold jointly.
\end{declaration}

\begin{remark}[Saturation gate (reference; see Chapter~11)]\label{rk:sat-gate-ch6}
We follow the Chapter~11 policy for a window \([0,\tau^\ast]\): \emph{(i)} eventually the maximal finite bar length in \(\mathbf{T}_{\tau^\ast}\mathbf{P}_i(F_t)\) is \(\le \eta\); \emph{(ii)} eventually \(d_{\mathrm{int}}\!\big(\mathbf{T}_{\tau^\ast}\mathbf{P}_i(F_t),\mathbf{T}_{\tau^\ast}\mathbf{P}_i(F_{t'})\big)\le \eta\); \emph{(iii)} the edge gap \(\delta:=\tau^\ast-\max\{b_r<\tau^\ast\}\) satisfies \(\delta>\eta\).
This chapter \emph{uses the gate only as a reference}; the quantitative policy and its verification are centralized in Chapter~11.
\end{remark}

\subsection*{6.3bis. Gate Cascade (P8): \(E_1\rightarrow (\mu,\nu)\rightarrow \Ext^1\rightarrow \mathrm{PH}_1\)}
\begin{theorem}[Gate Cascade rule (windowed, Core-safe)]\label{thm:cascade}
Fix a right-open window $W$ and $\tau>0$. Assume:
(a) after-collapse evaluation; (b) $W$ is a saturation window at $\tau$;
(c) tail isomorphism on $W$ in degree $1$ (i.e. $(\mu_{1,\tau},\nu_{1,\tau})=(0,0)$).
Then on $W$,
\[
\mathbf{T}_\tau\mathbf{P}_1(F|_W)=0
\ \Longrightarrow\ 
\mathrm{PH}_1(C_\tau F|_W)=0
\ \Longrightarrow\ 
\Ext^1(\mathcal{R}(C_\tau F|_W),k)=0,
\]
and moreover (by Theorem~\ref{thm:ch4-local-equiv}) on such $(W,\tau)$ we have
\[
\mathrm{PH}_1(C_\tau F|_W)=0 \iff \Ext^1(\mathcal{R}(C_\tau F|_W),k)=0.
\]
\end{theorem}

\subsection*{6.4. Scope, definable windows, and design patterns}\label{subsec:ch6-definable}
\begin{declaration}[Specification: Scope of admissible degenerations]\label{spec:scope}
The program encompasses: \emph{(a)} metric(\hyp measure) collapses modeled by subsampling and \(1\)\hyp Lipschitz retractions; \emph{(b)} simplicial refinements with bounded local degree; \emph{(c)} graph sparsifications preserving the normalized Laplacian construction and the \(1\)\hyp Lipschitz property of \(\mathcal{G}\), thereby keeping each \(\mathbf{P}_i\) non\hyp expansive under these maps.
Each case is functorially embedded by an admissible \(\mathcal{G}\).
\end{declaration}

\begin{remark}[Definable windows and finite Čech depth]\label{rem:definable-windows}
When windows \([u,u')\) (and, if present, domain covers) are definable in a fixed o\hyp minimal expansion of \((\mathbb{R},+,\cdot)\), one has: (i) only finitely many events (births/deaths) occur on each bounded window; (ii) the Čech nerve has finite depth; hence Overlap Gate checks reduce to finitely many overlaps and are fully auditable. This applies verbatim to geometric realizations, and integrates with the E\(_1\)\,–\,local gate (Chapter~3, Theorem~3.2) on definable windows.
\end{remark}

\begin{center}
\begin{tikzcd}[
  row sep=1.6em, column sep=3.4em,
  cells={nodes={font=\footnotesize}},
  every label/.append style={font=\scriptsize},
  scale=0.92, transform shape
]
{\text{Geometric input } X} \arrow[r, "\mathcal{G}"] 
  & {F\in \mathsf{FiltCh}(k)} \arrow[r, "\mathbf{P}_i"] 
    & {\mathbf{P}_i(F)} \arrow[r, "\mathbf{T}_\tau"] \arrow[d, dashed, "\mathrm{PE}_i^{\le\tau}"']
      & {\mathbf{T}_\tau\mathbf{P}_i(F)} \arrow[d, dashed, "\mathrm{PE}_i^{\le\tau}"] \\
{} & {} & {\text{Spectral } L(C_\tau F)}
      \arrow[r, dashed, "\text{heat trace / tails on }L(C_\tau F)"] 
      & {{\mathrm{HT},\,\mathrm{ST}_{\beta}}} \\
{} & {\mathcal{R}(F)}
      \arrow[from=3-2, to=1-3, bend left=20, dashed, "\mathrm{(LC)}"]
      \arrow[from=3-2, to=3-4, dashed, "{\begin{matrix}C_\tau\text{ then }\\ \Ext^1(-,Q)\end{matrix}}"]
  & {}
  & {{\Ext^1\big(\mathcal{R}(C_\tau F),Q\big)=0\ \ (\text{check})}}
\end{tikzcd}
\end{center}

\subsection*{6.5. Failure geometry and diagnostics}
\begin{definition}[Geometric failure types at scale]\label{def:geom-failure}
Within the monitored window, a sample is \emph{Type~IV at scale \(\tau\)} if \(\mathrm{PE}^{\le\tau}\) and spectral indicators decay while \(\DiagNonzero\).
The \emph{pure cokernel type} denotes \(\muc=0\) and \(\nuc>0\).
\end{definition}

\begin{declaration}[Specification: Diagnostic actions]\label{spec:diagnostic}
When \(\DiagNonzero\), refine the index diagram or adjust \(\tau\)\hyp sweep granularity until either \emph{(a)} the obstruction vanishes, or \emph{(b)} the failure persists across refinements, in which case the regime is recorded as non\hyp collapsible at the monitored scale.
\end{declaration}

\subsection*{6.6. Symplectic hook: Fukaya realization (\textbf{[Spec]})}
\begin{declaration}[Spec--Fukaya realization]\label{spec:fukaya}
Let \(\mathsf{Symp}^{\mathrm{adm}}\) be exact/monotone Liouville domains or sectors with stops.
\(\mathcal{G}_{\mathrm{Fuk}}\) assigns action\hyp filtered Floer complexes on a fixed window \([a,\tau]\) with \(a\le \tau\) over a field.
Assume:
\emph{(F1)} finite action spectrum in \([a,\tau]\);
\emph{(F2)} continuation maps shift actions by \(\le\varepsilon\) uniformly (hence are \(1\)\hyp Lipschitz for interleavings);
\emph{(F3)} stop additions/Liouville contractions are deletion\hyp type (Appendix~E).
Then for each degree \(i\) and scale \(\tau\) the comparison maps \(\phi_{i,\tau}\) are isomorphisms, hence \(\DiagZero\) on the monitored window.
Proof sketches and scope limits appear in Appendix~O.
\end{declaration}

\begin{remark}[Scope and bridge domain]
The specification above does \emph{not} extend the proved bridge beyond \(D^{\mathrm{b}}(k\text{-mod})\);
it provides a stable geometric hook whose persistence\hyp level behavior feeds the Part~I pipeline.
\end{remark}

\subsection*{6.7. Permitted operations catalog and \(\delta\)\hyp ledger (reinforced policy)}\label{subsec:ops-delta}
We record the admissible A\hyp side operations, their expected persistence\hyp level behavior \emph{after collapse}, and the mandatory \(\delta\) logging.

\begin{definition}[Permitted operations]\label{def:ops-catalog}
Each A\hyp side step \(U\) is labeled:
\begin{itemize}
  \item \emph{Deletion\hyp type (monotone; P5).} Examples: stop addition / sector shrinking (symplectic), mollification (low\hyp pass filtering), viscosity increment (PDE), threshold lowering, filter upper\hyp cap. \emph{Guarantee:} after applying \(C_\tau\) (and, if adopted, \(\mathrm{LP}_\tau\)), windowed persistence energies and spectral auxiliaries (aux\hyp bars) are \emph{non\hyp increasing}.
  \item \emph{\(\varepsilon\)\hyp continuation (non\hyp expansive).} Examples: small Hamiltonian continuation; micro time\hyp step; minor stop shift. \emph{Guarantee:} \(d_{\mathrm{int}}(\mathbf{P}_i(F),\mathbf{P}_i(UF))\le \varepsilon\); after \(C_\tau\), indicators are \emph{stable} up to the prescribed \(\varepsilon\).
  \item \emph{Inclusion\hyp type (stable only).} Examples: domain enlargement, inclusion maps not covered by the deletion\hyp type list. \emph{Guarantee:} no monotonicity claim; only stability (non\hyp expansiveness) if the induced map is \(1\)\hyp Lipschitz on persistence.
\end{itemize}
\end{definition}

\begin{declaration}[Mandatory \(\delta\)\hyp ledger]\label{dec:delta-ledger-ch6}
For each step \(U\) with collapse \(C_{\tau}\) and a fixed degree \(i\), record a three\hyp part non\hyp commutation budget
\[
\delta(i,\tau)\ =\ \delta^{\mathrm{alg}}(i,\tau)\ +\ \delta^{\mathrm{disc}}(i,\tau)\ +\ \delta^{\mathrm{meas}}(i,\tau),
\]
where \(\delta^{\mathrm{alg}}\) is the theoretical Mirror/Transfer–Collapse mismatch, \(\delta^{\mathrm{disc}}\) the discretization error, and \(\delta^{\mathrm{meas}}\) the numerical/estimation error. The per\hyp window pipeline budget is \(\Sigma\delta(i)=\sum_{U\in W}\delta(i,\tau)\) and must satisfy \(\mathrm{gap}_\tau>\Sigma\delta(i)\) to pass B\hyp Gate\({}^{+}\) (Chapter~1).
\end{declaration}

\subsection*{6.8. Gate template (per step, per window) and saturation usage}\label{subsec:gate-template}
The following operational template is used for each A\hyp side step within a fixed domain window \(W=[u,u')\) and a fixed collapse threshold \(\tau>0\):
\begin{enumerate}
  \item \emph{Apply step \(U\) and collapse.} Execute \(U\) (labeled as in Definition~\ref{def:ops-catalog}), then apply \(C_\tau\); clip to \(W\) and, only if Theorem~\ref{thm:LP-adopt} holds, apply \(\mathrm{LP}_\tau\).
  \item \emph{Measure on B\hyp side single layer.} Compute \(\mathbf{T}_\tau\mathbf{P}_i(F)\), \(\mathrm{PE}_i^{\le\tau}\), spectral indicators on \(L(C_\tau F)\) under the fixed policy, and (if in scope) \(\Ext^1(\mathcal{R}(C_\tau F),k)\).
  \item \emph{Record \(\delta\).} Append \(\delta^{\mathrm{alg}},\delta^{\mathrm{disc}},\delta^{\mathrm{meas}}\) for this step to the per\hyp window ledger and update \(\Sigma\delta(i)\).
  \item \emph{Evaluate B\hyp Gate\({}^{+}\).} Use the Cascade rule (Theorem~\ref{thm:cascade}) with the windowed safety margin \(\mathrm{gap}_\tau\); require \(\mathrm{gap}_\tau>\Sigma\delta(i)\).
  \item \emph{Log verdict.} If all pass, issue a windowed certificate; otherwise, classify failure (Type~I–IV) and proceed with diagnostics (Declaration~\ref{spec:diagnostic}).
\end{enumerate}
On windows declared \emph{saturated} in the sense of Chapter~11, one may use the saturation gate as a reference to shorten step (4) (remain within its quantitative policy).

\subsection*{6.9. Windowed workflow and logging (MECE enforcement)}\label{subsec:window-workflow}
Let \(\{[u_k,u_{k+1})\}_k\) be a MECE partition (Chapter~2, Def.~\ref{def:ch2-mece}). For each window:
\begin{itemize}
  \item Fix \(\tau\) by the adaptation rule (Chapter~2, Def.~\ref{def:tau-adapt}); if spectral auxiliaries are used, fix \((\beta,[a,b])\).
  \item Run the gate template (Subsection~\ref{subsec:gate-template}) for each step; aggregate \(\Sigma\delta(i)\) and evaluate B\hyp Gate\({}^{+}\).
  \item Record coverage checks (sum of lengths; sum of events) and all parameters in the manifest (Appendix~G).
\end{itemize}
Global claims are obtained by pasting windowed certificates via Restart (Appendix~J, Lemma~J.\ref{J:lem:restart}) and Summability (Appendix~J, Definition~J.\ref{J:def:summability}); \(\tau\) is selected inside stable bands (Appendix~J, Definition~J.\ref{J:def:stab-band}). When multiple torsion reflectors are used (e.g.\ length plus birth window), apply the soft\hyp commuting policy (Chapter~5, Definition~\ref{def:soft-commute}); otherwise fix a deterministic order and record the commutation defect in \(\delta^{\mathrm{alg}}\).

\subsection*{6.10. Compliance checklist (per run)}\label{subsec:compliance}
\begin{enumerate}
  \item MECE windows recorded; coverage checks pass; if windows are definable, include their formulas and the o\hyp minimal structure used.
  \item Collapse threshold \(\tau\) adapted to resolution; spectral bin policy fixed and logged.
  \item Each step labeled (deletion/\(\varepsilon\)/inclusion) with \(1\)\hyp Lipschitz rationale; \(\delta^{\mathrm{alg}},\delta^{\mathrm{disc}},\delta^{\mathrm{meas}}\) recorded.
  \item Indicators computed on B\hyp side single layer only; B\hyp Gate\({}^{+}\) evaluated (Cascade rule, safety margin).
  \item Tower audit \(\DiagZero\) on the window; stable band identified for \(\tau\); if applicable, E\(_1\)–local gate used on definable windows (Chapter~3, Theorem~3.2).
  \item Verdict (accept/reject) and failure type logged; Restart/Summability plan updated for the next window.
\end{enumerate}

\subsection*{6.11. Summary}
This chapter specifies the operational program for geometric collapse in the implementable range. Admissible realizations (Definition~\ref{def:geom-real}) feed the persistence layer, where collapse \(C_\tau\) is applied and all indicators are computed on the B\hyp side single layer. The pipeline is standardized as
\[
C_\tau\ \to\ W_{\mathrm{clip}}\ \to\ \mathrm{LP}_\tau\ \ (\text{optional, only if Theorem~\ref{thm:LP-adopt} holds}).
\]
Deletion\hyp type steps are \emph{non\hyp increasing} after collapse (and optional safe low\hyp pass); \(\varepsilon\)–continuations are \emph{stable}. Mirror/Transfer non\hyp commutation with collapse is \emph{externalized} via a \(\delta\)–ledger and accumulated additively along pipelines in a fixed commutative quantale. Windowed certificates are issued per MECE window by B\hyp Gate\({}^{+}\); the Gate Cascade (Theorem~\ref{thm:cascade}) organizes decisions as \(E_1\to(\mu,\nu)\to\Ext^1\to\mathrm{PH}_1\). Global claims are obtained by pasting certificates using Restart and Summability, with \(\tau\) selected inside stable bands. The soft\hyp commuting policy (Chapter~5) governs multi\hyp axis torsions. All assertions remain confined to the persistence layer and respect the one\hyp way bridge (Chapter~3) and the tower calculus (Chapter~4).

\subsection*{6.12. Tropical Mirror/Transfer: natural 2\hyp cell and energy monotonicity (\textbf{[Spec]})}
We now specify a tropical endofunctor and its quantitative commutation with collapse; this is used solely as a \emph{post\hyp collapse comparator} and an \emph{energy monotonicity} trigger.

\begin{definition}[Tropical comparator and 2-cell bound]\label{def:trop-2cell}
\textbf{[Spec]}
Let $\Trop_\lambda^{\mathrm{pers}}:\Pers^{\mathrm{cons}}_k\to\Pers^{\mathrm{cons}}_k$ be a 1-Lipschitz
post-collapse comparator (e.g. barcode endpoint shifts/shortening) indexed by $\lambda\in(0,1]$.
Assume a natural 2-cell on the persistence layer
\[
\theta_{i,\tau}:\ \Trop_\lambda^{\mathrm{pers}}\circ \mathbf{T}_\tau
\Rightarrow
\mathbf{T}_\tau\circ \Trop_\lambda^{\mathrm{pers}}
\]
with defect bounded by $\delta_{\mathrm{trop}}(i,\tau)$ in the chosen quantale.
\end{definition}


\begin{theorem}[\textnormal{\textbf{[Spec]}} After\hyp collapse energy non\hyp increase under tropical shortening]\label{thm:trop-energy}
Under Definition~\ref{def:trop-2cell}, for each degree \(i\), window \([0,\tau]\), and \(\alpha>0\),
\[
\mathrm{PE}_{i,\alpha}^{\le \tau}\!\big(C_\tau(\mathcal{G}\circ \Trop_{\lambda'} X)\big)\ \le\ \mathrm{PE}_{i,\alpha}^{\le \tau}\!\big(C_\tau(\mathcal{G}\circ \Trop_{\lambda} X)\big)\qquad(\lambda'\le \lambda),
\]
with strict decrease whenever \(\kappa(\lambda',\lambda)<1\) acts on a positive\hyp mass subset of clipped bars. Moreover, the comparison is \(\delta_\mathrm{trop}\)\hyp controlled at the persistence layer in the sense of Appendix~L.
\end{theorem}

\begin{proof}[Proof sketch]
Energy non\hyp increase follows from the shortening proxy (Appendix~M, Theorems~M.\ref{M:thm:conv}–M.\ref{M:thm:mono-dominance}) and the fact that \(\mathbf{T}_\tau\) is \(1\)\hyp Lipschitz. The \(2\)\hyp cell bound provides a quantitative defect \(\delta_\mathrm{trop}(i,\tau)\) to be recorded in the \(\delta\)\hyp ledger.
\end{proof}

\subsection*{6.13. PF/BC after\hyp collapse comparison protocol (arithmetic comparator)}
We promote the projection\hyp formula/base\hyp change (\textbf{PF/BC}) comparison to an after\hyp collapse arithmetic protocol at fixed windows and thresholds; see Chapter~6, §§6.12–6.17 of the main text for full details, and Appendix~N for the PF/BC transport contracts. The post\hyp collapse metric drift (if any) is logged in \(\delta^{\mathrm{disc}},\delta^{\mathrm{meas}}\) for the window’s budget (Appendix~G).

\subsection*{6.14. Collapse classification and the Defect functor (Iwasawa\hyp style notation)}
As in Chapter~6, §§6.13–6.17 and Appendix~D/J, the windowed verdict \(\mathrm{Verdict}(W,\tau)\) classifies invisible failures via \((\mu_{i,\tau},\nu_{i,\tau})\), aggregates budgets in the chosen quantale, and enforces \(\mathrm{gap}_\tau>\Sigma\delta\).

\subsection*{6.15. run.yaml augmentation (synchronization with Appendix~G)}
To make tropical, definable, and quantale choices audit\hyp ready, the manifest must include:

\begin{verbatim}
quantale:
  name: "R_plus"            # e.g., R_plus, R_max, product
  op: "add"                 # add|max|product
  unit: 0.0
  order: "le"               # <= (Lawvere orientation)

definable:
  o_minimal_structure: "R_an,exp"
  window_formulae:
    - id: "W01"
      expr: "0 <= t < 1.0"
    - id: "W02"
      expr: "1.0 <= t < 2.0"

tropical:
  bins: { a: 0.0, beta: 0.02, bins: 96, boundary: "right-open" }
  kappa: { lambda_prime: 0.4, lambda: 0.7, value: 0.85 }
  two_cell_bound:
    degree: 1
    tau: 0.25
    delta: 0.010
\end{verbatim}

All other fields (windows/coverage checks, operations, persistence verdicts, spectral policies, \(\delta\)\hyp ledger) remain as specified in Appendix~G.

\noindent\emph{Policy.} The tropical comparator is a [Spec]-only, post-collapse comparator and is never used as a gate; any $2$-cell defect contributes solely to the quantale-valued $\delta$-ledger.



\section{Chapter 7: Arithmetic Layers and Iwasawa Refinement (Design)}
\addcontentsline{toc}{section}{Arithmetic Layers and Iwasawa Refinement (Design)}

\noindent\textbf{Index separation.}
The collapse obstruction \(\muc\) used in this chapter is a persistence--level diagnostic and is \emph{unrelated} to the classical Iwasawa \(\mu\)–invariant; no identity or implication between them is asserted (see also §\ref{rk:separation}).

\begin{remark}[Monotonicity convention]
Throughout this chapter we adopt the corrected monotonicity convention of
Chapter~6, Remark~\ref{rem:stability-vs-monotonicity}:
\emph{deletion--type} updates are non–increasing for spectral tails and windowed energies,
while \emph{inclusion--type} updates are only stable (non–expansive);
see Appendix~E for sufficient conditions and counterexamples.
\end{remark}

\subsection*{7.0. Standing hypotheses and admissible arithmetic realization}
All statements in this chapter are made within the \emph{constructible range}
(we identify \(\Pers^{\mathrm{ft}}_k\) with the constructible subcategory as in Chapters~2 and~6).
Fix a base field \(k\) and adopt the notation and posture of Part~I:
\(\FiltCh{k}\) denotes finite–type filtered chain complexes,
\(\mathbf{P}_i:\FiltCh{k}\to\Pers^{\mathrm{cons}}_k\) the degreewise persistence functor,
and we write \(\Ttau:=\mathbf{T}_\tau\) for the Serre (bar–deletion) reflector at scale \(\tau\ge 0\) (with \(\Ttau=\mathrm{Id}\) at \(\tau=0\)).
Its filtered lift \(C_\tau\) is used \emph{up to filtered quasi–isomorphism} (Chapter~2, §§2.2–2.3).
A fixed realization \(\Rfun:\FiltCh{k}\to D^{\mathrm{b}}(k\text{-mod})\) is \(t\)–exact.
Unless explicitly marked \textbf{[Spec]}, \emph{equalities and Lipschitz claims are asserted only at the persistence layer};
at the filtered–complex layer they hold \emph{up to filtered quasi–isomorphism}.
Endpoint conventions and the treatment of infinite bars are as in Chapter~2, Remark~\ref{rk:2-endpoints}.

Arithmetic input is organized as towers
\[
  \mathbb{T}\ :=\ \{X_t\}_{t\in I}\ \longrightarrow\ X_\infty,
\]
indexed by a directed set \(I\cup\{\infty\}\) with transition maps \(X_{t'}\to X_t\) for \(t'\ge t\) (e.g.\ norm/corestriction, specialization, level–lowering).
Typical instances include cyclotomic/ray–class towers of number fields, modular–level towers, or Selmer–complex towers.
Filtered (co)limits, when used, are computed objectwise in \([\mathbb{R},\mathsf{Vect}_k]\) and used only under the scope policy of Appendix~A (compute in the functor category and verify return to \(\Pers^{\mathrm{cons}}_k\)); no claim is made outside this regime.

\begin{remark}[Denef–Pas windows]
In arithmetic sections we \emph{adopt Denef–Pas definable height windows} (right–open, MECE) whenever possible; see Appendix~Q for the Denef–Pas framework and quantifier–elimination tools used to ensure finiteness of event sets and finite Čech depth on height slices.
All window–local audits and gates below are meant to operate on such definable windows when declared. \label{rk:DP-windows}
\end{remark}

\begin{definition}[Iwasawa tower \(\Rightarrow\) persistence; height/local–intensity pattern]\label{def:Iwasawa-tower->pers}
A \emph{classical Iwasawa tower} consists of a directed system \(\{K_t\}_{t\in I}\) of global fields (e.g.\ \(t=n\) for \(\mathbb{Z}_p\)–extensions), together with arithmetic objects \(\{A_t\}\) (e.g.\ class groups, Selmer groups, cohomology complexes with local conditions \(\mathcal{L}_t\)). We encode this data into the persistence pipeline via:
\begin{enumerate}
  \item \textbf{Filtered realization.} Choose a filtered chain model \(F_t\in\FiltCh{k}\) for each \(t\), functorially in \((K_t,A_t,\mathcal{L}_t)\), such that:
  \begin{itemize}
    \item the \emph{height} filtration \(F_t(\,\cdot\,)\) is non–decreasing in a height parameter \(h\) (conductor/level/weight);
    \item \emph{local intensity} (imposed by \(\mathcal{L}_t\)) is implemented by Serre–class reflectors on \(F_t\) (pre–collapse).
  \end{itemize}
  \item \textbf{Transitions.} Corestriction/norm/specialization maps \(A_{t'}\to A_t\) (\(t'\ge t\)) induce filtered maps \(F_{t'}\to F_t\) that are non–expansive after applying \(\mathbf{P}_i\) (interleaving metric), up to filtered quasi–isomorphism.
  \item \textbf{Collapse and persistence.} Apply \(C_\tau\) and then \(\mathbf{P}_i\) to obtain truncated persistence modules \(\Ttau\mathbf{P}_i(F_t)\) and barcodes; energies and spectra are computed on \(C_\tau F_t\).
\end{enumerate}
Thus the tower pattern is
\[
\text{Iwasawa tower } (K_t,A_t,\mathcal{L}_t)\ \longmapsto\ F_t\ \xrightarrow{\,\mathbf{P}_i\,}\ \mathbf{P}_i(F_t)\ \xrightarrow{\,\Ttau\,}\ \Ttau\mathbf{P}_i(F_t),
\]
which we call the \emph{Iwasawa\(\to\)persistence} pattern. All claims are at the persistence layer, with filtered–level statements interpreted up to filtered quasi–isomorphism.
\end{definition}

\begin{definition}[Admissible arithmetic realization]\label{def:arith-real}
An \emph{admissible arithmetic realization} is a functor
\[
\begin{aligned}\n\mathcal{A}\colon\ & \mathsf{ArithTower}\longrightarrow \FiltCh{k},\\[-0.25em]\n& \mathbb{T}\longmapsto F_\bullet=\{F_t\}_{t\in I\cup\{\infty\}},\n\end{aligned}
\]
subject to: (1) functorial non–expansiveness at persistence (with interleaving bounds \(\varepsilon_{t',t}\)); (2) finite–type preservation and objectwise filtered (co)limits; (3) realization coherence \(\Rfun(C_\tau F_t)\simeq \tau_{\ge 0}\Rfun(F_t)\) up to f.q.i.; (4) endpoint policy of Part~I.
\end{definition}

\begin{remark}[Cone extension for the tower]\label{rk:cone-extension}
We work in the filtered index category \(I\cup\{\infty\}\) with \(t\le \infty\) and \emph{cone maps} \(X_t\to X_\infty\).
The realization \(\mathcal{A}\) carries these to filtered maps \(F_t\to F_\infty\), yielding the comparison maps in Definition~\ref{def:phi-munu}, mirroring Chapter~4.
\end{remark}

% ---- §7.1–§7.17: retained with minor edits for DP windows cross-ref (omitted here for brevity in this excerpt) ----
% The following subsections reproduce the user's approved §7.1–§7.17 content verbatim,
% except for harmless cross-references to Remark~\ref{rk:DP-windows} where “definable windows” are invoked.

\subsection*{7.1. Class/Selmer visualization at the persistence layer}
\begin{definition}[Arithmetic visualization data]\label{def:vis}
Given \(\mathbb{T}\mapsto F_\bullet\) via \(\mathcal{A}\), define for each \(t\in I\) and degree \(i\):
\[
  \mathcal{B}_i(F_t):=\mathrm{bars}\big(\mathbf{P}_i(F_t)\big),\qquad
  \mathrm{PE}_i^{\le\tau}(F_t)\ \text{ as in §6.1 (evaluated on }\Ttau\mathbf{P}_i(F_t)\text{)}.
\]
\end{definition}

\begin{remark}[Spectral layer and \(\Ext^1\)–check]\label{rk:spectral-ext}
Form the normalized Hodge Laplacian \(L_i(C_\tau F_t)\) and record spectral tails/heat traces as in §6.1.
At the categorical layer, check \(\Ext^1\!\big(\Rfun(C_\tau F_t),Q\big)=0\) for \(Q\in\Qtest=\{k[0]\}\).
\end{remark}

\subsection*{7.2. Tower diagnostics and obstructions}
\begin{definition}[Tower comparison and obstruction indices]\label{def:phi-munu}
For each degree \(i\) and scale \(\tau\), the comparison map
\[
  \phi_{i,\tau}:\ \varinjlim_{t\in I}\ \Ttau\!\big(\mathbf{P}_i(F_t)\big)\ \longrightarrow\ \Ttau\!\big(\mathbf{P}_i(F_\infty)\big)
\]
yields obstruction counts
\(
  \mu_{i,\tau}:=\dim_k\ker\phi_{i,\tau},\quad
  \nu_{i,\tau}:=\dim_k\,\mathrm{coker}\,\phi_{i,\tau}
\),
with \(\muc=\sum_i\mu_{i,\tau}\), \(\nuc=\sum_i\nu_{i,\tau}\), where \(\dim_k\) denotes the \emph{generic–fiber} dimension after truncation.
\end{definition}

\begin{declaration}[Spec–Arithmetic towers (non–expansion)]\label{spec:arith-nonexp}
Index transitions are non–expansive (interleaving sense), uniformly controlled.
Under finite–type and objectwise filtered colimits (Appendix~A), each \(\phi_{i,\tau}\) is an isomorphism, hence \(\DiagZero\).
\end{declaration}

\begin{declaration}[Specification: Tower stability at the persistence layer]\label{spec:tower-stability}
For each fixed \(\tau\) and all \(i\),
\(
  \varinjlim_{t} \Ttau\mathbf{P}_i(F_t)\xrightarrow{\ \cong\ }\Ttau\mathbf{P}_i(F_\infty)
\);
thus \(\DiagZero\) at scale \(\tau\).
\end{declaration}

\begin{remark}[Excluding Type~IV under tower stability]\label{rk:exclude-typeIV}
Under Declaration~\ref{spec:tower-stability} we have \(\DiagZero\); hence Type~IV cannot occur at that scale.
\end{remark}

\begin{remark}[Failure patterns]\label{rk:fail}
If \(\DiagNonzero\), record: \emph{pure cokernel} \((\muc=0,\nuc>0)\), \emph{pure kernel} \((\muc>0,\nuc=0)\), or \emph{mixed}.
\end{remark}

\begin{example}[Toy towers at the persistence layer]\label{ex:toy-towers}
(As in the approved draft; omitted here for space.)
\end{example}

\subsection*{7.3. Non–identity with classical Iwasawa \(\mu\)}
\begin{remark}[Separation of indices]\label{rk:separation}
The persistence obstruction \(\muc\) is extracted from kernels/cokernels of \(\phi_{i,\tau}\) between \emph{truncated} persistence modules, whereas the classical Iwasawa \(\mu\) measures \(p\)–primary growth of \(\Lambda=\mathbb{Z}_p[[T]]\)–modules.
No identity or implication is asserted; any relation, if present, is programmatic and confined to \textbf{[Conjecture]} statements.
\end{remark}

\begin{remark}[Alignment conditions for \(\mu_{\mathrm{Collapse}}\) and classical \(\mu\)]\label{rem:mu-align}
Programmatic alignment may be arranged on selected windows under:
(1) windowed torsion reflection;
(2) deletion–dominance (or uniformly bounded non–expansive shifts);
(3) stability of local conditions;
see the approved draft for details.
\end{remark}


\subsection*{7.4. Program specifications for arithmetic towers}
\begin{declaration}[Specification: Admissible indexings and maps]\label{spec:indexing}
An indexing of the tower by conductor, level, or height that renders the transition maps non\hyp expansive (hence \(1\)\hyp Lipschitz under each \(\mathbf{P}_i\) in the interleaving sense of Definition~\ref{def:arith-real}(1)) is \emph{admissible}.
Under such indexings, energy and spectral indicators are stable (non\hyp expansive) in general and \emph{non\hyp increasing for deletion\hyp type steps} (Appendix~E), up to f.q.i.; no non\hyp increase is claimed for inclusion\hyp type updates.
\end{declaration}

\subsection*{7.5. Conjectural propagation along arithmetic towers}
\begin{conjecture}[AK–Arithmetic tower propagation]\label{conj:arith-prop}
Assume an admissible arithmetic realization $\mathcal{A}$ and \LC.
If, along a non\hyp expansive tower segment and for a scale interval in $\tau$, we have $\DiagZero$ and the persistence energies (deletion\hyp type: non\hyp increasing; general: stable) together with the spectral indicators are controlled as above, then the arithmetic visualization stabilizes at that scale:
the proxies registered by persistence/spectral layers remain bounded, and the categorical check
\(\Ext^1\big(\Rfun(C_\tau F_t),Q\big)=0\) persists along the segment.
No number\hyp theoretic identity, and no identification with the classical Iwasawa invariants, is asserted.
\end{conjecture}

\subsection*{7.6. Diagram and data flow}
\begin{center}
\begin{tikzcd}[
  ampersand replacement=\&,
  column sep=1.8em, row sep=1.8em,
  cells={nodes={font=\footnotesize}},
  every label/.append style={font=\scriptsize},
  scale=0.92, transform shape
]
{\text{Arithmetic tower } \mathbb{T}} \arrow[r, "\mathcal{A}"]
  \& {\{F_t\}\subset \FiltCh{k}} \arrow[r, "\mathbf{P}_i"]
    \& {\mathbf{P}_i(F_t)} \arrow[r, "\Ttau"]
       \arrow[from=1-3, to=2-3, dashed, "{\mathrm{PE}_i^{\le \tau}}"']
      \& {\Ttau\mathbf{P}_i(F_t)}
       \arrow[from=1-4, to=2-4, dashed, "{\mathrm{PE}_i^{\le \tau}}"] \\\n{\phantom{\cdot}} \& {\phantom{\cdot}}
  \& {\text{Spectral } L(C_\tau F_t)}
    \arrow[r, dashed, "{\text{heat trace on }L(C_\tau F_t)}"]
  \& {\mathrm{HT},\ \mathrm{ST}_{\beta}} \\\n{\phantom{\cdot}} \& {\Rfun(F_t)}
  \arrow[from=3-2, to=1-3, bend left=40, dashed, "\LC"]
  \arrow[from=3-2, to=3-4, dashed, "{C_\tau\ \text{ then }\Ext^1(-,Q)}"]
  \& {\phantom{\cdot}} \& {\Ext^1(\Rfun(C_\tau F_t),Q)=0} \\\n{\phantom{\cdot}} \& {\varinjlim_{t}\, \Ttau\mathbf{P}_i(F_t)}
  \arrow[from=4-2, to=4-4, "\phi_{i,\tau}"]
  \& {\phantom{\cdot}} \& {\Ttau\mathbf{P}_i(F_\infty)}
    \arrow[from=4-4, to=5-2, bend right=25, dashed, swap, "{\text{log }(\DiagZero / \DiagNonzero)}"] \\\n{\phantom{\cdot}} \& {\text{failure log (pure/mixed)}} \& {\phantom{\cdot}} \& {\phantom{\cdot}}
\end{tikzcd}
\end{center}

\subsection*{7.7. Minimal assumptions per arithmetic class (design templates)}
\begin{declaration}[Specification: Template hypotheses]\label{spec:templates}
For practical deployment, the following minimal templates ensure admissibility (one\hyp line concrete instances shown):
\begin{itemize}
  \item \textbf{(MM spaces from arithmetic data).} Index by conductor/level; realize transitions as \(1\)\hyp Lipschitz retractions between metric(\hyp measure) models (e.g.\ modular curves under level\hyp lowering with Gromov–Hausdorff \(1\)\hyp Lipschitz maps); preserve finite\hyp type per degree.
  \item \textbf{(Simplicial/complex models).} Use bounded\hyp degree subdivisions for level changes (e.g.\ barycentric refinement at fixed depth); ensure objectwise degreewise colimits; non\hyp expansiveness under each \(\mathbf{P}_i\).
  \item \textbf{(Graphs/quotients).} Sparsify while preserving normalized Laplacians and the \(1\)\hyp Lipschitz property of \(\mathcal{G}\)/\(\mathcal{A}\) (e.g.\ degree\hyp bounded sparsification of Cayley graphs); compute spectra on \(C_\tau F_t\).
\end{itemize}
\end{declaration}

\subsection*{7.8. Reproducibility and logs}
\begin{remark}[Run logs and parameters]\label{rk:logs}
For each run, log: the tower index range \(t\in[t_{\min},t_{\max}]\), the scale sweep \(\tau\in[\tau_{\min},\tau_{\max}]\) with step, spectral parameters \((\beta,M(\tau),t_{\mathrm{HT}})\), and the obstruction tuple \((\DiagZero / \DiagNonzero)\) per \(\tau\) (with failure type).
Record also the degree set used for aggregation (per\hyp degree vs.\ summed across \(i\)) to ensure consistent replays.
These logs are part of the program specification and enable exact reruns.
For end\hyp to\hyp end validation scripts and datasets (class group and Selmer rank \(0/1\) scenarios), see Chapter~12 (Test Benches), which binds the logging format here with the executable test harness.
\end{remark}

\subsection*{7.9. Final guard-rails}
\begin{remark}[Scope and non-claims]\label{rk:nonclaims}
This chapter provides a design blueprint at the persistence/spectral/categorical layers for arithmetic towers.
It does \emph{not} assert number\hyp theoretic identities or decide deep conjectures; all forward\hyp looking statements are explicitly labeled \textbf{[Conjecture]} and rely on the implementable range and \LC.
No claim of \(\mathrm{PH}_1\Leftrightarrow\Ext^1\) is made; only the one\hyp way bridge under \textup{(B1)–(B3)} from Part~I is used.
\end{remark}

% -------------------- Reinforcements: height windows, PF/BC, commutativity, δ-ledger, gates --------------------

\subsection*{7.10. Height windows (MECE), PF/BC audit, and $\delta$-naturality}
\begin{definition}[Height windows and MECE partition]\label{def:height-windows}
Let the index set \(I\) carry a \emph{height} function \(h:I\to \mathbb{R}\) (e.g.\ conductor/level/weight) that is non\hyp decreasing along transitions. A \emph{height windowing} is a MECE partition \(\{W_k=[u_k,u_{k+1})\}_k\) of the height range such that the subdiagram of indices \(\{t\in I:\ h(t)\in W_k\}\) is filtered. All audits, gates, and certificates are performed \emph{per window}.
\end{definition}

\begin{declaration}[PF/BC audit after collapse]\label{dec:pf-bc-audit}
For external comparison functors (Projection Formula/Base Change) denoted \(\mathrm{PF},\mathrm{BC}\) at the arithmetic layer, we \emph{first} pass to persistence and \emph{then} collapse:
\[
X_t\ \xrightarrow{\ \mathcal{A}\ }\ F_t\ \xrightarrow{\ \mathbf{P}_i\ }\ \mathbf{P}_i(F_t)\ \xrightarrow{\ \Ttau\ }\ \Ttau\mathbf{P}_i(F_t).
\]
Pseudonaturality and PF/BC equalities are checked \emph{after} \(\Ttau\), i.e.\ on \(\Ttau\mathbf{P}_i(F_t)\), uniformly on each window. Any mismatch is recorded in the \(\delta\)\hyp ledger as
\[
\delta^{\mathrm{alg}}_{\mathrm{PF/BC}}(i,\tau;W_k)\ :=\ d_{\mathrm{int}}\!\Big(\Ttau \mathbf{P}_i\big(\mathrm{PF/BC}\circ \mathcal{A}\big),\ \Ttau \mathbf{P}_i\big(\mathcal{A}\circ \mathrm{PF/BC}\big)\Big).
\]
Discretization and measurement contributions are added as \(\delta^{\mathrm{disc}},\delta^{\mathrm{meas}}\); the per\hyp window budget is \(\Sigma\delta(i)\) (cf.\ Chapter~5, Specification~\ref{spec:delta-ledger} and Chapter~6, Declaration~\ref{dec:delta-ledger-ch6}).
\end{declaration}

\begin{remark}[Mirror/level transfer and $\delta$-ledger]
Let \(\Mirror\) denote level transfer (e.g.\ norm, corestriction, specialization). We measure the $2$-cell defect \(\epsilon_{i,\tau}:\Mirror\circ C_\tau\Rightarrow C_\tau\circ \Mirror\) after collapse with bound \(\delta(i,\tau)\) as in Chapter~5, Definition~\ref{def:delta-2cell}, and record it in the window ledger, aggregated by the ledger operation \(\bigotimes\) (equivalently: componentwise aggregation before scalarization).
Pipeline aggregation and $1$-Lipschitz post\hyp processing follow from Proposition~\ref{prop:pipeline-budget}.
\end{remark}

\subsection*{7.11. Commutativity and pseudonaturality tests after collapse}
\begin{declaration}[Pseudonaturality verification policy]\label{dec:pseudonat}
All naturality/compatibility diagrams involving level transfer, PF/BC, or auxiliary reflectors are verified \emph{on the collapsed persistence layer} (\(\Ttau\mathbf{P}_i(F_t)\)). This avoids pre\hyp collapse torsion noise and aligns the audit with the gate posture (B\hyp side single layer).
\end{declaration}

\begin{definition}[A/B commutativity test and fallback]\label{def:ab-test}
Given two persistence\hyp level reflectors \(T_A,T_B\) (e.g.\ length\hyp threshold and birth\hyp window) we define
\[
\Delta_{\mathrm{comm}}(M;A,B)\ :=\ d_{\mathrm{int}}\!\big(T_A T_B M,\ T_B T_A M\big).
\]
On each height window \(W_k\) we run the A/B test on \(M=\Ttau\mathbf{P}_i(F_t)\). If \(\Delta_{\mathrm{comm}}\le \eta\) (tolerance), we accept \emph{soft\hyp commuting} (Chapter~5, Definition~\ref{def:soft-commute}); otherwise we fix a deterministic order (e.g.\ \(T_B\circ T_A\)), and record \(\Delta_{\mathrm{comm}}\) into \(\delta^{\mathrm{alg}}\).
\end{definition}

\begin{remark}[Nested torsions and order independence]
If the Serre classes are nested, order independence holds and no A/B test is required (Chapter~5, Proposition~\ref{prop:nested-commute}); otherwise soft\hyp commuting governs adoption.
\end{remark}

\subsection*{7.12. Gate template for arithmetic windows and saturation usage}
\begin{enumerate}
  \item \textbf{Window selection.} Choose a height window \(W_k=[u_k,u_{k+1})\) (Definition~\ref{def:height-windows}); fix \(\tau\) inside a stable band (Chapter~4, Definition~\ref{def:stable-band}).
  \item \textbf{Collapse then measure.} For each \(t\) with \(h(t)\in W_k\), compute \(\Ttau\mathbf{P}_i(F_t)\), energies \(\mathrm{PE}_i^{\le\tau}\), spectral indicators on \(L(C_\tau F_t)\), and (if in scope) \(\Ext^1(\Rfun(C_\tau F_t),k)\).
  \item \textbf{$\delta$ logging.} Audit PF/BC and Mirror transfer after collapse (Declaration~\ref{dec:pf-bc-audit}); run A/B tests (Definition~\ref{def:ab-test}); accumulate \(\Sigma\delta(i)\).
  \item \textbf{B\hyp Gate$^{+}$.} Require: \(\mathrm{PH}_1(C_\tau F_t)=0\), \(\DiagZero\) per window (Declarations~\ref{spec:tower-stability}), Ext\(^1\) pass (if checked), and safety margin \(\mathrm{gap}_\tau>\Sigma\delta(i)\).
  \item \textbf{Certificate \& paste.} Issue the window certificate; paste across windows via Restart and Summability (Chapter~4, Lemma~\ref{lem:restart}, Definition~\ref{def:summability}).
\end{enumerate}
On windows declared \emph{saturated} (Chapter~11), the gate may reference the saturation criteria directly.

\subsection*{7.13. Compliance checklist (arithmetic run)}
\begin{enumerate}
  \item Height windows form a MECE partition; coverage log recorded.
  \item \(\tau\)\hyp sweep and stable bands documented; spectral parameters fixed.
  \item PF/BC and Mirror audits executed \emph{after collapse}; \(\delta^{\mathrm{alg}},\delta^{\mathrm{disc}},\delta^{\mathrm{meas}}\) ledger complete.
  \item A/B commutativity tests run per window; soft\hyp commuting adopted or deterministic order fixed with \(\Delta_{\mathrm{comm}}\) logged.
  \item B\hyp side only measurements; B\hyp Gate$^{+}$ passed with safety margin; \(\DiagZero\) per window.
  \item Certificates issued and pasted with Restart/Summability; failure types logged if any.
\end{enumerate}


\subsection*{7.14. Mirror/Transfer on arithmetic towers: 2\hyp cell defect, Control\(\to\)Overlap Gate, additivity, and non\hyp increase (IMRN/AiM)}

\begin{definition}[Mirror/Transfer on arithmetic towers]\label{def:arith-mirror}
Let \(\Mirror:\FiltCh{k}\to\FiltCh{k}\) be a functor representing a level transfer (norm/corestriction/specialization) on arithmetic towers. We assume:
\begin{itemize}
  \item \textbf{(M1) Non\hyp expansiveness at persistence:} for all \(i\), \(d_{\mathrm{int}}(\mathbf{P}_i(\Mirror F),\mathbf{P}_i(\Mirror G))\le d_{\mathrm{int}}(\mathbf{P}_i(F),\mathbf{P}_i(G))\).
  \item \textbf{(M2) 2\hyp cell after collapse:} there exists a natural \(2\)\hyp cell \(\epsilon_{i,\tau}:\Mirror\circ C_\tau\Rightarrow C_\tau\circ \Mirror\) with uniform bound \(\delta(i,\tau)\ge 0\) in \(d_{\mathrm{int}}\), invariant under f.q.i.
\end{itemize}
\end{definition}

\begin{proposition}[\textnormal{\textbf{[Spec]}} Mirror × Collapse: additivity and non\hyp increase]\label{prop:arith-mirror-pipeline}
Let \(U_m,\dots,U_1\) be A\hyp side steps (each deletion\hyp type or \(\varepsilon\)\hyp continuation), interlaced with collapses \(C_{\tau_j}\). Under \textup{(M1)}–\textup{(M2)} and Chapter~5, Proposition~\ref{prop:pipeline-budget}, for any fixed \(\tau\) and degree \(i\),
\[
d_{\mathrm{int}}\!\Big(\Ttau\mathbf{P}_i(\Mirror(C_{\tau_m}U_m\cdots C_{\tau_1}U_1F)),\ \Ttau\mathbf{P}_i(C_{\tau_m}U_m\cdots C_{\tau_1}U_1\Mirror F)\Big)\ \le\ \sum_{j=1}^m \delta_j(i,\tau_j),
\]
and $1$\hyp Lipschitz post\hyp processing (including PF/BC comparators of §7.10) does not increase the right\hyp hand side.
\end{proposition}

\begin{proof}[Proof sketch]
Compose the natural \(2\)\hyp cells and apply the triangle inequality; use $1$\hyp Lipschitzness of \(\Ttau\) and post\hyp processors as in Chapter~5.
\end{proof}

\begin{remark}[Overlap Gate (persistence layer), recall]
We use the \emph{Overlap Gate} from Chapter~5 as the acceptance criterion that two persistence\hyp level pipelines agree up to a controlled finite defect after collapse, i.e.\ their outputs are isomorphic modulo a finite number of bars of length \(\le \tau\), with the total defect charged to the \(\delta\)\nobreakdash ledger.
\end{remark}

\begin{proposition}[\textnormal{\textbf{[Spec]}} Control\(\Rightarrow\)Overlap Gate; finite defect recorded]\label{prop:control-overlap-gate}
Assume a classical arithmetic Control Theorem on a tower segment (e.g.\ Mazur/Greenberg\hyp style) providing that the transfer map on arithmetic objects \(A_{t'}\to A_t\) is an isomorphism modulo finite kernel/cokernel uniformly on the segment. For an admissible realization \(\mathcal{A}\) and fixed \(\tau\),
the induced persistence comparison
\[
\phi_{i,\tau}:\ \varinjlim_{t}\ \Ttau\big(\mathbf{P}_i(F_t)\big)\ \longrightarrow\ \Ttau\big(\mathbf{P}_i(F_\infty)\big)
\]
is an isomorphism up to a \emph{finite} kernel/cokernel consisting of bars of length \(\le \tau\). Consequently, the Overlap Gate accepts the segment, and the total finite defect is recorded as an algebraic budget
\[
\delta^{\mathrm{alg}}_{\mathrm{Ctrl}}(i,\tau)\ :=\ \dim_k\ker\phi_{i,\tau}\ +\ \dim_k\mathrm{coker}\,\phi_{i,\tau}.
\]
The quantity \(\delta^{\mathrm{alg}}_{\mathrm{Ctrl}}(i,\tau)\) is stable under $1$\hyp Lipschitz post\hyp processing and invariant under f.q.i.\ of the filtered models.
\end{proposition}

\begin{proof}[Proof sketch]
Finite kernel/cokernel at the arithmetic layer is carried by \(\mathcal{A}\) to finite\hyp rank changes in \(F_t\). After collapse, these manifest as a finite multiset of bars of length \(\le \tau\); all other (generic\hyp fiber) summands match. The Overlap Gate accepts by definition; stability follows from $1$\hyp Lipschitzness and f.q.i.\ invariance.
\end{proof}

\begin{remark}[Window arithmetic comparator with Mirror]
Combine Proposition~\ref{prop:arith-mirror-pipeline} with Proposition~\ref{prop:control-overlap-gate} and the PF/BC audit (Declaration~\ref{dec:pf-bc-audit}) to certify window\hyp level comparators \emph{after collapse}; aggregate the budgets in the \(\delta\)\nobreakdash ledger.
\end{remark}

\subsection*{7.15. Tropical shortening at the arithmetic layer (\textbf{[Spec]})}
We encode a \emph{tropical} base contraction on arithmetic heights/regulators as a window\hyp level barcode shortener.

\begin{definition}[Tropical base contraction (\textbf{[Spec]})]\label{def:tropical}
Let \(\Trop_\lambda:\mathsf{ArithTower}\to\mathsf{ArithTower}\) be an endofunctor with parameter \(\lambda\in(0,1]\) such that the induced filtered map on \(\FiltCh{k}\) is non\hyp expansive under \(\mathbf{P}_i\) and, on each window \(W\) and threshold \(\tau\), \emph{uniformly shortens} degreewise barcodes by factor \(\kappa(\lambda',\lambda)\le 1\) (Definition~8.1 in spirit), up to f.q.i., after applying \(\Ttau\).
\end{definition}

\begin{proposition}[Window energy non\hyp increase under tropical shortening (\textbf{[Spec]})]\label{prop:tropical-energy}
Assume Definition~\ref{def:tropical}. Then for each degree \(i\) and window \(W\),
\[
\mathrm{PE}_i^{\le\tau}\big(C_\tau (\mathcal{A}\circ \Trop_{\lambda'})(X)\big)\ \le\ \mathrm{PE}_i^{\le\tau}\big(C_\tau (\mathcal{A}\circ \Trop_{\lambda})(X)\big)\qquad (\lambda'\le \lambda),
\]
with strict decrease whenever a positive portion of clipped bars are shortened by a factor \(<1\).
\end{proposition}

\begin{proof}[Proof sketch]
Shortening reduces clipped lengths inside \([0,\tau]\) up to f.q.i.; sum of clipped lengths (Definition~\ref{def:PE}) therefore decreases, cf.\ Theorem~8.1 in the geometric setting and Chapter~6, §6.1.
\end{proof}

\begin{remark}[Scope]
Tropical shortening is a \textbf{[Spec]} design tool: no number\hyp theoretic identity is invoked; it supplies a proxy to enforce monotone decay of window energies under controlled base contractions (e.g.\ level pruning).
\end{remark}

\subsection*{7.16. Weak group collapse (linear proxy) at fixed windows}
We define a window\hyp level \emph{weak group collapse} proxy that can be tested on arithmetic symmetry/transfer actions.

\begin{definition}[Barcode space and linearization]\label{def:barcode-space}
Fix a window \(W\) and threshold \(\tau\). For degree \(i\), write \(\Ttau\mathbf{P}_i(F)\cong \bigoplus_{b\in \mathcal{B}_{i,\tau}(F;W)} I_b\), and let
\[
V_{i,\tau}(W)\ :=\ \bigoplus_{b\in \mathcal{B}_{i,\tau}(F;W)} k\cdot e_b.
\]
For a groupoid \(\mathsf{Aut}(F)\) of filtered self\hyp maps at arithmetic level, any \(g\in \mathsf{Aut}(F)\) induces (after \(\Ttau\)) a linear map on \(V_{i,\tau}(W)\), well\hyp defined up to conjugacy.
\end{definition}

\begin{definition}[Weak group collapse (window\hyp level gate)]\label{def:weak-group}
Fix a finite set \(S\subset \mathsf{Aut}(F)\). We say \emph{weak group collapse holds at \((W,\tau)\)} if:
\begin{itemize}
  \item (Semi\hyp contraction) \(\mathrm{spr}(\rho_{i,\tau}(g))\le 1\) for all \(g\in S\), all \(i\) (spectral radius bound over an algebraic closure).
  \item (Bounded unipotent length) There is \(m\in\mathbb{N}\) with \((\rho_{i,\tau}(g)-I)^m=0\) for all \(g\in S\), uniformly in \(i\).
\end{itemize}
\end{definition}

\begin{proposition}[\textnormal{\textbf{[Spec]}} Acceptance criterion and stability]\label{prop:weak-group-accept}
If weak group collapse holds at \((W,\tau)\) and the tower diagnostics vanish \(\DiagZero\), then the group action is semi\hyp contractive on the bar basis \emph{after collapse}, uniformly on \(W\), and B\hyp Gate$^{+}$ may adopt weak\hyp group\hyp collapse as an \emph{auxiliary} acceptance tag. The property is invariant under f.q.i.\ and stable under $1$\hyp Lipschitz post\hyp processing.
\end{proposition}

\begin{proof}[Proof sketch]
The linear proxy abstracts window\hyp level action on truncated bars; semi\hyp contraction and bounded unipotent length ensure no growth modes survive after collapse. Tower stability excludes Type~IV across the window. Invariance and stability follow from persistence\hyp level functoriality (Chapter~5) and the $1$\hyp Lipschitz policy.
\end{proof}

\begin{remark}[IMRN/AiM posture]
Weak group collapse does \emph{not} assert group trivialization; it is a linear, persistence\hyp level proxy on a fixed window and threshold, compatible with the budgeted pipeline and local gates. It fits the acceptance\hyp test toolbox, not a global equivalence claim.
\end{remark}

\subsection*{7.17. Summary (Mirror/Tropical/Langlands extensions, budgeted and windowed)}
We have consolidated the Mirror × Collapse \(2\)\hyp cell (with additive, non\hyp increasing bounds), a window\hyp level tropical shortening \textbf{[Spec]} that enforces energy non\hyp increase at fixed \(\tau\), and a weak group collapse proxy (semi\hyp contraction + bounded unipotent length) as auxiliary gate criteria. Each tool is \emph{windowed} and \emph{after collapse}, integrated with the PF/BC comparator (§7.10), the tower Defect calculus (§7.2), and the typed verdict of Chapter~6. All deviations are recorded in the \(\delta\)\hyp ledger; arithmetic decisions remain entirely at the persistence layer and within the constructible range, with the one\hyp way bridge only (Chapter~3). This unified policy delivers an IMRN/AiM\hyp ready, auditable program for arithmetic layers and Iwasawa\hyp style refinement that requires no further reinforcement beyond implementation details in Appendices~K–O.

\subsection*{7.18. The Iwasawa Gate: tri–state alignment between \(\mu_{\mathrm{Collapse}}\) and classical \(\mu\)}
We introduce a \emph{window–local} acceptance predicate that compares the persistence obstruction \(\mu_{\mathrm{Collapse}}\) with a classical Iwasawa \(\mu\)–proxy, without asserting any identity. The gate returns one of three outcomes.

\begin{definition}[Iwasawa Gate (tri–state)]\label{def:iwasawa-gate}
Fix a Denef–Pas height window \(W=[u,u')\) (Remark~\ref{rk:DP-windows}) and a threshold \(\tau\) in a stable band (Chapter~4, Definition~\ref{def:stable-band}).
Let \(\widehat{\mu}_{\mathrm{Iw}}(W)\) denote a \emph{classical \(\mu\)–proxy} extracted from the arithmetic tower on \(W\)
(e.g.\ the \(\mathbb{Z}_p\)–length growth rate of a chosen \(\Lambda\)–torsion module restricted to \(W\), computed by standard control on the segment; no identity with \(\muc\) is assumed).
Define:
\[
\mathrm{Gate}_{\mathrm{Iw}}(W,\tau)\ \in\ \{\mathrm{Aligned},\ \mathrm{Inconclusive},\ \mathrm{Misaligned}\},
\]
by the following budgeted, post–collapse rules:
\begin{itemize}
  \item \textbf{Aligned.} On \(W\), the sign pattern and window–wise monotonic trend of \(\muc(\tau)\) match those of \(\widehat{\mu}_{\mathrm{Iw}}(W)\) within the declared tolerance \(\eta_{\mathrm{Iw}}\), and the tower diagnostics vanish: \(\DiagZero\).
  \item \textbf{Inconclusive.} Either \(\DiagNonzero\) but can be driven to zero by mesh refinement/\(\tau\)–refinement within the existing \(\delta\)–budget, or the proxy is undefined on a subwindow due to missing arithmetic inputs.
    \item \textbf{Misaligned.} The sign/monotone pattern conflicts beyond tolerance after budgeted post–processing, or \(\DiagNonzero\) persists on \(W\) (Type~IV).
\end{itemize}
All comparisons are made \emph{after collapse} on \(\Ttau\mathbf{P}_i(F_t)\), with all drifts accounted for in the quantale–valued \(\delta\)–ledger (Chapter~5, Specification~\ref{spec:delta-ledger}; Chapter~6, Declaration~\ref{dec:delta-ledger-ch6}).
\end{definition}

\begin{figure}[t]
\centering
\begin{tikzpicture}[>=latex, node distance=12mm, scale=0.8, transform shape]
\tikzstyle{box}=[draw, rounded corners, align=center, inner sep=3pt, font=\small]
\node[box] (start) {Pick Denef--Pas window \(W\)\\ and \(\tau\) in a stable band};
\node[box, below=of start] (diag0) {Compute \((\DiagZero / \DiagNonzero)\)\\ and \(\widehat{\mu}_{\mathrm{Iw}}(W)\)};
\node[box, below left=12mm and 20mm of diag0] (aligned) {\textbf{Aligned}\\ patterns match within \(\eta_{\mathrm{Iw}}\)\\ and \(\DiagZero\)};
\node[box, below=12mm of diag0] (inc) {\textbf{Inconclusive}\\ refine mesh/\(\tau\) or inputs};
\node[box, below right=12mm and 20mm of diag0] (mis) {\textbf{Misaligned}\\ conflict beyond tolerance\\ or persistent Type~IV};
\draw[->] (start) -- (diag0);
\draw[->] (diag0.west) -- ++(-13mm,0) |- (aligned.north);
\draw[->] (diag0) -- (inc);
\draw[->] (diag0.east) -- ++(13mm,0) |- (mis.north);
\end{tikzpicture}
\caption{Iwasawa Gate: tri–state alignment on a Denef–Pas window, post–collapse and budgeted.}
\label{fig:iwasawa-gate}
\end{figure}


\begin{remark}[Scope and guarantees]
The gate is \emph{diagnostic}: it does not assert any equality \(\muc=\widehat{\mu}_{\mathrm{Iw}}\).
``Aligned'' certifies window–level agreement of \emph{trends} under stable–band selection and vanishing tower defects; ``Misaligned'' flags a genuine arithmetic/persistence discrepancy or unresolved Type~IV; ``Inconclusive'' directs refinement or input completion.
\end{remark}

\subsection*{7.19. Denef–Pas windows: adoption and cross–reference}
We \emph{adopt} Denef–Pas definable height windows whenever arithmetic parameters admit such descriptions (Appendix~Q).
This yields: (i) finite event sets per bounded window; (ii) finite Čech depth for window–wise gluing; (iii) a uniform setting for PF/BC and Mirror $2$–cells \emph{after collapse}.
All invocations of ``definable windows'' in §7 are to be read in this Denef–Pas sense unless explicitly stated otherwise.

\subsection*{7.20. Test \texttt{T–Iwasawa–Alignment} (new)}
We add a machine–checkable test that implements Definition~\ref{def:iwasawa-gate}.

\begin{specification}[\texttt{T–Iwasawa–Alignment}]\label{spec:T-Iwasawa}
\emph{Inputs:}
a height window \(W=[u,u')\) (Denef–Pas definable), a threshold \(\tau\) in a stable band, a tolerance \(\eta_{\mathrm{Iw}}\in \mathbb{R}_{\ge 0}\), and a classical \(\mu\)–proxy \(\widehat{\mu}_{\mathrm{Iw}}(W)\) (if available).

\emph{Checks (all after collapse):}
\begin{enumerate}
  \item \textbf{Tower stability:} \(\DiagZero\) on \(W\) (pointwise in \(\tau\) if sweeping).
  \item \textbf{Trend comparison:} the monotone trend (non–increase/non–decrease) and sign pattern of \(\muc(\tau)\) match those of \(\widehat{\mu}_{\mathrm{Iw}}(W)\) within \(\eta_{\mathrm{Iw}}\).
  \item \textbf{Budget dominance:} safety margin \(\mathrm{gap}_\tau>\Sigma\delta(i)\) for the monitored degrees; PF/BC and Mirror $2$–cell defects included.
\end{enumerate}

\emph{Output:}
\(\mathrm{Gate}_{\mathrm{Iw}}(W,\tau)\in\{\mathrm{Aligned},\mathrm{Inconclusive},\mathrm{Misaligned}\}\) as in Definition~\ref{def:iwasawa-gate}, with the \(\delta\)–ledger excerpt and the stable–band certificate attached.

\emph{Usage:}
attach \texttt{T–Iwasawa–Alignment} to the windowed checklist (§7.13): if ``Aligned'', annotate the window certificate with an ``Iw–aligned'' tag; if ``Inconclusive'', trigger refinement; if ``Misaligned'', raise an arithmetic comparator warning.
\end{specification}

\subsection*{7.21. Amendments to the windowed checklist}
In §7.13 (Compliance checklist), insert:
``(7) If a Denef–Pas window and a \(\mu\)–proxy are declared, run \texttt{T–Iwasawa–Alignment} (Spec.~\ref{spec:T-Iwasawa}) and record the tri–state outcome and budgets.''

\subsection*{7.22. Summary addendum}
The Iwasawa Gate provides a \emph{tri–state}, window–local diagnostic linking \(\mu_{\mathrm{Collapse}}\) to a classical \(\mu\)–proxy without asserting any identity. Denef–Pas windows (Appendix~Q) supply the definability and finiteness needed for auditable, nerve–level checks; the new test \texttt{T–Iwasawa–Alignment} integrates with the existing PF/BC/Mirror comparators, stable–band selection, and the quantale \(\delta\)–ledger. All decisions remain after collapse on the B–side single layer and within the constructible range, respecting the one–way bridge of Part~I.



\section{Chapter 8: Mirror/Tropical Collapse (Weak Group Collapse)}
\addcontentsline{toc}{section}{Mirror/Tropical Collapse (Weak Group Collapse)}
\label{sec:ch8}

\noindent\textbf{Windowed policy and B-side judgement.}
All decisions in this chapter are made \emph{after collapse} on the B-side, i.e.\ on single-layer objects \(\Ttau\mathbf{P}_i(F)\); when convenient, we identify this at the persistence layer with \(\mathbf{P}_i(\Ctau F)\) under the fixed policy of Part~I (up to f.q.i.\ at the filtered level).
Filtered-complex statements are always taken \emph{up to filtered quasi-isomorphism} (f.q.i.; Appendix~B).
All equalities and Lipschitz bounds are asserted \emph{only at the persistence layer} in the constructible range.

\begin{remark}[Monotonicity convention]\label{rk:8-mono}
We adopt Chapter~6, Remark~\ref{rem:stability-vs-monotonicity}:
\emph{deletion-type} updates are non-increasing for spectral tails and windowed energies;
\emph{inclusion-type} updates are stable (non-expansive).
See Appendix~E.
\end{remark}

\subsection*{8.0. Standing hypotheses and post-collapse policy}
Fix a field \(k\) and work within the \emph{implementable range} of Part~I.
Identify \(\Pers^{\mathrm{ft}}_k\) with the constructible subcategory (Chapter~6).
Let \(\FiltCh{k}\) be finite-type filtered chain complexes, and \(\mathbf{P}_i:\FiltCh{k}\to\Pers^{\mathrm{cons}}_k\) the degreewise persistence functor.
Write \(\Ttau:=\mathbf{T}_\tau\) for the Serre (bar-deletion) reflector at scale \(\tau\ge 0\); its filtered lift \(\Ctau\) is used up to f.q.i.\ (Chapter~2, §§2.2–2.3).
A fixed \(t\)-exact realization \(\Rfun:\FiltCh{k}\to D^{\mathrm{b}}(k\text{-mod})\) is retained, and \LC\ holds whenever \(\Ctau\) is compared with \(\tau_{\ge 0}\!\circ\!\Rfun\).
Endpoint conventions and infinite bars follow Chapter~2, Remark~\ref{rk:2-endpoints}.
Kernel/cokernel diagnostics \((\mu_{\mathrm{Collapse}},\nu_{\mathrm{Collapse}})\) at scale \(\tau\) are computed from comparison maps as in Chapter~4, §4.2, with \(\dim_k\) interpreted as generic-fiber dimension after truncation (multiplicity of \(I[0,\infty)\)); see Appendix~D, Remark~\ref{D:rem:generic-dim}.

\medskip
\noindent\textbf{Quantitative commutation and the product-ledger quantale (P7).}
We fix a commutative unital quantale \((\mathsf{Q},\otimes,\mathbf{1},\le)\) to aggregate \(2\)-cell defects.
The \emph{product-ledger} policy is standard:
budgets \(\delta\in\mathsf{Q}\) compose multiplicatively via \(\otimes\) along pipelines and are compared using \(\le\).
Default: \(\mathsf{Q}=\overline{\mathbb{R}}_{\ge 0}\) with \(\otimes=+\), \(\mathbf{1}=0\) (Lawvere order).
A tolerance \(\eta\in\mathsf{Q}\) is fixed per window.

\begin{declaration}[Post-collapse Non-expansion Policy (P4)]\label{dec:post-collapse-policy}
All comparisons, gates, and error budgets are evaluated \emph{after} applying \(\Ttau\).
Let \(L:\FiltCh{k}\to\FiltCh{k}\) be non-expansive degreewise for each \(\mathbf{P}_i\) and admit a natural \(2\)-cell
\(\theta^L:L\circ C_\tau\Rightarrow C_\tau\circ L\) (up to f.q.i.).
Then, for all degrees \(i\),
\[
  d_{\mathrm{int}}\!\big(\Ttau\mathbf{P}_i(LF),\,\Ttau\mathbf{P}_i(LG)\big)\ \le\ d_{\mathrm{int}}\!\big(\mathbf{P}_i(F),\,\mathbf{P}_i(G)\big),
\]
and for the same \(F\),
\[
  d_{\mathrm{int}}\!\big(\Ttau\mathbf{P}_i\big((L\!\circ\! C_\tau)F\big),\,\Ttau\mathbf{P}_i\big((C_\tau\!\circ\! L)F\big)\big)\ \le\ \delta^L(i,\tau)\in\mathsf{Q}.
\]
All bounds are one-sided ``\(\le\)'' (safe-side only). All equalities are at persistence; filtered-level statements hold up to f.q.i.
\end{declaration}

\begin{remark}[No pre-collapse observation]\label{rk:8-no-pre}
Intermediate pre-collapse observations are not used for decisions.
Only post-collapse quantities \(\Ttau\mathbf{P}_i(-)\) and their indicators enter gates and budgets.
\end{remark}

We consider admissible realizations (Chapter~6, Definition~\ref{def:geom-real}; Chapter~7, Definition~\ref{def:arith-real})
\[
  \mathsf{Geom}_A \xrightarrow{\ \mathcal{G}_A\ } \FiltCh{k},\qquad
  \mathsf{Geom}_B \xrightarrow{\ \mathcal{G}_B\ } \FiltCh{k}.
\]
A \emph{tropical base contraction} at parameter \(\lambda\in(0,1]\) is an endofunctor
\(\Trop_\lambda:\mathsf{Geom}_A\to \mathsf{Geom}_A\)
whose induced filtered map on \(F:=\mathcal{G}_A(X)\) is non-expansive under each \(\mathbf{P}_i\) and monotone (deletion-type) as \(\lambda\searrow 0\).
A \emph{mirror transfer} is a functor \(\Mirror:\FiltCh{k}\to\FiltCh{k}\) that is non-expansive for each \(\mathbf{P}_i\), compatible with \(\Ctau\) up to f.q.i., and subject to \LC\ for comparisons after realization \(\Rfun\).

\begin{remark}[Endpoints and infinite bars]\label{rk:8-endpoints}
All statements are insensitive to open/closed endpoints. Infinite bars are not removed by \(\Ttau\); windowed indicators clip their contributions (Chapter~6).
\end{remark}

\begin{remark}[Cone extension for the tropical flow]\label{rk:8-cone}
For a directed parameter set \(\Lambda\subset(0,1]\) with \(\lambda'\le \lambda\), adjoin a terminal element \(\lambda_\ast\) (representing \(\lambda\to 0\)) and cone maps \(\Trop_{\lambda}\Rightarrow \Trop_{\lambda_\ast}\).
Under \(\mathcal{G}_A\), these induce filtered maps \(F_\lambda\to F_{\lambda_\ast}\), providing the comparison maps used to compute \((\mu_{\mathrm{Collapse}},\nu_{\mathrm{Collapse}})\) at fixed \(\tau\) along the \(\lambda\)-tower (Chapter~4).
\end{remark}

\subsection*{8.1. Tropical contraction and barcode shortening}
\begin{definition}[Uniform shortening proxy \textbf{[Spec]}]\label{def:shorten}
Let \(F\in\FiltCh{k}\) and fix \(\tau\ge 0\).
A filtered map \(F\to F'\) \emph{uniformly shortens} degreewise barcodes at factor \(\kappa\in(0,1]\) \emph{up to f.q.i.} if, for every \(i\), the multiset of lengths in \(\Ttau\mathbf{P}_i(F')\) is obtained from that of \(\Ttau\mathbf{P}_i(F)\) by multiplying lengths by \(\le \kappa\) and possibly deleting some bars, modulo f.q.i.
Infinite bars are unaffected; shortening is enforced only within the monitored window \([0,\tau]\) after \(\Ttau\).
The factor may depend on \(i,\tau\) (write \(\kappa_i(\tau)\)).
\end{definition}

\begin{declaration}[Specification: Tropical reduction vs.\ barcode shortening]\label{spec:trop-short}
Within the implementable range, \(\Trop_{\lambda'\le\lambda}\) induces filtered maps
\[
  \mathcal{G}_A(\Trop_{\lambda'}X)\longrightarrow \mathcal{G}_A(\Trop_{\lambda}X)
\]
that uniformly shorten degreewise barcodes at a factor \(\kappa(\lambda',\lambda)\le 1\) up to f.q.i.
Consequently, for fixed \(\tau\), the truncated energies \(\mathrm{PE}_i^{\le\tau}\) are non-increasing along \(\lambda\searrow 0\), and strictly decrease whenever \(\kappa(\lambda',\lambda)<1\) on a subset of bars whose cumulative length within the \(\tau\)-window has positive proportion of the total windowed bar length.
\end{declaration}

\subsection*{8.2. Weak group collapse: linear proxies on automorphism groupoids}
\begin{definition}[Automorphism groupoid and linear proxies]\label{def:group-proxy}
Let \(\mathsf{Aut}(F)\) be the groupoid of filtered self-maps of \(F\) in \(\FiltCh{k}\).
Fix \(\tau\) and \(i\).
Choose an interval-decomposition (up to f.q.i.) of \(\Ttau\mathbf{P}_i(F)\cong\bigoplus_{b\in \mathcal{B}_{i,\tau}(F)} I_b\).
Define the barcode vector space \(V_{i,\tau}:=\bigoplus_{b\in \mathcal{B}_{i,\tau}(F)} k\cdot e_b\).
Any \(g\in \mathsf{Aut}(F)\) induces an automorphism of \(\Ttau\mathbf{P}_i(F)\), hence (after choosing a decomposition) a linear map on \(V_{i,\tau}\), well-defined up to conjugacy:
\[
  \rho_{i,\tau}:\ \mathsf{Aut}(F)\to \mathsf{GL}(V_{i,\tau}).
\]
For a finite generating set \(S\subset \mathsf{Aut}(F)\) set
\[
  \rho_{\max,i,\tau}(S):=\sup_{g\in S}\ \mathrm{spr}\big(\rho_{i,\tau}(g)\big),\qquad
  \mathrm{nilp}_{i,\tau}(S):=\min\{m\ge 0\mid (\rho_{i,\tau}(g)-I)^m=0\ \forall g\in S\}.
\]
We say \(F\) has \emph{weak group collapse (WGC) at scale \(\tau\)} if for some finite \(S\),
\(\rho_{\max,i,\tau}(S)\le 1\) for all \(i\) and \(\sup_i \mathrm{nilp}_{i,\tau}(S)<\infty\).
\end{definition}

\begin{remark}[Meaning and invariance]\label{rk:group-meaning}
“Non-expansive’’ means the induced maps on \(\Ttau\mathbf{P}_i(\cdot)\) are \(1\)-Lipschitz for \(d_{\mathrm{int}}\) at the fixed \(\tau\).
The numbers \(\rho_{\max,i,\tau}(S),\mathrm{nilp}_{i,\tau}(S)\) are conjugacy invariants, well-defined up to f.q.i.
\end{remark}

\begin{remark}[Auxiliary tag]\label{rk:wgc-aux}
WGC is a \emph{persistence-level linear proxy} and used \emph{only as an auxiliary tag} on a fixed window \([0,\tau]\).
It is \emph{not} a gate criterion.
\end{remark}

\subsection*{8.3. Post-collapse transfer, Mirror, and δ-budget}
\begin{declaration}[Non-expansion and 2-cell bounds (post-collapse)]\label{spec:mirror}
Let \(L:\FiltCh{k}\to\FiltCh{k}\) be non-expansive degreewise for each \(\mathbf{P}_i\) and admit a natural transformation \(\epsilon_{i,\tau}:L\circ C_\tau \Rightarrow C_\tau\circ L\) up to f.q.i., with persistence-level bound \(\delta^L(i,\tau)\in\mathsf{Q}\).
Then for all \(i\),
\[
  d_{\mathrm{int}}\!\Big(\Ttau\mathbf{P}_i\big((L\!\circ\! C_\tau)F\big),\ \Ttau\mathbf{P}_i\big((C_\tau\!\circ\! L)F\big)\Big)\ \le\ \delta^L(i,\tau).
\]
Post-processing by \(1\)-Lipschitz persistence maps (degree projections \(\mathbf{P}_i\), shifts \(S^\varepsilon\), further truncations \(\mathbf{T}_{\tau'}\)) does not increase this bound.
\end{declaration}

\begin{theorem}[Pipeline error budget (product-ledger, P7)]\label{thm:pipe-budget-8}
Let \(U_m,\dots,U_1\) be A-side steps (each deletion-type or \(\varepsilon\)-continuation) with interleaved collapses \(C_{\tau_j}\).
Let \(L\) be non-expansive with \(2\)-cells \(\epsilon_{i,\tau_j}\) bounded by \(\delta^L(i,\tau_j)\in\mathsf{Q}\).
Fix a B-side threshold \(\tau\) and degree \(i\).
Then
\[
d_{\mathrm{int}}\!\Big(\Ttau \mathbf{P}_i\big(L(C_{\tau_m}U_m\cdots C_{\tau_1}U_1F)\big),\ \Ttau \mathbf{P}_i\big(C_{\tau_m}U_m\cdots C_{\tau_1}U_1(LF)\big)\Big)\ \le\ \bigotimes_{j=1}^m \delta^L(i,\tau_j).
\]
Under the default \(\mathsf{Q}=\overline{\mathbb{R}}_{\ge 0}\) this reads ``\(\le \sum_{j}\delta^L(i,\tau_j)\)''.
\end{theorem}

\begin{remark}[Soft-commuting and \(\delta^{\mathrm{alg}}\) accounting]\label{rk:soft-comm-8}
For reflectors \(T_A,T_B\), test
\(\Delta_{\mathrm{comm}}:=d_{\mathrm{int}}(T_AT_BM,\ T_BT_AM)\) on \(M=\Ttau\mathbf{P}_i(F)\) per window/degree.
If \(\Delta_{\mathrm{comm}}\le \eta\), adopt soft-commuting; else fix an order and add \(\Delta_{\mathrm{comm}}\) to \(\delta^{\mathrm{alg}}(i,\tau)\in\mathsf{Q}\).
\end{remark}

\begin{remark}[Mirror bounds (safe-side, P4)]\label{rk:mirror-quant}
For \(L=\Mirror\) with \(2\)-cell bound \(\delta^{\mathrm{Fun}}(i,\tau)\in\mathsf{Q}\),
\[
  d_{\mathrm{int}}\!\big(\Ttau\mathbf{P}_i((\Mirror\!\circ\! C_\tau)F),\ \Ttau\mathbf{P}_i((C_\tau\!\circ\!\Mirror)F)\big)\ \le\ \delta^{\mathrm{Fun}}(i,\tau).
\]
For \(F,G\),
\[
  d_{\mathrm{int}}\!\big(\Ttau\mathbf{P}_i((\Mirror\!\circ\! C_\tau)F),\ \Ttau\mathbf{P}_i((C_\tau\!\circ\!\Mirror)G)\big)
  \ \le\ d_{\mathrm{int}}\!\big(\mathbf{P}_i(F),\,\mathbf{P}_i(G)\big) \ \oplus\ \delta^{\mathrm{Fun}}(i,\tau),
\]
where \(\oplus\) denotes \(\otimes\) in \(\mathsf{Q}\) (sum under the default).
\end{remark}

\begin{conjecture}[Mirror correspondences under collapse monitoring]\label{conj:mirror}
Assuming \(\DiagZero\) and \LC, mirror correspondences preserve the monitored indicators across \(\Mirror\) on the same \(\tau\)-range.
WGC (Definition~\ref{def:group-proxy}) propagates as an \emph{auxiliary tag}.
\end{conjecture}

\subsection*{8.3.1.\ Spec–Saturation gate (tropical/mirror)}
\begin{declaration}[Saturation gate \textbf{[Spec]} (tropical/mirror)]\label{gate:8-saturation}
Fix \(\tau^\ast>0\), tolerance \(\eta>0\), and edge gap
\(\mathrm{Gap}:=\tau^\ast-\max\{\, b_r<\tau^\ast \,\}>0\).
On \([0,\tau^\ast]\), assume:
(i) eventually the maximal \emph{finite} bar length in \(\T_{\tau^\ast}\mathbf{P}_i(F_\lambda)\) is \(\le \eta\);
(ii) eventually \(d_{\mathrm{int}}(\T_{\tau^\ast}\mathbf{P}_i(F_\lambda),\T_{\tau^\ast}\mathbf{P}_i(F_{\lambda'}))\le \eta\);
(iii) \(\mathrm{Gap}>\eta\).
Then, \textbf{within this window only}, temporarily adopt
\[
\mathrm{PH}_1(\C_{\tau^\ast}F_\lambda)=0\ \Longrightarrow\ \Ext^1(\Rfun(\C_{\tau^\ast}F_\lambda),k)=0,
\]
and within this window we adopt the \emph{paired check} \(\mathrm{PH}_1=0\) and \(\Ext^1=0\) as a \textbf{[Spec]} operational criterion; any mismatch is logged and triggers refinement.

Quantitative verification and usage are centralized in Chapter~11.
\end{declaration}

\subsection*{8.4. Monitoring protocol for tropical/mirror flows}
\begin{declaration}[Specification: Monitoring protocol]\label{spec:prot}
Fix a sweep \(\lambda\searrow 0\) and finite scales \(\tau\).
For each sample:
\begin{enumerate}
  \item Compute \(\Ttau\mathbf{P}_i(\mathcal{G}_A(\Trop_\lambda X))\) and \(\mathrm{PE}_i^{\le\tau}\) on truncated barcodes (equivalently on \(\Ctau\)).
  \item Compute spectral indicators \(\mathrm{ST}_{\beta}^{\ge M(\tau)}\), \(\mathrm{HT}(t;\cdot)\) on \(L(\Ctau(\mathcal{G}_A(\Trop_\lambda X)))\) with fixed \((\beta,M(\tau),t)\).
  \item Check \(\Ext^1(\Rfun(\Ctau{-}),Q)=0\) for \(Q\in\Qtest\).
  \item Evaluate \((\mu_{\mathrm{Collapse}},\nu_{\mathrm{Collapse}})\) from the \(\lambda\)-tower at \(\tau\) (Remark~\ref{rk:8-cone}).
  \item (Auxiliary) Choose finite \(S\subset\mathsf{Aut}(\mathcal{G}_A(\Trop_\lambda X))\) and record \(\rho_{\max,i,\tau}(S)\), \(\mathrm{nilp}_{i,\tau}(S)\) on \(V_{i,\tau}\) (WGC-tag).
  \item Apply \(\Mirror\) and repeat (1)–(5); compare via the \(\delta^{\mathrm{Fun}}(i,\tau)\) budget.
\end{enumerate}
The \emph{stable regime} at \(\tau\) is where \(\DiagZero\) and (1)–(3) are non-increasing along \(\lambda\searrow 0\).
\end{declaration}

\subsection*{8.5. Diagram (post-collapse indicators and mirror transfer)}
\begin{center}
% spacing knobs
\def\hA{5.0em}\def\hC{5.0em}\def\hB{5.0em}\def\vA{5.0em}\def\vB{9.0em}
\begin{tikzpicture}[baseline,>=latex]
\tikzset{box/.style={draw,rounded corners,inner sep=4pt},
         widebox/.style={box,inner xsep=6pt,inner ysep=4pt,minimum width=8em,align=center},
         lblUp/.style={midway,above=0.45em},
         lblDn/.style={midway,below=0.45em}}
\node (X) {$X$};
\node (F)       [right=\hA of X,box] {$F=\mathcal{G}_A(X)$};
\node (PiRaw)   [right=\hA of F,box] {$\mathbf{P}_i(F)$};
\node (PiF)     [right=\hC of PiRaw,box] {$\Ttau\mathbf{P}_i(F)$};
\node (MirPiF)  [right=\hB of PiF,box] {$\Ttau\mathbf{P}_i(\Mirror F)$};
\node (PEF)     [below=\vA of PiF,widebox]   {$\mathrm{PE}_i^{\le\tau}$\\$\mathrm{ST}_\beta,\ \mathrm{HT}$};
\node (PEMir)   [below=\vA of MirPiF,widebox]{$\mathrm{PE}_i^{\le\tau}$\\$\mathrm{ST}_\beta,\ \mathrm{HT}$};
\node (Ext)     [below=\vB of PiF,box] {$\Ext^1(\Rfun(\Ctau F),Q)$};
\draw[->,shorten >=0.2em,shorten <=0.2em] (X) -- node[lblUp] {$\mathcal{G}_A$} (F);
\draw[->,shorten >=0.2em,shorten <=0.2em] (F) -- node[lblUp] {$\mathbf{P}_i$} (PiRaw);
\draw[->,shorten >=0.2em,shorten <=0.2em] (PiRaw) -- node[lblDn] {$\Ttau$} (PiF);
\draw[->,shorten >=0.2em,shorten <=0.2em] (PiF) -- node[lblUp] {$\Mirror$} (MirPiF);
\draw[->,dashed,shorten >=0.2em,shorten <=0.2em] (PiF) -- (PEF);
\draw[->,dashed,shorten >=0.2em,shorten <=0.2em] (MirPiF) -- (PEMir);
\draw[->,dashed,shorten >=0.2em,shorten <=0.2em] (F) to[bend right=18] node[left] {$\Ctau$} (Ext);
\end{tikzpicture}
\end{center}

\subsection*{8.6. Toy instance (persistence layer)}
\begin{example}[Uniform shortening under tropical scaling]\label{ex:trop}
Let \(\mathbf{P}_i(F)\) have lengths \(\{\ell_j\}_j\) in \([0,\tau]\).
Suppose \(\Trop_\lambda\) induces \(\ell_j\mapsto \ell'_j\le \kappa\ell_j\) for a fixed \(\kappa<1\) on a subset \(S\) whose cumulative \(\tau\)-clipped length is positive, and possibly deletes other bars.
Then \(\mathrm{PE}_i^{\le\tau}\) strictly decreases.
If additionally \(\DiagZero\) at \(\tau\), one may annotate with a WGC-tag (Definition~\ref{def:group-proxy}); this tag is not used for gating.
\end{example}

\subsection*{8.7. Final guard-rails}
\begin{remark}[Scope and non-claims]\label{rk:8-guard}
All claims are at the persistence/spectral/categorical layers in the implementable range, with \LC\ when comparing after realization.
No group-theoretic trivialization is asserted; WGC is an auxiliary tag.
No claim of \(\mathrm{PH}_1\Leftrightarrow\Ext^1\) is made; only the one-way bridge of Part~I is used.
The obstruction \(\mu_{\mathrm{Collapse}}\) is unrelated to the classical Iwasawa \(\mu\)-invariant.
\end{remark}

\subsection*{8.8. Langlands tri-layer gates (Galois $\to$ Transfer $\to$ Functorial)}
We formalize gate criteria across a three-layer pipeline, each verified \emph{after collapse} on \(\Ttau\mathbf{P}_i(-)\), window by window.

\begin{definition}[Layer functors and objects]\label{def:layers}
Let \(F\in\FiltCh{k}\) and fix \(\tau\).
\begin{enumerate}
  \item Galois layer: a subgroup \(G\subset \mathsf{Aut}(F)\) acts by filtered maps. After collapse, \(G\) acts on \(V_{i,\tau}\) via \(\rho_{i,\tau}\) (Definition~\ref{def:group-proxy}).
  \item Transfer layer: a finite family \(\Trans=\{T_a:\FiltCh{k}\to\FiltCh{k}\}_a\) of non-expansive transfers (norm, corestriction, Hecke, BC/PF adapters), each with a \(2\)-cell \(T_a\circ C_\tau\Rightarrow C_\tau\circ T_a\) bounded by \(\delta^{\mathrm{Tr}}_a(i,\tau)\in\mathsf{Q}\).
  \item Functorial layer: a non-expansive \(\Funct:\FiltCh{k}\to\FiltCh{k}\) (e.g.\ Mirror, Langlands lift) with a \(2\)-cell \(\Funct\circ C_\tau\Rightarrow C_\tau\circ \Funct\) bounded by \(\delta^{\mathrm{Fun}}(i,\tau)\in\mathsf{Q}\).
\end{enumerate}
\end{definition}

\begin{definition}[Layer collapse maps and kernels]\label{def:layer-kernels}
For each \(i,\tau\):
\begin{itemize}
  \item Galois kernel: for \(S\subset G\) finite,
  \(
  \phi^{\mathrm{Gal}}_{i,\tau}(g):=\Ttau\mathbf{P}_i(g)-\mathrm{Id},
  \quad
  \mu^{\mathrm{Gal}}_{i,\tau}(S):=\dim_k\bigcap_{g\in S}\ker\phi^{\mathrm{Gal}}_{i,\tau}(g).
  \)
  \item Transfer kernel: for each \(T_a\),
  \(
  \phi^{\mathrm{Tr}}_{i,\tau}(a):\Ttau\mathbf{P}_i(F)\to \Ttau\mathbf{P}_i(T_a F),\ 
  \mu^{\mathrm{Tr}}_{i,\tau}(a):=\dim_k\ker\phi^{\mathrm{Tr}}_{i,\tau}(a).
  \)
  \item Functorial kernel:
  \(
  \phi^{\mathrm{Fun}}_{i,\tau}:\Ttau\mathbf{P}_i(F)\to \Ttau\mathbf{P}_i(\Funct F),\ 
  \mu^{\mathrm{Fun}}_{i,\tau}:=\dim_k\ker\phi^{\mathrm{Fun}}_{i,\tau}.
  \)
\end{itemize}
Define cokernels by \(\nu^{L}_{i,\tau}=\dim_k\mathrm{coker}(\phi^L_{i,\tau})\), \(L\in\{\mathrm{Gal},\mathrm{Tr},\mathrm{Fun}\}\).
All counts are invariant under f.q.i.\ and cofinal reindexing.
\end{definition}

\begin{declaration}[Layer gates (windowed)]\label{gate:tri-layer}
Fix a window and \(\tau\).
We accept a layer when:
\begin{itemize}
  \item Galois gate: \(\mu^{\mathrm{Gal}}_{i,\tau}(S)=0\) for all monitored \(i\).
        (WGC, if present, is \emph{recorded} but not required.)
  \item Transfer gate: for all \(T_a\), \(\mu^{\mathrm{Tr}}_{i,\tau}(a)=0\) and \(\bigotimes_a \delta^{\mathrm{Tr}}_a(i,\tau)\ \le\ \eta\).
  \item Functorial gate: \(\mu^{\mathrm{Fun}}_{i,\tau}=0\) and \(\delta^{\mathrm{Fun}}(i,\tau)\le \eta\).
\end{itemize}
The \emph{tri-layer gate} passes when all three pass and \(\DiagZero\) at \(\tau\); energies/spectral tails are non-increasing within the window.
Under the default \(\mathsf{Q}\), ``\(\le \eta\)'' reads ``sum of bounds \(<\) edge gap'' (Chapter~11).
\end{declaration}

\begin{remark}[Rationale]\label{rk:tri-rationale}
The Galois gate forbids nontrivial fixed sectors after collapse;
the transfer gate forbids losses invisible to energy/spectral proxies but detected by \(\ker\phi^{\mathrm{Tr}}\);
the functorial gate controls changes under functorial lifts via \(\ker\phi^{\mathrm{Fun}}\) with budget control.
\end{remark}

\subsection*{8.9. Type IV failure (μ\_Collapse–based) — layerwise visibility}
\begin{definition}[Visibility and Type IV codes]\label{def:type-iv}
At fixed \(\tau\) and window:
\begin{itemize}
  \item Visible failure if any energy/spectral indicator exceeds tolerance beyond the \(\delta\)-budget.
  \item Invisible failure if indicators stay within budget but some \(\mu^L_{i,\tau}+\nu^L_{i,\tau}>0\), \(L\in\{\mathrm{Gal},\mathrm{Tr},\mathrm{Fun}\}\).
\end{itemize}
Record layerwise codes \(\mathrm{IV}\text{-}L[\mathrm{vis}/\mathrm{inv}]\), \(L\in\{\mathrm{Gal},\mathrm{Tr},\mathrm{Fun}\}\).
The \emph{global} Type~IV flag follows the canon diagnostic: Type~IV at \(\tau\) iff \(\DiagNonzero\).
If \(\DiagZero\) but some layer has \(\mu^L+\nu^L>0\), record it as an \emph{auxiliary layer-defect} (not Type~IV) and treat it as a local warning for refinement.

\end{definition}

\begin{remark}[Transfer–kernel trigger]\label{rk:transfer-trigger}
The transfer collapse kernel is gate-decisive: \(\mu^{\mathrm{Tr}}_{i,\tau}(a)=0\) for all \(a\) is necessary to avoid \(\mathrm{IV}\text{-}\mathrm{Tr}[\mathrm{inv}]\).
\end{remark}

\subsection*{8.10. δ-ledger split by layers and cross-layer soft-commuting}
\begin{definition}[Layered product-ledger]\label{def:layer-delta}
We refine the ledger by
\[
\delta(i,\tau)\ =\ \delta^{\mathrm{alg}}(i,\tau)\ \otimes\ \delta^{\mathrm{disc}}(i,\tau)\ \otimes\ \delta^{\mathrm{meas}}(i,\tau),
\]
with the algebraic bin factorized as
\[
\delta^{\mathrm{alg}}(i,\tau)\ =\ \delta^{\mathrm{Gal}}(i,\tau)\ \otimes\ \Big(\bigotimes_{a}\delta^{\mathrm{Tr}}_a(i,\tau)\Big)\ \otimes\ \delta^{\mathrm{Fun}}(i,\tau)\ \in\ \mathsf{Q}.
\]
Under the default \(\mathsf{Q}\) this is ordinary addition split by layers.
\end{definition}

\begin{declaration}[Cross-layer soft-commuting]\label{dec:cross-soft}
For reflectors \(T_A,T_B\) and a layer functor \(L\in\{g\in G,\ T_a,\ \Funct\}\),
test
\[
\Delta_{\mathrm{comm}}(M;A,B|L)\ :=\ d_{\mathrm{int}}\!\big(T_A T_B (L\cdot M),\ T_B T_A (L\cdot M)\big)
\]
on \(M=\Ttau\mathbf{P}_i(F)\).
If \(\Delta_{\mathrm{comm}}\le \eta\), adopt soft-commuting; else fix an order and ledger \(\Delta_{\mathrm{comm}}\) into the corresponding layer bin of \(\delta^{\mathrm{alg}}\).
\end{declaration}

\subsection*{8.11. Tri-layer diagram and gate placement}
\begin{center}
\def\hstep{4.0em}\def\vstep{4.0em}\def\padL{1.0em}\def\padR{2.0em}
\begin{tikzpicture}[baseline,>=latex,node distance=2.0em]
\tikzset{box/.style={draw,rounded corners,inner sep=3pt,minimum width=10em,align=center}, ok/.style={inner sep=2pt}}
\node (M0) [box] {$\Ttau\mathbf{P}_i(F)$};
\node (G1) [right=\hstep of M0,box] {$\Ttau\mathbf{P}_i(F)$};
\node (G2) [right=\hstep of G1,box] {$\Ttau\mathbf{P}_i(F)$};
\node (Gok) [right=\hstep of G2,ok] {$\checkmark_{\mathrm{Gal}}$};
\draw[->,dashed,shorten >=0.2em,shorten <=0.2em] (M0) -- node[above] {$g\in G$} (G1);
\draw[->,dashed,shorten >=0.2em,shorten <=0.2em] (G1) -- node[above] {$\phi^{\mathrm{Gal}}_{i,\tau}(g)$} (G2);
\draw[->,dashed,shorten >=0.2em,shorten <=0.2em] (G2) -- node[above] {Gate: \(\mu^{\mathrm{Gal}}=0\)} (Gok);
\node (T1) [below=\vstep of G1,box] {$\Ttau\mathbf{P}_i(T_a F)$};
\node (T2) [right=\hstep of T1,box] {$\Ttau\mathbf{P}_i(T_a F)$};
\node (Tok) [right=\hstep of T2,ok] {$\checkmark_{\mathrm{Tr}}$};
\draw[->,dashed,shorten >=0.2em,shorten <=0.2em] (M0) -- node[left] {$T_a$} (T1);
\draw[->,dashed,shorten >=0.2em,shorten <=0.2em] (T1) -- node[above] {$\phi^{\mathrm{Tr}}_{i,\tau}(a)$} (T2);
\draw[->,dashed,shorten >=0.2em,shorten <=0.2em] (T2) -- node[above] {Gate: \(\mu^{\mathrm{Tr}}=0\)} (Tok);
\node (F2) [below=\vstep of T2,box] {$\Ttau\mathbf{P}_i(\Funct F)$};
\node (Fok) [right=\hstep of F2,ok] {$\checkmark_{\mathrm{Fun}}$};
\draw[->,dashed,shorten >=0.2em,shorten <=0.2em] (M0) to[bend right=10] node[below] {$\Funct$} (F2);
\draw[->,dashed,shorten >=0.2em,shorten <=0.2em] (F2) -- node[above] {Gate: \(\mu^{\mathrm{Fun}}=0\)} (Fok);
\end{tikzpicture}
\end{center}

\subsection*{8.12. Integration with tropical shortening}
\begin{proposition}[Shortening, stability, and tri-layer acceptance]\label{prop:short-trilayer}
Assume on a fixed window and \(\tau\):
(i) \(\DiagZero\);
(ii) tropical shortening with factor \(\kappa<1\) on a positive-mass subset (Definition~\ref{def:shorten});
(iii) all layer functors are non-expansive with \(2\)-cell budgets recorded via \(\otimes\) in \(\mathsf{Q}\).
Then indicators are non-increasing; tri-layer acceptance (Declaration~\ref{gate:tri-layer}) holds \emph{iff} all layer kernels vanish and the ledgered budgets stay within tolerance~\(\eta\).
Any WGC evidence may be recorded as an auxiliary tag but is not required for acceptance.
\end{proposition}

\begin{proof}[Proof sketch]
(i) ensures baseline stability under truncation.
(ii) yields strict decrease of windowed energy when shortening affects a positive-mass subset.
(iii) ensures post-collapse non-expansion; acceptance reduces to kernel vanishing plus ledger tolerance.
\end{proof}

\subsection*{8.13. Operational checklist (tri-layer, windowed)}
\begin{enumerate}
  \item Fix MECE windows and a \(\tau\)-sweep; set spectral parameters and tolerances \(\eta\); initialize the layered product-ledger in \(\mathsf{Q}\).
  \item Run tropical flow \(\lambda\searrow 0\); per sample compute \(\Ttau\mathbf{P}_i\), energies, spectra, \(\Ext^1\).
  \item Compute \((\mu_{\mathrm{Collapse}},\nu_{\mathrm{Collapse}})\) at \(\tau\) along the \(\lambda\)-tower.
  \item Galois layer: evaluate \(\mu^{\mathrm{Gal}}_{i,\tau}(S)\); run soft-commute tests; optionally record WGC-tag via \(\rho_{\max},\mathrm{nilp}\).
  \item Transfer layer: for each \(T_a\), evaluate \(\mu^{\mathrm{Tr}}_{i,\tau}(a)\); record \(\delta^{\mathrm{Tr}}_a\) in \(\mathsf{Q}\); run soft-commute tests.
  \item Functorial layer: evaluate \(\mu^{\mathrm{Fun}}_{i,\tau}\); record \(\delta^{\mathrm{Fun}}\); run soft-commute tests.
  \item Gate: accept layers; accept tri-layer if all pass and \(\DiagZero\); otherwise log Type~IV codes (Definition~\ref{def:type-iv}).
\end{enumerate}

\subsection*{8.14. Notes on sheaf proxies and towers (optional \textbf{[Spec]})}
\begin{remark}[Sheaf proxies]\label{rk:8-sheaf}
If a constructible sheaf model \(\mathcal{F}\) on a geometric carrier is available, compute windowed barcodes of \(\mathrm{R}\Gamma(\mathcal{F})\) and run the same gates on \(\Ttau\mathbf{P}_i(\mathrm{Sing}(\mathcal{F}))\), remaining at the persistence layer; no new identities are claimed.
\end{remark}

\begin{remark}[Tower compatibility]\label{rk:8-tower}
All layer kernels and Type~IV codes are invariant under f.q.i.\ and cofinal reindexing (Appendix~J). This ensures pasteability across windows and levels.
\end{remark}

\subsection*{8.15. Compliance summary (IMRN/AiM posture)}
\begin{enumerate}
  \item All decisions are B-side, windowed, and ledgered; equalities only at persistence; filtered-level statements up to f.q.i.
  \item Non-expansive policies are one-sided (\(\le\)) and \(2\)-cell budgets are aggregated by the product-ledger quantale (P7).
  \item Langlands tri-layer gates expose failure loci; transfer collapse kernel passes iff zero.
  \item Type~IV classification is layerwise with visibility labels and \(\mu/\nu\) diagnostics.
  \item WGC is an auxiliary persistence-level tag; it is never a gate criterion.
\end{enumerate}



\section{Chapter 9: Langlands Collapse (Three Layers)}
\addcontentsline{toc}{section}{Langlands Collapse (Three Layers)}
\label{sec:chapter9}

\noindent\textbf{Scope note (after–collapse policy, PF/BC discipline, and comparison order).}
We work in the constructible range over a fixed field \(k\), with degreewise finite-type filtered chain complexes
\(\FiltCh{k}\), degreewise persistence \(\mathbf{P}_i:\FiltCh{k}\to\Perskft\), the exact Serre reflector \(\Ttau:=\mathbf{T}_\tau\) (bar-deletion at scale \(\tau\ge 0\)), and a filtered lift \(\Ctau\) \emph{up to f.q.i.} as in Part~I.
All \emph{comparisons, metrics, and indicators are evaluated only after collapse}: the standard operating order is
\[
\text{for each }t\quad\Longrightarrow\quad \mathbf{P}_i\quad\Longrightarrow\quad \Ttau\quad\Longrightarrow\quad \text{compare in }\Perskft.
\]
Projection Formula / Base Change (PF/BC) is used as tabulated in Appendix~N \emph{objectwise in \(t\)}, transported to persistence, and \emph{all PF/BC comparators are measured only on the collapsed layer \(\Ttau\mathbf{P}_i(-)\)}.
Residuals (non-commutation, discretization, numerics) are externalized in the \(\delta\)-ledger and, unless explicitly algebraic, are charged to \(\delta_{\disc}\) and \(\delta_{\meas}\) (see Remark~\ref{rk:9-delta-policy} and Spec.~\ref{spec:9-T-PFBC-after}).
Equalities and Lipschitz statements are asserted at persistence after truncation; at the filtered level they hold up to filtered quasi-isomorphism (f.q.i.). Endpoint conventions and infinite bars follow Chapter~2, Remark~\ref{rk:2-endpoints}.

\begin{remark}[Monotonicity convention]
\label{rem:9-monotonicity}
We adopt Chapter~6, Remark~\ref{rem:stability-vs-monotonicity}: \emph{deletion-type} updates are non-increasing for windowed energies and spectral tails after truncation, whereas \emph{inclusion-type} updates are stability-only (non-expansive). See Appendix~E.
\end{remark}

\subsection*{9.0. Standing hypotheses, Gate Cascade alignment, and definable cover}
\emph{Implementable range.} We identify \(\Perskft\) with the constructible subcategory as in Chapter~6. Fix a \(t\)-exact realization \(\Rfun:\FiltCh{k}\to D^{\mathrm{b}}(k\text{-mod})\) (amplitude \(\le 1\) in use) and the lifting–coherence hypothesis \(\LC\) when comparing \(\Ctau\) with \(\tau_{\ge 0}\!\circ\!\Rfun\).

\emph{Three data layers.}
\[
\mathsf{Gal}\ \xrightarrow{\ \mathsf{Trans}\ }\ \mathsf{Par}\ \xrightarrow{\ \mathsf{Funct}\ }\ \mathsf{Aut},
\]
heuristically “Galois \(\to\) Transfer \(\to\) Functoriality”.
An admissible Langlands triple consists of functors
\(\mathcal{L}_{\mathrm{Gal}},\mathcal{L}_{\mathrm{Tr}},\mathcal{L}_{\mathrm{Aut}}:\FiltCh{k}\to\FiltCh{k}\) satisfying:
\begin{itemize}
  \item \textbf{Non-expansiveness.} Each layer induces filtered maps whose images under every \(\mathbf{P}_i\) are \(1\)-Lipschitz for interleavings; deletion-type steps make windowed indicators non-increasing after truncation; inclusion-type steps are stability-only.
  \item \textbf{\(\Ctau\)-compatibility.} For each \(i\): \(\mathbf{P}_i(\Ctau{-})\cong \Ttau\,\mathbf{P}_i(-)\) in \(\Perskft\).
  \item \textbf{Finite-type \& (co)limits.} Degreewise finite-type outputs; degreewise filtered (co)limits computed objectwise in \([\mathbb{R},\mathsf{Vect}_k]\) and used under the scope policy of Appendix~A.
  \item \textbf{Realization coherence.} Under \(\LC\), functorially up to f.q.i., \(\Rfun(\Ctau F)\simeq \tau_{\ge 0}\,\Rfun(F)\).
\end{itemize}

\begin{definition}[Langlands Gate Cascade and tri-layer gate]
\label{def:9-gate-cascade}
Fix a right-open window \(W\) definable in the declared tame structure
(Denef--Pas on \(\mathrm{Arith}\) windows as in Chapter~7, Remark~\ref{rk:DP-windows},
and o-minimal on \(\mathrm{Geom}\) windows when declared),
 a threshold \(\tau\), and a commutative unital quantale \((\mathsf{Q},\otimes,\mathbf{1},\le)\).
(If desired, write \(\oplus:=\otimes\) to match “additive” notation under the default Lawvere case.).
The \emph{Gate Cascade} on \(W\) at scale \(\tau\) is:
\[
\text{(GC1) B–Gate+ (single layer, Ch.~1)}\ \Rightarrow\ \text{(GC2) Overlap Gate (gluing, Ch.~1/5)}\ \Rightarrow\ \text{(GC3) Tri-layer Gate (this chapter)}.
\]
\emph{Tri-layer Gate} passes on \(W\) if, after truncation and on the same window/\(\tau\):
\begin{enumerate}
  \item \textbf{Layerwise acceptance:} Each \(\ast\in\{\mathrm{Gal},\mathrm{Tr},\mathrm{Aut}\}\) passes B–Gate+ in monitored degrees.
  \item \textbf{T–PFBC–AfterCollapse:} All PF/BC comparators used between layers are checked only on \(\Ttau\mathbf{P}_i(-)\) (Spec.~\ref{spec:9-T-PFBC-after}); total residuals are within budget.
  \item \textbf{Pseudonaturality after truncation:} The inter-layer comparisons of Remark~\ref{rk:9-compare} become isomorphisms in \(\Perskft\) up to the aggregated tolerance \(\boldsymbol{\delta}_{\mathrm{tot}}\) (Remark~\ref{rk:9-delta-policy}).
\end{enumerate}
\end{definition}

\begin{remark}[Run manifest (mandatory \texttt{run.yaml} fields)]
\label{rk:9-runyaml}
\texttt{quantale:\{name,op,unit,order\}}, \texttt{definable:\{structure,window\_formulae\}}, \texttt{layered\_\(\delta\):\{\(\delta^{\Gal}\),\(\delta^{\Tr}\),\(\delta^{\Fun}\)\}}, a fixed \(\tau\), window convention, and 2-cell bounds if AWFS/2-cell auditing is enabled.
\end{remark}

\begin{remark}[Operational order (PF/BC and collapse)]
\label{rk:9-operational}
Every PF/BC step (Appendix~N) is evaluated via
\[
\text{(i) objectwise in \(t\)}\ \Longrightarrow\ \text{(ii) }\mathbf{P}_i\ \Longrightarrow\ \text{(iii) }\Ttau\ \Longrightarrow\ \text{(iv) compare in }\Perskft,
\]
and \emph{never} before collapse in metrics/energies. Same window/\(\tau\) and same \(\delta\)-policy across all checks.
\end{remark}

\begin{declaration}[Spec–derived realizations and non-expansive transfers]
\label{spec:9-derived}
Besides \(\Rfun:\FiltCh{k}\!\to\!D^{\mathrm{b}}(k\text{-mod})\), we may use
\(\mathsf{R}_{\mathrm{coh}}:\FiltCh{k}\to D^{\mathrm{b}}\!\operatorname{Coh}(\mathfrak{X})\) and
\(\mathsf{R}_{\acute{e}t}:\FiltCh{k}\to D^{\mathrm{b}}_{\mathrm{c}}(\mathfrak{Y}_{\acute{e}t},\Lambda)\) with field \(\Lambda\).
PF/BC is assumed as in Appendix~N. Normalized transfers (degree-normalized pull/push, kernel/Hecke, parabolic induction/Jacquet) are \emph{non-expansive after truncation}:
\[
d_{\mathrm{int}}\!\big(\Ttau\mathbf{P}_i(\Phi F),\Ttau\mathbf{P}_i(\Phi G)\big)\ \le\ d_{\mathrm{int}}\!\big(\Ttau\mathbf{P}_i(F),\Ttau\mathbf{P}_i(G)\big).
\]
All claims reside at persistence after truncation; the bridge \(\mathrm{PH}_1\Rightarrow\Ext^1\) is used only in \(D^{\mathrm{b}}(k\text{-mod})\).
\end{declaration}

\begin{declaration}[Deletion-type operations (PDE)]
\label{spec:9-pde}
Maps implemented by the PDE repertoire of Appendix~E (Dirichlet restriction/absorbing boundaries, p.s.d.\ Loewner contractions, principal submatrices/Schur complements) are \emph{deletion-type} and make windowed energies and spectral tails \emph{non-increasing} after truncation. Inclusion-type steps are stability-only.
\end{declaration}

\begin{remark}[Endpoints and infinite bars]
\label{rk:9-endpoints}
Open/closed endpoint conventions are immaterial; \(\Ttau\) never removes infinite bars; windowed indicators clip them (Chapter~6).
\end{remark}

\begin{remark}[Indexing and cone extension]
\label{rk:9-cone}
For a directed index \(I\), adjoin a terminal \(\infty\) and cone maps \(t\to\infty\). Under realizations these yield filtered maps \(F_t\to F_\infty\) per layer/degree and provide the comparison maps \((\mu_{\mathrm{Collapse}},\nu_{\mathrm{Collapse}})\) at fixed~\(\tau\) (Chapter~4).
\end{remark}

\begin{remark}[Inter-layer comparison and pseudonaturality]
\label{rk:9-compare}
Fix natural transformations
\[
\alpha:\ \mathcal{L}_{\mathrm{Tr}}\!\circ\!\mathsf{Trans}\Rightarrow\mathcal{L}_{\mathrm{Gal}},\qquad
\beta:\ \mathcal{L}_{\mathrm{Aut}}\!\circ\!\mathsf{Funct}\Rightarrow\mathcal{L}_{\mathrm{Tr}},
\]
which become filtered quasi-isomorphisms degreewise after applying \(\Ctau\).
Equivalently, at persistence there are natural isomorphisms
\[
\Ttau\mathbf{P}_i(\mathcal{L}_{\mathrm{Tr}}\!\circ\!\mathsf{Trans}(-))\ \cong\ \Ttau\mathbf{P}_i(\mathcal{L}_{\mathrm{Gal}}(-)),\quad
\Ttau\mathbf{P}_i(\mathcal{L}_{\mathrm{Aut}}\!\circ\!\mathsf{Funct}(-))\ \cong\ \Ttau\mathbf{P}_i(\mathcal{L}_{\mathrm{Tr}}(-)),
\]
assembling to \emph{pseudonatural equivalences after truncation}.
\end{remark}

\begin{remark}[Unified \(\delta\)-policy and \(\delta\)-ledger]
\label{rk:9-delta-policy}
Fix \(\boldsymbol{\delta}=(\delta_{\mathrm{int}},\delta_{\mathrm{win}},\delta_{\mathrm{spec}})\) at scale \(\tau\). Layered budgets \(\delta^{\Gal},\delta^{\Tr},\delta^{\Fun}\in V\) aggregate as
\(\delta_{\mathrm{tot}}=\delta^{\Gal}\oplus\delta^{\Tr}\otimes\delta^{\Fun}\).
The \(\delta\)-ledger decomposes residuals as
\[
\delta = \delta_{\alg}\ \oplus\ \delta_{\disc}\ \oplus\ \delta_{\meas},
\]
where \(\delta_{\alg}\) logs provable algebraic non-commutation (A/B order, non-nested reflectors), while \(\delta_{\disc}\) and \(\delta_{\meas}\) log discretization and numerical tolerances (PF/BC transport, sampling, solvers). \emph{All PF/BC residuals in this chapter are charged to \(\delta_{\disc}\oplus \delta_{\meas}\)} unless stated algebraic. When \(\boldsymbol{\delta}_{\mathrm{tot}}=\mathbf{0}\), equalities are taken in \(\Perskft\) (up to iso).
\end{remark}

\subsection*{9.1. Persistence-layer interface for the three layers}
For \(F=\mathcal{L}_{\ast}(x)\) in layer \(\ast\) and degree \(i\), we monitor
\[
\Ttau\,\mathbf{P}_i(F),\qquad
\mathrm{PE}_i^{\le \tau}(F),\qquad
\mathrm{ST}_{\beta_{\mathrm{spec}}}^{\ge M(\tau)}(F),\ \mathrm{HT}(s;F),\qquad
\Ext^1\big(\Rfun(\Ctau F),Q\big)=0\ (Q\in\Qtest).
\]
All metrics/energies/indicators are evaluated after truncation, on a fixed window/\(\tau\), and within \(\boldsymbol{\delta}\).

\subsection*{9.2. Diagnostics along the index \(I\)}
For fixed layer and degree \(i\), at scale \(\tau\) define
\[
\phi_{i,\tau}:\ \varinjlim_{t\in I}\ \Ttau\,\mathbf{P}_i(F_t)\ \longrightarrow\ \Ttau\,\mathbf{P}_i(F_\infty),
\]
\[
\mu_{i,\tau}:=\dim_k\ker\phi_{i,\tau},\qquad
\nu_{i,\tau}:=\dim_k\mathrm{coker}(\phi_{i,\tau}),\qquad
\muc:=\sum_i\mu_{i,\tau},\ \nuc:=\sum_i\nu_{i,\tau}.
\]
All computed after truncation and on the same window/\(\tau\) (Remark~\ref{rk:9-alignment}).

\begin{remark}[Window/scale/metric alignment]
\label{rk:9-alignment}
Colimit and target of \(\phi_{i,\tau}\) use the same \(\tau\), window, and \(\boldsymbol{\delta}\)-policy.
\end{remark}

\subsection*{9.3. Propagation across the three layers}
\begin{declaration}[Propagation under \(\DiagZero\)]
\label{spec:9-prop}
Assume \(\DiagZero\) at a fixed \(\tau\) across the three layers, \(\LC\), and the inter-layer data of Remark~\ref{rk:9-compare}.
Then
\[
\mathsf{Gal}\ \xrightarrow{\ \mathsf{Trans}\ }\ \mathsf{Par}\ \xrightarrow{\ \mathsf{Funct}\ }\ \mathsf{Aut}
\]
commutes after truncation, up to isomorphism in \(\Perskft\) in each degree \(i\). Any obstruction is detected by \((\DiagZero / \DiagNonzero)\). Residual slack is bounded by \(\boldsymbol{\delta}_{\mathrm{tot}}\); when \(\boldsymbol{\delta}_{\mathrm{tot}}=\mathbf{0}\), comparisons are strict (up to iso).
\end{declaration}

\subsection*{9.4. Monitoring protocol (Langlands three-layer)}
\begin{declaration}[Joint monitoring protocol]
\label{spec:9-protocol}
Fix \(\tau\in[\tau_{\min},\tau_{\max}]\), a window convention, an index range \(I\), and a \(\boldsymbol{\delta}\)-policy with \(\delta\)-ledger split \(\delta_{\alg}\oplus\delta_{\disc}\oplus\delta_{\meas}\).
For each layer \(\ast\in\{\mathrm{Gal},\mathrm{Tr},\mathrm{Aut}\}\) and sample \(t\):
\begin{enumerate}
  \item Record \(\Ttau\,\mathbf{P}_i(F^{(\ast)}_t)\) and \(\mathrm{PE}_i^{\le \tau}(F^{(\ast)}_t)\) on \(\Ttau\,\mathbf{P}_i\) (or \(\Ctau\)), within \(\delta_{\win}\).
  \item Record spectral indicators on \(L(\Ctau F^{(\ast)}_t)\) with fixed \((\beta_{\mathrm{spec}},M(\tau),s_{\mathrm{HT}})\) within \(\delta_{\spec}\).
  \item Check \(\Ext^1\big(\Rfun(\Ctau F^{(\ast)}_t),Q\big)=0\) for \(Q\in\Qtest\).
  \item Evaluate \((\DiagZero / \DiagNonzero)\) via \(\phi_{i,\tau}\) along \(I\); log failure types (pure/mixed). Compare distances within \(\delta_{\int}\).
  \item Cross-layer check: non-expansiveness and \(\Ctau\)-compatibility up to f.q.i.; test pseudonaturality after truncation under \(\boldsymbol{\delta}_{\mathrm{tot}}\).
  \item PF/BC comparators \emph{only after collapse}: apply Spec.~\ref{spec:9-T-PFBC-after}; charge residuals to \(\delta_{\disc}\oplus\delta_{\meas}\) unless algebraic.
\end{enumerate}
Declare the collapse-stable regime where (1)–(3) hold jointly and \(\DiagZero\) across layers.
\end{declaration}

\subsection*{9.5. Diagram (three layers, indicators, obstructions)}
\label{sec:9.5}
\begin{center}
\begin{tikzcd}[
  column sep=5.5em, row sep=1.8em,
  cells={nodes={font=\footnotesize}},
  every label/.append style={font=\scriptsize}
]
\text{Galois} \arrow[r, "\mathsf{Trans}"]
  & \text{Transfer} \arrow[r, "\mathsf{Funct}"]
    & \text{Automorphic} \\
\mathcal{L}_{\mathrm{Gal}} \arrow[r, dashed, "{\text{non-expansive},\ \Ctau\text{-compat}}"']
  & \mathcal{L}_{\mathrm{Tr}} \arrow[r, dashed, "{\text{non-expansive},\ \Ctau\text{-compat}}"']
    & \mathcal{L}_{\mathrm{Aut}} \\
\mathbf{P}_i \arrow[r]
  & \mathbf{P}_i \arrow[r]
    & \mathbf{P}_i \\
\Ttau\,\mathbf{P}_i \arrow[r, dashed, "{\cong\ \text{in }\Perskft\ \text{after truncation}}"]
  & \Ttau\,\mathbf{P}_i \arrow[r, dashed, "{\cong\ \text{in }\Perskft\ \text{after truncation}}"]
    & \Ttau\,\mathbf{P}_i \arrow[d, dashed, "\mathrm{PE}_i^{\le \tau}"] \\
& & \text{spectra on }L(\Ctau F^{(\ast)}_t),\ \ \Ext^1(\Rfun(\Ctau F^{(\ast)}_t),k)=0 \arrow[d, dashed] \\
\varinjlim_{t}\,\Ttau\,\mathbf{P}_i(F_t) \arrow[rr, "\phi_{i,\tau}"]
  & & \Ttau\,\mathbf{P}_i(F_\infty) \arrow[dl, bend right=25, dashed, "{(\DiagZero / \DiagNonzero)\ \text{log}}"'] \\
& \text{failure log (pure/mixed)} &
\end{tikzcd}
\end{center}

\subsection*{9.6. Stability, non-expansiveness, and \(\delta\)-aggregation}
\begin{declaration}[Layerwise non-expansiveness with \(\delta\)-aggregation]
\label{spec:9-delta-agg}
Let \(\Phi_1:\mathsf{Gal}\!\to\!\mathsf{Par}\) and \(\Phi_2:\mathsf{Par}\!\to\!\mathsf{Aut}\) be admissible layer maps. For each \(i,\tau\),
\[
d_{\mathrm{int}}\!\Big(\Ttau\mathbf{P}_i(\Phi_2\!\circ\!\Phi_1(F)),\ \Ttau\mathbf{P}_i(\Phi_2\!\circ\!\Phi_1(G))\Big)\ \le\ d_{\mathrm{int}}\!\big(\Ttau\mathbf{P}_i(F),\ \Ttau\mathbf{P}_i(G)\big),
\]
and empirical/normalization slack is bounded by \(\delta_{\mathrm{int,tot}}=\delta_{\mathrm{int}}^{(\Phi_1)}\oplus\delta_{\mathrm{int}}^{(\Phi_2)}\).
Deletion-type steps make windowed energies/spectral tails non-increasing after truncation; inclusion-type steps are stable within \(\delta_{\mathrm{win,tot}},\delta_{\mathrm{spec,tot}}\).
\end{declaration}

\subsection*{9.7. Conjectural stability of functorial transfer}
\begin{conjecture}[Stability of functorial transfer under collapse]
\label{conj:9-stability}
Within the implementable range, assume non-expansive layer maps, \(\LC\), and \(\DiagZero\) on a \(\tau\)-interval.
Then functorial transfer is stabilized at that scale: deletion-type steps do not increase persistence energies; spectral indicators do not grow; and \(\Ext^1\big(\Rfun(\Ctau -),Q\big)=0\) persists across the three layers.
All comparisons are performed after truncation with the same window/\(\tau\) and the same \(\boldsymbol{\delta}\)-policy; \(\boldsymbol{\delta}_{\mathrm{tot}}\) is the sum of per-step budgets.
No number-theoretic identity is asserted.
\end{conjecture}

\subsection*{9.8. Guard-rails, A/B testing, and non-claims}
\begin{remark}[A/B pseudonaturality test after collapse]
\label{rk:9-AB}
For two composites \(\gamma_A,\gamma_B\) through the three layers, test after truncation:
\[
d_{\mathrm{int}}\!\Big(\Ttau\mathbf{P}_i(\gamma_A(F)),\ \Ttau\mathbf{P}_i(\gamma_B(F))\Big)\ \le\ \delta_{\mathrm{int,tot}},\quad
\big|\mathrm{PE}_i^{\le\tau}(\gamma_A)-\mathrm{PE}_i^{\le\tau}(\gamma_B)\big|\ \le\ \delta_{\mathrm{win,tot}},
\]
with spectral/heat discrepancies \(\le\delta_{\mathrm{spec,tot}}\), all on the same window/\(\tau\).
Excess beyond budget is logged as Type~III (spec-mismatch) unless explained by \((\DiagZero / \DiagNonzero)\).
\end{remark}

\begin{remark}[Scope and non-claims]
\label{rk:9-guard}
All statements are persistence/spectral/categorical under (B1)–(B3) of Part~I; no claim of \(\mathrm{PH}_1\Leftrightarrow\Ext^1\).
Obstructions are recorded by \((\DiagZero / \DiagNonzero)\) and are unrelated to classical Iwasawa \(\mu\).
Binary saturation gates (PH\(_1\)\(\Leftrightarrow\)\(\Ext^1\) policies) are organized in Chapter~11.
This chapter provides a design/specification blueprint; it does not decide Langlands correspondence.
\end{remark}

\subsection*{9.9. Boundary models: geometric vs.\ arithmetic regions and the region map}
\begin{definition}[Region map and boundary model]
\label{def:region}
Let \(\mathsf{Dom}\) be the ambient index/parameter space. A \emph{region map} is
\(\mathrm{Reg}:\mathsf{Dom}\to\{\mathrm{Geom},\mathrm{Arith}\}\)
piecewise constant (finitely many jumps per window, recorded in the manifest).
\end{definition}

\begin{remark}[Usage]
\(W\in\mathrm{Geom}\): audit with Chapter~6 (energies/spectra, monotonicity/stability, \((\DiagZero / \DiagNonzero)\)).
\(W\in\mathrm{Arith}\): audit with PF/BC comparators, transfer kernels, and tri-layer gates (Ch.~7–8).
Mixed windows are refined to a MECE partition where \(\mathrm{Reg}\) is constant.
\end{remark}

\subsection*{9.10. Window predicates: \(\mathrm{Ext\_trivial}\Rightarrow\mathrm{WeakGroup\_collapse}\) (typed)}
\begin{definition}[Typed window predicates]
\label{def:predicates}
Fix right-open \(W\), scale \(\tau\), degree set \(\mathcal{I}\).
\begin{itemize}
  \item \(\mathrm{Ext\_trivial}(W,\tau)\) iff \(\Ext^1(\Rfun(\Ctau F|_W),k)=0\).
  \item \(\mathrm{Tower\_stable}(W,\tau)\) iff \(\DiagZero\) on \(W\) for all \(i\in\mathcal{I}\).
  \item \(\mathrm{PF/BC\_ok}(W,\tau)\) iff all PF/BC comparators pass \emph{after collapse} within \(\delta_{\int}\) and residuals are booked to \(\delta_{\disc}\oplus\delta_{\meas}\).
  \item \(\mathrm{Transfer\_ker\_zero}(W,\tau)\) iff each transfer at \(\tau\) has \(\mu^{\mathrm{Tr}}_{i,\tau}=0\) for all \(i\in\mathcal{I}\).
  \item \(\mathrm{WeakGroup\_collapse}(W,\tau)\) iff Def.~\ref{def:group-proxy} holds for fixed finite \(S\subset\mathsf{Aut}(F|_W)\).
\end{itemize}
All predicates are computed on the B-side single layer and logged in the manifest.
\end{definition}

\begin{conjecture}[Predicate schema (WGC as an observed auxiliary tag)]
\label{conj:9-wgc-schema}
Assume on \(W,\tau\): \(\mathrm{Ext\_trivial}(W,\tau)\) and \(\mathrm{Tower\_stable}(W,\tau)\),
together with a declared compression regime (tropical shortening or deletion-type strict energy decay).
Then \(\mathrm{WeakGroup\_collapse}(W,\tau)\) is \emph{expected} to hold and must be \emph{verified directly}
by measuring \(\rho_{\max,i,\tau}(S)\) and \(\mathrm{nilp}_{i,\tau}(S)\) for a chosen finite \(S\).
\end{conjecture}


\begin{remark}[Region-aware instantiation]
On \(\mathrm{Geom}\) windows, (b) is typical (deletion-type smoothing). On \(\mathrm{Arith}\) windows, (a) is supplied by tropical proxies (Ch.~7). The implication is persistence-layer, windowed, and budgeted.
\end{remark}

\subsection*{9.11. Region-aware diagnostics and acceptance}
\begin{definition}[Region-specific acceptance]
\label{def:region-accept}
Fix \(W,\tau\).
\begin{itemize}
  \item \(\mathrm{Accept}_{\mathrm{Geom}}(W,\tau)\): \(\mathrm{Tower\_stable}(W,\tau)\) and deletion-type indicators non-increasing; if also \(\mathrm{Ext\_trivial}(W,\tau)\) and \(\mathrm{WeakGroup\_collapse}(W,\tau)\) is verified,
then tag the window with WGC (auxiliary, non-gate).

  \item \(\mathrm{Accept}_{\mathrm{Arith}}(W,\tau)\): \(\mathrm{Tower\_stable}(W,\tau)\), \(\mathrm{PF/BC\_ok}(W,\tau)\), and \(\mathrm{Transfer\_ker\_zero}(W,\tau)\); if also \(\mathrm{Ext\_trivial}(W,\tau)\) then declare \(\mathrm{WeakGroup\_collapse}(W,\tau)\).
\end{itemize}
The global window verdict is \texttt{Valid} iff the corresponding regional predicate holds and the \(\delta\)-budget is dominated by the edge gap.
\end{definition}

\begin{remark}[Boundary jumps]
If \(\mathrm{Reg}\) jumps inside a coarse window, refine to a MECE partition; evaluate per refined window and paste via Restart/Summability (Chapter~4).
\end{remark}

\subsection*{9.12. Examples (boundary map and predicates)}
\begin{example}[Geometric region]
If \(W\in\mathrm{Geom}\) with viscosity ramping (deletion-type), then \(\mathrm{Tower\_stable}(W,\tau)\) and monotone \(\mathrm{PE}^{\le\tau}\) hold; if \(\mathrm{Ext\_trivial}(W,\tau)\), Conjecture~\ref{conj:9-wgc-schema} suggests \(\mathrm{WeakGroup\_collapse}(W,\tau)\).
\end{example}

\begin{example}[Arithmetic region]
If \(W\in\mathrm{Arith}\) with PF/BC-admissible transfers and tropical shortening \(\kappa<1\), then \(\mathrm{PF/BC\_ok}(W,\tau)\), \(\mathrm{Transfer\_ker\_zero}(W,\tau)\), and \(\mathrm{Tower\_stable}(W,\tau)\) certify \(\mathrm{Accept}_{\mathrm{Arith}}(W,\tau)\).
If also \(\mathrm{Ext\_trivial}(W,\tau)\), conclude \(\mathrm{WeakGroup\_collapse}(W,\tau)\).
\end{example}

\subsection*{9.13. Summary (boundary models and predicates)}
We separated \(\mathrm{Geom}\) and \(\mathrm{Arith}\) windows via a region map \(\mathrm{Reg}\), introduced typed predicates that formalize “collapse after truncation’’ decisions at fixed \(\tau\), and used the core schema
\(\mathrm{Ext\_trivial}\wedge\mathrm{Tower\_stable}\Rightarrow\mathrm{WeakGroup\_collapse}\)
(under tropical or deletion-type compression).
Together with tri-layer gates and PF/BC comparators \emph{checked only after collapse}, this boundary view makes explicit \emph{where} collapse holds and \emph{which} layer blocks it, with Type~IV failures reported via \((\DiagZero / \DiagNonzero)\). All acceptance criteria are windowed and budgeted (\(\delta_{\alg}\oplus\delta_{\disc}\oplus\delta_{\meas}\)).

\subsection*{9.14. T–PFBC–AfterCollapse (explicit specification)}
\begin{declaration}[T–PFBC–AfterCollapse]
\label{spec:9-T-PFBC-after}
Let \(\Psi\) be any PF/BC move admissible in Appendix~N (projection formula, base change, push–pull along Cartesian squares, base-change for kernels/Hecke, etc.). For each \(i\) and window \(W\) at threshold \(\tau\):
\begin{enumerate}
  \item \textbf{Transport to the collapsed layer.} Comparators for \(\Psi\) are formed and tested \emph{only} on \(\big(\Ttau\mathbf{P}_i(-)\big)\big|_{W}\). No metric/indicator is read before truncation.
  \item \textbf{Non-expansive baseline.} There is a canonical map (iso in the ideal PF/BC setting)
  \[
  \Ttau W\,\mathbf{P}_i(\Psi F)\ \longrightarrow\ \Ttau W\,\mathbf{P}_i(\Psi G)
  \]
  compatible with the corresponding map \(\Ttau W\,\mathbf{P}_i(F)\to\Ttau W\,\mathbf{P}_i(G)\) and \(1\)-Lipschitz in \(d_{\mathrm{int}}\).
  \item \textbf{Residual accounting.} Any deviation of PF/BC comparators from isomorphism after truncation is charged to \(\delta_{\disc}\oplus\delta_{\meas}\) on \(W\) (unless an algebraic obstruction is identified, in which case it is charged to \(\delta_{\alg}\)). The tri-layer and A/B tests use only these after-collapse residuals.
\end{enumerate}
In particular, all PF/BC-based \emph{distance} or \emph{energy} comparisons in this chapter are meaningful \emph{only} after applying \(\Ttau\) (equivalently, on \(\Ctau\)).
\end{declaration}



\section{Chapter 10: Application Program (PDE / BSD rank 0/1 / RH up to T)}
\addcontentsline{toc}{section}{Application Program (PDE / BSD rank 0/1 / RH up to T)}
\label{sec:chapter10}

\begin{remark}[Stability vs.\ monotonicity; corrected]
For non-expansive maps, indicators are stable (non-expansive).
Deletion-type operations satisfying Appendix~E (e.g.\ Dirichlet restriction, principal submatrices/Schur complements, positive-semidefinite Loewner contractions)
make spectral tails and windowed energies \emph{non-increasing}.
Inclusion-type updates generally do not guarantee non-increase; we only claim stability.
\end{remark}

\subsection*{10.0. Standing hypotheses and admissible realizations}
We fix a field \(k\) and work within the \emph{implementable range} of Part~I.
\emph{All statements in this chapter are made within the constructible range}
(we identify \(\Perskft\) with the constructible subcategory as in Chapter~6).
Let \(\FiltCh{k}\) be finite-type filtered chain complexes, and \(\mathbf{P}_i:\FiltCh{k}\to\Perskft\) the degreewise persistence functor.
We write \(\Ttau:=\mathbf{T}_\tau\) for the Serre (bar-deletion) reflector at scale \(\tau\ge 0\), and use its filtered lift \(\Ctau\) \emph{up to filtered quasi-isomorphism} (Chapter~2, §§2.2–2.3).
A fixed \(t\)-exact realization \(\Rfun:\FiltCh{k}\to D^{\mathrm{b}}(k\text{-mod})\) is retained; the lifting–coherence hypothesis \(\LC\) is assumed when comparing \(\Ctau\) with \(\tau_{\ge 0}\!\circ\!\Rfun\).
\emph{Equalities and Lipschitz claims are asserted only at the persistence layer; at the filtered-complex layer they hold up to filtered quasi-isomorphism.}
Endpoint conventions and infinite bars follow Chapter~2, Remark~\ref{rk:2-endpoints}.

Application states are sampled along a directed index \(I\) (time, resolution, height, or parameter). An \emph{admissible realization} is a functor
\[
  \mathsf{State}\ \xrightarrow{\ \mathcal{P}\ }\ \FiltCh{k},\qquad U\longmapsto F=\mathcal{P}(U),
\]
satisfying:
\begin{itemize}
  \item \textbf{Finite-type and (co)limits:} \(F\) is degreewise finite-type; degreewise filtered (co)limits in \(\FiltCh{k}\) are computed objectwise in \([\mathbb{R},\mathsf{Vect}_k]\) and used only under the scope policy of Appendix~A (compute in the functor category and verify return to \(\Pers^{\mathrm{cons}}_k\)).
  \item \textbf{Non-expansiveness under persistence:} along each directed update (e.g.\ time step, parameter step, height step, down-/up-sampling), the induced filtered map is non-expansive degreewise under \(\mathbf{P}_i\);
  in \emph{deletion-type} steps (Appendix~E) indicators are non-increasing up to f.q.i., while inclusion-type updates guarantee only stability.
  \item \textbf{Compatibility with truncation:} for each \(i\), naturally in \(\Perskft\),
  \(
    \mathbf{P}_i(\Ctau F)\ \cong\ \Ttau\,\mathbf{P}_i(F).
  \)
  \item \textbf{Realization coherence:} \(\Rfun\) is \(t\)-exact and compatible with \(\LC\), so functorially up to f.q.i.,
  \(\Rfun(\Ctau F)\simeq \tau_{\ge 0}\,\Rfun(F)\).
\end{itemize}

\subsection*{10.0a. Window certificates, manifests, and \(\delta\)-ledgers (generic)}
A \emph{window} is an interval \(W=[u,u')\subset\mathbb{R}\) in the index axis (time/resolution/height/parameter).
Fix \(\tau>0\) (resolution-adapted; Chapter~2). A \emph{window certificate} at \((W,\tau)\) records:
\begin{itemize}
  \item the single-layer objects \(\Ttau\,\mathbf{P}_i(F_s)\) for \(s\in W\cap I\),
  \item windowed persistence energies \(\mathrm{PE}_i^{\le \tau}\), spectral tails \(\mathrm{ST}_\beta^{\ge M(\tau)}\), and heat traces \(\mathrm{HT}(t;\cdot)\) computed on \(L(\Ctau F_s)\),
  \item the obstruction counts \((\DiagZero / \DiagNonzero)\) computed via \(\phi_{i,\tau}\) (cf.\ §\ref{subsec:10.3}),
  \item the categorical check \(\Ext^1\big(\Rfun(\Ctau F_s),Q\big)=0\) for \(Q\in\Qtest\),
  \item a \emph{manifest} (run log) including discretization/sampling controls and thresholds,
  \item a \(\delta\)-ledger with the decomposition
\(
  \delta(i,\tau)=\delta_{\alg}(i,\tau)\ \otimes\ \delta_{\disc}(i,\tau)\ \otimes\ \delta_{\meas}(i,\tau),
\)
aggregated as \(\delta_{\mathrm{tot}}=\bigotimes_{U\in W}\delta_U\).
(For the default Lawvere quantale, \(\otimes=+\).)
\end{itemize}
Passing the gate (§\ref{subsec:10.8}) with safety margin \(\mathrm{gap}_\tau>\Sigma\delta\) produces a \emph{window certificate} for \((W,\tau)\).
Window pasting (Restart/Summability; §\ref{subsec:10.5}) aggregates certificates into global coverage.

\begin{remark}[Triggers (generic)]
A \emph{trigger} is a domain-specific necessary condition for gate failure within \(W\); it does not replace the gate but augments diagnostics. We use three canonical categories:
\begin{itemize}
  \item \textbf{Blow-up signs:} sustained growth in high-frequency/height/complexity channels after \(\Ctau\).
  \item \textbf{Tower accumulation:} repeated kernel/cokernel obstructions \((\DiagZero / \DiagNonzero)\ne(0,0)\) or aux-bar persistence across windows.
  \item \textbf{PF/BC deviations:} violations of domain-specific physical/arithmetic fidelity or sampling/contour budgets (e.g.\ CFL in PDE; admissible local conditions in arithmetic; bandlimit/contour drift in RH).
\end{itemize}
All triggers are logged with parameters and timestamps in the manifest.
\end{remark}

\subsection*{10.0b. Definable windows and the \texorpdfstring{\(E_1\)}{E1} trigger}
\label{subsec:10.0b}
Work on right-open windows \(W\) definable in the declared tame structure:
o-minimal on \(\mathrm{Geom}\) windows and Denef--Pas on \(\mathrm{Arith}\) windows (cf.\ Chapter~9, \S9.9).
Then \emph{finite-event} and \emph{finite-Čech-depth} properties hold, and Chapter~3, Theorem~\textup{\ref{thm:E1-local}} (see also Appendix~C) yields, on \(W\),
\[
E_1(W)=0\ \Longrightarrow\ \PH_1(\Ctau F|_W)=0\ \text{ and }\ \Ext^1\!\big(\Rfun(\Ctau F|_W),k\big)=0.
\]
Thus, on windows where the hypotheses of Theorem~\ref{thm:E1-local} apply,
a vanishing \(E_1(W)\) provides a certified shortcut: it simultaneously certifies
the PH- and Ext-checks on \(W\). No global equivalence is asserted.

\subsection*{10.1. Permitted operations and NS-specific examples (with CFL/CN controls)}
Each A-side step \(U\) is labeled and immediately followed by collapse \(\Ctau\); all measurements and gate decisions are taken on the B-side single layer \(\Ttau\mathbf{P}_i\) (Chapter~1, B-Gate\(^{+}\)). The \emph{Courant number} \(\mathrm{CN}\) and \emph{CFL} condition are recorded in the run manifest (Appendix~G) and justify quantitative non-expansiveness (\(\varepsilon\)-interleavings) for time stepping.

\paragraph{Operation labels and NS examples.}
\begin{itemize}
  \item \emph{Deletion-type (monotone):} low-pass mollification (filter width \(\sigma\)), viscosity increment \(\nu\mapsto \nu+\delta\nu\), threshold lowering in levelset filtrations, Dirichlet/absorbing boundary introduction, conservative averaging, Schur complements on blocks of the discrete operators. After \(\Ctau\), windowed persistence energies and spectral tails/heat traces on \(L(\Ctau F)\) are \emph{non-increasing} (Appendix~E).
  \item \emph{\(\varepsilon\)-continuation (non-expansive):} small time step \(\Delta t\) respecting CFL (e.g.\ \(\mathrm{CN}=\frac{u\,\Delta t}{\Delta x}\le \mathrm{CN}_{\max}\)), small parameter drifts (forcing amplitude, boundary condition perturbations), micro-updates of numerical flux limiters. Collapse-after stability holds with interleaving drift \(\varepsilon\!\sim\!C\Delta t\) (recorded).
  \item \emph{Inclusion-type (stable only):} domain enlargement, mesh refinement without smoothing, addition of couplings/sources (as long as the induced filtered map is 1-Lipschitz for \(\mathbf{P}_i\)). No monotonicity is claimed; stability only.
\end{itemize}
For each step, record in the \emph{\(\delta\)-ledger} (Chapter~5; Appendix~L) the decomposition
\(\delta=\delta^{\mathrm{alg}}+\delta^{\mathrm{disc}}+\delta^{\mathrm{meas}}\).

\begin{declaration}[Deletion-type operations (PDE)]
\label{spec:10-pde-del}
Operations covered by Appendix~E (Dirichlet restriction/absorbing boundaries, positive-semidefinite Loewner contractions with trace monotonicity, principal submatrices and Schur complements, conservative averaging) are treated as \emph{deletion-type}.
After truncation they are non-expansive for each \(\mathbf{P}_i\), and windowed energies \(\mathrm{PE}_i^{\le\tau}\) as well as spectral tails/heat traces on \(L(\Ctau F)\) are \emph{non-increasing}.
Inclusion-type updates are asserted only to be stable (non-expansive).
\end{declaration}

\begin{remark}[Quantitative non-expansiveness]
\label{rk:10-epsilon}
Let \(d_{\mathrm{int}}\) denote the interleaving distance on degreewise persistence. Along an update \(F_{s+1}\to F_s\), assume
\(
  d_{\mathrm{int}}\big(\mathbf{P}_i(F_{s+1}),\mathbf{P}_i(F_s)\big)\le \varepsilon_s\ (\varepsilon_s\ge 0).
\)
If \(\sup_s\varepsilon_s\le \varepsilon\), we call the tower \emph{\(\varepsilon\)-Lipschitz}. In deletion-type steps typically \(\varepsilon_s=0\); inclusion-type need not be zero. Time stepping under a CFL bound provides a concrete \(\varepsilon_s\sim C\Delta t\), recorded in the manifest.
\end{remark}

\begin{remark}[Truncation is \(1\)-Lipschitz]
Since \(\Ttau\) is \(1\)-Lipschitz for \(d_{\mathrm{int}}\), the same \(\varepsilon\)-Lipschitz control holds after truncation:
\(
d_{\mathrm{int}}\!\big(\Ttau\,\mathbf{P}_i(F_{s+1}),\ \Ttau\,\mathbf{P}_i(F_s)\big)\le \varepsilon_s.
\)
\end{remark}

\begin{remark}[Endpoints and infinite bars]
\label{rk:10-endpoints}
Open/closed endpoint choices are immaterial; infinite bars are not removed by \(\Ttau\) and are clipped by windowed indicators (as in Chapter~6).
\end{remark}

\begin{remark}[Index set and cone extension]
\label{rk:10-cone}
Work in the \emph{filtered index category} \(I\cup\{\infty\}\) with cone apex \(\infty\): for \(s\in I\), adjoin cone maps \(s\to\infty\).
Under \(\mathcal{P}\), these yield filtered maps \(F_s\to F_\infty\) used to define the comparison morphisms \(\phi_{i,\tau}\) at fixed~\(\tau\) (cf.\ Chapter~4).
\end{remark}

\subsection*{10.2. Construction principles for \texorpdfstring{$\mathcal{P}$}{P} (PDE)}
We list domain-agnostic templates; any one suffices for admissibility.
\begin{itemize}
  \item \textbf{Scalar-field cubical pipeline.} From a field \(q\) (e.g.\ vorticity magnitude, enstrophy density, \(Q\)-criterion) on a grid, build a cubical filtration by superlevel/sublevel sets; chains are \(k\)-valued on cubes.
  \item \textbf{Graph/simplicial pipeline.} From point samples, build Vietoris–Rips/alpha complexes with scale \(\varepsilon\); chains are \(k\)-valued on simplices.
  \item \textbf{Hybrid pipeline.} Combine topology of coherent structures with connectivity of level sets; filtration is vectorized but evaluated degreewise.
\end{itemize}
All three preserve finite-type per degree and admit non-expansive updates for standard PDE integrators (viscous steps are smoothing; down-sampling is deletion-type).
\emph{Spectral proxies are computed on \(L(\Ctau F)\) (positive eigenvalues; zero modes excluded or via pseudoinverse).}

\begin{remark}[Normalization and logging]
Normalization (graph vs.\ Hodge, symmetric vs.\ random-walk), zero-mode handling, and the window policy are fixed throughout a run and recorded alongside \((\beta,M(\tau),t)\) (Appendix~G).
All spectral indicators are computed on \(L(\Ctau F)\) to align with the truncation window.
\emph{Spectral indicators are not f.q.i.\ invariants; we only claim stability under a fixed policy \((\beta,M(\tau),t)\) on \(L(\Ctau F)\) (cf.\ Chapter~11).}
\end{remark}

\subsection*{10.2a. Construction principles for arithmetic and RH realizations}
\paragraph{BSD rank 0/1 (arithmetic).}
Let \(E/\mathbb{Q}\) be an elliptic curve; \(A\) denotes an \emph{arithmetic state} (e.g.\ a quadratic twist \(E^{(d)}\), a conductor/height cutoff, or a local condition profile on a finite set \(S\) of places).
We construct filtered complexes by any of:
\begin{itemize}
  \item \emph{Selmer filtration:} complexes whose chains encode \(p\)-Selmer data filtered by local condition strength or height; arrows reflect tightening/loosening local conditions.
  \item \emph{Descent graph pipeline:} graphs whose vertices are local condition classes; edges encode compatibility constraints; build a filtration by penalty thresholds.
  \item \emph{Hybrid pipeline:} combine Selmer layers with isogeny factors or visibility relations; evaluate degreewise.
\end{itemize}
Deletion-type updates include restriction to a smaller \(S\), tightening a local condition, or projecting along an isogeny with positive-semidefinite trace contraction on the chosen Laplacian model (Appendix~E analogues).
\(\varepsilon\)-continuation steps include small changes in a twist parameter \(d\) within a controlled family and height cutoffs; inclusion-type includes enlarging \(S\) or adding local conditions. Spectral proxies are computed on \(L(\Ctau F)\).

\paragraph{RH up to \(T\) (analytic).}
Let the \emph{state} encode samples of \(\xi(1/2+it)\), argument \(S(t)\), or zero counts \(N(t)\) on a window of heights. Build filtered complexes via:
\begin{itemize}
  \item \emph{Gram-graph pipeline:} nodes at Gram points/mesh points; edges connect near neighbors; filtration by magnitude thresholds or discrepancy of the argument from expected trends.
  \item \emph{Bandlimited scalar pipeline:} sub/superlevel filtrations of smoothed \(|\zeta(1/2+it)|\), \(|\xi|\), or of explicit-formula residuals; smoothing widths serve as deletion-type operations.
  \item \emph{Hybrid pipeline:} combine zero-locator events with discrepancy fields from the explicit formula.
\end{itemize}
Deletion-type updates include convolution smoothing (Gaussian/Fejér), restriction to subwindows, or projection onto bandlimited subspaces; \(\varepsilon\)-continuation includes small height increments \(\Delta t\) under Nyquist/bandlimit controls; inclusion includes window enlargement or resolution increase. Spectral proxies are computed on \(L(\Ctau F)\).

\subsection*{10.3. Indicators and diagnostics}
\label{subsec:10.3}
For each sample \(s\in I\) and degree \(i\) we monitor:
\[
  \Ttau\,\mathbf{P}_i(F_s),\qquad
  \mathrm{PE}_i^{\le \tau}(F_s)\ \text{(truncated energies on }\Ttau\,\mathbf{P}_i(F_s)\text{)},\qquad
  \mathrm{ST}_\beta^{\ge M(\tau)}(F_s),\ \mathrm{HT}(t;F_s)\ \text{on }L(\Ctau F_s),
\]
together with the categorical check \(\Ext^1\big(\Rfun(\Ctau F_s),Q\big)=0\) for \(Q\in\Qtest\).
For fixed \(\tau\), define the comparison map
\[
  \phi_{i,\tau}:\ \varinjlim_{s\in I}\ \Ttau\,\mathbf{P}_i(F_s)\ \longrightarrow\ \Ttau\,\mathbf{P}_i(F_\infty),
\]
and obstruction counts
\(
\mu_{i,\tau}=\dim_k\ker\phi_{i,\tau},\quad
\nu_{i,\tau}=\dim_k\mathrm{coker}\,\phi_{i,\tau},
\)
with \(\muc=\sum_i\mu_{i,\tau}\), \(\nuc=\sum_i\nu_{i,\tau}\) (finite by bounded degrees).
\emph{The obstructions \((\DiagZero / \DiagNonzero)\) are invariant under filtered quasi-isomorphisms and under cofinal reindexing of the tower} (Appendix~J).

\begin{declaration}[Specification: Tower stability at the persistence layer]
\label{spec:10-stab}
Under the finite-type and objectwise degreewise-colimit hypotheses, for each fixed \(\tau\) and all \(i\) the map
\(
  \phi_{i,\tau}:\ \varinjlim_{s}\ \Ttau\,\mathbf{P}_i(F_s)\ \xrightarrow{\ \cong\ }\ \Ttau\,\mathbf{P}_i(F_\infty)
\)
is an isomorphism; hence \(\DiagZero\) at that scale and Type~IV is excluded at \(\tau\).
\end{declaration}

\subsection*{10.4. Trigger pack ([Spec], domain-restricted necessary conditions)}
\paragraph{PDE (Navier–Stokes).}
\begin{itemize}
  \item \textbf{High-frequency surge:} sustained growth of enstrophy or high-wavenumber density in \(W\) \(\Rightarrow\) aux-bars (Chapter~11) persist \(>0\) after \(\Ctau\) or \(\mu>0\) is detected at \(\tau\).
  \item \textbf{Under-resolved advection:} CFL violation or \(\mathrm{CN}>\mathrm{CN}_{\max}\) \(\Rightarrow\) \(\varepsilon\)-continuation drift \(\varepsilon\) exceeds the safety margin \(\mathrm{gap}_\tau\) and B-Gate\(^{+}\) fails.
  \item \textbf{Unbalanced dissipation:} lack of smoothing under nominally viscous steps \(\Rightarrow\) non-decrease of \(\mathrm{PE}_i^{\le\tau}\) or spectral tails; repeated violations within \(W\) mark the window as non-regularizing.
  \item \textbf{PF/BC deviations:} boundary condition mismatches or energy budget imbalances beyond tolerance \(\Rightarrow\) flag window as suspect.
\end{itemize}

\paragraph{BSD rank 0/1.}
\begin{itemize}
  \item \textbf{Rank-proxy surge:} persistent increase of \(p\)-Selmer size proxies or regulator surges under deletion-type tightening \(\Rightarrow\) aux-bars persist \(>0\) or \(\mu>0\).
  \item \textbf{Local inconsistency:} repeated flips of local condition satisfaction under small parameter moves \(\Rightarrow\) \(\varepsilon\)-drift exceeds \(\mathrm{gap}_\tau\).
  \item \textbf{PF/BC deviations:} admissibility violations for chosen local models (e.g.\ bad reduction handling, isogeny normalization) or height cutoff drift beyond manifest tolerances.
\end{itemize}

\paragraph{RH up to \(T\).}
\begin{itemize}
  \item \textbf{Argument anomaly:} excursions of \(S(t)\) or explicit-formula residual discrepancies exceeding tolerance within \(W\) \(\Rightarrow\) aux-bars persist or \(\mu>0\).
  \item \textbf{Sampling under-resolution:} Nyquist/bandlimit violation for the chosen smoothing/kernel parameters \(\Rightarrow\) \(\varepsilon\)-drift exceeds \(\mathrm{gap}_\tau\).
  \item \textbf{PF/BC deviations:} contour/normalization policies (e.g.\ Gram grid misalignment) outside manifest tolerances.
\end{itemize}
All triggers are logged (Appendix~G) with quantitative thresholds and do \emph{not} replace B-Gate\(^{+}\); they augment diagnostics.

\subsection*{10.5. Window pasting: Restart and Summability}
\label{subsec:10.5}
Let \(\{W_k=[u_k,u_{k+1})\}_k\) be a MECE partition (Chapter~2). On each \(W_k\), fix \(\tau_k\) (resolution-adapted; Chapter~2) and compute the pipeline budget \(\Sigma\delta_k(i)=\sum_{U\in W_k}\delta_U(i,\tau_k)\). If B-Gate\(^{+}\) passes with a safety margin \(\mathrm{gap}_{\tau_k}>\Sigma\delta_k(i)\), the Restart lemma (Chapter~4) yields
\[
\mathrm{gap}_{\tau_{k+1}}\ \ge\ \kappa\ \bigl(\mathrm{gap}_{\tau_k}-\Sigma\delta_k(i)\bigr)\quad(\kappa\in(0,1]).
\]
If moreover \(\sum_k\Sigma\delta_k(i)<\infty\) (Summability; e.g.\ geometric decay of \(\tau_k,\beta_k\)), windowed certificates paste to a global one (Chapter~4).

\subsection*{10.6. Persistence-guided regularization ([Spec])}
\begin{declaration}[Specification: Persistence-guided regularization]
\label{spec:10-PGR}
A numerical or data-analytic regime is \emph{persistence-regularizing at scale \(\tau\)} if, along \(s\in I\),
\begin{enumerate}
  \item \(\mathrm{PE}_i^{\le \tau}\) are non-increasing (strictly decreasing on steps with genuine deletion-type smoothing),
  \item spectral indicators \(\mathrm{ST}_\beta^{\ge M(\tau)}\), \(\mathrm{HT}(t;\cdot)\) on \(\Ctau F_s\) are non-increasing (stability in general, monotone decrease in smoothing steps),
  \item \(\DiagZero\),
  \item \(\Ext^1\big(\Rfun(\Ctau F_s),Q\big)=0\) for \(Q\in\Qtest\).
\end{enumerate}
When these hold across a \(\tau\)-interval, the regime aligns with established regularization/verification frameworks at that scale (domain-specific hypotheses to be listed separately). No analytic identity is claimed.
\end{declaration}

\subsection*{10.7. AK--NS hypothesis (programmatic)}
\begin{conjecture}[AK--NS hypothesis]
\label{conj:10-AKNS}
For Navier–Stokes-type flows, under an admissible realization \(\mathcal{P}\) and \(\LC\), if a monitored segment satisfies Declaration~\ref{spec:10-PGR} across a \(\tau\)-interval, then the designed persistence structure \emph{collapses} at that scale (bars shorten/vanish in aggregate), spectral tails decay, and the categorical check persists.
Programmatically, this corresponds to convergence toward known regularity scenarios at that scale.
No equivalence \(\mathrm{PH}_1\Leftrightarrow\Ext^1\) is asserted; only the one-way bridge under (B1)–(B3) is used.
\end{conjecture}

\subsection*{10.8. Gate template (PDE)}
\label{subsec:10.8}
On a fixed window \(W=[u,u')\), collapse threshold \(\tau>0\), and degree \(i\):
\begin{enumerate}
  \item Apply step \(U\) (labeled as above), then collapse \(\Ctau\).
  \item Measure on the B-side single layer: \(\Ttau\mathbf{P}_i\), \(\mathrm{PE}_i^{\le\tau}\), spectral auxiliaries (aux-bars; Chapter~11), and (in scope) \(\Ext^1\).
  \item Record \(\delta^{\mathrm{alg}},\delta^{\mathrm{disc}},\delta^{\mathrm{meas}}\) and update \(\Sigma\delta\).
  \item Evaluate B-Gate\(^{+}\): require \(\PH_1=0\), (in scope) \(\Ext^1=0\), \(\DiagZero\) after \(\Ttau\), and \(\mathrm{gap}_\tau>\Sigma\delta\).
  \item Log verdict; issue a windowed certificate on success; otherwise classify failure (Type I–IV).
\end{enumerate}

\subsection*{10.9. Monitoring protocol (PDE)}
\begin{declaration}[Specification: Joint monitoring protocol]
\label{spec:10-protocol}
Fix scales \(\tau\in[\tau_{\min},\tau_{\max}]\) and an index set \(I\) (time/resolution/parameter).
For each sample \(s\in I\):
\begin{enumerate}
  \item \emph{Compute and record} \(\Ttau\,\mathbf{P}_i(F_s)\) and \(\mathrm{PE}_i^{\le \tau}\) on \(\Ttau\,\mathbf{P}_i(F_s)\) (equivalently on \(\Ctau F_s\)).
  \item \emph{Compute and record} spectral indicators \(\mathrm{ST}_\beta^{\ge M(\tau)}\), \(\mathrm{HT}(t;F_s)\) on \(L(\Ctau F_s)\) with a fixed \((\beta,M(\tau),t)\) policy.
  \item \emph{Check} \(\Ext^1\big(\Rfun(\Ctau F_s),Q\big)=0\) for \(Q\in\Qtest\).
  \item \emph{Evaluate} \((\DiagZero / \DiagNonzero)\) via \(\phi_{i,\tau}\) along \(s\in I\) and log failure types \emph{(pure kernel/cokernel/mixed)} if present.
  \item \emph{Stability declaration:} declare the \emph{persistence-regularizing regime} where (1)–(4) hold across the monitored \(\tau\)-range.
\end{enumerate}
\end{declaration}

\subsection*{10.10. Diagram (PDE pipeline and diagnostics)}
\begin{center}
\begin{tikzcd}[
  ampersand replacement=\&,
  column sep=2.8em, row sep=2.2em,
  cells={nodes={font=\footnotesize}},
  every label/.append style={font=\scriptsize},
  scale=0.9, transform shape
]
% --- Row 1 ---
{\text{PDE state }U} \arrow[r, "\mathcal{P}"] \&
{F\in \FiltCh{k}} \arrow[r, "\mathbf{P}_i"] \&
{\mathbf{P}_i(F)} \arrow[r, "\Ttau"] \&
{\Ttau\,\mathbf{P}_i(F)} \\
% --- Row 2 ---
\& \& {L(\Ctau F_s)} \arrow[r, dashed, "{\text{heat trace on }L(\Ctau F_s)}"] \&
{\mathrm{HT},\ \mathrm{ST}_{\beta}} \\
% --- Row 3 ---
\& {\Rfun(F_s)} \& \& {\Ext^1(\Rfun(\Ctau F_s),k)=0} \\
% --- Row 4 ---
\& {\varinjlim_{s}\,\Ttau\,\mathbf{P}_i(F_s)} \arrow[rr, "{\phi_{i,\tau}}"] \& \&
{\Ttau\,\mathbf{P}_i(F_\infty)} \\
% --- Row 5 ---
\& {\text{failure log (pure kernel/cokernel/mixed)}} \& \&
%
% ===== extra arrows with absolute coordinates =====
\arrow[from=1-3, to=2-3, dashed, "{\mathrm{PE}_i^{\le \tau}}"']
\arrow[from=1-4, to=2-4, dashed, "{\mathrm{PE}_i^{\le \tau}}"]
\arrow[from=3-2, to=1-2, bend left=40, dashed, "\LC"]
\arrow[from=3-2, to=3-4, dashed, "{\Ctau\ \text{ then }\Ext^1(-,k)\ \text{test}}"]
\arrow[from=4-4, to=5-2, bend right=25, dashed, swap,
       "{(\DiagZero / \DiagNonzero)\ \text{log (pure kernel/cokernel/mixed)}}"]
\end{tikzcd}
\end{center}

\subsection*{10.11. Toy instances (persistence layer)}
\begin{example}[Viscous smoothing]
\label{ex:10-visc}
Let \(s\mapsto U_s\) be viscous steps for which the induced maps are deletion-type on the filtration.
Then bar lengths within the \(\tau\)-window decrease (or vanish), \(\mathrm{PE}_i^{\le \tau}\) strictly decreases, and \(\DiagZero\) at fixed \(\tau\) by Declaration~\ref{spec:10-stab}.
Spectral proxies are evaluated as tails/heat traces of \(L(\Ctau F_s)\).
\end{example}

\begin{example}[Refinement/averaging pair]
A refinement \(F\to F'\) (inclusion-type) followed by conservative averaging \(F'\to \bar{F}\) (deletion-type) yields a non-expansive two-step update.
Under stability, \(\mathrm{PE}_i^{\le \tau}\) is non-increasing; failure logs isolate kernel/cokernel imbalance when present.
Spectral indicators are computed on \(L(\Ctau \bar{F})\).
\end{example}

\subsection*{10.12. Reproducibility (PDE)}
\begin{remark}[Run logs and parameters]
\label{rk:10-logs}
For each run, log: index range \(s\in[s_{\min},s_{\max}]\) (e.g.\ time), scales \(\tau\in[\tau_{\min},\tau_{\max}]\) (step width), spectral parameters \((\beta,M(\tau),t)\), discretization choices (cubical/simplicial/hybrid), CFL/CN numbers, \emph{a barcode-matching seed for reproducible vineyard tracking} (Appendix~G), \((\DiagZero / \DiagNonzero)\) per \(\tau\) with failure types, and the \(\delta\)-ledger decomposition at step level.
These logs enable exact reruns and pipeline audits.
\end{remark}

\noindent A minimal \texttt{run.yaml} PDE block:

\begin{verbatim}
quantale:
  name: "[0,inf]_plus"
  op: "+"
  unit: 0.0
  order: "<="
definable:
  structure: "o-minimal"
  window_formulae:
    - "u <= t < u'"
layered_delta:
  deltaGal: 0.0
  deltaTr:  0.0
  deltaFun: 0.0

windows:
  domain: [[0,1), [1,2), [2,3)]
  collapse_tau: 0.08
  spectral_bins: {a: 0.0, beta: 0.02, bins: 96, boundary: "right-open"}
coverage_check:
  length_sum: 3.0
  length_target: 3.0
  events_sum_equals_global: true
cfl:
  courant_number_max: 0.5
  courant_number_measured: 0.32
operations:
  - U: mollify; type: deletion; tau: 0.08; delta: {alg:0.004, disc:0.003, meas:0.001}
  - U: timestep; type: epsilon;  tau: 0.08; eps: 0.006; delta: {alg:0.000, disc:0.002, meas:0.001}
persistence:
  PH1_zero: true
  Ext1_zero: true
  mu: 0
  nu: 0
  phi_iso_tail: true
spectral:
  aux_bars_remaining: 0
budget:
  sum_delta: 0.011
  safety_margin: 0.025
gate:
  accept: true
\end{verbatim}

\subsection*{10.13. Guard-rails and non-claims}
\begin{remark}[Scope and non-claims]
\label{rk:10-guard}
All statements operate at the persistence/spectral/categorical layers in the implementable range.
No analytic regularity theorem is proved; the AK–NS hypothesis is programmatic.
No claim of \(\mathrm{PH}_1\Leftrightarrow\Ext^1\) is made; only the one-way bridge under (B1)–(B3) is used.
The obstruction \(\muc\) is unrelated to classical Iwasawa \(\mu\).
\end{remark}

\subsection*{10.14. Completion note}
\begin{remark}[No further supplementation required]
This chapter integrates: (i) MECE windowing and resolution-adapted \(\tau\) with stability bands (via Chapters~2 and 4), (ii) the permitted operations catalog with NS-specific examples under CFL/CN controls and \(\delta\)-ledger accounting (Chapter~5; Appendix~L), (iii) B-side single-layer gate B-Gate\(^{+}\) with \(\PH_1/\Ext^1/(\mu,\nu)/\)safety-margin, (iv) triggers as \textbf{[Spec]} with a complete monitoring protocol, (v) Restart/Summability for window pasting, and (vi) reproducibility (run.yaml) with audit fields. All claims remain within the v16.0 guard-rails and cross-reference the proven core.
\end{remark}

\subsection*{10.15. Application II: BSD rank 0/1 (definable windows, E\texorpdfstring{\(_1\)}{1}-trigger, Iwasawa interface)}
\label{subsec:10.15}
We monitor families where analytic/algebraic rank is expected to be \(0\) or \(1\) (e.g.\ twists \(E^{(d)}\), isogeny classes, conductor windows) using the same gate/certificate format. \emph{No BSD assertion is made.} We provide a persistence/spectral protocol with reproducible manifests and window short-circuiting via \(E_1\).

\paragraph{Admissible realization \(\mathcal{P}_{\mathrm{BSD}}\).}
Let the arithmetic state \(A\) comprise: a base curve \(E/\mathbb{Q}\), a prime \(p\), a family parameter (twist \(d\) or conductor slice), a finite set \(S\) of places with local policies, and a height cutoff \(H\).
Define \(F=\mathcal{P}_{\mathrm{BSD}}(A)\) by one of:
\begin{itemize}
  \item \emph{Selmer complex filtration:} degrees encode \(p\)-Selmer cochains filtered by local-condition penalties; deletion-type steps tighten local conditions or decrease \(H\).
  \item \emph{Descent graph:} vertices are local symbols/classes; edges capture compatibility; filtration by cumulative penalty; deletion-type is edge/vertex contraction under verified dominance (Appendix~E analogues).
  \item \emph{Hybrid:} combine isogeny pushforwards with Selmer layers; normalize Laplacians as per a fixed policy recorded in the manifest.
\end{itemize}

\paragraph{Definable windows and \(E_1\)-short-circuit.}
Fix a right-open, o-minimal definable window \(W\) (Archimedean) or Denef–Pas definable window (non-Archimedean; Appendix~Q).
Then Chapter~3, Theorem~\ref{thm:E1-local} implies, \emph{on \(W\)},
\[
E_1(W)=0\ \Longleftrightarrow\ \PH_1(\Ctau F|_W)=0\ \Longleftrightarrow\ \Ext^1\!\big(\Rfun(\Ctau F|_W),k\big)=0,
\]
so B-Gate\(^{+}\) reduces to verifying \(E_1(W)=0\) plus tower stability \(\DiagZero\) and budget dominance \(\mathrm{gap}_\tau>\Sigma\delta\).
This is a \emph{window-local} equivalence; global equivalence is not asserted.

\paragraph{Control \(\Rightarrow\) Overlap Gate (Iwasawa interface).}
Using Chapter~7, Proposition~\textup{\ref{prop:control-overlap-gate}} and Appendix~R, control theorems translate to the Overlap Gate: finite kernel/cokernel contributions are absorbed into \(\delta^{\mathrm{alg}}\) in the ledger, with explicit bounds recorded as \texttt{control\_finite\_bounds}. Layered \(\delta\)-boxes \((\delta^{\Gal},\delta^{\Tr},\delta^{\Fun})\) are mandatory (Chapter~9).

\paragraph{Indicators and gate.}
Compute \(\Ttau\,\mathbf{P}_i(F_s)\), \(\mathrm{PE}_i^{\le\tau}\), \(\mathrm{ST}_\beta^{\ge M(\tau)}\), \(\mathrm{HT}(t;\cdot)\) on \(L(\Ctau F_s)\) per window.
Gate requires (after collapse, on \(W\)): \(\PH_1=0\), \(\Ext^1=0\) (by \(E_1=0\)), \(\DiagZero\), and \(\mathrm{gap}_\tau>\Sigma\delta\).

\paragraph{Triggers ([Spec]).}
\begin{itemize}
  \item \emph{Rank-proxy surge:} persistent increase of rank proxies under deletion-type updates.
  \item \emph{Local inconsistency:} instability of local conditions under small parameter moves.
  \item \emph{PF/BC deviations:} policy violations in bad reduction handling or height normalization.
\end{itemize}

\paragraph{Window certificate (BSD).}
A certificate for \((W,\tau)\) contains: single-layer persistence objects, spectral proxies on \(L(\Ctau F_s)\), the obstruction log with failure types, Ext checks (by \(E_1=0\) on \(W\)), and the \(\delta\)-ledger.
The manifest includes: prime \(p\), family parameters, local policy for \(S\), height cutoff \(H\), Laplacian normalization, layered \(\delta\)-boxes, and seeds for deterministic matching.

\noindent A minimal \texttt{run.yaml} BSD/Iwasawa block (IMRN/AiM-ready):

\begin{verbatim}
quantale:
  name: "[0,inf]_plus"
  op: "+"
  unit: 0.0
  order: "<="
definable:
  o_minimal_structure: "R_an,exp"
  window_formulae:
    - "u <= t < u'"
  p_adic:
    structure: "Denef-Pas"
    window_formulae:
      - "val(x) in [a,b) and ac_n(x)=c"
layered_delta:
  deltaGal: 0.004
  deltaTr:  0.003
  deltaFun: 0.002
iwasawa:
  tower_level: ["N=1","N=2","N=4","N=8"]
  control_finite_bounds:
    kernel_le: 2
    cokernel_le: 2
windows:
  domain: [[1e3,2e3), [2e3,3e3)]
  collapse_tau: 0.12
  spectral_bins: {a: 0.0, beta: 0.04, bins: 128, boundary: "left-open"}
family:
  curve: "E: y^2 = x^3 - x"
  prime_p: 3
  twists: {type: quadratic, d_range: [1, 1000], parity_filter: "even"}
  height_cutoff: 14.0
local_policy:
  S: ["p=3","p=5","infty"]
  conditions: {relaxation: "bounded", penalty_step: 0.5}
awfs_2cell:
  awfs_enabled: true
  two_cell_bounds: {delta_upper: 0.006}
operations:
  - U: tighten_local; type: deletion; tau: 0.12; delta: {alg:0.002, disc:0.001, meas:0.001}
  - U: twist_step;    type: epsilon;  tau: 0.12; eps: 0.005; delta: {alg:0.000, disc:0.002, meas:0.001}
persistence:
  E1_zero_window: true
  PH1_zero: true
  Ext1_zero: true
  mu: 0
  nu: 0
  phi_iso_tail: true
spectral:
  aux_bars_remaining: 0
budget:
  delta_total: 0.009
  safety_margin: 0.021
gate:
  accept: true
\end{verbatim}

\begin{remark}[Guard-rails for BSD]
The protocol monitors persistence-layer stability under fixed policies and logs reproducible manifests.
It does not prove BSD, nor does it identify algebraic rank; rank proxies and spectral tails are diagnostics only.
All claims remain persistence/spectral/categorical and policy-dependent; PH/Ext equivalence is window-local via \(E_1(W)=0\).
\end{remark}

\subsection*{10.16. Application III: RH up to \texorpdfstring{$T$}{T} (template)}
We provide a verification\hyp style template to monitor windows in height and produce reproducible certificates.\ \emph{No RH claim is made}; zero\hyp locating or discrepancy detection is carried out at the persistence/spectral layer under fixed sampling/smoothing policies.

\paragraph{Admissible realization \(\mathcal{P}_{\mathrm{RH}}\).}
Let the state record: a height window \(W=[u,u')\), sampling mesh \(\Delta t\), smoothing kernel and bandwidth, and a normalization policy for \(\xi\), \(S(t)\), or explicit\hyp formula residuals. Construct \(F=\mathcal{P}_{\mathrm{RH}}(W)\) via:
\begin{itemize}
  \item \emph{Gram\hyp graph pipeline:} nodes at mesh/Gram points; edges connect neighbors; filtration by discrepancy thresholds \(|S(t)-S_{\mathrm{ref}}(t)|\).
  \item \emph{Scalar pipeline:} sub/superlevel filtrations of smoothed \(|\zeta(1/2+it)|\), \(|\xi|\), or residual fields.
  \item \emph{Hybrid:} couple zero\hyp candidate events with discrepancy fields; evaluate degreewise.
\end{itemize}
Deletion\hyp type: convolution smoothing; restriction to subwindows; projection to bandlimited subspaces. Epsilon\hyp continuation: small height steps under Nyquist control; inclusion: window enlargement or mesh refinement.

\paragraph{Indicators and gate.}
Compute \(\Ttau\,\mathbf{P}_i(F_s)\), \(\mathrm{PE}_i^{\le\tau}\), \(\mathrm{ST}_\beta^{\ge M(\tau)}\), \(\mathrm{HT}(t;\cdot)\) on \(L(\Ctau F_s)\) per window, with fixed normalization and bandlimit policies.
Gate: PH1\(=0\), Ext1\(=0\) (for test objects reflecting normalization checks), \(\DiagZero\), and \(\mathrm{gap}_\tau>\Sigma\delta\).

\paragraph{Triggers ([Spec]).}
\begin{itemize}
  \item \emph{Argument anomaly:} excursions of \(S(t)\) beyond tolerances, or explicit\hyp formula residual spikes.
  \item \emph{Sampling under\hyp resolution:} Nyquist/bandlimit violations for the chosen kernel/bandwidth.
  \item \emph{PF/BC deviations:} contour/grid normalization mismatches (e.g.\ Gram grid drift) or policy inconsistencies.
\end{itemize}

\paragraph{Window certificate (RH).}
A certificate for \((W,\tau)\) contains: single\hyp layer persistence, spectral proxies on \(L(\Ctau F_s)\), obstruction counts and failure types, Ext checks, and a \(\delta\)\hyp ledger. The manifest includes: mesh \(\Delta t\), kernel/bandwidth, bandlimit/Nyquist checks, normalization policy, and deterministic seeds.

\noindent A minimal \texttt{run.yaml} RH block:

\begin{verbatim}
quantale:
  name: "[0,inf]_plus"
  op: "+"
  unit: 0.0
  order: "<="
definable:
  structure: "o-minimal"
  window_formulae:
    - "u <= t < u'"
layered_delta:
  deltaGal: 0.0
  deltaTr:  0.0
  deltaFun: 0.0

windows:
  domain: [[1e9,1e9+5e5), [1e9+5e5, 1e9+1e6)]
  collapse_tau: 0.06
  spectral_bins: {a: 0.0, beta: 0.03, bins: 256, boundary: "right-open"}
sampling:
  dt: 1.0e-3
  bandlimit: 3000.0
  nyquist_check: true
smoothing:
  kernel: "gaussian"
  bandwidth: 2.5e-3
operations:
  - U: smooth;     type: deletion; tau: 0.06; delta: {alg:0.001, disc:0.002, meas:0.001}
  - U: heightstep; type: epsilon;  tau: 0.06; eps: 0.004; delta: {alg:0.000, disc:0.002, meas:0.001}
persistence:
  PH1_zero: true
  Ext1_zero: true
  mu: 0
  nu: 0
  phi_iso_tail: true
spectral:
  aux_bars_remaining: 0
budget:
  sum_delta: 0.006
  safety_margin: 0.018
gate:
  accept: true
\end{verbatim}

\begin{remark}[Guard\hyp rails for RH]
The protocol verifies stability of persistence/spectral indicators under fixed sampling/smoothing and normalization policies and produces reproducible manifests.
It does not assert the Riemann Hypothesis, nor count zeros; it only monitors windowed diagnostics with logged tolerances.
\end{remark}

% ============================================================
% Cross-application integration and auditability
% ============================================================

\subsection*{10.17. Cross\hyp application gate, triggers, and pasting}
All three applications (PDE, BSD rank~0/1, RH up to \(T\)) share:
\begin{itemize}
  \item \textbf{Gate B\hyp Gate\(^{+}\):} single\hyp layer decisions on \(\Ttau\mathbf{P}_i\) with PH1/Ext1/(\(\mu,\nu\))/safety\hyp margin.
  \item \textbf{Triggers:} blow\hyp up signs, tower accumulation, PF/BC deviations (domain\hyp specific specializations).
  \item \textbf{Restart/Summability:} window pasting with budgeted \(\Sigma\delta\) and geometric decay options for \(\tau,\beta\).
  \item \textbf{Reproducibility:} unified manifest keys (window domain, collapse\_tau, spectral bins, operations with \(\delta\)\hyp ledger, persistence verdicts, spectral auxiliaries, budget, gate).
\end{itemize}
Domain\hyp specific policies (normalization, local conditions, sampling/bandlimit) are fixed per run and logged; spectral indicators are always computed on \(L(\Ctau F)\).

\subsection*{10.18. Effect and operational readiness}
The templates make \emph{immediate operational deployment} possible: each application ships with (i) a gate specification, (ii) a trigger pack, (iii) a run manifest schema, and (iv) a window certificate format, yielding clear outcomes (certificate + reproducible logs).
Relative to v16.0, the deliverables are explicit and auditable.

\subsection*{10.19. Final guard\hyp rails (IMRN/AiM\hyp style)}
\begin{remark}[Non\hyp equivalences and scope]
All interleaving/Lipschitz/monotonicity claims are asserted at the persistence layer and, when stated for spectral proxies, under a fixed normalization policy on \(L(\Ctau F)\). Ext tests are scope\hyp restricted to a finite \(\Qtest\).
No analytic equivalences (e.g.\ BSD, RH, PDE regularity) are claimed or used. The program provides certificate\hyp style diagnostics with reproducible manifests and budgeted stability, suitable for audit and re\hyp execution.
\end{remark}

% ============================================================
% PoC reinforcements requested (PDE Burgers/NS, BSD Overlap, RH explicit formula)
% ============================================================

\subsection*{10.20. PoC I (PDE): Burgers / 2D--NS and a dissipation certificate}
\label{subsec:10.20}
\begin{definition}[Dissipation ECF and collapse witness]
\label{def:10-ecf}
For a window \(W=[u,u')\) and scale \(\tau\), define the \emph{energy--cumulative dissipation} (ECF)
\[
\mathrm{ECF}_\tau(W)\ :=\ \int_{u}^{u'} \mathcal{D}_\tau(s)\,ds,\qquad
\mathcal{D}_\tau(s)\ \coloneqq\ \mathrm{PE}_i^{\le\tau}(F_{s^-})-\mathrm{PE}_i^{\le\tau}(F_{s^+})\ \ge 0,
\]
where \(s^-,s^+\) denote instants immediately before/after a deletion\hyp type step at \(s\).
The \emph{collapse witness} is
\(
\mathrm{Col}_\tau(W):=\sum_i \bigl(\mathrm{bars}_{i,\tau}(u)-\mathrm{bars}_{i,\tau}(u')\bigr)\ge 0.
\)
\end{definition}

\begin{declaration}[PoC inequality (Spec)]
\label{spec:10-ecf-mu}
For Burgers/2D\hyp NS pipelines under the admissible operations of §10.1 and fixed normalization on \(L(\Ctau F)\), there exists a constant \(C(\tau)\ge 0\) (policy\hyp dependent) such that on any window \(W\)
\[
\muc(W)\ \le\ C(\tau)\,\mathrm{ECF}_\tau(W)\qquad\text{and}\qquad
\mathrm{Col}_\tau(W)\ \le\ C(\tau)\,\mathrm{ECF}_\tau(W).
\]
This is a \textbf{[Spec]} certificate: it is logged and verified numerically; it is not claimed as an analytic identity.
\end{declaration}

\begin{example}[Burgers shock--smoothing window]
On \(W\) containing a mollify\(\rightarrow\)advect cycle with CFL below bound, \(\mathrm{ECF}_\tau(W)>0\) and \(\mathrm{Col}_\tau(W)>0\).
If \(\sum_k\mathrm{ECF}_\tau(W_k)<\infty\) over a MECE cover, Restart/Summability (Chapter~4) yields global bar shortening at scale \(\tau\).
\end{example}

\subsection*{10.21. PoC II (BSD rank 0/1): Overlap globalization of the \(E_1\)-bridge}
\label{subsec:10.21}
\begin{declaration}[Spec: Overlap-based globalization of window certificates]
\label{spec:10-overlap-global}
Let \(\{W_k\}\) be a MECE cover by definable windows with \(E_1(W_k)=0\) for all \(k\).
Assume the Overlap Gate holds on pairwise overlaps and \(\sum_k \Sigma\delta_k<\infty\).
Then on \(\bigcup_k W_k\) we have \(\PH_1(\Ctau F)=0\) and \(\Ext^1(\Rfun(\Ctau F),k)=0\) at scale \(\tau\).
\end{declaration}

\begin{remark}[Sketch][Proof sketch]
By §10.0b, \(E_1(W_k)=0\) implies \(\PH_1(\Ctau F|_{W_k})=0\) iff \(\Ext^1(\Rfun(\Ctau F|_{W_k}),k)=0\).
The Overlap Gate identifies tails on overlaps up to the ledger budget; Restart/Summability pastes window certificates, yielding global vanishing at \(\tau\).
\end{remark}

\subsection*{10.22. PoC III (RH): explicit formula as trace, window\hyp local Weil positivity}
\label{subsec:10.22}
\begin{declaration}[Explicit\hyp formula comparator (T–PFBC–AfterCollapse)]
\label{spec:10-rh-pfbc}
Let \(\Phi_{\mathrm{EF}}\) denote the explicit\hyp formula transform and \(\Phi_{\mathrm{Tr}}\) the trace comparator under a fixed normalization policy.
All distance/energy comparisons between \(\Phi_{\mathrm{EF}}\) and \(\Phi_{\mathrm{Tr}}\) are evaluated \emph{only after} \(\Ttau\) (cf.\ Chap.~9, Spec.~\ref{spec:9-T-PFBC-after}); residuals are booked in \(\delta_{\disc}\oplus\delta_{\meas}\).
\end{declaration}

\begin{remark}[Weil\hyp type positivity, window\hyp local]
Fix a test function \(\varphi\) in the allowed class; the quadratic form \(Q(\varphi)\ge 0\) is verified \emph{window\hyp locally} by checking the corresponding spectral proxy on \(L(\Ctau F)\) and recording residuals in the ledger.
No global spectral statement is asserted.
\end{remark}

\subsection*{10.23. p\hyp adic interface (GL(1)\(\to\)GL(2))}
\label{subsec:10.23}
\paragraph{Local kernels and AK measurement.}
For GL(1), use Tate integrals/Igusa local zeta to define local kernels \(K_p\) that act as deletion\hyp type or \(\varepsilon\)\hyp continuation steps depending on cutoffs; their effects are measured on \(\Ttau\mathbf{P}_i\) and charged to the \(\delta\)\hyp ledger.
For GL(2) at small conductor, introduce local test functions stagewise; PF/BC comparators with Hecke kernels are evaluated only after collapse (T–PFBC–AfterCollapse).

\noindent A minimal \texttt{run.yaml} p\hyp adic block:

\begin{verbatim}
padic:
  primes: [3,5,7]
  local_kernels:
    - {p: 3, type: "Tate", cutoff: 9, action: "deletion"}
    - {p: 5, type: "Igusa", cutoff: 7, action: "epsilon"}
gl2:
  conductor: 64
  hecke_normalization: "unitary"
  stage_intro: ["T_p", "T_p^2", "U_p"]
operations:
  - U: local_Tate_p3; type: deletion; tau: 0.10; delta: {alg:0.001, disc:0.002, meas:0.001}
  - U: local_Igusa_p5; type: epsilon;  tau: 0.10; eps: 0.003; delta: {alg:0.000, disc:0.001, meas:0.001}
persistence:
  PH1_zero: true
  Ext1_zero: true
  mu: 0
  nu: 0
gate:
  accept: true
\end{verbatim}

\subsection*{10.24. Chapter summary (PoC addendum)}
The PoC additions (PDE ECF certificate, BSD Overlap globalization, RH explicit\hyp formula comparator, p\hyp adic local kernels) are fully \emph{after\hyp collapse}, windowed, and budgeted by the \(\delta\)\hyp ledger.
They integrate with B\hyp Gate\(^{+}\), Restart/Summability, and T–PFBC–AfterCollapse, with no analytic number theory or PDE regularity claims beyond the persistence/spectral/categorical guard\hyp rails of v16.0.



\section{Chapter 11: Collapse Energy, Spectral Indicators, and TDA Notes}
\addcontentsline{toc}{section}{Collapse Energy, Spectral Indicators, and TDA Notes}
\label{sec:ch11}

\subsection*{11.0. Scope, standing hypotheses, and notation}
We work within the \emph{implementable range} of Part~I, using realizations into $\FiltCh{k}$ as in Chs.~6–10. All persistence quantities are computed degreewise \emph{after} truncation by $\Ctau$; spectral indicators are computed on the normalized combinatorial Hodge Laplacian of $\Ctau F$; categorical checks use a fixed $t$-exact $\Rfun:\FiltCh{k}\to\Db(k\text{-mod})$ compatible with \LC. Filtered (co)limits, when used, are computed objectwise in $[\mathbb{R},\mathsf{Vect}_k]$ and only under the scope policy of Appendix~A (compute in the functor category and verify return to $\Pers^{\mathrm{cons}}_k$). No claim of $\mathrm{PH}_1\Leftrightarrow\Ext^1$ is made; only the one-way bridge under (B1)–(B3) is used. The obstruction pair $(\DiagZero / \DiagNonzero)$ is the \emph{collapse} diagnostic and is unrelated to the classical Iwasawa $\mu$. We adopt the global $\delta$-policy $\boldsymbol{\delta}=(\delta_{\mathrm{int}},\delta_{\mathrm{win}},\delta_{\mathrm{spec}})$ and the additive ledger on a fixed commutative quantale $V$ (Appendix~S); layered boxes $(\delta^{\Gal},\delta^{\Tr},\delta^{\Fun})$ are mandatory (Ch.~9).

\begin{remark}[Endpoints and infinite bars]\label{rk:11-endpoints}
Open/closed endpoint conventions are immaterial; infinite bars are not removed by $\Ttau$ and are clipped by the window $\tau$ in all windowed quantities (cf.\ Ch.~6).
\end{remark}

\subsection*{11.0+. First-class determinant \(E_1\); definable windows and finite events}
On right-open windows $W$ definable in a fixed o-minimal (Archimedean) or Denef–Pas (non-Archimedean) structure, Betti integrals are piecewise constant with finitely many jumps and Čech depth is finite; see Appendices~H and~J. In this regime, the Page-$E_1$ term is our \emph{first-class determinant}:

\begin{theorem}[Local bridge on definable windows; reprise of Ch.~3]\label{thm:11-E1-local-reprise}
Let $W$ be definable and right-open. Then
\[
E_1(W)=0 \ \Longrightarrow\ \PH_1(\Ctau F|_W)=0\ \text{ and }\ \Ext^1\!\big(\Rfun(\Ctau F|_W),k\big)=0.
\]
The implication is window-local and after collapse; no global equivalence is asserted.
\end{theorem}

\begin{remark}[Unique comparison order]\label{rk:11-order}
All measurements obey the \emph{single} order
\[
\boxed{\ \text{for each } t\ \Longrightarrow\ \mathbf{P}_i \ \Longrightarrow\ \mathbf{T}_\tau \ \Longrightarrow\ \text{compare in }\Pers^{\mathrm{cons}}_k\ }.
\]
Pre-collapse comparisons are out of scope. Overlap checks use the same order (Overlap Gate; Chs.~1 \& 5).
\end{remark}

\subsection*{11.0++.\ \(\Lambda_{\mathrm{len}}\) audit (default) and T–PFBC–AfterCollapse}
Throughout this chapter we \emph{adopt by default} the length-spectrum audit $\Lambda_{\mathrm{len}}$ (Ch.~2, Def.~\ref{def:len-operator}) for stability-band diagnostics. Any PF/BC-based comparator or transfer (Appendix~N) is evaluated only \emph{after} truncation (\textbf{T–PFBC–AfterCollapse}; cf.\ Ch.~9), and residuals are charged to $\delta_{\disc}\oplus\delta_{\meas}$ in the ledger (Appendix~L).

\subsection*{11.1. Length spectrum and invariance}
Let $\Lambda_{\mathrm{len}}(M;W)$ denote the length-spectrum operator of Ch.~2, Definition~\ref{def:len-operator} (we also write $\Len(M;W)$).

\begin{proposition}[Length spectrum equals clipped bar lengths; invariance]\label{prop:11-len}
If $M\simeq \bigoplus_j I[b_j,d_j)$ in $\Pers^{\mathrm{cons}}_k$ and $W=[u,u')$, then the eigenvalue multiset of $\Lambda_{\mathrm{len}}(M;W)$ is $\{\ell_W(I[b_j,d_j))\}_j$, with $\ell_W$ the Lebesgue length of $I[b_j,d_j)\cap W$. Hence the total collapse energy $\mathrm{PE}^{\le\tau}_i$ (Definition~\ref{def:11-PE}) equals the $L^1$-mass of $\Lambda_{\mathrm{len}}\!\big(\Ttau\mathbf{P}_i(F);[0,\tau]\big)$ and is invariant under isomorphisms in $\Pers^{\mathrm{cons}}_k$.
\end{proposition}

\begin{definition}[Stability bands via \(\Lambda_{\mathrm{len}}\)]\label{def:11-stability-band}
Fix $(i,\tau)$ and a window $W$. The \emph{$\Lambda_{\mathrm{len}}$-stability band} on $W$ is the maximal subunion $B\subseteq W$ such that
\[
\big\|\Lambda_{\mathrm{len}}(\Ttau\mathbf{P}_i(F_t);[0,\tau])-\Lambda_{\mathrm{len}}(\Ttau\mathbf{P}_i(F_{t'});[0,\tau])\big\|_{1}\ \le\ \delta_{\mathrm{win}}
\quad\text{for all }t,t'\in B.
\]
Here $\|\cdot\|_1$ is the trace (sum of eigenvalues) norm; the policy fixes the norm choice in the manifest.
\end{definition}

\subsection*{11.2. Collapse energy (windowed persistence energies)}
Fix degree $i$, scale $\tau\ge 0$, and exponent $\alpha>0$ (default $\alpha=1$). For a bar $b=[b_\ell,b_r)$ in the barcode of $\mathbf{P}_i(F)$ set
\[
\ell_\tau(b)=\bigl(\min\{b_r,\tau\}-\min\{b_\ell,\tau\}\bigr)_+,\qquad (x)_+=\max\{x,0\}.
\]
With weights $w_i:\mathcal{B}_i(F)\to[0,\infty)$ (default $1$):
\begin{definition}[Windowed energies]\label{def:11-PE}
\[
\mathrm{PE}^{\le\tau}_i(F;w_i,\alpha)=\sum_{b\in\mathcal{B}_i(F)}w_i(b)\,\big(\ell_\tau(b)\big)^\alpha,\qquad
\mathbf{CE}^{\le\tau}(F)=\big(\mathrm{PE}^{\le\tau}_i(F)\big)_i,\ \ \|\mathbf{CE}^{\le\tau}(F)\|_1=\sum_i\mathrm{PE}^{\le\tau}_i(F).
\]
All quantities are computed on $\Ttau\mathbf{P}_i(F)$ (equivalently on $\Ctau F$).
\end{definition}

\begin{remark}[Stability and deletion-type monotonicity]\label{rk:11-PE-stab}
$1$-Lipschitz updates under $\mathbf{P}_i$ yield non-expansive changes of $\mathrm{PE}^{\le\tau}_i$; \emph{deletion-type} updates (Appendix~E) make $\mathrm{PE}^{\le\tau}_i$ \emph{non-increasing} up to \fqi. Since $\Ttau$ is $1$-Lipschitz for interleaving distance, the same bounds hold after truncation. Under the tower hypotheses (Ch.~4), $\DiagZero$ at fixed $\tau$ excludes Type~IV.
\end{remark}

\subsection*{11.3. Spectral indicators on \(L(\Ctau F)\)}
Let $L(\Ctau F)$ be the normalized combinatorial Hodge Laplacian (per degree). Write $\{\lambda_j\}_{j\ge1}$ for its positive eigenvalues (zero modes omitted or handled by the Moore–Penrose pseudoinverse).

\begin{definition}[Spectral tails and heat traces]\label{def:11-spectral}
Fix $\beta>0$ and a cutoff policy $M(\tau)\in\mathbb{N}$. Set
\[
\mathrm{ST}^{\ge M(\tau)}_\beta(F)=\sum_{j\ge M(\tau)}\lambda_j^{-\beta},\qquad
\mathrm{HT}(t;F)=\sum_{j\ge1}e^{-t\lambda_j},\ \ t>0,
\]
with policies such as: (i) $M(\tau)=\lfloor c\,\tau^\gamma\rfloor$ ($c>0$, $\gamma\in[0,2]$); (ii) $t\in[c_1\tau^{-2},c_2\tau^{-2}]$ ($0<c_1\le c_2$).
\end{definition}

\begin{remark}[Spectral proof obligations; App.~E only]\label{rk:11-spectral-proof}
We rely solely on Appendix~E: deletion-type steps imply \emph{non-increase} of $\mathrm{ST}_\beta^{\ge M(\tau)}$ and $\mathrm{HT}(t;-)$; $\varepsilon$-continuations imply \emph{stability}. No further spectral claims are used.
\end{remark}

\begin{remark}[Mandatory ordering and norms]\label{rk:11-spectral-order}
Eigenvalues are stored in \emph{ascending} order; the matrix norm for tolerances is logged as \texttt{norm}$\in\{\texttt{op},\texttt{fro}\}$ (Appendix~G).
\end{remark}

\subsection*{11.4. Auxiliary spectral bars (aux-bars)}
Fix a spectral window $[a,b]$ and bin width $\beta>0$; bins are right-open $I_r=[a+r\beta,a+(r+1)\beta)$, $r=0,\dots,R-1$.

\begin{definition}[Occupancies and aux-bars]\label{def:11-aux}
For sample index $j$ (e.g.\ time) let $\{\lambda_m(j)\}_{m\ge1}$ be the positive spectrum of $L(\Ctau F_j)$ and $E_r(j)=\#\{m:\lambda_m(j)\in I_r\}$, with under/overflow $E_{<a},E_{\ge b}$ recorded. For each $r$, an \emph{aux-bar} is a maximal consecutive run $J$ with $E_r(j)>0$ for all $j\in J$; its lifetime is $|J|$ (or a rescaling).
\end{definition}

\begin{proposition}[Cumulative profile monotonicity and stability]\label{prop:11-aux}
For $C_r(j)=\sum_{s=r}^{R-1}E_s(j)$:
\begin{enumerate}
\item (\emph{Deletion-type}) $C_r(j\!+\!1)\le C_r(j)$ for all $r$.
\item Let \(A_j:=L(\Ctau F_j)\).
(\emph{$\varepsilon$-continuation}) If $\|A_{j+1}-A_j\|_{\mathrm{op}}\le\varepsilon$, then
$C_{r+q}(j+1)\le C_r(j)\le C_{\max\{0,r-q\}}(j+1)$ with $q=\lceil\varepsilon/\beta\rceil$.
\end{enumerate}
(See Appendix~E.)
\end{proposition}

\begin{remark}[Bin policy]\label{rk:11-bin}
Aux-bars are computed \emph{after} collapse and with a fixed policy $(a,b,\beta)$ per window; boundary is right-open; under/overflow are part of the manifest (Appendix~G).
\end{remark}

\subsection*{11.5. Categorical check (one-way bridge)}
With $\Qtest=\{k[0]\}$, we monitor
\[
\Ext^1\big(\Rfun(\Ctau F),Q\big)=0\qquad(Q\in\Qtest),
\]
performed \emph{after truncation} and only in the one-way direction under (B1)–(B3).

\subsection*{11.6. Collapse diagnostics along towers}
For index category $I$ with cone apex $\infty$,
\[
\phi_{i,\tau}:\ \varinjlim_{t\in I}\ \Ttau\mathbf{P}_i(F_t)\ \longrightarrow\ \Ttau\mathbf{P}_i(F_\infty),
\]
\[
\mu_{i,\tau}=\dim_k\ker\phi_{i,\tau}\footnote{Here $\dim_k$ denotes the \emph{generic fiber} dimension after truncation, i.e.\ the multiplicity of $I[0,\infty)$ summands; see Appendix~D, Remark~\ref{D:rem:generic-dim}.},\quad
\nu_{i,\tau}=\dim_k\mathrm{coker}\,\phi_{i,\tau},\quad
\muc=\sum_i\mu_{i,\tau},\ \nuc=\sum_i\nu_{i,\tau}.
\]
Under the hypotheses of Ch.~4, each $\phi_{i,\tau}$ is an isomorphism and $\DiagZero$.

\subsection*{11.7. Overlap Gate and measurement}
\begin{declaration}[Overlap-aware measurement policy]\label{dec:11-overlap}
For a right-open cover $\{W_\alpha\}$ and fixed $(i,\tau)$:
\begin{enumerate}
\item \textbf{Local:} compute $\Ttau\mathbf{P}_i(F|_{W_\alpha})$ and all indicators \emph{after} truncation.
\item \textbf{Overlaps:} require collapse-compatibility within the $\delta$-budget (Appendix~L), Čech–$\Ext^1$-acyclicity in degree $1$ after truncation, and $\DiagZero$ with near-$\tau$ non-accumulation.
\item \textbf{Global:} when overlaps pass, glue to a global truncated object (Ch.~5) and audit \emph{after} truncation.
\end{enumerate}
The manifest (Remark~\ref{rk:11-impl}) must log overlap checks (boolean), Čech–$\Ext^1$ status, and overlap $\delta$-budgets.
\end{declaration}

\subsection*{11.8. Joint monitoring protocol}
\begin{declaration}[Specification: joint monitoring]\label{spec:11-monitor}
Fix a finite sweep $\tau\in[\tau_{\min},\tau_{\max}]$ and a policy $(\alpha,w_i;\ \beta,M(\tau),t)$.
For each sample $t\in I$ and degree $i$:
\begin{enumerate}
\item \emph{Compute \& record} $\Ttau\mathbf{P}_i(F_t)$; evaluate $\mathrm{PE}^{\le\tau}_i(F_t)$ on $\Ttau\mathbf{P}_i(F_t)$ (equivalently on $\Ctau F_t$).
\item \emph{Compute \& record} $\mathrm{ST}_\beta^{\ge M(\tau)}(F_t)$ and $\mathrm{HT}(t;\Ctau F_t)$ using $L(\Ctau F_t)$; compute aux-bars via Definition~\ref{def:11-aux} with fixed $(a,b,\beta)$.
\item \emph{Check} $\Ext^1\big(\Rfun(\Ctau F_t),Q\big)=0$ for $Q\in\Qtest$.
\item \emph{Evaluate} $(\DiagZero / \DiagNonzero)$ at each $\tau$ via $\phi_{i,\tau}$; log failure type (pure kernel/cokernel/mixed) if $\DiagNonzero$.
\item \emph{Declare stable} at $\tau$ when (1)–(3) hold jointly and $\DiagZero$. On definable windows, $E_1(W)=0$ short-circuits the PH/Ext checks (Theorem~\ref{thm:11-E1-local-reprise}).
\end{enumerate}
All persistence-layer statements are \fqi-invariant by construction; spectral/aux-bar steps are \emph{stable} under the fixed policy on $L(\Ctau F_t)$ (Appendix~E).
\end{declaration}

\subsection*{11.9. Noise, discretization, and $\delta$-ledger}
\begin{declaration}[Specification: noise/discretization policy]\label{spec:11-noise}
Let $\varepsilon>0$ be the noise scale.
\begin{itemize}
\item \textbf{Barcode denoising:} remove bars of length $\le\varepsilon$ within the $\tau$-window (\emph{$\varepsilon$-clipping}); bottleneck perturbations $\le\varepsilon$ preserve f.q.i.~invariants.
\item \textbf{Energy stability:} there exists $C_{i,\tau,\alpha}$ with
$\big|\mathrm{PE}^{\le\tau}_i(F)-\mathrm{PE}^{\le\tau}_i(\tilde F)\big|\le C_{i,\tau,\alpha}\,\varepsilon^{\min\{1,\alpha\}}$
whenever $d_{\mathrm{int}}(\Ttau\mathbf{P}_i(F),\Ttau\mathbf{P}_i(\tilde F))\le\varepsilon$.
\item \textbf{Spectral stabilization:} compute spectra on $L(\Ctau F)$; keep $(\beta,M(\tau),t)$ fixed. Averaging over $N$ runs reduces variance as $N^{-1/2}$. Ignore aux-bar lifetimes $\le 2$ frames if declared in the manifest.
\item \textbf{Resolution rule:} minimal resolvable feature length $\ge 3$ grid steps; sweep $\tau$ on a lattice $\Delta\tau\le\frac12$ of the minimal resolvable bar length.
\item \textbf{$\delta$-ledger:} decompose per-step
$\delta(i,\tau)=\delta_{\alg}(i,\tau)\ \otimes\ \delta_{\disc}(i,\tau)\ \otimes\ \delta_{\meas}(i,\tau)$ in \(V\),
and aggregate as $\delta_{\mathrm{tot}}=\bigotimes \delta_U$.
(For the default Lawvere quantale, \(\otimes=+\).); with layered boxes $(\delta^{\mathrm{Gal}},\delta^{\mathrm{Tr}},\delta^{\mathrm{Fun}})$ for cross-layer runs.
\end{itemize}
\end{declaration}

\subsection*{11.10. Saturation gate \textbf{[Spec]}}
\begin{declaration}[Window-local saturation]\label{gate:11-sat}
Fix $\tau^\ast>0$ and parameters $\eta,\delta>0$. On $[0,\tau^\ast]$, assume:
(i) eventually the maximal \emph{finite} bar length in $\Ttau^\ast\mathbf{P}_i(F_t)$ is $\le\eta$;
(ii) eventually $d_{\mathrm{int}}\!\big(\Ttau^\ast\mathbf{P}_i(F_t),\Ttau^\ast\mathbf{P}_i(F_{t'})\big)\le\eta$;
(iii) the edge gap $\delta=\tau^\ast-\max\{b_r<\tau^\ast\}$ satisfies $\delta>\eta$.
Then, \textbf{within this window only}, adopt the temporary binary policy
\[
\textbf{Decision rule (Spec):}\quad
E_1(W)=0\ \text{or}\ \Ext^1(\Rfun(\C_{\tau^\ast}F),k)=0\ \Longrightarrow\ \text{treat the PH-check as passed on this window.}
\]
\end{declaration}

\subsection*{11.11. Artifacts, manifests, and minimal schema}
\begin{remark}[Implementation notes]\label{rk:11-impl}
\emph{Artifacts.} (i) \texttt{bars.json/h5}: records $\langle i,b_\ell,b_r,w\rangle$; (ii) \texttt{spec.json/h5}: positive eigenvalues of $L(\Ctau F)$ per degree; (iii) \texttt{aux.json/h5}: occupancies $E_r(j)$ and bin metadata; (iv) \texttt{ext.json}: boolean for $\Ext^1(\Rfun(\Ctau F),Q)$ with minimal witness; (v) \texttt{phi.json}: ranks of $\phi_{i,\tau}$ and $(\mu_{i,\tau},\nu_{i,\tau})$. \emph{Run log.} Store: sweep $\tau_{\min}$:$\Delta\tau$:$\tau_{\max}$; $(\alpha,w_i;\beta,M(\tau),t)$; discretization (grid/complex, step sizes); seeds; software versions; $\delta$-ledger per step; bin window $[a,b]$, width $\beta$, under/overflow; mandatory spectral fields \texttt{order=ascending}, \texttt{norm}$\in\{\texttt{op},\texttt{fro}\}$; overlap checks; Čech–$\Ext^1$ status; tail-isomorphism flag; optional length-spectrum summary.
\end{remark}

\noindent Minimal \texttt{run.yaml} block (augmented):
\begin{verbatim}
quantale:
  name: "[0,inf]_plus"
  op: "+"
  unit: 0.0
  order: "<="
layered_delta: {deltaGal: 0.002, deltaTr: 0.003, deltaFun: 0.002}
windows:
  domain: [[0,1), [1,2), [2,3)]
  collapse_tau: 0.08
  spectral_bins: {a: 0.0, beta: 0.02, bins: 96, boundary: "right-open"}
coverage_check: {length_sum: 3.0, length_target: 3.0, events_sum_equals_global: true}
overlap_checks: {local_equal_after_collapse: true, cech_ext1_ok: true, stability_band_ok: true}
spectral_policy: {order: "ascending", norm: "op"}
operations:
  - U: mollify; type: deletion; tau: 0.08; delta: {alg:0.004, disc:0.003, meas:0.001}
  - U: timestep; type: epsilon;  tau: 0.08; eps:0.006; delta: {alg:0.000, disc:0.002, meas:0.001}
persistence:
  E1_zero_window: true
  PH1_zero: true
  Ext1_zero: true
  mu: 0
  nu: 0
  phi_iso_tail: true
length_spectrum: {degree: 1, tau: 0.08, eigenvalues: [0.24, 0.51, 0.78]}
spectral:
  ST_beta: 2
  ST_M_of_tau: "floor(0.5 * tau^1.5)"
  HT_t: [0.5*tau^-2, 1.0*tau^-2]
  aux_bars_remaining: 0
budget: {sum_delta: 0.011, safety_margin: 0.025}
gate: {accept: true}
\end{verbatim}

\subsection*{11.12. Compliance checks and unit tests}
\begin{declaration}[Minimal test suite]\label{spec:11-tests}
Every deployment must pass:
\begin{itemize}
\item \textbf{Stability test.} Under synthetic $\varepsilon$-perturbations, verify non-expansiveness of $\mathrm{PE}^{\le\tau}_i$ and stability of spectral indicators and aux-bars under the fixed policy.
\item \textbf{Monotone-update test.} For a deletion-type update, confirm non-increase of $\mathrm{PE}^{\le\tau}_i$, spectral tails, and active-bin mass; record $\DiagZero$ at fixed $\tau$.
\item \textbf{Cone-extension test.} Verify that $\phi_{i,\tau}$ is an isomorphism on a model tower (hence $\DiagZero$) and that Type~IV is excluded at that $\tau$.
\item \textbf{Categorical check.} On a curated sample, confirm stability of $\Ext^1(\Rfun(\Ctau F),Q)=0$ under admissible \fqi-updates.
\item \textbf{T–$\Lambda_{\mathrm{len}}$–Consistency (optional).} Check that $\big\|\Lambda_{\mathrm{len}}(\Ttau\mathbf{P}_i(F_{t});[0,\tau])-\Lambda_{\mathrm{len}}(\Ttau\mathbf{P}_i(F_{t'});[0,\tau])\big\|_{1}\le \delta_{\mathrm{win}}$ along declared stability bands (Def.~\ref{def:11-stability-band}), and that
$\mathrm{PE}^{\le\tau}_i(F_{t})=\mathrm{trace}\big(\Lambda_{\mathrm{len}}(\Ttau\mathbf{P}_i(F_{t});[0,\tau])\big)$ holds to numerical tolerance.
\end{itemize}
\end{declaration}

\subsection*{11.13. Diagram (pipeline and logs)}
\begin{center}
\begin{tikzcd}[
  ampersand replacement=\&,
  column sep=2.4em, row sep=2.0em,
  cells={nodes={font=\footnotesize}},
  every label/.append style={font=\scriptsize},
  scale=0.9, transform shape
]
\text{Input }F \arrow[r, "\mathbf{P}_i"]
  \& \mathbf{P}_i(F) \arrow[r, "\Ttau"] \arrow[d, dashed, "{\mathrm{PE}_i^{\le \tau}}"]
    \& \Ttau\mathbf{P}_i(F) \arrow[d, dashed, "{\mathrm{PE}_i^{\le \tau}}"] \\
\& L(\Ctau F) \arrow[r, dashed, "{\text{heat/aux on }L(\Ctau F)}"]
    \& \mathrm{HT},\ \mathrm{ST}_{\beta},\ \text{aux-bars} \\
\Rfun(F) \arrow[uu, bend left=40, dashed, "\LC"]
  \arrow[rr, dashed, "{\Ctau\ \text{ then }\Ext^1(-,Q)}"]
    \&\& \Ext^1(\Rfun(\Ctau F),Q)=0 \\
\varinjlim_{t}\ \Ttau\mathbf{P}_i(F_t) \arrow[rr, "{\phi_{i,\tau}}"]
  \&\& \Ttau\mathbf{P}_i(F_\infty) \arrow[lld, bend right=25, swap, dashed, "{(\DiagZero / \DiagNonzero)\ \text{log}}"] \\
\text{failure log (pure kernel/cokernel/mixed)} \&\&
\end{tikzcd}
\end{center}

\subsection*{11.14. Completion note}
\begin{remark}[No further supplementation required]\label{rk:11-done}
This chapter fully integrates: (i) $E_1$ as first-class determinant on definable windows with finite-event/finite-Čech guarantees (App.~H/J) and a reprise of the local bridge (Thm.~\ref{thm:11-E1-local-reprise}); (ii) default $\Lambda_{\mathrm{len}}$ audit for stability-band diagnostics with invariance via Prop.~\ref{prop:11-len}; (iii) spectral auxiliaries restricted to deletion-type non-increase and $\varepsilon$-stability with proofs deferred to App.~E; (iv) Overlap Gate enforcement and the unique comparison order; (v) tower diagnostics $(\DiagZero / \DiagNonzero)$ and the joint monitoring protocol; (vi) noise/discretization rules and a quantale $\delta$-ledger (with layered boxes); (vii) artifacts/manifests with mandatory audit fields; (viii) a minimal, IMRN/AiM-ready test suite including the optional T–$\Lambda_{\mathrm{len}}$–Consistency check. All claims remain in the B-side single-layer scope; no additional supplementation is needed for operational use.
\end{remark}

\subsection*{11.15. Guard-rails}
\begin{remark}[Scope and non-claims]\label{rk:11-guard}
This chapter specifies measurement protocols and auxiliaries at the persistence/spectral/categorical layers. No analytic regularity, group trivialization, or number-theoretic identity is asserted. All statements respect the guard-rails of Part~I; in particular, no claim of $\mathrm{PH}_1\Leftrightarrow\Ext^1$ is made, and $\muc$ differs from the classical Iwasawa $\mu$.
\end{remark}



\section{Chapter 12: Formal Test Suite and Open Problems}
\addcontentsline{toc}{section}{Formal Test Suite and Open Problems}
\label{sec:ch12}

% --- Badge policy (explicit at chapter head) ---
\noindent\textbf{Badge policy.}
\begin{itemize}
  \item \textbf{[Prop]}: mathematics proved in Part~I (core results; cite exact source).
   \item \textbf{[Thm]}: same as \textbf{[Prop]} but stated at theorem-level (cite exact source).
  \item \textbf{[Declaration]}: programmatic specification in the implementable range, verifiable by the test suite in this chapter.
  \item \textbf{[Conjecture]}: forward-looking statement; no claim beyond the stated scope.
\end{itemize}

% --- Notation & Conventions block for the whole chapter ---
\subsection*{12.0. Notation \& conventions}
\begin{itemize}
  \item \textbf{Constructible range.} We identify $\Perskft$ with the constructible subcategory of $[\mathbb{R},\mathsf{Vect}_k]$ and use $\Perskft$ uniformly.
  \item \textbf{Truncation phrase.} “after applying $\mathbf{T}_\tau$; equivalently on $\Ctau F$’’ indicates computation at the persistence layer after truncation (hence equivalently on the filtered lift $\Ctau F$).
  \item \textbf{Comparison order (unique).} All PF/BC and metric/spectral comparisons use the \emph{single} order
  \[
  \text{for each } t\ \Longrightarrow\ \mathbf{P}_i\ \Longrightarrow\ \mathbf{T}_\tau\ \Longrightarrow\ \text{compare in }\Perskft,
  \]
  i.e.\ \textbf{T–PFBC–AfterCollapse}. Pre\hyp collapse comparisons are out of scope (Chs.~1, 9, 11).
  \item \textbf{Generic–fiber dimension.} For a comparison map $\phi_{i,\tau}$ at fixed $\tau$, $\dim_k$ denotes the \emph{generic–fiber dimension after truncation}, i.e.\ the multiplicity of $I[0,\infty)$ summands in $\Ttau\mathbf{P}_i(-)$; informally, the $t\to\infty$ stable rank within the $\tau$–window.
  \item \textbf{Spectral ordering and norms.} Positive eigenvalues of $L(\Ctau F)$ are listed in ascending order $\lambda_1\le \lambda_2\le\cdots$. Matrix/operator norms are $\|\cdot\|_{\mathrm{op}}$ and $\|\cdot\|_{\mathrm{fro}}$; each test declares and logs its choice.
  \item \textbf{Obstruction totals and macros.} $\mu_{i,\tau}=\dim_k\ker\phi_{i,\tau}$, $\nu_{i,\tau}=\dim_k\mathrm{coker}\,\phi_{i,\tau}$, totals $\muc=\sum_i\mu_{i,\tau}$, $\nuc=\sum_i\nu_{i,\tau}$ (finite by bounded degree).
  \item \textbf{Endpoints.} Endpoint conventions and infinite bars follow the global policy (Appendix~A); infinite bars are not removed by $\Ttau$ and are clipped by the window in all windowed quantities.
  \item \textbf{Non\hyp expansiveness.} We use the spelling “non\hyp expansive’’/“non\hyp expansiveness’’ uniformly.
  \item \textbf{Quantale \& $V$-enrichment.} All $\delta$-budgets live in a fixed commutative quantale $V$ (Appendix~S); $V$-Lipschitz and $V$-nucleus properties are tested in~T16. Layered boxes $(\delta^{\Gal},\delta^{\Tr},\delta^{\Fun})$ are mandatory (Ch.~9).
  \item \textbf{Definable windows.} Windows $W$ are definable in a fixed o\hyp minimal (Archimedean) or Denef–Pas (non\hyp Archimedean) structure; definability and finite Čech depth are tested in~T17.
  \item \textbf{Quantale convention.} We write $\otimes$ for the quantale operation (YAML \texttt{op});
in the default Lawvere case, $\otimes=+$.
\end{itemize}

\subsection*{12.1. Badge inventory (representative items)}
\begin{center}
\begingroup
\renewcommand{\arraystretch}{1.12}
\begin{tabular}{@{}l >{\raggedright\arraybackslash}p{0.74\textwidth}@{}}
\toprule
\textbf{Badge} & \textbf{Representative items (label / location)}\\
\midrule
\textbf{[Prop]} &
  Stability, idempotence, and exactness of $\Ttau$ (Prop.~\ref{prop:stability}, Ch.~2);\\
& Shift–commutation / $1$\nobreakdash-Lipschitz for $\Ttau$ (Lemma~\ref{lem:Vshift}, Ch.~2);\\
& Operational coreflection $\mathsf{C}_\tau^{\mathrm{comb}}$ on the implementable range
  (Prop.~\ref{prop:operational-coreflection}, Ch.~5);\\
& Tower diagnosis: $(\DiagZero / \DiagNonzero)$ via cone extension; isomorphism criterion excluding Type~IV (Prop.~\ref{J:prop:diagzero}, Appendix~J).\\
\textbf{[Thm]} &
  One\hyp way bridge: $\mathrm{PH}_1(F)=0 \Rightarrow \Ext^1(\mathcal{R}(F),k)=0$ under (B1)–(B3) (Thm.~\ref{thm:PH1-to-Ext1}, Ch.~3);\\
& Local bridge on definable windows (Thm.~\ref{thm:E1-local}, Ch.~3; reprise Ch.~11).\\
\textbf{[Declaration]} &
  Ch.~2: (co)limit and pullback compatibility \emph{at the persistence layer only} (after $\Ttau$);\\
& Ch.~6: filtered\hyp colimit stability in geometry; joint indicators and protocol (after truncation);\\
& Ch.~7: arithmetic tower stability; non\hyp identity of $\muc$ with Iwasawa $\mu$;\\
& Ch.~8: tropical shortening $\Rightarrow$ weak group collapse; mirror transfer \emph{non\hyp expansive after truncation};\\
& Ch.~9: three\hyp layer (Gal$\to$Trans$\to$Funct) compatibility as isomorphisms in $\Perskft$ \emph{after} $\Ttau\mathbf{P}_i$;\\
& Ch.~10: persistence\hyp guided regularization; AK–NS hypothesis (programmatic);\\
& Ch.~11: joint monitoring, noise/discretization policy, minimal test suite; Saturation gate.\\
\textbf{[Conjecture]} &
  Cross\hyp domain collapse propagation (Chs.~6–10); AK–NS (Ch.~10); mirror\hyp side propagation (Ch.~8);\\
& Functorial transfer stability (Ch.~9).\\
\bottomrule
\end{tabular}
\endgroup
\end{center}

\subsection*{12.2. Formal test suite (unit / integration / regression)}
All tests operate at the truncated persistence, spectral (on $L(\Ctau F)$), and categorical layers and are \fqi\ invariant at the persistence layer. A test \emph{passes} iff all stated pass\hyp criteria are met and logs are complete. Pass\hyp criteria must state whether indicators are evaluated \emph{per degree} or \emph{aggregated} across degrees; the choice must be fixed and logged for the run. Spectra use ascending order $\lambda_1\le\lambda_2\le\cdots$, and the chosen norm $\|\cdot\|_{\mathrm{op}}$ or $\|\cdot\|_{\mathrm{fro}}$ must be declared and logged.

\paragraph{(T1) Stability under non\hyp expansive updates [Unit].}
\emph{Input:} pairs $F\to F'$ with $d_{\mathrm{int}}(\mathbf{P}_i(F),\mathbf{P}_i(F'))\le \varepsilon$.\\
\emph{Assertions:} $|\mathrm{PE}^{\le\tau}_i(F)-\mathrm{PE}^{\le\tau}_i(F')|\le C_{i,\tau,\alpha}\varepsilon^{\min\{1,\alpha\}}$ (after $\Ttau$; equivalently on $\Ctau$); spectra of $L(\Ctau F)$ vs.\ $L(\Ctau F')$ satisfy the fixed $(\beta,M(\tau),t)$\hyp policy stability bounds in the declared norm; $\Ext^1(\Rfun(\Ctau-),Q)$ is stable under admissible \fqi\ updates ($Q\in\Qtest$).\\
\emph{Artifacts:} \texttt{bars.json}, \texttt{spec.json}, \texttt{ext.json}; \texttt{run.yaml} (norm and spectral policy recorded).

\paragraph{(T2) Monotone update (deletion–/inclusion–type) [Unit].}
\emph{Input:} $F\to F'$ monotone.\\
\emph{Assertions:} \emph{Deletion–type:} $\mathrm{PE}^{\le\tau}_i$ and spectral indicators are non\hyp increasing (after $\Ttau$).In the synthetic two-term cone test case satisfying the Ch.~4 cone-extension hypotheses,
the map $\phi_{i,\tau}$ is an isomorphism, hence $\DiagZero$ at fixed $\tau$.
\emph{Inclusion–type:} stability only.\\
\emph{Artifacts:} \texttt{bars.json}, \texttt{spec.json}, \texttt{ext.json}, \texttt{phi.json}; \texttt{run.yaml}.

\paragraph{(T3) Filtered\hyp colimit stability [Integration].}
\emph{Input:} tower $\{F_\lambda\}_\lambda$.\\
\emph{Assertions:} for fixed $\tau$, $\phi_{i,\tau}:\varinjlim_\lambda \Ttau\mathbf{P}_i(F_\lambda)\xrightarrow{\cong}\Ttau\mathbf{P}_i(F_{\lambda_\ast})$; thus $\DiagZero$ at that scale. Terminal symbol consistency is logged.\\
\emph{Artifacts:} \texttt{phi.json}, \texttt{run.yaml}.

\paragraph{(T4) Mirror/tropical pipeline [Integration].}
\emph{Input:} $X$, tropical flow $\Trop_\lambda$, realization $F_\lambda$, mirror functor $\Mirror$.\\
\emph{Assertions:} shortening factor $\kappa\le 1$ implies non\hyp increase of $\mathrm{PE}^{\le\tau}_i$ (after $\Ttau$); mirror transfer is non\hyp expansive after truncation; group proxies (if used) meet weak\hyp collapse thresholds (Ch.~8). Spectral eigenvalues ascending; norm declared.\\
\emph{Artifacts:} per\hyp $\lambda$ \texttt{bars/spec/ext/phi.json}; \texttt{run.yaml}.

\paragraph{(T5) Three\hyp layer compatibility [Integration].}
\emph{Input:} Gal$\to$Trans$\to$Funct data with comparison natural transformations (Ch.~9).\\
\emph{Assertions:} after $\Ctau$ and $\mathbf{P}_i$, commutativity holds up to isomorphism in $\Perskft$ per degree; indicators consistent; failures typed and logged.\\
\emph{Artifacts:} per\hyp layer \texttt{bars/spec/ext.json}; global \texttt{phi.json}; \texttt{run.yaml}.

\paragraph{(T6) PDE monitoring loop [Regression].}
\emph{Input:} index set $I$ (time/resolution/parameter), realization $\mathcal{P}$.\\
\emph{Assertions:} Ch.~11 protocol holds; stable regime flags match logs of $\mathrm{PE}^{\le\tau}$, spectral indicators \& aux\hyp bars, $\Ext^1$, $(\DiagZero / \DiagNonzero)$; reporting choice (per degree vs.\ aggregated) fixed and logged; spectra ascending; norm declared.\\
\emph{Artifacts:} \texttt{bars/spec/aux/ext/phi.json} over $I$; \texttt{run.yaml}.

\paragraph{(T7) Saturation gate verification [Integration].}
\emph{Input:} window $[0,\tau^\ast]$ with candidate saturation (Ch.~11).\\
\emph{Assertions:} verify saturation parameters $(\eta,\delta)$ and record the window-local
\textbf{decision rule} enabled by the Saturation Gate:
within $[0,\tau^\ast]$, if $E_1(W)=0$ \emph{or} $\Ext^1(\Rfun(\C_{\tau^\ast}F),k)=0$,
then the PH-check may be treated as \texttt{passed} on this window.
(No logical equivalence is asserted.)
\emph{Artifacts:} \texttt{bars.json}, \texttt{ext.json}; \texttt{run.yaml}.

\paragraph{(T8) $\varepsilon$\hyp clipping regression [Unit].}
\emph{Input:} paired runs with unclipped vs.\ $\varepsilon$\hyp clipped $\Ttau\mathbf{P}_i$ (Ch.~11).\\
\emph{Assertions:} energy stability bound; $(\DiagZero / \DiagNonzero)$ computed on unclipped data and identical across the pair; logs distinguish clipping.\\
\emph{Artifacts:} \texttt{bars.json} (both), \texttt{phi.json}; \texttt{run.yaml}.

\paragraph{(T9) MECE window coverage \& event accounting [Unit].}
\emph{Input:} MECE windowing $\{[u_k,u_{k+1})\}_k$ and global range $[u_0,U)$.\\
\emph{Assertions:} coverage equality; event counts add up (up to tolerance); uniform $\tau$ and bin policy unless justified and logged.\\
\emph{Artifacts:} \texttt{run.yaml} (coverage\_check), per\hyp window \texttt{bars.json}.

\paragraph{(T10) A/B commutativity test for reflectors [Unit/Integration].}
\emph{Input:} two persistence\hyp level reflectors $T_A,T_B$, tolerance $\eta\ge 0$.\\
\emph{Assertions:} $\Delta_{\mathrm{comm}}(M;A,B)=d_{\mathrm{int}}(T_AT_BM,\ T_BT_AM)$; pass if $\Delta_{\mathrm{comm}}\le\eta$, else fallback order is used and defect added to $\delta^{\alg}$ (Appendix~L).\\
\emph{Artifacts:} \texttt{run.yaml} (A/B policy), \texttt{bars.json} before/after.

\paragraph{(T11) Restart \& Summability [Integration/Regression].}
\emph{Input:} windows with thresholds $\tau_k$, budgets $\Sigma\delta_k(i)$, margins $\mathrm{gap}_{\tau_k}$.\\
\emph{Assertions:} Restart: $\mathrm{gap}_{\tau_{k+1}}\ge \kappa(\mathrm{gap}_{\tau_k}-\Sigma\delta_k(i))$ (record $\kappa$). Summability: $\sum_k \Sigma\delta_k(i)<\infty$. Certificates paste.\\
\emph{Artifacts:} \texttt{run.yaml} (restart/summability), gate logs, global certificate.

\paragraph{(T12) Trigger pack verification (domain\hyp restricted) [Integration].}
\emph{Input:} declared triggers (e.g.\ PDE, Ch.~10).\\
\emph{Assertions:} each trigger implies a B\hyp Gate$^{+}$ failure on the window; detection rate and false positives logged. Triggers are \textbf{[Spec]} and complement (not replace) B\hyp Gate$^{+}$.\\
\emph{Artifacts:} \texttt{run.yaml} (thresholds), \texttt{aux.json}, \texttt{phi.json}, gate verdicts.

\paragraph{(T13) $\delta$-ledger additivity \& pipeline budget [Integration].}
\emph{Input:} steps $U_m,\dots,U_1$ with per\hyp step collapses $C_{\tau_j}$ and bounds $\delta_j(i,\tau_j)$.\\
\emph{Assertions:} verify
\[
d_{\mathrm{int}}\!\Big(\cdots\Big)\ \preceq\ \bigotimes_{j=1}^m \delta_j(i,\tau_j),
\qquad\text{where }\otimes\text{ is the quantale operation (YAML: }\texttt{quantale.op}\texttt{).}
\]
and that post\hyp processing by $1$\hyp Lipschitz maps does not increase the bound (Appendix~L).\\
\emph{Artifacts:} \texttt{run.yaml} (\,$\delta$-ledger), \texttt{bars.json}, distance logs.

\paragraph{(T14) Overlap Gate gluing test [Integration].}
\emph{Input:} charts $X_1,X_2$ with windows $W_1,W_2$ and overlap $X_{12}$; fixed $\tau$; reflectors $T_A,T_B$.\\
\emph{Assertions:} after $\mathbf{P}_i$ and $\mathbf{T}_\tau$, verify post\hyp collapse local equivalence on $X_{12}$ within budget, Čech–$\Ext^1$ vanishing, A/B soft\hyp commuting or logged fallback; construct glued truncated object and confirm B\hyp Gate$^{+}$.\\
\emph{Artifacts:} \texttt{run.yaml} (overlap/A\!/\!B/\,$\delta$), per\hyp chart/overlap \texttt{bars/ext.json}; global verdict.

\paragraph{(T15) Length spectrum audit [Unit].}
\emph{Input:} $\Ttau\mathbf{P}_i(F)$ and $\Len(\Ttau\mathbf{P}_i(F);[0,\tau])$.\\
\emph{Assertions:} eigenvalue multiset equals clipped bar\hyp length multiset (up to permutation); $L^1$-mass equals $\mathrm{PE}^{\le\tau}_i(F)$. Hash or canonical ordering recorded.\\
\emph{Artifacts:} \texttt{bars.json}, \texttt{Lambda\_len.json} (optional list/hash), \texttt{run.yaml}.

\paragraph{(T16) \textbf{V\hyp shift (Quantale) test} [Unit].}
\emph{Input:} a fixed commutative quantale $V$ with operation $\oplus$, unit $e$, order $\preceq$; Lawvere $V$-distance and $V$-shift operators $S^{v}$ ($v\in V$); truncation $\Ttau$.\\
\emph{Assertions:} (i) $V$-Lipschitz: $d_V(\Ttau S^{v}M,\Ttau S^{v}N)\preceq d_V(\Ttau M,\Ttau N)$ for all $v$ (Ch.~2, Lemma~“$V$-shift’’); (ii) commutation: $\Ttau\circ S^{v}\cong S^{v}\circ \Ttau$ at the persistence layer; (iii) composition: $S^{v_2}\circ S^{v_1}\cong S^{v_1\oplus v_2}$ and quantitative defect (if any) is recorded in $\delta^{\alg}$.\\
\emph{Artifacts:} \texttt{run.yaml} (\texttt{quantale:\{name, op, unit, order, mode\}}), \texttt{bars.json} before/after $S^{v}$, distance logs.

\paragraph{(T17) \textbf{Definable coverage \& Čech finiteness} [Integration].}
\emph{Input:} a window cover by formulas $\{\varphi_\alpha(x)\}$ in a fixed o\hyp minimal or Denef–Pas structure; right\hyp open windows $W_\alpha=\{x:\varphi_\alpha(x)\}$.\\
\emph{Assertions:} (i) definability check passes for each $\varphi_\alpha$; (ii) finite event count on each $W_\alpha$ (piecewise constant Betti integrals; Appendix~H); (iii) finite Čech depth on the cover (Appendix~J); (iv) Overlap Gate checks pass with logged $\delta$-budgets; (v) optional confirmation of Ch.~3 local bridge when $E_1(W_\alpha)=0$.\\
\emph{Artifacts:} \texttt{run.yaml} (\texttt{definable:\{structure, window\_formulae\}}), per\hyp window \texttt{bars.json}, overlap logs.

\paragraph{(T18) \textbf{Iwasawa control $\Rightarrow$ Overlap Gate} [Integration].}
\emph{Input:} arithmetic tower (Ch.~7) with a Control theorem yielding finite kernel/cokernel on comparison maps; Overlap Gate configuration.\\
\emph{Assertions:} (i) finite kernel/cokernel are absorbed into $\delta^{\alg}$ as per policy; (ii) post\hyp collapse comparison maps satisfy $\phi_{i,\tau}$ isomorphism on windows where the control bound holds, hence $\DiagZero$ at fixed $\tau$; (iii) Overlap Gate passes with recorded $\delta$; (iv) explicit bounds \texttt{control\_finite\_bounds} are logged.\\
\emph{Artifacts:} \texttt{run.yaml} (\texttt{iwasawa:\{tower\_level, control\_finite\_bounds\}}), \texttt{phi.json}, gate logs.

% ---------------- Mandatory named tests (aliases and additions) ----------------
\subsection*{12.2a. Mandatory named tests (aliases/additions)}
The following \emph{named} tests are \textbf{mandatory}. Each comes with a canonical alias into (T1)–(T18) for reporting.

\paragraph{(T–ExtZero$\Rightarrow$PHZero) \quad [Integration].}
\emph{Scope:} Only on definable windows with $E_1(W)=0$ (Thm.~\ref{thm:E1-local}) \emph{or} under the Saturation Gate (Ch.~11, Decl.~\ref{gate:11-sat}).\\
\emph{Assertions:} \emph{Assertions:} On definable windows with $E_1(W)=0$, verify that both
$\PH_1(\Ctau F|_W)=0$ and $\Ext^1(\Rfun(\Ctau F|_W),k)=0$ hold (window-local, after collapse).
Outside this scope, the test is marked \texttt{inapplicable}.
 (window\hyp local). Outside this scope, the test is marked \texttt{inapplicable}.\\
\emph{Alias:} T7 (saturation) and T17 (definable). Artifacts as in T7/T17.

\paragraph{(T–Countable–Cover) \quad [Integration].}
\emph{Input:} a countable MECE cover $W=\bigsqcup_{n\ge 1}W_n$ with local finiteness on compact subintervals.\\
\emph{Assertions:} (i) coverage/equality checks of T9 extend to the countable case; (ii) Overlap Gate passes on each finite subcover; (iii) certificates paste by Restart/Summability (T11).\\
\emph{Alias:} T9$+$T11$+$T14. Artifacts as in those tests.

\paragraph{(T–Delta–Sum–Converges) \quad [Regression].}
\emph{Assertions:} $\sum_k \Sigma\delta_k(i)<\infty$ with logged tail bounds; global certificate exists.\\
\emph{Alias:} T11 (Summability). Artifacts: restart/summability block.

\paragraph{(T–Lipschitz–AfterCollapse) \quad [Unit].}
\emph{Assertions:} (i) $\Ttau$ is $1$–Lipschitz: $d_{\mathrm{int}}(\Ttau M,\Ttau N)\le d_{\mathrm{int}}(M,N)$; (ii) each declared update is $1$–Lipschitz \emph{after} $\Ttau$ within recorded $\varepsilon$; (iii) deletion–type steps achieve non\hyp increase of $\mathrm{PE}^{\le\tau}_i$.\\
\emph{Alias:} T1$+$T2. Artifacts: as in T1/T2.

\paragraph{(T–Exactness–Persistence) \quad [Unit].}
\emph{Input:} short exact sequences in $\Perskft$ (implementable range).\\
\emph{Assertions:} $\Ttau$ is exact on these sequences; induced maps on barcodes respect subquotients; equality verified after collapse.\\
\emph{Alias:} supports [Prop] “exactness of $\Ttau$’’ and audits via T10 (A/B) when multiple reflectors are present. Artifacts: before/after \texttt{bars.json}.

\paragraph{(T–Iwasawa–Alignment) \quad [Integration].}
\emph{Assertions:} control\hyp theorem comparators align with $\delta$–ledger: finite kernel/cokernel absorbed into $\delta^{\alg}$; Overlap Gate passes; $\DiagZero$ at fixed $\tau$.\\
\emph{Alias:} T18. Artifacts: as in T18.

\paragraph{(T–PFBC–AfterCollapse) \quad [Unit/Integration, optional].}
\emph{Assertions:} PF/BC steps computed objectwise in $t$ then compared only \emph{after} $\Ttau$; non\hyp expansiveness verified post\hyp truncation; any discretization/sampling residuals are charged to $\delta_{\disc}\oplus\delta_{\meas}$.\\
\emph{Alias:} extends T5/T13 with PF/BC flags. Artifacts: PF/BC comparator logs.

\paragraph{(T–$\Lambda_{\mathrm{len}}$) \quad [Unit, optional].}
\emph{Assertions:} identical to T15; provide canonical hash for eigenvalue multiset of $\Len(\Ttau\mathbf{P}_i(F);[0,\tau])$.\\
\emph{Alias:} T15. Artifacts: \texttt{Lambda\_len.json}.

% ---------------- Reproducibility & logs ----------------
\subsection*{12.3. Reproducibility and logs}
Every run ships with a manifest \texttt{run.yaml} declaring: sweep $\tau_{\min}\!:\Delta\tau\!:\tau_{\max}$; spectral policy $(\beta,M(\tau),t)$; discretization (grid/complex, steps); seeds; software versions; tower index set and cone extension (including the terminal symbol); pass\hyp criteria (per\hyp degree vs.\ aggregated); norm choice $\|\cdot\|_{\mathrm{op}}$ or $\|\cdot\|_{\mathrm{fro}}$; A/B tolerance $\eta$; Restart constants ($\kappa$) and Summability evidence; Overlap Gate status; file pointers to \texttt{bars/spec/aux/ext/phi.json} (optionally \texttt{.h5}). Persistence quantities are computed after $\Ttau$; equivalently on $\Ctau F$; and remain invariant under \fqi\ at the persistence layer.

\begin{declaration}[Schema extension and mandatory fields \textbf{[Spec]}]\label{dec:12-schema}
The following \texttt{run.yaml} fields are \emph{mandatory} for auditability (synchronized with Appendix~G):
\begin{itemize}
  \item \textbf{Quantale block} \texttt{quantale:\{name, op, unit, order, mode\}} (e.g.\ $[0,\infty]_{+}$, $(\max,+)$, probabilistic/product modes).
  \item \textbf{Layered $\delta$} \texttt{layered\_delta:\{deltaGal, deltaTr, deltaFun\}}.
  \item \textbf{Definable windows} \texttt{definable:\{structure, window\_formulae\}} (structure $\in\{\texttt{R\_an,exp},\texttt{Denef\!-\!Pas}\}$).
  \item \textbf{Iwasawa} \texttt{iwasawa:\{tower\_level, control\_finite\_bounds\}}.
  \item \textbf{AWFS/2\hyp cell} \texttt{awfs:\{enabled: bool, two\_cell\_bounds: value\}}.
  \item \textbf{Overlap Gate} \texttt{overlap\_checks:\{local\_equiv, cech\_ext1\_ok, stability\_band\_ok\}}.
  \item \textbf{Length spectrum} \texttt{Lambda\_len} per degree on $[0,\tau]$ (list or hash) for T15.
  \item \textbf{Spectral policy} \texttt{spectral\_policy:\{order: "ascending", norm: "op"|"fro"\}}, and \texttt{spectral\_bounds:\{lambda\_min, lambda\_max, lip\_tol?\}}.
  \item \textbf{Persistence} explicit $(\mu,\nu)$ totals and tail\hyp isomorphism flag \texttt{phi\_iso\_tail}.
  \item \textbf{Budget} \texttt{sum\_delta}, \texttt{safety\_margin}; for Restart, per\hyp window \texttt{gap\_tau}.
  \item \textbf{A/B} \texttt{ab\_test:\{eta, policy, fallback\}}.
\end{itemize}
\end{declaration}

\begin{remark}[Audit checklist]\label{rk:12-audit}
(i) Constructibility verified;\;
(ii) Coefficient field fixed (Novikov allowed at \textbf{[Spec]});\;
(iii) Deletion– vs.\ inclusion–type correctly labeled;\;
(iv) Uniform interleaving shifts $\varepsilon_n$ bounded;\;
(v) Same window for $\mathrm{PE}$, spectral, aux\hyp bars, $\Ext^1$, and $(\mu,\nu)$ after $\Ttau$;\;
(vi) LC order: $\Ctau$ then $\Rfun$ (one\hyp way bridge only);\;
(vii) PF/BC prechecks for derived transfers (App.~N); non\hyp expansiveness only \emph{after} truncation;\;
(viii) Spectra ascending; norm declared; terminal symbol consistent;\;
(ix) MECE coverage and event accounting satisfied (finite or countable);\;
(x) A/B commutativity configured (T10) and logged;\;
(xi) Restart/Summability evidenced (T11/T–Delta–Sum–Converges);\;
(xii) Overlap Gate fields complete (T14);\;
(xiii) Length spectrum audit recorded (T15/T–$\Lambda_{\mathrm{len}}$);\;
(xiv) Quantale/definable/Iwasawa/AWFS blocks complete (T16–T18);\;
(xv) T–PFBC–AfterCollapse flags present when PF/BC is used.
\end{remark}

\noindent\emph{Manifest template (YAML).}
\small
\begin{verbatim}
coeff_field: "k"                  # or "Novikov(q)" [Spec-level]
tau_window: [0.05, 1.0]           # start, end
tau_step: 0.05
quantale:
  name: "[0,inf]_plus"
  op: "+"
  unit: 0.0
  order: "<="
  mode: "standard"                # or "probabilistic", "product"
layered_delta: {deltaGal: 0.002, deltaTr: 0.003, deltaFun: 0.002}
definable:
  structure: "R_an,exp"           # or "Denef-Pas"
  window_formulae:
    - "u <= t < u'"
    - "t in union_{j=1..m} (a_j, b_j]"
iwasawa:
  tower_level: 5
  control_finite_bounds: {kernel_le: 2, cokernel_le: 3}
awfs:
  enabled: true
  two_cell_bounds: 0.01
spectral:
  tail_beta: 2
  tail_cutoff_M_of_tau: "floor(0.5 * tau^1.5)"
  heat_t: [0.5*tau^-2, 1.0*tau^-2]
  aux_bins: {a: 0.0, beta: 0.02, bins: 96, boundary: "right-open"}
spectral_policy:
  order: "ascending"
  norm: "op"
spectral_bounds:
  lambda_min: 1.0e-6
  lambda_max: 10.0
  lip_tol: 0.02
tower:
  eps_interleave_max: 0.02
  terminal_symbol: "infty"        # or "lambda_star"
  cone_extension: true
ab_test:
  eta: 0.01
  policy: "soft-commuting"        # or "fallback:A_then_B"
restart_summability:
  kappa_min: 0.8
  sum_delta_bound: 0.05
windows:
  domain: [[0,1), [1,2), [2,3)]   # finite or countable MECE cover
  collapse_tau: 0.08
coverage_check:
  length_sum: 3.0
  length_target: 3.0
  events_sum_equals_global: true
overlap_checks:
  local_equiv: true
  cech_ext1_ok: true
  stability_band_ok: true
pfbc:
  policy: "after_collapse"        # T–PFBC–AfterCollapse
  residual_ledger: ["disc","meas"]
persistence:
  PH1_zero: true
  Ext1_zero: true
  mu: 0
  nu: 0
  phi_iso_tail: true
Lambda_len:
  degree: 1
  tau: 0.08
  audit: "hash:2f4c...d1"
record:
  bars: true
  PE: {report: "per-degree", clipping: "epsilon=0.02"}
  aux: {lifetime_min_frames: 3}
  heat_trace: {ordering: "ascending", norm: "op"}
  ext1: true
  mu_nu: true
budget:
  sum_delta: 0.011
  safety_margin: 0.025
  gap_tau: 0.03
gate:
  accept: true
notes: "deletion-type only for monotonicity; LC with Rfun after truncation; PF/BC verified."
\end{verbatim}
\normalsize

\subsection*{12.4. Open problems (selected)}
\begin{remark}[Open problems]\label{rk:12-open}
\hfill
\begin{enumerate}
  \item \textbf{Quantitative bridge.} Domain\hyp wise sufficient conditions implying $\Ext^1=0$ from decay of $\|\mathbf{CE}^{\le \tau}\|_1$ and $\mathrm{ST}_\beta^{\ge M(\tau)}$ (window\hyp local).
  \item \textbf{Colimit criteria.} Sharp hypotheses guaranteeing $\DiagZero$ beyond objectwise degreewise colimits.
  \item \textbf{Failure lattice.} Finer invariants separating pure/mixed failures and anticipating Type~IV at nearby scales.
  \item \textbf{Spectral–persistence calibration.} Robust bounds between collapse energy and spectral tails under noise and discretization.
  \item \textbf{Weak group collapse.} Persistence\hyp level proxies vs.\ algebraic invariants without leaving the implementable range.
  \item \textbf{Arithmetic towers.} Templates linking collapse diagnostics to Selmer/class growth while keeping $\muc\neq \mu_{\mathrm{Iwasawa}}$.
  \item \textbf{Langlands layers.} Minimal comparison data for truncated commutativity across Gal$\to$Trans$\to$Funct.
  \item \textbf{PDE program.} Conditions under which persistence\hyp guided regularization predicts classical regimes programmatically.
  \item \textbf{Universality of $\mathbf{T}_\tau$.} Characterization of $\Ttau$ as Serre localization in the implementable range.
\end{enumerate}
\end{remark}

\subsection*{12.5. Final guard\hyp rails}
\begin{remark}[Scope and non\hyp claims]\label{rk:12-guards}
All specifications are confined to the persistence/spectral/categorical layers in the implementable range and are verifiable by the test suite above. No number\hyp theoretic identity, analytic regularity theorem, or group trivialization is asserted. In particular, no claim of $\mathrm{PH}_1\Leftrightarrow \Ext^1$ is made; only the one\hyp way implication under (B1)–(B3) is used. The obstruction $\muc$ is a collapse diagnostic and differs from the classical Iwasawa $\mu$.
\end{remark}

\subsection*{12.6. Effect and auditability}
\begin{remark}[Effect of the extensions]
Relative to v16.0, the harness is strengthened by: (i) Overlap Gate gluing (T14) with post\hyp collapse local equivalence and A/B soft\hyp commuting; (ii) A/B tests with manifest\hyp level tolerance (T10); (iii) Restart/Summability quantification (T11); (iv) Saturation Gate anchoring (T7); \emph{and additionally} (v) $V$\hyp shift verification for quantale\hyp enriched runs (T16); (vi) definable coverage \& Čech finiteness checks (T17); (vii) Iwasawa control integration with $\delta^{\alg}$ absorption (T18); (viii) \textbf{mandatory} named tests T–ExtZero$\Rightarrow$PHZero, T–Countable–Cover, T–Delta–Sum–Converges, T–Lipschitz–AfterCollapse, T–Exactness–Persistence, T–Iwasawa–Alignment; (ix) optional T–PFBC–AfterCollapse and T–$\Lambda_{\mathrm{len}}$. The schema (Decl.~\ref{dec:12-schema}) mandates quantale/definable/Iwasawa/AWFS/PFBC blocks, improving third\hyp party auditability.
\end{remark}

\subsection*{12.7. Conclusion}
This chapter consolidates a complete, testable interface: a precise badge policy, a uniform notation layer, and a formal test suite spanning stability, monotone updates, filtered\hyp colimits, mirror/tropical flows, Langlands triples, PDE pipelines, Overlap Gate gluing, quantale\hyp enriched shifts, definable coverage, and Iwasawa control. All persistence\hyp layer quantities are computed after $\Ttau$; spectral indicators are normalized (eigenvalues ascending; norm declared); categorical checks are performed only in the one\hyp way direction. Reproducibility is enforced by a single manifest with mandatory fields. The \emph{implementable range} is thus executable and auditable, with conservative guard\hyp rails and clear open directions.

\subsection*{12.8. Completion note}
\begin{remark}[No further supplementation required]
This chapter fully integrates: (i) MECE (finite/countable) window tests and event accounting; (ii) A/B commutativity with tolerance and fallback; (iii) Restart/Summability verification; (iv) Trigger pack validation; (v) $\delta$-ledger additivity; (vi) Overlap Gate gluing (T14); (vii) Length spectrum audit (T15/T–$\Lambda_{\mathrm{len}}$); (viii) Quantale $V$-shift tests (T16); (ix) definable coverage/Čech finiteness (T17); (x) Iwasawa control $\Rightarrow$ Overlap Gate (T18/T–Iwasawa–Alignment); (xi) PF/BC enforcement after collapse (T–PFBC–AfterCollapse); (xii) a manifest schema synchronized with Appendix~G. All items are consistent with the v16.0 guard\hyp rails and cross\hyp reference the proven core; no additional supplementation is needed for operational use as a formal test suite.
\end{remark}

% ---------------- Machine-readable badge & test index ----------------
\subsection*{12.9. Machine-readable badge \& test index (for automated extraction)}
\noindent\textit{This block is purely auxiliary for reproducibility tools and can be ignored in print.}
\begin{verbatim}
badge_index:
  proof_labels: ["prop:stability","lem:Vshift","prop:operational-coreflection","J:prop:diagzero"]
  theorem_labels: ["thm:PH1-to-Ext1","thm:E1-local"]
  declaration_chapters: [2,6,7,8,9,10,11,12]
mandatory_tests:
  - name: "T-ExtZero->PHZero"        ; alias: ["T7","T17"]  ; scope: "definable_or_saturation"
  - name: "T-Countable-Cover"        ; alias: ["T9","T11","T14"]
  - name: "T-Delta-Sum-Converges"    ; alias: ["T11"]
  - name: "T-Lipschitz-AfterCollapse"; alias: ["T1","T2"]
  - name: "T-Exactness-Persistence"  ; alias: []
  - name: "T-Iwasawa-Alignment"      ; alias: ["T18"]
optional_tests:
  - name: "T-PFBC-AfterCollapse"     ; alias: ["T5","T13"]
  - name: "T-Lambda_len"             ; alias: ["T15"]
pfbc_policy: "after_collapse"
spectral_policy: {order: "ascending", norm: "op"}
\end{verbatim}



% =========================================================================
% Chapter 13 : High-Dimensional Projection Search and the Map of Validity
% =========================================================================

\section*{Chapter 13: The Map of Validity and Defect Potential}
\label{sec:ch13}

\subsection*{13.0. Overview and Motivation}
Part~I established the \emph{Unified Collapse Contract (UCC)} as a rigorous
auditor: given an input \(F\) and collapse threshold \(\tau\), the system certifies
validity through the inequality
\[
\|\Sigma\delta(x)\|_V \ <\ \mathrm{Gap}_\tau,
\]
This certificate is sufficient for verification but inadequate for
\emph{exploration}.  
To study global mathematical families—such as flows of PDEs, arithmetic
families of elliptic curves, or geometric variations—we require
a framework that provides directional information:
a means to navigate the parameter space \(\mathcal{M}\).

This chapter introduces such a mechanism.
We reinterpret the \(\delta\)-ledger as a \textbf{scalar potential}
\(\Phi : \mathcal{M} \to \mathbb{R}_{\ge 0} \cup \{\infty\}\),
providing a navigable landscape over which AI agents—defined
in Chapter~14—may perform gradient descent, local search,
or dimensional lifting.  
Paired with a definable partition of \(\mathcal{M}\) into \emph{Terrain Cells},
the framework evolves from a passive auditor into an active navigation system.

% -------------------------------------------------------------------------
\subsection*{13.1. From Collapse Diagnosis to Navigation}
Let \(\mathcal{M}\) denote the moduli space of admissible inputs.
We assume throughout that \(\mathcal{M}\) admits a stratification by definable
sets (Appendix~Q).

\begin{definition}[Navigation Mode]
In \emph{Navigation Mode}, the AK pipeline does not reject an input
\(x \in \mathcal{M}\) when
\(\Sigma\delta(x) \ge \mathrm{gap}_\tau\).
Instead, it returns both the \textbf{magnitude} of the defect and,
when available, an approximate gradient direction
indicating how to reduce the defect within \(\mathcal{M}\).
\end{definition}

\begin{specification}[Gradient Oracle (canon) \textbf{[Spec]}]\label{spec:13-grad-oracle}
In \emph{Navigation Mode}, any reference to a “gradient”
\(\nabla\Phi\) means a \emph{logged, reproducible gradient estimate}
\(\widehat{\nabla}\Phi\) produced by the following oracle \(\mathsf{GradEst}\).

\paragraph{Oracle interface.}
Given a state \(x\) in the navigation/search space, a scalar potential
\(\Phi(x)\) (Def.~\ref{def:13-defect-potential}), and a policy block
\texttt{grad\_policy} recorded in \texttt{run.yaml} (Appendix~G),
the oracle returns:
\[
\mathsf{GradEst}(x;\texttt{grad\_policy}) \;\leadsto\;
(\widehat{g},\;\widehat{\sigma}^2,\;\texttt{diag}),
\]
where \(\widehat{g}\) is the gradient estimate, \(\widehat{\sigma}^2\) is an
estimated variance (or an upper bound), and \texttt{diag} is a structured
diagnostic record (below).

\paragraph{Allowed methods (enumerated).}
\texttt{grad\_policy.method} is one of:
\[
\texttt{finite\_difference}\;|\;\texttt{SPSA}\;|\;\texttt{surrogate}.
\]
For \texttt{finite\_difference}, \texttt{grad\_policy.stencil} fixes the stencil
(one-/two-sided, coordinate subset, step size \(\varepsilon\)).
For \texttt{SPSA}, \texttt{grad\_policy.seed} and perturbation distribution are fixed.
For \texttt{surrogate}, the surrogate family and fitting seed are fixed.

\paragraph{Norms and step semantics.}
All step decisions use the norm recorded in \texttt{grad\_policy.norm}
and must be consistent with the spectral/persistence norms declared in
\texttt{spectral\_policy} / \texttt{pipeline.metric} (Appendix~G).
Any smoothing/regularization used by \(\mathsf{GradEst}\) must be declared in
\texttt{grad\_policy.regularization}.

\paragraph{Audit and \(\delta\)-ledger charging.}
Gradient estimation introduces an approximation component that must be
charged into \texttt{operations[*].delta.sources} and aggregated into
\texttt{budget.sum\_delta} via the manifest quantale (Ch.~12, Dec.~\ref{dec:12-schema}).
At minimum, the diagnostic record \texttt{diag} contains:
\begin{itemize}[leftmargin=1.25em]
\item \texttt{method}, \texttt{stencil} (if applicable), \texttt{seed},
\item \texttt{eval\_count}, \texttt{eps} (if applicable),
\item \texttt{variance} (or certified bound), and
\item \texttt{delta\_charge} (the amount charged for this oracle call).
\end{itemize}
The corresponding per-step action log entry is mandatory (Appendix~U).

\paragraph{Core/Search separation guard.}
No \(\nabla\Phi\) claim is admissible unless it is backed by a
\(\mathsf{GradEst}\) record in \texttt{run.yaml} and Appendix~U for the same \texttt{run\_id}.
\end{specification}


\noindent\textbf{[Spec] Gradient output.}
When available, the pipeline returns an approximate descent direction for the
Defect Potential \(\Phi_\tau\) (see Remark~\ref{rk:13-grad-policy}).

Thus the goal is no longer merely to certify a point but to
identify the region:
\[
Z_{\mathrm{Valid}} \ := \ \Phi^{-1}([0,\, \mathrm{gap}_\tau]).
\]

% -------------------------------------------------------------------------
\subsection*{13.2. The Defect Potential \texorpdfstring{\(\Phi(x)\)}{Phi(x)}}
The \(\delta\)-ledger encodes algebraic, numerical, and functorial deviations
from ideal collapse. To combine these into a single invariant, we apply
a monotone scalarization.

\begin{definition}[Scalarization]
Let \(V\) be the Quantale of δ-budgets.
A map
\(\|\cdot\|_V : V \to \mathbb{R}_{\ge 0} \cup \{\infty\}\)
is a \emph{scalarization} if:
\begin{enumerate}
    \item \(\|0_V\|_V = 0\);
    \item \(\delta_1 \preceq \delta_2 \Rightarrow \|\delta_1\|_V \le \|\delta_2\|_V\);
    \item \(\|\delta_{\mathrm{alg}} \oplus \delta_{\mathrm{disc}}\|_V
            \ge \|\delta_{\mathrm{alg}}\|_V\)
          (algebraic defects contribute persistently).
\end{enumerate}
Typical examples include \(\ell^1\) or \(\ell^\infty\) norms when
\(V=[0,\infty]^k\).
\end{definition}

\begin{definition}[Defect Potential \(\Phi_\tau\)]\label{def:13-defect-potential}
Let \(x\in \mathcal{M}\), and let \(\Sigma\delta(x)\) denote the δ-budget
\emph{after} applying collapse \(T_\tau\).
Let \((\mu(x),\nu(x))\) be the tower obstruction indices (Chapter~4).
The \emph{Defect Potential} is:
\[
\Phi_\tau(x)
\ :=\
\big\|\Sigma\delta(x)\big\|_V
\;+\;
\lambda_{\mathrm{sing}}\, \mathcal{I}_{\mathrm{IV}}(x),
\]
where \(\lambda_{\mathrm{sing}}\gg 1\) and
\[
\mathcal{I}_{\mathrm{IV}}(x)
=
\begin{cases}
1 & \text{if } (\mu(x),\nu(x))\neq(0,0),\\
0 & \text{otherwise}.
\end{cases}
\]
\end{definition}

\begin{remark}[Gradient estimation and logging \textbf{[Spec]}]\label{rk:13-grad-policy}
On each Terrain Cell, an approximate descent direction is computed for \(\Phi_\tau\)
using a declared policy \texttt{grad\_policy}:
(i) finite differences, (ii) random perturbations, or (iii) a surrogate model.
The Lipschitz metric on \(\mathcal{M}\) is evaluated in the declared norm
\(\|\cdot\|_{\mathcal{M}}\) (e.g.\ \(\ell^2\) or \(\ell^\infty\)) and logged.
Any estimation/sampling residual is charged to
\(\delta_{\mathrm{meas}}\oplus\delta_{\mathrm{disc}}\) and recorded in the \(\delta\)-ledger.
\end{remark}

\begin{remark}[Geometric Stratification and Lifting Trigger]
The potential \(\Phi_\tau\) stratifies the parameter space
\(\mathcal{M}\) into three operational regimes, each prescribing a distinct
AI-agent behavior:

\begin{itemize}
    \item \textbf{Plain of Truth}
    (\(\Phi(x)<\mathrm{gap}_\tau\)).  
    Collapse is certified by UCC.
    The agent \emph{records the region as valid} and explores boundaries.

    \item \textbf{Ridge of Noise}
    (\(\mathrm{gap}_\tau \le \Phi(x) < \lambda_{\mathrm{sing}}\)).  
    Only Types~I--III defects occur.
    The agent performs \textbf{local gradient descent} (or random-walk refinement)
    within the Terrain Cell to search for a descending path.

    \item \textbf{Peak of Singularity}
    (\(\Phi(x)\ge \lambda_{\mathrm{sing}}\)).  
    Indicates essential Type~IV obstruction.
    The agent must trigger a \textbf{Dimensional Lifting} request (Chapter~14),
    adding auxiliary axes to escape the singular fiber.
\end{itemize}

This stratification gives \(\Phi\) an operational semantics and links it directly
to the autonomous behaviors of Chapter~14.
\end{remark}

% -------------------------------------------------------------------------
\subsection*{13.3. Terrain Cells and Definable Geometry}
Optimization on \(\mathcal{M}\) requires discretization compatible
with definability and uniformity.

\begin{definition}[Terrain Cell]
A \emph{Terrain Cell} \(W_\alpha\subset\mathcal{M}\) is a definable,
bounded subset satisfying:
\begin{enumerate}
    \item \textbf{Uniform Constructibility:}
    The persistence diagrams \(\mathbf{P}_i(F_x)\) admit uniform
    constructible description for all \(x\in W_\alpha\).

    \item \textbf{Lipschitz Budget:}
    The assignment \(x\mapsto \Sigma\delta(x)\) is Lipschitz on \(W_\alpha\).

    \item \textbf{MECE Partition:}
    The family \(\{W_\alpha\}\) is mutually exclusive and collectively exhaustive.
\end{enumerate}
\end{definition}

\begin{declaration}[Local convexity proxy \textbf{[Spec]}]
Away from Type~IV regions, \(\Phi_\tau\) is treated as approximately convex
on sufficiently small Terrain Cells for the purpose of local search.
\end{declaration}


% -------------------------------------------------------------------------
\subsection*{13.4. Structural Regularity (programmatic)}
\begin{declaration}[AK Global Validity via bounded potential \textbf{[Spec]}]\label{dec:ak-reg}
Let \(\mathcal{M}\) be a path-connected definable component.
Assume (1)–(3). Then, for every \(x\in\mathcal{M}\),
the UCC checks pass in the implementable range after truncation:
B-Gate\(^+\) holds at the declared \(\tau\), and the local certificates glue
(across the declared cover) to a global post-collapse certificate in
\(\Pers_k^{\mathrm{cons}}\) (up to the declared \(\delta\)-ledger budgets).
\end{declaration}

\begin{conjecture}[Interpretation under domain hypotheses]
Under the AK--NS hypothesis (Ch.~10) (resp.\ arithmetic realization hypotheses in Ch.~7),
a globally bounded defect potential suggests absence of singular behavior
(resp.\ alignment of ranks) in the monitored regime.
\end{conjecture}

\begin{remark}[\textnormal{\textbf{[Spec]}} Rationale (replacing the proof environment)]\label{rem:ak-reg-rationale}
This paragraph records the intended implication chain of the programmatic declaration
\ref{dec:ak-reg} under the stated hypotheses.
Bounded potential is used as a proxy for uniform feasibility of B-Gate\(^+\) (with declared margins),
the Type~IV veto removes invisible obstructions,
and definable coverage together with Restart/Summability is the gluing mechanism for local certificates.
This is not a completed mathematical proof; it is an auditable design rationale for HDPS navigation.
\end{remark}

% -------------------------------------------------------------------------
\subsection*{13.5. Summary}
This chapter transforms AK-HDPST from a passive diagnostic mechanism
into a geometric navigation framework.
The Defect Potential \(\Phi\) supplies a scalar field governing movement,
and Terrain Cells provide local uniformity for optimization.
Chapter~14 introduces the autonomous agents—Hunter, Mapper, Lifter—
that operate over this landscape.



% =========================================================================
% Chapter 14 : AI Agents — Hunter, Mapper, Lifter
% =========================================================================
% AK-HDPST v17.0 — Fully Revised & Corrected Version (ChatGPT 5.1)
% =========================================================================

\section*{Chapter 14: AI Agents — Hunter, Mapper, and Lifter}
\label{sec:ch14}

\subsection*{14.0. Overview and Agent Taxonomy}
The scalar field \(\Phi\) (Chapter~13) transforms collapse diagnostics into
a navigable landscape on the parameter space \(\mathcal{M}\).
To explore this landscape efficiently and safely, we introduce three
autonomous but contract-bounded agents:

\begin{itemize}
    \item \textbf{Hunter} — performs local optimization of \(\Phi\) within
          Terrain Cells, discovering valid regions and local minima.
    \item \textbf{Mapper} — assembles validated Terrain Cells into a coherent
          global structure using Overlap Gates.
    \item \textbf{Lifter} — resolves essential singularities
          (Type~IV points) through controlled dimensional extension.
\end{itemize}

All agents operate under the \emph{Unified Collapse Contract (UCC)}:
they may propose actions, but acceptance and certification is performed
only by the AK Core.

% -------------------------------------------------------------------------
\subsection*{14.1. The Hunter: Regime-Aware Search Strategy}

Navigation Mode (Definition~13.1) assigns each point \(x\in\mathcal{M}\)
to one of the three regimes determined by \(\Phi(x)\):
\[
\text{Plain} \quad (\Phi < \mathrm{gap}_\tau), \qquad
\text{Ridge} \quad (\mathrm{gap}_\tau \le \Phi < \lambda_{\mathrm{sing}}), \qquad
\text{Peak} \quad (\Phi \ge \lambda_{\mathrm{sing}}).
\]

The Hunter realizes these semantics operationally.

% --- insert right after the regime split in 14.1 ---
\begin{remark}[Gradient oracle and logging \textbf{[Spec]}]\label{rk:14-grad-oracle}
In this chapter, “\(\nabla\Phi(x)\)” denotes an \emph{estimated descent direction}
\(g(x)\) for the scalar potential \(\Phi_\tau\) (Ch.~13), computed under a declared
\texttt{grad\_policy} (finite differences / random perturbations / surrogate).
The metric on \(\mathcal{M}\) is evaluated in the declared norm
\(\|\cdot\|_{\mathcal{M}}\).
All estimation and sampling residuals are charged to
\(\delta_{\mathrm{meas}}\oplus\delta_{\mathrm{disc}}\) and recorded in the \(\delta\)-ledger.
(See Remark~\ref{rk:13-grad-policy} and the manifest schema in Ch.~12.)
\end{remark}


\begin{definition}[Hunter State]
A Hunter maintains a state
\[
S_k = (x_k,\; W_k,\; \Phi_\tau(x_k),\; g_k,\; \mathsf{meta}_k),
\]
where \(g_k\) is an estimated descent direction produced by the gradient oracle
(Remark~\ref{rk:14-grad-oracle}), and \(\mathsf{meta}_k\) logs
\texttt{grad\_policy}, \(\|\cdot\|_{\mathcal{M}}\), step size, and seeds.
\end{definition}


\begin{specification}[Hunter Protocol]
At iteration \(k\), the Hunter acts according to the regime at \(x_k\):

\begin{enumerate}

    \item \textbf{Plain Regime (\(\Phi(x_k)\le \mathrm{gap}_\tau\)): Verification.}  
    Trigger the full B-Gate\(^+\) check.  
    If verified, mark \(W_k\) as \texttt{valid} and initiate exploration of
    its boundary \(\partial W_k\).

   \item \textbf{Ridge Regime (\(\mathrm{gap}_\tau < \Phi(x_k) < \lambda_{\mathrm{sing}}\)): Descent.}  
Compute \(g_k=\mathsf{GradEst}(\Phi_\tau,x_k;\texttt{grad\_policy})\) and propose
\[
x_{k+1}^{\mathrm{prop}} := x_k - \alpha_k\, g_k,
\qquad
x_{k+1}:=\Pi_{W_k}(x_{k+1}^{\mathrm{prop}}),
\]
where \(\Pi_{W_k}\) is the declared projection/clipping policy to keep the iterate inside
the current Terrain Cell.
Accept the step only if \(\Phi_\tau(x_{k+1})\le \Phi_\tau(x_k)-\rho\) (margin \(\rho\ge 0\) declared and logged);
otherwise shrink \(\alpha_k\) and retry up to a declared limit.
If \(\|g_k\|_{\mathcal{M}}\) falls below a declared threshold, apply a controlled random perturbation
(seed logged). All estimation/projection residuals are charged to
\(\delta_{\mathrm{meas}}\oplus\delta_{\mathrm{disc}}\).


    \item \textbf{Peak Regime (\(\Phi(x_k)\ge \lambda_{\mathrm{sing}}\)): Escalation.}  
    Terminate local search and invoke the Lifter to escape essential obstructions.

\end{enumerate}

Every action is recorded in the \emph{Hunter Action Log} (Appendix~U)
to guarantee reproducibility.
\end{specification}

% -------------------------------------------------------------------------
\subsection*{14.2. The Mapper: Global Assembly of Certificates}

The Mapper ensures that local certificates produced by Hunters
combine into a globally coherent proof artifact.

\begin{definition}[Coverage Graph]
Let \(\mathcal{G}=(\mathcal{V},\mathcal{E})\) be a graph whose vertices
are Terrain Cells marked \texttt{valid}.
An edge \((\alpha,\beta)\in\mathcal{E}\) exists when the Overlap Gate
(Chapter~5) passes on the intersection \(W_\alpha\cap W_\beta\).
\end{definition}

\begin{specification}[Mapper Protocol]
The Mapper performs the following loop:

\begin{enumerate}
    \item \textbf{Ingest Validated Cells.}
          Receive validated \(W_\alpha\) from Hunters.

    \item \textbf{Execute Overlap Gates.}
          Verify consistency on each intersection \(W_\alpha\cap W_\beta\).

    \item \textbf{Update Coverage Graph.}
          Add edges or merge components accordingly.

    \item \textbf{Check Global Coverage.}
          In bounded domains: check whether a connected component
          covers the target domain.  
          In unbounded domains: verify asymptotic stability of certificates.
\end{enumerate}

When a connected component covers the domain of interest,
the Mapper issues a \textbf{Global Certificate}.
\end{specification}

% -------------------------------------------------------------------------
\subsection*{14.3. The Lifter: Controlled Dimensional Extension}

The Lifter responds exclusively to essential singularities
(Type~IV points: \(\mu,\nu\neq 0\)).
Unlike the Hunter, its role is not optimization but \emph{geometric escape}.

\begin{definition}[Dimensional Lifting]
If \(\mathcal{M}_n\) is the base parameter space, a \emph{lifting}
is an embedding
\[
\iota: \mathcal{M}_n \hookrightarrow \mathcal{M}_{n+k}
\]
adding \(k\) auxiliary coordinates (e.g.\ smoothing width, spectral softness,
auxiliary weights, or arithmetic depth).
The object \(F\) is pulled back to a lifted object \(\tilde{F}\).
\end{definition}

\begin{specification}[Lifter Protocol]
Given a singular point \(x_{\mathrm{sing}}\):

\begin{enumerate}
    \item \textbf{Select Axis Type.}
          Choose an auxiliary axis appropriate to the obstruction
          (e.g.\ spectral, geometric, arithmetic).

    \item \textbf{Construct Lifted Neighborhood.}
          Build a definable neighborhood \(\tilde{W}\subset\mathcal{M}_{n+1}\)
          around \((x_{\mathrm{sing}},0)\).

    \item \textbf{Spawn New Hunter.}
          Launch a new Hunter on \(\tilde{W}\) to seek descent directions
          in the enlarged space.
\end{enumerate}

All lifting steps are subject to the Lifting Penalty (Section~14.4).
\end{specification}

% -------------------------------------------------------------------------
\subsection*{14.4. Safety: The Lifting Penalty \texorpdfstring{\(\delta^{\mathrm{lift}}\)}{delta-lift}}

To prevent trivial “solve-by-infinite-lifting”, dimensional lifting incurs
a cost charged against the global δ-budget.

\begin{definition}[Lifting Penalty]
For a lifting depth \(k\), the penalty is a monotone function
\[
\delta^{\mathrm{lift}}(k)
\quad\in V,
\]
commonly exponential, e.g.
\[
\delta^{\mathrm{lift}}(k) = C_{\mathrm{lift}}\, 2^k.
\]
\end{definition}

\begin{definition}[Augmented Gap Constraint]
Any lifted search path must satisfy the UCC constraint in the declared comparison mode:
either in \(V\)-order (via an embedding of \(\mathrm{gap}_\tau\) into \(V\)), or after scalarization:
\[
\big\|\Sigma\delta(x)\oplus\delta^{\mathrm{lift}}(k(x))\big\|_{V} \ <\ \mathrm{gap}_\tau.
\]
The chosen mode is declared in \texttt{run.yaml}.
\end{definition}

\begin{remark}[Operational Meaning]
The Lifter is not a universal escape mechanism.  
It is a \emph{high-cost extension operator} whose use must be justified
by essential obstructions.  
In practice, only a small number of lifts are admissible under the
UCC budget.
\end{remark}

\noindent\textbf{[Spec] Hunter Action Log.}
Each iteration logs \((x_k, W_k, \Phi_\tau(x_k))\), \texttt{grad\_policy}, \(\|\cdot\|_{\mathcal{M}}\),
\(g_k\), \(\alpha_k\), projection events, accept/reject, seeds, and the charged
\(\delta_{\mathrm{meas}}\oplus\delta_{\mathrm{disc}}\).

% -------------------------------------------------------------------------
\subsection*{14.5. Summary}
Hunter, Mapper, and Lifter jointly implement the active exploration
engine of AK-HDPST.
Hunter minimizes the Defect Potential \(\Phi\),
Mapper assembles local certificates into a global proof,
and Lifter resolves essential singularities via controlled dimensional extension.
Together they provide a sound, reproducible, and bounded
search mechanism for the validity landscape described in Chapter~13.



% =========================================================================
% Chapter 15 : Collapse-Based Optimization Protocols
% =========================================================================
% AK-HDPST v17.0 — Revised Version (ChatGPT 5.1 + Gemini)
% =========================================================================

\section*{Chapter 15: Collapse-Based Optimization Protocols}
\label{sec:ch15}

\subsection*{15.0. Overview and Motivation}
Chapter~14 defined the agents (Hunter, Mapper, Lifter) as the actors
of the navigation system.
This chapter specifies the \emph{optimization scripts} they follow.

Unlike classical optimization where the cost function is smooth and
explicit, the Defect Potential \(\Phi_\tau(x)\) is derived from
persistence diagnostics and quantale-valued error ledgers.
It can be non-convex, piecewise-defined, and expensive to evaluate.
We therefore require robust protocols combining:
\begin{itemize}
    \item \textbf{Local Descent:} Exploit approximate gradients where
          \(\Phi\) is locally regular (Ridge regime).
    \item \textbf{Restart Logic:} Use the Restart Lemma (Appendix~J)
          to escape shallow local minima caused by window scale.
    \item \textbf{Lifting Heuristics:} Recognize when failure is due
          to essential obstructions rather than numerical noise.
\end{itemize}

\noindent\textbf{[Spec] Logging.}
CBGD stencils/samples, norm \(\|\cdot\|_{\mathcal{M}}\), \texttt{grad\_policy}, accept/reject,
restart events, and the charged \(\delta_{\mathrm{meas}}\oplus\delta_{\mathrm{disc}}\) are mandatory log fields.

% -------------------------------------------------------------------------
\subsection*{15.1. Search Strategies and Restart Logic}
Within a Terrain Cell \(W\), the primary operation of a Hunter is to
minimize \(\Phi\) under the Ridge regime
(\(\mathrm{gap}_\tau < \Phi < \lambda_{\mathrm{sing}}\)).

\begin{remark}[Discrete gradient oracle, norm, and ledger \textbf{[Spec]}]\label{rk:15-grad}
In this chapter, “\(\nabla_{\!\delta}\Phi(x)\)” denotes an \emph{estimated} descent direction
\(g(x)\) produced by a declared gradient policy \texttt{grad\_policy}
(e.g.\ finite stencil / random perturbations / surrogate).
All distances and Lipschitz checks on \(\mathcal{M}\) use a declared norm
\(\|\cdot\|_{\mathcal{M}}\).
Any estimation/sampling/projection residuals introduced by the oracle are charged to
\(\delta_{\mathrm{meas}}\oplus\delta_{\mathrm{disc}}\) and recorded in the Hunter Action Log and \texttt{run.yaml}.
(Consistency with Remark~\ref{rk:14-grad-oracle} and Ch.~12 is mandatory.)
\end{remark}


\begin{definition}[Collapse-Based Gradient Descent (CBGD)]\label{def:15-cbgd}
Let \(x_k\in W\subset\mathcal{M}\).
Compute a discrete descent direction
\(g_k=\mathsf{GradEst}(\Phi_\tau,x_k; \texttt{grad\_policy})\)
using a declared stencil/sampling policy on \(W\) (Remark~\ref{rk:15-grad}).
If \(\|g_k\|_{\mathcal{M}}=0\) (below a declared threshold), use a controlled perturbation
(seed logged) to obtain \(g_k\neq 0\).
Propose
\[
x_{k+1}^{\mathrm{prop}}:=x_k-\alpha_k\,\frac{g_k}{\|g_k\|_{\mathcal{M}}},\qquad
x_{k+1}:=\Pi_W(x_{k+1}^{\mathrm{prop}}),
\]
where \(\Pi_W\) is the declared projection/clipping rule to keep iterates in \(W\).
All oracle/projection residuals are charged to \(\delta_{\mathrm{meas}}\oplus\delta_{\mathrm{disc}}\).
The step size \(\alpha_k\) is adaptive and governed by the acceptance rule in
Spec.~\ref{spec:15-restart}.
\end{definition}

\begin{specification}[Restart Logic via Convergence Manager]\label{spec:15-restart}
Fix declared tolerances \(\rho\ge 0\) (descent margin), \(N\in\mathbb{N}\) (stagnation horizon),
and a \(\tau\)-restart schedule \(\tau_{k}\) on a declared lattice.
At each step:

\begin{enumerate}
  \item \textbf{Accept/Reject.}
        Accept \(x_{k+1}\) only if \(\Phi_\tau(x_{k+1})\le \Phi_\tau(x_k)-\rho\).
        Otherwise shrink \(\alpha_k\) and retry up to a declared limit; log retries.

  \item \textbf{Stagnation.}
        If \(\Phi_\tau(x_k)>\mathrm{gap}_\tau\) and
        \(|\Phi_\tau(x_{k+\ell})-\Phi_\tau(x_k)|\le \rho\) for \(1\le \ell\le N\),
        declare stagnation and perform a \textbf{Window Restart}:
        subdivide \(W\) into smaller Terrain Cells \(W'\) (MECE, policy logged),
        and rerun CBGD on each \(W'\) under the same \(\texttt{grad\_policy}\) and \(\|\cdot\|_{\mathcal{M}}\).

  \item \textbf{\(\tau\)-Restart (scale artifact check).}
        If stagnation persists, switch to the next \(\tau_{k+1}\) in the declared schedule and recompute
        \(\Phi_{\tau_{k+1}}\) and the UCC check \(\mathrm{Gap}_{\tau_{k+1}}>\Sigma\delta\).
        Record \((\tau_k,\tau_{k+1})\) and the incremental ledger consumption.

  \item \textbf{Local Minimum (resolution-bounded).}
        If no restart (cell refinement nor \(\tau\)-restart) yields a descending move,
        mark \(x_k\) as a \emph{resolution-bounded local minimum} and hand off to the Lifter heuristic (Sec.~15.2).
\end{enumerate}

All decisions must be recorded in the Hunter Action Log, including
\texttt{grad\_policy}, \(\|\cdot\|_{\mathcal{M}}\), stencil size / samples, seeds,
\(\alpha_k\), accept/reject, and the charged \(\delta_{\mathrm{meas}}\oplus\delta_{\mathrm{disc}}\).
\end{specification}

\begin{remark}[Quantale Awareness]
CBGD minimizes the scalarized quantity \(\|\Sigma\delta\|_V\), but the
full quantale-valued vector \(\boldsymbol{\delta}\) is always retained
in the Hunter Action Log.
This allows post hoc analysis of which component (algebraic vs.\ numerical,
commutation vs.\ truncation) dominates the defect at a local minimum.
\end{remark}

% -------------------------------------------------------------------------
\subsection*{15.2. Handling Local Minima via Lifting Heuristics}
When a Hunter is trapped at a local minimum \(x^*\) satisfying
\(\Phi(x^*) \ge \mathrm{gap}_\tau\), the system must decide whether
this reflects:
\begin{itemize}
    \item a genuine candidate counterexample, or
    \item a projection artifact caused by insufficient dimension.
\end{itemize}

\begin{definition}[Lifting Condition]
A local minimum \(x^*\) is a candidate for Dimensional Lifting if:
\begin{enumerate}
    \item \textbf{Persistent Obstruction:}
          \(\Phi(x^*)\) remains above \(\mathrm{gap}_\tau\) under
          a \(\tau\)-sweep (varying collapse scale within admissible bounds).

    \item \textbf{Type IV Signature:}
          The tower diagnostics at \(x^*\) satisfy
          \((\mu(x^*),\nu(x^*)) \neq (0,0)\).

    \item \textbf{Cost Feasibility:}
          The base Lifting Penalty obeys
          \(\delta^{\mathrm{lift}}(1) < \mathrm{gap}_\tau\),
          so that a single lift is admissible under the UCC.
\end{enumerate}
\end{definition}

\begin{specification}[Lifting Heuristic Protocol]
Given a local minimum \(x^*\) satisfying the Lifting Condition:

\begin{enumerate}
    \item \textbf{Freeze Local Search.}
          Suspend further CBGD updates at \(x^*\).

    \item \textbf{Propose Auxiliary Axis.}
          The Lifter proposes an axis \(\mathcal{A}\) in accordance
          with Appendix~U (e.g.\ smoothing parameter, spectral softening,
          arithmetic depth).

    \item \textbf{Test Lift Gradient.}
          Evaluate the one-sided directional derivative (or finite difference)
          of \(\Phi\) along \(\mathcal{A}\) at \((x^*,0)\) in the lifted
          space \(\mathcal{M}\times \mathcal{A}\).

    \item \textbf{Commit or Reject Lift.}
          \begin{itemize}
              \item If a direction with \(\partial_{\mathcal{A}}\Phi < 0\)
                    exists and the augmented budget
                    \big\|\(\Sigma\delta \oplus \delta^{\mathrm{lift}}(1)\big\|_{V}
                    < \mathrm{gap}_\tau\) holds, commit the lift and
                    spawn a new Hunter on \(\mathcal{M}\times\mathcal{A}\).
              \item If all tested axes yield non-decreasing \(\Phi\),
      label \(x^*\) as a \textbf{Terminal Singularity} (hard instance under the current lifting policy)
      and halt lifting; no mathematical counterexample claim is made.

          \end{itemize}
\end{enumerate}
\end{specification}

\subsection*{15.3. Summary}
This chapter specifies the optimization core of the AK-HDPST search engine:

\begin{itemize}
    \item \textbf{CBGD} provides a regime-aware gradient descent adapted
          to the structure of \(\Phi\).
    \item \textbf{Restart Logic} uses window refinement to distinguish
          spurious local minima from resolution artifacts.
    \item \textbf{Lifting Heuristics} decide when and how to escape
          topological traps by controlled dimensional extension, subject
          to the Lifting Penalty of Chapter~14.
\end{itemize}

Together with Chapters~13 and~14, these protocols endow the Hunter and
Lifter with a mathematically disciplined behavior over the Defect
Potential landscape.



% =========================================================================
% Chapter 16 : Bridge Programs and Spectral-Gap Windows  (A-plan, conservative)
% =========================================================================

\section*{Chapter 16: Bridge Programs and Spectral-Gap Windows}
\addcontentsline{toc}{section}{Bridge Programs and Spectral-Gap Windows}
\label{sec:ch16}

\subsection*{16.0. Overview and motivation (Search layer only)}
Chapter~3 established the \emph{one-way bridge}
\[
\PH_1(\Ctau F)=0 \ \Longrightarrow\ \Ext^1\!\big(\Rfun(\Ctau F),k\big)=0
\quad\text{under (B1)--(B3)}.
\]
This direction is a \textbf{Core} result (Part~I) and is used by the AK Core.

In contrast, many global programs (PDE families, arithmetic families, etc.) would like to
use categorical evidence to guide topological decisions operationally.  In particular,
one may wish---\emph{at the Search Layer only}---to treat
\[
\Ext^1\!\big(\Rfun(\Ctau F|_W),k\big)\approx 0
\]
as evidence for
\(\PH_1(\Ctau F|_W)=0\) on a window/cell \(W\).
This chapter defines \textbf{Bridge Programs} B1--B3 that govern such \emph{reverse use}
under strict guard-rails.

\medskip
\noindent\textbf{A-plan principle (scope-first).}
Reverse use \(\Ext^1 \Rightarrow \PH_1\) is \emph{never} permitted globally or generically.
It is permitted only in the \texttt{definable\_or\_saturation} scope that is already
audited by the mandatory named test
\(\text{T--ExtZero}\Rightarrow\text{PHZero}\) (Ch.~12, \S12.2a; alias: T7/T17).
Spectral information may be used only as an \emph{auxiliary} safety check and does \emph{not}
expand the scope.

\subsection*{16.0+. Standing conventions (inherit)}
All comparisons obey the unique order:
\[
\text{for each } t \ \Longrightarrow\ \mathbf{P}_i \ \Longrightarrow\ \Ttau
\ \Longrightarrow\ \text{compare in }\Perskft,
\]
i.e.\ \textbf{T--PFBC--AfterCollapse}.  All quantities used in this chapter are computed
\emph{after} \(\Ttau\) (equivalently on \(\Ctau F\)).
Spectral conventions follow Ch.~11/12: eigenvalues in ascending order; norm choice
\(\|\cdot\|_{\mathrm{op}}\) or \(\|\cdot\|_{\mathrm{fro}}\) declared and logged.

% -------------------------------------------------------------------------
\subsection*{16.1. Program B1: the reverse-use problem (Search-layer) \textbf{[Spec]}}

\begin{definition}[The reverse-use problem]\label{def:16-reverse-problem}
Fix a Terrain Cell \(W\subset\mathcal{M}\) and a collapse scale \(\tau>0\).
The \emph{reverse-use problem} asks for conditions under which the Search Layer is allowed
to \emph{treat} the operational implication
\[
\Ext^1\!\big(\Rfun(\Ctau F|_W),k\big)=0
\ \Longrightarrow\
\PH_1(\Ctau F|_W)=0
\]
as a local proxy, without asserting any new Core equivalence.
\end{definition}

\begin{specification}[Bridge Program B1 (A-plan): reverse use under \texttt{definable\_or\_saturation} only]\label{spec:16-B1}
Bridge Program~B1 issues a \textbf{Reverse Certificate} on a Terrain Cell \(W\) only if the
following \emph{Collapse-Consistent Conditions (CCC)} hold.

\begin{enumerate}
  \item \textbf{Scope Gate (mandatory).}
  Program~B1 is \texttt{applicable} only if either:
  \begin{enumerate}
    \item[(a)] \(W\) is definable and \(E_1(W)=0\) (local bridge scope; Thm.~\ref{thm:E1-local}), or
    \item[(b)] the Saturation Gate is active on \(W\) at \(\tau^\ast\) (Decl.~\ref{gate:11-sat}).
  \end{enumerate}
  Otherwise Program~B1 is \texttt{inapplicable} and \emph{no reverse propagation is permitted}.

  \item \textbf{UCC compliance and ordering.}
  All computations are performed \emph{after} \(\Ttau\) (equivalently on \(\Ctau F\)),
  and all comparisons use \textbf{T--PFBC--AfterCollapse}.  Pre-collapse comparisons are out of scope.

  \item \textbf{Tower stability.}
  The tower diagnostics vanish on \(W\) at the declared \(\tau\):
  \(\DiagZero\) (Ch.~4).

  \item \textbf{Potential bound (Search-layer safety).}
  The Defect Potential satisfies \(\Phi_\tau(x) < \mathrm{gap}_\tau\) for all \(x\in W\)
  (Ch.~13).  This ensures UCC-style budget safety at the Search Layer.

  \item \textbf{Test gating (mandatory named test).}
  The mandatory named test \(\text{T--ExtZero}\Rightarrow\text{PHZero}\)
  (Ch.~12, \S12.2a; alias: T7/T17) is \texttt{applicable} and \texttt{passes},
  with full logs and artifacts.
\end{enumerate}

\noindent\textbf{Certificate semantics (Search layer only).}
If CCC holds, B1 records the local statement
\[
\Ext^1\!\big(\Rfun(\Ctau F|_W),k\big)=0
\ \Longrightarrow\
\PH_1(\Ctau F|_W)=0
\quad
\text{(\emph{window-local, Search layer only})}.
\]
This does \emph{not} assert any Core equivalence
\(\PH_1 \Leftrightarrow \Ext^1\); it is a gated proxy rule used only by the Search Layer.
\end{specification}

\begin{remark}[Failure modes and dispositions]\label{rk:16-B1-disposition}
If Program~B1 is \texttt{inapplicable} (outside \texttt{definable\_or\_saturation}),
the cell \(W\) remains \texttt{undecided} under reverse use.
The Search Layer may (i) continue exploration using \(\Phi\) and the Hunter protocol,
(ii) request refinement/restart, or (iii) flag for lifting when Type~IV is indicated.
\end{remark}

% -------------------------------------------------------------------------
\subsection*{16.2. Optional spectral safety: spectral-gap windows \textbf{[Spec]}}

\noindent\textbf{Positioning (A-plan).}
Spectral information is an \emph{auxiliary safety check}. It can reduce false positives
from near-zero numerical artifacts, but it \emph{never expands the applicability} of B1.
In particular, satisfying a spectral gap \emph{cannot} make B1 applicable outside
\texttt{definable\_or\_saturation}.

\begin{definition}[Spectral gap at scale \(\tau\)]\label{def:16-spectral-gap}
Let \(L_1(\Ctau F_x)\) denote the normalized combinatorial Hodge Laplacian
on degree \(1\) for the truncated object \(\Ctau F_x\) (Ch.~11).
Let \(\lambda_1(x)\le \lambda_2(x)\le\cdots\) be its positive eigenvalues (ascending order).
Define the \emph{spectral gap}:
\[
\gamma_\tau(x) \ :=\ \lambda_1(x),
\]
with the convention \(\gamma_\tau(x)=+\infty\) if there is no positive spectrum.
\end{definition}

\begin{definition}[Noise floor from the ledger]\label{def:16-noise-floor}
Let \(\Sigma\delta(x)\in V\) be the aggregated post-collapse budget (Appendix~S/L),
and let \(\|\cdot\|_V\) be the fixed scalarization used in Ch.~13.
Define the \emph{noise floor} on a cell \(W\) by
\[
\mathrm{NF}(W)\ :=\ \sup_{x\in W}\big\|\Sigma\delta(x)\big\|_V.
\]
Optionally, if the implementation separates measurement/discretization components,
record \(\mathrm{NF}_{\mathrm{meas+disc}}(W):=\sup_{x\in W}\|\delta_{\mathrm{meas}}(x)\oplus\delta_{\mathrm{disc}}(x)\|_V\).
\end{definition}

\begin{definition}[Spectral-gap condition (auxiliary)]\label{def:16-spectral-gap-cond}
A Terrain Cell \(W\) satisfies the \emph{Spectral-Gap Condition} at scale \(\tau\)
if there exists a constant \(c>1\) such that
\[
\inf_{x\in W}\gamma_\tau(x)\ >\ c\cdot \mathrm{NF}(W).
\]
The policy fixes \(c\) (default \(c=2\)) and logs it in \texttt{run.yaml}.
\end{definition}

\begin{specification}[Design guarantee: spectral robustness (auxiliary safety)]\label{spec:16-spectral-robustness}
When Program~B1 is \texttt{applicable} (Scope Gate satisfied) and CCC holds,
the Spectral-Gap Condition provides an auxiliary guard against
misclassifying a small positive mode as “zero” due to numerical noise.
All spectral parameters must be fixed and logged:
\begin{itemize}
  \item eigenvalue order: \texttt{ascending};
  \item norm choice for matrix perturbations: \texttt{op} or \texttt{fro};
  \item spectral bounds (at least \(\lambda_{\min},\lambda_{\max}\)) and optional \texttt{lip\_tol};
  \item the constant \(c\) in Def.~\ref{def:16-spectral-gap-cond};
  \item the backend precision / solver tolerance used for eigen-computation.
\end{itemize}
Passing the spectral check may be recorded as \texttt{spectral\_aux\_ok: true}
but \emph{cannot} override any failure of the Scope Gate or the mandatory named test.
\end{specification}

\begin{remark}[Minimal log block]\label{rk:16-spectral-log}
A minimal manifest extension (consistent with Ch.~12, Decl.~\ref{dec:12-schema}) is:
\begin{verbatim}
reverse_bridge:
  enabled: true
  scope: "definable_or_saturation"
  mandatory_test: "T-ExtZero->PHZero"
  spectral_aux:
    enabled: true
    c: 2.0
    gamma_inf_min: 0.15
    noise_floor_sup: 0.05
\end{verbatim}
\end{remark}

% -------------------------------------------------------------------------
\subsection*{16.3. Program B2: Global assembly at the Search Layer \textbf{[Spec]}}

\begin{specification}[Program B2: Search-layer regularity via coverage]\label{spec:16-B2}
\textbf{Input:} a parameter space \(\mathcal{M}\) and a realization \(F\) (Part~II).\\
\textbf{Goal:} produce a \emph{Search-layer} coverage artifact indicating that the domain of interest
is covered by locally validated cells, with all gluing checks passing.

\textbf{Procedure:}
\begin{enumerate}
  \item \textbf{Decomposition.}
  Hunters propose a cover of \(\mathcal{M}\) by Terrain Cells \(\{W_\alpha\}\)
  satisfying definability/MECE and Restart/Summability requirements (Ch.~12, T9/T11/T17).

  \item \textbf{Local evaluation.}
  On each \(W_\alpha\), run UCC/B-Gate\(^+\) checks and diagnostics.
  If Program~B1 is \texttt{applicable} and passes CCC (including T--ExtZero\(\Rightarrow\)PHZero),
  mark \(W_\alpha\) as \texttt{reverse-validated}.
  If B1 is \texttt{inapplicable}, keep the cell \texttt{undecided} under reverse use.

  \item \textbf{Singularity handling.}
  If the Peak regime is detected (Type~IV indicated by \((\mu,\nu)\neq(0,0)\)),
  invoke the Lifter under the Lifting Penalty (Ch.~14) and record outcomes.
  Unresolved peaks are marked \texttt{barrier}.

  \item \textbf{Gluing.}
  The Mapper runs Overlap Gate checks (Ch.~5; Ch.~12 T14) on intersections
  and builds a coverage graph of \texttt{reverse-validated} cells.

  \item \textbf{Search-layer report.}
  If the union of \texttt{reverse-validated} cells covers the target domain and
  no \texttt{barrier} remains, emit a \textbf{Search-layer Coverage Report}
  with full logs and certificates. This report is not a Core theorem.
\end{enumerate}
\end{specification}

\begin{remark}[Non-claims]\label{rk:16-B2-nonclaims}
A successful Program~B2 run does not prove analytic regularity or number-theoretic identities.
It is a reproducible Search-layer artifact asserting that, \emph{under the declared gates and tests},
the pipeline found no surviving barriers on the explored domain.
\end{remark}

% -------------------------------------------------------------------------
\subsection*{16.4. Program B3: Counterexample hunt at the Search Layer \textbf{[Spec]}}

\begin{specification}[Program B3: robust barrier discovery]\label{spec:16-B3}
\textbf{Goal:} search for stable Peak-regime regions (Type~IV) that persist across admissible repairs and lifts.

\textbf{Procedure:}
\begin{enumerate}
  \item \textbf{Maximization / targeting.}
  Hunters target points where \(\Phi_\tau\) is large, prioritizing cells that repeatedly fail UCC budgets.

  \item \textbf{Repair attempts.}
  Apply Restart/refinement (Ch.~12 T11) and admissible deletion/inclusion updates (Ch.~11/12 T2)
  to rule out discretization artifacts.

  \item \textbf{Lifting attempts.}
  Invoke the Lifter under the Lifting Penalty (Ch.~14), recording
  whether peaks disappear under admissible lifts.

  \item \textbf{Candidate output.}
  If a Peak persists across admissible refinements and lifts and continues to show
  \((\mu,\nu)\neq(0,0)\) at declared scales, output a
  \textbf{Certified Counterexample Candidate (Search-layer)} with complete artifacts:
  \texttt{bars/spec/aux/ext/phi.json} and the manifest \texttt{run.yaml}.
\end{enumerate}
Such candidates require independent mathematical analysis outside the automated pipeline.
\end{specification}

% -------------------------------------------------------------------------
\subsection*{16.5. Summary and guard-rails}
\begin{remark}[Summary]\label{rk:16-summary}
Bridge Programs B1--B3 formalize the Search-layer handling of reverse uses:
\begin{itemize}
  \item \textbf{B1 (A-plan)} allows reverse propagation only in the pre-audited
  \texttt{definable\_or\_saturation} scope and only when the mandatory named test
  T--ExtZero\(\Rightarrow\)PHZero passes (Ch.~12, \S12.2a).
  \item \textbf{Spectral gaps} are optional auxiliary safety checks; they do not expand scope.
  \item \textbf{B2/B3} define Search-layer coverage vs.\ barrier discovery workflows using Hunter/Mapper/Lifter.
\end{itemize}
All results remain within Part~II and do not modify Part~I theorems.
No claim of \(\PH_1\Leftrightarrow\Ext^1\) is made.
\end{remark}



% =========================================================================
% Chapter 17 : The AK-HDPST AI Platform (v17.0, aligned with Ch.11--16 + Ch.12 schema)
% =========================================================================

\section*{Chapter 17: The AK-HDPST AI Platform}
\addcontentsline{toc}{section}{The AK-HDPST AI Platform}
\label{sec:ch17}

\subsection*{17.0. Overview: from theory to execution}
The preceding chapters established (i) the mathematical Core (Part~I) and
(ii) the Search Layer protocols (Part~II), including the UCC ledger,
test suite, Bridge Programs (A-plan), and agent behaviors.
This chapter specifies the \emph{execution environment}---the AK-HDPST AI Platform.

In this framework, a “proof artifact” is not a static text but a
\emph{reproducible computational process}:
a manifest plus verifiable artifacts that can be re-executed by an independent auditor.
To preserve rigor, we enforce a strict separation of roles:
\begin{itemize}
  \item \textbf{AI Agents (Proposers):}
  untrusted agents (Hunters / Mappers / Lifters) proposing points, moves, restarts, and lifts.
  \item \textbf{AK Core (Verifier):}
  a trusted deterministic kernel implementing Part~I, the UCC logic, and the formal test suite (Ch.~12),
  auditing every proposal and issuing accept/reject verdicts.
\end{itemize}

\noindent\textbf{Standing convention.}
All persistence-layer quantities used for acceptance are computed \emph{after} \(\Ttau\)
(equivalently on \(\Ctau F\)), and all comparisons follow the unique order
\textbf{T--PFBC--AfterCollapse} (Ch.~12).

% -------------------------------------------------------------------------
\subsection*{17.1. \texttt{run.yaml} as a formal proof object (manifest + artifacts)}

\begin{definition}[Proof Object]\label{def:17-proof-object}
A \emph{Proof Object} in AK-HDPST is a tuple
\[
\mathcal{P} \;=\; (\texttt{run.yaml},\ \mathcal{A},\ \mathcal{H}),
\]
where \(\texttt{run.yaml}\) is a manifest, \(\mathcal{A}\) is the set of referenced artifacts
(barcodes, spectra, Ext-data, comparison maps, logs), and \(\mathcal{H}\) is the set of cryptographic hashes
binding the manifest to the artifacts and to the Core implementation/version used in the run.
\end{definition}

\begin{specification}[Manifest conformance and mandatory fields]\label{spec:17-manifest-conformance}
A \texttt{run.yaml} is \emph{conformant} if it satisfies the schema requirements of
Ch.~12 (Decl.~\ref{dec:12-schema}) and records at least:
\begin{itemize}
  \item coefficient field and backend policy;
  \item \(\tau\)-sweep specification and window policies;
  \item quantale block \texttt{quantale:\{name, op, unit, order, mode\}};
  \item layered deltas \texttt{layered\_delta:\{deltaGal, deltaTr, deltaFun\}} when applicable;
  \item definable windows \texttt{definable:\{structure, window\_formulae\}};
  \item Iwasawa block and AWFS/2-cell block when enabled;
  \item spectral policy (\texttt{order: ascending}, norm declaration) and bounds;
  \item Restart/Summability evidence and Overlap Gate fields;
  \item A/B commutativity policy and tolerance;
  \item explicit \((\mu,\nu)\) totals, \texttt{phi\_iso\_tail}, and pass criteria
        (\texttt{per-degree} vs.\ \texttt{aggregated}).
\end{itemize}
\end{specification}

\begin{specification}[Validity of a Proof Object (deterministic verification)]\label{spec:17-validity}
A Proof Object \(\mathcal{P}\) is \emph{valid} if and only if the following hold under
deterministic re-execution by the AK Core (same Core hash/version as recorded in \(\mathcal{H}\)):

\begin{enumerate}
  \item \textbf{MECE coverage (finite or countable).}
  The referenced Terrain Cells form a declared MECE cover of the target domain,
  with event accounting and coverage checks (Ch.~12, T9; countable via T--Countable--Cover).

  \item \textbf{Cell-level acceptance (UCC).}
  Every cell marked \texttt{passed: true} includes an AK Core certificate that
  verifies UCC acceptance (gap check and ledger bound) and the tower diagnostics policy
  (including Type~IV handling via \((\mu,\nu)\)) at the declared \(\tau\).

  \item \textbf{Gluing and overlap.}
  All claimed adjacencies/merges are supported by Overlap Gate logs and artifacts
  (Ch.~12, T14), including A/B soft-commuting or fallback with \(\delta^{\alg}\) charging
  (Ch.~12, T10).

  \item \textbf{Restart/Summability.}
  The global certificate includes Restart constants and Summability evidence
  (Ch.~12, T11 and T--Delta--Sum--Converges), preventing hidden accumulation of defects.

  \item \textbf{Reverse certificates (Bridge Program A-plan).}
  Any cell labeled \texttt{reverse\_certified: true} must satisfy:
  \begin{itemize}
    \item scope \texttt{definable\_or\_saturation} (Ch.~16, Program B1 A-plan),
    \item the mandatory named test \texttt{T-ExtZero->PHZero} is \texttt{applicable} and \texttt{passes}
          (Ch.~12, \S12.2a; alias T7/T17),
    \item CCC fields and artifacts are present (tower stability, potential bound, logs).
  \end{itemize}
  Optional spectral-gap checks may be recorded as \texttt{spectral\_aux\_ok}, but
  \emph{cannot} make an inapplicable reverse certificate valid.
\end{enumerate}

\noindent Verification reduces to re-running \texttt{run.yaml} against the AK Core;
acceptance is deterministic given the recorded hashes and policies.
\end{specification}

\begin{remark}[Artifact minimal set]\label{rk:17-artifacts}
A minimal auditable bundle referenced by \texttt{run.yaml} includes
\texttt{bars.json}, \texttt{spec.json}, \texttt{ext.json}, \texttt{phi.json},
and the stepwise logs (e.g.\ \texttt{hunter\_log.jsonl}, \texttt{mapper\_log.jsonl}),
optionally an \texttt{.h5} container for large runs (Ch.~12, \S12.3).
\end{remark}

% -------------------------------------------------------------------------
\subsection*{17.2. Reproducibility: Hunter Action Log schema (with gradient norm discipline)}

\begin{definition}[Hunter Action Log]\label{def:17-hunter-log}
A \emph{Hunter Action Log} is a sequential record
\(\mathcal{L}=(S_0,A_0,S_1,A_1,\dots)\) where
\(S_k\) is the state and \(A_k\) is the action at step \(k\),
sufficient to replay the exploration under the same manifest and Core.
\end{definition}

\begin{specification}[Required fields for replayability]\label{spec:17-log-replay}
A valid Hunter log must contain at least:

\begin{enumerate}
  \item \textbf{Initialization.}
  RNG seeds; initial point \(x_0\); initial Terrain Cell \(W_0\);
  hyperparameters (\(\alpha\), step caps, \(\lambda_{\mathrm{sing}}\), restart thresholds).

  \item \textbf{Regime classification data.}
  For each step: \(\Phi_\tau(x_k)\), the regime label (Plain/Ridge/Peak),
  the declared \(\tau\), and whether acceptance checks were invoked.

  \item \textbf{Approximate gradient specification (mandatory when used).}\label{it:17-grad-spec}
  If Navigation Mode returns an \emph{approximate gradient direction},
  the log must record the following as \textbf{[Spec]}:
  \begin{itemize}
    \item \textbf{Estimator type:} finite differences / random perturbation / surrogate model;
    \item \textbf{Stencil / perturbation details:} neighbor set, step radius, distribution, sample count;
    \item \textbf{Norm choice:} which norm on \(\mathcal{M}\) is used for Lipschitz control and step normalization;
    \item \textbf{Estimator error charging:} the gradient estimation error (and sampling residuals) is charged to
      \(\delta_{\meas}\oplus\delta_{\disc}\) (and logged distinctly from \(\delta^{\alg}\));
    \item \textbf{Acceptance rule:} whether the step is accepted by descent check and/or Core veto.
  \end{itemize}

  \item \textbf{Restart trace.}
  When restart/refinement occurs: the trigger condition, the refined cells \(W'\),
  updated local thresholds (if any), and the linkage to Restart/Summability evidence.

  \item \textbf{Lifting trace.}
  For each committed lift: axis type, local one-sided derivative / finite difference along the axis,
  charged Lifting Penalty \(\delta^{\mathrm{lift}}\), lifted coordinates, and the new Hunter spawn record.

  \item \textbf{Cross-check hooks.}
  References to artifacts produced/consumed at each step:
  \texttt{bars/spec/ext/phi.json} pointers and their hashes.
\end{enumerate}

\noindent\textbf{Replayability requirement.}
Given the same seeds, the same \texttt{run.yaml}, and the same Core hash/version,
a third party must be able to replay \(\mathcal{L}\) and recover the same
set of validated cells and certificates, up to declared numerical tolerances.
\end{specification}

% -------------------------------------------------------------------------
\subsection*{17.3. Architecture: proposer--verifier separation (white-box audit)}

The AK-HDPST AI Platform adheres to a strict Proposer--Verifier model:

\begin{itemize}
  \item \textbf{Proposer.}
  AI agents propose candidate moves: points \(x\), schedules of \(\tau\)-sweeps,
  restart/refinement steps, and candidate lifts.

  \item \textbf{Verifier (AK Core).}
  The Core computes \(\Phi\), updates the \(\delta\)-ledger, evaluates tower diagnostics \((\mu,\nu)\),
  runs formal tests (Ch.~12), checks gates (B-Gate\(^+\), Overlap Gate, CCC, saturation/definable scopes),
  and returns accept/reject for each proposed certificate.

  \item \textbf{Proof store.}
  Accepted cell certificates, overlap edges, and global coverage evidence are written to the proof store
  and referenced by hashes in \texttt{run.yaml}.
\end{itemize}

\begin{remark}[White-box principles]\label{rk:17-white-box}
The platform enforces:
\begin{enumerate}
  \item \textbf{AI visibility only.}
  Agents may observe diagnostics and potential values but cannot modify Core thresholds,
  comparison orders, or acceptance rules.

  \item \textbf{Core sovereignty.}
  If \(\Sigma\delta \nprec \mathrm{gap}_\tau\), if Type~IV policies fail,
  or if mandatory tests do not pass, the Core vetoes the certificate
  regardless of any agent confidence.

  \item \textbf{No black-box acceptance.}
  Claims are accepted only as a consequence of deterministic Core verification of the manifest and artifacts.
\end{enumerate}
\end{remark}

% -------------------------------------------------------------------------
\subsection*{17.4. Platform outputs: Map of Validity (as a verifiable object)}

\begin{definition}[Map of Validity]\label{def:17-map-validity}
A \emph{Map of Validity} is the union of:
(i) a MECE cell decomposition of the target domain,
(ii) cell labels (\texttt{valid}, \texttt{undecided}, \texttt{barrier}),
(iii) overlap edges and connected components produced by the Mapper,
and (iv) the complete set of certificates and logs referenced in \texttt{run.yaml}.
\end{definition}

\begin{remark}[Interpretation and non-claims]\label{rk:17-nonclaims}
The Map of Validity is a \emph{Search-layer} object: it is auditable, replayable,
and bounded by the implementable-range rules and tests.
It does not, by itself, assert new Core theorems (e.g.\ analytic regularity or number-theoretic identities).
Any global mathematical conclusion requires an explicit statement of scope, gates, and tests
used to construct the map.
\end{remark}

% -------------------------------------------------------------------------
\subsection*{17.5. Final conclusion}
AK-HDPST v17.0 completes the transition from a static theoretical framework
to a \emph{computational proof engine} in the implementable range.
By combining:
\begin{enumerate}
  \item the rigorous \textbf{Collapse Core} (Part~I),
  \item the disciplined \textbf{Search Layer} with formal tests and A-plan Bridge Programs (Part~II),
  \item and the auditable \textbf{AI Platform} (Chapter~17),
\end{enumerate}
the theory supports large-scale exploration of parameter spaces via a reproducible,
verifier-certified pipeline.

In this setting, the central deliverable is a \textbf{Map of Validity}:
a high-dimensional, replayable certificate bundle proposed by agents and
certified by the AK Core under UCC, the test suite, and strict guard-rails.



```tex
% =========================
\section*{Notation and Conventions (reinforced v17.0)}
% =========================
\phantomsection
\addcontentsline{toc}{section}{Notation and Conventions}

\paragraph{Base field and ambient categories.}
Fix a coefficient field \(k\) (Appendices~N/O may instead use a field \(\Lambda\); when used, replace \(k\) by \(\Lambda\) everywhere).
Let \(\Vectk\) be the abelian category of finite\hyp dimensional \(k\)\hyp vector spaces and write \([\bbR,\Vectk]\) for functors \((\bbR,\le)\to \Vectk\).

\paragraph{Constructible persistence and standing identification.}
We write
\[
\Perskft\ \subset\ [\bbR,\Vectk]
\]
for the full subcategory of \emph{constructible} persistence modules (pointwise finite\hyp dimensional with locally finite critical set on bounded windows).
Throughout the paper we \emph{identify} the “finite\hyp type’’ category with this constructible subcategory and use the symbol \(\Perskft\) uniformly.

\paragraph{Badges and layer discipline (Core vs.\ Search).}
Statements are tagged by their layer:
\(\,[\mathrm{Prop}]/[\mathrm{Thm}]\,\) are reserved for Part~I (Core) results;
\(\,[\mathrm{Declaration}]/[\mathrm{Spec}]\,\) record operational commitments (tests, manifests, logging, search heuristics);
\(\,[\mathrm{Conjecture}]\,\) denotes forward-looking, non-audited claims.
Reverse inferences (e.g.\ Ext\(\Rightarrow\)PH) are \emph{never} asserted as Core theorems; if used, they appear only as Search-layer specifications with explicit certificates and ledger charges.

\paragraph{Filtered objects and persistence.}
\(\FiltCh{k}\) denotes filtered chain complexes of finite\hyp dimensional \(k\)\hyp spaces; filtered quasi\hyp isomorphism is abbreviated f.q.i.
For \(i\in\ZZ\) the degree–\(i\) persistence functor is
\[
\Pfun_i:\ \FiltCh{k}\longrightarrow \Perskft,\qquad \Pfun_i(F)(t)=\rH=\mathrm{H}(F^t).
\]
Realizations from other formalisms into \(\FiltCh{k}\) are denoted \(\Rfun(-)\) (and \(\mathcal{F}(-)\), etc.\ as appropriate).

\paragraph{Quantale enrichment, \texorpdfstring{$\delta$}{delta}-ledger, and definable windows (UCC layer).}
Unless stated otherwise, the measurement layer is enriched over a fixed \emph{commutative quantale} \(V\) (write \((V,\oplus,\le,0)\)).
Distances are Lawvere \(V\)\hyp valued, and all aggregates (budgets, residuals, tolerances) are taken using the single operation \(\oplus\) (``quantale\hyp sum'').
The \(\delta\)\hyp ledger is recorded in (possibly product) quantales and is decomposed as
\[
\delta\ =\ \delta_{\alg}\ \oplus\ \delta_{\disc}\ \oplus\ \delta_{\meas},
\qquad\text{with optional layered bookkeeping }(\delta^{\Gal},\delta^{\Tr},\delta^{\Fun})\text{ when enabled.}
\]
Window sets are \emph{right\hyp open} and \(\MECE\) along a \(\tau\)\hyp sweep, and—when required—\emph{Denef--Pas definable}; definability guarantees finite event sets and finite Čech depth on bounded windows (Appendix~Q; cf.\ Appendices~H/J for the real o\hyp minimal surrogate \(\bbR_{\mathrm{an},\exp}\)).
The triple \((V,\text{definable windows},\text{AWFS/2\hyp cells})\) is referred to as the \emph{Unified Collapse Contract (UCC)}.

\paragraph{Reflection/truncation vs.\ window clipping.}
For \(\tau\ge 0\), let \(\Ecat\subset\Perskft\) be the Serre subcategory generated by bars of length \(\le \tau\).
The \emph{reflector} (bar\hyp deletion truncation)
\[
\Ttau:\ \Perskft\longrightarrow \Orth
\]
is exact, idempotent, and left adjoint to the inclusion \(\Orth\hookrightarrow\Perskft\) (Appendix~A, Theorem~\ref{A:thm:localization}); it is \(1\)\hyp Lipschitz for \(\intdist\) and, more generally, \(V\)\hyp \(1\)\hyp Lipschitz in the \(V\)\hyp enriched metric (Appendix~A, Proposition~\ref{A:prop:lipschitz}).
On filtered complexes we use a collapser \(\Ctau\) with a natural (up to f.q.i.) identification
\[
\Pfun_i(\Ctau F)\ \cong\ \Ttau\big(\Pfun_iF\big)\qquad\text{(natural in \(F,i\)).}
\]
For a right\hyp open window \(W=[u,u')\), the \emph{window clip} is denoted \(\Crop_W\) (a.k.a.\ \(W_{\mathrm{clip}}\); Definition~\ref{def:Wclip}): it restricts to \(W\) and extends by zero; it is exact and \(1\)\hyp Lipschitz for \(\intdist\) (Appendix~A and Chapter~6).
\emph{Warning.} \(\Ttau\) (delete bars of length \(\le\tau\)) and \(\Crop_W\) (clip to a window) play distinct roles and must not be conflated.

\paragraph{Optional safe low\hyp pass (after\hyp collapse only).}
A safe low\hyp pass operator \(\mathrm{LP}_\tau\) may be applied \emph{after} \(\Ttau\) and \(\Crop_W\), but \emph{only} to spectral auxiliaries derived from \(L(\Ctau F|_W)\) (Chapter~6.0bis, Definition~\ref{def:LP}).
It never modifies the persistence\hyp layer objects \(\Ttau\Pfun_i(F|_W)\) used for gates and certification; non\hyp compliant filters are not adopted for certification.

\paragraph{AWFS viewpoint and \(2\)\hyp cells.}
When enabled, we use an algebraic weak factorization system on \(\Ho(\FiltCh{k})\),
\[
\mathrm{Id}\ \Rightarrow\ L\ \dashv\ R\ \Rightarrow\ \mathrm{Id},\qquad R=\Ctau,
\]
to organize preprocessing/realization. Any resulting \(2\)\hyp cell defects are accounted for as algorithmic ledger entries \(\delta_{\alg}\) (Chapter~5; Appendices~K/L) and logged in the run manifest (Appendix~G).

\paragraph{Interleaving metric and shifts.}
On \(\Perskft\) the interleaving metric \(\intdist\) equals the bottleneck distance in the constructible \(1\)D setting.
The time shift is \((S^\varepsilon M)(t):=M(t+\varepsilon)\); shifts commute canonically with \(\Ttau\), hence \(\Ttau\) is \(1\)\hyp Lipschitz (and \(V\)\hyp \(1\)\hyp Lipschitz when the measurement layer is \(V\)\hyp enriched).

\paragraph{Unique comparison order (after\hyp collapse policy).}
All comparisons (local, overlap, global) follow the unique order
\[
\boxed{\ \text{for each }t\ \Longrightarrow\ \Pfun_i\ \Longrightarrow\ \Ttau\ \Longrightarrow\ \text{compare in }\Perskft\ }.
\]
Equivalently, comparisons are performed \emph{after applying \(\Ttau\)} (equivalently on \(\Ctau F\)) and \emph{never} on pre\hyp collapse objects.
This order is mandatory for audits, spectral alignment, and overlap gluing (Chapter~11, \S11.0 bis; Chapter~5).

\paragraph{Barcodes, events, and endpoint convention.}
Barcodes use half\hyp open intervals \(I=[b,d)\) with \(d\in\bbR\cup\{\infty\}\) and multiplicity \(m(I)\in\ZZ_{\ge1}\).
Any consistent open/closed choice yields the same clipped lengths and event sets.
For \(\tau\ge 0\) the clipped length is
\[
\ell_{[0,\tau]}(I):=\max\{0,\min\{d,\tau\}-\max\{b,0\}\}.
\]
Given \(\tau_0>0\), the finite event set in degree \(i\) is
\[
\mathsf{Ev}_i(F;\tau_0)=\{0,\tau_0\}\ \cup\ \bigl(\{b\in[0,\tau_0]\}\cap\mathrm{births}\bigr)\ \cup\ \bigl(\{d\in[0,\tau_0]\}\cap\mathrm{deaths}\bigr),
\]
with endpoint conventions and infinite bars as in Appendix~A, Remark~\ref{A:rk:endpoints}.

\paragraph{Betti curves and Betti integral.}
\(\beta_i(F;t):=\dim_k H_i(F^t)\)is c\`adl\`ag and piecewise constant on bounded windows.
The (clipped) Betti integral is
\[
\mathrm{PE}_i^{\le\tau}(F)=\int_0^\tau \beta_i(F;t)\,dt\ =\ \sum_{I\in\mathcal{B}_i(F)} m(I)\,\ell_{[0,\tau]}(I)\qquad\text{(Appendix~H).}
\]

\paragraph{Length spectrum operator.}
For \(M\in\Perskft\) with barcode \(M\simeq\bigoplus_j I[b_j,d_j)\) and a right\hyp open window \(W=[u,u')\), the \emph{length spectrum} operator \(\Len(M;W)\) (Chapter~2) is diagonal on a bar\hyp basis with eigenvalues \(\ell_W(I[b_j,d_j))\).
Its multiset of eigenvalues equals the multiset of clipped bar lengths and is invariant under isomorphisms \(M\simeq M'\).
In particular, for \(\Ttau\Pfun_i(F)\) and \(W=[0,\tau]\),
\[
\mathrm{PE}_i^{\le \tau}(F)\ =\ \|\Len(\Ttau\Pfun_i(F);[0,\tau])\|_{L^1}\,,
\]
hence the total collapse energy is an isomorphism invariant of the truncated persistence (Chapter~11, \S11.0+).

\paragraph{Spectral policy, ordering, and matrix norms (after\hyp collapse only).}
Spectral indicators are computed on \(L(\Ctau F)\) (normalized combinatorial Hodge Laplacian), never on pre\hyp collapse objects.
Positive eigenvalues are reported in \emph{ascending} order.
The matrix norm used for tolerances is declared as \(\|\cdot\|_{\op}\) (operator) or \(\|\cdot\|_{\mathrm{fro}}\) (Frobenius) and recorded in the run manifest (Appendix~G).
Allowed uses of spectrum are limited to: (i) deletion\hyp type non\hyp increase and (ii) \(\varepsilon\)\hyp continuation stability under \(\intdist\) (Appendix~E); no analytic/geometric conclusions are asserted beyond these audited regimes.

\paragraph{Towers and diagnostics; \texorpdfstring{$\DiagZero/\DiagNonzero$}{DiagZero/DiagNonzero} and Type IV.}
A \emph{tower} is a directed system \(F=(F_n)_{n\in I}\) with colimit \(F_\infty\).
For \(i\in\ZZ\), \(\tau\ge 0\), the comparison map is
\[
\phi_{i,\tau}(F):\ \varinjlim_{n}\ \Ttau\!\big(\Pfun_i(F_n)\big)\longrightarrow \Ttau\!\big(\Pfun_i(F_\infty)\big).
\]
Set \(\mu_{i,\tau}(F):=\gdim\ker\phi_{i,\tau}(F)\) and \(\nu_{i,\tau}(F):=\gdim\coker\phi_{i,\tau}(F)\); the totals are
\[
\muc(F):=\sum_{i}\mu_{i,\tau}(F),\qquad \nuc(F):=\sum_{i}\nu_{i,\tau}(F).
\]
We write \(\DiagZero\) for \((\muc,\nuc)=(0,0)\) and \(\DiagNonzero\) for \((\muc,\nuc)\neq(0,0)\).
The regime \(\DiagNonzero\) is the \emph{Type IV (invisible defect)} diagnostic of v17.0.
Cofinal restriction leaves \((\DiagZero / \DiagNonzero)\) unchanged; finite direct sums add; composition is subadditive and the rules extend to \(V\)\hyp distances (Appendix~J).
If \(\phi_{i,\tau}\) is an isomorphism for all relevant \(i\), then \(\DiagZero\).
The diagnostic \(\muc\) (and \(\nuc\)) is a persistence\hyp layer obstruction and is \emph{distinct} from the classical Iwasawa \(\mu\)\hyp invariant.

\paragraph{Gate cascade (audited evaluation order; no illicit reverse theorems).}
The default gate cascade is \emph{operated} as
\[
E_1{=}0\ \Longrightarrow\ (\mu,\nu){=}(0,0)\ \Longrightarrow\ \Ext^1{=}0\ \Longrightarrow\ \mathrm{PH}_1{=}0,
\]
as a sequent calculus with cut elimination (Chapter~6.3bis; Chapter~1/3/5): success at a later stage never overturns failure at an earlier stage.
Logical implications are only claimed where explicitly proved or certified; in particular, no global equivalence \(\mathrm{PH}_1\Leftrightarrow\Ext^1\) is asserted.

\paragraph{Overlap Gate (local\texorpdfstring{$\to$}{->}global gluing).}
Given a windowed cover \(\{(X_\alpha,W_\alpha)\}\) by right\hyp open \emph{definable} windows, the \emph{Overlap Gate} (Chapter~1; Chapter~5) requires, after truncation by \(\Ttau\),
\begin{itemize}
  \item local post\hyp collapse equivalence on overlaps up to the recorded \(\delta\)\hyp budget (quantitative commutation);
  \item Čech–\(\Ext^1\) acyclicity in degree \(1\) (finite Čech depth holds by definability);
  \item stability\hyp band condition \(\DiagZero\) and near\hyp \(\tau\) non\hyp accumulation.
\end{itemize}
When all overlaps pass, local truncated objects glue (uniquely up to isomorphism) in \(\Perskft\).
Run manifests record these under \texttt{overlap\_checks} (Appendix~G): \texttt{local\_equiv}, \texttt{cech\_ext1\_ok}, \texttt{stability\_band\_ok}.

\paragraph{A/B soft\hyp commuting; Mirror/Transfer; pipeline budget; B\hyp Gate\(^{+}\).}
For exact reflectors \(T_A,T_B\) in \(\Perskft\), the commutation defect is
\[
\Delta_{\mathrm{comm}}(M;A,B)=\intdist\big(T_AT_BM,\ T_BT_AM\big).
\]
Given tolerance \(\eta\), if \(\Delta_{\mathrm{comm}}\le\eta\) we accept \emph{soft\hyp commuting}; else fix an order (e.g.\ \(T_B\circ T_A\)) and \emph{record} \(\Delta_{\mathrm{comm}}\) as \(\delta_{\alg}\) (Appendices~K/L).
For Mirror/Transfer functors \(\Mirror\), assume a natural \(2\)\hyp cell \(\Mirror\circ \Ctau\Rightarrow \Ctau\circ\Mirror\) with a uniform bound \(\delta(i,\tau)\ge 0\) in \(\intdist\), additive along pipelines and \emph{non\hyp increasing} under \(1\)\hyp Lipschitz post\hyp processing (Appendix~L).
On a window \(W\) and degree \(i\), the pipeline budget aggregates (in the quantale)
\[
\Sigma\delta(i,\tau)\ =\ \bigoplus_{\text{Mirror--Collapse}}\delta(i,\tau)\ \oplus\ \bigoplus_{\text{A/B fails}}\Delta_{\mathrm{comm}}\ \oplus\ \bigoplus_{\text{audits}}\bigl(\delta_{\disc}\oplus\delta_{\meas}\bigr).
\]
The \emph{safety margin} \(\mathrm{gap}_\tau>0\) is configured per window and degree; \(\mathrm{B\text{--}Gate}^{+}\) requires \(\mathrm{gap}_\tau>\Sigma\delta(i,\tau)\).
Across windows, Restart and Summability hold if there exists \(\kappa\in(0,1]\) with
\[
\mathrm{gap}_{\tau_{k+1}}\ \ge\ \kappa\bigl(\mathrm{gap}_{\tau_k}-\Sigma\delta_k(i)\bigr),\qquad \bigoplus_k \Sigma\delta_k(i)\ <\ \infty
\]
(Appendix~J), with the second condition interpreted in the chosen quantale.

\paragraph{Arithmetic alignment conventions (Iwasawa Gate).}
When applicable, the Iwasawa diagnostic \(\mu_{\mathrm{Iwasawa}}\) is compared with the collapse diagnostic \(\mu_{\mathrm{Collapse}}:=\muc\) in a three\hyp state regime (lower bound / match / drift\hyp corrected) used by the Gate cascade to suppress Type\hyp IV drift (Chapter~7; Appendix~R).
Finite kernel/cokernel effects are absorbed into \(\delta_{\alg}\).

\paragraph{Categorical check and window\hyp local trigger.}
With \(\mathcal{Q}:=\{k[0]\}\), we test \(\Ext^1(\Rfun(\Ctau F),Q)=0\) \emph{after truncation}.
The one\hyp way bridge \(\mathrm{PH}_1\Rightarrow \Ext^1\) is used only under \textup{(B1)–(B3)} (field coefficients; constructible range; \(t\)\hyp exact realization of amplitude \(\le 1\)).
On a definable right\hyp open window \(W\) of finite Čech depth we adopt the \emph{window\hyp local trigger} via the first energy page \(E_1\):
\[
E_1(W)=0\ \Longleftrightarrow\ \mathrm{PH}_1\!\big(\Ctau F\!\mid_W\big)=0\ \Longleftrightarrow\ \Ext^1\!\big(\Rfun(\Ctau F)\!\mid_W,k\big)=0,
\]
while globally only sanctioned one\hyp way implications (and certified pasting via Restart/Summability) are used.

\paragraph{Reproducibility (manifest) and mandatory fields.}
The run manifest \texttt{run.yaml} (v17) is treated as a proof object (Chapter~17; Appendix~G) and must include, at minimum:
\texttt{version}, \texttt{suite\_version}, \texttt{seed};
UCC hooks \texttt{quantale}, \texttt{definable}, \texttt{awfs\_2cell}, \texttt{layered\_delta};
\texttt{windows} (right\hyp open, \MECE, and coverage), \texttt{coverage\_check};
\texttt{operations} (including A/B commuting policy and fallbacks), \texttt{budget} (including \texttt{gap\_tau} and \(\oplus\)\hyp totals),
\texttt{gate} and \texttt{overlap\_checks};
\texttt{persistence} (PH/Ext/\((\muc,\nuc)\) and tail isomorphism flags);
\texttt{spectral\_policy} with mandatory \texttt{order: "ascending"} and \texttt{norm: "op"|"fro"} plus any declared bounds/tolerances;
and, when enabled, \texttt{iwasawa} and other cross\hyp domain control blocks.
All persistence\hyp layer quantities are computed after applying \(\Ttau\) (equivalently on \(\Ctau F\)); spectral auxiliaries follow the fixed normalization policy; any residual slack is charged to the ledger and logged.

\paragraph{Global guard\hyp rails and non\hyp claims.}
All statements are confined to the persistence/spectral/categorical layers in the implementable range.
No number\hyp theoretic identity, analytic regularity theorem, or group trivialization is asserted.
No claim of \(\mathrm{PH}_1\Leftrightarrow \Ext^1\) is made; only the one\hyp way implication \(\mathrm{PH}_1\Rightarrow \Ext^1\) (under \textup{(B1)–(B3)}) is used.
The collapse obstruction \(\muc\) is a persistence\hyp level diagnostic and is distinct from the classical Iwasawa \(\mu\).

\medskip
\noindent\emph{Abbreviations.}
f.q.i.\ = filtered quasi\hyp isomorphism;\quad
c\`adl\`ag = right\hyp continuous with left limits;\quad
UCC = Unified Collapse Contract;\quad
AWFS = algebraic weak factorization system;\quad
DP = Denef--Pas definable;\quad
\(\mathrm{LP}_\tau\) = safe low\hyp pass;\quad
“window” \(=\) right\hyp open interval \(W=[u,u')\) (often \(W=[0,\tau]\)).
```




\appendix
\section*{Appendix A. Constructible Persistence: Abelianity, Serre Localization, and the V-Nucleus View (reinforced v17.0)}
% =========================

\addcontentsline{toc}{section}{Appendix A. Constructible Persistence: Abelianity, Serre Localization, and the V-Nucleus View (reinforced v17.0)}

\begingroup
Throughout this appendix, fix a field \(k\).
Write \(\Pers_k\) for the category of right-continuous persistence modules
\(M:(\mathbb{R},\le)\to\Vect_k\) with structure maps \(M(t\le t')\).
We denote by \(\Pers^{\mathrm{ft}}_k\subset\Pers_k\) the \emph{constructible} (finite-type) subcategory used in the main text.

\medskip
\noindent\textbf{Global conventions (v17.0 alignment).}
(i) All \(\Ext\)-tests are taken against \(Q=k[0]\) (the unit interval module supported at a point), unless explicitly stated otherwise.\\
(ii) Windowed energies use an exponent \(\alpha>0\) (default \(\alpha=1\)).\\
(iii) References to appendices use the tilde style (e.g.\ Appendix~D); failure types use the dash style
\(\mathrm{Type\ I\text{--}II}\), \(\mathrm{Type\ III}\), \(\mathrm{Type\ IV}\).\\
(iv) Notational disambiguation: the reflector (Serre localization) functor is denoted \(\mathbf{T}_\tau\); truncations/clippings on domains are denoted \(\mathbf{Tr}_\tau\) or \(\clip_{[a,b)}\). This resolves the potential collision sometimes found in informal notes where \(\mathbf{T}_\tau\) was used for truncation.\\
(v) When window partitions are used (MECE; §A.6), \emph{half-open, right-open} (i.e.\ left-closed/right-open) is the standing endpoint convention for domain windows and spectral bins; coverage checks are mandatory and any discretization slack is recorded (Appendix~G).\\
(vi) All results in §§A.2–A.5 are stated and used in the one-parameter (1D), field-coefficient, right-continuous, constructible setting.\\
(vii) \emph{\(\delta\)-ledger interface.} Quantitative non-commutation and implementation defects are externalized as
\[
\delta\ =\ \delta_{\mathrm{alg}}\ \oplus\ \delta_{\mathrm{disc}}\ \oplus\ \delta_{\mathrm{meas}}\ \in V,
\]
summed in a fixed commutative quantale \(V\) (see §A.0). Windowed pipelines aggregate \(\delta\)'s using \(\oplus\).
When the layered ledger is enabled, entries may additionally be reported in boxed form \((\delta^{\Gal},\delta^{\Tr},\delta^{\Fun})\) and then aggregated in the product quantale (Appendix~S; schema: Appendix~G).

\begin{remark}[Scope / after-collapse alignment]\label{A:rk:scope-after}
Appendix~A asserts equalities and exactness statements purely at the persistence layer \(\Pers^{\mathrm{ft}}_k\).
All \emph{operational} quantitative comparisons (distances, Lipschitz constants, monotonicity, budgets) used in the paper are evaluated in the mandated order \(\mathbf{P}_i\Rightarrow \mathbf{T}_\tau\Rightarrow \Pers^{\mathrm{ft}}_k\) on fixed right-open windows (after-collapse policy).
Equivalently, whenever the main text views these objects via a filtered model \(F\), all such comparisons are made \emph{after applying} \(\mathbf{T}_\tau\) (equivalently on \(\mathcal{C}_\tau F\)) and then returned to \(\Pers^{\mathrm{ft}}_k\); no filtered-level equalities are asserted in this appendix.
\end{remark}

% -------------------------
\subsection*{A.0. $V$-nucleus viewpoint and the $\delta$-interface}
% -------------------------
Let \(V\) be a commutative quantale \((V,\oplus,\le,0)\), and endow \(\Pers^{\mathrm{ft}}_k\) with a Lawvere \(V\)-metric \(d_V\) induced by the interleaving (bottleneck) distance.
(The canonical choice is \(V=[0,\infty]\) with \(\oplus=+\) and the usual order \(\le\).)

\begin{definition}[$V$-nucleus (Lawvere setting)]\label{A:def:Vnucleus}
A \emph{$V$-nucleus} on a \(V\)-metric category \((\mathcal{C},d_V)\) is an endofunctor \(N:\mathcal{C}\to\mathcal{C}\) together with a natural transformation \(\eta:\mathrm{Id}\Rightarrow N\) such that:
\begin{itemize}
  \item (Idempotence) \(N\circ N\simeq N\).
  \item (Non-expansiveness / \(V\)-\(1\)-Lipschitz) \(d_V(NX,NY)\le d_V(X,Y)\) for all \(X,Y\).
  \item (Exactness when an abelian structure is present) \(N\) is exact on the underlying abelian category (equivalently, preserves finite limits and finite colimits).
\end{itemize}
\end{definition}

\begin{proposition}[$\mathbf{T}_\tau$ is a $V$-nucleus]\label{A:prop:Vnucleus}
In the constructible 1D range, the Serre reflector
\(\mathbf{T}_\tau:\Pers^{\mathrm{ft}}_k\to\Pers^{\mathrm{ft}}_{k,\tau\text{-loc}}\)
(Theorem~\ref{A:thm:localization}) is a $V$-nucleus in the sense of Definition~\ref{A:def:Vnucleus}:
it is idempotent (Cor.~\ref{A:cor:idemp-cons}), \(1\)-Lipschitz for interleavings (Prop.~\ref{A:prop:lipschitz}), and exact (Theorem~\ref{A:thm:localization}).
\end{proposition}

\begin{remark}[$\delta$-commutation schema]\label{A:rk:delta-schema}
At the categorical level, functors that preserve the \(\tau\)-ephemeral Serre class commute strictly with \(\mathbf{T}_\tau\).
Operationally, we record any implementation-level discrepancy by a nonnegative \(\delta\in V\), decomposed as in (vii).
For clipping on a right-open window \(W=[u,v)\) we define a \emph{clipping defect} \(\delta_{\mathrm{clip}}(W,\tau)\in V\) with the contract
\[
d_V\bigl(\,\mathbf{T}_\tau\circ\clip_{W}(M),\ \clip_{W}\circ\mathbf{T}_\tau(M)\,\bigr)\ \le\ \delta_{\mathrm{clip}}(W,\tau),
\]
which must be logged (Appendix~G) and is \emph{subadditive} over pipelines and windows (with respect to \(\oplus\)).
By default, \(\delta_{\mathrm{clip}}\) is charged to \(\delta_{\mathrm{disc}}\oplus\delta_{\mathrm{meas}}\) unless an algorithmic \(2\)-cell origin is explicitly declared (Appendices~K/L).
In the ideal (exact) model, \(\delta_{\mathrm{clip}}(W,\tau)=0\) (Lemma~\ref{A:lem:clip}); any nonzero value reflects discretization/rounding policies and is accounted for in the \(\delta\)-ledger.
\end{remark}

% -------------------------
\subsection*{A.1. Constructible objects}
% -------------------------
\begin{definition}[Constructible / finite-type]\label{A:def:constructible}
A persistence module \(M\in\Pers_k\) is \emph{constructible} (finite-type) if on every bounded interval
\([a,b]\subset\mathbb{R}\) it has a \emph{finite critical set}: there exist
\(a=t_0<t_1<\dots<t_N=b\) such that each structure map
\(M(t\le t')\) is an isomorphism whenever \(t,t'\in (t_j,t_{j+1})\) for some \(j\).
Equivalently, \(M\) is pointwise finite-dimensional and admits a barcode decomposition as a
\emph{locally finite direct sum of interval modules}, i.e.\ only finitely many intervals intersect any bounded window.
We write \(\Pers^{\mathrm{ft}}_k\) for the full subcategory of such modules.
\end{definition}

\begin{remark}
In the 1D, field-coefficient, right-continuous setting, the equivalence above is standard (barcode decomposition).
All constructions below (kernels, cokernels, torsion, Serre localization, clipping) preserve constructibility and are controlled by finitely many events on bounded windows; see the references at the end of this appendix.
\end{remark}

% -------------------------
\subsection*{A.2. Abelianity}
% -------------------------
\begin{proposition}\label{A:prop:abelian}
\(\Pers^{\mathrm{ft}}_k\) is an abelian category.
Moreover, for a morphism \(f:M\to N\) in \(\Pers^{\mathrm{ft}}_k\),
kernels and cokernels are computed pointwise in \(\Vect_k\) and remain constructible.
\end{proposition}

\begin{proof}
Evaluation at each \(t\in\mathbb{R}\) is exact in \(\Vect_k\), hence pointwise kernels and cokernels define functorial sub/quotient persistence modules.
Constructibility is preserved: on any bounded window one refines the break sets of \(M,N\) to a finite set controlling \(\Ker f\) and \(\Coker f\).
Exactness axioms follow objectwise; hence \(\Pers^{\mathrm{ft}}_k\) is abelian with pointwise exactness.
\end{proof}

% -------------------------
\subsection*{A.3. The \texorpdfstring{$\tau$}{tau}-ephemeral Serre subcategory}
% -------------------------
Fix \(\tau>0\).
Let \(I[a,b)\) denote the interval module supported on \([a,b)\) (half-open, right-open convention).

\begin{definition}[\(\tau\)-ephemeral subcategory]\label{A:def:Etau}
Let \(\mathsf{E}_\tau\subset\Pers^{\mathrm{ft}}_k\) be the smallest full subcategory
containing all interval modules \(I[a,b)\) with length \(b-a\le \tau\) and closed under subobjects, quotients, and extensions.
We call \(\mathsf{E}_\tau\) the \emph{\(\tau\)-ephemeral} (or \(\tau\)-torsion) subcategory.
\end{definition}

\begin{lemma}\label{A:lem:Serre}
\(\mathsf{E}_\tau\) is a Serre subcategory of \(\Pers^{\mathrm{ft}}_k\), and it is hereditary as a torsion class.
\end{lemma}

\begin{remark}[Endpoint conventions]\label{A:rk:endpoints}
All statements in this appendix are insensitive to the choice of open/closed endpoints on interval modules.
We fix \([a,b)\) for definiteness; changing endpoint conventions does not affect lengths, barcode decompositions, interleaving/bottleneck distances, or any categorical constructions below.
\end{remark}

% -------------------------
\subsection*{A.3.1. The torsion pair and maximal \texorpdfstring{$\tau$}{tau}-ephemeral subobject}
% -------------------------
Define the \emph{\(\tau\)-local (orthogonal)} subcategory
\[
\Pers^{\mathrm{ft}}_{k,\tau\text{-loc}}
\ :=\
\bigl\{\,X\in\Pers^{\mathrm{ft}}_k\ \big|\ \Hom(E,X)=0=\Ext^1(E,X)\ \text{ for all }E\in\mathsf{E}_\tau\,\bigr\}.
\]
Then \((\mathsf{E}_\tau,\ \Pers^{\mathrm{ft}}_{k,\tau\text{-loc}})\) is a torsion pair: for each \(M\) there is a functorial short exact sequence
\[
0\ \longrightarrow\ t_\tau(M)\ \longrightarrow\ M\ \longrightarrow\ f_\tau(M)\ \longrightarrow\ 0
\]
with \(t_\tau(M)\in\mathsf{E}_\tau\) and \(f_\tau(M)\in \Pers^{\mathrm{ft}}_{k,\tau\text{-loc}}\).

% -------------------------
\subsection*{A.4. The reflector \texorpdfstring{$\mathbf{T}_\tau\dashv\iota_\tau$}{T\_\tau ⊣ ι\_\tau}, exactness, and the $V$-nucleus corollary}
% -------------------------
Let \(\iota_\tau:\Pers^{\mathrm{ft}}_{k,\tau\text{-loc}}\hookrightarrow\Pers^{\mathrm{ft}}_k\) be the inclusion.

\begin{theorem}[Exact reflective localization]\label{A:thm:localization}
The Serre quotient functor
\[
\pi_\tau:\ \Pers^{\mathrm{ft}}_k\ \longrightarrow\ \Pers^{\mathrm{ft}}_k/\mathsf{E}_\tau
\]
is exact.
In the 1D constructible setting there is a canonical exact equivalence of abelian categories
\[
\Pers^{\mathrm{ft}}_k/\mathsf{E}_\tau \ \simeq\ \Pers^{\mathrm{ft}}_{k,\tau\text{-loc}}.
\]
Composing \(\pi_\tau\) with this equivalence yields a functor
\[
\mathbf{T}_\tau:\ \Pers^{\mathrm{ft}}_k\ \longrightarrow\ \Pers^{\mathrm{ft}}_{k,\tau\text{-loc}}
\]
which is left adjoint to \(\iota_\tau\) and is exact.
\end{theorem}

\begin{corollary}[Idempotence, conservativity, $V$-nucleus]\label{A:cor:idemp-cons}
\(\mathbf{T}_\tau\circ \mathbf{T}_\tau \cong \mathbf{T}_\tau\) and \(\mathbf{T}_\tau|_{\Pers^{\mathrm{ft}}_{k,\tau\text{-loc}}}\cong \mathrm{Id}\).
Together with Proposition~\ref{A:prop:lipschitz}, \(\mathbf{T}_\tau\) is a $V$-nucleus (Definition~\ref{A:def:Vnucleus}) as recorded in Proposition~\ref{A:prop:Vnucleus}.
\end{corollary}

\begin{proposition}[Behavior on barcodes]\label{A:prop:barcode-behavior}
Let \(M\simeq \bigoplus_{j} I[a_j,b_j)\). Then
\[
\mathbf{T}_\tau M\ \simeq\ \bigoplus_{\,b_j-a_j>\tau}\ I[a_j,b_j) \,,\qquad
t_\tau(M)\ \simeq\ \bigoplus_{\,b_j-a_j\le \tau}\ I[a_j,b_j)\,.
\]
\end{proposition}

\begin{remark}[Filtered colimits: functor-category computation and return to constructible]
\label{A:rk:filtered-colimits}
Filtered colimits are computed objectwise in \([\mathbb{R},\Vect_k]\), and \(\mathbf{T}_\tau\) commutes with those colimits there (as a left adjoint).
A filtered colimit of constructible modules may exit \(\Pers^{\mathrm{ft}}_k\).
In applications we either: (i) restrict to towers that remain constructible degreewise; or (ii) compute in \([\mathbb{R},\Vect_k]\), apply \(\mathbf{T}_\tau\), and \emph{verify} return to \(\Pers^{\mathrm{ft}}_k\) on each declared window (coverage + finiteness checks logged in \texttt{run.yaml}; Appendix~G).
\end{remark}

% -------------------------
\subsection*{A.5. Shift-commutation, monotonicity in \texorpdfstring{$\tau$}{tau}, and 1-Lipschitz continuity}
% -------------------------
For \(\varepsilon\ge 0\), let \(S^\varepsilon\) be the shift \((S^\varepsilon M)(t):=M(t+\varepsilon)\).

\begin{lemma}[Shift commutation]\label{A:lem:shift}
For all \(\varepsilon\ge 0\), \(\mathbf{T}_\tau\circ S^\varepsilon \;\cong\; S^\varepsilon\circ \mathbf{T}_\tau\).
\end{lemma}

\begin{lemma}[Monotonicity in \(\tau\)]\label{A:lem:monotone-tau}
If \(0<\tau\le \tau'\), there is a natural epimorphism \(\mathbf{T}_{\tau'}M\to \mathbf{T}_{\tau}M\), functorial in \(M\).
\end{lemma}

\begin{proposition}[Non-expansiveness (interleaving/bottleneck)]\label{A:prop:lipschitz}
\(\mathbf{T}_\tau\) is \(1\)-Lipschitz for the interleaving (equivalently, bottleneck) distance on \(\Pers^{\mathrm{ft}}_k\).
Equivalently, \(\mathbf{T}_\tau\) is \(V\)-\(1\)-Lipschitz for the induced Lawvere \(V\)-metric \(d_V\) in the canonical \(V=[0,\infty]\) regime.
\end{proposition}

% -------------------------
\subsection*{A.6. Windowing (MECE), coverage checks, and \texorpdfstring{$\tau$}{tau}-adaptation}
% -------------------------
\begin{definition}[MECE domain windowing and coverage]\label{A:def:MECE}
A \emph{domain windowing} is a finite or countable collection of half-open, right-open intervals
\(\{[u_k,u_{k+1})\}_{k\in K}\) such that:
\begin{itemize}
  \item \emph{(Disjointness)} \([u_k,u_{k+1})\cap [u_\ell,u_{\ell+1})=\varnothing\) for \(k\neq \ell\).
  \item \emph{(Contiguity)} \(u_{k+1}=u_k+\len_k\) with \(\len_k>0\).
  \item \emph{(Coverage)} \(\bigsqcup_{k\in K}[u_k,u_{k+1})=[u_0,U)\) for some finite \(U>u_0\).
\end{itemize}
Coverage checks require
\[
\sum_{k\in K}(u_{k+1}-u_k)=U-u_0,
\qquad
\#\mathrm{Events}([u_0,U))=\sum_{k\in K}\#\mathrm{Events}([u_k,u_{k+1}))\ \ (\pm\ \text{recorded discretization}).
\]
Any rounding/discretization discrepancy is recorded as \(\delta_{\disc}\) (and, if measurement-induced, as \(\delta_{\meas}\)) in the run manifest (Appendix~G).
\end{definition}

\begin{remark}[Alignment policy (after-collapse, same window)]
When persistence and spectral measurements are combined, domain windows \(\{[u_k,u_{k+1})\}\), collapse thresholds \(\tau\), and spectral bins must be \emph{fixed per window} and logged.
All B-side measurements are taken \emph{after} applying \(\mathbf{T}_\tau\) and on the same window; pre-collapse comparisons are out of scope.
When definability is required (finite event sets, finite Čech depth), we assume Denef--Pas definability (Appendix~Q), or the real o-minimal surrogate used in Appendices~H/J.
\end{remark}

\begin{definition}[\(\tau\)-adaptation, sweep, and stability bands]\label{A:def:tau-adapt}
A threshold \(\tau\) is \emph{resolution-adapted} if \(\tau=\alpha\cdot \max\{\Delta t,\Delta x\}\) for a fixed \(\alpha>0\).
A \emph{\(\tau\)-sweep} is a discrete set \(\{\tau_\ell\}\) on which diagnostics are evaluated.
A \emph{stability band} is a contiguous range \(B\subset (0,\infty)\) such that the chosen natural transformations are isomorphisms for all \(\tau\in B\) (hence the relevant tower diagnostics vanish on that band).
\end{definition}

% -------------------------
\subsection*{A.7. Clipping, strict commutation, and the $\delta$-commutation contract}
% -------------------------
Let \(\clip_{[u,v)}:\Pers^{\mathrm{ft}}_k\to\Pers^{\mathrm{ft}}_k\) denote clipping to \([u,v)\) (half-open, right-open).

\begin{lemma}[Strict commutation in the exact model]\label{A:lem:clip}
\(\mathbf{T}_\tau\) commutes with clipping: \(\mathbf{T}_\tau\circ \clip_{[u,v)} \cong \clip_{[u,v)}\circ \mathbf{T}_\tau\).
\end{lemma}

\begin{proof}
Clipping is exact and preserves interval lengths, hence preserves \(\mathsf{E}_\tau\). It therefore descends to the Serre quotient and commutes with \(\pi_\tau\); transporting across the equivalence in Theorem~\ref{A:thm:localization} gives the claim.
\end{proof}

\begin{proposition}[Operational $\delta$-commutation]\label{A:prop:delta-clip}
In implementations that incur discretization/rounding, there exists a nonnegative defect
\(\delta_{\mathrm{clip}}([u,v),\tau)\in V\) such that for all \(M\in\Pers^{\mathrm{ft}}_k\)
\[
d_V\!\left(\mathbf{T}_\tau\clip_{[u,v)}(M),\ \clip_{[u,v)}\mathbf{T}_\tau(M)\right)\ \le\ \delta_{\mathrm{clip}}([u,v),\tau),
\]
with \(\delta_{\mathrm{clip}}\) \emph{aggregated by} \(\oplus\) over concatenated windows and \emph{subadditive} along pipelines.
In the mathematical model, \(\delta_{\mathrm{clip}}=0\) by Lemma~\ref{A:lem:clip}.
\end{proposition}

% -------------------------
\subsection*{A.8. $V$-enriched metric: benignity and operational role}
% -------------------------
\begin{remark}[$V$-enrichment is benign; \(\mathbf{T}_\tau\) as $V$-nucleus]\label{rem:V-enriched}
Let \(V\) be a commutative quantale. The abelian/Serre-localization statements remain unchanged on the underlying category \(\Pers^{\mathrm{ft}}_k\).
The barcode semantics and length calculus (interval Jordan–Hölder and Serre quotient by \(\mathsf{E}_\tau\)) are invariant as \emph{statements in \(\Pers^{\mathrm{ft}}_k\)}.
The \(V\)-structure is used only to \emph{measure} distances, sums, and gluing budgets uniformly; it does not alter the algebraic backbone.
In particular, \(\mathbf{T}_\tau\) is a $V$-nucleus (Prop.~\ref{A:prop:Vnucleus}); operational non-commutations are summarized by \(\delta\)-contracts (Remark~\ref{A:rk:delta-schema}, Prop.~\ref{A:prop:delta-clip}) and logged as part of the proof object (\texttt{run.yaml}; Appendix~G).
\end{remark}

% -------------------------
\subsection*{A.9. Operational checklist and glued output}
% -------------------------
The reinforcement yields a single, self-contained persistence-layer backbone compatible with the v17.0 after-collapse contract:
\begin{itemize}
  \item Abelianity and pointwise exactness (Proposition~\ref{A:prop:abelian}).
  \item Hereditary Serre \(\tau\)-ephemeral class \(\mathsf{E}_\tau\) (Lemma~\ref{A:lem:Serre}); exact reflective localization (Theorem~\ref{A:thm:localization}).
  \item \(\mathbf{T}_\tau\) is an exact, idempotent, \(1\)-Lipschitz $V$-nucleus (Corollary~\ref{A:cor:idemp-cons}, Proposition~\ref{A:prop:lipschitz}, Proposition~\ref{A:prop:Vnucleus}).
  \item Strict clipping-commutation in the categorical model (Lemma~\ref{A:lem:clip}); operational $\delta$-commutation contract for audits (Proposition~\ref{A:prop:delta-clip}).
  \item Shift-commutation and monotonicity in \(\tau\) (Lemmas~\ref{A:lem:shift}, \ref{A:lem:monotone-tau}); barcode-level description (Proposition~\ref{A:prop:barcode-behavior}).
  \item MECE windowing and coverage checks (Definition~\ref{A:def:MECE}); \(\tau\)-adaptation and stability bands (Definition~\ref{A:def:tau-adapt}).
  \item Filtered-colimit policy with return-to-constructible verification on declared windows (Remark~\ref{A:rk:filtered-colimits}), logged in the manifest (Appendix~G).
\end{itemize}
\emph{Output:} the globally glued object obtained from the windowed pipeline (MECE partition, localization, and subsequent certified steps in the main text) is accepted only under the mandated after-collapse order and under declared coverage/adaptation policies; all non-commutations are explicitly accounted for in the \(\delta\)-ledger via the $V$-nucleus contracts above.

\medskip
\noindent\emph{References for Appendix A.}
Crawley–Boevey (2015): Decomposition of pointwise finite-dimensional persistence modules. IMRN. 
Chazal–de~Silva–Glisse–Oudot (2016): The Structure and Stability of Persistence Modules. 
Gabriel (1962): Des catégories abéliennes. 
Popescu: Abelian Categories. 
Stacks Project, Tag~02MO.
\endgroup



% =========================
\appendix
\section*{Appendix B. Lifting \texorpdfstring{$\mathbf{T}_\tau$}{T\_\tau} to \texorpdfstring{$C_\tau$}{C\_\tau} and the Homotopy Setting (reinforced v17.0)}
% =========================

\addcontentsline{toc}{section}{Appendix B. Lifting $T_\tau$ to $C_\tau$ and the Homotopy Setting (reinforced v17.0)}

\begingroup

Throughout, fix a field \(k\).
Let \(\mathsf{FiltCh}(k)\) denote the category of \emph{bounded-in-degree} filtered chain complexes of finite-dimensional \(k\)-vector spaces with filtration-preserving chain maps.
(“Bounded” refers to homological degree; filtrations are assumed \emph{locally finite on bounded windows} as in Appendix~A.)
For each homological degree \(i\), write
\[
\mathbf{P}_i:\ \mathsf{FiltCh}(k)\longrightarrow \Pers^{\mathrm{ft}}_k,\qquad
F\longmapsto \bigl(t\mapsto H_i(F^{t}C_\bullet)\bigr),
\]
the degreewise persistence functor into the constructible subcategory (Appendix~A).

\medskip
\noindent\textbf{Global scope and conventions (v17.0 alignment).}
(i) All claims at the filtered–complex layer hold \emph{up to filtered quasi-isomorphism (f.q.i.)}; all identities at the persistence layer hold \emph{strictly} in \(\Pers^{\mathrm{ft}}_k\).\\
(ii) Any operational quantitative comparison (distance, Lipschitz, monotonicity, budgets) is interpreted under the mandatory order
\[
\boxed{\ \mathbf{P}_i\ \Rightarrow\ \mathbf{T}_\tau\ \Rightarrow\ \text{compare in } \Pers^{\mathrm{ft}}_k\ }
\]
on fixed right-open windows (after-collapse policy; Appendix~A, Remark~\ref{A:rk:scope-after}).\\
(iii) Filtered (co)limits, when invoked, are computed objectwise in \([\mathbb{R},\Vect_k]\), and we then \emph{verify} that the result lies in (or returns to) \(\Pers^{\mathrm{ft}}_k\) on declared windows (Appendix~A, Remark~\ref{A:rk:filtered-colimits}); no claim is made outside this regime.\\
(iv) Deletion-type updates are non-increasing for windowed energies and spectral tails \emph{after truncation}, whereas inclusion-type updates are only stable (non-expansive) (Appendix~E).\\
(v) Endpoint conventions follow Appendix~A (Remark~\ref{A:rk:endpoints}); in particular, infinite bars are not removed by \(\mathbf{T}_\tau\) and their contributions are clipped by windowing.\\
(vi) For notational economy we sometimes write \(\Ttau=\mathbf{T}_\tau\).\\
(vii) \emph{Amplitude guard-rail.} Realizations used operationally are taken with \emph{amplitude \(\le 1\) after collapse}: the fixed \(t\)-exact realization \(\mathcal{R}\) is required to have cohomological amplitude contained in \([0,1]\) on the after-collapse objects \(\mathcal{R}(C_\tau F)\), with this requirement enforced up to f.q.i.\ (\S\ref{B:thm:Ctau}).\\
(viii) \emph{\(\delta\)-ledger interface.} Any implementation-level non-commutation/defect is charged into a commutative quantale \((V,\oplus,\le,0)\) as
\(\delta=\delta_{\mathrm{alg}}\oplus\delta_{\mathrm{disc}}\oplus\delta_{\mathrm{meas}}\in V\)
(Appendices~A/L/S). Additivity along pipelines is always understood with respect to \(\oplus\).

% -------------------------
\subsection*{B.1. The interval-realization assignment \texorpdfstring{$\mathcal{U}$}{U} (up to f.q.i.)}
% -------------------------
\begin{definition}[Elementary interval blocks (two-term/one-term model)]
Let \(I[a,b)\) be an interval module (fixed endpoint convention; Appendix~A, Remark~\ref{A:rk:endpoints}).
\begin{itemize}
  \item If \(b<+\infty\), realize \(I[a,b)\) in homological degree \(i\) by a \emph{two-term filtered block}
  \[
    k\cdot y \xrightarrow{\,d\,} k\cdot x,\qquad |y|=i+1,\ |x|=i,\quad
       \mathrm{fil}(x)=a,\ \mathrm{fil}(y)=b,\ d(y)=x,\ d(x)=0.
  \]
  Then \(x\) contributes a bar born at \(a\) and killed at \(b\).
  \item If \(b=+\infty\), realize \(I[a,\infty)\) by a \emph{one-term} block \(k\cdot x\) in degree \(i\) with \(\mathrm{fil}(x)=a\) and \(d=0\).
\end{itemize}
In all blocks, the differential preserves the filtration: \(d(F^t)\subseteq F^t\) for every \(t\).
Taking \emph{locally finite on bounded windows} direct sums of such blocks and applying degree shifts produces a filtered complex whose persistence recovers the prescribed bars.
We call any such model an \emph{elementary interval complex} and denote a representative by \(\mathcal{I}[a,b)\).
\end{definition}

\begin{proposition}[Barcode realization for bounded families (up to f.q.i.)]\label{B:prop:U}
There exists an assignment
\[
\mathcal{U}:\ \Pers^{\mathrm{ft}}_k\longrightarrow \mathsf{FiltCh}(k)
\]
such that for any \emph{degree-bounded} family \(\{M_i\}_{i\in\mathbb{Z}}\) of constructible persistence modules
(only finitely many \(i\) nonzero) there are natural isomorphisms in \(\Pers^{\mathrm{ft}}_k\),
\[
\mathbf{P}_i\!\Big(\,\bigoplus_{j}\mathcal{U}(M_j)[-j]\Big)\ \cong\ M_i\qquad(\forall i).
\]
The construction is canonical \emph{up to} filtered quasi-isomorphism, additive, and functorial in the homotopy category \(\Ho(\mathsf{FiltCh}(k))\).
In particular, for a single module \(M\) realized in a base degree (say \(0\)) one has \(\mathbf{P}_0(\mathcal{U}(M))\cong M\) and \(\mathbf{P}_j(\mathcal{U}(M))=0\) for all \(j\neq 0\).
\end{proposition}

\begin{remark}[Pseudofunctoriality of \(\mathcal{U}\)]
The assignment \(\mathcal{U}\) extends to a \emph{pseudofunctor}
\(\mathcal{U}:\Pers^{\mathrm{ft}}_k\to \Ho(\mathsf{FiltCh}(k))\): on a morphism of persistence modules, choose interval decompositions and a bar-matching; the induced blockwise filtered chain map is well-defined in \(\Ho\) \emph{up to} f.q.i., and compositions are respected up to coherent isomorphism.
Consequently, constructions below that use \(\mathcal{U}\) on morphisms (e.g.\ \(C_\tau\)) are functorial on \(\Ho(\mathsf{FiltCh}(k))\).
\end{remark}

% -------------------------
\subsection*{B.2. Filtered quasi-isomorphisms and \texorpdfstring{$\Ho(\mathsf{FiltCh}(k))$}{Ho(FiltCh(k))}}
% -------------------------
\begin{definition}[Filtered quasi-isomorphism]
A filtration-preserving chain map \(f:F\to G\) is a \emph{filtered quasi-isomorphism (f.q.i.)} if for every \(t\in\mathbb{R}\) the map \(F^{t}C_\bullet\to G^{t}C_\bullet\) is a quasi-isomorphism.
Equivalently, for all \(i\), \(\mathbf{P}_i(f)\) is an isomorphism in \(\Pers^{\mathrm{ft}}_k\).
\end{definition}

\begin{lemma}[Characterization of f.q.i.]
For bounded-in-degree filtered complexes of finite-dimensional vector spaces, \(f:F\to G\) is an f.q.i. iff \(\mathbf{P}_i(f)\) is an isomorphism in \(\Pers^{\mathrm{ft}}_k\) for all \(i\).
\end{lemma}

\begin{definition}[Homotopy category]
Let \(\Ho(\mathsf{FiltCh}(k))\) be the localization of \(\mathsf{FiltCh}(k)\) at f.q.i.’s.
Identities stated in \(\Ho(\mathsf{FiltCh}(k))\) are to be understood \emph{up to f.q.i.} at the model level.
All endofunctors considered below (e.g.\ \(C_\tau\) and Mirror/Transfer templates) preserve f.q.i.’s; thus they descend to \(\Ho(\mathsf{FiltCh}(k))\).
\end{definition}

% -------------------------
\subsection*{B.2.1. Amplitude \texorpdfstring{$\le 1$}{≤1}: modeling caution and \texorpdfstring{\textup{[Spec]}}{[Spec]}}
% -------------------------
\begin{remark}[Modeling caution: amplitude \(\le 1\) after collapse]\label{B:rk:ampl}
The operational guard-rail requires that, after applying \(C_\tau\), the realization \(\mathcal{R}(C_\tau F)\) has cohomology concentrated in degrees \([0,1]\).
The \emph{block-diagonal assembly} (no off-diagonal couplings across distinct bars or non-adjacent degrees) ensures this and is enforced throughout.
\end{remark}

\begin{declaration}[\textup{[Spec]} amplitude \(\le 1\) witness]\label{B:lem:ampl}
Let \(F\in\mathsf{FiltCh}(k)\) and construct \(C_\tau(F)\) by replacing each finite bar by a two-term block \((i{+}1)\!\to\! i\) and each infinite bar by a one-term block in degree \(i\), with zero differentials between distinct blocks and between non-adjacent degrees.
Then the spectral sequence of the stupid filtration on \(\mathcal{R}(C_\tau F)\) degenerates at \(E_1\), and \(H^q(\mathcal{R}(C_\tau F))=0\) for \(q\notin\{0,1\}\).
\end{declaration}

\begin{proof}[Proof sketch]
Each bar contributes either a length-one complex or a single object; hence every differential raises degree by at most one.
No off-diagonal maps exist by construction.
Therefore the only potentially nonzero \(d_r\) occur with \(r\le 1\), so the spectral sequence collapses at \(E_1\), forcing cohomology to lie in degrees \(0,1\).
\end{proof}

% -------------------------
\subsection*{B.2.2. f.q.i. checklist (\texorpdfstring{T-Exactness-Persistence}{T-Exactness-Persistence})}
% -------------------------
\begin{definition}[Test \texttt{T-Exactness-Persistence}]\label{B:def:T-Exactness}
Given \(F\in\mathsf{FiltCh}(k)\), the test consists of the following verifications:
\begin{enumerate}\itemsep0.1em
  \item[(E0)] \emph{Degree bound}: \(F\) is bounded in homological degree.
  \item[(E1)] \emph{Local finiteness}: the filtration is locally finite on bounded windows.
  \item[(E2)] \emph{Barcode audit}: for each \(i\), \(\mathbf{P}_i(F)\) is constructible (Appendix~A).
  \item[(E3)] \emph{Realization amplitude}: after \(C_\tau\), \(\mathcal{R}(C_\tau F)\) has amplitude \(\le 1\) (Declaration~\ref{B:lem:ampl}).
  \item[(E4)] \emph{Exactness match}: for every short exact sequence of filtered complexes \(0\to F'\to F\to F''\to 0\), the induced sequence of persistence modules is exact in \(\Pers^{\mathrm{ft}}_k\).
  \item[(E5)] \emph{Functoriality under f.q.i.}: if \(f:F\to G\) is an f.q.i., then \(\mathbf{P}_i(f)\) is iso for all \(i\).
  \item[(E6)] \emph{Shift/clip compatibility}: \(\mathbf{P}_i\) commutes with shifts and clipping; \(\Ttau\) commutes with clipping (Appendix~A, Lemma~\ref{A:lem:clip}).
  \item[(E7)] \emph{Non-expansiveness (after-collapse)}:
  \[
  d_{\mathrm{int}}\!\bigl(\Ttau\,\mathbf{P}_i(F),\ \Ttau\,\mathbf{P}_i(G)\bigr)\ \le\ d_{\mathrm{int}}\!\bigl(\mathbf{P}_i(F),\ \mathbf{P}_i(G)\bigr).
  \]
\end{enumerate}
A dataset \((F,\tau)\) \emph{passes} \texttt{T-Exactness-Persistence} if (E0)–(E7) hold.
\end{definition}

\begin{proposition}[Effect of \texttt{T-Exactness-Persistence}]\label{B:prop:T-Exactness}
If \(F\) passes \texttt{T-Exactness-Persistence}, then for all \(i\)
\[
\mathbf{P}_i\!\big(C_\tau(F)\big)\ \cong\ \Ttau\,\mathbf{P}_i(F),
\]
functorially in \(\Ho(\mathsf{FiltCh}(k))\).
Moreover, \(C_\tau\) preserves f.q.i. and is idempotent up to f.q.i.
\end{proposition}

\begin{proof}[Proof sketch]
(E2) and (E4) place persistence computations in the abelian constructible regime where \(\Ttau\) is exact (Appendix~A, Thm.~\ref{A:thm:localization}); (E6) gives compatibility with clipping/shifts; (E7) is Appendix~A, Prop.~\ref{A:prop:lipschitz} applied under the mandated after-collapse order.
The construction of \(C_\tau\) (Theorem~\ref{B:thm:Ctau}) together with (E3) ensures the amplitude guard-rail; f.q.i.\ detection by persistence (E5) yields the functorial identification in \(\Ho\).
Idempotence follows from \(\Ttau\circ \Ttau=\Ttau\) and the block-diagonal lift.
\end{proof}

% -------------------------
\subsection*{B.3. Lifting \texorpdfstring{$\mathbf{T}_\tau$}{T\_\tau} to \texorpdfstring{$C_\tau$}{C\_\tau} and (co)limit/pullback compatibilities}
% -------------------------

\paragraph*{Existence, functoriality, and uniqueness (homotopy-functor level): block-diagonal assembly.}
\begin{theorem}[\textup{[Declaration]} Thresholded collapse in \(\Ho\)]\label{B:thm:Ctau}
For each \(\tau\ge 0\) there exists an endofunctor
\[
C_\tau:\ \Ho(\mathsf{FiltCh}(k))\longrightarrow \Ho(\mathsf{FiltCh}(k))
\]
and natural isomorphisms in \(\Pers^{\mathrm{ft}}_k\)
\[
\mathbf{P}_i\!\big(C_\tau(F)\big)\ \xrightarrow{\ \cong\ }\ \mathbf{T}_\tau\!\big(\mathbf{P}_i(F)\big)\qquad(\forall i,F),
\]
such that:
\begin{enumerate}\itemsep0.2em
  \item (\emph{Idempotence/monotonicity in \(\Ho\)})
  \(C_\tau\circ C_\sigma \simeq C_{\max\{\tau,\sigma\}}\simeq C_\sigma\circ C_\tau\).
  \item (\emph{Non-expansiveness at persistence; after-collapse})
  \(d_{\mathrm{int}}\!\bigl(\mathbf{P}_i(C_\tau F),\,\mathbf{P}_i(C_\tau G)\bigr)\ \le\ d_{\mathrm{int}}\!\bigl(\mathbf{P}_i(F),\,\mathbf{P}_i(G)\bigr)\).
\end{enumerate}
Any two such lifts are uniquely isomorphic in \(\Ho(\mathsf{FiltCh}(k))\).
For \(\tau=0\), \(C_0\simeq \mathrm{id}\) in \(\Ho(\mathsf{FiltCh}(k))\).
\end{theorem}

\begin{proof}[Construction/Proof sketch]
Replace \(\mathbf{P}_i(F)\) with \(\Ttau(\mathbf{P}_i(F))\) (Appendix~A) and realize via \(\mathcal{U}\) (Prop.~\ref{B:prop:U}); assemble differentials \emph{block-diagonally} as in Remark~\ref{B:rk:ampl}.
Functoriality and uniqueness follow from pseudofunctoriality of \(\mathcal{U}\) and the universal property of \(\Ttau\);
non-expansiveness reflects Appendix~A, Prop.~\ref{A:prop:lipschitz}.
\end{proof}

\paragraph*{(Co)limits and pullbacks: persistence layer is strict; filtered layer up to f.q.i.}
\begin{proposition}[Compatibility at the persistence layer]\label{B:prop:limits}
Assume filtered colimits in \(\mathsf{FiltCh}(k)\) are computed degreewise and the results return to \(\Pers^{\mathrm{ft}}_k\) on declared windows.
Then for every filtered diagram \(\{F_\lambda\}\) and every \(i\),
\[
\mathbf{P}_i\!\big(C_\tau(\varinjlim\nolimits_\lambda F_\lambda)\big)\ \cong\ \varinjlim\nolimits_\lambda\, \mathbf{P}_i\!\big(C_\tau(F_\lambda)\big)\quad\text{in } \Pers^{\mathrm{ft}}_k.
\]
If, in addition, \textup{[Spec]} finite pullbacks in \(\mathsf{FiltCh}(k)\) are computed degreewise and \(\mathcal{U}\) preserves finite limits up to f.q.i.\ under \emph{lifting–coherence} \((\LC)\), then for any pullback square \(F\times_H G\),
\[
\mathbf{P}_i\!\big(C_\tau(F\times_H G)\big)\ \cong\ \mathbf{P}_i\!\big(C_\tau(F)\times_{C_\tau(H)} C_\tau(G)\big)\quad\text{in } \Pers^{\mathrm{ft}}_k.
\]
At the filtered level, compatibilities hold \emph{up to f.q.i.}
\end{proposition}

\begin{remark}[On \((\LC)\)]
\((\LC)\) is a \emph{finite-diagram} coherence ensuring that interval realizations can be chosen compatibly (up to f.q.i.) with pullback/pushout shapes encountered in practice.
It holds for the block model and finite matching diagrams induced by monotone filtrations; we use it only in \textup{[Spec]} statements.
\end{remark}

\begin{remark}[Realization functor; comparison maps \textup{[Spec]}]
Let \(\mathcal{R}:\mathsf{FiltCh}(k)\to D^{\mathrm{b}}(k\text{-mod})\) be the fixed \(t\)-exact realization (amplitude guard-rail as above).
Within the implementable range there are natural comparison morphisms
\[
\mathcal{R}\circ C_\tau\ \Longrightarrow\ \tau_{\ge 0}\circ \mathcal{R},
\]
compatible with \(\mathbf{P}_i\) after homology (Appendix~C). These maps are treated up to f.q.i.\ in \(\Ho(\mathsf{FiltCh}(k))\).
\end{remark}

% -------------------------
\subsection*{B.4. Non-expansive Mirror/Transfer templates \texorpdfstring{[Spec]}{[Spec]}}
% -------------------------
\begin{definition}[Admissible Mirror/Transfer endofunctors]\label{B:def:mirror}
An endofunctor \(\Mirror:\mathsf{FiltCh}(k)\to \mathsf{FiltCh}(k)\) is \emph{admissible} if:
\begin{enumerate}\itemsep0.2em
  \item (\emph{Persistence non-expansiveness})
  \(d_{\mathrm{int}}\!\big(\mathbf{P}_i(\Mirror F),\,\mathbf{P}_i(\Mirror G)\big)\ \le\ d_{\mathrm{int}}\!\big(\mathbf{P}_i(F),\,\mathbf{P}_i(G)\big)\) for all \(F,G,i\).
  \item (\emph{Constructible stability})
  \(\Mirror\) carries finite-type objects to finite-type objects degreewise (on declared windows).
  \item (\emph{f.q.i.-invariance})
  If \(f\) is an f.q.i., then \(\Mirror(f)\) is an f.q.i.; hence \(\Mirror\) descends to \(\Ho(\mathsf{FiltCh}(k))\).
  \item (\emph{Conditional commutation with \(C_\tau\)})
  There exists a natural 2-cell
  \(\theta:\ \Mirror\circ C_\tau\Rightarrow C_\tau\circ \Mirror\)
  whose effect at persistence is \(\delta\)-controlled:
  \[
  d_{\mathrm{int}}\!\Big(\Ttau\,\mathbf{P}_i(\Mirror(C_\tau F)),\ \Ttau\,\mathbf{P}_i(C_\tau(\Mirror F))\Big)\ \le\ \delta(i,\tau),
  \]
  with \(\delta(i,\tau)\) uniform in \(F\) and aggregated along pipelines via \(\oplus\).
\end{enumerate}
\end{definition}

\begin{theorem}[Quantitative commutation in \(\Ho\)]\label{B:thm:quant}
Assume \(\Mirror\) is admissible and \(\theta\) exists with bound \(\delta(i,\tau)\).
Then for all \(F,i,\tau\),
\[
  d_{\mathrm{int}}\!\Big(\Ttau\,\mathbf{P}_i(\Mirror(C_\tau F)),\ \Ttau\,\mathbf{P}_i(C_\tau(\Mirror F))\Big)\ \le\ \delta(i,\tau).
\]
For a pipeline \(\Mirror_m,\dots,\Mirror_1\) the bounds aggregate (in the quantale) as
\[
  d_{\mathrm{int}}(\cdots)\ \le\ \delta_m(i,\tau_m)\ \oplus\ \cdots\ \oplus\ \delta_1(i,\tau_1).
\]
Any subsequent \(1\)-Lipschitz persistence post-processing does not increase the bound.
\end{theorem}

% -------------------------
\subsection*{B.5. Commutable torsion reflectors and A/B policy (homotopy interface)}
% -------------------------
Let \(T_A,T_B:\Pers^{\mathrm{ft}}_k\to\Pers^{\mathrm{ft}}_k\) be exact reflectors with Serre classes \(E_A,E_B\).

\begin{proposition}[Nested torsions \(\Rightarrow\) order independence]\label{B:prop:nested}
If \(E_A\subseteq E_B\) or \(E_B\subseteq E_A\), then
\(T_A\circ T_B= T_B\circ T_A= T_{A\vee B}\).
In particular, for 1D length thresholds, \(\mathbf{T}_\tau\circ \mathbf{T}_\sigma=\mathbf{T}_{\max\{\tau,\sigma\}}\).
\end{proposition}

\begin{definition}[A/B soft-commuting policy]\label{B:def:ab}
For arbitrary reflectors \(T_A,T_B\) and \(M\in\Pers^{\mathrm{ft}}_k\), set
\(\Delta_{\mathrm{comm}}(M;A,B):=d_{\mathrm{int}}(T_A T_B M,\ T_B T_A M)\).
Given a tolerance \(\eta\ge 0\), accept \emph{soft-commuting} if \(\Delta_{\mathrm{comm}}\le \eta\); otherwise fix an order and charge \(\Delta_{\mathrm{comm}}\) to \(\delta^{\mathrm{alg}}\) (Appendix~L) with pipeline aggregation via \(\oplus\).
\end{definition}

% -------------------------
\subsection*{B.6. AWFS on \texorpdfstring{$\Ho(\mathsf{FiltCh})$}{Ho(FiltCh)} and 2-cell accounting}
% -------------------------
\begin{declaration}[AWFS on $\Ho(\mathsf{FiltCh})$]\label{B:dec:awfs}
We adopt an algebraic weak factorization system on $\Ho(\mathsf{FiltCh})$ with $L$ (preprocess/left map) and $R$ (collapse/right map) such that $R \simeq C_\tau$ up to f.q.i.
Triangle/zigzag identities hold up to f.q.i.; all 2-cell deviations are recorded as $\delta_{\mathrm{alg}}$ (Appendix~L).
\end{declaration}

\begin{theorem}[AWFS triangle 2-cells]\label{B:thm:awfs-triangle}
There exist coherent 2-cells (quantitatively bounded after collapse)
\[
C_\tau\!\circ C_\tau \simeq C_\tau,\qquad
\mathbf{T}_\tau\!\circ \mathbf{T}_\tau = \mathbf{T}_\tau,\qquad
L\!\circ R \simeq R\!\circ L,
\]
whose implementation non-commutation is absorbed into the $\delta_{\mathrm{alg}}$-budget (Appendix~L).
\end{theorem}

\begin{corollary}[A/B policy as 2-cell accounting]\label{B:cor:ab-awfs}
For two exact reflectors on persistence, any measured defect $\Delta_{\mathrm{comm}}$ (Definition~\ref{B:def:ab}) is realizable as a 2-cell deviation within the AWFS picture and must be logged as $\delta_{\mathrm{alg}}$; pipeline aggregation follows from the quantale-sum rule in Appendix~L.
\end{corollary}

\begin{remark}[V-shifts and $C_\tau$ \emph{in situ}]
If $S^v$ denotes a Lawvere $V$-shift compatible with degreewise filtrations, then $C_\tau\circ S^v\simeq S^v\circ C_\tau$ in $\Ho$ and, after applying $\mathbf{P}_i$ and then \(\Ttau\), the induced comparison is \(V\)-\(1\)-Lipschitz (Appendix~A, Lemma~\ref{A:lem:shift}).
\end{remark}

% -------------------------
\subsection*{B.7. Completion note and implementation recipe}
% -------------------------
\begin{remark}[No further supplementation required]
This appendix integrates: (i) the lift \(C_\tau\) of the exact reflector \(\Ttau\) to the homotopy setting (existence, functoriality, uniqueness up to f.q.i.; non-expansiveness at persistence); (ii) strict persistence-layer compatibilities with (co)limits and pullbacks (filtered level up to f.q.i.); (iii) admissible Mirror/Transfer templates with a uniform, additive 2-cell bound \(\delta(i,\tau)\) stable under \(1\)-Lipschitz post-processing; (iv) a commutable-torsion policy (nested \(\Rightarrow\) order-independent; else A/B with ledger); and \emph{(v) the amplitude \(\le 1\) guard-rail and the f.q.i. checklist \texttt{T-Exactness-Persistence}}.
No additional supplementation is required for operational use under the global after-collapse scope of Appendix~A.
\end{remark}

\paragraph{Implementation recipe (engineering checklist).}
\begin{itemize}
  \item Build \(C_\tau\) by block-diagonal assembly of interval blocks; forbid off-diagonal couplings across blocks/degrees (enforces amplitude \(\le 1\)).
  \item For each morphism, lift \(\Ttau\mathbf{P}_i(f)\) blockwise via \(\mathcal{U}\) and take direct sums across \(i\); functorial in \(\Ho\).
  \item Enforce \texttt{T-Exactness-Persistence} (Def.~\ref{B:def:T-Exactness}); expose pass/fail and witnesses in the manifest/log layer.
  \item For Mirror/Transfer, provide a 2-cell \(\theta\) with a \emph{uniform} \(\delta(i,\tau)\); aggregate along pipelines via \(\oplus\); ensure post-processors are \(1\)-Lipschitz at persistence (after-collapse).
  \item For reflectors, run A/B tests; if \(\Delta_{\mathrm{comm}}>\eta\), fix an order and charge the surplus to \(\delta^{\mathrm{alg}}\) (Appendix~L).
  \item Keep clipping/windowing separate from localization; use Appendix~A, Lemma~\ref{A:lem:clip} as needed.
\end{itemize}

\medskip
\noindent\emph{References for Appendix B.}
Crawley–Boevey (2015): Decomposition of pointwise finite-dimensional persistence modules. IMRN. 
Chazal–de~Silva–Glisse–Oudot (2016): The Structure and Stability of Persistence Modules. 
Standard sources on AWFS and homotopical algebra (e.g.\ Riehl, \emph{Categorical Homotopy Theory}) are used at the level of \emph{up to f.q.i.} coherence only.

\endgroup



% =========================
\appendix
\section*{Appendix C. The Bridge \texorpdfstring{$\mathrm{PH}_1 \Rightarrow \Ext^1$}{PH1⇒Ext1} and its Local Reverse under \texorpdfstring{$E_1{=}0$}{E1=0} (reinforced, complete v17.0)}
% =========================

\addcontentsline{toc}{section}{Appendix C. The Bridge PH$_1\Rightarrow$Ext$^1$ and its Local Reverse under $E_1{=}0$ (reinforced, complete)}

\begingroup

Throughout, fix a field \(k\).
Let \(\mathsf{FiltCh}(k)\) be the category of \emph{bounded-in-degree} filtered chain complexes of finite-dimensional \(k\)-vector spaces with filtration-preserving maps.
For \(F\in\mathsf{FiltCh}(k)\) and each degree \(i\), the degreewise persistence functor
\[
\mathbf{P}_i(F):\ \mathbb{R}\longrightarrow \Vect_k,\qquad t\longmapsto H_i(F^{t}C_\bullet)
\]
is assumed \emph{constructible} (pointwise finite-dimensional, finitely many critical parameters on bounded windows), i.e.\ \(\mathbf{P}_i(F)\in \Pers^{\mathrm{ft}}_k\) (Appendix~A).
We also fix a realization functor
\[
\mathcal{R}:\ \mathsf{FiltCh}(k)\longrightarrow D^{\mathrm{b}}(k\text{-mod})
\]
into the bounded derived category of finite-dimensional \(k\)-vector spaces.

\medskip
\noindent\textbf{Bridge hypotheses, scope, and gate policy (v17.0 canon).}
We work under the standing assumptions \textup{(B1)–(B3)} used in the main text:
\begin{itemize}
  \item \textup{(B1)} field coefficients and constructibility of all \(\mathbf{P}_i(F)\) on declared windows;
  \item \textup{(B2)} \emph{two-term (amplitude \(\le 1\)) guard-rail after collapse:}
  on any window where the \(\Ext^1\)-test is \emph{eligible}, the after-collapse object satisfies
  \[
    \mathcal{R}(C_\tau F)\ \in\ D^{[-1,0]}(k\text{-mod})
  \]
  (equivalently: homological amplitude \(\le 1\) under the chain/derived sign convention);
  \item \textup{(B3)} functoriality/naturality of all constructions (including clipping, collapse, and realization) up to f.q.i.\ where applicable.
\end{itemize}
All statements in this appendix are confined to the implementable range of Appendix~A and the homotopy interface of Appendix~B.

\smallskip
\noindent\textbf{Eligibility for B-Gate\(^{+}\) (fixed).}
The \(\Ext^1\)-test is included in B-Gate\(^{+}\) \emph{only} on windows/scales where
\(\mathcal{R}(C_\tau F)\in D^{[-1,0]}(k\text{-mod})\) and the test object is \(k[0]\).
Outside this regime, \(\Ext^1\) may be logged but is \emph{not} used for gating.

\smallskip
\noindent\textbf{Operational order (collapse-first; fixed).}
All gate-relevant categorical tests follow the mandated order
\[
\boxed{\ F\ \xrightarrow{\ C_\tau\ }\ C_\tau F\ \xrightarrow{\ \clip_W\ }\ (C_\tau F)|_W\ \xrightarrow{\ \mathcal{R}\ }\ \mathcal{R}\big((C_\tau F)|_W\big)\ \xrightarrow{\ \Ext^1(-,k[0])\ }\ 0\ }.
\]
This is compatible with the persistence-layer canon \(\mathbf{P}_i\Rightarrow\mathbf{T}_\tau\Rightarrow\) compare, via Appendix~B:
\(\mathbf{P}_i(C_\tau F)\cong \mathbf{T}_\tau(\mathbf{P}_i(F))\).

\smallskip
\noindent\textbf{Meaning of \(\mathrm{PH}_1=0\).}
For a filtered complex \(F\), we write
\[
\mathrm{PH}_1(F)=0 \quad:\Longleftrightarrow\quad \mathbf{P}_1(F)=0 \text{ in } \Pers^{\mathrm{ft}}_k
\quad\Longleftrightarrow\quad H_1(F^t)=0\ \forall t\in\mathbb{R}.
\]
All operational uses are on after-collapse/windowed objects, i.e.\ \(\mathrm{PH}_1((C_\tau F)|_W)=0\).

\smallskip
\noindent\textbf{Windows, clipping, and commutation.}
A \emph{right-open window} is a half-open interval \(W=[u,v)\) (right endpoint open), consistent with the global MECE/window convention (Appendix~A, §A.6).
We write \(\clip_W\) for clipping/restriction to \(W\) (Appendix~A, §A.7).
In the exact model, \(\mathbf{T}_\tau\) commutes strictly with clipping (Appendix~A, Lemma~\ref{A:lem:clip}); any implementation defect is charged as \(\delta_{\mathrm{clip}}\) (Appendix~A, Proposition~\ref{A:prop:delta-clip}).

% -------------------------
\subsection*{C.1. Two-term amplitude and a canonical two-term model}
% -------------------------

\begin{proposition}[Two-term amplitude normal form]\label{C:prop:two-term}
If \(A\in D^{[-1,0]}(k\text{-mod})\), then there is a natural isomorphism in \(D^{\mathrm{b}}(k\text{-mod})\)
\[
A\ \simeq\ \Big[\, H^{-1}(A)\ \xrightarrow{\,d_A\,}\ H^{0}(A)\,\Big],
\]
where the complex on the right is concentrated in cohomological degrees \(-1\) and \(0\).
\end{proposition}

\begin{remark}
All statements below are invariant under filtered quasi-isomorphism on \(F\) and under isomorphism in \(D^{\mathrm{b}}(k\text{-mod})\) on \(\mathcal{R}(F)\).
\end{remark}

% -------------------------
\subsection*{C.2. The edge: \texorpdfstring{$H^{-1}(\mathcal{R}(F))\cong \varinjlim_t H_1(F^t)$}{H^{-1}(R(F)) ≅ colim H_1(F^t)}}
% -------------------------

\begin{assumption}[Edge-compatibility of the realization]\label{C:as:edge-compat}
On any eligible window (i.e.\ where \(\mathcal{R}(C_\tau F)\in D^{[-1,0]}\)), the realization \(\mathcal{R}\) is assumed to be compatible with filtered colimits along the filtration parameter in the following sense:
\[
H^{-1}\!\big(\mathcal{R}(F)\big)\ \cong\ \varinjlim_{t\in\mathbb{R}} H_1(F^{t}C_\bullet),
\]
naturally in \(F\) (and similarly after clipping to a fixed right-open window \(W\) by restricting the directed system to \(t\in W\)).
This is the only place where a compatibility between \(\mathcal{R}\) and the filtration-directed system is used.
\end{assumption}

\begin{proposition}[Edge identification and naturality]\label{C:prop:edge}
Assume \textup{(B1)–(B3)} and Assumption~\ref{C:as:edge-compat}.
Then for every eligible \(F\in \mathsf{FiltCh}(k)\) there is a natural isomorphism
\[
H^{-1}\!\big(\mathcal{R}(F)\big)\ \cong\ \varinjlim_{t\in\mathbb{R}}\, H_1(F^{t}C_\bullet).
\]
If \(f:F\to G\) preserves filtrations, the induced square with the maps on colimits commutes.
The same statement holds after applying \(C_\tau\) and/or clipping to a right-open window \(W\).
\end{proposition}

% -------------------------
\subsection*{C.3. Computing \texorpdfstring{$\Ext^1$}{Ext1} in amplitude \([-1,0]\)}
% -------------------------

\begin{lemma}[Edge lemma for \(\Ext^1\) in amplitude \(\lbrack -1,0\rbrack\)]\label{C:lem:edge-ext}
For \(A\in D^{[-1,0]}(k\text{-mod})\) there is a natural isomorphism
\[
\Ext^1\!\big(A,k[0]\big)\ \cong\ \Hom_k\!\big(H^{-1}(A),\,k\big).
\]
\end{lemma}

\begin{proof}[Proof sketch]
Use the standard truncation triangle
\(\tau_{\le -1}A \to A \to \tau_{\ge 0}A \to\),
where \(\tau_{\le -1}A\simeq H^{-1}(A)[1]\) and \(\tau_{\ge 0}A\simeq H^0(A)[0]\).
Apply \(\RHom(-,k[0])\) and take \(H^1\), using \(\Ext^{>0}(V,k)=0\) for finite-dimensional \(k\)-spaces \(V\).
\end{proof}

\begin{corollary}[Dimension and perfect pairing]\label{C:cor:dim}
For any eligible \(F\) (so \(\mathcal{R}(F)\in D^{[-1,0]}\)),
\[
\dim_k \Ext^1\!\big(\mathcal{R}(F),k[0]\big)\ =\ \dim_k\, H^{-1}\!\big(\mathcal{R}(F)\big)\ =\ \dim_k\Big(\varinjlim_t H_1(F^t)\Big),
\]
and the canonical pairing
\(\Ext^1(\mathcal{R}(F),k[0])\otimes H^{-1}(\mathcal{R}(F))\to k\)
is perfect.
\end{corollary}

% -------------------------
\subsection*{C.4. The forward bridge and its after-collapse/windowed form}
% -------------------------

\begin{theorem}[Bridge \(\mathrm{PH}_1\Rightarrow \Ext^1\) (eligible regime)]\label{C:thm:bridge}
Let \(F\in \mathsf{FiltCh}(k)\) be eligible (so \(\mathcal{R}(F)\in D^{[-1,0]}\)).
If \(\mathrm{PH}_1(F)=0\) (equivalently, \(H_1(F^t)=0\) for all \(t\)), then
\[
\Ext^1\!\big(\mathcal{R}(F),\,k[0]\big)\ =\ 0.
\]
\end{theorem}

\begin{proof}
If \(\mathrm{PH}_1(F)=0\), then \(\varinjlim_t H_1(F^t)=0\).
By Proposition~\ref{C:prop:edge}, \(H^{-1}(\mathcal{R}(F))=0\).
Apply Lemma~\ref{C:lem:edge-ext}.
\end{proof}

\begin{corollary}[After-collapse/windowed bridge (gate form)]\label{C:cor:windowed}
Fix \(\tau>0\) and a right-open window \(W=[u,v)\).
Assume eligibility on \(W\), i.e.\ \(\mathcal{R}((C_\tau F)|_W)\in D^{[-1,0]}\).
If \(\mathrm{PH}_1\!\big((C_\tau F)|_W\big)=0\), then
\[
\Ext^1\!\Big(\mathcal{R}\big((C_\tau F)|_W\big),\,k[0]\Big)=0.
\]
\end{corollary}

% -------------------------
\subsection*{C.5. The local reverse under \texorpdfstring{$E_1{=}0$}{E1=0} (formal statement of P3)}
% -------------------------

\begin{definition}[Tail stabilization on a right-open window]\label{C:def:tail}
Let \(W=[u,v)\) be right-open.
We say that a persistence module \(M\in\Pers^{\mathrm{ft}}_k\) has \emph{tail stabilization on \(W\)} if there exists \(t_0\in(u,v)\) such that for all \(t_0\le t\le t'<v\), the structure map \(M(t)\to M(t')\) is an isomorphism.
We say \(F\in\mathsf{FiltCh}(k)\) has tail stabilization on \(W\) in degree \(1\) if \(\mathbf{P}_1(F|_W)\) has tail stabilization.
\end{definition}

\begin{remark}[About \(E_1(W)=0\)]
The predicate \(E_1(W)=0\) is the window-local trigger defined in Chapter~11 (energy/spectral layer) and is evaluated \emph{after collapse} on \(W\).
In this appendix we only use that \(E_1(W)=0\), together with tail stabilization, identifies the stabilized degree-\(1\) edge on \(W\) with the windowed persistent \(H_1\) in the eligible regime.
\end{remark}

\begin{lemma}[Edge identification on \(E_1\)-degenerate windows]\label{C:lem:E1-edge}
Let \(W=[u,v)\) be right-open.
Assume \(\,E_1(W)=0\) and that \((C_\tau F)\) has tail stabilization on \(W\) in degree \(1\).
Assume also eligibility on \(W\): \(\mathcal{R}((C_\tau F)|_W)\in D^{[-1,0]}\).
Then there is a natural isomorphism
\[
H^{-1}\!\Big(\mathcal{R}\big((C_\tau F)|_W\big)\Big)\ \cong\ \mathrm{PH}_1\!\big((C_\tau F)|_W\big).
\]
\end{lemma}

\begin{proof}[Proof sketch]
Tail stabilization on \(W\) identifies the directed colimit \(\varinjlim_{t\in W}H_1(((C_\tau F)|_W)^t)\) with the stabilized value of the degree-\(1\) persistence on \(W\).
The hypothesis \(E_1(W)=0\) provides the \(E_1\)-degeneracy needed to identify this stabilized edge with the windowed persistent class used by \(\mathrm{PH}_1\) in the gate layer (Chapter~11).
Combine with the edge compatibility in Proposition~\ref{C:prop:edge}.
\end{proof}

\begin{theorem}[Local Reverse under \(E_1{=}0\) (P3; eligible regime)]\label{C:thm:local-reverse}
Let \(W=[u,v)\) be right-open.
Assume \(E_1(W)=0\), tail stabilization on \(W\) for \((C_\tau F)\) in degree \(1\), and eligibility on \(W\):
\(\mathcal{R}((C_\tau F)|_W)\in D^{[-1,0]}\).
If
\[
\Ext^1\!\Big(\mathcal{R}\big((C_\tau F)|_W\big),\,k[0]\Big)=0,
\]
then
\[
\mathrm{PH}_1\!\big((C_\tau F)|_W\big)=0.
\]
\end{theorem}

\begin{proof}
By Lemma~\ref{C:lem:E1-edge},
\(H^{-1}(\mathcal{R}((C_\tau F)|_W))\cong \mathrm{PH}_1((C_\tau F)|_W)\).
By Lemma~\ref{C:lem:edge-ext}, \(\Ext^1(-,k[0])\cong \Hom(H^{-1}(-),k)\) in the eligible regime.
Hence \(\Ext^1=0\Rightarrow H^{-1}=0\Rightarrow \mathrm{PH}_1=0\).
\end{proof}

\begin{corollary}[Local equivalence on suitable windows]\label{C:cor:definable-local-bridge}
Let \(W=[u,v)\) be right-open and (UCC-)definable with finite event set and finite \v{C}ech depth (Appendix~Q; cf.\ Appendix~A for window conventions).
Assume eligibility on \(W\) and tail stabilization on \(W\) for \((C_\tau F)\) in degree \(1\).
If \(E_1(W)=0\), then for every \(F\) and \(\tau>0\),
\[
\mathrm{PH}_1\!\big((C_\tau F)|_{W}\big)=0
\ \Longleftrightarrow\
\Ext^1\!\Big(\mathcal{R}\big((C_\tau F)|_{W}\big),\,k[0]\Big)=0,
\]
with \(\Rightarrow\) by Theorem~\ref{C:thm:bridge} and \(\Leftarrow\) by Theorem~\ref{C:thm:local-reverse}.
\end{corollary}

% -------------------------
\subsection*{C.6. Naturality, exact triangles, and stability}
% -------------------------

\begin{proposition}[Naturality of the \(\Ext^1\) edge isomorphism]\label{C:prop:naturality}
For any morphism \(f:F\to G\) preserving filtrations, under eligibility the natural square
\[
\begin{tikzcd}[row sep=1.2em, column sep=2.4em]
\Ext^1\!\big(\mathcal{R}(F),k[0]\big)  \arrow[r, "\sim"]
  \arrow[d, "{\Ext^1(\mathcal{R}(f),\,k[0])}"'] &
\Hom\!\big(H^{-1}(\mathcal{R}(F)),k\big)
  \arrow[d, "{\Hom(H^{-1}(\mathcal{R}(f)),\,k)}"]\\
\Ext^1\!\big(\mathcal{R}(G),k[0]\big)
  \arrow[r, "\sim"] &
\Hom\!\big(H^{-1}(\mathcal{R}(G)),k\big)
\end{tikzcd}
\]
commutes.
\end{proposition}

\begin{lemma}[Exact triangles and 2-out-of-3 for \(\Ext^1\)]\label{C:lem:2of3}
Let \(A_1\to A_2\to A_3\to A_1[1]\) be a distinguished triangle in \(D^{\mathrm{b}}(k\text{-mod})\) with each \(A_i\in D^{[-1,0]}\).
If two of \(\Ext^1(A_i,k[0])\) vanish, then so does the third.
\end{lemma}

\begin{remark}[Stability under admissible updates (eligible regime)]
If \(F\mapsto F'\) is an admissible update that is non-expansive at the persistence layer after collapse (e.g.\ deletion-type updates), and both \(\mathcal{R}(C_\tau F)\) and \(\mathcal{R}(C_\tau F')\) are eligible, then the verdict \(\Ext^1(\mathcal{R}(C_\tau F),k[0])=0\) is stable under \(F\mapsto F'\) whenever the stabilized edge group (equivalently \(H^{-1}\)) is preserved.
\end{remark}

% -------------------------
\subsection*{C.7. Implementation details, reproducibility, and logging}
% -------------------------

\begin{remark}[Run-time policy and manifest fields]\label{C:rk:runtime}
Enforce the order \(\,F\to C_\tau F\to (C_\tau F)|_W\to \mathcal{R}((C_\tau F)|_W)\to \Ext^1(-,k[0])\,\).
The manifest \texttt{run.yaml} must include (per window \(W\) and \(\tau\)):
\begin{itemize}
  \item \texttt{ext1\_eligible}: boolean; \texttt{true} iff \(\mathcal{R}((C_\tau F)|_W)\in D^{[-1,0]}\);
  \item \texttt{ext1\_used\_in\_gate}: boolean; \texttt{true} iff \texttt{ext1\_eligible} and gate policy enables it;
  \item \texttt{amplitude}: reported as \texttt{"[-1,0]"} or \texttt{">1"};
  \item \texttt{gate\_order}: \texttt{"collapse→clip→realize→ext1"};
  \item \texttt{q\_test}: \texttt{"k[0]"};
  \item \texttt{window\_E1\_zero}: boolean (Chapter~11 predicate, evaluated after collapse);
  \item \texttt{tail\_stable\_H1}: boolean (Definition~\ref{C:def:tail});
  \item \texttt{notes}: free text; if \texttt{window\_E1\_zero:true} and \texttt{tail\_stable\_H1:true}, then “Local reverse (P3) enabled”.
\end{itemize}
Outside eligibility, set \texttt{ext1\_eligible:false}, \texttt{ext1\_used\_in\_gate:false}, and continue the gate using persistence/tower/safety criteria only.
\end{remark}

\begin{verbatim}
# run.yaml (excerpt; per window W and tau)
ext1_eligible: true
ext1_used_in_gate: true
amplitude: "[-1,0]"
gate_order: "collapse→clip→realize→ext1"
q_test: "k[0]"
window_E1_zero: true
tail_stable_H1: true
notes: "Local reverse (P3) enabled on E1=0 window with tail stabilization."
\end{verbatim}

% -------------------------
\subsection*{C.8. Counterexamples and boundary cases}
% -------------------------

\begin{example}[Failure of the global reverse implication \(\Ext^1\Rightarrow \mathrm{PH}_1\)]
A single \emph{finite} degree-\(1\) bar yields \(\mathrm{PH}_1(F)\neq 0\) while the directed colimit
\(\varinjlim_{t\to +\infty}H_1(F^t)=0\), hence \(\Ext^1(\mathcal{R}(F),k[0])=0\) in the eligible regime.
Thus the converse fails globally (and this is why the reverse direction is only asserted locally under \(E_1(W)=0\) with tail stabilization).
\end{example}

\begin{example}[Amplitude breach: diagnostics only]
If \(\mathcal{R}((C_\tau F)|_W)\notin D^{[-1,0]}\) (e.g.\ it lies in \(D^{[-2,0]}\) with \(H^{-2}\neq 0\)), then Lemma~\ref{C:lem:edge-ext} is inapplicable.
The \(\Ext^1\)-test is \emph{ineligible} and must not be used for gating.
\end{example}

\begin{example}[Non-constructible tails]
If \(\mathbf{P}_1(F)\notin \Pers^{\mathrm{ft}}_k\), filtered colimits in \([\mathbb{R},\Vect_k]\) may exit the constructible regime (Appendix~A, Remark~\ref{A:rk:filtered-colimits}).
Eligibility fails by \textup{(B1)}.
\end{example}

% -------------------------
\subsection*{C.9. Additional safeguards and best practices (policy layer)}
% -------------------------

\begin{remark}[Monotonicity across thresholds (implementation policy)]
In the block-diagonal reference implementation (Appendix~B), eligibility is monotone under strengthening collapse: if \(\mathcal{R}((C_{\tau'}F)|_W)\in D^{[-1,0]}\) for some \(\tau'\ge \tau\), then \(\mathcal{R}((C_{\tau}F)|_W)\in D^{[-1,0]}\).
Accordingly, eligibility is logged per \((W,\tau)\), but the implementation may reuse witnesses across a \(\tau\)-sweep when monotonicity is verified.
\end{remark}

\begin{remark}[Uniformity under base change (benignity)]
If \(k\subset K\) is a field extension, then (within the eligible regime) the vanishing verdict
\(\Ext^1(\mathcal{R}((C_\tau F)|_W),k[0])=0\) is stable under scalar extension to \(K\).
Thus the gate decision is field-independent within the amplitude \([-1,0]\) regime.
\end{remark}

% -------------------------
\subsection*{C.10. Scope and non-claims}
% -------------------------

\begin{remark}[Scope and non-claims]\label{C:rk:scope}
The forward bridge \(\mathrm{PH}_1\Rightarrow \Ext^1\) is asserted and used only in the eligible regime \(\mathcal{R}((C_\tau F)|_W)\in D^{[-1,0]}\), and only for the test object \(k[0]\).
The global converse is \emph{false}.
The \emph{local reverse} (Theorem~\ref{C:thm:local-reverse}, formal statement of P3) holds on right-open windows where \(E_1(W)=0\) and the degree-\(1\) tail stabilizes, in the eligible regime.
All uses respect the after-collapse order and the filtered-colimit scope policy of Appendix~A.
No claims are made outside the persistence/derived layers in the implementable range.
\end{remark}

\medskip
\noindent\emph{Summary of Appendix C (reinforced and complete).}
Under \textup{(B1)–(B3)} and eligibility (\(\mathcal{R}((C_\tau F)|_W)\in D^{[-1,0]}\)), the edge compatibility identifies
\(H^{-1}(\mathcal{R}((C_\tau F)|_W))\) with the stabilized windowed \(H_1\), and the amplitude \([-1,0]\) calculus yields
\(\Ext^1(\mathcal{R}((C_\tau F)|_W),k[0])\cong \Hom(H^{-1}(\cdot),k)\).
This gives the forward bridge
\(\mathrm{PH}_1((C_\tau F)|_W)=0\Rightarrow \Ext^1(\mathcal{R}((C_\tau F)|_W),k[0])=0\),
and, on \(E_1(W)=0\) windows with tail stabilization, the local reverse
\(\Ext^1=0\Rightarrow \mathrm{PH}_1=0\) (P3).
Operationally, the order is \textbf{collapse} \(\to\) \textbf{clip} \(\to\) \textbf{realize} \(\to\) \textbf{Ext}; outside eligibility, \(\Ext^1\) is excluded from gate decisions.

\endgroup



% =========================
\section*{Appendix D. Towers, \texorpdfstring{$\mu,\nu$}{mu,nu}, and Examples [Proof/Example] (reinforced, complete v17.0)}
% =========================
\addcontentsline{toc}{section}{Appendix D. Towers, $\mu,\nu$, and Examples [Proof/Example]}
\refstepcounter{section} % stable anchors for \label/\ref in unnumbered appendix headings

\begingroup

\usetikzlibrary{arrows.meta,cd}

% ---- local safety (only if not already defined globally) ----
\providecommand{\gdim}{\operatorname{gdim}}

Throughout, fix a field \(k\).
We work in the constructible regime (Appendix~A).
Let \(\mathsf{FiltCh}(k)\) denote the category of bounded-in-degree filtered chain complexes of finite-dimensional \(k\)-vector spaces with filtration-preserving chain maps (Appendix~B).
For each degree \(i\in\mathbb{Z}\), the degreewise persistence functor is
\[
\mathbf{P}_i:\ \mathsf{FiltCh}(k)\longrightarrow \Pers^{\mathrm{ft}}_k,\qquad
F\longmapsto \big(t\mapsto H_i(F^{t}C_\bullet)\big).
\]
The truncation/deletion reflector \(\mathbf{T}_\tau:\Pers^{\mathrm{ft}}_k\to\Pers^{\mathrm{ft}}_{k,\tau\text{-loc}}\) is exact and \(1\)-Lipschitz (Appendix~A).
Filtered colimits are computed objectwise in \([\mathbb{R},\Vect_k]\), subject to the scope rule of Appendix~A, Remark~\ref{A:rk:filtered-colimits}.
All statements at the filtered–complex layer are \emph{up to filtered quasi-isomorphism (f.q.i.)}; persistence-layer statements take place \emph{strictly} in \(\Pers^{\mathrm{ft}}_k\).
All quantities below may depend on the threshold \(\tau>0\); \emph{no monotonicity in \(\tau\)} is asserted.

\medskip
\noindent\textbf{Notation for “Type IV” verdicts at scale \(\tau\).}
For a fixed tower and threshold \(\tau\), write
\[
\DiagZero(\tau)\quad:\Longleftrightarrow\quad
\mu_{i,\tau}=0\ \text{and}\ \nu_{i,\tau}=0\ \text{for all relevant }i,
\]
and \(\DiagNonzero(\tau)\) otherwise.
This is the \emph{tower-level (Type~IV)} defect detector at scale \(\tau\).

% -------------------------
% V-metric calculus (reinforcement)
% -------------------------
\begin{proposition}[Calculus of \(\mu,\nu\) in the \(V\)-metric]\label{D:prop:V-subadd}
Let \(V\) be a commutative quantale and endow \(\Pers^{\mathrm{ft}}_k\) with a Lawvere \(V\)-metric (Chapter~2; Appendix~A, Remark~\ref{rem:V-enriched}).
For any tower with apex and any \(\tau>0\), the obstruction indices computed \emph{after} \(\mathbf{T}_\tau\),
\[
\mu_{i,\tau}:=\gdim\Ker(\phi_{i,\tau}),\qquad
\nu_{i,\tau}:=\gdim\coker(\phi_{i,\tau}),
\]
satisfy (uniformly in the choice of \(V\)):
\begin{enumerate}\itemsep0.2em
\item \textbf{Cofinal/f.q.i.\ invariance:} \(\mu,\nu\) are invariant under cofinal reindexing of the tower and under levelwise f.q.i.\ replacements.
\item \textbf{Composition subadditivity:} for composable comparison maps \(\psi,\phi\),
\[
\mu(\psi\!\circ\!\phi)\le \mu(\phi)+\mu(\psi),\qquad
\nu(\psi\!\circ\!\phi)\le \nu(\psi)+\mu(\phi).
\]
\item \textbf{Additivity on finite sums:} \(\mu,\nu\) are additive under finite direct sums of towers.
\item \textbf{Window finiteness on definable covers:} on o\hbox{-}minimal definable windows with finite \v{C}ech/Leray depth (Appendix~H/J), all \(\mu_{i,\tau},\nu_{i,\tau}\) are finite and only finitely many \(\tau\)-criticalities occur per window.
\end{enumerate}
All multiplicities are read as the number of \(I[0,\infty)\) summands in the barcode of the relevant kernel/cokernel after \(\mathbf{T}_\tau\).
\end{proposition}

\begin{proof}[Proof sketch]
By Appendix~A, Remark~\ref{rem:V-enriched}, \(V\)-enrichment does not alter the underlying abelian/Serre calculus: kernels, cokernels, and \(\mathbf{T}_\tau\) are computed in \(\Pers^{\mathrm{ft}}_k\) as in the unenriched case.
Hence (1)–(3) follow from the barcode calculus and the linear-algebra inequalities used below (Propositions~\ref{D:prop:gdim-iff}, \ref{D:prop:subadd}, \ref{D:prop:additive}).
For (4), definable covers yield finite depth, so only finitely many events contribute on a window (Appendix~H/J), and finiteness follows.
\end{proof}

\medskip
\noindent\textbf{Comparison map and obstruction indices.}
Let \(\{F_n\}_{n\in\mathbb{N}}\) be a directed system in \(\mathsf{FiltCh}(k)\).
Let \(F_\infty\) be an apex equipped with a cocone \(F_n\to F_\infty\) (indexing category \(\mathbb{N}\cup\{\infty\}\) with unique morphisms \(n\to\infty\)).
For each \(i\) and \(\tau>0\) define the comparison morphism in \([\mathbb{R},\Vect_k]\)
\[
\phi_{i,\tau}:\ \colim_{n\in\mathbb{N}}\, \mathbf{T}_\tau\!\big(\mathbf{P}_i(F_n)\big)\ \longrightarrow\ \mathbf{T}_\tau\!\big(\mathbf{P}_i(F_\infty)\big).
\]
Define the \emph{tower obstruction indices}
\[
\mu_{i,\tau}:=\gdim\Ker(\phi_{i,\tau}),\qquad
\nu_{i,\tau}:=\gdim\coker(\phi_{i,\tau}),
\qquad
\mu_{\mathrm{tot},\tau}:=\sum_i\mu_{i,\tau},\ \ 
\nu_{\mathrm{tot},\tau}:=\sum_i\nu_{i,\tau}.
\]
These sums are finite because complexes are bounded in homological degrees and we work in the constructible range.

\begin{remark}[Generic dimension after truncation]\label{D:rem:generic-dim}
In \(\Pers^{\mathrm{ft}}_k\), after applying \(\mathbf{T}_\tau\) the kernel and cokernel of any morphism decompose (noncanonically) as finite direct sums of interval modules.
We write \(\gdim(-)\) for the \emph{generic fiber} dimension, i.e.\ the multiplicity of the infinite bar \(I[0,\infty)\) in that decomposition.
Finite bars contribute zero generic fiber.
\end{remark}

\begin{remark}[Invariance of \(\DiagZero/\DiagNonzero\)]
The indices \((\mu,\nu)\) (hence \(\DiagZero/\DiagNonzero\)) are invariant under levelwise f.q.i.\ replacements of the tower and apex: if \(F_n\simeq_{\mathrm{f.q.i.}}F'_n\) and \(F_\infty\simeq_{\mathrm{f.q.i.}}F'_\infty\), then \(\mathbf{P}_i\) sends these to isomorphisms in \(\Pers^{\mathrm{ft}}_k\), hence kernels/cokernels (and thus \(\mu,\nu\)) are unchanged.
They are also invariant under cofinal reindexing of the tower, since filtered colimits over cofinal subdiagrams are canonically isomorphic.
\end{remark}

\begin{figure}[t]
\centering
\begin{tikzcd}[row sep=1.0em, column sep=2.0em]
F_1 \arrow[r] \arrow[dr] &
F_2 \arrow[r] \arrow[dr] &
\cdots \arrow[r] &
F_n \arrow[r] \arrow[dr] &
\cdots \arrow[r] &
F_\infty \\
& \mathbf{P}_i(F_1) \arrow[r] &
\mathbf{P}_i(F_2) \arrow[r] &
\cdots \arrow[r] &
\mathbf{P}_i(F_n) \arrow[r] &
\mathbf{P}_i(F_\infty)
\end{tikzcd}
\caption{A tower with apex \(F_\infty\) and its image under \(\mathbf{P}_i\).
The comparison \(\phi_{i,\tau}\) (defined after applying \(\mathbf{T}_\tau\)) measures the failure of the cocone to exhibit a colimit at scale \(\tau\).}
\end{figure}

% -------------------------
\subsection*{D.1. Calculus of defects: generic-fiber interpretation, naturality, and subadditivity}
% -------------------------

\begin{proposition}[Generic-fiber interpretation]\label{D:prop:gdim-iff}
Let \(f:M\to N\) be a morphism in \(\Pers^{\mathrm{ft}}_k\) and fix \(\tau>0\).
Then, in the barcode decomposition of \(\Ker(\mathbf{T}_\tau f)\) and \(\coker(\mathbf{T}_\tau f)\), the multiplicity of \(I[0,\infty)\) equals \(\gdim\Ker(\mathbf{T}_\tau f)\) and \(\gdim\coker(\mathbf{T}_\tau f)\), respectively.
Equivalently, \(\gdim\) is the generic fiber dimension of the corresponding functor \(\mathbb{R}\to\Vect_k\).
\end{proposition}

\begin{proof}
Standard for constructible \(1\)D persistence over a field: after \(\mathbf{T}_\tau\), objects split as finite sums of interval modules (Appendix~A).
The generic fiber (dimension on a cofinal ray) counts infinite bars, i.e.\ copies of \(I[0,\infty)\).
\end{proof}

\begin{definition}[Morphisms of towers and naturality of \(\phi\)]\label{D:def:tower-mor}
A \emph{morphism of towers with apex} \((F_\bullet,F_\infty)\to(G_\bullet,G_\infty)\) is a collection of maps \(u_n:F_n\to G_n\) and \(u_\infty:F_\infty\to G_\infty\) commuting with all structure maps to the apex.
Applying \(\mathbf{P}_i\), then \(\mathbf{T}_\tau\), and passing to the filtered colimit yields a commutative square
\[
\begin{tikzcd}
\colim_n \mathbf{T}_\tau \mathbf{P}_i(F_n) \arrow[r,"\phi^{F}_{i,\tau}"] \arrow[d,"\colim\,\mathbf{T}_\tau\mathbf{P}_i(u_n)"'] &
\mathbf{T}_\tau \mathbf{P}_i(F_\infty) \arrow[d,"\mathbf{T}_\tau \mathbf{P}_i(u_\infty)"] \\
\colim_n \mathbf{T}_\tau \mathbf{P}_i(G_n) \arrow[r,"\phi^{G}_{i,\tau}"] &
\mathbf{T}_\tau \mathbf{P}_i(G_\infty),
\end{tikzcd}
\]
i.e.\ \(\phi_{i,\tau}\) is natural in the tower.
\end{definition}

\begin{proposition}[Subadditivity under composition]\label{D:prop:subadd}
Let \((F_\bullet,F_\infty)\xrightarrow{u}(G_\bullet,G_\infty)\xrightarrow{v}(H_\bullet,H_\infty)\) be morphisms of towers with apex.
Then, for each \(i,\tau\),
\[
\mu\big(\phi^{H}_{i,\tau}\!\circ\!\colim\,\mathbf{T}_\tau\mathbf{P}_i(v_n\circ u_n)\big)
\ \le\
\mu\big(\phi^{G}_{i,\tau}\!\circ\!\colim\,\mathbf{T}_\tau\mathbf{P}_i(u_n)\big)
+\mu\big(\phi^{H}_{i,\tau}\!\circ\!\colim\,\mathbf{T}_\tau\mathbf{P}_i(v_n)\big),
\]
\[
\nu\big(\phi^{H}_{i,\tau}\!\circ\!\colim\,\mathbf{T}_\tau\mathbf{P}_i(v_n\circ u_n)\big)
\ \le\
\nu\big(\phi^{H}_{i,\tau}\!\circ\!\colim\,\mathbf{T}_\tau\mathbf{P}_i(v_n)\big)
+\mu\big(\phi^{G}_{i,\tau}\!\circ\!\colim\,\mathbf{T}_\tau\mathbf{P}_i(u_n)\big).
\]
In particular, for any factorization of a fixed comparison map \(\phi\), one has
\(\mu(\psi\circ\phi)\le \mu(\phi)+\mu(\psi)\) and \(\nu(\psi\circ\phi)\le \nu(\psi)+\mu(\phi)\).
\end{proposition}

\begin{proof}
Apply \(\mathbf{T}_\tau\) and use exactness together with the standard inequalities for kernels and cokernels of compositions in finite-dimensional linear algebra, then pass to generic fibers via Proposition~\ref{D:prop:gdim-iff}.
\end{proof}

\begin{proposition}[Additivity on finite direct sums]\label{D:prop:additive}
For two towers \((F_\bullet,F_\infty)\) and \((G_\bullet,G_\infty)\),
\[
\mu\big((F\oplus G)_\bullet,(F\oplus G)_\infty\big)=\mu(F_\bullet,F_\infty)+\mu(G_\bullet,G_\infty),
\qquad
\nu\big((F\oplus G)_\bullet,(F\oplus G)_\infty\big)=\nu(F_\bullet,F_\infty)+\nu(G_\bullet,G_\infty).
\]
\end{proposition}

\begin{proof}
\(\mathbf{P}_i\) and \(\mathbf{T}_\tau\) preserve finite direct sums; kernels/cokernels preserve finite direct sums; \(\gdim\) is additive on direct sums.
\end{proof}

\begin{proposition}[Invariance under f.q.i.\ and cofinal reindexing]\label{D:prop:invariance}
If \(F_n\simeq_{\mathrm{f.q.i.}}F'_n\) levelwise and \(F_\infty\simeq_{\mathrm{f.q.i.}}F'_\infty\), then \(\mu,\nu\) agree for the two towers.
If \(J\subset \mathbb{N}\) is cofinal, then restricting the tower to \(J\) does not change \(\mu,\nu\).
\end{proposition}

\begin{proof}
\(\mathbf{P}_i\) sends f.q.i.\ maps to isomorphisms in \(\Pers^{\mathrm{ft}}_k\); \(\mathbf{T}_\tau\) is exact; filtered colimits over cofinal subdiagrams are canonically isomorphic.
\end{proof}

% -------------------------
\subsection*{D.2. Toy towers: pure kernel / pure cokernel / mixed}
% -------------------------

\begin{example}[Pure cokernel at a fixed scale]\label{D:ex:pure-coker}
Fix \(\tau>0\) and degree \(i=1\).
Let \(\mathbf{P}_1(F_n)=I[0,\tau-\tfrac1n)\) with transition maps the evident inclusions.
Let \(F_\infty\) satisfy \(\mathbf{P}_1(F_\infty)=I[0,\infty)\).
Then \(\mathbf{T}_\tau(\mathbf{P}_1(F_n))=0\) for all \(n\), whereas \(\mathbf{T}_\tau(\mathbf{P}_1(F_\infty))\cong I[0,\infty)\).
Hence \(\phi_{1,\tau}:0\to I[0,\infty)\) has trivial kernel and nontrivial cokernel, so \(\mu_{1,\tau}=0\) and \(\nu_{1,\tau}=1\) (pure cokernel).
\end{example}

\begin{example}[Pure kernel at a fixed scale]\label{D:ex:pure-ker}
Fix \(\tau>0\).
Let \(\mathbf{P}_1(F_n)=I[0,\infty)\) for all \(n\), with transition maps the identities.
Let \(F_\infty\) satisfy \(\mathbf{P}_1(F_\infty)=0\), and take the cocone maps \(\mathbf{P}_1(F_n)\to \mathbf{P}_1(F_\infty)\) to be \(0\) for all \(n\).
Then \(\mathbf{T}_\tau(\mathbf{P}_1(F_n))\cong I[0,\infty)\) for all \(n\), so the source of \(\phi_{1,\tau}\) is \(I[0,\infty)\), while the target is \(0\).
Thus \(\phi_{1,\tau}: I[0,\infty)\to 0\) has nontrivial kernel and zero cokernel, hence \(\mu_{1,\tau}=1\), \(\nu_{1,\tau}=0\) (pure kernel).
\end{example}

\begin{example}[Mixed]\label{D:ex:mixed}
Fix \(\tau>0\) and set
\[
\mathbf{P}_1(F_n)\ =\ I[0,\tau-\tfrac1n)\ \oplus\ I[0,\infty),
\]
with transition maps the obvious inclusions on the first summand and the identities on the second.
Take \(F_\infty\) with \(\mathbf{P}_1(F_\infty)=I[0,\infty)\oplus 0\), using cocone maps that send the second summand to \(0\).
Then the first summand yields a cokernel contribution as in Example~\ref{D:ex:pure-coker}, while the second yields a kernel contribution as in Example~\ref{D:ex:pure-ker}.
Hence \(\mu_{1,\tau}=\nu_{1,\tau}=1\) (mixed).
\end{example}

\begin{remark}[Realizability by filtered complexes]
All persistence-level towers above are realizable by filtered complexes via the interval-realization assignment \(\mathcal{U}\) from Appendix~B, up to f.q.i.; constructibility is preserved.
\end{remark}

% -------------------------
\subsection*{D.3. When \texorpdfstring{$\phi_{i,\tau}$}{phi} is an isomorphism: \texorpdfstring{$\DiagZero$}{(mu,nu)=(0,0)}}
% -------------------------

\begin{proposition}[Isomorphism criterion]\label{D:prop:iso-zero}
Assume:
\begin{enumerate}[label=(\roman*)]
\item degreewise filtered colimits in \(\mathsf{FiltCh}(k)\) are computed objectwise on chains and filtrations;
\item each \(\mathbf{P}_i(F_n)\) lies in \(\Pers^{\mathrm{ft}}_k\);
\item \(\mathbf{T}_\tau\) commutes with the filtered colimit of \(\{\mathbf{P}_i(F_n)\}\) in \([\mathbb{R},\Vect_k]\), and the result is constructible (Appendix~A);
\item the cocone exhibits a colimit at persistence level: the canonical map
\(\colim_{n}\mathbf{P}_i(F_n)\xrightarrow{\ \cong\ }\mathbf{P}_i(F_\infty)\)
is an isomorphism in \([\mathbb{R},\Vect_k]\).
\end{enumerate}
Then for every \(i\) and \(\tau>0\), the comparison map \(\phi_{i,\tau}\) is an isomorphism in \(\Pers^{\mathrm{ft}}_k\).
Consequently \(\DiagZero(\tau)\).
\end{proposition}

\begin{remark}
Condition (iv) is automatic if \(F_\infty\) is the colimit of \(\{F_n\}\) in a filtered-complex model where \(\mathbf{P}_i\) is computed objectwise and the scope rule of Appendix~A applies; no claim is made beyond that regime.
\end{remark}

% -------------------------
\subsection*{D.4. Sufficient conditions ensuring \texorpdfstring{$\DiagZero$}{(mu,nu)=(0,0)}}
% -------------------------

The summability condition \(\sum_{n} d_{\mathrm{int}}\!\big(\mathbf{P}_i(F_{n+1}),\mathbf{P}_i(F_n)\big)<\infty\) \emph{alone} does not guarantee \(\DiagZero(\tau)\); see \S D.5.
The following hypotheses are sufficient.

\begin{theorem}[Sufficient conditions for \(\phi_{i,\tau}\) to be an isomorphism]\label{D:thm:sufficient}
Fix \(i\) and \(\tau>0\).
Each of the following implies that \(\phi_{i,\tau}\) is an isomorphism (hence \(\DiagZero(\tau)\)):
\begin{enumerate}[label=(S\arabic*)]
\item \textbf{Commutation and apex colimit:} \(\mathbf{T}_\tau\) commutes with the filtered colimit of \(\{\mathbf{P}_i(F_n)\}\) in \([\mathbb{R},\Vect_k]\), the outcome is constructible, and the cocone exhibits a colimit at persistence level (Proposition~\ref{D:prop:iso-zero}(iv)).
\item \textbf{No \(\tau\)-accumulation from below:} there exists \(\eta>0\) such that, for all sufficiently large \(n\), no bar in \(\mathbf{P}_i(F_n)\) has length in \((\tau-\eta,\tau)\).
Equivalently, there is no sequence of bar lengths strictly increasing to \(\tau\).
\item \textbf{\(\mathbf{T}_\tau\)-Cauchy with compatible cocone:} the sequence \(\mathbf{T}_\tau(\mathbf{P}_i(F_n))\) is Cauchy in the interleaving metric, and the cocone to \(\mathbf{T}_\tau(\mathbf{P}_i(F_\infty))\) identifies the metric limit with the colimit target.
(Here we use only completeness/uniqueness of limits for p.f.d.\ barcodes under bottleneck/interleaving distance.)
\end{enumerate}
\end{theorem}

\begin{proof}
(S1) is Proposition~\ref{D:prop:iso-zero}.
For (S2), the gap prevents creation at the apex of new bars of length \(>\tau\): every long bar in \(\mathbf{T}_\tau(\mathbf{P}_i(F_\infty))\) must appear at some finite stage and stabilize, yielding bijectivity on interval factors.
For (S3), completeness gives a unique metric limit; the stated compatibility identifies it with the colimit target, so \(\phi_{i,\tau}\) is an isometry and hence an isomorphism in \(\Pers^{\mathrm{ft}}_k\).
\end{proof}

% -------------------------
\subsection*{D.5. A counterexample: \(\sum d_{\mathrm{int}}<\infty\) yet \(\DiagNonzero\)}\label{D:subsec:counter}
% -------------------------

\begin{example}[Summable increments, pure cokernel at the apex]\label{D:ex:summable}
Fix \(\tau>0\) and set \(\ell_n=\tau-\sum_{m\ge n}2^{-m}\uparrow \tau\), so that
\(\sum_{n}(\ell_{n+1}-\ell_{n})=\sum_{n}2^{-n}<\infty\).
Let \(M_n:=I[0,\ell_n)\) with \(M_{n}\hookrightarrow M_{n+1}\) the standard inclusions.
Then
\[
d_{\mathrm{int}}(M_{n},M_{n+1})=\tfrac12(\ell_{n+1}-\ell_{n})=2^{-(n+1)},\qquad
\sum_{n} d_{\mathrm{int}}(M_{n},M_{n+1})<\infty.
\]
Let \(\mathbf{P}_1(F_n)=M_n\), and choose an apex with \(\mathbf{P}_1(F_\infty)=I[0,\infty)\).
For every \(n\), \(\mathbf{T}_\tau(M_n)=0\), while \(\mathbf{T}_\tau(\mathbf{P}_1(F_\infty))\cong I[0,\infty)\).
Thus \(\mu_{\mathrm{tot},\tau}=0\), \(\nu_{\mathrm{tot},\tau}=1\) (pure cokernel), despite summable interleaving distances along the tower.
\end{example}

This shows that \(\sum d_{\mathrm{int}}<\infty\) alone is insufficient to force \(\DiagZero(\tau)\).

% -------------------------
\subsection*{D.6. Converse failures and the Type~IV catalog}
% -------------------------

\paragraph{D.6.1. \(\Ext^1=0\) does not imply \(\PH_1=0\).}
Let \(A\in D^{[-1,0]}(k\text{-mod})\) with \(H^{-1}(A)=0\) and \(H^{0}(A)\ne 0\), e.g.\ \(A=V[0]\) for a nonzero \(k\)-space \(V\).
Then \(\Ext^1(A,k[0])\cong \Hom(H^{-1}(A),k)=0\) by Appendix~C (Lemma~\ref{C:lem:edge-ext}).
Choose \(F\in\mathsf{FiltCh}(k)\) with \(\mathbf{P}_1(F)\neq 0\) (e.g.\ a single finite interval) and \(\mathcal{R}(F)\simeq A\); this can be arranged up to f.q.i.\ using the interval-realization template (Appendix~B).
Hence \(\Ext^1(\mathcal{R}(F),k[0])=0\) while \(\PH_1(F)\neq 0\), refuting the converse of the bridge globally.

\paragraph{D.6.2. Type~IV (pure cokernel) at fixed \(\tau\).}
Example~\ref{D:ex:pure-coker} exhibits \(\mu_{\mathrm{tot},\tau}=0\), \(\nu_{\mathrm{tot},\tau}>0\) with \(\mathbf{T}_\tau(\mathbf{P}_1(F_n))=0\) for all \(n\) but \(\mathbf{T}_\tau(\mathbf{P}_1(F_\infty))\neq 0\).

\paragraph{D.6.3. Type~IV (mixed).}
Example~\ref{D:ex:mixed} yields \(\mu_{\mathrm{tot},\tau}>0\) and \(\nu_{\mathrm{tot},\tau}>0\) simultaneously.

\paragraph{D.6.4. Realization notes.}
All persistence-level constructions above are realizable by filtered complexes via \(\mathcal{U}\) (Appendix~B), up to f.q.i.; constructibility is preserved.

% -------------------------
\subsection*{D.7. Restart/Summability for window pasting}\label{D:subsec:restart}
% -------------------------

All persistence-layer statements in this subsection are made \emph{after} applying \(\mathbf{T}_{\tau_k}\) on each window.

\begin{definition}[Per-window safety margin and pipeline budget]\label{D:def:window-budget}
Let \(\{W_k=[u_k,u_{k+1})\}_{k\in K}\) be a MECE partition (Appendix~A, Definition~\ref{A:def:MECE}).
On each window \(W_k\), fix a collapse threshold \(\tau_k>0\).
For a degree \(i\), define the \emph{pipeline budget}
\[
\Sigma\delta_k(i)\ :=\ \sum_{U\subseteq W_k}\Big(\delta^{\mathrm{alg}}_{U}(i,\tau_k)+\delta^{\mathrm{disc}}_{U}(i,\tau_k)+\delta^{\mathrm{meas}}_{U}(i,\tau_k)\Big),
\]
and the \emph{safety margin} \(\mathrm{gap}_{\tau_k}(i)>0\) as the configured slack for \(\mathrm{B\text{-}Gate}^{+}\) on \(W_k\) and degree \(i\).
(Here \(\sum_{U\subseteq W_k}\) ranges over the finite set of logged update-steps/units on \(W_k\).)
\end{definition}

\begin{lemma}[Restart inequality]\label{D:lem:restart}
Assume that, on window \(W_k\), \(\mathrm{B\text{-}Gate}^{+}\) passes with \(\mathrm{gap}_{\tau_k}(i)>\Sigma\delta_k(i)\), and that the transition to \(W_{k+1}\) is realized by a finite composition of \emph{deletion-type} steps and \(\varepsilon\)-continuations (both measured after \(\mathbf{T}_{\tau_k}\)).
Then there exists \(\kappa\in(0,1]\), depending only on the admissible step class and the \(\tau\)-adaptation policy, such that
\[
\mathrm{gap}_{\tau_{k+1}}(i)\ \ge\ \kappa\bigl(\mathrm{gap}_{\tau_k}(i)-\Sigma\delta_k(i)\bigr).
\]
\end{lemma}

\begin{proof}[Proof sketch]
Deletion-type steps are non-increasing for the monitored indicators after \(\mathbf{T}_{\tau_k}\) (Appendix~E), and \(\varepsilon\)-continuations are \(1\)-Lipschitz.
Aggregating drifts yields the stated retention factor \(\kappa\).
\end{proof}

\begin{definition}[Summability]\label{D:def:summability}
A run satisfies \emph{Summability} (on a degree set \(I\subset\mathbb{Z}\)) if
\[
\sum_{k\in K}\Sigma\delta_k(i)\ <\ \infty\qquad(\forall\,i\in I).
\]
A sufficient design is geometric decay of the windowed thresholds \(\tau_k\) (hence of bins) and bounded per-window step counts.
\end{definition}

\begin{theorem}[Pasting windowed certificates]\label{D:thm:pasting}
Let \(\{W_k\}_k\) be MECE, and on each \(W_k\) let \(\mathrm{B\text{-}Gate}^{+}\) pass with \(\mathrm{gap}_{\tau_k}(i)>\Sigma\delta_k(i)\) for all \(i\in I\).
If the Restart inequality (Lemma~\ref{D:lem:restart}) holds at every transition and Summability (Definition~\ref{D:def:summability}) holds, then the concatenation of windowed certificates yields a global certificate on \(\bigcup_k W_k\) for degrees \(i\in I\).
\end{theorem}

\paragraph{Convergence Manager (\(\Sigma\delta\)) --- auditable pseudocode.}
\begin{verbatim}
# Inputs:
#  windows: list of MECE windows W_k
#  degrees: monitored degree set I
#  tau:     list of collapse thresholds tau_k aligned with windows
#  deltas:  per-window lists of triples (delta_alg, delta_disc, delta_meas) per degree
#  policy:  either "geometric(r<1)" or "p_series(p>1)"
#
def convergence_manager(windows, degrees, tau, deltas, policy):
    total = {i: 0.0 for i in degrees}
    for k, Wk in enumerate(windows):
        for i in degrees:
            sigma = 0.0
            for (d_alg, d_disc, d_meas) in deltas[k][i]:
                sigma += d_alg + d_disc + d_meas    # Quantale-additive
            assert gap_tau[k][i] > sigma, "Restart inequality failed"
            total[i] += sigma
        if policy.kind == "geometric":
            r = policy.r
            assert 0 < r < 1 and tau[k+1] <= r * tau[k]
        elif policy.kind == "p_series":
            p = policy.p
            assert p > 1
    return {i: (total[i] < float("inf")) for i in degrees}
\end{verbatim}

% -------------------------
\subsection*{D.8. Stability bands, \texorpdfstring{$\tau$}{tau}-sweeps, and detection algorithm}\label{D:subsec:stab-bands}
% -------------------------

\begin{definition}[Stability band via \(\tau\)-sweep]\label{D:def:stab-band}
Fix a window \(W\) and degree \(i\).
Let \(\{\tau_\ell\}_{\ell=1}^L\) be an increasing \(\tau\)-sweep.
A contiguous block \(\{\tau_a,\dots,\tau_b\}\) is a \emph{stability band} if
\[
\mu_{i,\tau_\ell}=\nu_{i,\tau_\ell}=0\quad\text{for all }\ \ell\in\{a,\dots,b\},
\]
and the verdict persists upon \emph{refining} the sweep (inserting new \(\tau\)-values) without introducing \(\mu\) or \(\nu\) in the band.
\end{definition}

\begin{proposition}[Robust detection of stability bands]\label{D:prop:robust-band}
Assume (S1)–(S3) of Theorem~\ref{D:thm:sufficient} hold on \(W\).
Then any sufficiently fine \(\tau\)-sweep admits stability bands covering all \(\tau\) at which \(\phi_{i,\tau}\) is an isomorphism; conversely, detecting a stability band by a sweep and its refinement certifies \(\DiagZero(\tau)\) on the band.
\end{proposition}

\begin{proof}[Proof sketch]
Under (S1)–(S3), \(\phi_{i,\tau}\) is an isomorphism on open neighborhoods of the corresponding \(\tau\)-values.
A fine sweep samples each neighborhood; refinement removes aliasing.
The converse follows by Definition~\ref{D:def:stab-band}.
\end{proof}

\begin{remark}[Caveat: non-monotonicity in \(\tau\)]
There is no general monotonicity of \(\mu_{i,\tau}\) or \(\nu_{i,\tau}\) in \(\tau\).
Stability bands may be separated by isolated \(\tau\)-values where \(\phi_{i,\tau}\) fails to be an isomorphism.
\end{remark}

\paragraph{Detection algorithm (auditable).}
Given a sweep \(\mathsf{TauSweep}=\{\tau_\ell\}_{\ell=1}^L\):
\begin{enumerate}[label=(A\arabic*)]
\item For each \(\ell\), compute \(\phi_{i,\tau_\ell}\) and record \((\mu_{i,\tau_\ell},\nu_{i,\tau_\ell})\).
\item Extract maximal contiguous index ranges \([a,b]\) with \((\mu,\nu)=(0,0)\).
\item Refine by inserting midpoints \(\tau'=\frac12(\tau_\ell+\tau_{\ell+1})\) within each candidate band and recompute \((\mu,\nu)\).
\item Accept a band iff all refined points also yield \((0,0)\).
\item Emit a certificate with hashes of inputs, tower metadata, and flags indicating which of (S1)–(S3) were used.
\end{enumerate}

% -------------------------
\subsection*{D.9. Implementation guide: APIs, stubs, and tests}\label{D:subsec:impl}
% -------------------------

All persistence-layer computations are understood after applying \(\mathbf{T}_\tau\).

\subsubsection*{Lean stubs (illustrative).}
\begin{lstlisting}[language=,caption={Lean 4 stubs for towers and obstruction indices}]
namespace PH
structure Tower (ι : Type) :=
  (F    : ι → FiltCh)
  (apex : FiltCh)
  (toApex : ∀ i : ι, ChainMap (F i) apex)

def P_i (i : ℤ) : FiltCh → Pers := -- assumed given
def T_tau (τ : ℝ) : Pers → Pers := -- exact, 1-Lipschitz

def phi (i : ℤ) (τ : ℝ) (T : Tower ℕ) : PersHom :=
  have src := colim (fun n => T_tau τ (P_i i (T.F n)))
  have tgt := T_tau τ (P_i i T.apex)
  comparison src tgt -- canonical

def gdim (M : Pers) : Nat := -- multiplicity of I[0,∞) in barcode

def mu (i : ℤ) (τ : ℝ) (T : Tower ℕ) : Nat := gdim (kernel (phi i τ T))
def nu (i : ℤ) (τ : ℝ) (T : Tower ℕ) : Nat := gdim (cokernel (phi i τ T))
end PH
\end{lstlisting}

\subsubsection*{Coq stubs (illustrative).}
\begin{lstlisting}[language=,caption={Coq stubs for towers and obstruction indices}]
Module PH.
Record Tower := {
  F      : nat -> FiltCh;
  apex   : FiltCh;
  toApex : forall n, ChainMap (F n) apex
}.

Parameter P_i   : Z -> FiltCh -> Pers.
Parameter T_tau : R -> Pers -> Pers. (* exact, 1-Lipschitz *)

Definition phi (i:Z) (tau:R) (T:Tower) : PersHom :=
  let src := colim (fun n => T_tau tau (P_i i (F T n))) in
  let tgt := T_tau tau (P_i i (apex T)) in
  comparison src tgt.

Parameter gdim : Pers -> nat.

Definition mu (i:Z) (tau:R) (T:Tower) : nat := gdim (kernel (phi i tau T)).
Definition nu (i:Z) (tau:R) (T:Tower) : nat := gdim (cokernel (phi i tau T)).
End PH.
\end{lstlisting}

\paragraph{Sample tests.}
\begin{enumerate}[label=({T}\arabic*)]
\item \textbf{T3 (Filtered-colim stability).} Construct a tower whose apex is the filtered colimit and for which \(\mathbf{T}_\tau\) commutes with colim. Verify \(\phi_{i,\tau}\) is an iso and \(\DiagZero(\tau)\).
\item \textbf{T7 (Toy towers).} Instantiate the pure kernel, pure cokernel, and mixed towers of \S D.2 and confirm \((\mu,\nu)=(1,0),(0,1),(1,1)\), respectively.
\item \textbf{T9 (No \(\tau\)-accumulation).} Create barcodes with an \(\eta\)-gap below \(\tau\); confirm \(\phi_{i,\tau}\) is iso.
\item \textbf{T10 (Cauchy+compatibility).} Build a Cauchy sequence in the bottleneck metric whose limit equals the apex post-\(\mathbf{T}_\tau\); confirm iso.
\item \textbf{T11 (Restart+Summability).} Simulate windows and transitions satisfying Lemma~\ref{D:lem:restart} and Definition~\ref{D:def:summability}; verify global certificate via Theorem~\ref{D:thm:pasting}.
\end{enumerate}

% -------------------------
\subsection*{D.10. Audit schema: \texttt{run.yaml}, JSON, and HDF5 layout}\label{D:subsec:audit}
% -------------------------

\subsubsection*{YAML fields (mandatory).}
\begin{lstlisting}[language=,caption={Minimal audit fields in run.yaml (bands included)}]
phi:
  idx:
    - i: 1
      tau: 0.75
      iso: true          # φ_{i,τ} isomorphism?
      mu: 0
      nu: 0
      flags:
        S1: true         # used commutation+apex-colim
        S2: false
        S3: false
      iso_tail:
        passed: true     # tail check on refined sweep
        refinement_levels: 2
  bands:
    - i: 1
      start_tau: 0.70
      end_tau: 0.82
      certified: true    # (μ,ν)=(0,0) across band
      method: "sweep+refine"
windows:
  collapse:
    tau_sweep: [0.5, 0.6, 0.7, 0.75, 0.8]
persistence:
  phi_iso_tail: "strict"
summability:
  policy: "geometric"
  r: 0.8
  total_delta:
    i=0: 0.137
    i=1: 0.092
tower:
  edges:
    - src: 0; dst: 1; kind: "inclusion"
    - src: 1; dst: 2; kind: "inclusion"
hash:
  inputs: "sha256:..."
  code:   "sha256:..."
\end{lstlisting}

\subsubsection*{JSON snippet (optional).}
\begin{lstlisting}
{
  "phi": {
    "idx": [
      {"i":1, "tau":0.75, "iso":true, "mu":0, "nu":0,
       "flags":{"S1":true,"S2":false,"S3":false},
       "iso_tail":{"passed":true,"refinement_levels":2}}
    ],
    "bands": [
      {"i":1, "start_tau":0.70, "end_tau":0.82,
       "certified":true, "method":"sweep+refine"}
    ]
  },
  "summability":{"policy":"geometric","r":0.8,
                 "total_delta":{"i=0":0.137,"i=1":0.092}}
}
\end{lstlisting}

\paragraph{HDF5 groups (canonical order; bands included).}
We store comparison data under \texttt{/phi} and tower metadata under \texttt{/tower}.
A minimal layout is:

\medskip
\begin{center}
\begin{tabular}{ll}
\toprule
Group/Dataset & Contents \\
\midrule
\texttt{/phi/idx/i} & integer degrees \(i\) \\
\texttt{/phi/idx/tau} & real thresholds \(\tau\) \\
\texttt{/phi/idx/iso} & boolean flags \\
\texttt{/phi/idx/mu} & nonnegative integers \(\mu_{i,\tau}\) \\
\texttt{/phi/idx/nu} & nonnegative integers \(\nu_{i,\tau}\) \\
\texttt{/phi/idx/flags} & bitmask/provenance for (S1,S2,S3) \\
\texttt{/phi/idx/iso\_tail/passed} & boolean \\
\texttt{/phi/bands/i} & integer degrees \(i\) \\
\texttt{/phi/bands/start\_tau} & real \(\tau\)-starts \\
\texttt{/phi/bands/end\_tau} & real \(\tau\)-ends \\
\texttt{/phi/bands/certified} & boolean certification flags \\
\texttt{/phi/bands/method} & string provenance (e.g.\ sweep+refine) \\
\texttt{/tower/edges/src} & integer source indices \\
\texttt{/tower/edges/dst} & integer target indices \\
\texttt{/tower/edges/kind} & categorical: inclusion/deletion/epsilon \\
\bottomrule
\end{tabular}
\end{center}

% -------------------------
\subsection*{D.11. Additional formalities: \texorpdfstring{$\tau$}{tau}-criticality and bandwise certification}\label{D:subsec:critical}
% -------------------------

\begin{proposition}[Piecewise constancy off critical thresholds]\label{D:prop:piecewise}
Fix \(i\) and a tower.
There exists a finite set \(S\subset(0,\infty)\) consisting of relevant bar lengths and their finite sums/differences (within the monitored window/degree budget) such that \(\mu_{i,\tau}\) and \(\nu_{i,\tau}\) are locally constant on each connected component of \((0,\infty)\setminus S\).
\end{proposition}

\begin{proof}[Proof sketch]
In the constructible regime, changes in \(\Ker/\coker\) after \(\mathbf{T}_\tau\) can occur only when \(\tau\) crosses a length at which truncation toggles an interval from “deleted” to “kept” in a kernel/cokernel decomposition.
Fixing a bounded set of degrees and a finite window budget yields a finite critical set.
\end{proof}

\begin{corollary}[Bandwise certification]\label{D:cor:bandwise}
If \(\phi_{i,\tau_0}\) is an isomorphism for some \(\tau_0\) lying in a connected component of \((0,\infty)\setminus S\), then it is an isomorphism on the entire component.
\end{corollary}

% -------------------------
\subsection*{D.12. Completion note and cross-module conventions}
% -------------------------

\begin{remark}[No further supplementation required]\label{D:rem:complete}
This appendix provides:
(i) the definition and calculus of the tower obstruction indices \((\mu,\nu)\) (generic fiber dimensions after truncation), including \(V\)-metric invariance and window finiteness on definable covers;
(ii) naturality of \(\phi\), subadditivity under composition, and additivity under finite direct sums;
(iii) toy towers (pure kernel/cokernel/mixed) and a counterexample showing \(\sum d_{\mathrm{int}}<\infty\) does not force \(\DiagZero(\tau)\);
(iv) sufficient conditions (S1)–(S3) guaranteeing \(\DiagZero(\tau)\);
(v) a Restart/Summability framework (with auditable pseudocode) to paste windowed certificates into global ones;
(vi) a robust \(\tau\)-sweep procedure and stability bands to certify \(\DiagZero(\tau)\) on contiguous \(\tau\)-ranges; and
(vii) implementation-grade audit schemas (YAML/JSON/HDF5), API stubs (Lean/Coq), and test items.
All statements remain within the global guard-rails (constructible \(1\)D persistence over a field; persistence-layer equalities after truncation; f.q.i.\ at the filtered-complex layer).
\end{remark}

\medskip
\noindent\textbf{Cross-module conventions.}
\begin{itemize}\itemsep0.2em
\item Ext-tests (Appendix~C) are always taken against \(k[0]\): \(\Ext^1(\mathcal{R}((C_\tau F)|_W),k[0])\).
\item Update monotonicity follows the global rule (Appendix~E): \emph{deletion-type} updates are non-increasing for windowed energies and spectral tails \emph{after truncation}, whereas \emph{inclusion-type} updates are stable (non-expansive).
\item Type labels follow the global convention \emph{Type~I–II / Type~III / Type~IV}; tower defects \((\mu,\nu)\neq(0,0)\) are \emph{Type~IV} at the declared scale(s) \(\tau\).
\end{itemize}

\endgroup



% =========================
\section*{Appendix E. Spectral Indicators: Monotonicity, Stability, Counterexamples [Proof/Spec] (reinforced)}
% =========================
\addcontentsline{toc}{section}{Appendix E. Spectral Indicators: Monotonicity, Stability, Counterexamples}
\refstepcounter{section} % stabilize cross-references with an unnumbered section

% (Local) listings style for YAML/JSON snippets.
% Requires: xcolor, listings (globally).
\definecolor{AKlightgray}{gray}{0.95}
\lstset{
  basicstyle=\ttfamily\small,
  backgroundcolor=\color{AKlightgray},
  frame=single,
  columns=fullflexible,
  keepspaces=true,
  showstringspaces=false,
  upquote=true,
  breaklines=true
}

\paragraph{Standing conventions (Appendix E).}
Throughout this appendix:
\begin{itemize}[leftmargin=1.25em,itemsep=0.2em]
\item Matrices/operators are Hermitian unless stated otherwise.
Eigenvalues are always listed in \emph{ascending} order:
\(\lambda_1(H)\le\lambda_2(H)\le\cdots\le\lambda_n(H)\).
\item The \emph{upper-tail counting function} is
\[
N_\theta(H)\ :=\ \#\{j:\lambda_j(H)\ge \theta\}\qquad(\theta\in\RR),
\]
with left/right limits \(N_{\theta-0}(H)\), \(N_{\theta+0}(H)\) taken in \(\theta\).
\item All spectral audits are \emph{windowed} and performed \emph{after} collapse on the B-side single layer.
When we speak about operators arising from filtered complexes, we mean:
form \(C_\tau F\) (collapse), realize to a linear operator \(L(C_\tau F)\) under the fixed realization policy (Appendix~G),
and \emph{then} compute spectral indicators on each window.
\item Equalities asserted about persistence objects are strict in \(\Pers^{\mathrm{ft}}_k\); filtered–complex statements are \emph{up to filtered quasi-isomorphism (f.q.i.)}.
Filtered colimits follow the scope rule of Appendix~A, Remark~\ref{A:rk:filtered-colimits}.
\item Cross-module conventions (global):
Ext-tests are taken against \(k[0]\) (Appendix~C), i.e.\ \(\Ext^1(\mathcal{R}(C_\tau F),k)=0\);
energy exponents are uniform \(\alpha>0\) (default \(\alpha=1\));
type labels use \emph{Type~I--II / Type~III / Type~IV}.
\item \(V\)-enrichment is governed by Appendix~A, Remark~\ref{rem:V-enriched}: the \(V\)-metric/aggregation layer does not alter the underlying algebraic backbone.
\end{itemize}

\medskip
For \(\tau>0\), define the \emph{clipped spectrum}
\[
\mathrm{clip}_\tau(H):=(\min\{\lambda_j(H),\tau\})_{j=1}^n,
\]
the \emph{clipped sum}
\[
S^{\le \tau}(H):=\sum_{j=1}^n \min\{\lambda_j(H),\tau\},
\]
and the \emph{sub-threshold deficit}
\[
D^{<\tau}(H)\ :=\ \sum_{j=1}^n (\tau-\lambda_j(H))_+,\qquad x_+:=\max\{x,0\}.
\]
We use the operator norm \(\|\cdot\|_{\mathrm{op}}\) and Frobenius norm \(\|\cdot\|_{\mathrm{fro}}\).

\medskip
\noindent\textbf{Deletion vs.\ inclusion (policy).}
When \(H\) arises by restricting admissible degrees of freedom, imposing Dirichlet constraints, eliminating internal dofs by shorting (Schur complement/Kron reduction), or taking principal submatrices, we call the step \emph{deletion-type}.
When \(H\) is obtained by adding degrees of freedom, couplings, or enlarging a domain, we call it \emph{inclusion-type}.
Deletion-type steps admit one-sided monotonicity (with a specified Loewner orientation);
inclusion-type steps admit only stability (non-expansive) unless additional order hypotheses are imposed.
\emph{All inclusion-type claims are explicitly labeled as “non-expansive only.”}

% -------------------------
\subsection*{E.0. Scope and window policy}
% -------------------------

All spectral audits in this appendix are \emph{windowed} and performed \emph{after} collapse on the B-side single layer.
Comparisons follow the mandatory order:
\[
\boxed{\ \text{for each } t\ \Longrightarrow\ \text{apply } \mathbf{P}_i \ \Longrightarrow\ \text{apply } \mathbf{T}_\tau \ \Longrightarrow\ \text{compare in }\Pers^{\mathrm{ft}}_k\ }.
\]
Spectral indicators are computed only as \emph{auxiliary diagnostics} on \(L(C_\tau F)\) under a fixed policy (Appendix~G); they never replace B--Gate$^{+}$.

When a spectral window \([a,b]\) with bin width \(\beta>0\) is used, bins are \emph{half-open, right-attribution}
\(I_r=[a+r\beta,a+(r+1)\beta)\), and eigenvalues at a right boundary are counted in the next bin.
Underflows/overflows are recorded.
This policy ensures reproducibility and compatibility with Overlap Gate and B--Gate$^{+}$ (Appendix~G; Chapter~1).

% -------------------------
\subsection*{E.1. Deletion-type monotonicity (principal/Dirichlet, Schur complement, Loewner)}
% -------------------------

\begin{proposition}[Principal/Dirichlet restriction: interlacing and tail-count control]\label{E:prop:dirichlet}
Let \(A\in\RR^{n\times n}\) be Hermitian and let \(B\) be a principal \((n-1)\times(n-1)\) submatrix (obtained, e.g., by pinning a coordinate---Dirichlet restriction).
Then Cauchy interlacing holds:
\[
\lambda_1(A)\ \le\ \lambda_1(B)\ \le\ \lambda_2(A)\ \le\ \cdots\ \le\ \lambda_{n-1}(B)\ \le\ \lambda_n(A).
\]
In particular, for every \(\theta\in\RR\),
\[
N_\theta(B)\ \le\ N_\theta(A),
\qquad
N_\theta(B)\ \ge\ \max\{0,\,N_\theta(A)-1\}.
\]
\end{proposition}

\begin{proposition}[Schur complement (shorting) monotonicity]\label{E:prop:schur}
Partition \(M=\begin{psmallmatrix}A & B\\ B^\top & C\end{psmallmatrix}\succeq 0\) with \(C\succ 0\) and form the Schur complement \(S:=A-BC^{-1}B^\top\).
Then \(S\preceq A\).
Consequently, for all \(j\) and all \(\theta\ge 0\),
\[
\lambda_j(S)\ \le\ \lambda_j(A),
\qquad
N_\theta(S)\ \le\ N_\theta(A).
\]
\end{proposition}

\begin{proposition}[Loewner-order monotonicity]\label{E:prop:loewner}
If \(0\preceq A\preceq B\) (Loewner order), then for each \(j\),
\(\lambda_j(A)\le \lambda_j(B)\),
and for every \(\theta\ge 0\), \(N_\theta(A)\le N_\theta(B)\).
\end{proposition}

\begin{remark}[Heat traces and spectral tails]\label{E:rem:heat-tail}
For PSD matrices and \(t>0\), the heat trace \(\mathrm{HT}(t;H)=\sum_j e^{-t\lambda_j(H)}\) satisfies:
if \(A'\preceq A\) (contraction), then \(\mathrm{HT}(t;A')\ge \mathrm{HT}(t;A)\);
if \(A'\succeq A\) (hardening), then \(\mathrm{HT}(t;A')\le \mathrm{HT}(t;A)\).
Likewise, for spectral tails \(\mathrm{ST}_\beta(H)=\sum_{j\ge 1}\lambda_j(H)^{-\beta}\) with \(\beta>0\),
one has \(\mathrm{ST}_\beta(A')\ge \mathrm{ST}_\beta(A)\) under \(A'\preceq A\) and the reverse inequality under \(A'\succeq A\),
provided all \(\lambda_j>0\).
In practice, tails are computed on \(L(C_\tau F)\) with zero modes removed or handled by pseudoinverses; see Appendix~G.
\end{remark}

\begin{corollary}[Conservative averaging]\label{E:cor:avg}
If \(A_1,\dots,A_m\succeq 0\) satisfy \(A_\ell\preceq A\) for all \(\ell\), then for any convex combination \(\bar A:=\sum_\ell w_\ell A_\ell\) with \(w_\ell\ge 0\), \(\sum_\ell w_\ell=1\),
\(\bar A\ \preceq\ A\).
Therefore \(\lambda_j(\bar A)\le \lambda_j(A)\), \(N_\theta(\bar A)\le N_\theta(A)\) for \(\theta\ge 0\), and the heat trace/tail inequalities of Remark~\ref{E:rem:heat-tail} apply.
\end{corollary}

\begin{remark}[Orientation for deletions]\label{E:rem:orientation}
Two Loewner orientations occur in practice.
\emph{Contractions} (e.g.\ Schur complements, Kron reduction) produce \(A'\preceq A\);
\emph{hardening} operations (e.g.\ some PDE Dirichlet comparisons across different media) may yield \(A'\succeq A\).
Deletion-type monotonicities in this appendix are always stated with the relevant orientation explicitly indicated.
\end{remark}

% -------------------------
\subsection*{E.2. Inclusion-type counterexamples}
% -------------------------

Deletion-type monotonicity does \emph{not} extend naively to inclusion-type operations without additional order hypotheses.

\begin{example}[Neumann/domain inclusion reverses direction]\label{E:ex:neumann}
For the Neumann Laplacian on an interval, enlarging the domain decreases the nonzero eigenvalues:
on \([0,L]\), the first nonzero Neumann eigenvalue is \((\pi/L)^2\), so passing \(L:1\to 2\) reduces it from \(\pi^2\) to \((\pi/2)^2\).
Thus any “inclusion \(\Rightarrow\) increase” heuristic fails under Neumann-type constraints.
\end{example}

\begin{example}[Indefinite coupling can move eigenvalues both ways]\label{E:ex:indef}
Let \(A=I_2=\mathrm{diag}(1,1)\) and
\(B=\begin{psmallmatrix}1 & M\\ M & 1\end{psmallmatrix}\) with \(M>1\).
Then \(B\) has eigenvalues \(1-M\) and \(1+M\), so for \(\theta=0\),
\(N_\theta(B)=1<N_\theta(A)=2\), while the top eigenvalue \(\lambda_2\) increases.
Without a Loewner relation (\(B-A\) indefinite), no monotone law survives.
\end{example}

\begin{example}[Principal extension lacks a fixed direction]\label{E:ex:principal-extend}
Let \(B=[0]\) (eigenvalue \(0\)) and
\(A=\begin{psmallmatrix}0 & t\\ t & 0\end{psmallmatrix}\) with \(t\neq 0\).
Going from \(B\) to \(A\) (adding one dof and a coupling) produces eigenvalues \(-|t|\) and \(|t|\):
the maximum increases to \(|t|\), but the minimum decreases to \(-|t|\).
Hence no uniform increase/decrease holds under inclusion.
\end{example}

These examples justify restricting monotone claims to the deletion/Loewner settings formalized in \S E.1.

% -------------------------
\subsection*{E.3. Continuity, stability, and truncated functionals (with $V$-metric lift)}
% -------------------------

\begin{proposition}[Weyl and Hoffman--Wielandt]\label{E:prop:weyl}
For Hermitian \(A,B\in\RR^{n\times n}\),
\[
\max_{1\le j\le n}\,|\lambda_j(A)-\lambda_j(B)|\ \le\ \|A-B\|_{\mathrm{op}},\qquad
\Big(\sum_{j=1}^n |\lambda_j(A)-\lambda_j(B)|^2\Big)^{1/2}\ \le\ \|A-B\|_{\mathrm{fro}}.
\]
Hence \(A\mapsto(\lambda_1(A),\dots,\lambda_n(A))\) is \(1\)-Lipschitz from \((\|\cdot\|_{\mathrm{op}})\) into \((\RR^n,\|\cdot\|_\infty)\).
\end{proposition}

\begin{corollary}[Lipschitz stability of clipped spectra]\label{E:cor:clip}
For any \(\tau>0\) and Hermitian \(A,B\),
\[
\sum_{j=1}^n\Big|\min\{\lambda_j(A),\tau\}-\min\{\lambda_j(B),\tau\}\Big|
\ \le\ \sum_{j=1}^n|\lambda_j(A)-\lambda_j(B)|
\ \le\ \sqrt{n}\,\|A-B\|_{\mathrm{fro}}
\ \le\ n\,\|A-B\|_{\mathrm{op}}.
\]
Consequently, \(S^{\le \tau}\) is \(\sqrt{n}\)-Lipschitz in \(\|\cdot\|_{\mathrm{fro}}\) and \(n\)-Lipschitz in \(\|\cdot\|_{\mathrm{op}}\).
\end{corollary}

\begin{proposition}[Semicontinuity of counting indicators]\label{E:prop:count-semi}
If \(A_m\to A\) in operator norm and \(\theta\) is not an eigenvalue of \(A\), then \(N_\theta(A_m)=N_\theta(A)\) for all large \(m\) (local constancy).
In general,
\[
\limsup_{m\to\infty} N_\theta(A_m)\ \le\ N_{\theta-0}(A),\qquad
\liminf_{m\to\infty} N_\theta(A_m)\ \ge\ N_{\theta+0}(A).
\]
\end{proposition}

\begin{proposition}[Truncated functionals: monotonicity and stability]\label{E:prop:trunc-func}
Fix \(\tau>0\).
For an \(n\times n\) positive semidefinite (PSD) matrix \(A\), set
\[
S^{\le\tau}(A):=\sum_{j=1}^n \min\{\lambda_j(A),\tau\},\qquad
D^{<\tau}(A):=\sum_{j=1}^n (\tau-\lambda_j(A))_{+}.
\]
Then:
\begin{enumerate}[leftmargin=1.15em,label=(\arabic*)]
\item \emph{Deletion; Loewner contraction \(A'\preceq A\).}
For all \(j\), \(\lambda_j(A')\le \lambda_j(A)\), hence \(N_\theta(A')\le N_\theta(A)\) for every \(\theta\ge 0\), and
\[
S^{\le\tau}(A')\le S^{\le\tau}(A),\qquad D^{<\tau}(A')\ge D^{<\tau}(A).
\]
\item \emph{Deletion; Loewner hardening \(A'\succeq A\).}
All inequalities in \emph{(1)} reverse:
\[
\lambda_j(A')\ge \lambda_j(A),\quad N_\theta(A')\ge N_\theta(A)\ (\theta\ge 0),\quad
S^{\le\tau}(A')\ge S^{\le\tau}(A),\quad D^{<\tau}(A')\le D^{<\tau}(A).
\]
\item \emph{Lipschitz stability.} For any Hermitian \(A,B\),
\[
\bigl|D^{<\tau}(A)-D^{<\tau}(B)\bigr|
\ \le \sum_{j=1}^n \bigl|\lambda_j(A)-\lambda_j(B)\bigr|
\ \le \sqrt{n}\,\|A-B\|_{\mathrm{fro}}
\ \le n\,\|A-B\|_{\mathrm{op}}.
\]
\end{enumerate}
\end{proposition}

\begin{theorem}[Deletion-type monotonicity and $V$-Lipschitz after collapse]\label{thm:spec-v-lip}
Let \(V\) be a commutative quantale and endow the per-window spectral dashboard
(counts \(N_\theta\), clipped sums \(S^{\le\tau}\), deficits \(D^{<\tau}\), heat traces, tails, cumulative profiles \(C_r\))
with a Lawvere \(V\)-metric aggregator (Appendix~A, Remark~\ref{rem:V-enriched}).
Then, measured \emph{after} applying \(C_\tau\) and evaluating under the fixed realization policy (Appendix~G):
\begin{enumerate}[leftmargin=1.15em,label=(\alph*)]
\item For every deletion-type update \(U\), the dashboard is \emph{componentwise monotone} in the appropriate Loewner orientation as in Propositions~\ref{E:prop:dirichlet}--\ref{E:prop:trunc-func} and Remark~\ref{E:rem:heat-tail}.
\item For every \(\varepsilon\)-continuation with \(\|A'-A\|_{\mathrm{op}}\le\varepsilon\), each scalar component changes by at most its explicit Lipschitz bound (Weyl/Hoffman--Wielandt and Corollary~\ref{E:cor:clip}), and any \(V\)-aggregation preserves the same (non-expansive) estimate.
\end{enumerate}
Spectral indicators are \emph{auxiliary}: they never serve as sole gate criteria.
\end{theorem}

\begin{remark}[Heat traces and spectral tails: stability]\label{E:rem:heat-tail-stab}
Let \(\mathrm{HT}(t;H)=\sum_j e^{-t\lambda_j(H)}\), \(\mathrm{ST}_\beta(H)=\sum_j \lambda_j(H)^{-\beta}\) with zero modes removed.
If \(\|A-B\|_{\mathrm{op}}\le\varepsilon\), then on windows with \(\lambda_j\ge \lambda_{\min}^+>0\),
both \(\mathrm{HT}(t;\cdot)\) and \(\mathrm{ST}_\beta(\cdot)\) admit Lipschitz-type bounds in terms of \(\sum_j|\lambda_j(A)-\lambda_j(B)|\),
hence in terms of \(\|\cdot\|_{\mathrm{fro}}\) (and \(\|\cdot\|_{\mathrm{op}}\) with an \(n\) factor) by Proposition~\ref{E:prop:weyl}.
In practice, tails/heat traces are evaluated on \(L(C_\tau F)\) under a fixed normalization (Appendix~G).
\end{remark}

% -------------------------
% New: Safe low-pass post-processing + 1-Lipschitz verification
% -------------------------
\subsection*{E.3.5. Safe low-pass post-processing and 1-Lipschitz verification (final)}

We record two safe low-pass mechanisms used as \emph{post}--collapse auxiliaries.
Both are \emph{non-expansive} under the stated conditions and compatible with the dashboard above.
They never replace B--Gate$^{+}$.

\paragraph{(A) Temporal/index low-pass on a sequence \(\{A_j\}_j\) of PSD operators.}
Fix a finitely supported kernel \(h^{(\tau)}:\ZZ\to[0,1]\) with:
\[
\begin{aligned}
\text{\textbf{(LP1) evenness:}} &\quad h^{(\tau)}[r]=h^{(\tau)}[-r],\\
\text{\textbf{(LP2) unit mass:}} &\quad \sum_{r\in\ZZ} h^{(\tau)}[r]=1,\\
\text{\textbf{(LP3) scale:}} &\quad \mathrm{supp}\!\big(h^{(\tau)}\big)\subseteq[-R(\tau),R(\tau)].
\end{aligned}
\]
Define the smoothed sequence \(\widetilde A_j:=\sum_{r} h^{(\tau)}[r]\;A_{j-r}\) (finite sum).
Then:
\begin{enumerate}[leftmargin=1.15em,label=(T\arabic*)]
\item \emph{PSD and Loewner convexity.} \(\widetilde A_j\succeq 0\) and, if \(A_{j-r}\preceq A^\star\) for all \(r\), then \(\widetilde A_j\preceq A^\star\) (Corollary~\ref{E:cor:avg}).
\item \emph{Non-expansive in \(\|\cdot\|_{\mathrm{op}}\).} For any sequences \(\{A_j\},\{B_j\}\),
\[
\|\widetilde A_j-\widetilde B_j\|_{\mathrm{op}}
\le \sum_r h^{(\tau)}[r]\;\|A_{j-r}-B_{j-r}\|_{\mathrm{op}}
\le \sup_m\|A_m-B_m\|_{\mathrm{op}}.
\]
Thus the temporal low-pass is \(1\)-Lipschitz in \(\ell_\infty\)-in-\(j\) operator norm.
\item \emph{Dashboard compatibility.} Apply the dashboard to each \(\widetilde A_j\) after \(C_\tau\).
Deletion-type monotonicity is preserved when the original updates are deletion-type and the kernel averages only within the same update class; otherwise only non-expansiveness is claimed.
\end{enumerate}

\paragraph{(B) Spectral low-pass via functional calculus (filter scale \(\sigma\), not the collapse threshold).}
Let \(A\succeq 0\) and choose one of the \emph{safe families} (parameter \(\sigma>0\)):

\smallskip
\(\bullet\) \emph{Heat filter: } \(f_\sigma(\lambda):=e^{-\sigma\lambda}\).
Then
\[
f_\sigma(A)-f_\sigma(B)=-\!\int_0^\sigma e^{-(\sigma-s)A}\,(A-B)\,e^{-sB}\,ds,
\quad\Rightarrow\quad
\|f_\sigma(A)-f_\sigma(B)\|_{\mathrm{op}}\le \sigma\|A-B\|_{\mathrm{op}}.
\]
Thus \(\sigma\le 1\Rightarrow\) operator \(1\)-Lipschitz.

\smallskip
\(\bullet\) \emph{Resolvent filter: } \(r_\sigma(\lambda):=(1+\lambda/\sigma)^{-1}\).
By the resolvent identity,
\[
r_\sigma(A)-r_\sigma(B)=(I+A/\sigma)^{-1}\,\frac{B-A}{\sigma}\,(I+B/\sigma)^{-1},
\]
and \(\|(I+A/\sigma)^{-1}\|_{\mathrm{op}},\|(I+B/\sigma)^{-1}\|_{\mathrm{op}}\le 1\).
Hence
\[
\|r_\sigma(A)-r_\sigma(B)\|_{\mathrm{op}}\le \frac{1}{\sigma}\|A-B\|_{\mathrm{op}},
\]
so \(\sigma\ge 1\Rightarrow\) operator \(1\)-Lipschitz.

\smallskip
Both filters satisfy \(0\le f_\sigma(\lambda),r_\sigma(\lambda)\le 1\) and \(f_\sigma(0)=r_\sigma(0)=1\) (unit mass at DC), and they are operator-monotone decreasing in \(\lambda\).
When applied to \(L(C_\tau F)\), the resulting operators keep the dashboard non-expansive; if a deletion update satisfies a Loewner orientation, the filtered indicators inherit the same orientation.

\paragraph{(C) 1-Lipschitz verification checklist (runtime, per window).}
\begin{enumerate}[leftmargin=1.15em,label=(V\arabic*)]
\item \emph{Temporal low-pass:} check (LP1)--(LP3) and \(\sum h=1\). Report \texttt{lp.kind="temporal"}, \texttt{lp.mass=1.0}, \texttt{lp.support}, and \texttt{lp.even=true}. Non-expansiveness holds with constant \(1\).
\item \emph{Heat filter:} record \(\sigma\) and ensure \(\sigma\le 1\) for \(1\)-Lipschitz; otherwise log \texttt{lip\_const=\(\sigma\)} and use it in the \(V\)-aggregator.
\item \emph{Resolvent filter:} ensure \(\sigma\ge 1\) for \(1\)-Lipschitz; otherwise log \texttt{lip\_const=\(1/\sigma\)}.
\item \emph{Inclusion-type steps:} tag \texttt{inclusion="non-expansive-only"}; no monotone claim is emitted.
\end{enumerate}

\subsection*{(D) Manifest snippet (auditable).}
\begin{lstlisting}[language=,caption={run.yaml (spectral low-pass metadata)}]
spectral_post:
  lowpass:
    mode: "temporal"            # or "heat" | "resolvent"
    kernel:
      even: true
      mass: 1.0
      support: [-2, 2]          # indices r with h[r] ≠ 0
      taps: [0.1, 0.2, 0.4, 0.2, 0.1]
    lipschitz:
      constant: 1.0
      verified: true
    inclusion_policy: "non-expansive-only"
# If mode is "heat":
#   param_sigma: 0.75
#   lipschitz.constant: 0.75
# If mode is "resolvent":
#   param_sigma: 2.0
#   lipschitz.constant: 0.5
\end{lstlisting}

% -------------------------
\subsection*{E.4. Auxiliary spectral bars (aux-bars): definition, stability, and policy [Spec]}
% -------------------------

We formalize \emph{auxiliary spectral bars} as diagnostics alongside persistence.
They never replace B--Gate$^{+}$ and are used only as auxiliary evidence.

\paragraph{E.4.1. Binning and endpoint convention.}
Fix a spectral window \([a,b]\) and a bin width \(\beta>0\).
Let \(R:=\lfloor (b-a)/\beta\rfloor\).
Define half-open, right-attribution bins
\(I_r=[a+r\beta,a+(r+1)\beta)\)
for \(r=0,1,\dots,R-1\).
An eigenvalue at the bin’s right boundary is counted in the next bin.
Record underflow \(U(H):=\#\{j:\lambda_j(H)<a\}\) and overflow \(O(H):=\#\{j:\lambda_j(H)\ge b\}\).
For a Hermitian \(H\), define the bin occupancy
\[
E_r(H):=\#\{j:\lambda_j(H)\in I_r\},
\]
and the cumulative (upper-tail) profile
\[
C_r(H):=\sum_{s=r}^{R-1}E_s(H)=N_{a+r\beta}(H)-O(H).
\]

\paragraph{E.4.2. Aux-bars across an index (time/tower).}
Let \((H_j)_{j\in J}\) be a sequence (time or tower).
For fixed \(r\), the set \(\{j:E_r(H_j)>0\}\) decomposes into maximal consecutive runs \(J_{r,\ell}\).
Each run \(J_{r,\ell}\) defines an aux-bar \((r,J_{r,\ell})\) with lifetime \(|J_{r,\ell}|\) (or a rescaled duration).
We log \(\mathrm{aux\text{-}count}=\sum_{r,\ell}1\), \(\mathrm{aux\text{-}mass}=\sum_r E_r(H_j)\) (per index \(j\)), and \(\mathrm{active\ bins}=\#\{r:E_r(H_j)>0\}\).

\paragraph{E.4.3. Monotonicity/stability.}
\begin{proposition}[Cumulative-profile monotonicity under deletion/Loewner]\label{E:prop:aux-cum-mono}
If \(A'\preceq A\) (PSD contraction) or \(A'\) is a principal/Dirichlet restriction of \(A\), then for every \(r\),
\[
C_r(A')\le C_r(A).
\]
\end{proposition}

\begin{proposition}[Cumulative-profile stability]\label{E:prop:aux-cum-stab}
If \(\|A-B\|_{\mathrm{op}}\le \varepsilon\) and \(q:=\lceil \varepsilon/\beta\rceil\), then for all \(r\),
\[
C_{r+q}(B)\le C_r(A)\le C_{\max\{0,r-q\}}(B).
\]
In particular, if \(\varepsilon<\beta\), the cumulative profile can shift by at most one bin.
\end{proposition}

\begin{corollary}[Definable windows: finiteness and piecewise constancy]\label{E:cor:definable}
On an \(o\hbox{-}minimal\) definable spectral window with finite Leray/\v{C}ech depth (Appendix~H/J),
the sequences \(r\mapsto E_r(H_j)\) and \(r\mapsto C_r(H_j)\) admit only finitely many bin-transition events per window and are piecewise constant in \(j\).
Aux-bars are therefore finite in number per window and auditable.
\end{corollary}

\begin{remark}[Policy]\label{E:rem:aux-policy}
\begin{itemize}[leftmargin=1.25em,itemsep=0.2em]
\item Deletion-type steps: enforce monotonicity on the \emph{cumulative} profile \(C_r\). Per-bin occupancies and lifetimes are diagnostics only.
\item \(\varepsilon\)-continuations: with \(\|A_{j+1}-A_j\|_{\mathrm{op}}\le \varepsilon\), declare stability up to \(\pm q=\lceil \varepsilon/\beta\rceil\) bin shifts; record \texttt{eps\_cont\_shift\_bins} in the manifest.
\item Inclusion-type steps: claim no monotonicity; only stability bounds are used (\emph{non-expansive only}).
\item Under/overflow must be logged. Optional conservative rules (policy-only): \(O=0\), \(C_{R-1}=0\).
\end{itemize}
\end{remark}

\paragraph{E.4.4. Reproducibility fields.}
The run manifest \texttt{run.yaml} should include (Appendix~G):
\begin{itemize}[leftmargin=1.25em]
  \item \texttt{spectral.range} \([a,b]\), \texttt{bin\_width} \(\beta\), \texttt{bins} \(R\), endpoint policy \texttt{half-open/right-attribution};
  \item \texttt{underflow}/\texttt{overflow} per index \(j\);
  \item \texttt{cum\_profile}: the sequence \(C_r(H_j)\) per \(j\);
  \item \texttt{aux\_bars} (optional): list of runs \((r,J_{r,\ell})\) with lifetimes;
  \item \texttt{eps\_cont\_bound}: \(\varepsilon\) and derived \texttt{eps\_cont\_shift\_bins} \(= \lceil \varepsilon/\beta\rceil\);
  \item \texttt{spectral\_policy.order}: \texttt{"ascending"}; \texttt{spectral\_policy.norm}: \texttt{"op"} or \texttt{"fro"};
        bounds \texttt{lambda\_min}, \texttt{lambda\_max}, optional \texttt{lip\_tol}.
\end{itemize}

% -------------------------
\subsection*{E.5. Implementation and reproducibility: JSON/HDF5 schemas}
% -------------------------

\subsection*{JSON layout (mandatory fields).}
\begin{lstlisting}[language=,caption={Minimal spec.json layout (one run, multiple operators)}]
{
  "meta": {
    "schema_version": "2025-03-15",
    "eigen_units": "dimensionless",
    "order": "ascending",
    "sorted": true,
    "norm": "op",               // or "fro"
    "Ntheta_convention": { "left": "N_{θ-0}", "right": "N_{θ+0}" },
    "window": { "range": [0.0, 2.0], "semantics": "closed" },
    "clip_tau": 1.0,
    "tol_eig": 1e-8,
    "aux_policy": { "bin": 0.02, "right_attribution": true },
    "coverage_check": { "thetas_in_window": true },
    "links": { "run_id": "...", "run_yaml_hash": "sha256:..." }
  },
  "operators": [
    {
      "id": "sha256:...A",
      "kind": "laplacian_dirichlet",
      "n": 500,
      "spectrum": { "eigs": [0.10, 0.12, 0.45, ...] },  // ascending
      "clip":     { "tau": 1.0, "sum": 37.219, "deficit": 12.004 },
      "underflow": 0, "overflow": 0
    },
    {
      "id": "sha256:...B",
      "kind": "principal_submatrix",
      "parent": "sha256:...A",
      "N_theta": [
        { "theta": 0.20, "left": 17, "right": 16 },
        { "theta": 0.50, "left": 10, "right": 10 }
      ],
      "cum_profile": [ 13, 11, 8, 2, 0 ],
      "underflow": 2, "overflow": 0,
      "monotonicity": { "type": "deletion", "passed": true }
    }
  ],
  "lowpass": {
    "mode": "heat",           // "temporal" | "resolvent"
    "param_sigma": 0.75,
    "lipschitz_constant": 0.75,
    "verified": true,
    "inclusion_policy": "non-expansive-only"
  },
  "hash": "sha256:...spec"
}
\end{lstlisting}

\paragraph{HDF5 layout (canonical).}
\begin{itemize}[leftmargin=1.25em]
  \item Datasets:
  \texttt{/spec/ops/\{id\}/eig} (float64 ascending),
  \texttt{/spec/ops/\{id\}/clip/sum} (float64),
  \texttt{/spec/ops/\{id\}/clip/deficit} (float64),
  \texttt{/spec/ops/\{id\}/Ntheta/theta,left,right} (parallel arrays),
  \texttt{/spec/ops/\{id\}/cum\_profile} (int32),
  \texttt{/spec/ops/\{id\}/underflow} (int32),
  \texttt{/spec/ops/\{id\}/overflow} (int32),
  \texttt{/spec/lowpass/mode} (fixed string),
  \texttt{/spec/lowpass/param\_sigma} (float64),
  \texttt{/spec/lowpass/lipschitz\_constant} (float64),
  \texttt{/spec/lowpass/verified} (bool).
  \item Attributes: \texttt{order="ascending"}, \texttt{norm="op"|"fro"}, \texttt{eigen\_units}, \texttt{tol\_eig}, \texttt{schema\_version}, bin policy, and canonical HDF5 flags
  (\texttt{track\_times=false}, UTF-8 fixed strings, fixed chunking); see Appendix~G.
\end{itemize}

% -------------------------
\subsection*{E.6. Tests and operational checklist}
% -------------------------

\paragraph{Core tests.}
\begin{enumerate}[leftmargin=1.25em]
  \item \textbf{T2 (Deletion-type monotonicity).}
  For principal/Dirichlet or Schur complements, verify \(N_\theta\), \(S^{\le\tau}\), \(D^{<\tau}\), heat trace, and tails satisfy the monotone direction consistent with the Loewner orientation.
  Log \texttt{monotonicity.passed=true}.
  \item \textbf{T1 (\(\varepsilon\)-continuation stability).}
  Given \(\|A_{j+1}-A_j\|_{\mathrm{op}}\le \varepsilon\), validate the bin-shift bounds in Proposition~\ref{E:prop:aux-cum-stab} and Lipschitz bounds for \(S^{\le\tau}\), \(D^{<\tau}\).
  \item \textbf{T9 (Coverage).}
  Confirm that all \(\theta\)-queries, bin windows, and eigenvalues fall within declared windows; if not, log underflow/overflow and set \texttt{coverage\_check.*=false} with a justification.
  \item \textbf{T12 (Low-pass safety).}
  For temporal low-pass, check (LP1)--(LP3) and \(\sum h=1\);
  for heat/resolvent, assert the scale choices yield operator Lipschitz constant \(\le 1\) (or record the constant if \(>1\)) and verify against \(\|A-B\|_{\mathrm{op}}\).
\end{enumerate}

\paragraph{Operational checklist (per window).}
\begin{itemize}[leftmargin=1.25em]
  \item Fix spectral policy: \texttt{order="ascending"}, \texttt{norm} selection, bin width \(\beta\), spectral window \([a,b]\).
  \item Compute spectra on \(L(C_\tau F)\) and clip at the same \(\tau\) used by persistence.
  \item Optionally apply a \emph{safe} low-pass (temporal, heat, resolvent) with recorded scale \(\sigma\) and verified Lipschitz constant.
  \item Log underflow/overflow, cumulative profiles \(C_r\), optional aux-bars with lifetimes (diagnostic).
  \item For deletion-type steps, assert \(C_r\) monotonicity;
        for \(\varepsilon\)-continuations, assert bin-shift stability with \(\lceil \varepsilon/\beta\rceil\);
        for inclusion-type steps, \emph{non-expansive only}.
\end{itemize}

% -------------------------
\subsection*{E.7. Completion note}
% -------------------------

\begin{remark}[No further supplementation required]
This appendix provides a complete, IMRN/AiM-ready treatment of spectral indicators consistent with the v16.0 guard-rails:
(i) deletion-type monotonicities (principal/Dirichlet, Schur, Loewner) for \(N_\theta\), \(S^{\le\tau}\), \(D^{<\tau}\), heat traces, and tails;
(ii) inclusion-type counterexamples;
(iii) Lipschitz stability via Weyl/Hoffman--Wielandt and induced bounds for clipped sums/deficits, heat traces, and tails;
(iv) a windowed, half-open binning policy with cumulative-profile monotonicity and stability under \(\varepsilon\)-continuations;
(v) $V$-metric reinforcement (Theorem~\ref{thm:spec-v-lip}) ensuring deletion-type monotonicity and $V$-stable aggregation after collapse;
(vi) \textbf{safe low-pass post-processing} with even-kernel/unit-mass/\(\tau\)-scale (temporal) and operator-Lipschitz filters (heat/resolvent) together with an auditable Lipschitz verification checklist; and
(vii) reproducibility and canonical schemas (JSON/HDF5) with a minimal test suite (T1/T2/T9/T12).
All claims are made after collapse on the B-side single layer, per-window, and integrate with \(\delta\)-ledger accounting, Overlap Gate, and B--Gate$^{+}$ elsewhere in the manuscript.
No further supplementation is required for operational deployment or audit.
\end{remark}



% =========================
\section*{Appendix F. Formalization Sketch (Lean/Coq) [Spec] (reinforced)}
% =========================
\addcontentsline{toc}{section}{Appendix F. Formalization Sketch (Lean/Coq)}
\refstepcounter{section} % stabilize cross-references with an unnumbered section

% (Local) listings style for YAML/JSON and code stubs.
% Requires: xcolor, listings (globally).
\definecolor{AKFgray}{gray}{0.95}
\lstset{
  basicstyle=\ttfamily\small,
  backgroundcolor=\color{AKFgray},
  frame=single,
  columns=fullflexible,
  keepspaces=true,
  showstringspaces=false,
  upquote=true,
  breaklines=true
}

% ---- Interface stub (must appear at the very beginning) ----
\begin{declaration}[Lean/Coq stubs]\label{stub:formal}
Expose minimal, reusable interfaces:
\texttt{pers\_Ttau\_exact}, \texttt{pers\_Ttau\_lipschitz}, \texttt{Ctau\_lift}, \texttt{Ctau\_colim}, \texttt{Ctau\_pullback},
\texttt{mu\_nu\_vanish}, \texttt{PH1\_to\_Ext1\_under\_B}, \texttt{delta\_pipeline\_additivity},
enriched metrics \texttt{V\_metric\_shift}, and the canonical normal form \texttt{cnf\_after\_Ttau}
(interval/Smith-type decomposition for kernels/cokernels).
Equalities are confined to the persistence layer; filtered-level facts are packaged ``up to f.q.i.''.
AWFS stubs, \(o\hbox{-}minimal\) \v{C}ech, and Iwasawa-style control catalysts are provided as portable axioms with clear call sites.
\end{declaration}

\paragraph{Purpose (Appendix F).}
This appendix provides a fully integrated, implementation-oriented \emph{Spec} for mechanizing the core claims of
Appendices~A--E in Lean/Coq with a module decomposition tailored for a minimal, portable ``mini-library.''
The categorical spine consists of the Serre localization and the reflector \(\mathbf{T}_\tau\) (exact, idempotent),
its \(1\)-Lipschitz property on barcodes (interleaving metric),
tower diagnostics \((\mu,\nu)\) via the comparison map \(\phi_{i,\tau}\),
and the one-way bridge \(\mathrm{PH}_1\Rightarrow\Ext^1\) under the amplitude \(\le 1\) realization policy.
We work in the constructible (p.f.d.) range and adhere to the filtered-colimit scope rule
(Appendix~A, Remark~\ref{A:rk:filtered-colimits}).
Cross-module conventions: \(\Ext\)-tests are always against \(k[0]\) (Appendix~C), i.e.\ \(\Ext^1(\mathcal{R}(C_\tau F),k)=0\);
the energy exponent is globally \(\alpha>0\) (default \(\alpha=1\));
type labels use \emph{Type~I--II / Type~III / Type~IV} for tower diagnostics.
Spectral monotonicity is invoked only for deletion-type operations; inclusion-type operations are used solely with stability bounds (Appendix~E).

\paragraph{Refereeing style (IMRN/AiM).}
Statements are modular with explicit hypotheses and reusable APIs.
All constructions remain within abelian categories, exact localizations, and derived categories with bounded \(t\)-structures.
Proof obligations used in the code stubs are isolated and cited to Appendices~A--E; replacing \texttt{admit}/\texttt{Axiom} by library lemmas yields a fully checked artifact.

% -------------------------
\subsection*{F.0. Reading guide and module map}
% -------------------------

We split the development into five modules and three thin catalysts:

\begin{itemize}[leftmargin=1.25em]
  \item \textbf{AK.Core} (F.1--F.6, F.11): \(\Pers^{\mathrm{ft}}_k\), the Serre subcategory \(\mathsf{E}_\tau\), the reflector \(\mathbf{T}_\tau\) (exact, idempotent), interleaving/shift calculus and \(1\)-Lipschitz, the collapse on filtered complexes \(C_\tau\), canonical normal form (CNF) after \(\mathbf{T}_\tau\), and invariance under filtered quasi-isomorphisms.
  \item \textbf{AK.LocalEquiv} (F.7): Window-local bridge \(\mathrm{PH}\leftrightarrow \Ext\) under amplitude \(\le 1\), saturation/no-accumulation triggers, and tail identification for \(\phi_{i,\tau}\).
  \item \textbf{AK.Tower} (F.8): Comparison map \(\phi_{i,\tau}\), diagnostics \((\mu,\nu)\), functoriality, stability bands, direct sums, compositions, and cofinal invariance.
  \item \textbf{AK.Gluing} (F.9, F.18): Overlap Gate (collapse-compatibility, A/B checks, \v{C}ech--\(\Ext^1\), stability bands) and MECE windowing glue to a global verdict, including Restart/Summability via a DP-style manager.
  \item \textbf{AK.Spectral} (F.10): Spectral auxiliaries (Appendix~E) used after collapse, with deletion-type monotonicity only; inclusion-type steps are \emph{non-expansive only}.
  \item \textbf{Catalysts (thin):} \(V\)-enriched metric/shift (F.A), AWFS skeleton (F.B) for functorial factorization used by \(C_\tau\), and \(o\hbox{-}minimal\) \v{C}ech + Iwasawa-style control (F.C) to streamline Overlap Gate.
\end{itemize}

Test fixtures include T7 (saturation gate), T10 (A/B overlap), T13 (\(\delta\)-budget and restart chain).

% -------------------------
\subsection*{F.1. Environment and objects [Spec] (AK.Core)}
% -------------------------

Fix a field \(k\).
Let \(\mathsf{Vect}_k\) be the abelian category of finite-dimensional \(k\)-vector spaces and
\([\mathbb{R},\mathsf{Vect}_k]\) the functor category (index \((\mathbb{R},\le)\)).
Let \(\Pers^{\mathrm{ft}}_k\subset[\mathbb{R},\mathsf{Vect}_k]\) be the full subcategory of constructible persistence modules.

Let \(\mathsf{FiltCh}(k)\) be filtered chain complexes of finite-dimensional \(k\)-spaces, bounded in homological degree,
with filtration-preserving maps. For \(i\in\mathbb{Z}\) write
\(\mathbf{P}_i:\mathsf{FiltCh}(k)\to\Pers^{\mathrm{ft}}_k\)
for the degree-\(i\) persistence functor.
The bar-deletion reflector \(\mathbf{T}_\tau:\Pers^{\mathrm{ft}}_k\to\Pers^{\mathrm{ft}}_k\)
is recalled from Appendix~A.

\begin{remark}[Generic fiber dimension and stabilization]\label{F:rk:generic-fiber}
We adopt Appendix~D, Remark~\ref{D:rem:generic-dim}.
For \(M\in\Pers^{\mathrm{ft}}_k\), the generic fiber dimension is the multiplicity of the infinite interval \(I[0,\infty)\)
in the barcode of \(M\); equivalently,
\[
\mathrm{gdim}(M)\ =\ \lim_{t\to +\infty}\dim_k M(t),
\]
which stabilizes in the constructible range.
After applying \(\mathbf{T}_\tau\), kernels and cokernels again lie in \(\Pers^{\mathrm{ft}}_k\), and \(\mathrm{gdim}\)
is computed there.
\end{remark}

\begin{specification}[Stabilization lemma for constructible modules]\label{F:spec:stabilize}
If \(M\in\Pers^{\mathrm{ft}}_k\), then there exist \(T_0\in\mathbb{R}\) and \(c\in\mathbb{N}\) such that
\(\dim_k M(t)=c=\mathrm{gdim}(M)\) for all \(t\ge T_0\).
\emph{Use:} define \(\mathrm{gdim}(M)\) by this stabilized value \(c\) and prove iso-invariance of \(\mathrm{gdim}\).
\end{specification}

\begin{remark}[Tower verdict normal form]\label{F:rem:diagzero}
We fix the diagnostic normal form:
\[
\mathrm{DiagZero}\quad:\Longleftrightarrow\quad (\mu,\nu)=(0,0),
\]
with \(\mu,\nu\) defined from \(\phi_{i,\tau}\) in \S F.8.
Type~IV corresponds to \((\mu,\nu)\neq(0,0)\) (Appendix~D).
\end{remark}

% -------------------------
\subsection*{F.2. Serre subcategory and localization [Spec] (AK.Core)}
% -------------------------

Let \(\mathsf{E}_\tau\subset \Pers^{\mathrm{ft}}_k\) be the Serre subcategory generated by intervals of length \(\le\tau\).
By Appendix~A, \(\mathsf{E}_\tau\) is hereditary Serre and the inclusion
\(\iota_\tau:\mathsf{E}_\tau^\perp\hookrightarrow \Pers^{\mathrm{ft}}_k\) admits an exact left adjoint
\(\mathbf{T}_\tau:\Pers^{\mathrm{ft}}_k\to \mathsf{E}_\tau^\perp\) (reflector), inducing an equivalence
\[
\Pers^{\mathrm{ft}}_k/\mathsf{E}_\tau\ \simeq\ \mathsf{E}_\tau^\perp.
\]
Basic laws (API):
\[
\mathbf{T}_\tau\circ \mathbf{T}_\tau \cong \mathbf{T}_\tau,\qquad \mathbf{T}_\tau \dashv \iota_\tau .
\]
Moreover, \(\mathbf{T}_\tau\) is exact and preserves finite (co)limits (Appendix~A).

% -------------------------
\subsection*{F.3. Interleavings, shifts, and \(1\)-Lipschitz [Spec] (AK.Core)}
% -------------------------

The interleaving pseudometric \(d_{\mathrm{int}}\) on \(\Pers^{\mathrm{ft}}_k\) is implemented via shift functors
\(\mathrm{Shift}_\varepsilon\) and \(\varepsilon\)-interleavings.
Appendix~A yields natural isomorphisms
\(\mathrm{Shift}_\varepsilon\circ \mathbf{T}_\tau\simeq \mathbf{T}_\tau\circ \mathrm{Shift}_\varepsilon\)
that transport interleavings, hence
\[
d_{\mathrm{int}}(\mathbf{T}_\tau M,\mathbf{T}_\tau N)\ \le\ d_{\mathrm{int}}(M,N).
\]
A \(V\)-enriched (Lawvere) lift is provided in \S F.A.

% -------------------------
\subsection*{F.4. Canonical Normal Form (CNF) after \texorpdfstring{$\mathbf{T}_\tau$}{Tτ} [Spec] (AK.Core)}
% -------------------------

\begin{definition}[CNF for morphisms after truncation]\label{F:def:cnf}
For \(f:M\to N\) in \(\Pers^{\mathrm{ft}}_k\), define its canonical normal form
\(\mathrm{CNF}_\tau(f)\) as the barcode-level decomposition of \(\mathbf{T}_\tau f\) (Appendix~A):
\[
\mathbf{T}_\tau M \ \cong\ \bigoplus_a I_a,\qquad
\mathbf{T}_\tau N \ \cong\ \bigoplus_b I_b,
\]
and, under these decompositions,
\[
\mathbf{T}_\tau f\ \leftrightsquigarrow\
\Big(\bigoplus_c \mathrm{id}_{I_c}\Big)\ \oplus\
\Big(\bigoplus_d 0_{I_d}\Big)\ \oplus\
\Big(\bigoplus_e \iota_e\Big),
\]
where each summand is either an isomorphism on an interval, the zero map, or a standard inclusion
\(\iota_e:I[\ell,\infty)\to I[\ell',\infty)\) with \(\ell\ge \ell'\).
This CNF is unique up to permutation of factors.
\end{definition}

\begin{proposition}[Reading \(\mu,\nu\) from CNF]\label{F:prop:cnf-mu-nu}
Let \(f\) be as above. In \(\mathrm{CNF}_\tau(f)\), the multiplicity of \(I[0,\infty)\) in
\(\Ker(\mathbf{T}_\tau f)\) (resp.\ \(\coker(\mathbf{T}_\tau f)\)) equals the tower invariant \(\mu\) (resp.\ \(\nu\))
defined in \S F.8. Finite bars contribute zero to \(\mu,\nu\).
\end{proposition}

\begin{remark}[CNF as a proof/automation device]
The CNF provides a canonical target for normalization tactics (\S F.19). It is functorial under isomorphism and stable under cofinal tower reindexings.
\end{remark}

% -------------------------
\subsection*{F.5. Filtered colimits, scope rule, and constructibility [Spec] (AK.Core)}
% -------------------------

\textbf{Scope rule.}
All filtered (co)limit computations are performed objectwise in \([\mathbb{R},\mathsf{Vect}_k]\), where filtered colimits are exact;
they are invoked only under Appendix~A, Remark~\ref{A:rk:filtered-colimits}.
Whenever the result might exit \(\Pers^{\mathrm{ft}}_k\),
we either (i) verify constructibility, or (ii) compute outside and \emph{return} via \(\mathbf{T}_\tau\)
(or an explicit finite-type truncation). No claim is made outside this regime.

% -------------------------
\subsection*{F.6. Collapse on filtered complexes and f.q.i. [Spec] (AK.Core)}
% -------------------------

We use a collapse/threshold operation \(C_\tau\) at the level of filtered complexes:
\[
C_\tau:\ \mathsf{FiltCh}(k)\longrightarrow \mathsf{FiltCh}(k),
\]
compatible with \(\mathbf{P}_i\) by construction, and preserving filtered quasi-isomorphisms (f.q.i.) up to localization.
In practice, \(\mathbf{P}_i\circ C_\tau\) is compared to \(\mathbf{T}_\tau\circ \mathbf{P}_i\) via a natural transformation
that becomes an isomorphism after applying \(\mathbf{T}_\tau\) (Appendix~A).
All \(\Ext\)-tests are taken after \(\mathcal{R}(C_\tau F)\) with \(\mathcal{R}\) of amplitude \([-1,0]\) (Appendix~C).
AWFS scaffolding for \(C_\tau\) is given in \S F.B.

% -------------------------
\subsection*{F.7. Local equivalences within a window [Spec] (AK.LocalEquiv)}
% -------------------------

We formalize the window-local bridge \(\mathrm{PH}\leftrightarrow \Ext\) under amplitude \(\le 1\) and mild regularity.

Let a right-open window \(W\) be fixed; assume:
(i) the filtered complex \(F\) is concentrated in homological degrees \(\le 1\) on \(W\),
(ii) the collapse \(C_\tau\) is stable on \(W\) (saturation),
(iii) no \(\tau\)-accumulation of critical values on \(W\) (Appendix~D),
(iv) tail identification for \(\phi_{i,\tau}\) on \(W\) (Appendix~D, (S1)).

Let \(\mathcal{R}\) be a derived realization of amplitude \([-1,0]\) (Appendix~C).
Then there are natural identifications
\[
H^{-1}\bigl(\mathcal{R}(C_\tau F)\bigr)\ \simeq\ \varinjlim_{t\in W} H_1(F^t),\qquad
\Ext^1\bigl(\mathcal{R}(C_\tau F),k\bigr)\ \simeq\ \Hom\!\bigl(H^{-1}(\mathcal{R}(C_\tau F)),k\bigr).
\]
Thus, window-locally,
\[
\mathrm{PH}_1(F|_W)=0\quad\Longrightarrow\quad \Ext^1(\mathcal{R}(C_\tau F),k)=0.
\]

\begin{remark}[Local reverse under \(E_1(W)=0\)]
Under the definable-window trigger \(E_1(W)=0\) and finite Leray depth (Appendix~C, Corollary~\ref{C:cor:definable-local-bridge}),
\(\Ext^1(\mathcal{R}(C_\tau F|_W),k)=0\) also implies \(\mathrm{PH}_1(C_\tau F|_W)=0\).
We encapsulate this as the tactic \emph{reverse\_bridge!} in \S F.19.
\end{remark}

% -------------------------
\subsection*{F.8. Towers, \texorpdfstring{$\phi_{i,\tau}$}{phi}, and diagnostics \texorpdfstring{$(\mu,\nu)$}{(mu,nu)} [Spec] (AK.Tower)}
% -------------------------

We define towers, the comparison map \(\phi_{i,\tau}\), and the invariants
\[
\mu_{i,\tau}(T)\ :=\ \mathrm{gdim}\,\ker\phi_{i,\tau}(T),\qquad
\nu_{i,\tau}(T)\ :=\ \mathrm{gdim}\,\mathrm{coker}\,\phi_{i,\tau}(T),
\]
computed in \(\Pers^{\mathrm{ft}}_k\) after applying \(\mathbf{T}_\tau\) (Appendix~D).
They are invariant under filtered quasi-isomorphisms of towers and under cofinal reindexings,
and vanish under (S1)--(S3) (Appendix~D). CNF (\S F.4) provides a canonical device to read \((\mu,\nu)\).
Type labels are assigned by \((\mu,\nu)\) per Appendix~D and Remark~\ref{F:rem:diagzero}.

% -------------------------
\subsection*{F.9. Overlap Gate, MECE windows, and gluing [Spec] (AK.Gluing)}
% -------------------------

We formalize the operational glue from windows to a global verdict.
MECE coverage, \v{C}ech--\(\Ext^1\) consistency on overlaps, Restart/Summability, and stability bands follow Appendices~C--E.
All overlap checks are carried out \emph{after} collapse on the B-side single layer.

% -------------------------
\subsection*{F.10. Spectral calculus and Lipschitz bounds [Spec] (AK.Spectral)}
% -------------------------

Spectral indicators are auxiliary and computed only after collapse, under the fixed window policy (Appendix~E).
Deletion-type spectral monotonicity is used only in the Loewner-orientable cases (principal/Dirichlet, Schur complement, Loewner order);
inclusion-type operations are \emph{non-expansive only}.
Safe low-pass post-processing (temporal kernels; heat/resolvent filters with filter-scale parameter \(\sigma\))
is non-expansive under the recorded Lipschitz constant (Appendix~E).

% -------------------------
\subsection*{F.11. Edge identification \texorpdfstring{$\mathrm{PH}_1 \Rightarrow \Ext^1$}{PH1⇒Ext1} [Spec] (AK.Core)}
% -------------------------

Let \(\mathcal{R}:\mathsf{FiltCh}(k)\to D(\mathsf{Vect}_k)\) be of amplitude \([-1,0]\).
There is a natural edge identification
\[
H^{-1}(\mathcal{R}(F))\ \cong\ \varinjlim_{t} H_1(F^t).
\]
For \(A\in D^{[-1,0]}\) we have \(\Ext^1(A,k)\cong \Hom(H^{-1}(A),k)\).
Combining these we obtain, for any \(F\),
\[
\mathrm{PH}_1(F)=0\quad\Longrightarrow\quad \Ext^1(\mathcal{R}(F),k)=0,
\]
and the same implication after insertion of the collapse \(C_\tau\).
On definable right-open windows with \(E_1(W)=0\), the converse holds after collapse (Appendix~C).

% -------------------------
\subsection*{F.12. Lean~4 sketch (representative stubs) [Spec]}
% -------------------------

\begin{verbatim}
-- AK.Core: categories, Serre reflector, interleavings, scope rule, collapse, CNF
namespace AK.Core
open scoped BigOperators Classical
noncomputable section

variable (k : Type*) [Field k]
abbrev Vect := FinVect k
structure RIdx := (α : Type) (str : Preorder α)
abbrev Diag := (RIdx → Vect k)

abbrev Pers := { M : Diag k // Constructible M }

-- Serre reflector T_τ : Pers → E_τ^⊥
def Eτ (τ : ℝ≥0) : SerreSubcategory (Pers k) := by admit
noncomputable def Tτ (τ : ℝ≥0) : Pers k ⥤ (Eτ k τ).orthogonal := by admit
noncomputable def iotaτ (τ) : (Eτ k τ).orthogonal ⥤ Pers k := by admit
theorem Tτ_exact (τ) : (Tτ k τ).IsExact := by admit
theorem Tτ_idem  (τ) : (Tτ k τ) ⋙ (Tτ k τ) ≅ (Tτ k τ) := by admit
theorem Tτ_adj   (τ) : (Tτ k τ) ⊣ (iotaτ k τ) := by admit

-- Interleaving stability (1-Lipschitz)
class Interleaving (C : Type*) :=
  (dist : C → C → ℝ≥0∞) (isPseudoMetric : PseudoMetricSpace C)
def d_int := (Interleaving.dist : Pers k → Pers k → ℝ≥0∞)
noncomputable def Shift (ε : ℝ≥0) : Pers k ⥤ Pers k := by admit
axiom shift_comm (τ ε) : Shift k ε ⋙ (Tτ k τ) ≅ (Tτ k τ) ⋙ Shift k ε
theorem Tτ_non-expansive (τ) :
  ∀ M N : Pers k, d_int k ((Tτ k τ).obj M) ((Tτ k τ).obj N) ≤ d_int k M N := by admit

-- Scope rule hooks
theorem filtered_colim_exact :
  ∀ {J} [IsFiltered J] (F : J ⥤ Vect), ExactFilteredColim F := by admit
axiom return_to_constructible :
  ∀ (D : SomeFilteredDiagram), Constructible (colim D)

-- Filtered complexes, persistence, collapse
abbrev FiltCh := FiltChCat k
def P_i (i : ℤ) : FiltCh k ⥤ Pers k := by admit
noncomputable def Cτ (τ : ℝ≥0) : FiltCh k ⥤ FiltCh k := by admit
theorem Cτ_preserves_fqi (τ) : PreservesFQI (Cτ k τ) := by admit

-- CNF after T_τ (barcode-normal form)
structure CNF where
  iso_parts  : List Interval
  zero_parts : List Interval
  incl_parts : List (Interval × Interval)  -- standard inclusions
noncomputable def cnf_after_Tτ {M N : Pers k} (τ) (f : M ⟶ N) : CNF := by admit

end AK.Core
\end{verbatim}

% -------------------------
\subsection*{F.13. Coq sketches (mathcomp/coq-category-theory) [Spec]}
% -------------------------

\begin{verbatim}
From mathcomp Require Import all_ssreflect all_algebra.
From CoqCT Require Import Category Abelian Functor Limits Colimits.
Set Implicit Arguments. Unset Strict Implicit. Unset Printing Implicit Defensive.

Module AK.

Parameter k : fieldType.
Axiom Vect : AbelianCat.                     (* f.d. k-vector spaces *)
Axiom Rposet : PreOrder.                     (* (ℝ, ≤), schematic *)
Definition Diag := FunctorCat Rposet Vect.
Parameter Constructible : Diag -> Prop.
Record Pers := { M : Diag; pfd : Constructible M }.

Axiom Eτ : SerreSubcat Pers.
Axiom Tτ : Functor Pers (Orthogonal Eτ).
Axiom iotaτ : Functor (Orthogonal Eτ) Pers.
Axiom Tτ_exact : ExactFunctor Tτ.
Axiom Tτ_idem  : FunctorComp Tτ Tτ ≅ Tτ.
Axiom Tτ_adj   : Adjunction Tτ iotaτ.

Parameter dint : Pers -> Pers -> R.
Parameter Shift : R -> Functor Pers Pers.
Axiom shift_comm : forall eps, FunctorComp (Shift eps) Tτ ≅ FunctorComp Tτ (Shift eps).
Axiom Tτ_non-expansive :
  forall (X Y : Pers), dint (Tτ X) (Tτ Y) <= dint X Y.

(* Collapse and persistence *)
Parameter FiltCh : Type.
Parameter P_i : Z -> Functor FiltCh Pers.
Parameter Cτ : R -> Functor FiltCh FiltCh.
Axiom Cτ_preserves_fqi : forall τ, PreservesFQI (Cτ τ).

(* CNF after T_τ *)
Record CNF := { iso_parts : seq Interval; zero_parts : seq Interval;
                incl_parts : seq (Interval * Interval) }.
Parameter cnf_after_Tτ : forall M N (f : Hom M N) τ, CNF.

End AK.
\end{verbatim}

% -------------------------
\subsection*{F.14. Tests and fixtures (T7, T10, T13) [Spec]}
% -------------------------

\paragraph{T7 (Saturation gate).}
Construct a tower with pure cokernel defect and one with a stationary summand, and their direct sum.
Verify that:
(i) \(\mu,\nu\) equal the multiplicity of \(I[0,\infty)\) after \(\mathbf{T}_\tau\) (via CNF);
(ii) cofinal reindexing \(n\mapsto n+1\) preserves \((\mu,\nu)\);
(iii) under (S1) the comparison \(\phi_{i,\tau}\) is an isomorphism and \(\mathrm{DiagZero}\).

\paragraph{T10 (A/B overlap).}
Instantiate an \(\eta\)-tolerant A/B test on overlaps and verify soft commuting; check that
window-local \(\Ext^1\)-vanishing glues via \v{C}ech--\(\Ext^1\) (Appendix~C).

\paragraph{T13 (\(\delta\)-budget).}
Generate a \(\delta\)-ledger per window/degree; check additivity/post-stability and the restart inequality
\[
\mathrm{gap}_{k+1}\ge \kappa\bigl(\mathrm{gap}_k-\Sigma\delta_k\bigr)_+.
\]
Verify summability \(\sum_k\Sigma\delta_k<\infty\) and that B--Gate$^{+}$ accepts precisely when
\(\mathrm{gap}>\mathrm{dsum}\) along with \(\mathrm{PH}_1=0\), \(\Ext^1=0\), and \(\mathrm{DiagZero}\).

% -------------------------
\subsection*{F.15. What is proved, what is assumed [Spec]}
% -------------------------

\begin{itemize}[leftmargin=1.25em]
\item (\emph{Localization}) \(\mathsf{E}_\tau\) is hereditary Serre; \(\mathbf{T}_\tau\) exists, is exact,
idempotent, and induces \(\Pers^{\mathrm{ft}}_k/\mathsf{E}_\tau\simeq \mathsf{E}_\tau^\perp\).
\item (\emph{Stability}) \(\mathbf{T}_\tau\) is \(1\)-Lipschitz for the interleaving metric via shift-commutation (Appendix~A).
\item (\emph{Towers}) \(\phi_{i,\tau}\) is functorial; under (S1)--(S3) (Appendix~D) it is an isomorphism, hence \(\mathrm{DiagZero}\).
\((\mu,\nu)\) are invariant under f.q.i.\ and cofinal reindexings; finiteness holds by degree bounds.
\item (\emph{Bridge}) For \(F\in\mathsf{FiltCh}(k)\), \(\mathcal{R}\) has amplitude \([-1,0]\);
\(H^{-1}(\mathcal{R}(F))\simeq \varinjlim_t H_1(F^t)\) and
\(\Ext^1(A,k)\simeq \Hom(H^{-1}(A),k)\) for \(A\in D^{[-1,0]}\), hence
\(\mathrm{PH}_1(F)=0\Rightarrow \Ext^1(\mathcal{R}(F),k)=0\).
On definable right-open windows with \(E_1(W)=0\), the local reverse holds after \(C_\tau\) (Appendix~C).
\item (\emph{Spectral}) Deletion-type operators are spectrally monotone in the appropriate Loewner orientation; inclusion-type operators are controlled via stability bounds; safe low-pass is non-expansive under its recorded Lipschitz constant (Appendix~E).
\end{itemize}

% -------------------------
\subsection*{F.16. Notes on libraries and portability [Spec]}
% -------------------------

The Lean sketch targets \textsf{mathlib} (abelian categories, Serre subcategories, localization, derived categories).
The Coq sketch targets \textsf{mathcomp}+\textsf{coq-category-theory} (or \textsf{UniMath}).
Nontrivial steps are isolated behind \texttt{admit}/\texttt{Axiom} with explicit references to Appendices~A--E.
Replacing them by library lemmas yields a complete development.
In Lean, define \(\phi_{i,\tau}\) via \texttt{Limits.colimit.desc} and use \texttt{colimit.hom\_ext} for naturality;
in Coq, use \texttt{Colim.desc}/\texttt{colim\_map} with right-to-left \texttt{compose} convention.

% -------------------------
\subsection*{F.17. Thin catalysts (V-enrichment, AWFS, o-minimal \v{C}ech, Iwasawa control)}
% -------------------------

\paragraph{F.A. \(V\)-enriched metric/shift (Spec).}
Let \(V\) be a commutative quantale.
Equip \(\Pers^{\mathrm{ft}}_k\) with a Lawvere \(V\)-metric \(d_V\) obtained from \(d_{\mathrm{int}}\) via a monotone embedding and a \(V\)-aggregator.
Require:
(i) \(d_V\) extends \(d_{\mathrm{int}}\) on scalars,
(ii) \(\mathrm{Shift}_\varepsilon\) is \(V\)-1-Lipschitz,
(iii) \(\mathbf{T}_\tau\) is \(V\)-1-Lipschitz and commutes with shifts up to enriched natural isomorphism.
\emph{Use:} stability of dashboards and Overlap Gate tolerances.

\paragraph{F.B. AWFS skeleton for \(C_\tau\) (Spec).}
Postulate an algebraic weak factorization system \((\mathcal{L}_\tau,\mathcal{R}_\tau)\) on \(\mathsf{FiltCh}(k)\) with functorial factorization
\(F\xrightarrow{\ell_\tau} C_\tau F \xrightarrow{r_\tau} F\),
where \(\ell_\tau\) is a \(\tau\)-collapse cofibration (acyclic at persistence level after \(\mathbf{T}_\tau\))
and \(r_\tau\) is a \(\tau\)-local fibration.
\emph{Use:} functoriality of \(C_\tau\), pullbacks along \(\mathcal{R}_\tau\)-maps, and stability of f.q.i.\ under collapse.

\paragraph{F.C. \(o\hbox{-}minimal\) \v{C}ech and Iwasawa control (Spec).}
Let \(\mathcal{U}=\{U_i\}\) be a definable right-open cover of a window \(W\) with finite Leray depth.
Then the \v{C}ech complex \(\check{C}(\mathcal{U})\) computes window-local \(\Ext^1\) (under amplitude \(\le 1\)) and patches across overlaps.
An \emph{Iwasawa control catalyst} is a tuple \((\Gamma,\rho,\upsilon)\) with a directed index \(\Gamma\),
a non-expansive control \(\rho:\Gamma\to \RR_{\ge 0}\), and a compatibility map \(\upsilon\) such that the windowed diagnostics are \(\rho\)-Cauchy and tail-identify with the apex.
\emph{Use:} Overlap Gate---uniform control across towers implies stability-band persistence and global gluing.

% -------------------------
\subsection*{F.18. Convergence Manager (DP windows) [Spec]}
% -------------------------

We wrap Restart/Summability (Appendix~D) into a Convergence Manager for dynamic-programming (DP) windowing.

\begin{specification}[DP-Convergence Manager]\label{F:spec:dp-manager}
Maintain, per degree \(i\): current safety margin \(\mathrm{gap}_k(i)\), per-window budget \(\Sigma\delta_k(i)\),
and retention factor \(\kappa\in(0,1]\). On transition \(k\to k+1\),
\[
\mathrm{gap}_{k+1}(i)\ \gets\ \max\bigl\{0,\ \kappa\bigl(\mathrm{gap}_k(i)-\Sigma\delta_k(i)\bigr)\bigr\}.
\]
Accept window \(k\) if \(\mathrm{gap}_k(i)>\Sigma\delta_k(i)\) and \(\mathrm{DiagZero}\);
declare convergence on an interval if acceptance persists across the MECE chain and \(\sum_k\Sigma\delta_k(i)<\infty\).
\end{specification}

\paragraph{Lean~4 tactic skeleton.}
\begin{verbatim}
namespace AK.Gluing
open AK.Core
meta def converge_windows! :
  Π (κ : ℝ≥0) (deg : ℤ) (gaps budgets : List ℝ≥0),
    tactic (List ℝ≥0) := by admit
-- Intended behavior: compute next gaps via κ⋅(gap - budget)⁺ and
-- produce certificates that BGATE⁺ holds on accepted windows.
end AK.Gluing
\end{verbatim}

\paragraph{Coq hint database (sketch).}
\begin{verbatim}
Create HintDb DPconvergence.
Hint Resolve restart_ok summable_budget : DPconvergence.
(* A Ltac 'converge_windows' computes κ⋅(gap - Σδ)^+ and applies the hints. *)
\end{verbatim}

% -------------------------
\subsection*{F.19. Tactic stubs: \texttt{cnf!}, \texttt{ext1\_hom!}, \texttt{reverse\_bridge!}, \texttt{converge\_windows!}}
% -------------------------

\paragraph{Lean~4 (meta-level).}
\begin{verbatim}
namespace AK.Tactics
open AK.Core AK.Tower AK.LocalEquiv

/-- Put a morphism into CNF after T_τ and read (μ,ν). -/
meta def cnf! (τ : ℝ≥0) : tactic Unit :=
  `[refine (AK.Core.cnf_after_Tτ _ _ ?f τ); all_goals admit]

/-- Solve goals Ext¹(A,k) ≅ Hom(H^{-1}A,k) for A ∈ D^{[-1,0]}. -/
meta def ext1_hom! : tactic Unit :=
  `[apply AK.Core.ext1_edge; try { exact ‹_› }]

/-- Local reverse bridge under E₁(W)=0 (definable window, amplitude ≤ 1). -/
meta def reverse_bridge! : tactic Unit := `
  [ apply AK.LocalEquiv.local_reverse_bridge; all_goals admit ]

/-- DP-window convergence manager (Appendix D/E). -/
meta def converge_windows! := AK.Gluing.converge_windows!
end AK.Tactics
\end{verbatim}

\paragraph{Coq (Ltac skeleton).}
\begin{verbatim}
Ltac cnf τ :=
  (* normalize T_τ f to barcode CNF and compute μ, ν *) idtac.

Ltac ext1_hom :=
  (* reduce Ext¹(A,k) to Hom(H^{-1}A,k) when A ∈ D^{[-1,0]} *) idtac.

Ltac reverse_bridge :=
  (* apply local reverse under E₁(W)=0 to deduce PH₁=0 from Ext¹=0 *) idtac.

Ltac converge_windows :=
  (* apply restart_ok and summable_budget to chain windows *) eauto with DPconvergence.
\end{verbatim}

% -------------------------
\subsection*{F.20. Completion note}
% -------------------------

\begin{remark}[Completion note]
This appendix delivers IMRN/AiM-ready formalization stubs:
\(\mathbf{T}_\tau\) (exact, idempotent, \(1\)-Lipschitz), CNF after truncation to read \((\mu,\nu)\),
tower calculus with invariances and sufficient conditions (S1--S3), the bridge \(\mathrm{PH}_1\Rightarrow \Ext^1\)
and its local reverse under \(E_1(W)=0\), spectral auxiliaries (non-expansive low-pass with filter-scale \(\sigma\)),
and a DP-style Convergence Manager for MECE window pasting.
Lean/Coq tactic skeletons (\texttt{cnf!}, \texttt{ext1\_hom!}, \texttt{reverse\_bridge!}, \texttt{converge\_windows!})
provide an auditable path from the paper’s hypotheses to machine-checked goals.
All claims are confined to the constructible, amplitude-\(\le 1\) regime with filtered colimits used only under the scope rule.
No further supplementation is required for operational use in the proof framework.
\end{remark}



% =========================
\section*{Appendix G. Reproducibility: Logs and Schemas [Spec] (reinforced)}
% =========================
\phantomsection
\addcontentsline{toc}{section}{Appendix G. Reproducibility: Logs and Schemas}
\refstepcounter{section}
\label{G:repro}

This appendix specifies the provenance log (\texttt{run.yaml}) and the machine-readable schemas
for artifacts produced in this work—barcodes (\texttt{bars}), spectral indicators (\texttt{spec}),
Ext-tests (\texttt{ext}), tower comparison maps (\texttt{phi}), and the windowed length-spectrum audit
(\texttt{Lambda\_len}).
All files may be emitted in either JSON or HDF5; JSON keys coincide with HDF5 group/dataset names.
Filtered colimits are used only under the scope policy (Appendix~A, Remark~\ref{A:rk:filtered-colimits}).
Type labels follow \emph{Type I--II / Type III / Type IV}.
Cross-module conventions: the Ext-test is always against \(k[0]\), i.e.\
\(\Ext^1(\mathcal{R}(C_\tau F),k)=0\)
(with \(C_\tau\) understood up to f.q.i.\ on \(\Ho(\mathsf{FiltCh}(k))\));
the energy exponent satisfies \(\alpha>0\) (default \(\alpha=1\)).
Spectral monotonicity is asserted only for \emph{deletion-type} operations (Dirichlet/principal/Loewner),
with directions fixed by Appendix~E; inclusion-type operations are used solely with stability bounds.

\paragraph{Mandatory order (After-Collapse).}
All comparisons follow the mandatory order:
\[
\boxed{\ \text{for each } t\ \Longrightarrow\ \text{apply } \mathbf{P}_i\ \Longrightarrow\ \text{apply } \mathbf{T}_\tau\ \Longrightarrow\ \text{compare in }\Pers^{\mathrm{ft}}_k\ }.
\]

% ---- Canonical schema kernel (mandatory) + controlled extensions (optional) ----
\begin{declaration}[run.yaml schema: mandatory kernel + controlled extensions \textbf{[Spec]}]\label{dec:runyaml-extended}
\textbf{Normative status.}
This appendix is normative: the \texttt{run.yaml} manifest is a \emph{proof object} for auditability
(Chapter~17) and is synchronized with Chapter~12 (Decl.~\ref{dec:12-schema}).
Accordingly, the \emph{kernel blocks} below are \textbf{mandatory} whenever a run makes any quantitative claim.

\paragraph{Canonical keys and alias policy (for backward compatibility).}
The \emph{canonical} (preferred) keys are those listed in the kernel below.
For backward compatibility with older drafts and external tools, the following aliases \emph{may} appear,
but MUST be treated as synonyms and MUST be normalized to canonical keys during verification:
\[
\begin{array}{lll}
\texttt{definable.o\_minimal\_structure} &\equiv& \texttt{definable.structure},\\
\texttt{layered\_delta.delta\_Gal/delta\_Tr/delta\_Fun} &\equiv& \texttt{layered\_delta.deltaGal/deltaTr/deltaFun},\\
\texttt{iwasawa.control\_finite\_bounds.kernel\_leq/cokernel\_leq} &\equiv& \texttt{kernel\_le/cokernel\_le},\\
\texttt{awfs\_2cell.*} &\equiv& \texttt{awfs.*}.\\
\end{array}
\]
Use of aliases is \emph{discouraged}: canonical keys should be used in this manuscript and all released artifacts.

\paragraph{Mandatory kernel blocks.}
The following blocks (and minimal keys) are mandatory for auditability:
\begin{verbatim}
quantale:
  name: "[0,inf]_plus"     # or "max-plus", "product", ...
  op: "+"                  # value-level monoid op (aggregator for distances/budgets)
  unit: 0
  order: "<="
  mode: "standard"         # "standard" | "probabilistic" | "product" (Ch.12)

layered_delta:
  deltaGal: ...            # geometric-algebraic (δ^{Gal})
  deltaTr:  ...            # discretization/rounding/truncation (δ^{Tr})
  deltaFun: ...            # functorial/commutation residuals (δ^{Fun})

definable:
  structure: "R_an,exp"    # or "Denef-Pas"
  window_formulae:
    - "u <= t < u'"        # right-open windows; finite or countable MECE cover

iwasawa:
  tower_level: ...
  control_finite_bounds: { kernel_le: ..., cokernel_le: ... }

awfs:
  enabled: true
  two_cell_bounds: ...     # scalar or structured value; see below

overlap_checks:
  local_equiv: true
  cech_ext1_ok: true
  stability_band_ok: true

spectral_policy:
  order: "ascending"
  norm: "op"               # or "fro"
spectral_bounds:
  lambda_min: ...
  lambda_max: ...
  lip_tol: ...             # optional if unused

persistence:
  PH1_zero: true
  Ext1_zero: true
  mu: ...
  nu: ...
  phi_iso_tail: ...

budget:
  sum_delta: ...           # quantale-valued; may be scalar or structured
  safety_margin: ...
  gap_tau: ...             # mandatory when Restart/Summability is invoked

ab_test:
  eta: ...
  policy: ...
  fallback: ...
\end{verbatim}

\paragraph{Controlled optional extensions (recommended when used).}
The following blocks are optional but, if present, become \emph{auditable obligations}:
\begin{verbatim}
pfbc:
  policy: "after_collapse"         # mandatory when PF/BC is used
  residual_ledger: ["disc","meas"] # charging policy (Appendix N, L)

restart_summability:
  kappa_min: ...
  sum_delta_bound: ...

tropical:
  bins: { width: ..., range: [a,b] }   # diagnostic binning for aux-bars

policy:
  after_collapse_only: true
  windows: "right-open"

# NOTE on awfs.two_cell_bounds:
#   may be a scalar bound, or a structured map, e.g.
#   two_cell_bounds: { mirror_collapse: 0.005, transfer_collapse: 0.005 }
\end{verbatim}
\end{declaration}

\paragraph{Version note (suite v17.0).}
This version integrates: (i) window declarations and coverage checks,
(ii) operation logs with \(\delta\)-breakdowns,
(iii) Overlap Gate checks,
(iv) \(\Lambda_{\mathrm{len}}\) (length spectrum audit) cross-linked from all artifacts,
(v) canonical spectral policy (ascending order; declared norm; bounded safe low-pass flags),
(vi) Restart/Summability bookkeeping via \texttt{gap\_tau} when invoked,
and (vii) HDF5 canonicalization with fixed-length UTF-8 strings.
These fields ensure replayability and third-party auditability.

% -------------------------
\subsection*{G.1. Provenance, determinism, and gating}
% -------------------------
Each run records (i) source/inputs, (ii) algorithmic choices and thresholds, (iii) numeric tolerances and units,
(iv) code/environment fingerprints, (v) RNG details, (vi) strong identifiers (content hashes) for all artifacts,
and (vii) \emph{gating} decisions that determine acceptance of results.
Randomness is controlled by explicit seeds. Floating-point claims report both an \emph{asserted} tolerance and a \emph{measured} slack.
Windows (domain/collapse/spectral) must be declared, and a coverage check attests that all measured quantities
fall inside their stated windows. A budget aggregates operation-level error contributions \(\delta\) and yields a safety margin
relative to the governing tolerance; finally, \texttt{gate.accept} records the run-level decision (accept/reject) together with reasons.

% -------------------------
\subsection*{G.2. \texttt{run.yaml} schema (versions, windows, overlap, budget, gate)}
% -------------------------
\paragraph{Intent.}
A single file per execution, sufficient to reproduce the pipeline end-to-end, including all windows,
coverage checks, operation logs, overlap checks, and the final acceptance gate.

\noindent\textbf{Canonical layout (YAML).}
\begin{verbatim}
version: 17
schema_version: "2025-03-15"
suite_version: "v17.0"
run_id: "2025-03-15T09:12:07Z-7f5c1b1"

# Recommended: record coefficients explicitly (must agree with bars/ext artifacts).
coeff_field: "k"                   # e.g. "k"; at [Spec] may be "Novikov(q)" etc.

seed: 1337
rng:
  python: "default_rng"
  numpy:  "PCG64"

platform:
  os: "Ubuntu 22.04"
  cpu: "Intel(R) Xeon(R) Platinum 8370C"
  cuda: "12.2"
  blas: "OpenBLAS 0.3.23"
  hdf5: "1.14.3"
  lapack: "OpenBLAS-LAPACK"
  glibc: "2.35"
  kernel: "5.15.0-105"
  locale: "C.UTF-8"

env:
  python: "3.11.7"
  packages:
    numpy: "1.26.4"
    scipy: "1.13.1"
    h5py:  "3.10.0"
    networkx: "3.2.1"
  threads:
    OMP_NUM_THREADS: 1
    MKL_NUM_THREADS: 1
    OPENBLAS_NUM_THREADS: 1

container:
  image: "docker.io/example/persistence:2025.03"
  digest: "sha256:deadbeef..."

git:
  repo: "git@host:ak/persistence.git"
  commit: "a1b2c3d4"

units:
  filtration: "dimensionless"
  eigenvalues: "dimensionless"

# ============================================================
# ---- Mandatory kernel blocks (Ch.12, Dec. dec:12-schema) ----
# ============================================================
quantale:
  name: "[0,inf]_plus"       # examples: "[0,inf]_plus", "max-plus", "product"
  op: "+"                    # aggregator used for budgets/residuals
  unit: 0.0
  order: "<="
  mode: "standard"           # optional enumerator; used by auditors

layered_delta:
  deltaGal: 0.020
  deltaTr:  0.015
  deltaFun: 0.015

definable:
  structure: "R_an,exp"      # or "Denef-Pas"
  window_formulae: ["u <= t < u'"]
  # optional unless overlap/gluing uses Čech controls:
  cech_depth_bound: 2

iwasawa:
  tower_level: 128
  control_finite_bounds:
    kernel_le: 2
    cokernel_le: 0

awfs:
  enabled: true
  two_cell_bounds:
    mirror_collapse: 0.005
    transfer_collapse: 0.005

# Spectral policy is declared once here (Stage params must agree).
spectral_policy:
  order: "ascending"
  norm: "fro"                # "fro" or "op"
spectral_bounds:
  lambda_min: 1.0e-12
  lambda_max: 1.0e+05
  lip_tol: 0.02              # optional; omit if not used

windows:
  domain:
    filtration_range: [0.0, 2.0]
    degrees: [0, 2]
  collapse:
    tau_sweep: [0.25, 0.50, 1.00]
  spectral:
    range: [0.0, 2.0]
    order: "ascending"

coverage_check:
  domain_window_covers_bars: true
  spectral_window_covers_thetas: true
  collapse_tau_sweep_covers_reports: true

stability_bands:
  - { i: 1, tau_lo: 0.60, tau_hi: 0.95 }   # Appendix D.8

overlap_checks:
  local_equiv: true
  cech_ext1_ok: true
  stability_band_ok: true

# Ch.12: A/B mandatory
ab_test:
  eta: 0.01
  policy: "soft-commuting"
  fallback: "A_then_B"

# ============================================================
# Inputs and pipeline (auditable; stage params must be explicit)
# ============================================================
inputs:
  dataset: "AK-bench-v3"
  graphs:
    - path: "data/G_001.edgelist"
      hash: "sha256:..."
  filters:
    type: "height"
    params: { axis: 2 }

pipeline:
  metric: "interleaving"          # exactly one of: interleaving | bottleneck
  stages:
    - name: "barcode"
      params:
        field: "k"                # must agree with coeff_field
        reduction: "clearing"
    - name: "collapse"            # C_τ (up to f.q.i.)
      params: { tau: 0.50 }
    - name: "spec"
      params:
        window: [0.0, 2.0]        # right-open per policy.windows
        # Must agree with spectral_policy:
        norm: "fro"
        order: "ascending"
        # Bounds must agree with top-level spectral_bounds:
        spectral_bounds: { lambda_min: 1.0e-12, lambda_max: 1.0e+05, lip_tol: 0.02 }
        clip: 1.00
        loewner_assumption: "Aprime_preceq_A"   # Aprime_preceq_A | Aprime_succeq_A | none
        low_pass:
          kernel: "heat"
          even: true
          mass: 1.0
          clip_tau_eq_pipeline: true
        eig_solver:
          method: "lanczos"
          k: 128
          maxiter: 1000
          tol: 1e-12
          reorthogonalize: true
          rng_seed: 1337
    - name: "ext-test"            # Ext^1(R(C_τ F), k)
      params: { amplitude_check: true }

operations:
  - step: 1
    U: [0,1,3]
    type: "inclusion"
    tau: 0.50
    delta:
      distance: { interleaving: 0.050 }
      sources:
        discretization: 0.030
        rounding: 1.0e-12
        heuristic: 0.020
      total: 0.050
      note: "Edge contraction in subgraph U"

persistence:
  PH1_zero: true
  Ext1_zero: true
  mu: 1
  nu: 0
  phi_iso_tail: false            # run-level summary flag (details in phi artifact)

spectral:
  aux_bars_remaining: 0

thresholds:
  alpha: 1.0
  tol:
    distance:
      interleaving: 1e-6
    eig: 1e-8
    witness: 1e-9

# Ch.12: Budget mandatory (quantale-sum target)
budget:
  sum_delta: 0.150
  safety_margin: 0.850
  gap_tau: 0.025
  rationale: "All deltas accounted for; slack remains >0"

# Ch.12: Length spectrum audit mandatory (T15)
Lambda_len:
  degree: 1
  tau: 0.50
  audit: "hash:2f4c...d1"

gate:
  accept: true
  reason: "Coverage ok; safety margin positive; all assertions satisfied"

# ============================================================
# Optional but auditable obligations (present iff used)
# ============================================================
pfbc:
  policy: "after_collapse"        # enforces T–PFBC–AfterCollapse order
  residual_ledger: ["disc","meas"]

restart_summability:
  kappa_min: 0.8
  sum_delta_bound: 0.05

tropical:
  bins: { width: 0.02, range: [0.0, 2.0] }

# If navigation uses gradients (Ch.13–15), this block is mandatory:
grad_policy:
  method: "finite_difference"     # finite_difference | SPSA | surrogate
  norm: "fro"                     # norm for step decisions
  stencil: { kind: "two_sided", eps: 1.0e-3, coords: "all" }
  seed: 1337
  variance_reporting: "estimate"  # estimate | upper_bound
  delta_charge_target: "operations[*].delta.sources.heuristic"

policy:
  after_collapse_only: true
  windows: "right-open"

serialization:
  float_dtype: "ieee754-f64-le"
  json_sort_keys: true
  hdf5_canonical:
    compression: { algo: "gzip", level: 4 }
    shuffle: false
    fletcher32: false
    track_times: false
    fillvalue: 0.0
    string_encoding: "utf8-fixed"
    chunk_shapes:
      bars: { i: 4096, birth: 4096, death: 4096, death_is_inf: 4096, mult: 4096 }
      spec_eigs: { eig: 4096 }
      spec_Ntheta: { theta: 512, left: 512, right: 512 }
      phi_idx: { i: 256, tau: 256, iso: 256, mu: 256, nu: 256 }

status:
  success: true
  errors: []

outputs:
  bars: "out/bars_7f5c1b1.json"
  spec: "out/spec_7f5c1b1.json"
  ext:  "out/ext_7f5c1b1.json"
  phi:  "out/phi_7f5c1b1.h5"
  Lambda_len: "out/Lambda_len_7f5c1b1.json"
\end{verbatim}



% -------------------------
\subsection*{G.3. \texttt{bars} (barcodes) schema}
% -------------------------
\paragraph{Semantics.}
A constructible barcode is a multiset of half-open intervals \(I=[b,d)\) with degree \(i\). Deaths may be \(+\infty\).

\noindent\textbf{JSON layout (infinity convention, units, cross-links, optional clip report).}
\begin{verbatim}
{
  "meta": {
    "schema_version": "2025-03-15",
    "suite_version": "v17.0",
    "field": "k",
    "filtration_units": "dimensionless",
    "endpoint_convention": "[b,d) (see Chapter 2)",
    "infinity": { "json": "inf" },
    "clip_tau": 0.50,
    "float_dtype": "ieee754-f64-le",
    "string_encoding": "utf8-fixed",
    "links": {
      "run_id": "2025-03-15T09:12:07Z-7f5c1b1",
      "run_yaml_hash": "sha256:...run",
      "Lambda_len": "sha256:...Lambda"
    }
  },
  "bars": [
    { "i": 0, "birth": 0.0, "death": 0.3,   "mult": 1 },
    { "i": 1, "birth": 0.2, "death": "inf", "mult": 1 }
  ],
  "hash": "sha256:...bars"
}
\end{verbatim}

\noindent\textbf{HDF5 layout (split representation for \(+\infty\); fixed UTF-8).}
\begin{itemize}[leftmargin=1.25em]
  \item Datasets:
    \texttt{/bars/i} (\texttt{int32}),
    \texttt{/bars/birth} (\texttt{float64}),
    \texttt{/bars/death} (\texttt{float64}),
    \texttt{/bars/death\_is\_inf} (\texttt{bool}),
    \texttt{/bars/mult} (\texttt{int32}).
  \item Attributes:
    \texttt{/bars}.attrs[\texttt{field}=\texttt{"k"}],
    \texttt{filtration\_units},
    \texttt{schema\_version},
    \texttt{suite\_version},
    \texttt{float\_dtype},
    \texttt{death\_encoding}=\texttt{"split\_scalar\_bool"},
    \texttt{string\_encoding}=\texttt{"utf8-fixed"},
    optional \texttt{clip\_tau},
    and \texttt{links/Lambda\_len}.
\end{itemize}

% -------------------------
\subsection*{G.4. \texttt{spec} (spectral indicators) schema}
% -------------------------
\paragraph{Semantics.}
Spectral features include: clipped sums, counts above/below thresholds with left/right limits,
and deletion-type monotonicity diagnostics (Appendix~E). Matrices/operators are identified by content hashes.
All spectral reporting is post-collapse unless explicitly marked otherwise by policy.

\noindent\textbf{JSON layout (ascending storage; \(N_{\theta\pm 0}\); solver and low-pass params; coverage; cross-links).}
\begin{verbatim}
{
  "meta": {
    "schema_version": "2025-03-15",
    "suite_version": "v17.0",
    "eigen_units": "dimensionless",
    "order": "ascending",
    "sorted": true,
    "Ntheta_convention": { "left": "N_{θ-0}", "right": "N_{θ+0}" },
    "window": { "range": [0.0, 2.0], "semantics": "right-open" },
    "norm": "fro",
    "clip_tau": 1.0,
    "tol_eig": 1e-8,
    "loewner_assumption": "Aprime_preceq_A",
    "low_pass": { "kernel": "heat", "even": true, "mass": 1.0, "safe": true },
    "aux_bars_remaining": 0,
    "coverage_check": { "thetas_in_window": true },
    "tropical_bins": { "width": 0.02, "range": [0.0, 2.0] },
    "string_encoding": "utf8-fixed",
    "eig_solver": {
      "method": "lanczos", "k": 128, "maxiter": 1000,
      "tol": 1e-12, "reorthogonalize": true, "rng_seed": 1337
    },
    "links": {
      "run_id": "2025-03-15T09:12:07Z-7f5c1b1",
      "run_yaml_hash": "sha256:...run",
      "Lambda_len": "sha256:...Lambda"
    }
  },
  "operators": [
    {
      "id": "sha256:...A",
      "kind": "laplacian_dirichlet",
      "n": 500,
      "spectrum": { "eigs": [0.10, 0.12, 0.45, ...] },
      "clip":     { "tau": 1.0, "sum": 37.219, "deficit": 12.004 }
    },
    {
      "id": "sha256:...B",
      "kind": "principal_submatrix",
      "parent": "sha256:...A",
      "N_theta": [
        { "theta": 0.20, "left": 17, "right": 16 },
        { "theta": 0.50, "left": 10, "right": 10 }
      ],
      "monotonicity": { "type": "deletion", "passed": true }
    }
  ],
  "hash": "sha256:...spec"
}
\end{verbatim}

\noindent\textbf{HDF5 layout.}
\begin{itemize}[leftmargin=1.25em]
  \item \texttt{/spec/ops/\{id\}/eig} (float64, ascending),
        \texttt{/spec/ops/\{id\}/clip/sum} (float64),
        \texttt{/spec/ops/\{id\}/clip/deficit} (float64),
        \texttt{/spec/ops/\{id\}/Ntheta/theta},
        \texttt{/spec/ops/\{id\}/Ntheta/left},
        \texttt{/spec/ops/\{id\}/Ntheta/right} (parallel datasets).
  \item Attributes:
        \texttt{kind}, \texttt{parent}, \texttt{norm} \(\in\{\texttt{"fro"},\texttt{"op"}\}\),
        \texttt{order}=\texttt{"ascending"}, \texttt{sorted} (bool), \texttt{eigen\_units}, \texttt{tol\_eig},
        \texttt{schema\_version}, \texttt{suite\_version}, \texttt{loewner\_assumption},
        \texttt{low\_pass/*}, \texttt{aux\_bars\_remaining}, \texttt{coverage\_thetas\_in\_window},
        optional \texttt{tropical\_bins/width,range},
        \texttt{string\_encoding}=\texttt{"utf8-fixed"}, and \texttt{links/Lambda\_len}.
\end{itemize}

% -------------------------
\subsection*{G.5. \texttt{ext} (Ext-test) schema}
% -------------------------
\paragraph{Semantics.}
Outcome of the bridge \(\mathrm{PH}_1\Rightarrow \Ext^1\) for \(C_\tau F\), with amplitude checks for \(\mathcal{R}\) and recorded assumptions.

\noindent\textbf{JSON layout (assumptions, cross-links).}
\begin{verbatim}
{
  "meta": {
    "schema_version": "2025-03-15",
    "suite_version": "v17.0",
    "field": "k",
    "alpha": 1.0,
    "assumptions": {
      "field_is_k": true,
      "constructible_verified": true,
      "t_exact_and_amp_le_1": true
    },
    "string_encoding": "utf8-fixed",
    "links": {
      "run_id": "2025-03-15T09:12:07Z-7f5c1b1",
      "run_yaml_hash": "sha256:...run",
      "Lambda_len": "sha256:...Lambda"
    }
  },
  "tau": 0.50,
  "amplitude": { "ok": true, "range": [-1, 0] },
  "Hminus1": { "dim": 0, "witness_norm": 0.0 },
  "Ext1":    { "dim": 0, "passed": true, "tol": 1e-9, "slack": 0.0 },
  "links":   { "bars": "sha256:...bars", "phi": "sha256:...phi" },
  "hash": "sha256:...ext"
}
\end{verbatim}

\noindent\textbf{HDF5 layout.}
\begin{itemize}[leftmargin=1.25em]
\item Scalars:
\texttt{/ext/tau} (float64),
\texttt{/ext/Hminus1/dim} (int32),
\texttt{/ext/Ext1/dim} (int32),
\texttt{/ext/Ext1/passed} (bool),
\texttt{/ext/Ext1/tol} (float64),
\texttt{/ext/Ext1/slack} (float64).
\item Attributes:
\texttt{field}=\texttt{"k"},
\texttt{alpha} (float64),
\texttt{schema\_version},
\texttt{suite\_version},
\texttt{assumptions/*},
\texttt{string\_encoding}=\texttt{"utf8-fixed"},
and \texttt{links/Lambda\_len}.
\end{itemize}

% -------------------------
\subsection*{G.6. \texttt{phi} (tower comparison) schema}
% -------------------------
\paragraph{Semantics.}
Encodes \(\phi_{i,\tau}\) for towers, together with \((\mu,\nu)\) as generic-fiber dimensions after truncation,
structural flags for the sufficiency hypotheses (Appendix~D, \S D.4), and explicit stability bands (Appendix~D.8).

\noindent\textbf{JSON layout (with \(\tau\)-sweep, stability bands, witnesses, iso-tail, cross-link).}
\begin{verbatim}
{
  "meta": {
    "schema_version": "2025-03-15",
    "suite_version": "v17.0",
    "definition": "phi_{i,τ}: colim T_τ P_i(F_n) → T_τ P_i(F_∞)",
    "scope": "colim in [ℝ,Vect_k], return-to-constructible policy",
    "tau_sweep": [0.25, 0.50, 1.00],
    "edge_kinds": ["inclusion","projection","quasi_iso",
                   "filtration_preserving_map","schur_complement","other"],
    "stability_bands": [ { "i": 1, "tau_lo": 0.60, "tau_hi": 0.95 } ],
    "string_encoding": "utf8-fixed",
    "links": {
      "run_id": "2025-03-15T09:12:07Z-7f5c1b1",
      "run_yaml_hash": "sha256:...run",
      "Lambda_len": "sha256:...Lambda"
    }
  },
  "indices": [
    {
      "i": 1, "tau": 0.50,
      "iso": false,
      "mu": 1, "nu": 0,
      "flags": { "S1_commutes": false, "S2_noAccum": true, "S3_Cauchy": false },
      "witness": { "ker_generic_dim": 1, "coker_generic_dim": 0 },
      "iso_tail": { "passed": false }
    }
  ],
  "tower": {
    "nodes": [
      { "n": 0, "id": "sha256:...F0" },
      { "n": 1, "id": "sha256:...F1" }
    ],
    "edges": [
      { "src": 0, "dst": 1, "kind": "inclusion" }
    ],
    "limit": { "id": "sha256:...Finf" }
  },
  "hash": "sha256:...phi"
}
\end{verbatim}

\noindent\textbf{HDF5 layout.}
\begin{itemize}[leftmargin=1.25em]
\item \texttt{/phi/idx/i} (int32),
      \texttt{/phi/idx/tau} (float64),
      \texttt{/phi/idx/iso} (bool),
      \texttt{/phi/idx/mu} (int32),
      \texttt{/phi/idx/nu} (int32).
\item \texttt{/phi/idx/flags/S1\_commutes},
      \texttt{/phi/idx/flags/S2\_noAccum},
      \texttt{/phi/idx/flags/S3\_Cauchy} (bool).
\item Optional witnesses:
      \texttt{/phi/idx/witness/ker\_generic\_dim},
      \texttt{/phi/idx/witness/coker\_generic\_dim}.
\item Optional tail:
      \texttt{/phi/idx/iso\_tail/passed} (bool).
\item \texttt{/phi/meta/stability\_bands}: records \((i,\tau_{\mathrm{lo}},\tau_{\mathrm{hi}})\).
\item Optional tower edges:
      \texttt{/phi/tower/edges/src},
      \texttt{/phi/tower/edges/dst} (int32),
      \texttt{/phi/tower/edges/kind} (fixed-length UTF-8 string).
\item Attributes:
      \texttt{schema\_version},
      \texttt{suite\_version},
      \texttt{string\_encoding}=\texttt{"utf8-fixed"},
      \texttt{tau\_sweep} (float64 array),
      and \texttt{links/Lambda\_len}.
\end{itemize}

% -------------------------
\subsection*{G.7. \texttt{Lambda\_len} (windowed length spectrum) schema}
% -------------------------
\paragraph{Semantics.}
The length spectrum operator \(\Lambda_{\mathrm{len}}(M;[0,\tau])\) is diagonal on the bar-basis with eigenvalues
equal to clipped bar-lengths on \([0,\tau]\). Its unordered eigenvalue multiset equals the clipped bar-length multiset
(Appendix~H). The \texttt{Lambda\_len} audit records either the eigenvalue list (small instances) or a content hash.

\noindent\textbf{JSON layout (links to all major artifacts).}
\begin{verbatim}
{
  "meta": {
    "schema_version": "2025-03-15",
    "suite_version": "v17.0",
    "definition": "Lambda_len(T_τ P_i(C_τ F); [0,τ])",
    "degree": 1,
    "tau": 0.50,
    "string_encoding": "utf8-fixed",
    "links": {
      "bars": "sha256:...bars",
      "phi":  "sha256:...phi",
      "spec": "sha256:...spec",
      "ext":  "sha256:...ext",
      "run_yaml_hash": "sha256:...run"
    }
  },
  "eigs": [0.24, 0.51, 0.78],
  "hash": "sha256:2f4c...d1"
}
\end{verbatim}

\noindent\textbf{HDF5 layout.}
\begin{itemize}[leftmargin=1.25em]
  \item \texttt{/Lambda\_len/meta} attributes:
  \texttt{schema\_version}, \texttt{suite\_version}, \texttt{degree}, \texttt{tau},
  \texttt{string\_encoding}=\texttt{"utf8-fixed"}, and links as above.
  \item \texttt{/Lambda\_len/eigs} (optional; float64 array).
  \item \texttt{/Lambda\_len/hash} (fixed-length UTF-8 string).
\end{itemize}

% -------------------------
\subsection*{G.8. Content hashing and canonical serialization}
% -------------------------
Each artifact carries a content hash \texttt{sha256:...} over its canonical serialization
(JSON with sorted keys; HDF5 with fixed dataset/attribute creation order, chunk shapes, compression and filters).
Cross-file links (\texttt{bars} \(\leftrightarrow\) \texttt{phi} \(\leftrightarrow\) \texttt{ext}
\(\leftrightarrow\) \texttt{spec} \(\leftrightarrow\) \texttt{Lambda\_len}) use these hashes exclusively.
\emph{JSON numeric policy:} finite numbers only; positive infinity is encoded as the string \texttt{"inf"} where applicable
(see \texttt{bars.meta.infinity}). HDF5 encodes \(+\infty\) via the split representation
\texttt{/bars/death} (float64) + \texttt{/bars/death\_is\_inf} (bool).
\emph{Strings:} all JSON strings and HDF5 string datasets/attributes are fixed-length UTF-8
(\texttt{string\_encoding}=\texttt{"utf8-fixed"}) to ensure bitwise reproducibility.
\emph{HDF5 canonicalization:} set \texttt{track\_times=false}, \texttt{shuffle=false}, \texttt{fletcher32=false},
\texttt{fillvalue=0.0}, compression to GZIP level~4, and use the \texttt{chunk\_shapes} recorded in \texttt{run.yaml};
create datasets and attributes in the order shown in this appendix.

% -------------------------
\subsection*{G.9. Numeric tolerances, \(\delta\)-budgets, and audit trail}
% -------------------------
Every quantitative claim includes:
\begin{itemize}[leftmargin=1.25em]
\item \textbf{tolerance} (\texttt{tol}) declared in \texttt{run.yaml};
\item \textbf{slack} (\texttt{slack}) measured margin to the decision boundary;
\item \textbf{norm} used for spectral bounds (\texttt{fro} or \texttt{op}), consistent with Appendix~E;
\item \textbf{metric} for persistence distances (\texttt{interleaving} or \texttt{bottleneck});
\item \textbf{budget aggregation:} \texttt{operations[*].delta} entries quantale-sum into \texttt{budget.sum\_delta},
with \texttt{budget.safety\_margin} and \texttt{budget.gap\_tau};
\item \textbf{layered \(\delta\)-ledger:} the stratified entries \texttt{layered\_delta.*} must sum to the aggregate via \texttt{quantale.op};
\item \textbf{solver} details for spectral computations (Lanczos parameters, RNG seed);
\item \textbf{windows/coverage:} \texttt{windows.*} declare scopes; \texttt{coverage\_check.*} record pass/fail;
\item \textbf{gate decision:} \texttt{gate.accept} with \texttt{gate.reason}.
\end{itemize}

% -------------------------
\subsection*{G.10. Tests T14/T15 and reproducibility checklist}
% -------------------------
\paragraph{T14 (Overlap Gate gluing).}
On a windowed cover, verify:
(i) post-collapse equality (up to budget) on overlaps (\texttt{overlap\_checks.local\_equiv=true}),
(ii) \v{C}ech--\(\Ext^1\) acyclicity on overlaps (\texttt{cech\_ext1\_ok=true}),
(iii) stability band detection (\(\mu=\nu=0\), \texttt{stability\_band\_ok=true}),
(iv) A/B soft-commuting with logged residuals (added to the \(\delta\)-ledger),
(v) global gate acceptance with additive budgets.

\paragraph{T15 (Length spectrum audit).}
Compute \(\Lambda_{\mathrm{len}}(\mathbf{T}_\tau\,\mathbf{P}_i(C_\tau F);[0,\tau])\) and verify that its eigenvalue multiset
equals the clipped bar-length multiset of \(\mathbf{T}_\tau\,\mathbf{P}_i(C_\tau F)\) (Appendix~H).
Log either the eigenvalue list or a content hash under \texttt{Lambda\_len}, and ensure cross-links
(\texttt{bars/spec/ext/phi}) resolve to the same hash.

\paragraph{Minimal reproducibility checklist.}
\begin{enumerate}[leftmargin=1.25em]
\item Preserve \texttt{run.yaml} and all emitted \texttt{bars/spec/ext/phi/Lambda\_len} files (JSON or HDF5).
\item Confirm \(\alpha>0\) (default \(\alpha=1\)) and field \(k\) are consistent across files.
\item Verify content hashes and all cross-links resolve; each artifact carries \texttt{meta.links.run\_id} and \texttt{run\_yaml\_hash}; all artifacts provide a link to \texttt{Lambda\_len}.
\item Check that \texttt{pipeline.metric} matches \texttt{thresholds.tol.distance} (exactly one of \texttt{interleaving}/\texttt{bottleneck}); norms/tolerances are consistent across \texttt{spec}.
\item Verify declared windows and that \texttt{coverage\_check.*} is true; for \texttt{definable}, check \texttt{o\_minimal\_structure}, \texttt{window\_formulae}, and \texttt{cech\_depth\_bound}.
\item Verify eigenvalue order metadata: \texttt{spec.meta.order}=\texttt{"ascending"}, \texttt{spec.meta.sorted}=true, and HDF5 eigen arrays are non-decreasing; if \texttt{tropical\_bins} are present, confirm width/range match \texttt{tropical.bins}.
\item For Dirichlet Laplacians, confirm \(\lambda_{\min}>0\) (or log a certified lower bound consistent with Appendix~E).
\item Ensure each recorded \(\theta\) used for \(N_\theta\) lies within \texttt{spec.meta.window.range}; recompute spectral indicators using recorded norm/tolerance/solver settings; check slack \(\ge 0\).
\item Re-evaluate \(\phi_{i,\tau}\) under the scope policy; confirm \((\mu,\nu)\) match generic-fiber counts (CNF after \(\mathbf{T}_\tau\)); check \texttt{iso\_tail.passed}; if \texttt{iwasawa} is present, verify \texttt{kernel\_leq}, \texttt{cokernel\_leq}. Confirm listed \texttt{stability\_bands}.
\item For Ext-tests, verify amplitude \([-1,0]\), the assumption flags in \texttt{ext.meta.assumptions}, and \(\Ext^1(\mathcal{R}(C_\tau F),k)=0\); check \texttt{persistence.Ext1\_zero}.
\item Validate HDF5 canonicalization: chunk shapes, compression, filters, \texttt{string\_encoding="utf8-fixed"}, and \texttt{track\_times=false}; death is represented via the split float/bool fields.
\item Inspect \texttt{operations}, \texttt{budget}, and \texttt{layered\_delta}: the stratified sums match \texttt{budget.sum\_delta} via \texttt{quantale.op}; compute safety margin and \texttt{gap\_tau}; verify \texttt{gate.accept}.
\item If \texttt{awfs\_2cell.awfs\_enabled}, verify recorded two-cell bounds and that \texttt{policy.after\_collapse\_only} holds.
\end{enumerate}

% -------------------------
\subsection*{G.11. Extended policy notes (concise)}
% -------------------------
\begin{itemize}[leftmargin=1.25em]
\item \textbf{Quantale.} The \texttt{quantale} section fixes the value-level monoid and order used to aggregate distances and budgets; all layered \(\delta\) entries combine via \texttt{quantale.op}.
\item \textbf{Definable windows.} Right-open windows are specified by first-order formulae in the declared \(o\hbox{-}minimal\) structure; the \v{C}ech depth bound controls gluing on overlaps.
\item \textbf{Layered \(\delta\).} The stratified ledger \((\delta^{\mathrm{Gal}},\delta^{\mathrm{Tr}},\delta^{\mathrm{Fun}})\) provides a MECE breakdown that must quantale-sum to the aggregate.
\item \textbf{Iwasawa control.} Finite bounds for kernel/cokernel dimensions at the recorded tower level certify control of \((\mu,\nu)\) along the tower.
\item \textbf{AWFS 2-cell.} Optional two-cell bounds certify functoriality/transport of \(C_\tau\) across mirrors/transfers/pullbacks.
\item \textbf{Safe low-pass.} The \texttt{low\_pass} block records kernel parity (even), unit mass, and alignment of clipping with the pipeline; these ensure non-expansiveness (Appendix~E).
\item \textbf{Tropical bins.} Optional diagnostic binning for aux-bars; if present, the same bin policy is mirrored in \texttt{spec.meta.tropical\_bins}.
\end{itemize}

\medskip
\noindent\textbf{Outcome.}
The versioned schemas above—now with (i) overlap checks for gluing, (ii) a windowed length-spectrum audit,
(iii) canonical spectral policy (\texttt{order="ascending"}, \texttt{norm="op"|"fro"}, spectral bounds with safe low-pass),
(iv) \(\delta\)-ledger extensions with \texttt{layered\_delta} and \texttt{gap\_tau},
(v) quantale/definable/Iwasawa/AWFS/tropical hooks, (vi) explicit \texttt{stability\_bands},
and (vii) HDF5 canonicalization with fixed-length UTF-8 strings—are sufficient to regenerate all figures and claims in the main text
from first principles, within the constructible regime and under the filtered-colimit policy, while making accept/reject criteria explicit and auditable.
No further supplementation is required for operational deployment or third-party review.



% =========================
\section*{Appendix H. Betti Integral and Finite \texorpdfstring{$\tau$}{tau}-Events (Reinforced)}
% =========================
\phantomsection
\addcontentsline{toc}{section}{Appendix H. Betti Integral and Finite $\tau$-Events}
\refstepcounter{section}
\label{H:betti-integral}

\paragraph{Standing conventions.}
We work over a field \(k\).
All persistence modules are constructible (locally finite on bounded windows); filtered colimits, when used, are taken only under the scope policy of Appendix~A, Remark~\ref{A:rk:filtered-colimits}.
Endpoint conventions follow Appendix~A, Remark~\ref{A:rk:endpoints}; we use half-open bars \([b,d)\) (any consistent endpoint convention yields the same integrals, cf.\ Remark~\ref{H:rk:endpoints}).
Global conventions: Ext-tests are always against \(k[0]\) (we write \(\Ext^1(\mathcal{R}(C_\tau F),k)=0\), with \(C_\tau\) understood up to f.q.i.\ on \(\Ho(\mathsf{FiltCh}(k))\)); the \emph{energy exponent} satisfies \(\alpha>0\) (default \(\alpha=1\)).
Window and overlap claims, coverage checks, and acceptance gates are recorded in the manifest \texttt{run.yaml} (Appendix~\ref{G:repro}).
Type labels follow \emph{Type I--II / Type III / Type IV}.

% -------------------------
\subsection*{H.1. Betti curves and the Betti integral}
% -------------------------
Let \(F\) be a filtered chain complex (or any filtered object realizing a persistence module) and let \(\mathbf{P}_i(F)\) denote its degree-\(i\) persistence module.
Write the corresponding barcode decomposition (constructible, locally finite on bounded windows) as a multiset
\[
\mathbf{P}_i(F)\ \cong\ \bigoplus_{I\in \mathcal{B}_i(F)} I^{\oplus m(I)},\qquad
I=[b,d),\ \ d\in\mathbb{R}\cup\{\infty\},\ \ m(I)\in\mathbb{Z}_{\ge 1}.
\]
Define the Betti curve \(\beta_i(t):=\dim_k H_i(F^t)\) and the \emph{Betti integral} up to \(\tau\ge 0\) by
\[
\mathrm{PE}_i^{\le \tau}(F)\ :=\ \int_{0}^{\tau} \beta_i(t)\,dt.
\]
Under constructibility, \(\beta_i\) is right-continuous and piecewise constant, and on any bounded window only finitely many bars meet.

\begin{theorem}[Betti integral \(=\) clipped barcode mass]\label{H:thm:betti-integral}
For every \(\tau\ge 0\),
\[
\mathrm{PE}_i^{\le \tau}(F)\ =\ \sum_{I\in \mathcal{B}_i(F)} m(I)\cdot \lambda\!\left(I\cap[0,\tau]\right),
\]
where \(\lambda\) is Lebesgue measure and
\[
\lambda([b,d)\cap[0,\tau])=\max\{0,\min\{d,\tau\}-\max\{b,0\}\}\quad(\min\{\infty,\tau\}=\tau).
\]
In particular, an infinite bar alive at \(0\) contributes its clipped length \(\tau\).
\end{theorem}

\begin{proof}
By local finiteness, for each bounded window \([0,\tau]\) only finitely many bars intersect the window, so
\[
f(t)\ :=\ \sum_{I\in\mathcal{B}_i(F)} m(I)\,\mathbf{1}_I(t)
\]
is a nonnegative measurable function on \([0,\tau]\) given by a finite sum.
With the half-open convention \([b,d)\), for all \(t\) away from event times (births \(b\) and deaths \(d\)) we have \(\beta_i(t)=f(t)\);
and since \(\beta_i\) is right-continuous, \(\beta_i=f\) almost everywhere on \([0,\tau]\).
Thus, by Tonelli/Fubini on bounded windows,
\[
\int_{0}^{\tau}\!\beta_i(t)\,dt
=\int_{0}^{\tau}\!f(t)\,dt
=\sum_{I} m(I)\int_{0}^{\tau}\!\mathbf{1}_{I}(t)\,dt
=\sum_{I} m(I)\,\lambda\!\left(I\cap[0,\tau]\right).
\]
\end{proof}

\begin{corollary}[Monotonicity, a.e.\ derivative, piecewise linearity]\label{H:cor:pl}
The map \(\tau\mapsto \mathrm{PE}_i^{\le \tau}(F)\) is nondecreasing, continuous, and piecewise linear on every bounded interval.
Its derivative satisfies
\[
\frac{d}{d\tau}\,\mathrm{PE}_i^{\le \tau}(F)=\beta_i(\tau)\quad\text{for a.e.\ }\tau,
\]
and at event points (births/deaths) the right derivative equals \(\beta_i(\tau)\) while the left derivative equals \(\beta_i(\tau-)\).
All breakpoints on \([0,\tau_0]\) lie in \(\{0,\tau_0\}\cup\{b\in[0,\tau_0]\}\cup\{d\in[0,\tau_0]\}\).
\end{corollary}

\begin{remark}[Endpoint and baseline conventions]\label{H:rk:endpoints}
Changing open/closed endpoint conventions modifies \(\beta_i\) only on a set of measure zero; the integral and breakpoint set remain unchanged.
The baseline \(0\) is a reference; negative births are allowed and handled by intersecting \(I\) with \([0,\tau]\).
\end{remark}

\begin{remark}[Energy exponent and \(\alpha\)-Betti integral]\label{H:rk:alpha}
For \(\alpha>0\), define the \(\alpha\)-Betti integral up to \(\tau\ge 0\) by
\[
\mathrm{PE}_{i,\alpha}^{\le \tau}(F)\ :=\ \int_{0}^{\tau} \bigl(\beta_i(t)\bigr)^{\alpha}\,dt.
\]
On each component of \([0,\tau]\) between consecutive event times, \(\beta_i\) is constant; hence \(\mathrm{PE}_{i,\alpha}^{\le \tau}\) is continuous, nondecreasing, and piecewise linear in \(\tau\),
with slope \(\bigl(\beta_i(\tau)\bigr)^{\alpha}\) on right-open pieces.
The case \(\alpha=1\) recovers Theorem~\ref{H:thm:betti-integral}.
For \(\alpha\neq 1\), \(\mathrm{PE}_{i,\alpha}^{\le \tau}\) is not additive over bars (because \(x\mapsto x^\alpha\) is nonlinear),
but all finite-check and implementation statements below remain valid verbatim (replace \(\beta_i\) by \(\beta_i^{\alpha}\) when computing segment slopes).
\end{remark}

\begin{corollary}[Trace/length-spectrum identity for \(\alpha=1\)]\label{H:cor:lambdalen-trace}
Let \(\Lambda_{\mathrm{len}}(\mathbf{P}_i(F);[0,\tau])\) denote the length-spectrum operator of Appendix~G (\S G.7), diagonal in the bar-basis with diagonal entries
\(\ell_\tau(I):=\lambda(I\cap[0,\tau])\) repeated with multiplicity \(m(I)\).
Then
\[
\operatorname{tr}\!\bigl(\Lambda_{\mathrm{len}}(\mathbf{P}_i(F);[0,\tau])\bigr)
=\sum_{I} m(I)\,\ell_\tau(I)
=\mathrm{PE}_i^{\le \tau}(F).
\]
Consequently, the \texttt{Lambda\_len} audit (Appendix~G) is a trace-level certificate of Theorem~\ref{H:thm:betti-integral} when \(\alpha=1\).
\end{corollary}

% -------------------------
\subsection*{H.2. Finite \texorpdfstring{$\tau$}{tau}-events and finite checking sets}
% -------------------------
Fix \(\tau_0>0\) and define the finite \(\tau\)-event set
\[
\mathsf{Ev}_i(F;\tau_0)\ :=\ \{0,\tau_0\}\ \cup\
\bigl(\{b\mid [b,d)\in\mathcal{B}_i(F)\}\cap[0,\tau_0]\bigr)\ \cup\
\bigl(\{d\mid [b,d)\in\mathcal{B}_i(F)\}\cap[0,\tau_0]\bigr).
\]
By constructibility, \(\mathsf{Ev}_i(F;\tau_0)\) is finite.

\begin{proposition}[Finite checking set]\label{H:prop:finite-check}
Let \(g:[0,\tau_0]\to\mathbb{R}\) be continuous and affine on each connected component of \([0,\tau_0]\setminus \mathsf{Ev}_i(F;\tau_0)\)
(e.g.\ a piecewise linear benchmark with breakpoints in \(\mathsf{Ev}_i\)).
Then, for either inequality direction,
\[
\mathrm{PE}_i^{\le \tau}(F)\ \ge\ g(\tau)\quad(\text{resp.\ }\le)\quad\text{for all }\tau\in[0,\tau_0]
\]
holds if and only if it holds for all \(\tau\in \mathsf{Ev}_i(F;\tau_0)\).
\end{proposition}

\begin{proof}
Between consecutive event times, both \(\mathrm{PE}_i^{\le \tau}(F)\) and \(g(\tau)\) are affine in \(\tau\) (Corollary~\ref{H:cor:pl}).
Hence their difference \(h(\tau):=\mathrm{PE}_i^{\le \tau}(F)-g(\tau)\) is affine on each closed component \(J=[u,v]\subset[0,\tau_0]\setminus \mathsf{Ev}_i(F;\tau_0)\).
An affine function on a compact interval attains its extremum at an endpoint, so \(h\ge 0\) (resp.\ \(h\le 0\)) on \(J\) iff \(h(u)\ge 0\) and \(h(v)\ge 0\) (resp.\ \(\le 0\)).
Taking the union over all such \(J\) plus the singleton event points yields the claim.
\end{proof}

\begin{remark}[Definable piecewise constancy (o\mbox{-}minimal); finite checks]\label{H:rk:omin}
Under the definable windowing policy of Appendix~G (Declaration~\ref{dec:runyaml-extended}, \texttt{definable.o\_minimal\_structure}/\texttt{window\_formulae}),
each \(\beta_i\) on a bounded window is definable and integer-valued, hence piecewise constant with finitely many jumps.
Equivalently, \(\mathsf{Ev}_i(F;\tau_0)\) is finite and computable from definable data.
Consequently, all constraints on \(\mathrm{PE}_{i,\alpha}^{\le \tau}\) that are affine on components between events reduce to finitely many evaluations at \(\tau\in\mathsf{Ev}_i(F;\tau_0)\);
record the event list (or its hash), window declaration, and counts in the manifest \texttt{run.yaml} (Appendix~G).
\end{remark}

\begin{remark}[Algorithmic evaluation]\label{H:rk:algo}
Algorithm for evaluating \(\mathrm{PE}_{i,\alpha}^{\le \tau}\) on \([0,\tau_0]\):
\begin{enumerate}[leftmargin=1.25em]
  \item Collect all births and deaths intersecting \([0,\tau_0]\); sort to form \(\mathsf{Ev}_i(F;\tau_0)=\{0=t_0<t_1<\cdots<t_M=\tau_0\}\).
  \item For each segment \([t_j,t_{j+1})\), compute \(c_j:=\beta_i(t)\) for any \(t\in[t_j,t_{j+1})\) and set \(s_j:=c_j^{\alpha}\).
  \item For \(\tau\in[t_j,t_{j+1})\),
  \[
  \mathrm{PE}_{i,\alpha}^{\le \tau}
  =\sum_{\ell<j} s_\ell\,(t_{\ell+1}-t_\ell)\ +\ s_j\,(\tau-t_j).
  \]
\end{enumerate}
Complexity: \(O(M\log M)\) to form/sort \(\mathsf{Ev}_i\) and \(O(M)\) for accumulation.
Record the window definition, endpoint policy (right-open), tie-break rules, and event counts in \texttt{run.yaml} (Appendix~G).
\end{remark}

% -------------------------
\subsection*{H.3. Consequences for shifts, truncations, and window variation}
% -------------------------
\paragraph{(i) Baseline/interval form and shifts.}
For \(a<b\), define the interval Betti integral
\[
\mathrm{PE}_{i,\alpha}^{[a,b]}(F)\ :=\ \int_{a}^{b} \bigl(\beta_i(t)\bigr)^{\alpha}\,dt.
\]
Then \(\mathrm{PE}_{i,\alpha}^{\le \tau}(F)=\mathrm{PE}_{i,\alpha}^{[0,\tau]}(F)\).
For \(\varepsilon\in\mathbb{R}\), let \((S^\varepsilon F)^t:=F^{t+\varepsilon}\). Then for any \(a<b\),
\[
\mathrm{PE}_{i,\alpha}^{[a,b]}(S^\varepsilon F)=\mathrm{PE}_{i,\alpha}^{[a+\varepsilon,b+\varepsilon]}(F).
\]
In particular, when \(\varepsilon\ge 0\) and \(\sigma\ge 0\),
\[
\mathrm{PE}_{i,\alpha}^{\le \sigma}(S^\varepsilon F)=\mathrm{PE}_{i,\alpha}^{\le \sigma+\varepsilon}(F)-\mathrm{PE}_{i,\alpha}^{\le \varepsilon}(F).
\]

\paragraph{(ii) Truncation monotonicity (deletion-type).}
Let \(\mathbf{T}_{\tau'}\) denote bar-deletion at scale \(\tau'>0\) (Appendix~A). Then, for every \(\sigma>0\) and \(\alpha>0\),
\[
\mathrm{PE}_{i,\alpha}^{\le \sigma}\!\big(\mathbf{T}_{\tau'}(\mathbf{P}_i(F))\big)\ \le\ \mathrm{PE}_{i,\alpha}^{\le \sigma}\!\big(\mathbf{P}_i(F)\big),
\]
since \(\beta_i\) decreases pointwise under deletion and \(x\mapsto x^\alpha\) is nondecreasing on \(\mathbb{R}_{\ge 0}\) for \(\alpha>0\).
Moreover, \(\mathbf{T}_{\tau'}\) is \(1\)-Lipschitz in interleaving distance (Appendix~A).

\paragraph{(iii) Lipschitz in the window parameter.}
For \(0\le s\le \tau\) and \(\alpha>0\),
\[
\bigl|\mathrm{PE}_{i,\alpha}^{\le \tau}(F)-\mathrm{PE}_{i,\alpha}^{\le s}(F)\bigr|
=\int_{s}^{\tau}\bigl(\beta_i(t)\bigr)^{\alpha}\,dt
\ \le\ (\tau-s)\cdot \sup_{t\in[s,\tau]}\bigl(\beta_i(t)\bigr)^{\alpha}.
\]

% -------------------------
\subsection*{H.4. Stability under matchings and perturbations on bounded windows}
% -------------------------
For a window \([a,b]\), write \(N_i([a,b])\) for the total number of bars (counted with multiplicity) in \(\mathcal{B}_i(F)\cup\mathcal{B}_i(G)\) that intersect \([a,b]\).
(Any equivalent locally finite counting convention is acceptable, provided it is fixed and recorded.)

\begin{proposition}[Windowed perturbation bounds]\label{H:prop:bounded-stability}
Fix \(\tau_0>0\). Suppose two barcodes \(\mathcal{B}_i(F)\) and \(\mathcal{B}_i(G)\) admit a bottleneck \(\delta\)-matching on \([-\delta,\tau_0+\delta]\):
each matched pair \([b,d)\leftrightarrow[b',d')\) satisfies \(|b-b'|\le\delta\), \(|d-d'|\le\delta\) (with \(d=\infty\) allowed), and every unmatched bar has length \(\le 2\delta\).
Then:
\begin{enumerate}[leftmargin=1.25em]
\item (\(\alpha=1\)) For all \(\tau\in[0,\tau_0]\),
\[
\bigl|\mathrm{PE}_{i}^{\le \tau}(F)-\mathrm{PE}_{i}^{\le \tau}(G)\bigr|
\ \le\ 2\,\delta\cdot N_i([-\delta,\tau_0+\delta]).
\]
\item (\(\alpha\ge 1\)) Let \(B:=\sup_{t\in[-\delta,\tau_0+\delta]}\max\{\beta_i^F(t),\beta_i^G(t)\}\) (finite by constructibility).
Then for all \(\tau\in[0,\tau_0]\),
\[
\bigl|\mathrm{PE}_{i,\alpha}^{\le \tau}(F)-\mathrm{PE}_{i,\alpha}^{\le \tau}(G)\bigr|
\ \le\ 2\,\alpha\,B^{\alpha-1}\,\delta\cdot N_i([-\delta,\tau_0+\delta]).
\]
\item (\(0<\alpha\le 1\)) For all \(\tau\in[0,\tau_0]\),
\[
\bigl|\mathrm{PE}_{i,\alpha}^{\le \tau}(F)-\mathrm{PE}_{i,\alpha}^{\le \tau}(G)\bigr|
\ \le\ 2\,\delta^{\alpha}\cdot N_i([-\delta,\tau_0+\delta]).
\]
\end{enumerate}
\end{proposition}

\begin{proof}[Proof sketch]
For \(\alpha=1\), Theorem~\ref{H:thm:betti-integral} expresses \(\mathrm{PE}_i^{\le \tau}\) as a sum of clipped lengths on \([0,\tau]\).
Endpoint perturbations by \(\delta\) change each matched clipped length by at most \(2\delta\); each unmatched bar contributes at most \(2\delta\).
Summing over all bars intersecting \([-\delta,\tau_0+\delta]\) yields (1).
For \(\alpha\ge 1\), use the pointwise bound \(|x^\alpha-y^\alpha|\le \alpha\,\max\{x,y\}^{\alpha-1}|x-y|\) and the fact that \(\beta_i\) changes only at event times, whose total count on the enlarged window is controlled by \(N_i([-\delta,\tau_0+\delta])\).
For \(0<\alpha\le 1\), use \(|x^\alpha-y^\alpha|\le |x-y|^\alpha\) on \(\mathbb{R}_{\ge 0}\) and the same windowed counting argument.
\end{proof}

\begin{remark}[Metric policy]
The manifest \texttt{run.yaml} (Appendix~G) records whether the pipeline uses \texttt{interleaving} or \texttt{bottleneck} for distance reporting.
Proposition~\ref{H:prop:bounded-stability} is stated for a bottleneck-style endpoint control; it applies whenever a bottleneck bound is derived or logged
(e.g.\ by stability reductions or explicit matchings). For interleaving-only runs, use a conversion lemma (if available in the implementation) or log the induced bottleneck bound explicitly.
\end{remark}

% -------------------------
\subsection*{H.5. Implementation notes and numerics}
% -------------------------
For large barcodes, the following practices improve reproducibility and numerical stability:
\begin{enumerate}[leftmargin=1.25em]
  \item \textbf{Event extraction:} derive \(\mathsf{Ev}_i(F;\tau_0)\) directly from the barcode; for streamed persistence, emit births/deaths as they occur and maintain a running count \(c_j\).
  \item \textbf{Accumulation:} use compensated summation (e.g.\ Kahan) when aggregating \(s_j\,(t_{j+1}-t_j)\).
  \item \textbf{Types:} store event times as float64; store counts \(c_j\) as 64-bit integers; compute slopes as float64.
  \item \textbf{Idempotence:} with \([b,d)\), repeated evaluation on the same event sequence is bitwise deterministic (fixed sort and tie-break rules on equal times).
  \item \textbf{Window policy:} record in \texttt{run.yaml} the baseline, window \([0,\tau_0]\), endpoint convention (right-open), and whether negative births are present (Appendix~G).
\end{enumerate}

% -------------------------
\subsection*{H.6. Testing and validation}
% -------------------------
Minimal tests:
\begin{enumerate}[leftmargin=1.25em]
  \item \textbf{Synthetic bars:} verify Theorem~\ref{H:thm:betti-integral} and cross-check numerical integration vs.\ clipped-length summation (\(\alpha=1\)).
  \item \textbf{Endpoint consistency:} switch between \([b,d)\) and any other consistent convention and verify identical \(\mathrm{PE}_{i,\alpha}^{\le \tau}\) and breakpoint sets.
  \item \textbf{Shift equivariance:} random \(\varepsilon\); check \(\mathrm{PE}_{i,\alpha}^{[a,b]}(S^\varepsilon F)=\mathrm{PE}_{i,\alpha}^{[a+\varepsilon,b+\varepsilon]}(F)\).
  \item \textbf{Truncation monotonicity:} apply \(\mathbf{T}_{\tau'}\); verify pointwise decrease in \(\sigma\).
  \item \textbf{Stability:} simulate \(\delta\)-endpoint perturbations; confirm Proposition~\ref{H:prop:bounded-stability} (casework by \(\alpha\)).
  \item \textbf{Length-spectrum audit:} for \(\alpha=1\), verify Corollary~\ref{H:cor:lambdalen-trace} and the multiset identity logged in \texttt{Lambda\_len} (Appendix~G).
\end{enumerate}

% -------------------------
\subsection*{H.7. Variants and generalizations}
% -------------------------
\begin{itemize}[leftmargin=1.25em]
  \item \textbf{Weighted windows.}
  For nonnegative \(w\in L^1([0,\tau_0])\), define
  \[
  \mathrm{PE}_{i,\alpha}^{w}(F)\ :=\ \int_{0}^{\tau_0} w(t)\,\bigl(\beta_i(t)\bigr)^{\alpha}\,dt.
  \]
  If \(w\) is piecewise constant with breakpoints contained in \(\mathsf{Ev}_i(F;\tau_0)\), then \(\mathrm{PE}_{i,\alpha}^{w}\) reduces to a finite sum over event intervals.
  For \(\alpha=1\), one additionally has the barcode-level identity
  \[
  \mathrm{PE}_{i}^{w}(F)=\sum_{I\in\mathcal{B}_i(F)} m(I)\int_{I\cap[0,\tau_0]} w(t)\,dt,
  \]
  by linearity of \(\beta_i(t)=\sum_I m(I)\mathbf{1}_I(t)\).
  \item \textbf{Alternate baselines.}
  Replacing \([0,\tau]\) by \([a,b]\) yields (for \(\alpha=1\))
  \[
  \mathrm{PE}_i^{[a,b]}(F)=\sum_{I} m(I)\,\lambda(I\cap[a,b]),
  \]
  and for general \(\alpha>0\) one uses \(\mathrm{PE}_{i,\alpha}^{[a,b]}(F)=\int_a^b (\beta_i(t))^\alpha\,dt\).
  \item \textbf{Discrete filtrations.}
  Replace integrals by Riemann sums on a grid; all statements adapt with counting measure and the event set replaced by grid points.
\end{itemize}

% -------------------------
\subsection*{H.8. $E_1$-level determinacy and bridge notes (window-local) [Spec]}
% -------------------------
\paragraph{Determinacy of $E_1=0$ on finite $\tau$-events.}
In the Overlap Gate, any constraint of the form
\[
\mathrm{PE}_{i,\alpha}^{\le \tau}(F)\ \le\ g(\tau)\quad\text{or}\quad \ge g(\tau)
\]
with \(g\) continuous and affine on components between event times holds on \([0,\tau_0]\) iff it holds at the finite set
\(\tau\in\mathsf{Ev}_i(F;\tau_0)\) (Proposition~\ref{H:prop:finite-check}).
Under the definable windowing policy (Appendix~G, Declaration~\ref{dec:runyaml-extended}), \(\mathsf{Ev}_i(F;\tau_0)\) is computable and auditable
(Remark~\ref{H:rk:omin}). Record the event list (or hash), window declaration, and pass/fail results in \texttt{run.yaml} (Appendix~G).

\paragraph{Bridge note (field/cohomological amplitude) [Spec].}
Let \(R\in D^b(k\text{-}\mathrm{mod})\) have cohomological amplitude contained in \([-1,0]\).
Over a field \(k\), the universal coefficient spectral sequence collapses and yields a canonical isomorphism
\[
\Ext^1(R,k)\ \cong\ \Hom\bigl(H^{-1}(R),k\bigr).
\]
Therefore \(\Ext^1(R,k)=0 \Rightarrow H^{-1}(R)=0\).
In the pipeline, the operational gate uses only the proven forward implication
\(\mathrm{PH}_1 \Rightarrow \Ext^1\) (Appendix~C/D policy; logged assumptions in Appendix~G), and does \emph{not} rely on any global equivalence.
A reverse implication \(\Ext^1\Rightarrow \mathrm{PH}_1\) may be audited \emph{window-locally} only when additional, explicitly logged hypotheses identify
\(H^{-1}(\mathcal{R}(C_\tau F))\) with the relevant persistence obstruction on that window (e.g.\ via the Local Equiv/Overlap Gate prerequisites);
this optional audit is outside the mandatory gate and must be marked as such in \texttt{run.yaml}.

\medskip
\noindent\textbf{Summary.}
The Betti integral equals clipped barcode mass for \(\alpha=1\) (Theorem~\ref{H:thm:betti-integral}), hence \(\mathrm{PE}_i^{\le \tau}\) is continuous, nondecreasing, and piecewise linear with breakpoints among births/deaths (Corollary~\ref{H:cor:pl}).
Any affine-on-components constraint reduces to the finite event set (Proposition~\ref{H:prop:finite-check}); under the \(o\mbox{-}minimal\) definable policy (Appendix~G),
these events are a priori finite and computable (Remark~\ref{H:rk:omin}), so \(E_1\)-level checks in the Overlap Gate become finite, auditable checklists.
Deletion-type truncations make \(\mathrm{PE}_{i,\alpha}^{\le \tau}\) nonincreasing and preserve \(1\)-Lipschitz stability in interleaving distance (Appendix~A).
Windowed perturbation bounds (Proposition~\ref{H:prop:bounded-stability}) yield practical stability on bounded ranges.
Finally, the \texttt{Lambda\_len} audit (Appendix~G) provides an independent length-spectrum certificate of the \(\alpha=1\) identity via trace and multiset checks
(Corollary~\ref{H:cor:lambdalen-trace}); any reverse-bridge reasoning is optional, window-local, and must be explicitly logged as [Spec].



% =========================
\section*{Appendix I. \texorpdfstring{$\varepsilon$}{epsilon}-Survival Lemma and Grid-to-Continuum [Proof/Spec] (reinforced)}
% =========================
\phantomsection
\addcontentsline{toc}{section}{Appendix I. $\varepsilon$-Survival Lemma and Grid-to-Continuum}
\refstepcounter{section}
\label{I:eps-survival}

\paragraph{Standing conventions.}
We work over a field \(k\).
All persistence modules lie in \(\Pers^{\mathrm{ft}}_k\) (constructible; locally finite on bounded windows).
Any use of filtered colimits follows the scope policy of Appendix~A, Remark~\ref{A:rk:filtered-colimits}.
Global conventions: Ext-tests are always against \(k[0]\) (we write \(\Ext^1(\mathcal{R}(C_\tau F),k)=0\), with \(C_\tau\) understood up to f.q.i.\ on \(\Ho(\mathsf{FiltCh}(k))\));
the energy exponent satisfies \(\alpha>0\) (default \(\alpha=1\));
type dashes (Type~I--II / Type~III / Type~IV) are used uniformly.
We write \(d_{\mathrm{int}}\) for the interleaving metric, and in the 1-parameter constructible regime we identify it with the bottleneck metric via the usual isometry theorem (when this identification is invoked, it is explicitly stated below).
Window policy, right-open conventions, coverage checks, \(\delta\)-budgets, and gating are as in Appendix~\ref{G:repro}.
All survival/stability statements in the pipeline are applied \emph{after} the mandatory order
\[
\boxed{\ \text{for each }t\ \Longrightarrow\ \mathbf{P}_i\ \Longrightarrow\ \mathbf{T}_\tau\ \Longrightarrow\ \text{compare in }\Pers^{\mathrm{ft}}_k\ },
\]
and any additional window restriction is implemented by \(\mathbf{W}_{\le\tau_0}\) defined in \S I.1.

\begin{remark}[Endpoint conventions]\label{I:rk:endpoints}
Intervals are taken half-open \([b,d)\) with \(d\in\mathbb{R}\cup\{\infty\}\).
Any consistent open/closed choice yields the same clipped lengths and event sets on bounded windows (cf.\ Appendix~H, Remark~\ref{H:rk:endpoints});
all quantitative statements below are invariant under this choice.
\end{remark}

% -------------------------
\subsection*{I.1. Window clipping and nonexpansivity}
% -------------------------
Let \(\Pers^{\mathrm{ft}}_k\) denote constructible persistence modules on \((\mathbb{R},\le)\).
For \(\tau\ge 0\), write \(i_{\le\tau}:[0,\tau]\hookrightarrow\mathbb{R}\) for the inclusion and define the \emph{window clip} endofunctor
\[
\mathbf{W}_{\le\tau}\ :=\ (i_{\le\tau})^{0}_!\circ i_{\le\tau}^{\ast}:\ \Pers^{\mathrm{ft}}_k\longrightarrow \Pers^{\mathrm{ft}}_k,
\]
i.e.\ restriction to \([0,\tau]\) followed by extension by \(0\) outside \([0,\tau]\).
On barcodes, \(\mathbf{W}_{\le\tau}\) corresponds to intersecting bars with \([0,\tau]\) and discarding empty intersections.

For a bar \(I=[b,d)\), its clipped length on \([0,\tau_0]\) is
\[
\ell_{[0,\tau_0]}(I)
:=\lambda(I\cap[0,\tau_0])
=\max\{0,\min\{d,\tau_0\}-\max\{b,0\}\},
\qquad \min\{\infty,\tau_0\}=\tau_0.
\]

\begin{remark}[Shifts commute with clipping]\label{I:rk:shift-clip}
For every \(\varepsilon\ge 0\), there is a natural isomorphism
\[
S^\varepsilon\circ \mathbf{W}_{\le\tau}\ \cong\ \mathbf{W}_{\le\tau}\circ S^\varepsilon.
\]
Consequently, any \(\varepsilon\)-interleaving transports through \(\mathbf{W}_{\le\tau}\).
\end{remark}

\begin{lemma}[Clipping is \(1\)-Lipschitz]\label{I:lem:clip-1lip}
For all \(M,N\in\Pers^{\mathrm{ft}}_k\) and \(\tau\ge 0\),
\[
d_{\mathrm{int}}\bigl(\mathbf{W}_{\le\tau}M,\ \mathbf{W}_{\le\tau}N\bigr)\ \le\ d_{\mathrm{int}}(M,N).
\]
\end{lemma}

\begin{proof}
If \(M,N\) are \(\varepsilon\)-interleaved, there are morphisms \(f:M\to S^\varepsilon N\) and \(g:N\to S^\varepsilon M\) satisfying the standard triangle conditions.
Applying \(\mathbf{W}_{\le\tau}\) and using Remark~\ref{I:rk:shift-clip} yields morphisms
\(\mathbf{W}_{\le\tau}M\to S^\varepsilon \mathbf{W}_{\le\tau}N\) and \(\mathbf{W}_{\le\tau}N\to S^\varepsilon \mathbf{W}_{\le\tau}M\)
satisfying the same triangle conditions. Hence \(\mathbf{W}_{\le\tau}M,\mathbf{W}_{\le\tau}N\) are \(\varepsilon\)-interleaved.
Taking the infimum over \(\varepsilon\) gives the claim.
\end{proof}

% -------------------------
\subsection*{I.2. Survival under \texorpdfstring{$\varepsilon$}{epsilon}-interleavings}
% -------------------------
We state a sharp clipped-length survival estimate. When we refer to an \(\varepsilon\)-matching of bars, we use the 1D isometry theorem in the constructible regime to pass from
\(d_{\mathrm{int}}\le\varepsilon\) to a bottleneck-style endpoint matching.

\begin{lemma}[\(\varepsilon\)-survival; sharp two-sided clipped-length form]\label{I:lem:survive-2eps}
Let \(M,N\in\Pers^{\mathrm{ft}}_k\) satisfy \(d_{\mathrm{int}}(M,N)\le\varepsilon\), and fix \(\tau_0>0\).
Assume we choose an \(\varepsilon\)-matching of bars (equivalently, an \(\varepsilon\)-bottleneck matching under the 1D isometry theorem).
Then:
\begin{enumerate}[leftmargin=1.25em]\itemsep0.25em
\item If a bar \(I\) of \(M\) is matched to a bar \(J\) of \(N\), then
\[
\ell_{[0,\tau_0]}(J)\ \ge\ \max\{\ell_{[0,\tau_0]}(I)-2\varepsilon,\ 0\}.
\]
\item If \(\ell_{[0,\tau_0]}(I)>2\varepsilon\), then its matched partner \(J\) satisfies \(\ell_{[0,\tau_0]}(J)>0\).
\item (\emph{Multiplicity}) If at least \(r\) bars in \(M\) have \(\ell_{[0,\tau_0]}(\cdot)>2\varepsilon\) (counted with multiplicity),
then \(\mathbf{W}_{\le\tau_0}N\) has at least \(r\) nonzero bars (with multiplicity).
\end{enumerate}
\end{lemma}

\begin{proof}
Under an \(\varepsilon\)-matching, matched endpoints satisfy \(|b-b'|\le\varepsilon\) and \(|d-d'|\le\varepsilon\) (with \(d=\infty\) allowed).
Clipping to \([0,\tau_0]\) changes the left truncation point \(\max\{b,0\}\) by at most \(\varepsilon\), and changes the right truncation point \(\min\{d,\tau_0\}\) by at most \(\varepsilon\).
Therefore the clipped length changes by at most \(2\varepsilon\), yielding (1). Items (2) and (3) follow immediately.
\end{proof}

\begin{remark}[After-collapse composition]\label{I:rk:after-collapse}
The bar-deletion reflector \(\mathbf{T}_\tau\) is \(1\)-Lipschitz for \(d_{\mathrm{int}}\) (Appendix~A).
Therefore Lemma~\ref{I:lem:survive-2eps} remains valid after pre/post-composition by \(\mathbf{T}_\tau\), and in particular it applies to the pipeline objects
\(\mathbf{W}_{\le\tau_0}\mathbf{T}_\tau\mathbf{P}_i(F)\) and \(\mathbf{W}_{\le\tau_0}\mathbf{T}_\tau\mathbf{P}_i(F_h)\).
\end{remark}

% -------------------------
\subsection*{I.3. Grid-to-continuum transfer}
% -------------------------
Let \(F\) be filtered with degree-\(i\) persistence \(\mathbf{P}_i(F)\), and let \(F_h\) be a discretization (grid approximation) such that
\[
d_{\mathrm{int}}\bigl(\mathbf{P}_i(F_h),\ \mathbf{P}_i(F)\bigr)\ \le\ \varepsilon(h).
\]
Fix a pipeline truncation scale \(\tau>0\) and a window \([0,\tau_0]\).
Define the post-collapse/windowed modules
\[
M_h:=\mathbf{W}_{\le\tau_0}\mathbf{T}_\tau\mathbf{P}_i(F_h),
\qquad
M:=\mathbf{W}_{\le\tau_0}\mathbf{T}_\tau\mathbf{P}_i(F).
\]
By nonexpansivity of \(\mathbf{T}_\tau\) and \(\mathbf{W}_{\le\tau_0}\),
\(
d_{\mathrm{int}}(M_h,M)\le \varepsilon(h).
\)

\begin{theorem}[Grid-to-continuum survival]\label{I:thm:g2c}
Fix \(\tau_0>0\), \(r\in\mathbb{Z}_{\ge 1}\), and \(\eta>0\).
If \(M_h=\mathbf{W}_{\le\tau_0}\mathbf{T}_\tau\mathbf{P}_i(F_h)\) has at least \(r\) bars of clipped length \(\ge 2\varepsilon(h)+\eta\),
then \(M=\mathbf{W}_{\le\tau_0}\mathbf{T}_\tau\mathbf{P}_i(F)\) has at least \(r\) nonzero bars, each of clipped length \(\ge \eta\).
\end{theorem}

\begin{proof}
Apply Lemma~\ref{I:lem:survive-2eps} to \(M_h,M\) with \(\varepsilon=\varepsilon(h)\).
\end{proof}

% -------------------------
\subsection*{I.4. Budget-adjusted and \texorpdfstring{$V$}{V}-metric variants [Spec]}
% -------------------------
We now incorporate (i) the \(\delta\)-ledger/budget of Appendix~G and (ii) an optional \(V\)-enriched (quantale) distance used in the suite.

\paragraph{Setup.}
Let \(d_V\) be a Lawvere/quantale distance on \(\Pers^{\mathrm{ft}}_k\) such that both \(\mathbf{W}_{\le\tau}\) and \(\mathbf{T}_\tau\) are \(1\)-Lipschitz.
Let \(W=[0,\tau_0]\) be the active window. Let \(\Delta_W\) denote the window-restricted budget aggregate obtained by combining the operation-level residuals
\(\texttt{operations[*].delta.total}\) via the quantale operation \(\oplus\) recorded in \texttt{run.yaml.quantale.op} (Appendix~G).
(In the default \([0,\infty]_{+}\) quantale, \(\oplus\) is ordinary addition.)
Define the \emph{effective radius}
\[
\boxed{\ \varepsilon_{\mathrm{eff}}\ :=\ d_V\bigl(\mathbf{P}_i(F_h),\mathbf{P}_i(F)\bigr)\ \oplus\ \Delta_W\ }.
\]

\begin{lemma}[\(\varepsilon\)-survival, budgeted \(V\)-metric]\label{I:lem:eps-clip-V}
With notation as above, consider
\(
M_h:=\mathbf{W}_{\le\tau_0}\mathbf{T}_\tau\mathbf{P}_i(F_h)
\)
and
\(
M:=\mathbf{W}_{\le\tau_0}\mathbf{T}_\tau\mathbf{P}_i(F).
\)
Assume \(d_V(M_h,M)\le \varepsilon_{\mathrm{eff}}\).
Then for any bar \(I\) in \(M_h\):
\begin{enumerate}[leftmargin=1.25em]\itemsep0.25em
\item (\emph{Two-sided, sharp}) If \(\ell_{[0,\tau_0]}(I)>\,2\,\varepsilon_{\mathrm{eff}}\), then the matched bar in \(M\) is nonzero; moreover its clipped length is
\[
\ge\ \ell_{[0,\tau_0]}(I)\ \ominus\ 2\,\varepsilon_{\mathrm{eff}},
\]
where \(\ominus\) denotes the truncated subtraction in the default additive case (i.e.\ \(\max\{x-2\varepsilon_{\mathrm{eff}},0\}\)); equivalently, the loss is bounded by \(2\varepsilon_{\mathrm{eff}}\).
\item (\emph{One-sided improvement; optional} [Spec]) If the implementation enforces a one-sided control in which births are anchored and only the death-side may drift by \(\le \varepsilon_{\mathrm{eff}}\),
then the nonvanishing threshold improves to \(\ell_{[0,\tau_0]}(I)>\varepsilon_{\mathrm{eff}}\), and the residual margin degrades by at most \(\varepsilon_{\mathrm{eff}}\).
\end{enumerate}
\end{lemma}

\begin{proof}[Proof sketch]
Nonexpansivity of \(\mathbf{W}_{\le\tau_0}\) and \(\mathbf{T}_\tau\) in \(d_V\) yields \(d_V(M_h,M)\le \varepsilon_{\mathrm{eff}}\).
In the default additive regime, the two-sided survival bound follows by the same endpoint-motion argument as Lemma~\ref{I:lem:survive-2eps}, with \(\varepsilon\) replaced by \(\varepsilon_{\mathrm{eff}}\).
The one-sided improvement is implementation-specific and follows if only one endpoint is permitted to drift.
\end{proof}

\begin{corollary}[Budgeted grid-to-continuum]\label{I:cor:g2c-V}
If at least \(r\) clipped bars in \(M_h=\mathbf{W}_{\le\tau_0}\mathbf{T}_\tau\mathbf{P}_i(F_h)\) have length \(\ge 2\varepsilon_{\mathrm{eff}}+\eta\),
then \(M=\mathbf{W}_{\le\tau_0}\mathbf{T}_\tau\mathbf{P}_i(F)\) has at least \(r\) nonzero bars of clipped length \(\ge \eta\).
Under the one-sided hypothesis of Lemma~\ref{I:lem:eps-clip-V}(2), replace \(2\) by \(1\).
\end{corollary}

\begin{remark}[Tropical bins and aux-bars; persistence-side diagnostic]\label{I:rk:tropical-bins}
If a diagnostic bin width \(\beta>0\) is declared (Appendix~G, \texttt{tropical.bins.width}), then an \(\varepsilon_{\mathrm{eff}}\)-drift implies a histogram/bin-profile shift bounded by
\[
q=\Bigl\lceil \varepsilon_{\mathrm{eff}}/\beta\Bigr\rceil
\]
bins (in the default additive regime). When this diagnostic is used, record \texttt{eps\_cont\_shift\_bins}\(=q\) in \texttt{run.yaml}.
\end{remark}

\begin{remark}[Logging (minimal contract)]\label{I:rk:logging}
Record in \texttt{run.yaml} (Appendix~G): the metric name (\texttt{pipeline.metric} or \texttt{quantale.name}/\(d_V\) identifier);
the active window \(W=[0,\tau_0]\) under \path{windows.domain.filtration_range};
the contributing \(\delta\)-entries and their aggregation rule (\texttt{quantale.op}, \texttt{budget.sum\_delta}, \texttt{layered\_delta});
the computed \(\varepsilon_{\mathrm{eff}}\);
counts of bars above \(2\varepsilon_{\mathrm{eff}}+\eta\) (or \(\varepsilon_{\mathrm{eff}}+\eta\) in one-sided mode);
and confirmation that \texttt{policy.after\_collapse\_only=true}.
\end{remark}

\begin{corollary}[Betti-integral lower bound (link to Appendix~H)]\label{I:cor:betti-lb}
If \(M_h=\mathbf{W}_{\le\tau_0}\mathbf{T}_\tau\mathbf{P}_i(F_h)\) contains \(r\) bars of clipped length \(\ge 2\varepsilon_{\mathrm{eff}}+\eta\),
then
\[
\mathrm{PE}_i^{\le \tau_0}\bigl(\mathbf{T}_\tau\mathbf{P}_i(F)\bigr)\ \ge\ r\,\eta,
\]
by Theorem~\ref{H:thm:betti-integral} applied to \(\mathbf{T}_\tau\mathbf{P}_i(F)\) and Corollary~\ref{I:cor:g2c-V}.
Since \(\mathbf{T}_\tau\) is deletion-type, \(\mathrm{PE}_i^{\le \tau_0}(\mathbf{P}_i(F))\ge \mathrm{PE}_i^{\le \tau_0}(\mathbf{T}_\tau\mathbf{P}_i(F))\), so the same lower bound holds \emph{a fortiori} for \(\mathbf{P}_i(F)\).
\end{corollary}

% -------------------------
\subsection*{I.5. Variants, sharpness, and towers}
% -------------------------
\begin{itemize}[leftmargin=1.25em]\itemsep0.25em
\item \emph{Sharpness.} The constant \(2\) in Lemma~\ref{I:lem:survive-2eps} is optimal: a bar of clipped length \(2\varepsilon\) can be shifted by \(\varepsilon\) at both ends to vanish on \([0,\tau_0]\).
\item \emph{After-collapse pipeline.} Since \(\mathbf{T}_\tau\) is \(1\)-Lipschitz (Appendix~A), composing with \(\mathbf{T}_\tau\) does not change survival thresholds; all inequalities are preserved under the mandated order \(\mathbf{P}_i\to \mathbf{T}_\tau\).
\item \emph{Towers.} If \(d_V(\mathbf{P}_i(F_n),\mathbf{P}_i(F_\infty))\le\varepsilon_n\to 0\), then any fixed positive margin detected at stage \(n\) propagates to the limit by Corollary~\ref{I:cor:g2c-V}.
This interfaces with the tower comparison diagnostics \((\mu,\nu)\) and stability bands (Appendix~D, Appendix~G).
\end{itemize}

% -------------------------
\subsection*{I.6. Formalization stubs (Lean/Coq) [Spec]}
% -------------------------
\begin{verbatim}
-- Lean-style pseudocode (schematic)
namespace AK.I
open scoped Classical
noncomputable section

/-- Constructible persistence modules (schematic placeholder). -/
structure PersModule := (dummy : Unit)

/-- Window clip and truncation (schematic). -/
def W_le (τ : ℝ≥0) : PersModule → PersModule := sorry
def Tτ  (τ : ℝ≥0) : PersModule → PersModule := sorry

/-- V-distance (quantale-style) and nonexpansivity axioms. -/
def dV : PersModule → PersModule → ℝ≥0 := sorry
axiom W_le_non-expansive : ∀ τ M N, dV (W_le τ M) (W_le τ N) ≤ dV M N
axiom Tτ_non-expansive  : ∀ τ M N, dV (Tτ  τ M) (Tτ  τ N) ≤ dV M N

/-- Bars and clipped lengths (schematic). -/
constant Bar : Type
constant BarOf : PersModule → Type
def clipLen_on (τ0 : ℝ≥0) : Bar → ℝ≥0 := sorry

theorem eps_survival_two_sided
  (τ0 : ℝ≥0) (εeff : ℝ≥0) {Mh M : PersModule}
  (h : dV Mh M ≤ εeff)
  (I : BarOf Mh) (hI : clipLen_on τ0 (Classical.choice ⟨Bar, by trivial⟩) > 2*εeff) :
  ∃ J : BarOf M, True := by
  -- Push h through W_le and Tτ (non-expansive),
  -- then use endpoint drift ≤ εeff at both ends.
  admit

theorem eps_survival_one_sided  -- optional birth-anchored mode
  (τ0 : ℝ≥0) (εeff : ℝ≥0) {Mh M : PersModule}
  (h : dV Mh M ≤ εeff) (births_anchored : True)
  (I : BarOf Mh) (hI : clipLen_on τ0 (Classical.choice ⟨Bar, by trivial⟩) > εeff) :
  ∃ J : BarOf M, True := by
  admit

end AK.I
\end{verbatim}

% -------------------------
\subsection*{I.7. Summary}
% -------------------------
Clipping \(\mathbf{W}_{\le\tau}\) is restriction to \([0,\tau]\) followed by \(0\)-extension; it preserves constructibility and is \(1\)-Lipschitz (Lemma~\ref{I:lem:clip-1lip}).
Under an \(\varepsilon\)-interleaving (equivalently, an \(\varepsilon\)-matching in the 1D constructible regime), clipped lengths degrade by at most \(2\varepsilon\);
thus any bar with clip-length \(>\!2\varepsilon\) survives as a nonzero bar on the same window (Lemma~\ref{I:lem:survive-2eps}).
Applying this after the mandated order \(\mathbf{P}_i\to\mathbf{T}_\tau\) yields the grid-to-continuum survival transfer (Theorem~\ref{I:thm:g2c}).
In the budgeted \(V\)-metric setting, replace \(\varepsilon\) by the \emph{effective} radius
\[
\varepsilon_{\mathrm{eff}}=d_V\bigl(\mathbf{P}_i(F_h),\mathbf{P}_i(F)\bigr)\ \oplus\ \Delta_W,
\]
so the survival threshold becomes \(>\!2\,\varepsilon_{\mathrm{eff}}\) (or \(>\!\varepsilon_{\mathrm{eff}}\) under the optional one-sided, birth-anchored mode).
These statements commute with the after-collapse policy (\(\mathbf{T}_\tau\) is \(1\)-Lipschitz), tie into Betti-integral lower bounds (Appendix~H; Corollary~\ref{I:cor:betti-lb}),
and are logged via the \(\delta\)-ledger and window fields in Appendix~G.
No further supplementation is required for operational deployment or third-party audit.



% =========================
\section*{Appendix J. Calculus of $\mu,\nu$ [Proof + Stability Bands + Window Pasting] (reinforced)}
% =========================
\phantomsection
\addcontentsline{toc}{section}{Appendix J. Calculus of $\mu,\nu$}
\refstepcounter{section}
\label{J:calc}

\paragraph{Standing conventions.}
We work over a field \(k\).
All persistence modules lie in the one-parameter constructible persistence category \(\Pers^{\mathrm{cons}}_k\) (locally finite on bounded windows).
Filtered colimits, when used, are computed in the functor category \([\mathbb{R},\mathsf{Vect}_k]\) under the scope policy of Appendix~A, Remark~\ref{A:rk:filtered-colimits}, and are returned to the constructible range only when the policy conditions guarantee constructibility.
The bar-deletion reflector \(\mathbf{T}_\tau\dashv \iota_\tau\) is exact and \(1\)-Lipschitz (Appendix~A, Theorem~\ref{A:thm:localization} and Proposition~\ref{A:prop:lipschitz}).
Global conventions: \(\Ext^1\)-tests are always against \(k[0]\); the energy exponent \(\alpha>0\) (default \(\alpha=1\)).
All window policies and overlap conventions follow Appendix~G (MECE accounting with right-open primitives; overlap lists and finite-check manifests are mandatory).
All \((\mu,\nu)\) diagnostics are computed \emph{after} applying \(\mathbf{T}_\tau\) (B-side single-layer policy), and always under the same \(\tau\) and window record used by B-Gate\(^{+}\) and the \(\delta\)-ledger (Appendix~G).

% -------------------------
\subsection*{J.0. Setup: towers, comparison maps, and generic fiber dimension}
% -------------------------

\paragraph{Towers.}
Fix \(i\in\mathbb{Z}\).
Let \(F=(F_n)_{n\in I}\) be a directed system (``tower'') of filtered objects such that \(\mathbf{P}_i(F_n)\in\Pers^{\mathrm{cons}}_k\) for all \(n\).
Let \(F_\infty\) denote a chosen colimit object with cocone maps \(F_n\to F_\infty\).
For \(\tau\ge 0\) consider the \emph{comparison map} in \([\mathbb{R},\mathsf{Vect}_k]\),
\begin{equation}\label{J:eq:phi}
\phi_{i,\tau}(F):\quad
\varinjlim_{n}\ \mathbf{T}_\tau\big(\mathbf{P}_i(F_n)\big)\ \longrightarrow\ \mathbf{T}_\tau\big(\mathbf{P}_i(F_\infty)\big).
\end{equation}
Whenever \(\phi_{i,\tau}(F)\) is asserted to lie in \(\Pers^{\mathrm{cons}}_k\), this is by the scope policy hypotheses of Appendix~A and the window/definability policies of Appendix~G.

\paragraph{Generic fiber dimension.}
For a constructible module \(M\in\Pers^{\mathrm{cons}}_k\), there exists \(t_0\) such that \(\dim_k M(t)\) stabilizes for all \(t\ge t_0\).
Define the \emph{generic fiber dimension}
\[
\dim^{\mathrm{gen}}_k(M)\ :=\ \dim_k M(t)\ \ \text{for any (hence all) sufficiently large }t,
\]
equivalently the stabilized right-tail dimension.
In barcode terms, \(\dim^{\mathrm{gen}}_k(M)\) equals the total multiplicity of infinite bars \([b,\infty)\) (for any sufficiently large tail parameter), and after applying \(\mathbf{T}_\tau\) it is computed on the post-deletion barcode (Appendix~D, Remark~\ref{D:rem:generic-dim}).

\paragraph{Definition of \texorpdfstring{$\mu,\nu$}{mu,nu}.}
Define
\begin{equation}\label{J:eq:mu-nu}
\mu_{i,\tau}(F)\ :=\ \dim^{\mathrm{gen}}_k \ker \phi_{i,\tau}(F),
\qquad
\nu_{i,\tau}(F)\ :=\ \dim^{\mathrm{gen}}_k \operatorname{coker} \phi_{i,\tau}(F),
\end{equation}
with \(\ker,\operatorname{coker}\) taken in \([\mathbb{R},\mathsf{Vect}_k]\) and then interpreted in the constructible regime on the stabilized tail when permitted by policy.\footnote{Under constructibility, kernels/cokernels decompose into finite direct sums of interval modules on bounded windows, and \(\dim_k^{\mathrm{gen}}\) equals the stabilized tail dimension, hence is well-defined and integer-valued. The diagnostics \((\mu,\nu)\) are the tower-level ``infinite-bar'' defect measures after deletion, and are designed to detect Type~IV (invisible) failures.}
We write \(\DiagZero\) for the verdict \((\mu_{i,\tau}(F),\nu_{i,\tau}(F))=(0,0)\) and \(\DiagNonzero\) otherwise (Type~IV witness at \((i,\tau)\)).
We also write the totals
\begin{equation}\label{J:eq:totals}
\mu_{\mathrm{Collapse},\tau}(F) := \sum_i \mu_{i,\tau}(F),
\qquad
\nu_{\mathrm{Collapse},\tau}(F) := \sum_i \nu_{i,\tau}(F).
\end{equation}
which are finite when \(F\) is bounded in homological degrees (constructible range assumption).
All quantities depend on \(\tau\); \emph{no global monotonicity in \(\tau\) is asserted.}

% -------------------------
\subsection*{J.1. Functoriality, isomorphism invariance, and cofinality}
% -------------------------

\begin{definition}[Morphisms of towers]
A morphism \(u:F\to G\) consists of maps \(u_n:F_n\to G_n\) commuting with the structure maps and inducing a canonical map on colimits \(u_\infty:F_\infty\to G_\infty\).
\end{definition}

\begin{lemma}[Naturality of comparison maps]\label{J:lem:functoriality}
Under the scope policy, each tower morphism \(u:F\to G\) induces a commutative square
\[
\begin{tikzcd}[column sep=large,row sep=large]
\varinjlim_n \mathbf{T}_\tau\mathbf{P}_i(F_n) \arrow[r,"\phi_{i,\tau}(F)"] \arrow[d,"\varinjlim \mathbf{T}_\tau\mathbf{P}_i(u_n)"'] &
\mathbf{T}_\tau\mathbf{P}_i(F_\infty) \arrow[d,"\mathbf{T}_\tau\mathbf{P}_i(u_\infty)"] \\
\varinjlim_n \mathbf{T}_\tau\mathbf{P}_i(G_n) \arrow[r,"\phi_{i,\tau}(G)"'] &
\mathbf{T}_\tau\mathbf{P}_i(G_\infty),
\end{tikzcd}
\]
natural in both \(i\) and \(\tau\), and compatible with composition of tower morphisms.
\end{lemma}

\begin{proof}
Apply \(\mathbf{P}_i\), then \(\mathbf{T}_\tau\), and then filtered colimits in \([\mathbb{R},\mathsf{Vect}_k]\).
Exactness of \(\mathbf{T}_\tau\) and functoriality of colimits yield the square and compatibility with composition.
\end{proof}

\begin{remark}[Invariance under isomorphism of towers]\label{J:rk:iso-inv}
If \(u:F\to G\) is a tower isomorphism (levelwise and on apex), then \(\phi_{i,\tau}(F)\) and \(\phi_{i,\tau}(G)\) are isomorphic as morphisms; hence \(\mu_{i,\tau}(F)=\mu_{i,\tau}(G)\) and \(\nu_{i,\tau}(F)=\nu_{i,\tau}(G)\).
\end{remark}

\begin{definition}[Cofinal subindexing]
Let \(I\) be the directed index category for \(F\).
A full subcategory \(J\subset I\) is cofinal if for every \(i\in I\) there exists \(j\in J\) with a morphism \(i\to j\).
The restricted tower \(F|_J\) has the same colimit as \(F\) in \([\mathbb{R},\mathsf{Vect}_k]\).
\end{definition}

\begin{theorem}[Cofinal invariance]\label{J:thm:cofinal}
Let \(J\subset I\) be cofinal. Then for all \(\tau\ge 0\),
\[
\mu_{i,\tau}(F|_J)=\mu_{i,\tau}(F),
\qquad
\nu_{i,\tau}(F|_J)=\nu_{i,\tau}(F).
\]
Hence \(\DiagZero/\DiagNonzero\) agree as well.
\end{theorem}

\begin{proof}
Cofinal restriction does not change colimits in \([\mathbb{R},\mathsf{Vect}_k]\), hence does not change the source, target, or definition of \(\phi_{i,\tau}(F)\).
Kernels/cokernels and stabilized tail dimensions therefore agree.
\end{proof}

% -------------------------
\subsection*{J.2. Linear-algebra calculus for stabilized kernels/cokernels (P6)}
% -------------------------

\paragraph{A map-level notation (post-\texorpdfstring{$\mathbf{T}_\tau$}{Ttau}).}
For a morphism \(f:M\to N\) in the constructible regime (or in \([\mathbb{R},\mathsf{Vect}_k]\) under stabilization), define
\[
\mu^{\mathrm{gen}}(f)\ :=\ \dim^{\mathrm{gen}}_k\ker(f),
\qquad
\nu^{\mathrm{gen}}(f)\ :=\ \dim^{\mathrm{gen}}_k\operatorname{coker}(f).
\]
With this notation, \(\mu_{i,\tau}(F)=\mu^{\mathrm{gen}}(\phi_{i,\tau}(F))\) and \(\nu_{i,\tau}(F)=\nu^{\mathrm{gen}}(\phi_{i,\tau}(F))\).

\begin{theorem}[Subadditivity under composition (P6)]\label{J:thm:subadd}
Let \(f:M\to N\) and \(g:N\to P\) be composable morphisms in the stabilized constructible regime (in particular, after the mandated \(\mathbf{T}_\tau\) step).
Then
\[
\mu^{\mathrm{gen}}(g\circ f)\ \le\ \mu^{\mathrm{gen}}(f)\ +\ \mu^{\mathrm{gen}}(g),
\qquad
\nu^{\mathrm{gen}}(g\circ f)\ \le\ \nu^{\mathrm{gen}}(f)\ +\ \nu^{\mathrm{gen}}(g).
\]
\end{theorem}

\begin{proof}
Choose \(t\gg 0\) in a common stabilization tail for all modules involved, so that \(\dim_k(\ker(\cdot))(t)\) and \(\dim_k(\operatorname{coker}(\cdot))(t)\) equal the generic dimensions.
Evaluating at \(t\) yields finite-dimensional linear maps of \(k\)-spaces
\(
f_t:M_t\to N_t
\)
and
\(
g_t:N_t\to P_t
\).
For linear maps \(A:U\to V\), \(B:V\to W\), one has
\[
\dim\ker(B\circ A)\ \le\ \dim\ker(A)\ +\ \dim\ker(B),
\qquad
\dim\operatorname{coker}(B\circ A)\ \le\ \dim\operatorname{coker}(A)\ +\ \dim\operatorname{coker}(B),
\]
the first by the exact sequence
\(0\to \ker A \to \ker(BA)\to \ker B\cap \mathrm{Im}A\to 0\),
the second by duality or rank inequalities.
Applying these to \(A=f_t\), \(B=g_t\), and reading the stabilized values gives the claim.
\end{proof}

\begin{remark}[Metric-free; valid in \(V\)-enriched regimes]\label{J:rk:metric-free}
Theorem~\ref{J:thm:subadd} uses only tail-stabilized linear algebra; it is independent of any metric, and therefore remains valid verbatim in the \(V\)-enriched setting of Appendices~E--F.
\end{remark}

\begin{remark}[How P6 is used in the pipeline]\label{J:rk:P6-use}
In the operational stack, P6 is applied to compositions formed by post-processing maps that are required to be non-expansive and deletion-type (Appendix~G),
e.g.\ \(\mathbf{W}_{\le\tau_0}\), \(\mathbf{T}_\tau\), and permitted continuation maps.
Because \((\mu,\nu)\) are computed on the B-side after \(\mathbf{T}_\tau\), such post-processing never creates ``budget-escaping'' decreases in the diagnostic slack beyond what is logged in the \(\delta\)-ledger.
\end{remark}

% -------------------------
\subsection*{J.3. Additivity under finite direct sums}
% -------------------------

\begin{proposition}[Direct-sum additivity]\label{J:prop:sum}
Let \(F=F^{(1)}\oplus F^{(2)}\) be the levelwise direct sum of towers (same index category), and similarly for the colimit.
Then for every \(\tau\ge 0\),
\[
\mu_{i,\tau}(F)=\mu_{i,\tau}\!\big(F^{(1)}\big)+\mu_{i,\tau}\!\big(F^{(2)}\big),
\qquad
\nu_{i,\tau}(F)=\nu_{i,\tau}\!\big(F^{(1)}\big)+\nu_{i,\tau}\!\big(F^{(2)}\big).
\]
Hence \(\DiagZero/\DiagNonzero\) is additive under finite direct sums as well.
\end{proposition}

\begin{proof}
\(\mathbf{P}_i\) and \(\mathbf{T}_\tau\) preserve finite direct sums; filtered colimits commute with finite direct sums in \([\mathbb{R},\mathsf{Vect}_k]\).
Thus \(\phi_{i,\tau}(F)\) is block-diagonal with blocks \(\phi_{i,\tau}(F^{(1)})\), \(\phi_{i,\tau}(F^{(2)})\); kernels/cokernels and stabilized tail dimensions add.
\end{proof}

% -------------------------
\subsection*{J.4. Vanishing criteria (when the tower is ``well-behaved'')}
% -------------------------

\begin{proposition}[Sufficient conditions for $\DiagZero$]\label{J:prop:diagzero}
Assume one of the sufficient conditions of Appendix~D, \S D.3 holds for the tower \(F\) at the monitored \((i,\tau)\)
(e.g.\ commutation of \(\mathbf{T}_\tau\) with the relevant colimit in the scoped sense (S1), no \(\tau\)-accumulation plus constructible return (S2), or a \(\mathbf{T}_\tau\)-Cauchy tower with compatible cocone (S3)).
Then \(\phi_{i,\tau}(F)\) is an isomorphism on the stabilized tail, hence
\[
\mu_{i,\tau}(F)=\nu_{i,\tau}(F)=0.
\]
\end{proposition}

\begin{proof}
Under any of (S1)--(S3) (Appendix~D), the comparison map is an isomorphism after applying the mandated \(\mathbf{T}_\tau\), at least on the stabilized right tail where \(\dim_k^{\mathrm{gen}}\) is computed.
Thus \(\ker\phi_{i,\tau}(F)\) and \(\operatorname{coker}\phi_{i,\tau}(F)\) have zero generic dimension.
\end{proof}

\begin{remark}[Interpretation: Type IV detector]\label{J:rk:typeIV}
By design, \((\mu,\nu)\) detect ``invisible'' defects that may not appear in window-local \(E_1\)-style checks.
Operationally: \(\DiagZero\) is the normal state; \(\DiagNonzero\) is a Type~IV witness at the monitored \((i,\tau)\), and must be logged and surfaced by the Auditor layer (Appendix~G).
\end{remark}

% -------------------------
\subsection*{J.5. $\tau$-sweeps and stability bands}
% -------------------------

\begin{definition}[$\tau$-sweep and stability band]\label{J:def:stab-band}
Fix a window (MECE/right-open accounting) and a degree \(i\).
A \emph{\(\tau\)-sweep} is a finite increasing list \(\{\tau_\ell\}_{\ell=0}^L\subset(0,\infty)\) recorded in \texttt{run.yaml}.
A contiguous subarray \(\{\tau_a,\ldots,\tau_b\}\) is a \emph{stability band} for \(F\) in degree \(i\) if
\[
\mu_{i,\tau_\ell}(F)=\nu_{i,\tau_\ell}(F)=0\quad\text{for all }\ell\in\{a,\ldots,b\},
\]
and the verdict persists under refinement of the sweep (i.e.\ inserting additional \(\tau\)-values into \([\tau_a,\tau_b]\) does not create \(\DiagNonzero\) there).
\end{definition}

\begin{proposition}[Robust detection under refinement]\label{J:prop:robust-band}
Assume the definable/no-accumulation window policy of Appendix~G and that one of Appendix~D's sufficient hypotheses (S1)--(S3) holds on an interval of \(\tau\)-values around \(\tau_0\).
Then there exists an open neighborhood \(U\) of \(\tau_0\) such that \(\DiagZero\) holds for all \(\tau\in U\).
A sufficiently fine \(\tau\)-sweep detects \(U\) as a stability band and remains stable under refinement.
\end{proposition}

\begin{proof}
Under (S1)--(S3) on a neighborhood, \(\phi_{i,\tau}(F)\) is an isomorphism on the stabilized tail throughout that neighborhood, hence \(\mu=\nu=0\) there.
Refining the sweep cannot change the evaluated values, so the detected band is stable.
\end{proof}

% -------------------------
\subsection*{J.6. Piecewise constancy off a finite critical set (bounded $\tau$-ranges)}
% -------------------------

\begin{proposition}[Finite critical set and piecewise constancy]\label{J:prop:piecewise}
Fix a tower \(F\), degree \(i\), and a bounded interval \([a,b]\subset(0,\infty)\).
Assume the definable/no-\(\tau\)-accumulation policy of Appendix~G for the relevant bar-data after \(\mathbf{T}_\tau\).
Then there exists a finite set \(S\subset[a,b]\) such that \(\mu_{i,\tau}(F)\) and \(\nu_{i,\tau}(F)\) are locally constant on each connected component of \([a,b]\setminus S\).
\end{proposition}

\begin{proof}[Proof sketch]
Under definability (or an explicit no-accumulation contract) on bounded \(\tau\)-ranges, the set of \(\tau\)-values at which the post-\(\mathbf{T}_\tau\) barcode data can change is finite on \([a,b]\).
The stabilized tail linear-algebra type of \(\phi_{i,\tau}(F)\) (hence the generic dimensions of its kernel and cokernel) can change only at those critical \(\tau\)-values.
Therefore \(\mu_{i,\tau}\) and \(\nu_{i,\tau}\) are constant between successive critical values.
\end{proof}

\begin{corollary}[Band openness]\label{J:cor:band-open}
If \(\DiagZero\) holds at some \(\tau_0\in(a,b)\), then it holds on an open neighborhood of \(\tau_0\) inside \((a,b)\).
Hence stability bands are unions of open intervals intersected with the sweep list.
\end{corollary}

% -------------------------
\subsection*{J.7. Window pasting via Restart and Summability [Spec/Policy]}
% -------------------------

\begin{definition}[Per-window budget and safety gap]\label{J:def:budget-gap}
Let \(\{W_k\}\) be a MECE accounting of a domain, with right-open primitives and an explicit overlap list (Appendix~G).
On each window \(W_k\), let \(\tau_k>0\) be the selected collapse threshold (typically chosen inside a stability band when available).
Define the \emph{window budget} (degreewise) by aggregating the \(\delta\)-ledger entries recorded for that window:
\[
\Sigma\delta_k(i)\ :=\ \bigoplus_{\mathrm{op}\in\mathrm{Ops}(W_k)} \delta_{\mathrm{op}}(i,\tau_k),
\]
where \(\oplus\) is the quantale/budget aggregator declared in \texttt{run.yaml} (default: ordinary addition).
Let \(\mathrm{gap}_{\tau_k}(i)>0\) be the configured slack margin required by B-Gate\(^{+}\) on \(W_k\) in degree \(i\).
\end{definition}

\begin{lemma}[Restart inequality (policy form)]\label{J:lem:restart}
Assume the transition from \(W_k\) to \(W_{k+1}\) is realized by a finite composition of permitted \emph{deletion-type} steps and \(\varepsilon\)-continuations, all measured \emph{after} \(\mathbf{T}_{\tau_k}\) and recorded in the manifest.
If B-Gate\(^{+}\) passes on \(W_k\) with \(\mathrm{gap}_{\tau_k}(i)>\Sigma\delta_k(i)\), then there exists \(\kappa\in(0,1]\), determined by the admissible step class and the declared \(\tau\)-adaptation policy, such that
\[
\mathrm{gap}_{\tau_{k+1}}(i)\ \ge\ \kappa\ \Bigl(\mathrm{gap}_{\tau_k}(i)\ \ominus\ \Sigma\delta_k(i)\Bigr),
\]
where \(\ominus\) denotes truncated subtraction in the default additive budget regime (i.e.\ \(\max\{x-\Sigma\delta,0\}\)).
\end{lemma}

\begin{remark}[Status]
Lemma~\ref{J:lem:restart} is a policy contract: the implementation must either (i) supply a proof for the chosen step class and metric model, or (ii) treat \(\kappa\) and the inequality as an explicit assumption logged under \texttt{assumptions.restart\_inequality} (Appendix~G).
\end{remark}

\begin{definition}[Summability]\label{J:def:summability}
A run satisfies \emph{Summability} (on monitored degrees \(i\in I\)) if
\[
\sum_{k}\Sigma\delta_k(i)\ <\ \infty
\quad\text{for all }i\in I,
\]
with the series interpretation determined by \(\oplus\) (default: ordinary series in \([0,\infty]\)).
A sufficient pattern is geometric decay of window scales (e.g.\ \(\tau_k=\tau_0\rho^k\), \(\rho\in(0,1)\)) with uniformly bounded per-window operation counts and bounded per-operation \(\delta\) entries (Appendix~G).
\end{definition}

\begin{theorem}[Pasting windowed certificates (policy theorem)]\label{J:thm:pasting}
Let \(\{W_k\}\) be a MECE accounting of a domain with explicit overlap lists.
Suppose that on each \(W_k\), B-Gate\(^{+}\) passes with \(\mathrm{gap}_{\tau_k}(i)>\Sigma\delta_k(i)\) for all monitored degrees \(i\in I\),
that the Restart inequality (Lemma~\ref{J:lem:restart}) holds at every transition,
and that Summability (Definition~\ref{J:def:summability}) is satisfied.
Then the concatenation of per-window certificates yields a global certificate on \(\bigcup_k W_k\) for all monitored degrees \(i\in I\),
with all budgets and overlaps auditable from the manifest (Appendix~G).
\end{theorem}

% -------------------------
\subsection*{J.8. Countable covers and finite Čech depth (T-Countable-Cover)}
% -------------------------

\begin{theorem}[Countable cover; bounded overlap depth implies finite Čech depth]\label{J:thm:countable}
Let \(\mathcal{W}=\{W_j\}_{j\in\mathbb{N}}\) be a countable right-open cover of a bounded interval \(U\subset\mathbb{R}\).
Assume a \emph{bounded overlap multiplicity} certificate:
\[
m\ :=\ \sup_{t\in U}\#\{j\mid t\in W_j\}\ <\ \infty,
\]
either (i) provided explicitly as a checked bound, or (ii) derived from definability as in Theorem~\ref{J:thm:def-cech}.
Then:
\begin{enumerate}[leftmargin=1.25em]\itemsep0.25em
\item The Čech nerve of \(\mathcal{W}\) truncates in degree \(m-1\).
\item Overlap Glue therefore reduces to finitely many overlap checks on each compact subinterval of \(U\) (bounded by \(m\) and the logged overlap lists), consistent with the finite-check doctrine of Appendices~G--H.
\end{enumerate}
\end{theorem}

\begin{proof}
If \(m<\infty\), then any \(p\)-fold intersection for \(p>m\) is empty, so the Čech nerve has no simplices in degrees \(\ge m\), hence truncates in degree \(m-1\).
Finite-check reduction follows because only finitely many overlap patterns up to depth \(m\) can appear on compact subintervals once overlap lists are logged.
\end{proof}

\paragraph{Specification (T-Countable-Cover).}
\emph{Input:} a countable right-open cover \(\mathcal{W}\) of bounded \(U\); either an explicit bound \(m\) or definability metadata.
\emph{Checks:} (i) compute/verify \(m\), (ii) record Čech depth \(m-1\), (iii) store finite overlap lists used by Overlap Glue.
\emph{Accept iff} these fields appear in \texttt{run.yaml} and are referenced by the Overlap Glue routine.

% -------------------------
\subsection*{J.9. Summability contract (T-Delta-Sum-Converges)}
% -------------------------

\begin{proposition}[Summable $\delta$-ledger via comparison tests]\label{J:prop:sum-delta}
Let \(\{W_k\}_{k\in\mathbb{N}}\) be a (logged) countable window accounting on a bounded \(U\).
Suppose the per-window budgets satisfy either
\[
\Sigma\delta_k(i)\ \le\ C\,\rho^k\quad (C>0,\ \rho\in(0,1)),
\qquad\text{or}\qquad
\Sigma\delta_k(i)\ \le\ C\cdot \psi(k)\quad\text{with }\sum_k \psi(k)<\infty.
\]
Then \(\sum_k \Sigma\delta_k(i)<\infty\) for all monitored degrees \(i\), i.e.\ Summability holds.
\end{proposition}

\begin{proof}
Immediate by comparison with a convergent geometric series or the reference series \(\sum_k \psi(k)\).
\end{proof}

\paragraph{Specification (T-Delta-Sum-Converges).}
\emph{Input:} recorded sequence \(\{\Sigma\delta_k(i)\}\) and a proof obligation (geometric bound or reference-series bound).
\emph{Check:} compute partial sums; certify a finite bound \(B_i\) for each \(i\).
\emph{Accept iff} \(\sup_n \sum_{k\le n}\Sigma\delta_k(i)\le B_i<\infty\) is logged and referenced by B-Gate\(^{+}\) in \texttt{run.yaml}.

% -------------------------
\subsection*{J.10. Definable Čech finiteness (o-minimal)}
% -------------------------

\begin{theorem}[Definable Čech finiteness]\label{J:thm:def-cech}
Let \(\mathcal{W}=\{W_j\}_{j\in J}\) be a right-open cover of a bounded interval \(U\) such that endpoints and overlaps are \emph{definable} in an o-minimal structure (Appendix~G).
Then there exists a finite subcover, the overlap multiplicity \(m\) is finite, and the Čech nerve truncates in degree \(m-1\).
\end{theorem}

\begin{proof}
In an o-minimal structure, definable subsets of \(\mathbb{R}\) are finite unions of points and intervals.
Definability of endpoints/overlaps implies only finitely many combinatorial overlap types occur on bounded \(U\), hence a finite subcover exists and multiplicity is uniformly bounded.
Čech truncation follows from the finite bound \(m\).
\end{proof}

% -------------------------
\subsection*{J.11. Mandatory after-collapse order and coherence contract}
% -------------------------

All diagnostics \(\phi_{i,\tau}\), \(\mu_{i,\tau}\), and \(\nu_{i,\tau}\) are computed \emph{after} applying \(\mathbf{T}_\tau\) (B-side single layer),
and \emph{with the same window and the same \(\tau\)} as used by B-Gate\(^{+}\) and the \(\delta\)-ledger (Appendix~G).
This mandatory order
\[
\boxed{\ \text{for each }t\ \Longrightarrow\ \mathbf{P}_i\ \Longrightarrow\ \mathbf{T}_\tau\ \Longrightarrow\ \text{compare}\ }
\]
ensures that any subsequent \(1\)-Lipschitz post-processing (\(\mathbf{W}_{\le\tau_0}\), continuation maps, or allowed \(V\)-non-expansive functors) does not invalidate budgets,
does not create unlogged amplification, and keeps the Auditor/reproducibility manifest complete.

% -------------------------
\subsection*{J.12. Minimal API sketch (pseudocode)}
% -------------------------

\begin{verbatim}
# Compute (mu, nu) at scale tau for tower F and degree i (after T_tau)
def compute_mu_nu(F, i, tau):
    Ms = [ T_tau(P_i(F_n), tau) for F_n in F.levels ]
    M_inf = T_tau(P_i(F.apex), tau)
    colim_M = colim(Ms)                     # scoped colimit in [R, Vect_k]
    phi = comparison(colim_M, M_inf)
    mu = generic_tail_dim(kernel(phi))      # stabilized dimension t>>0
    nu = generic_tail_dim(cokernel(phi))
    return mu, nu

# Detect stability bands on a finite tau sweep; refinement-stable if piecewise-constant policy holds
def detect_stability_bands(F, i, tau_sweep):
    vals = [compute_mu_nu(F, i, tau) for tau in tau_sweep]
    zeros = [idx for idx,(mu,nu) in enumerate(vals) if (mu,nu)==(0,0)]
    return maximal_contiguous_subarrays(zeros)

# Summability check and cover depth contract
def summable_budget(deltas):                # T-Delta-Sum-Converges
    return (sum(deltas) < +infty)

def cech_depth_bound(cover):                # T-Countable-Cover
    m = max_overlap_multiplicity(cover)     # must be certified/derived
    return m-1
\end{verbatim}

\medskip
\noindent\textbf{Summary.}
Within the constructible regime and the filtered-colimit scope of Appendix~A, the tower comparison map \(\phi_{i,\tau}(F)\) is natural under tower morphisms (Lemma~\ref{J:lem:functoriality}) and invariant under cofinal restriction (Theorem~\ref{J:thm:cofinal}).
The diagnostics \((\mu_{i,\tau}(F),\nu_{i,\tau}(F))\) are defined as stabilized (generic tail) dimensions of \(\ker/\mathrm{coker}\) of \(\phi_{i,\tau}(F)\) after \(\mathbf{T}_\tau\); \(\DiagNonzero\) is the designated Type~IV witness.
A core calculus principle (P6) is the subadditivity of stabilized kernel/cokernel dimensions under composition (Theorem~\ref{J:thm:subadd}), independent of any metric.
Additivity holds under finite direct sums (Proposition~\ref{J:prop:sum}).
On bounded \(\tau\)-ranges, under the definable/no-accumulation policy, \((\mu_{i,\tau},\nu_{i,\tau})\) are piecewise constant off a finite critical set (Proposition~\ref{J:prop:piecewise}),
enabling robust stability-band detection (Definition~\ref{J:def:stab-band}, Proposition~\ref{J:prop:robust-band}).
Window pasting is governed by the \(\delta\)-ledger budgets, a Restart inequality, and Summability (Definitions~\ref{J:def:budget-gap}, \ref{J:def:summability}; Theorem~\ref{J:thm:pasting}), with countable-cover overlap glue reduced to finite Čech depth once bounded multiplicity is certified (Theorem~\ref{J:thm:countable}) or derived via definability (Theorem~\ref{J:thm:def-cech}).
All comparisons, budgets, and logs are made after collapse on the B-side single layer, preserving non-expansive bounds and reproducible audits (Appendix~G).



% =========================
\section*{Appendix K. Idempotent (Co)Monads for Collapse (up to f.q.i.) [Spec + Soft-Commuting + AWFS/2-Cell Budget] (reinforced)}
% =========================
\phantomsection
\addcontentsline{toc}{section}{Appendix K. Idempotent (Co)Monads for Collapse (up to f.q.i.)}
\refstepcounter{section}
\label{K:monads}

\paragraph{Standing conventions.}
We work over a field \(k\).
All persistence modules are taken in the one-parameter constructible persistence category \(\Pers^{\mathrm{cons}}_k\) (locally finite on bounded windows), and all equalities asserted at the persistence layer are equalities in \(\Pers^{\mathrm{cons}}_k\).
Filtered (co)limits, when invoked, are computed in the functor category \([\mathbb{R},\mathsf{Vect}_k]\) under the scope policy of Appendix~A, Remark~\ref{A:rk:filtered-colimits}; when stated, the result is returned to the constructible range only under the logged hypotheses of Appendix~A and the window/definability policies of Appendix~G.
The bar-deletion reflector \(\mathbf{T}_\tau\dashv \iota_\tau\) onto the \(\tau\)-local orthogonal subcategory \((\mathsf{E}_\tau)^\perp\subset \Pers^{\mathrm{cons}}_k\) is exact and \(1\)-Lipschitz (Appendix~A, Theorem~\ref{A:thm:localization}, Proposition~\ref{A:prop:lipschitz}).
Global conventions: \(\Ext^1\) is always against \(k[0]\); the energy exponent satisfies \(\alpha>0\) (default \(\alpha=1\)); windows are MECE and right-open with overlap lists logged (Appendix~G).
Distances are measured by the interleaving metric \(d_{\mathrm{int}}\) (which agrees with bottleneck distance on barcodes in the 1D constructible setting).
Identifications ``up to f.q.i.'' occur in \(\Ho(\mathsf{FiltCh}(k))\); they are preserved by \(\mathbf{P}_i\) whenever \(\mathbf{P}_i\) is applied.
All soft-commuting and 2-cell budget claims are evaluated \emph{after} the mandated order \(\mathbf{P}_i\to \mathbf{T}_\tau\) (B-side single layer), and must be measured on the same window/\(\tau\) as the gate and \(\delta\)-ledger (Appendix~G).

% ------------------------------------------------------------
\subsection*{K.1. Persistence layer: the idempotent monad $\ \mathbf{M}_\tau=\iota_\tau\mathbf{T}_\tau$}
% ------------------------------------------------------------

Let \(\iota_\tau:(\mathsf{E}_\tau)^\perp\hookrightarrow \Pers^{\mathrm{cons}}_k\) be the fully faithful inclusion and
\(\mathbf{T}_\tau:\Pers^{\mathrm{cons}}_k\to(\mathsf{E}_\tau)^\perp\) its left adjoint (Appendix~A).
Set
\[
\mathbf{M}_\tau\ :=\ \iota_\tau\circ \mathbf{T}_\tau:\ \Pers^{\mathrm{cons}}_k\longrightarrow \Pers^{\mathrm{cons}}_k.
\]

\begin{theorem}[Idempotent monad]\label{K:thm:monad}
With unit \(\eta:\mathrm{Id}\Rightarrow \mathbf{M}_\tau\) and multiplication
\[
\mu:\mathbf{M}_\tau^2=\iota_\tau\mathbf{T}_\tau\,\iota_\tau\mathbf{T}_\tau
\xRightarrow{\ \iota_\tau\,\varepsilon\,\mathbf{T}_\tau\ }
\iota_\tau\mathbf{T}_\tau=\mathbf{M}_\tau,
\]
induced by the counit \(\varepsilon:\mathbf{T}_\tau\iota_\tau\Rightarrow \mathrm{Id}\) on \((\mathsf{E}_\tau)^\perp\),
the triple \((\mathbf{M}_\tau,\eta,\mu)\) is a monad and \(\mu\) is a natural isomorphism (idempotence).
\end{theorem}

\begin{proof}
Any adjunction \(\mathbf{T}_\tau\dashv \iota_\tau\) yields a monad \(\iota_\tau\mathbf{T}_\tau\) with unit given by the adjunction unit and multiplication induced by the counit.
Because \(\iota_\tau\) is fully faithful, the counit \(\varepsilon:\mathbf{T}_\tau\iota_\tau\Rightarrow\mathrm{Id}\) is a natural isomorphism.
Hence \(\mu=\iota_\tau\,\varepsilon\,\mathbf{T}_\tau\) is a natural isomorphism, i.e.\ \(\mathbf{M}_\tau\) is idempotent.
\end{proof}

\begin{proposition}[Exact and non-expansive]\label{K:prop:exact-lip}
\(\mathbf{M}_\tau\) is exact and \(1\)-Lipschitz:
\[
d_{\mathrm{int}}\!\big(\mathbf{M}_\tau M,\mathbf{M}_\tau N\big)\ \le\ d_{\mathrm{int}}(M,N)
\qquad\text{for all }M,N\in\Pers^{\mathrm{cons}}_k.
\]
\end{proposition}

\begin{proof}
Exactness and \(1\)-Lipschitzness of \(\mathbf{T}_\tau\) are given in Appendix~A.
The inclusion \(\iota_\tau\) is fully faithful and exact on the essential image, and it does not increase interleaving distance.
Therefore the composite \(\mathbf{M}_\tau=\iota_\tau\mathbf{T}_\tau\) is exact and \(1\)-Lipschitz.
\end{proof}

\begin{remark}[Fixed points and algebras]\label{K:rk:algebras}
The Eilenberg--Moore category of \(\mathbf{M}_\tau\)-algebras identifies with \((\mathsf{E}_\tau)^\perp\) via \(\iota_\tau\).
Equivalently, \(M\) is a fixed point (\(\eta_M\) an isomorphism) iff \(M\in(\mathsf{E}_\tau)^\perp\).
Thus \(\mathbf{M}_\tau\) is the persistence-layer \(\tau\)-collapse, idempotent on its essential image.
\end{remark}

\begin{example}[Length threshold in 1D]\label{K:ex:length}
In the 1D constructible (barcode) case, \(\mathbf{T}_\tau\) deletes all interval summands of length \(<\tau\) (Appendix~A).
Hence \(\mathbf{M}_\tau\) is the canonical \(\tau\)-deleted representative, and \(\mathbf{M}_\tau^2\simeq \mathbf{M}_\tau\).
\end{example}

% ------------------------------------------------------------
\subsection*{K.2. Filtered layer: implementable idempotent comonad up to f.q.i.\ [Spec]}
% ------------------------------------------------------------

\paragraph{Spec context.}
This subsection is \emph{[Spec]}: it records the intended filtered-layer mechanism consistent with the proven persistence-layer reflector.
No claim here upgrades to a strict model-level statement unless explicitly discharged by an implementation proof and logged as such.

\medskip
\noindent\emph{[Spec]} There exists a coreflective ``implementable'' subcategory
\[
\Ho(\mathsf{FiltCh}(k))_\tau^{\mathrm{comb}}\ \subset\ \Ho(\mathsf{FiltCh}(k))
\]
and a fully faithful inclusion
\(
\iota:\Ho(\mathsf{FiltCh}(k))_\tau^{\mathrm{comb}}\hookrightarrow \Ho(\mathsf{FiltCh}(k))
\)
with right adjoint (coreflector) \(C_\tau^{\mathrm{comb}}\) (natural up to f.q.i.).
Define the endofunctor
\[
\mathbf{G}_\tau\ :=\ \iota\circ C_\tau^{\mathrm{comb}}:\ \Ho(\mathsf{FiltCh}(k))\longrightarrow \Ho(\mathsf{FiltCh}(k)).
\]

\begin{theorem}[\textnormal{\textbf{[Spec]}} Idempotent comonad up to f.q.i.]\label{K:thm:comonad}
With counit \(\varepsilon:\mathbf{G}_\tau\Rightarrow \mathrm{Id}\) (the coreflection counit) and comultiplication
\[
\delta:\mathbf{G}_\tau
\xRightarrow{\ \iota\,\eta\,C_\tau^{\mathrm{comb}}\ }
\mathbf{G}_\tau^2,
\]
\((\mathbf{G}_\tau,\varepsilon,\delta)\) is a comonad in \(\Ho(\mathsf{FiltCh}(k))\); moreover \(\delta\) is a natural isomorphism (idempotence).
All statements are invariant under filtered quasi-isomorphism.
\end{theorem}

\begin{proof}[Proof sketch]
Any coreflection \(\iota\dashv C_\tau^{\mathrm{comb}}\) yields a comonad \(\iota C_\tau^{\mathrm{comb}}\), with comultiplication induced by the adjunction unit.
Because \(\iota\) is fully faithful, the unit is an isomorphism on the essential image, which implies idempotence.
All constructions occur in \(\Ho(\mathsf{FiltCh}(k))\), hence are f.q.i.-invariant.
\end{proof}

\begin{proposition}[\textnormal{\textbf{[Spec]}} Compatibility with persistence; nonexpansivity after $\mathbf{P}_i$]\label{K:prop:compat}
Naturally in \(i\) and \(F\),
\[
\mathbf{P}_i(\mathbf{G}_\tau F)\ \cong\ \mathbf{M}_\tau(\mathbf{P}_iF).
\]
Moreover, for all \(F,G\) and all \(i\),
\[
d_{\mathrm{int}}\!\big(\mathbf{P}_i(\mathbf{G}_\tau F),\mathbf{P}_i(\mathbf{G}_\tau G)\big)\ \le\ d_{\mathrm{int}}\!\big(\mathbf{P}_iF,\mathbf{P}_iG\big).
\]
\end{proposition}

\begin{proof}[Proof sketch]
The first claim is the defining compatibility requirement for the implementable coreflector: it must realize the same post-\(\tau\) deletion as \(\mathbf{T}_\tau\) after applying \(\mathbf{P}_i\).
The second follows from \(1\)-Lipschitzness of \(\mathbf{M}_\tau\) (Proposition~\ref{K:prop:exact-lip}) and functoriality of \(\mathbf{P}_i\).
\end{proof}

\begin{remark}[Scope]\label{K:rk:scope}
The comonad is asserted only on the implementable, f.q.i.\ range.
This suffices for algorithms and stability accounting because all measured constraints are reduced to persistence-layer checks (after \(\mathbf{P}_i\to\mathbf{T}_\tau\)) and recorded in the manifest (Appendix~G).
\end{remark}

% ------------------------------------------------------------
\subsection*{K.3. Multiple reflectors: nesting, strict commutation, and soft-commuting policy}
% ------------------------------------------------------------

\paragraph{General torsion reflectors.}
Let \(E_A,E_B\) be hereditary Serre subcategories of \(\Pers^{\mathrm{cons}}_k\) with exact reflectors \(T_A,T_B\).
Let \(E_{A\vee B}\) denote the Serre join.

\begin{proposition}[\textnormal{\textbf{[Spec]}} Nested torsions $\Rightarrow$ strict order independence]\label{K:prop:nested}
If \(E_A\subseteq E_B\) or \(E_B\subseteq E_A\), then
\[
T_A\circ T_B\ =\ T_B\circ T_A\ =\ T_{A\vee B}.
\]
In particular, for deletion thresholds,
\[
\mathbf{T}_\tau\circ \mathbf{T}_\sigma\ =\ \mathbf{T}_{\max\{\tau,\sigma\}}
\qquad\text{and hence}\qquad
\mathbf{M}_\tau\circ \mathbf{M}_\sigma\ =\ \mathbf{M}_{\max\{\tau,\sigma\}}.
\]
\end{proposition}

\begin{proof}[Proof sketch]
For nested torsion theories, applying the larger reflector already enforces orthogonality to the smaller torsion class, yielding order independence.
The threshold identity follows because ``delete bars of length \(<\tau\)'' composed with ``delete bars of length \(<\sigma\)'' equals deleting bars of length \(<\max\{\tau,\sigma\}\).
\end{proof}

\begin{definition}[Commutation defect]\label{K:def:soft}
For \(M\in\Pers^{\mathrm{cons}}_k\), define the \emph{commutation defect}
\[
\Delta_{\mathrm{comm}}(M;A,B)\ :=\ d_{\mathrm{int}}\!\big(T_A T_B M,\ T_B T_A M\big).
\]
Given a window \(W\) and tolerance \(\eta\ge 0\) declared in \texttt{run.yaml}, we say \(T_A,T_B\) \emph{soft-commute} on \(M\) (on \(W\)) if
\(\Delta_{\mathrm{comm}}(M;A,B)\le \eta\).
Otherwise, a canonical order is fixed and the residual \(\Delta_{\mathrm{comm}}\) is logged into the \(\delta\)-ledger as an algorithmic defect.
\end{definition}

\begin{declaration}[Operational A/B policy (window-coherent)]\label{K:dec:soft-policy}
Per window \(W\) and degree \(i\):
\begin{enumerate}[leftmargin=1.25em]\itemsep0.2em
\item Compute \(\Delta_{\mathrm{comm}}\) \emph{after} the mandated B-side collapse \(\mathbf{P}_i\to \mathbf{T}_\tau\) and on the same \(W,\tau\) as the gate and ledger.
\item If \(\Delta_{\mathrm{comm}}\le\eta\), treat \(T_A,T_B\) as soft-commuting on that window and allow either order.
\item If \(\Delta_{\mathrm{comm}}>\eta\), enforce a canonical order, and log \(\Delta_{\mathrm{comm}}\) as \(\delta^{\mathrm{alg}}_{\mathrm{comm}}\) for \(W,i,\tau\).
\item Aggregate \(\delta^{\mathrm{alg}}_{\mathrm{comm}}\) into the per-window budget \(\Sigma\delta\) used by Restart/Summability (Appendix~J).
\end{enumerate}
\end{declaration}

\begin{remark}[Three or more nonnested axes]\label{K:rk:nonconfluent}
For \(A,B,C\) pairwise soft-commuting does not imply global confluence.
Operationally, fix a canonical total order on axes and measure/log adjacent commutation defects along that order; do not infer higher coherence without explicit 2-cell data.
\end{remark}

% ------------------------------------------------------------
\subsection*{K.4. Mirror/Transfer 2-cells and additive budget (persistence-level accounting)}
% ------------------------------------------------------------

\paragraph{2-cells as measured defects.}
Let \(\Mirror\) be an admissible endofunctor on the filtered or persistence layer, equipped with a natural \(2\)-cell
\[
\theta:\ \Mirror\circ C_\tau\ \Rightarrow\ C_\tau\circ \Mirror
\]
(on the filtered layer, understood up to f.q.i.), whose image under \(\mathbf{P}_i\) is measured in \(d_{\mathrm{int}}\) by a bound \(\delta^{\mathrm{mirror}}(i,\tau)\ge 0\) on the current window \(W\).

\begin{proposition}[\textnormal{\textbf{[Spec]}} Additive budget with (co)reflectors and Mirror]\label{K:prop:budget}
On a fixed window \(W\) and degree \(i\), the total slack required to accommodate:
\begin{enumerate}[leftmargin=1.25em]\itemsep0.2em
\item each Mirror--Collapse defect \(\delta^{\mathrm{mirror}}(i,\tau)\), and
\item each recorded A/B commutation defect \(\delta^{\mathrm{alg}}_{\mathrm{comm}}=\Delta_{\mathrm{comm}}(M;A,B)\),
\end{enumerate}
is bounded above by their quantale-aggregation (default: ordinary sum) in the \(\delta\)-ledger.
Any subsequent \(1\)-Lipschitz persistence-level post-processing does not increase this bound.
\end{proposition}

\begin{proof}[Proof sketch]
Each defect is a \(d_{\mathrm{int}}\)-bound for a natural transformation or commutator comparison.
Triangle inequalities and \(1\)-Lipschitzness of allowed post-processing yield subadditivity and non-expansive propagation of bounds.
Ledger aggregation is required to apply a single law exactly once (Appendix~G).
\end{proof}

\begin{corollary}[Windowwise additivity]\label{K:cor:window}
Across a MECE family of windows \(\{W_j\}\) with per-window contributions
\(\delta_j^{\mathrm{mirror}}\) and \(\delta_j^{\mathrm{comm}}\),
the end-to-end slack is bounded by the ledger aggregation of
\(
\{\delta_j^{\mathrm{mirror}},\delta_j^{\mathrm{comm}}\}_j
\),
and this bound is preserved under \(1\)-Lipschitz aggregators and the Restart/Summability mechanism (Appendix~J).
\end{corollary}

% ------------------------------------------------------------
\subsection*{K.5. Strict 2-cell accounting via product quantales (no double counting)}
% ------------------------------------------------------------

We strictify multi-channel 2-cell budget aggregation by declaring a base quantale and its finite products,
and enforcing a single aggregation law applied exactly once (Appendix~G, Declaration~\ref{dec:runyaml-extended}).

\begin{definition}[Base quantale]\label{K:def:quantale-base}
A \emph{quantale (budget object)} is a commutative, monotone monoid \((V,\oplus,\preceq,0)\),
where \(\preceq\) is a partial order compatible with \(\oplus\).
Default instance: \(V=[0,\infty]\), \(\oplus=+\), \(\preceq=\le\), \(0=0\).
Each pipeline step emits a defect value in \(V\) (e.g.\ \(\delta^{\mathrm{alg}},\delta^{\mathrm{disc}},\delta^{\mathrm{meas}}\)),
and per-window budgets use the \emph{single} operation \(\oplus\).
\end{definition}

\begin{definition}[Product quantale and collapse homomorphism]\label{K:def:quantale-prod}
For \(m\ge 1\) defect channels, define the \emph{product quantale}
\[
V^{\times m}
\quad\text{with}\quad
\mathbf{x}\preceq \mathbf{y}\iff(\forall r)\ x_r\preceq y_r,
\qquad
(\mathbf{x}\ \widehat{\oplus}\ \mathbf{y})_r:=x_r\oplus y_r,
\qquad
\mathbf{0}:=(0,\dots,0).
\]
Define the \emph{collapse homomorphism} \(\pi:V^{\times m}\to V\) by
\[
\pi(x_1,\dots,x_m)\ :=\ x_1\oplus\cdots\oplus x_m.
\]
Then \(\pi\) is monotone and \(\pi(\mathbf{x}\ \widehat{\oplus}\ \mathbf{y})=\pi(\mathbf{x})\oplus\pi(\mathbf{y})\).
\end{definition}

\begin{definition}[AWFS/2-cell bounds (contract)]\label{K:def:awfs}
\emph{[Spec/Contract]} Assume an algebraic weak factorization system (AWFS) on the filtered layer producing:
\begin{enumerate}[leftmargin=1.25em]\itemsep0.2em
\item a comonad \(L\) (cofibration-like) with counit \(L\Rightarrow\mathrm{Id}\),
\item a monad \(R\) (fibration-like) with unit \(\mathrm{Id}\Rightarrow R\),
\item 2-cell contracts for admissible functors, including
\(\mathbf{G}_\tau\Rightarrow \mathbf{G}_\tau^2\),
\(\Mirror\circ \mathbf{G}_\tau \Rightarrow \mathbf{G}_\tau\circ \Mirror\),
and adjacent reflector commutators.
\end{enumerate}
Each contract emits a per-degree/per-threshold defect vector
\[
\boldsymbol{\delta}(i,\tau)\ :=\ \big(\delta^{\mathrm{mirror}}(i,\tau),\ \delta^{\mathrm{transfer}}(i,\tau),\ \delta^{\mathrm{comm}}(i,\tau)\big)\ \in\ V^{\times m},
\]
measured after \(\mathbf{P}_i\to\mathbf{T}_\tau\) on the current window \(W\).
\end{definition}

\begin{proposition}[Strict product accounting]\label{K:prop:product}
Along any pipeline segment with component defect vectors \(\boldsymbol{\delta}_1,\ldots,\boldsymbol{\delta}_n\in V^{\times m}\),
the accumulated segment defect satisfies
\[
\boldsymbol{\delta}_{\mathrm{seg}}\ \preceq\
\boldsymbol{\delta}_1\ \widehat{\oplus}\ \cdots\ \widehat{\oplus}\ \boldsymbol{\delta}_n,
\qquad
\delta_{\mathrm{seg}}\ :=\ \pi(\boldsymbol{\delta}_{\mathrm{seg}})\ \preceq\
\pi(\boldsymbol{\delta}_1)\ \oplus\ \cdots\ \oplus\ \pi(\boldsymbol{\delta}_n).
\]
Thus:
(i) each channel is aggregated exactly once (coordinatewise),
(ii) the collapsed scalar budget \(\delta_{\mathrm{seg}}\) respects the same single law \(\oplus\),
and (iii) double counting is structurally prevented by the homomorphism \(\pi\).
\end{proposition}

\begin{proof}
Each defect channel is bounded in \(d_{\mathrm{int}}\) and obeys a triangle inequality, hence the coordinatewise sum \(\widehat{\oplus}\) bounds the accumulated vector.
Applying \(\pi\) and using \(\pi(\mathbf{x}\widehat{\oplus}\mathbf{y})=\pi(\mathbf{x})\oplus\pi(\mathbf{y})\) yields the scalar bound.
\end{proof}

\begin{remark}[run.yaml alignment; determinism]\label{K:rk:yaml}
Record in \texttt{run.yaml} (Appendix~G, Declaration~\ref{dec:runyaml-extended}):
\begin{enumerate}[leftmargin=1.25em]\itemsep0.15em
\item \texttt{quantale.\{name,op,unit,order\}} and \texttt{layered\_delta.compose: "quantale-sum"},
\item channel list and the product dimension \(m\),
\item per-step and per-window vectors \(\boldsymbol{\delta}\in V^{\times m}\) and their scalar collapse \(\pi(\boldsymbol{\delta})\),
\item reflector order decisions and soft-commuting verdicts, including \(\eta\) and \(\Delta_{\mathrm{comm}}\).
\end{enumerate}
This enforces deterministic aggregation and prevents inadvertent re-aggregation.
\end{remark}

% ------------------------------------------------------------
\subsection*{K.6. Windowed usage and minimal schema (operational contract)}
% ------------------------------------------------------------

Per window \(W\) (degree \(i\), threshold \(\tau\)), log at minimum:
\begin{enumerate}[leftmargin=1.25em]\itemsep0.2em
\item reflector axis list and canonical order; A/B soft-commuting checks with \(\eta\) and measured \(\Delta_{\mathrm{comm}}\),
\item admissible 2-cells (Mirror/Transfer) and their measured bounds \(\delta^{\mathrm{mirror}}(i,\tau)\), \(\delta^{\mathrm{transfer}}(i,\tau)\),
\item the defect vector \(\boldsymbol{\delta}(i,\tau)\in V^{\times m}\) and its scalar collapse \(\pi(\boldsymbol{\delta}(i,\tau))\),
\item quantale parameters and the aggregation law used exactly once,
\item B-Gate\(^{+}\) gap and the resulting per-window budget \(\Sigma\delta(i)\) for Restart/Summability (Appendix~J).
\end{enumerate}
Canonical YAML keys appear in Appendix~G.

% ------------------------------------------------------------
\subsection*{K.7. Formalization stubs (Lean/Coq) [Spec]}
% ------------------------------------------------------------
\begin{verbatim}
-- Schematic: reflectors, commutation defect, and quantale products

namespace AK.K
open scoped Classical
noncomputable section

-- Persistence reflectors (schematic)
constant Pers : Type
constant Reflector : Type
constant T : Reflector → (Pers → Pers)

-- Interleaving distance (schematic)
constant d_int : Pers → Pers → ℝ≥0

-- Commutation defect
def comm_defect (M : Pers) (A B : Reflector) : ℝ≥0 :=
  d_int (T A (T B M)) (T B (T A M))

-- Quantale
structure Quantale :=
  (V : Type) (op : V → V → V) (le : V → V → Prop) (unit : V)
  (op_comm  : ∀ x y, op x y = op y x)
  (op_assoc : ∀ x y z, op (op x y) z = op x (op y z))
  (op_mono  : ∀ {x x' y y'}, le x x' → le y y' → le (op x y) (op x' y'))
  (le_refl  : ∀ x, le x x) (le_trans : ∀ x y z, le x y → le y z → le x z)

-- Product quantale and collapse homomorphism (fold)
def prodQ (Q : Quantale) (m : Nat) : Quantale := by
  -- coordinatewise order and op
  admit

def collapse (Q : Quantale) (m : Nat) :
  (prodQ Q m).V → Q.V := by
  -- fold by Q.op along coordinates; homomorphism proof required
  admit
end AK.K
\end{verbatim}

% ------------------------------------------------------------
\subsection*{K.8. Guard-rails and failure modes}
% ------------------------------------------------------------

\begin{itemize}\itemsep0.2em
  \item \emph{Nonexact steps.} Heuristic, nonexact truncations may violate monad/comonad laws and stability; exclude them from the reflector class or log them as \texttt{nonexact} with separate handling.
  \item \emph{Window mismatch.} A/B commutation tests and Mirror--Collapse measurements must use the same window/\(\tau\) as the gate and \(\delta\)-ledger; otherwise the audit is invalid.
  \item \emph{Multi-axis nonconfluence.} For \(\ge 3\) nonnested axes, pairwise soft-commuting does not imply global coherence; fix a canonical order and log adjacent defects.
  \item \emph{Adaptive thresholds.} If \(\tau\) adapts online, every ledger entry must carry the in-force \(\tau\); do not retroactively reassign defects to different \(\tau\).
  \item \emph{f.q.i.\ boundaries.} Any statement involving \(\mathbf{G}_\tau\) or AWFS is [Spec] unless discharged; persistence-layer statements remain authoritative.
\end{itemize}

\medskip
\noindent\textbf{Summary.}
On \(\Pers^{\mathrm{cons}}_k\), collapse is governed by the exact, idempotent, \(1\)-Lipschitz monad \(\mathbf{M}_\tau=\iota_\tau\mathbf{T}_\tau\) (Theorem~\ref{K:thm:monad}, Proposition~\ref{K:prop:exact-lip}).
On \(\Ho(\mathsf{FiltCh}(k))\), an implementable idempotent comonad \(\mathbf{G}_\tau=\iota C_\tau^{\mathrm{comb}}\) is recorded as [Spec], required to be compatible with persistence after \(\mathbf{P}_i\) (Theorem~\ref{K:thm:comonad}, Proposition~\ref{K:prop:compat}).
For multiple reflector axes, nesting yields strict commutation; otherwise an A/B soft-commuting policy controls order via the commutation defect \(\Delta_{\mathrm{comm}}\) and logs residuals into the \(\delta\)-ledger (Definition~\ref{K:def:soft}, Declaration~\ref{K:dec:soft-policy}).
Mirror/Transfer 2-cells contribute additional measured defects.
All multi-channel defects are aggregated in a product quantale and collapsed by a single homomorphism \(\pi\), enforcing strict, non-duplicative accounting aligned with \texttt{run.yaml} (Proposition~\ref{K:prop:product}, Remark~\ref{K:rk:yaml}).
This integrates with Restart/Summability pasting and overlap auditing (Appendices~G and~J) under the mandatory after-collapse order.



% =========================
\section*{Appendix L. Quantitative Commutation for Mirror/Tropical [Spec + Pipeline Budget + A/B Policy] (reinforced)}
% =========================
\phantomsection
\addcontentsline{toc}{section}{Appendix L. Quantitative Commutation for Mirror/Tropical}
\refstepcounter{section}
\label{L:quant-comm}

\paragraph{Standing conventions.}
We work over a field \(k\).
All persistence modules are taken in the one-parameter constructible persistence category \(\Pers^{\mathrm{cons}}_k\) (locally finite on bounded windows).
Filtered (co)limits, when invoked, are computed in \([\mathbb{R},\mathsf{Vect}_k]\) under the scope policy of Appendix~A, Remark~\ref{A:rk:filtered-colimits}; returning to the constructible range is permitted only under the logged hypotheses of Appendix~A and the window/definability policies of Appendix~G.
Distances are measured by the interleaving metric \(d_{\mathrm{int}}\) (which agrees with bottleneck distance on barcodes in the 1D constructible case).
The bar-deletion reflector \(\mathbf{T}_\tau:\Pers^{\mathrm{cons}}_k\to\Pers^{\mathrm{cons}}_k\) is exact and \(1\)-Lipschitz (Appendix~A).
On the filtered layer we write \(C_\tau\) for the (filtered) lift of \(\mathbf{T}_\tau\), defined only up to filtered quasi-isomorphism (f.q.i.) in \(\Ho(\mathsf{FiltCh}(k))\), with endpoints/infinite-bar policy centralized as declared in Chapter~2 and Appendix~G.
Global conventions: \(\Ext^1\) is always against \(k[0]\); the energy exponent satisfies \(\alpha>0\) (default \(\alpha=1\)); windows are MECE and right-open with overlap lists logged (Appendix~G).
All measurements and comparisons are performed \emph{after} the mandated order
\[
\boxed{\ \text{for each }t\ \Rightarrow\ \mathbf{P}_i\ \Rightarrow\ \mathbf{T}_\tau\ \Rightarrow\ \text{compare}\ }
\]
on the B-side single layer (Appendix~G). Any statement not reduced to this persistence-layer order is \emph{[Spec]}.

% -------------------------
\subsection*{L.1. Hypotheses (Mirror/Tropical) [Spec]}
% -------------------------

\paragraph{Spec context.}
This appendix is \emph{[Spec]}: it records quantitative commutation assumptions and the resulting budgeting rules used by the pipeline.
The authoritative, proven layer is the persistence-layer \(\mathbf{T}_\tau\) and its \(1\)-Lipschitz behavior (Appendix~A); all filtered-layer commutation is accepted only through persistence-layer measurements after \(\mathbf{P}_i\to\mathbf{T}_\tau\).

\medskip
\noindent\emph{[Spec]} Let \(U\) be a filtered-level endofunctor (``Mirror/Tropical'') on \(\Ho(\mathsf{FiltCh}(k))\).
Via \(\mathbf{P}_i\), it induces endofunctors on \(\Pers^{\mathrm{cons}}_k\).
Assume:

\begin{itemize}\itemsep0.35em
  \item[(H1)] \textbf{\(1\)-Lipschitz after persistence.}
  For each degree \(i\),
  \[
    d_{\mathrm{int}}\!\Big(\mathbf{P}_i(UF),\ \mathbf{P}_i(UG)\Big)\ \le\ d_{\mathrm{int}}\!\Big(\mathbf{P}_i(F),\ \mathbf{P}_i(G)\Big)
    \qquad(\forall\,F,G).
  \]
  \item[(H2)] \textbf{\(\delta\)-controlled natural \(2\)-cell (up to f.q.i.).}
  There exists a natural \(2\)-cell
  \[
    \theta:\ U\!\circ\! C_\tau\ \Rightarrow\ C_\tau\!\circ\! U
  \]
  in \(\Ho(\mathsf{FiltCh}(k))\) (interpreted up to f.q.i.) such that, uniformly in \(F\),
  \[
    d_{\mathrm{int}}\!\Big(\mathbf{P}_i\big(U(C_\tau F)\big),\ \mathbf{P}_i\big(C_\tau(UF)\big)\Big)\ \le\ \delta(i,\tau),
  \]
  for all \(i\in\mathbb{Z}\) and \(\tau\ge 0\), and where the measurement is taken on the B-side after-collapse layer (Appendix~G).
\end{itemize}

\begin{remark}[Strict commutation]\label{L:rk:strict}
If \(\delta(i,\tau)=0\), then \(U\) and \(C_\tau\) commute up to isomorphism \emph{after} \(\mathbf{P}_i\) at the truncated persistence layer, for the declared window/\(\tau\).
\end{remark}

\begin{remark}[Tropical instances (Spec)]\label{L:rk:trop}
A ``Tropical'' operator may be any window-coherent, order-preserving endofunctor assembled from idempotent semiring primitives (e.g.\ min/max filters, morphological erosions/dilations, monotone reparametrizations) that satisfies \emph{(H1)} and admits a \(2\)-cell \(\theta\) with a bound \(\delta_{\mathrm{trop}}(i,\tau)\) independent of \(F\).
Its configuration (e.g.\ bin width/range and bin-shift policy) is logged under \texttt{tropical} in \texttt{run.yaml} (Appendix~G, Declaration~\ref{dec:runyaml-extended}).
\end{remark}

% -------------------------
\subsection*{L.2. Post-collapse bound from the 2-cell (direct persistence form)}
% -------------------------

\begin{theorem}[Post-collapse non-expansive commutation bound]\label{L:thm:q-comm}
Assume \emph{(H2)}. Then for all filtered inputs \(F\), degrees \(i\), and thresholds \(\tau\),
\[
d_{\mathrm{int}}\!\Big(\mathbf{T}_\tau\,\mathbf{P}_i\big(U(C_\tau F)\big),\ \mathbf{T}_\tau\,\mathbf{P}_i\big(C_\tau(UF)\big)\Big)\ \le\ \delta(i,\tau).
\]
Moreover, any further B-side post-processing \(\Phi\) that is \(1\)-Lipschitz for \(d_{\mathrm{int}}\) satisfies
\[
d_{\mathrm{int}}\!\Big(\Phi\,\mathbf{T}_\tau\mathbf{P}_i(U(C_\tau F)),\ \Phi\,\mathbf{T}_\tau\mathbf{P}_i(C_\tau(UF))\Big)
\ \le\ \delta(i,\tau).
\]
\end{theorem}

\begin{proof}
Apply \(\mathbf{T}_\tau\), which is \(1\)-Lipschitz, to the \emph{(H2)} estimate at the persistence layer, and then use \(1\)-Lipschitzness of \(\Phi\).
\end{proof}

\begin{corollary}[Two-stage additivity]\label{L:cor:two-stage}
If \(U_1,U_2\) satisfy \emph{(H2)} with bounds \(\delta_1,\delta_2\), then for all \(F,i,\tau\),
\[
d_{\mathrm{int}}\!\Big(\mathbf{T}_\tau\mathbf{P}_i\big(U_2U_1(C_\tau F)\big),\ \mathbf{T}_\tau\mathbf{P}_i\big(C_\tau(U_2U_1F)\big)\Big)\ \le\ \delta_1(i,\tau)\ +\ \delta_2(i,\tau).
\]
\end{corollary}

\begin{remark}[Necessity of a controlled 2-cell]\label{L:rk:need2cell}
Without a controlled \(2\)-cell, \(1\)-Lipschitzness of \(U\) and \(\mathbf{T}_\tau\) does not constrain the discrepancy between \(U\circ C_\tau\) and \(C_\tau\circ U\).
In particular, near-\(\tau\) accumulations can create discontinuous behavior invisible to naive nonexpansivity checks, and must be treated as Type~IV risk unless explicitly bounded (Appendix~D).
\end{remark}

% -------------------------
\subsection*{L.3. Product-ledger budgeting (direct accumulation; no relay chains)}
% -------------------------

We mainstream direct accumulation of commutation errors in a product-quantale ledger (Appendix~K, \S K.5), eliminating intermediate relay comparisons and preventing double counting.

\begin{definition}[Base and product quantales]\label{L:def:q-product}
Fix a commutative quantale (budget object) \((V,\oplus,\preceq,0)\) (default: \(V=[0,\infty]\), \(\oplus=+\), \(\preceq=\le\)).
For \(m\ge 1\) channels (e.g.\ Mirror, Transfer, A/B), set the product quantale
\[
V^{\times m},\quad
\mathbf{x}\preceq\mathbf{y}\iff(\forall r)\ x_r\preceq y_r,\quad
(\mathbf{x}\ \widehat{\oplus}\ \mathbf{y})_r:=x_r\oplus y_r,\quad
\mathbf{0}:=(0,\dots,0).
\]
Let \(\pi:V^{\times m}\!\to V\) be the collapse homomorphism \(\pi(\mathbf{x})=x_1\oplus\cdots\oplus x_m\) (Appendix~K).
\end{definition}

\begin{definition}[Per-window defect vector]\label{L:def:vec}
Fix a window \(W=[u,u')\) (MECE, right-open), a degree \(i\), and the threshold(s) \(\tau\) in force on \(W\).
Each Mirror/Tropical step \(U_j\) on \(W\) contributes a vector
\(
\boldsymbol{\delta}^{(j)}_W(i)\in V^{\times m}
\)
whose nonzero coordinates are the relevant bounds furnished by Theorem~\ref{L:thm:q-comm} (e.g.\ mirror/transfer coordinates).
Each A/B commutation failure contributes a vector whose only nonzero coordinate is the A/B channel value equal to the measured \(\Delta_{\mathrm{comm}}\) (Definition~\ref{L:def:ab}).
The window vector and its scalar collapse are
\[
\boldsymbol{\delta}_W(i)\ :=\ \widehat{\oplus}_{\,j}\ \boldsymbol{\delta}^{(j)}_W(i),
\qquad
\Sigma\delta_W(i)\ :=\ \pi\big(\boldsymbol{\delta}_W(i)\big).
\]
All entries must be logged with the window id, degree, and the in-force \(\tau\) (Appendix~G).
\end{definition}

\begin{proposition}[Strict, non-duplicative accounting]\label{L:prop:strict}
Coordinatewise aggregation in \(V^{\times m}\) followed by \(\pi\) yields
\[
\Sigma\delta_W(i)\ \preceq\ \bigoplus_{j}\ \pi\!\big(\boldsymbol{\delta}^{(j)}_W(i)\big),
\]
counts each channel exactly once, and is compatible with Appendix~K, Proposition~\ref{K:prop:product}.
\end{proposition}

\begin{proof}
Immediate from the definition of \(\widehat{\oplus}\) (coordinatewise application of \(\oplus\)) and the homomorphism property of \(\pi\).
\end{proof}

% -------------------------
\subsection*{L.4. Windowed pipeline bound (direct form)}
% -------------------------

\begin{theorem}[\textnormal{\textbf{[Spec]}} Direct pipeline bound]\label{L:thm:budget}
Fix a window \(W\), degree \(i\), and the realized left/right orderings \(\Pi_{\mathrm{lhs}}\), \(\Pi_{\mathrm{rhs}}\) on \(W\).
Assume that every commutation step in the segment is covered either by:
(i) a recorded \(2\)-cell bound (Mirror/Tropical/Transfer) contributing a coordinate to \(\boldsymbol{\delta}_W(i)\), or
(ii) an A/B commutation test contributing a coordinate to \(\boldsymbol{\delta}_W(i)\).
Then for any filtered input \(F\),
\[
d_{\mathrm{int}}\!\Big(\mathbf{T}_\tau\,\mathbf{P}_i(\Pi_{\mathrm{lhs}}(F)),\ \mathbf{T}_\tau\,\mathbf{P}_i(\Pi_{\mathrm{rhs}}(F))\Big)\ \preceq\ \Sigma\delta_W(i),
\]
where \(\tau\) is the threshold in force on \(W\).
The bound is uniform in \(F\), additive across steps via \(\widehat{\oplus}\), and non-increasing under any subsequent \(1\)-Lipschitz persistence post-processing.
\end{theorem}

\begin{proof}[Proof sketch]
Each step discrepancy is bounded in the appropriate channel either by Theorem~\ref{L:thm:q-comm} (2-cell) or by the A/B commutation measurement.
Triangle inequalities aggregate these bounds coordinatewise, giving \(\boldsymbol{\delta}_W(i)\).
Applying \(\pi\) yields the scalar bound \(\Sigma\delta_W(i)\).
Finally, \(1\)-Lipschitz post-processing cannot increase the discrepancy.
\end{proof}

\begin{corollary}[Across windows; Restart/Summability alignment]\label{L:cor:across}
Over a MECE partition \(\{W_k\}\),
\[
\bigoplus_k \Sigma\delta_{W_k}(i)
\]
controls the end-to-end discrepancy for degree \(i\).
Compare this per-degree accumulated budget to the per-window safety margins \(\mathrm{gap}_{\tau_k}(i)\) to apply Restart and Summability (Appendix~J).
\end{corollary}

% -------------------------
\subsection*{L.5. Operational A/B policy for reflector axes (soft-commuting)}
% -------------------------

\begin{definition}[A/B commutation test]\label{L:def:ab}
Let \(T_A,T_B\) be exact reflectors on \(\Pers^{\mathrm{cons}}_k\) (Appendix~K).
For \(M\in\Pers^{\mathrm{cons}}_k\), define
\[
\Delta_{\mathrm{comm}}(M;A,B)\ :=\ d_{\mathrm{int}}(T_AT_BM,\ T_BT_AM).
\]
Given a tolerance \(\eta\ge 0\) declared per window \(W\) in \texttt{run.yaml}, declare \emph{soft-commuting} if \(\Delta_{\mathrm{comm}}\le\eta\).
If \(\Delta_{\mathrm{comm}}>\eta\), fix a canonical order and ledger the residual as the A/B coordinate in \(\boldsymbol{\delta}_W(i)\).
Skip testing for nested torsions where \(T_A\circ T_B=T_{A\vee B}\) holds strictly (Appendix~K, Proposition~\ref{K:prop:nested}).
All tests are performed after \(\mathbf{P}_i\to\mathbf{T}_\tau\) on the window \(W\).
\end{definition}

% -------------------------
\subsection*{L.6. Tropical specification hooks and run.yaml alignment}
% -------------------------

For reproducibility and audit (Appendix~G, Declaration~\ref{dec:runyaml-extended}), record at minimum:
\begin{itemize}\itemsep0.25em
  \item \texttt{tropical:} configuration (e.g.\ \texttt{bins.\{width,range\}}) and \texttt{policy:"after\_collapse\_only"}.
  \item \texttt{awfs\_2cell.two\_cell\_bounds:} per-channel bounds (mirror/transfer) used as coordinates of \(\boldsymbol{\delta}\).
  \item Per window/degree, record both the vector and its scalar collapse \(\Sigma\delta_W(i)=\pi(\boldsymbol{\delta}_W(i))\), together with the window id and in-force \(\tau\).
\end{itemize}

\begin{remark}[Illustrative YAML snippet]\label{L:rk:yaml-snip}
\begin{verbatim}
tropical:
  bins: { width: 0.05, range: [0.0, 2.0] }
  policy: "after_collapse_only"
awfs_2cell:
  two_cell_bounds:
    mirror_collapse: 0.006
    transfer_collapse: 0.004
window:
  id: "W03"
  degree: 1
  tau: 0.40
  budget:
    delta_vector: { mirror: 0.006, transfer: 0.004, AB: 0.002 }
    sum_delta: 0.012   # π(delta_vector) under the declared quantale
  b_gate_plus:
    passed: true
    gap_tau: 0.09
\end{verbatim}
\end{remark}

% -------------------------
\subsection*{L.7. Minimal pseudocode (product-ledger, direct)}
% -------------------------

\begin{verbatim}
# steps on window W, degree i; each step carries a coordinate vector in V^{×m}
def window_budget_vector(steps, i):
    vec = {"mirror": 0.0, "transfer": 0.0, "AB": 0.0}   # V=[0,∞], ⊕=+
    for s in steps:
        kind = s["kind"]          # "mirror" | "transfer" | "AB"
        delta = s["delta"]        # measured after P_i -> T_tau on window W
        vec[kind] += delta        # coordinatewise ⊕
    return vec

def collapse_quantale(vec):
    # π: product -> base quantale (here: sum of coordinates)
    return vec["mirror"] + vec["transfer"] + vec["AB"]
\end{verbatim}

% -------------------------
\subsection*{L.8. Edge cases and guard-rails}
% -------------------------

\begin{itemize}\itemsep0.25em
  \item \emph{No (H2), no bound.} Without a controlled \(2\)-cell, no quantitative commutation estimate exists; treat the step as unbudgeted and reject (or mark as Type~IV risk) unless separately audited.
  \item \emph{Window/\(\tau\) coherence.} All measurements must use the same window and in-force \(\tau\) as B-Gate\(^{+}\) and the \(\delta\)-ledger; otherwise the audit is invalid.
  \item \emph{Per-degree reporting.} Log per-degree vectors \(\boldsymbol{\delta}_W(i)\) and per-degree totals \(\Sigma\delta_W(i)\); do not rely only on cross-degree sums, as small residuals may hide in totals.
  \item \emph{Multiple nonnested axes.} Pairwise soft-commuting does not imply global confluence; fix a canonical order, test adjacent pairs, and ledger residuals (Appendix~K, Remark~\ref{K:rk:nonconfluent}).
  \item \emph{Adaptive thresholds.} If \(\tau\) adapts online, every defect entry must carry the \(\tau\) in force at measurement time; do not retroactively migrate defects across \(\tau\).
\end{itemize}

\medskip
\noindent\textbf{Summary.}
Assuming a Mirror/Tropical operator \(U\) admits a \(\delta(i,\tau)\)-controlled \(2\)-cell against the collapse lift \(C_\tau\), the post-collapse discrepancy between \(U\!\circ\!C_\tau\) and \(C_\tau\!\circ\!U\) is bounded by \(\delta(i,\tau)\) and remains bounded under further \(1\)-Lipschitz B-side post-processing (Theorem~\ref{L:thm:q-comm}).
Commutation costs are accumulated directly per window in a product-quantale \(\delta\)-ledger and collapsed exactly once by a homomorphism \(\pi\), preventing double counting and enabling deterministic auditing (Definitions~\ref{L:def:q-product}--\ref{L:def:vec}, Proposition~\ref{L:prop:strict}).
The resulting scalar \(\Sigma\delta_W(i)\) integrates with B-Gate\(^{+}\), Restart, and Summability (Appendix~J), while Tropical configuration, 2-cell bounds, window ids, degrees, and in-force \(\tau\) are recorded in \texttt{run.yaml} per Appendix~G.



% =========================
\section*{Appendix M. (Optional) Lax Monoidal Compatibility [Spec + Windowed Usage + Budget Integration] (reinforced)}
% =========================
\phantomsection
\addcontentsline{toc}{section}{Appendix M. (Optional) Lax Monoidal Compatibility}
\refstepcounter{section}
\label{M:lax-monoidal}

\bigskip

\noindent
\textbf{Status (canon).}
Statements tagged \emph{[Spec]} are operational, windowed contracts.
All \emph{strict equalities and invariants} are asserted at the persistence layer in the one-parameter constructible category \(\Pers^{\mathrm{cons}}_\kk\).
Chain-/filtered-layer claims are used only \emph{up to filtered quasi-isomorphism (f.q.i.)} in \(\Ho(\mathsf{FiltCh}(\kk))\) and are accepted only insofar as they reduce to persistence-layer measurements after the mandatory order
\[
\boxed{\ \text{for each }t\ \Rightarrow\ \mathbf{P}_i\ \Rightarrow\ \mathbf{T}_\tau\ \Rightarrow\ \text{compare}\ }
\]
on the B-side single layer (Appendix~G).

% ------------------------------------------------------------
\subsection*{M.0. Standing conventions and scope}
% ------------------------------------------------------------

\paragraph{Ground field and categories.}
Fix a field \(\kk\).
Let \(\Pers^{\mathrm{cons}}_\kk\) denote the category of constructible (1-parameter) persistence modules \(M:\RR\to\Vect_\kk\), i.e.\ pointwise finite-dimensional with locally finite critical parameters on bounded windows (equivalently, locally finite barcodes).
Filtered chain complexes are denoted \(F=(F^t)_{t\in\RR}\) with each \(F^t\) a bounded (on windows) chain complex of finite-dimensional \(\kk\)-vector spaces and locally finite changes in \(t\).

\paragraph{Scope policy for (co)limits.}
Whenever filtered (co)limits are invoked, they are computed in the functor category \([\RR,\Vect_\kk]\) under the scope policy of Appendix~A, Remark~\ref{A:rk:filtered-colimits}, and only then (when explicitly stated) returned to the constructible range under the logged window/definability hypotheses (Appendix~G).

\paragraph{Interleaving metric and truncation.}
We use the interleaving metric \(\dint\) on \(\Pers^{\mathrm{cons}}_\kk\) (bottleneck distance on barcodes in the constructible 1D case).
For each \(\tau>0\), \(\mathbf{T}_\tau:\Pers^{\mathrm{cons}}_\kk\to\Pers^{\mathrm{cons}}_\kk\) denotes the exact bar-deletion reflector (Appendix~A): it is exact and \(1\)-Lipschitz for \(\dint\).
On the filtered layer, \(C_\tau\) denotes a chosen lift of \(\mathbf{T}_\tau\), defined only up to f.q.i.\ in \(\Ho(\mathsf{FiltCh}(\kk))\); endpoint/infinite-bar policies are centralized (Chapter~2, Appendix~G).
No claim of strict functorial equality \( \mathbf{P}_i(C_\tau F)=\mathbf{T}_\tau(\mathbf{P}_iF)\) is made unless reduced to persistence-layer comparison and logged as a policy claim.

\paragraph{Windowing and measurability.}
Fix MECE, right-open windows on \([0,\infty)\): \(\mathcal{W}=\{W_j=[w_j,w_{j+1})\}_{j\in J}\), locally finite on bounded ranges (Appendix~G).
For any \(M\in\Pers^{\mathrm{cons}}_\kk\), the rank function \(t\mapsto \dim_\kk M(t)\) is piecewise constant with locally finite jumps on bounded windows, hence measurable and integrable.

\paragraph{Energy functional (windowed).}
For \(\sigma>0\) and a persistence module \(M\), define the clipped rank integral
\begin{equation}\label{M:eq:energy-module}
  \PE^{\le \sigma}(M)\ :=\ \int_0^\sigma \dim_\kk M(t)\,dt.
\end{equation}
For a filtered complex \(F\) and degree \(i\), define
\begin{equation}\label{M:eq:energy-degree}
  \PE_i^{\le \sigma}(F)\ :=\ \PE^{\le \sigma}\!\big(\mathbf{P}_i(F)\big)\ =\ \int_0^\sigma \betti_i(F;t)\,dt,
  \qquad \betti_i(F;t):=\dim_\kk \mathbf{P}_i(F)(t).
\end{equation}
All integrals are Lebesgue integrals; constructibility implies they reduce to finite sums on bounded windows.

\paragraph{Pointwise tensor.}
On persistence modules and filtered complexes we use the pointwise tensor:
\[
  (M\otimes N)(t):=M(t)\otimes_\kk N(t),\qquad
  (F\otimes G)^t := F^t\otimes_\kk G^t.
\]
All tensors are over \(\kk\).

\paragraph{[Spec] scope.}
This appendix provides windowed, reproducible contracts about how collapse interacts with tensor \emph{laxly}.
No assertion of strict symmetric monoidality for \(C_\tau\) (or \(\mathbf{T}_\tau\)) is made beyond what is explicitly stated and (when needed) measured after \(\mathbf{P}_i\to\mathbf{T}_\tau\).

% ------------------------------------------------------------
\subsection*{M.1. Hypotheses and the laxator}
% ------------------------------------------------------------

We work under the following hypotheses.

\begin{description}[leftmargin=2.2em,labelindent=0em]
  \item[(M1) Exactness and constructibility of pointwise tensor.]
  The pointwise tensor \(\otimes\) on \(\Vect_\kk\) is exact and biadditive; hence the induced pointwise tensor on \([\RR,\Vect_\kk]\) is exact in each variable.
  Moreover, for \(M,N\in\Pers^{\mathrm{cons}}_\kk\), the tensor \(M\otimes N\) lies in \(\Pers^{\mathrm{cons}}_\kk\) on bounded windows.
  \item[(M2) K\"unneth over a field (pointwise, on windows).]
  For filtered complexes \(F,G\) satisfying the window-bounded finiteness assumptions above, for each \(t\in\RR\) and each \(i\in\ZZ\),
  \[
    H_i\big((F\otimes G)^t\big)\ \cong\ \bigoplus_{p+q=i} H_p(F^t)\otimes_\kk H_q(G^t),
  \]
  naturally in \((F,G)\) and \(t\).
  Equivalently (after assembling in \(t\)), in \(\Pers^{\mathrm{cons}}_\kk\) one has a natural isomorphism
  \begin{equation}\label{M:eq:kun}
    \mathbf{P}_i(F\otimes G)\ \cong\ \bigoplus_{p+q=i}\ \mathbf{P}_p(F)\otimes \mathbf{P}_q(G),
  \end{equation}
  on any bounded window where both sides are constructible.
\item[(M3) Lax compatibility for collapse (filtered layer) \SpecBadge.]
  \emph{[Spec]} There exists a natural transformation (the laxator), in \(\Ho(\mathsf{FiltCh}(\kk))\) up to f.q.i.,
  \begin{equation}\label{M:eq:laxator}
    \lambda_{\tau,F,G}:\ C_\tau(F\otimes G)\ \Longrightarrow\ C_\tau F\ \otimes\ C_\tau G,
  \end{equation}
  natural in \(F,G\) and \(\tau\), whose operational meaning is only through persistence-layer measurements after \(\mathbf{P}_i\to\mathbf{T}_\tau\).
\end{description}

We will occasionally strengthen (M3) to:

\medskip
\noindent
\(\mathbf{(M3^+)}\) \emph{[Spec]} For all \(t\in\RR\) and degrees \(i\), the induced map in homology
\[
  H_i\big(\lambda_{\tau,F,G}^t\big):\ H_i\big((C_\tau(F\otimes G))^t\big)\ \longrightarrow\ H_i\big((C_\tau F\otimes C_\tau G)^t\big)
\]
is a monomorphism (equivalently, ranks do not decrease under \(\lambda\) pointwise in \((t,i)\)).

\begin{remark}[Intervals under tensor]\label{M:rem:interval-tensor}
For interval modules over \(\kk\),
\[
  \kk_{[a,b)}\otimes \kk_{[c,d)} \ \cong\
  \begin{cases}
     \kk_{[\max\{a,c\},\ \min\{b,d\})}, & \text{if } [a,b)\cap[c,d)\neq\emptyset,\\
     0, & \text{otherwise}.
  \end{cases}
\]
Thus tensor intersects lifespans; at the barcode level this is the field K\"unneth rule.
\end{remark}

\begin{remark}[Constructibility under tensor]\label{M:rem:constructible}
On any bounded window, only finitely many bars meet; hence the critical set of \(M\otimes N\) is locally finite and pointwise dimensions remain finite. Therefore \(M\otimes N\in\Pers^{\mathrm{cons}}_\kk\) on bounded windows.
\end{remark}

\begin{remark}[Monotonicity scope]\label{M:rem:mono-scope}
Tensor is neither purely deletion-type nor purely inclusion-type in general. Accordingly, we confine monotonicity claims to:
(i) windowed energy bounds, and
(ii) regimes where a monomorphic laxator (M3\(^{+}\)) is verified or where a persistence-layer injection is used (Section~M.3\(^{\dagger}\)).
\end{remark}

% ------------------------------------------------------------
\subsection*{M.2. Windowed energy via overlap integrals}
% ------------------------------------------------------------

\begin{definition}[Windowed seminorms]\label{M:def:window-norms}
For \(M\in\Pers^{\mathrm{cons}}_\kk\) and window \([0,\sigma]\), define
\[
  \|M\|_{E_1}^{(\sigma)}\ :=\ \PE^{\le \sigma}(M),\qquad
  \|M\|_{\infty}^{(\sigma)}\ :=\ \sup_{t\in[0,\sigma]}\dim_\kk M(t).
\]
For a filtered complex \(F\) and degree \(i\), set
\(
\|F\|_{E_1}^{(i,\sigma)}:=\|\mathbf{P}_i(F)\|_{E_1}^{(\sigma)}
\)
and
\(
\|F\|_{\infty}^{(i,\sigma)}:=\|\mathbf{P}_i(F)\|_{\infty}^{(\sigma)}.
\)
\end{definition}

\begin{theorem}[Convolution identity (K\"unneth) and upper bounds]\label{M:thm:conv}
Assume \emph{(M1)--(M2)}. Then for all filtered complexes \(F,G\), degrees \(i\), and \(\sigma>0\),
\begin{equation}\label{M:eq:conv-id}
  \PE_i^{\le \sigma}(F\otimes G)\ =\
  \sum_{p+q=i}\ \int_0^\sigma \betti_p(F;t)\,\betti_q(G;t)\,dt,
\end{equation}
where the sum is finite on windows due to bounded homological range.
Consequently,
\begin{equation}\label{M:eq:conv-sup}
  \PE_i^{\le \sigma}(F\otimes G)\ \le\
  \sum_{p+q=i}\Big(\sup_{t\in[0,\sigma]}\betti_q(G;t)\Big)\,\PE_p^{\le \sigma}(F),
\end{equation}
and symmetrically with \((F,p)\leftrightarrow(G,q)\).
Equivalently, in seminorm form,
\[
  \|F\otimes G\|_{E_1}^{(i,\sigma)}\ \le\
  \sum_{p+q=i}\ \|F\|_{E_1}^{(p,\sigma)}\ \|G\|_{\infty}^{(q,\sigma)},
\quad
  \|F\otimes G\|_{E_1}^{(i,\sigma)}\ \le\
  \sum_{p+q=i}\ \|F\|_{\infty}^{(p,\sigma)}\ \|G\|_{E_1}^{(q,\sigma)}.
\]
\end{theorem}

\begin{proof}
By \eqref{M:eq:kun}, \(\dim_\kk \mathbf{P}_i(F\otimes G)(t)=\sum_{p+q=i}\dim_\kk \mathbf{P}_p(F)(t)\cdot \dim_\kk \mathbf{P}_q(G)(t)\) for each \(t\).
Integrate over \([0,\sigma]\) to obtain \eqref{M:eq:conv-id}.
Bounding one factor by its \(\sup\) on \([0,\sigma]\) gives \eqref{M:eq:conv-sup} and the seminorm inequalities.
\end{proof}

\begin{remark}[Barcode interpretation]\label{M:rem:barcode}
In the interval-decomposable case (1D constructible over a field), \(\betti_i(F;t)\) counts \(i\)-bars alive at \(t\).
The integrand \(\betti_p(F;t)\betti_q(G;t)\) counts ordered pairs of alive bars; integrating sums lengths of pairwise intersections, matching ``tensor \(=\) intersection'' (Remark~\ref{M:rem:interval-tensor}).
\end{remark}

% ------------------------------------------------------------
\subsection*{M.2$^\ast$. Clip/contract stability (deletion-type monotonicity)}
% ------------------------------------------------------------

\begin{proposition}[Clipping and collapse are nonincreasing]\label{M:prop:clip-contract-stable}
For any \(\sigma>0\):
\begin{enumerate}\itemsep0.2em
  \item \emph{Clipping.} Restricting to a subwindow (MECE, right-open) does not increase \(\|-\|_{E_1}^{(\sigma)}\) nor \(\|-\|_{\infty}^{(\sigma)}\), and leaves values unchanged on \([0,\sigma]\).
  \item \emph{Bar-deletion collapse.} For any \(\tau>0\) and \(M\in\Pers^{\mathrm{cons}}_\kk\),
  \[
    \|\mathbf{T}_\tau M\|_{E_1}^{(\sigma)}\ \le\ \|M\|_{E_1}^{(\sigma)},
    \qquad
    \|\mathbf{T}_\tau M\|_{\infty}^{(\sigma)}\ \le\ \|M\|_{\infty}^{(\sigma)}.
  \]
  Consequently, for filtered \(F\) and degree \(i\),
  \(
  \|\mathbf{T}_\tau\mathbf{P}_i(F)\|_{E_1}^{(\sigma)}\le \|\mathbf{P}_i(F)\|_{E_1}^{(\sigma)}
  \)
  and similarly for \(\|\cdot\|_\infty\).
\end{enumerate}
\end{proposition}

\begin{proof}
Clipping is restriction. Bar deletion implies pointwise \(\dim(\mathbf{T}_\tau M)(t)\le \dim M(t)\) on any window; integrate and take sup.
\end{proof}

% ------------------------------------------------------------
\subsection*{M.3. Collapse vs tensor: laxity and energy dominance}
% ------------------------------------------------------------

\begin{proposition}[Collapsed convolution bound]\label{M:prop:collapsed-conv}
Assume \emph{(M1)--(M2)}. For any \(\tau,\sigma>0\) and degree \(i\),
\[
  \PE^{\le \sigma}\!\Big(\mathbf{T}_\tau\,\mathbf{P}_i(F\otimes G)\Big)
  \ \le\
  \PE^{\le \sigma}\!\Big(\mathbf{P}_i(F\otimes G)\Big)
  \ =\ \sum_{p+q=i}\int_0^\sigma \betti_p(F;t)\betti_q(G;t)\,dt.
\]
\end{proposition}

\begin{proof}
Apply Proposition~\ref{M:prop:clip-contract-stable} to \(\mathbf{P}_i(F\otimes G)\), then Theorem~\ref{M:thm:conv}.
\end{proof}

\begin{theorem}[Energy dominance under a monomorphic laxator \({[Spec]}\)]\label{M:thm:mono-dominance}
Assume \emph{(M3)} and \emph{(M3$^+$)} \emph{[Spec]}. Then for all \(\tau,\sigma>0\) and degrees \(i\),
\[
  \PE_i^{\le \sigma}\big(C_\tau(F\otimes G)\big)\ \le\ \PE_i^{\le \sigma}\big(C_\tau F\otimes C_\tau G\big),
\]
and therefore, by Theorem~\ref{M:thm:conv} applied on the right-hand side where defined,
\[
  \PE_i^{\le \sigma}\big(C_\tau(F\otimes G)\big)\ \le\ \sum_{p+q=i}\ \int_0^\sigma \betti_p(C_\tau F;t)\,\betti_q(C_\tau G;t)\,dt.
\]
\end{theorem}

\begin{proof}
By (M3\(^{+}\)), \(\betti_i(C_\tau(F\otimes G);t)\le \betti_i(C_\tau F\otimes C_\tau G;t)\) for all \(t\in[0,\sigma]\). Integrate.
\end{proof}

% ------------------------------------------------------------
\subsection*{M.3$^\dagger$. Persistence-layer ECF low-pass laxator (explicit) \([Spec]\)}
% ------------------------------------------------------------

\paragraph{Definition (ECF low-pass injection).}
\emph{[Spec]} In the interval regime (1D constructible), view \(\mathbf{T}_\tau\) as deleting all bars of lifespan \(\le\tau\) and keeping all longer bars unchanged (no shrinking).
Define a natural transformation
\begin{equation}\label{M:eq:ecf}
  \lambda^{\mathrm{ECF}}_{\tau,M,N}:\ \mathbf{T}_\tau(M\otimes N)\ \longrightarrow\ (\mathbf{T}_\tau M)\ \otimes\ (\mathbf{T}_\tau N)
\end{equation}
by specifying it on interval summands using \( \kk_I\otimes \kk_J\cong \kk_{I\cap J}\) (Remark~\ref{M:rem:interval-tensor}):
\begin{itemize}\itemsep0.2em
  \item If \(\ell(I\cap J)\le \tau\), then the source summand is killed by \(\mathbf{T}_\tau\), hence maps to \(0\).
  \item If \(\ell(I\cap J)>\tau\), then necessarily \(\ell(I)>\tau\) and \(\ell(J)>\tau\); thus \(\mathbf{T}_\tau \kk_I=\kk_I\), \(\mathbf{T}_\tau \kk_J=\kk_J\), and the map on the \(\kk_{I\cap J}\) summand is the identity inclusion into \(\kk_I\otimes\kk_J\cong\kk_{I\cap J}\).
\end{itemize}
Extend additively to finite sums on bounded windows.

\paragraph{Consequences.}
\emph{[Spec]} The map \(\lambda^{\mathrm{ECF}}_{\tau,M,N}\) is pointwise monomorphic (hence rank-nonincreasing on sources), yielding for any window \([0,\sigma]\):
\[
  \PE^{\le \sigma}\!\big(\mathbf{T}_\tau(M\otimes N)\big)\ \le\ \PE^{\le \sigma}\!\big((\mathbf{T}_\tau M)\otimes(\mathbf{T}_\tau N)\big).
\]
When \(M=\mathbf{P}_p(F)\), \(N=\mathbf{P}_q(G)\), this provides a concrete, window-coherent candidate for the lax behavior used by filtered-level contracts.

\begin{remark}[Use with the canon order]
Any use of \eqref{M:eq:ecf} in the pipeline must be evaluated after the canon order
\(\mathbf{P}_i\to\mathbf{T}_\tau\to\)compare on the B-side, with the same window/\(\tau\) as the ledger (Appendix~G).
\end{remark}

% ------------------------------------------------------------
\subsection*{M.4. Interaction with \texorpdfstring{$\tau$}{tau}-sweeps and stability bands}
% ------------------------------------------------------------

\begin{definition}[\(\tau\)-sweep for energy]\label{M:def:tau-sweep-energy}
Fix a bounded window \([0,\sigma]\) and an object \(M\in\Pers^{\mathrm{cons}}_\kk\).
A \(\tau\)-sweep is a finite or countable increasing array \(\{\tau_\ell\}\subset(0,\infty)\).
A subarray is an \emph{energy stability band} if \(\dim(\mathbf{T}_\tau M)(t)\) is constant for all \(t\in[0,\sigma]\) and all \(\tau\) in the band (equivalently, no bar length of \(M\) lies in the band boundary in a way that affects \([0,\sigma]\)).
\end{definition}

\begin{proposition}[Piecewise constancy in \(\tau\) on bounded windows]\label{M:prop:tau-piecewise}
Fix \(M\in\Pers^{\mathrm{cons}}_\kk\) and \([0,\sigma]\).
There exists a finite critical set \(S\subset(0,\infty)\) (depending on \(M,[0,\sigma]\)) such that
\(\tau\mapsto \PE^{\le \sigma}(\mathbf{T}_\tau M)\) is constant on each connected component of \((0,\infty)\setminus S\).
In particular, energy stability bands are unions of open intervals intersected with the sweep.
\end{proposition}

\begin{proof}
On \([0,\sigma]\), only finitely many bars of \(M\) meet the window. \(\mathbf{T}_\tau\) changes only when \(\tau\) crosses one of these bar lengths; between such values the deleted/retained set is constant, hence the rank function and its integral are constant.
\end{proof}

% ------------------------------------------------------------
\subsection*{M.5. Budget integration and quantitative gaps \([Spec]\)}
% ------------------------------------------------------------

\begin{definition}[Monoidal-laxity budget entry]\label{M:def:budget-lax}
\emph{[Spec]} On a window \(W=[u,u')\), degree \(i\), and threshold \(\tau\), define a measurable persistence-layer laxity gap by
\begin{equation}\label{M:eq:lax-gap}
  \delta^{\mathrm{lax}}_{W}(i,\tau)\ :=\
  d_{\mathrm{int}}\!\Big(\mathbf{T}_\tau\,\mathbf{P}_i\big(C_\tau(F\otimes G)\big),\ \mathbf{T}_\tau\,\mathbf{P}_i\big(C_\tau F\otimes C_\tau G\big)\Big),
\end{equation}
computed after the canon order and using the same window/\(\tau\) as the ledger.
If only an upper bound is available (e.g.\ via ECF injection at the persistence layer), record that upper bound.
\end{definition}

\begin{proposition}[Ledger compatibility (quantale/product-quantale)]\label{M:prop:ledger}
\emph{[Spec]} The entry \(\delta^{\mathrm{lax}}_{W}(i,\tau)\) may be recorded as:
(i) a scalar defect in the base quantale \((V,\oplus)\), or
(ii) a dedicated coordinate (e.g.\ \texttt{lax\_tensor}) in the product quantale \(V^{\times m}\) (Appendix~K--L),
and then aggregated once via the declared quantale law (Appendix~G).
Any subsequent \(1\)-Lipschitz post-processing on the persistence layer does not increase the recorded bound.
\end{proposition}

\begin{remark}[run.yaml alignment]
Log in \texttt{run.yaml} (Appendix~G, Declaration~\ref{dec:runyaml-extended}) the window id, degree \(i\), in-force \(\tau\), and either:
\texttt{budget.delta\_vector.lax\_tensor} (product-ledger) or \texttt{budget.sum\_delta} (scalar),
together with \texttt{policy.after\_collapse\_only=true}. This prevents double counting.
\end{remark}

% ------------------------------------------------------------
\subsection*{M.6. Edge cases and pitfalls}
% ------------------------------------------------------------

\begin{itemize}[leftmargin=2em]\itemsep0.25em
  \item \emph{Nonexact tensor surrogates.} If an operation used in place of \(\otimes\) is not exact/biadditive, K\"unneth and the convolution identity may fail; treat all such cases as \([Spec]\) and require measured bounds.
  \item \emph{Unverified K\"unneth context.} If the hypotheses for \eqref{M:eq:kun} fail (e.g.\ out of the window-bounded regime), do not assert equality \eqref{M:eq:conv-id}; use only measured or ledgered bounds.
  \item \emph{Missing monomorphy.} Without (M3\(^{+}\)) (or without the ECF persistence injection), energy dominance of \(C_\tau(F\otimes G)\) by \(C_\tau F\otimes C_\tau G\) is not guaranteed.
  \item \emph{Window/\(\tau\) mismatch.} All measurements must use the same MECE right-open window and the in-force \(\tau\) as B-Gate\(^{+}\) and the \(\delta\)-ledger; otherwise Restart/Summability pasting can fail (Appendix~J).
\end{itemize}

% ------------------------------------------------------------
\subsection*{M.7. Worked examples and tests}
% ------------------------------------------------------------

\begin{example}[Single-interval bars (persistence level)]
Let \(M=\kk_{[a,b)}\), \(N=\kk_{[c,d)}\). Then \(M\otimes N\cong \kk_{[a,b)\cap[c,d)}\) (Remark~\ref{M:rem:interval-tensor}).
Hence for any \(\sigma>0\),
\[
  \PE^{\le \sigma}(M\otimes N)\ =\ \lambda\!\Big(([a,b)\cap[c,d))\cap[0,\sigma]\Big),
\]
where \(\lambda(\cdot)\) is Lebesgue measure (length).
\end{example}

\begin{example}[Effect of bar-deletion collapse]
If \(\ell([a,b))\le\tau\) or \(\ell([c,d))\le\tau\), then \(\mathbf{T}_\tau M=0\) or \(\mathbf{T}_\tau N=0\).
Moreover, if \(\ell([a,b)\cap[c,d))\le\tau\) then \(\mathbf{T}_\tau(M\otimes N)=0\).
The ECF injection \eqref{M:eq:ecf} implies
\(
\PE^{\le \sigma}(\mathbf{T}_\tau(M\otimes N))\le \PE^{\le \sigma}((\mathbf{T}_\tau M)\otimes(\mathbf{T}_\tau N)).
\)
\end{example}

\begin{proposition}[Test model for (M3\(^{+}\)) on interval sums]\label{M:prop:test-m3plus}
\emph{[Spec]} Let \(F\simeq \bigoplus_r \kk_{I_r}[-p_r]\) and \(G\simeq \bigoplus_s \kk_{J_s}[-q_s]\) be finite direct sums of interval modules (shifted in homological degree) on a bounded window, and let \(C_\tau\) implement bar deletion (low-pass) at scale \(\tau\).
Define \(\lambda_{\tau,F,G}\) on summands by the canonical inclusions induced by
\(
\mathbf{T}_\tau(\kk_{I_r}\otimes\kk_{J_s}) \hookrightarrow \mathbf{T}_\tau\kk_{I_r}\otimes \mathbf{T}_\tau\kk_{J_s}
\),
extended additively and lifted to complexes up to f.q.i.
Then (M3\(^{+}\)) holds on the window, and energy dominance of Theorem~\ref{M:thm:mono-dominance} follows.
\end{proposition}

% ------------------------------------------------------------
\subsection*{M.8. Formal underpinnings (constructibility, measurability, exactness)}
% ------------------------------------------------------------

\begin{lemma}[Constructibility preserved by tensor]\label{M:lem:constructible-tensor}
Assume (M1). If \(M,N\in\Pers^{\mathrm{cons}}_\kk\), then \(M\otimes N\in\Pers^{\mathrm{cons}}_\kk\) on bounded windows.
If \(F,G\) are constructible filtered complexes on windows, then \(F\otimes G\) is constructible on windows.
\end{lemma}

\begin{proof}
For persistence modules, use Remark~\ref{M:rem:constructible}. For filtered complexes, apply the same argument degreewise on windows and use window-bounded finiteness.
\end{proof}

\begin{lemma}[Measurability and finiteness of energy]\label{M:lem:measurable}
For constructible \(M\), \(t\mapsto\dim_\kk M(t)\) is piecewise constant with locally finite jumps on bounded windows, hence \(\PE^{\le \sigma}(M)<\infty\) for all \(\sigma>0\).
\end{lemma}

\begin{proof}
On a bounded window, constructibility implies finitely many rank changes; the rank function is a finite sum of characteristic functions of intervals. Integrability follows.
\end{proof}

\begin{lemma}[\(\mathbf{T}_\tau\) is exact and \(1\)-Lipschitz]\label{M:lem:lipschitz}
\(\mathbf{T}_\tau\) is exact on short exact sequences in \(\Pers^{\mathrm{cons}}_\kk\) and satisfies
\(
\dint(\mathbf{T}_\tau M,\mathbf{T}_\tau N)\le \dint(M,N)
\)
for all \(M,N\).
\end{lemma}

\begin{proof}
This is Appendix~A: \(\mathbf{T}_\tau\) is an exact bar-deletion reflector and \(1\)-Lipschitz for \(d_{\mathrm{int}}\).
\end{proof}

% ------------------------------------------------------------
\subsection*{M.9. Operational checklist (windowed, reproducible) \([Spec]\)}
% ------------------------------------------------------------

For each experiment window \(W=[u,u')\) (or \([0,\sigma]\) by translation):
\begin{itemize}[leftmargin=2em]\itemsep0.25em
  \item Log the tensor context: pairs \((F,G)\), monitored degrees \(i\), and the window id (MECE, right-open).
  \item Log the in-force \(\tau\) and confirm the canon order (after \(\mathbf{P}_i\to\mathbf{T}_\tau\)).
  \item State whether (M2) is invoked on the window; otherwise replace Theorem~\ref{M:thm:conv} by measured estimates only.
  \item If using filtered-layer laxity (M3), specify \(\lambda_{\tau,F,G}\) and whether (M3\(^{+}\)) is verified (e.g.\ rank checks on sampled \(t\) points).
  \item If a persistence-layer bound is used (Section~M.3\(^{\dagger}\)), log it as \texttt{lax\_tensor} (product ledger) or as a scalar defect.
  \item Aggregate defects using the declared quantale law exactly once (Appendix~G, Appendix~K--L) and compare against B-Gate\(^{+}\) gaps for Restart/Summability (Appendix~J).
\end{itemize}

% ------------------------------------------------------------
\subsection*{M.10. Summary of contracts \([Spec]\)}
% ------------------------------------------------------------

On bounded windows, tensor admits a deterministic energy calculus: under pointwise field K\"unneth, the windowed energy of \(F\otimes G\) in degree \(i\) equals an overlap integral of Betti curves (Theorem~\ref{M:thm:conv}).
Bar-deletion collapse \(\mathbf{T}_\tau\) is deletion-type and nonincreasing for both \(E_1\) (energy) and \(L^\infty\) seminorms (Proposition~\ref{M:prop:clip-contract-stable}).
Lax monoidal compatibility of collapse is used only as a windowed contract: either via a verified monomorphic laxator (M3\(^{+}\)) giving energy dominance (Theorem~\ref{M:thm:mono-dominance}), or via an explicit persistence-layer injection in the interval regime (Section~M.3\(^{\dagger}\)).
Any residual ``laxity gap'' may be recorded as a quantale-budget defect and integrated with the product-ledger accounting of Appendices~K--L and the Restart/Summability pasting of Appendix~J.

% ------------------------------------------------------------
\subsection*{M.11. Formalization blueprint (Lean/Coq) \([Spec]\)}
% ------------------------------------------------------------

A minimal API for formalization includes:
\begin{itemize}[leftmargin=2em]\itemsep0.25em
  \item \texttt{PersCons} (constructible persistence), pointwise tensor \texttt{tensor} with exactness and constructibility lemmas (Lemma~\ref{M:lem:constructible-tensor}).
  \item A windowed energy functional \texttt{PE} on persistence modules and \texttt{PEi} on filtered objects via \(\mathbf{P}_i\).
  \item A K\"unneth interface yielding \eqref{M:eq:kun} on windows and the energy identity \eqref{M:eq:conv-id}.
  \item The reflector \(\mathbf{T}_\tau\) with exactness and \(1\)-Lipschitz proofs (Appendix~A; Lemma~\ref{M:lem:lipschitz}).
  \item Optional \texttt{laxator} (filtered layer, up to f.q.i.) and predicate \texttt{laxator\_mono}; optional persistence-layer \texttt{ecf\_lax} implementing \eqref{M:eq:ecf}.
  \item Budget hooks aligned with \texttt{run.yaml} keys: quantale parameters, product-ledger coordinate \texttt{lax\_tensor}, and single aggregation law.
\end{itemize}



% =========================
\section*{Appendix N. Projection Formula and Base Change [Spec + Windowed Protocol + Budget Integration] (reinforced)}
% =========================
\phantomsection
\addcontentsline{toc}{section}{Appendix N. Projection Formula and Base Change}
\refstepcounter{section}
\label{N:pf-bc}

\paragraph{Standing conventions (canon).}
We work over a coefficient \emph{field} \(\Lambda\) (e.g.\ a base field \(k\), or at \emph{[Spec]} level a Novikov field), and all statements below are phrased uniformly for \(\Lambda\).
All persistence modules are constructible; all equality/identity claims are asserted at the persistence layer in the one-parameter constructible category \(\Pers^{\mathrm{cons}}_\Lambda\).
Filtered (co)limits, when used, are computed objectwise in \([\RR,\Vect_\Lambda]\) under the scope policy of Appendix~A, Remark~\ref{A:rk:filtered-colimits}, and only then (when stated) returned to the constructible range; any such return must be logged (Appendix~G) and may be enforced by \(\mathbf{T}_\tau\).
Distances are measured by the interleaving metric \(d_{\mathrm{int}}\) (=\ bottleneck in the constructible \(1\)D setting).
Truncation \(\mathbf{T}_\tau\) is the exact bar-deletion reflector and is \(1\)\hyp Lipschitz (Appendix~A, Proposition~\ref{A:prop:lipschitz}); on the filtered-complex side we use \(C_\tau\) \emph{only up to f.q.i.} in \(\Ho(\mathsf{FiltCh}(\Lambda))\), and write \(\mathbf{P}_i\) for degree–\(i\) persistence.
Global conventions: \(\Ext^1\) is always taken against \(\Lambda[0]\); the energy exponent satisfies \(\alpha>0\) (default \(\alpha=1\)); windows are MECE and right–open (Appendix~G).
References to “infinite bars/generic dimension” point to Appendix~D, Remark~\ref{D:rem:generic-dim}.
Monotonicity claims follow the global policy: deletion-type only (nonincreasing), inclusion-type merely stable/non-expansive.
All comparisons follow the mandatory order
\[
\boxed{\ \text{for each }t\ \Rightarrow\ \mathbf{P}_i\ \Rightarrow\ \mathbf{T}_\tau\ \Rightarrow\ \text{compare}\ }
\]
on the B-side single layer, after collapse (Appendix~G).

% Number subsections in Appendix N as N.1, N.2, ...
\setcounter{subsection}{0}
\renewcommand\thesubsection{N.\arabic{subsection}}
\makeatletter
\renewcommand\@seccntformat[1]{\csname the#1\endcsname.\quad}
\makeatother

% -------------------------
\subsection{Hypotheses (PF/BC layer) and normalizations}
\label{N:hyp}
% -------------------------
\emph{[Spec]} We fix a class of filtered spaces and maps \(f:X\to Y\) for which a six-functor formalism is available on
\(D^b_c(\mathrm{Shv}_\Lambda(-))\) (bounded derived category of \(\Lambda\)-constructible sheaves), and adopt:

\begin{itemize}\itemsep0.35em
  \item[(N0)] \textbf{Coefficients and Tor control.}
  \(\Lambda\) is a field.
  All objects invoked in tensor expressions have finite Tor-dimension, hence no \(\mathrm{Tor}\)-corrections appear in K\"unneth/projection-formula statements.
  \item[(N1)] \textbf{Constructibility/finiteness.}
  All sheaves are constructible; we use the standard \(t\)-structure on \(D^b_c\); (co)homology objects are finite dimensional on bounded windows.
  \item[(N2)] \textbf{PF/BC hypotheses.}
  Projection formula (PF) and base change (BC) are invoked only under the usual hypotheses:
  \begin{itemize}\itemsep0.1em
    \item PF: for \(f\) \emph{proper}, \(Rf_\ast(A\otimes^\mathbf{L} f^\ast B)\simeq Rf_\ast A\otimes^\mathbf{L} B\).
    \item BC: for a Cartesian square and \(f\) proper (or smooth with the appropriate \(f^!\) variant),
    \(Lg^\ast Rf_\ast A \simeq Rf'_\ast Lg'{}^\ast A\).
  \end{itemize}
  \item[(N3)] \textbf{Degree normalization and objectwise evaluation in \(t\).}
  We use \emph{cohomological} indexing on \(D^b_c\) and evaluate realizations objectwise in \(t\):
  \[
     \mathcal{R}(F)^t \cong \mathcal{R}(F^t),\qquad
     \mathbf{P}_i(F)(t)\ \cong\ H_i(F^t)\ \cong\ H^{-i}\!\big(\mathcal{R}(F^t)\big).
  \]
  Hence \(\mathbf{P}_i\) reads off the \((-i)\)-th cohomology sheaf along the filtration.
  Any geometric shift from \(f^!\) (smooth case) is absorbed by this bookkeeping and must be logged if used.
  \item[(N4)] \textbf{Tensor and window legality.}
  Tensor is pointwise in \(t\):
  \((A\otimes^\mathbf{L} B)^t \cong A^t\otimes^\mathbf{L} B^t\), exact over the field \(\Lambda\).
  Constructibility on bounded windows is preserved; integrals and event counts on bounded windows are justified by Appendix~H (Tonelli/finite-event regime).
\end{itemize}

\begin{remark}[Scope and return to constructible]\label{N:rk:scope-return}
All PF/BC comparisons below are formed in the derived category, evaluated objectwise in \(t\), and then passed to the persistence layer via \(\mathbf{P}_i\).
Any filtered colimit is taken in \([\RR,\Vect_\Lambda]\) under Appendix~A’s scope policy and is returned to \(\Pers^{\mathrm{cons}}_\Lambda\) only by verified constructibility on the window, or by applying \(\mathbf{T}_\tau\) (and logging the policy in \texttt{run.yaml}).
\end{remark}

% -------------------------
\subsection{Projection formula / base change at the persistence layer}
\label{N:pfbc-persistence}
% -------------------------
Let \(f:X\to Y\) and a Cartesian square
\[
\vcenter{\xymatrix{
X'\ar[r]^{g'}\ar[d]_{f'} & X\ar[d]^f\\
Y'\ar[r]^g & Y
}}
\]
satisfy \textup{(N0)–(N2)}.
For filtered complexes \(F\) on \(X\) and \(G\) on \(Y\), write \(\mathcal{R}(F),\mathcal{R}(G)\) for their realizations in \(D^b_c\), computed objectwise in \(t\).

\begin{theorem}[PF/BC transported to \texorpdfstring{$\mathbf{P}_i$ and $\mathbf{T}_\tau$}{P\_i and T\_\tau} \textup{[Spec]}]\label{N:thm:pf-bc}
Under \textup{(N0)--(N4)} the following canonical isomorphisms hold in \(\Pers^{\mathrm{cons}}_\Lambda\),
\emph{natural in} \((i,\tau,f,g,F,G)\), and are asserted \emph{after truncation by \(\mathbf{T}_\tau\)} (canon order):
\begin{align*}
\textup{(PF)}\quad
& \mathbf{T}_\tau\,\mathbf{P}_i\!\Big(Rf_\ast\big(\mathcal{R}(F)\otimes^{\mathbf{L}} f^\ast\mathcal{R}(G)\big)\Big)
 \ \cong\
 \mathbf{T}_\tau\,\mathbf{P}_i\!\Big(Rf_\ast\mathcal{R}(F)\otimes^{\mathbf{L}}\mathcal{R}(G)\Big),
\\[0.25em]
\textup{(BC)}\quad
& \mathbf{T}_\tau\,\mathbf{P}_i\!\Big(Lg^\ast Rf_\ast \mathcal{R}(F)\Big)
 \ \cong\
 \mathbf{T}_\tau\,\mathbf{P}_i\!\Big(Rf'_\ast Lg'{}^\ast \mathcal{R}(F)\Big).
\end{align*}
\end{theorem}

\begin{proof}[Proof sketch]
PF and BC are canonical isomorphisms in \(D^{b}_{c}\) under \textup{(N0)--(N2)}.
Evaluate objectwise in \(t\) (N3), identify \(\mathbf{P}_i\) with \((-i)\)-cohomology in \(t\), then apply \(\mathbf{T}_\tau\).
Exactness of \(\mathbf{T}_\tau\) (Appendix~A) preserves short exact sequences and hence preserves isomorphisms after passing to \(\mathbf{P}_i\).
Naturality in \((f,g,F,G)\) follows from naturality of PF/BC in the six-functor formalism; naturality in \((i,\tau)\) follows from functoriality of \(\mathbf{P}_i\) and \(\mathbf{T}_\tau\).
\end{proof}

\begin{declaration}[PF/BC after collapse (non-negotiable canon)]\label{N:dec:pfbc-after}
Projection formula and base change are audited and asserted at the persistence layer \emph{only after} applying \(\mathbf{T}_\tau\) (equivalently, after \(C_\tau\) on the filtered side, then \(\mathbf{P}_i\), then \(\mathbf{T}_\tau\)).
Any observed post-truncation discrepancy is treated as an implementation/hypothesis violation and is logged as budget terms \(\delta^{\mathrm{disc}}\) and \(\delta^{\mathrm{meas}}\) (never as an ``algebraic'' PF/BC defect).
\end{declaration}

\begin{corollary}[Compatibility with a filtered lift \(C_\tau\) \textup{[Spec]}]\label{N:cor:Ctau}
Assume in addition that \(C_\tau\) is a filtered lift of \(\mathbf{T}_\tau\) in the sense of Appendix~B: for each \(i\),
\(
\mathbf{P}_i(C_\tau F)\simeq \mathbf{T}_\tau \mathbf{P}_i(F)
\)
up to f.q.i.\ on the filtered side and equality in \(\Pers^{\mathrm{cons}}_\Lambda\) after applying \(\mathbf{T}_\tau\).
Then the PF/BC isomorphisms of Theorem~\ref{N:thm:pf-bc} may be expressed using \(\mathcal{R}(C_\tau F)\) and \(\mathcal{R}(C_\tau G)\) on the filtered-realization side, provided comparisons still follow the canon order \(\mathbf{P}_i\to\mathbf{T}_\tau\).
\end{corollary}

\begin{remark}[What is \emph{not} claimed]\label{N:rk:not-claimed}
No global Lipschitz control for PF/BC is asserted beyond:
(i) the \(1\)\nobreakdash-Lipschitz property of \(\mathbf{T}_\tau\) (Appendix~A), and
(ii) any additional commutation controls explicitly ledgered (Appendix~L).
PF/BC are exact identities at the sheaf layer; any persistence-level drift \emph{after truncation} indicates violated hypotheses, window mismatch, or implementation drift and must be logged (Section~\ref{N:window-protocol}).
\end{remark}

% -------------------------
\subsection{Windowed protocol and reproducible audit}
\label{N:window-protocol}
% -------------------------
All PF/BC audits are \emph{windowed}. The mandatory comparison order is:
\[
\boxed{\ \text{for each } t\ \Longrightarrow\ \text{apply } \mathbf{P}_i\ \Longrightarrow\ \text{apply } \mathbf{T}_\tau\ \Longrightarrow\ \text{compare in }\Pers^{\mathrm{cons}}_\Lambda\ }.
\]
Use the \emph{same} MECE, right-open windows and the \emph{same} \(\tau\) as the rest of the run (Appendix~G).

\begin{declaration}[Audit checklist (per window; \textup{[Spec]})]\label{N:dec:audit}
Record in \texttt{run.yaml}:
(i) the PF/BC hypothesis set used (proper/smooth, finite Tor, degree normalization and any \(f^!\) shift policy);
(ii) the functors and objects compared (e.g.\ \(Rf_\ast(\mathcal{R}(F)\otimes f^\ast\mathcal{R}(G))\) vs.\ \(Rf_\ast\mathcal{R}(F)\otimes\mathcal{R}(G)\));
(iii) the verdict (\texttt{passed:true/false}) and, if any post-truncation drift is observed, its breakdown into \(\delta^{\mathrm{disc}}\) and \(\delta^{\mathrm{meas}}\) together with tolerance(s);
(iv) the window id and \(\tau\) used for the comparison, matching those used by B-Gate\(^{+}\) and the \(\delta\)-ledger (Appendix~G).
\end{declaration}

\begin{remark}[Definable coverage]\label{N:rk:def-cov}
If windows and filtrations are definable in an o-minimal structure, event counts on bounded windows are finite (Appendix~H), coverage is decidable, and PF/BC checks reduce to finitely many window-wise persistence comparisons recorded in \texttt{run.yaml}.
\end{remark}

% -------------------------
\subsection{Budget integration and window pasting}
\label{N:budget}
% -------------------------
When the hypotheses hold and comparisons follow the canon order, PF/BC contribute \(\delta^{\mathrm{alg}}=0\) (algebraic defect zero).
Only discretization/measurement residuals are budgeted.

\begin{definition}[PF/BC residual defect (windowed)]\label{N:def:pfbc-defect}
On a window \(W\) and degree \(i\), define the PF and BC observed residuals (post-truncation) by
\begin{align*}
\delta^{\mathrm{PF}}_{W}(i;\tau)
&:= d_{\mathrm{int}}\!\Big(
\mathbf{T}_\tau\mathbf{P}_i(Rf_\ast(\mathcal{R}(F)\!\otimes^{\mathbf{L}}\! f^\ast\mathcal{R}(G))),
\mathbf{T}_\tau\mathbf{P}_i(Rf_\ast\mathcal{R}(F)\!\otimes^{\mathbf{L}}\!\mathcal{R}(G))
\Big),
\\
\delta^{\mathrm{BC}}_{W}(i;\tau)
&:= d_{\mathrm{int}}\!\Big(
\mathbf{T}_\tau\mathbf{P}_i(Lg^\ast Rf_\ast \mathcal{R}(F)),
\mathbf{T}_\tau\mathbf{P}_i(Rf'_\ast Lg'{}^\ast \mathcal{R}(F))
\Big).
\end{align*}
These are expected to be \(0\) in ideal algebraic settings; any nonzero value is recorded as \(\delta^{\mathrm{disc}}+\delta^{\mathrm{meas}}\) (and optionally split further by implementation source).
\end{definition}

\begin{proposition}[Pipeline budget integration]\label{N:prop:budget}
Fix a commutative quantale budget law \((V,\oplus,\le,0)\) (Appendix~K--L) and a window \(W\).
In a pipeline that includes Mirror/Transfer commutation bounds and A/B residuals (Appendix~L/K), the window budget in degree \(i\) is
\[
\Sigma\delta_W(i)\ =\
\Big(\widehat{\oplus}_{\text{Mirror/Transfer}} \boldsymbol{\delta}^{(j)}(i,\tau_j)\Big)
\ \widehat{\oplus}\
\Big(\widehat{\oplus}_{\text{A/B fails}} \boldsymbol{\delta}^{(\mathrm{AB})}\Big)
\ \widehat{\oplus}\
\boldsymbol{\delta}^{(\mathrm{PF/BC})}_{W}(i),
\]
where \(\boldsymbol{\delta}^{(\mathrm{PF/BC})}_{W}(i)\) has only the PF/BC coordinate(s) nonzero, equal to the recorded residuals (Definition~\ref{N:def:pfbc-defect}) under the chosen split.
Collapsing once by the declared homomorphism \(\pi:V^{\times m}\to V\) yields the scalar \(\Sigma\delta_W(i)\), compatible with Appendix~L and Appendix~K.
This scalar is nonincreasing under any subsequent \(1\)\hyp Lipschitz persistence post-processing.
\end{proposition}

\begin{corollary}[Window pasting (Restart/Summability alignment)]\label{N:cor:pasting}
Over a MECE partition \(\{W_k\}\), the sum (quantale aggregation) of \(\Sigma\delta_{W_k}(i)\) controls end-to-end discrepancy.
Compare \(\Sigma\delta_{W_k}(i)\) against the per-window safety margins \(\mathrm{gap}_{\tau_k}(i)\) to apply Restart and Summability (Appendix~J), enforcing \(\mathrm{gap}_{\tau_k}(i)>\Sigma\delta_{W_k}(i)\) per window.
\end{corollary}

% -------------------------
\subsection{Ext–tests under change of functor / coefficients}
\label{N:ext-tests}
% -------------------------
PF/BC isomorphisms transport \(\Ext^1\)-tests along canonical identifications.

\begin{proposition}[Portability of the \(\Ext^1\)–test (sheaf layer)]\label{N:prop:ext-port}
Under \textup{(N0)–(N2)}, any isomorphism \(A \xrightarrow{\sim} B\) in \(D^{b}_{c}\) induces a natural isomorphism
\[
\Ext^1(A,\Lambda)\ \xrightarrow{\ \sim\ }\ \Ext^1(B,\Lambda).
\]
In particular, if \(\Ext^1(\mathcal{R}(C_\tau F),\Lambda)=0\), then \(\Ext^1\) also vanishes for any PF/BC partner of \(\mathcal{R}(C_\tau F)\) (under the same hypothesis regime).
\end{proposition}

\begin{remark}[Bridge stays one–way]\label{N:rk:bridge}
The one–way bridge \(\mathrm{PH}_1\Rightarrow \Ext^1\) (Appendix~C) is unchanged.
PF/BC only \emph{transport} the \(\Ext^1\) test across equivalent sheaf-theoretic descriptions; no converse implication and no new equivalence is claimed.
\end{remark}

% -------------------------
\subsection{Functoriality and two–out–of–three (windowed Beck–Chevalley)}
\label{N:2of3}
% -------------------------
\begin{proposition}[\textnormal{\textbf{[Spec]}} Two–out–of–three for PF/BC squares (after collapse)]\label{N:prop:2of3}
Fix a Cartesian square, a window \(W\), and \(\tau>0\).
If any two among the PF/BC isomorphisms (evaluated objectwise in \(t\), then passed through \(\mathbf{P}_i\), then truncated by \(\mathbf{T}_\tau\)) hold as isomorphisms in \(\Pers^{\mathrm{cons}}_\Lambda\), then the third holds as well (all comparisons performed on \(W\) and in degree \(i\)).
\end{proposition}

\begin{proof}[Proof sketch]
Two–out–of–three holds in \(D^b_c\) for composable isomorphisms arising from PF/BC.
Objectwise evaluation and exactness of \(\mathbf{T}_\tau\) transport this to \(\Pers^{\mathrm{cons}}_\Lambda\) after applying the canon order.
\end{proof}

% -------------------------
\subsection{Implementation notes and checkpoints}
\label{N:impl}
% -------------------------
\begin{itemize}\itemsep0.35em
  \item \textbf{Finite windows / constructibility.}
  On bounded \(t\)-windows, bar events are finite (Appendix~H). PF/BC are computed objectwise in \(t\) and the resulting persistence objects remain constructible on windows; if not, enforce return via \(\mathbf{T}_\tau\) and log the policy.
  \item \textbf{Exactness bookkeeping (allowed primitives).}
  Reductions to persistence use only:
  (i) PF/BC hold in \(D^b_c\) under (N2);
  (ii) \(\mathbf{P}_i\) reads \((-i)\)-cohomology under (N3);
  (iii) \(\mathbf{T}_\tau\) is exact and \(1\)-Lipschitz (Appendix~A);
  (iv) filtered colimits respect the scope policy (Appendix~A, Remark~\ref{A:rk:filtered-colimits}).
  \item \textbf{Proper/smooth reminder.}
  We use cohomological conventions; any \(f^!\)-induced shifts in smooth variants are absorbed by (N3) and must be logged (degree normalization clause) if they affect comparisons.
  \item \textbf{Window coherence (non-negotiable).}
  PF/BC audits must use the \emph{same} windows and \(\tau\) as B-Gate\(^{+}\) and the \(\delta\)-ledger (Appendix~G); otherwise budget accounting and Restart/Summability pasting (Appendix~J) become invalid.
\end{itemize}

% -------------------------
\subsection{Formalization stubs (Lean/Coq) \textup{[Spec]}}
\label{N:formal}
% -------------------------
A minimal API (cf.\ Appendix~F) includes:
\begin{itemize}\itemsep0.25em
  \item \texttt{pf\_iso}: \(Rf_\ast(A\otimes f^\ast B)\cong Rf_\ast A\otimes B\) under properness and Tor-finiteness;
  \item \texttt{bc\_iso}: \(Lg^\ast Rf_\ast A\cong Rf'_\ast Lg'{}^\ast A\) for Cartesian squares (and smooth variants with \(f^!\) and the shift bookkeeping of (N3));
  \item \texttt{eval\_t}: objectwise evaluation in \(t\);
  \item \texttt{to\_pers}: extraction of \(\mathbf{P}_i\) and truncation by \(\mathbf{T}_\tau\) (canon order);
  \item \texttt{pfbc\_pers\_nat}: naturality in \((i,\tau,f,g,F,G)\) after collapse;
  \item \texttt{budget\_hook}: logging of post-truncation residuals as \(\delta^{\mathrm{disc}},\delta^{\mathrm{meas}}\) and insertion into the (product) ledger aligned with Appendix~G and Appendix~L/K.
\end{itemize}

\medskip
\noindent\textbf{Summary.}
Under standard PF/BC hypotheses over a field \(\Lambda\), projection formula and base change descend—via objectwise evaluation in \(t\), \(\mathbf{P}_i\), and exact truncation \(\mathbf{T}_\tau\)—to canonical, natural isomorphisms at the persistence layer.
All comparisons follow the windowed protocol “for each \(t\) \(\to\) persistence \(\to\) collapse \(\to\) compare,” using the same MECE right-open windows and the same \(\tau\) as the rest of the run.
PF/BC contribute zero algebraic defect; any post-truncation drift is treated as discretization/measurement residual and accounted for in the \(\delta\)-ledger, integrated with Mirror/Transfer commutation (Appendix~L), multi-axis reflectors (Appendix~K), and Restart/Summability pasting (Appendix~J), while keeping the one-way bridge \(\mathrm{PH}_1\Rightarrow \Ext^1\) intact (Appendix~C).



% =========================
\section*{Appendix O. Fukaya Realization \& Stability [Spec + Permitted Ops + $\delta$-Ledger + B-Gate$^{+}$] (Reinforced, canon-aligned)}
% =========================
\phantomsection
\addcontentsline{toc}{section}{Appendix O. Fukaya Realization \& Stability}
\refstepcounter{section}
\label{O:fukaya}

\paragraph{Standing conventions (canon).}
We work over a coefficient \emph{field} \(\Lambda\) (e.g.\ a ground field \(k\) or a Novikov field).
All persistence modules are constructible; all equalities asserted in this appendix are asserted at the persistence layer in the one-parameter constructible setting, and all audits are performed \emph{after collapse} on the B-side single layer.
Filtered (co)limits are computed objectwise in \([\RR,\Vect_\Lambda]\) under the scope policy of Appendix~A, Remark~\ref{A:rk:filtered-colimits}, and then (when stated) returned to the constructible range (by verified constructibility on the window or by applying \(\mathbf{T}_\tau\)).
Distances are measured by the interleaving metric \(d_{\mathrm{int}}\) (=\ bottleneck in the constructible \(1\)D setting).
Truncation \(\mathbf{T}_\tau\) is the exact bar-deletion reflector and is \(1\)\hyp Lipschitz (Appendix~A, Proposition~\ref{A:prop:lipschitz}); chain models are used only up to filtered quasi-isomorphism (f.q.i.) in \(\Ho(\mathsf{FiltCh}(\Lambda))\).
Global conventions: \(\Ext^1\) is always taken against \(\Lambda[0]\); the energy exponent satisfies \(\alpha>0\) (default \(\alpha=1\)).
References to “generic fiber dimension / infinite bars” point to Appendix~D, Remark~\ref{D:rem:generic-dim}.
Monotonicity claims follow the global policy: \emph{deletion-type only} (nonincreasing), inclusion-type merely stable/non-expansive (Appendix~E).
Windows are MECE and right–open; bars are half-open \([b,d)\) (Appendix~G/H). All comparisons follow the fixed order
\[
\boxed{\ \text{for each }t\ \Longrightarrow\ \mathbf{P}_i\ \Longrightarrow\ \mathbf{T}_\tau\ \Longrightarrow\ \text{compare}\ }
\]
after collapse on the B-side single layer (Appendix~G).

\begin{remark}[Right–open windows and half-open bars]\label{O:rmk:windows}
Right–open windows and half-open \([b,d)\) bars enforce MECE coverage and avoid double-counting at endpoints; events at a right boundary are attributed to the subsequent window.
\end{remark}

\begin{definition}[Filtered chain model and persistence]\label{O:def:filt-chain}
A filtered chain complex over \(\Lambda\) means a chain complex \(C_\bullet\) equipped with an exhaustive, increasing filtration \(\{F^{t}C_\bullet\}_{t\in\RR}\) by subcomplexes such that the differential preserves the filtration and continuation data act by filtered maps.
For each homological degree \(i\), the degree–\(i\) persistence module is \(t\mapsto H_i(F^{t}C_\bullet)\in \Vect_\Lambda\).
Constructibility on bounded windows means only finitely many jumps (break times) occur and ranks are finite on bounded intervals.
\end{definition}

\begin{definition}[Deletion-type operation at the persistence layer]\label{O:def:deletion}
A morphism \(M\to N\) in \(\Pers_\Lambda\) is \emph{deletion-type} on a window \(W\subset\RR\) if, after restricting to \(W\) and post-composing with \(\mathbf{T}_\tau\) (for any \(\tau\ge 0\)), the induced effect can only shorten or remove existing bars and cannot create new bars on \(W\).
Equivalently, all deletion-type monotone indicators of Appendix~E are nonincreasing under this operation (and inclusion-type indicators are not claimed monotone).
\end{definition}

% -------------------------
\subsection*{O.1. Realization functor and hypotheses}
% -------------------------
\emph{[Spec]} Fix a Liouville/Weinstein sector \((X,\lambda)\) with a (possibly empty) system of \emph{stops}.
Write \(\mathsf{Fuk}(X;\mathrm{stops})\) for a wrapped/exact/monotone Fukaya-type category for which Floer-theoretic chain models admit an \emph{action filtration}.
We normalize the filtration parameter as follows: \emph{the action value is the filtration parameter \(t\)}, increasing with larger action (so sublevel sets \(F^{t}\) mean action \(\le t\)).
Package the chain-level construction into a realization
\[
\mathcal{F}:\ \text{(geometric input)}\longrightarrow \mathsf{FiltCh}(\Lambda),
\]
natural in continuation data and stop operations, with degree–\(i\) persistence
\(\mathbf{P}_i(\mathcal{F}(-))\in \Pers^{\mathrm{cons}}_\Lambda\) (constructible on bounded windows).
We assume:

\begin{itemize}\setlength{\itemsep}{0.35em}
  \item[(O0)] \textbf{Coefficients/admissibility.}
  \(\Lambda\) is a field; in the monotone/exact regimes with admissible almost complex structures, the action and index filtrations are well defined; continuation solutions have finite energy.
  \item[(O1)] \textbf{Constructibility on bounded action windows.}
  On every bounded action window \([0,\sigma]\) the Floer complexes have finitely many generators and finitely many break times; hence \(\mathbf{P}_i(\mathcal{F}(-))\) is constructible on \([0,\sigma]\).
  The same local finiteness holds with Novikov coefficients on bounded windows.
  \item[(O2)] \textbf{Continuation shift bound.}
  Any continuation map for a homotopy of data with controlled action shift \(\varepsilon\) induces a filtered chain map whose filtration increase is \(\le \varepsilon\).
  \item[(O3)] \textbf{Stop operations are deletion-type (post-collapse).}
  Adding a stop or shrinking a sector removes generators and/or increases differentials in a way that corresponds to a deletion-type operation at the persistence layer: on any fixed action window, no new bars are created \emph{after applying \(\mathbf{T}_\tau\)}.
  \item[(O4)] \textbf{Up to filtered quasi-isomorphism.}
  Chain models are considered up to f.q.i.; all claims are invariant under f.q.i.\ and are asserted after passing to persistence and truncating by \(\mathbf{T}_\tau\).
\end{itemize}

\begin{remark}[Action filtration normalization]\label{O:rmk:action}
Our choice “action value equals filtration parameter” fixes the direction of filtration.
Monotone time reparametrizations that preserve order act by reindexings and, after normalization, give isometries in \(d_{\mathrm{int}}\) at the persistence layer.
\end{remark}

% -------------------------
\subsection*{O.2. Stability: continuation and stops}
% -------------------------
\begin{theorem}[Continuation control implies interleaving bound]\label{O:thm:cont}
Under \textup{(O2)}, for any two realizations related by a continuation with action shift \(\varepsilon\),
\[
d_{\mathrm{int}}\!\Big(\mathbf{P}_i(\mathcal{F}_0),\ \mathbf{P}_i(\mathcal{F}_1)\Big)\ \le\ \varepsilon,\qquad
d_{\mathrm{int}}\!\Big(\mathbf{T}_\tau\mathbf{P}_i(\mathcal{F}_0),\ \mathbf{T}_\tau\mathbf{P}_i(\mathcal{F}_1)\Big)\ \le\ \varepsilon
\]
for all \(i\) and all \(\tau\ge 0\).
\end{theorem}

\begin{proposition}[Deletion-type monotonicity for stops (post-collapse)]\label{O:prop:stops}
Under \textup{(O3)}, adding a stop or shrinking a sector induces, on any window and after \(\mathbf{T}_\tau\), a deletion-type morphism:
for every \(i\) and \(\tau\ge 0\),
\[
\mathbf{T}_\tau\,\mathbf{P}_i\big(\mathcal{F}_{\mathrm{with\ stop}}\big)\ \preceq\ \mathbf{T}_\tau\,\mathbf{P}_i\big(\mathcal{F}_{\mathrm{without\ stop}}\big),
\]
and all deletion-type monotone indicators of Appendix~E are nonincreasing under this operation.
\end{proposition}

\begin{theorem}[Stop/continuation policy — canon-aligned]\label{O:thm:fukaya-policy}
In the action-filtered Fukaya realization, stop addition is deletion-type (post-collapse nonincrease) and \(\varepsilon\)-continuation maps satisfy \(d_{\mathrm{int}}\le \varepsilon\) on \(\mathbf{T}_\tau\)-collapsed persistence.
In the \(\delta\)-ledger: continuation/shift steps record \(\delta^{\mathrm{alg}}=\varepsilon\); deletion-type stop steps record \(\delta^{\mathrm{alg}}=0\).
\end{theorem}

\begin{declaration}[Gate Cascade placement (after-collapse, B-side single layer)]\label{O:dec:gate-cascade}
Fukaya realizations enter the Gate Cascade at the \emph{B-side} after applying \(\mathbf{P}_i\) and truncating by \(\mathbf{T}_\tau\), and before any Mirror/Tropical post-processing (Appendix~L) or multi-axis reflector interactions (Appendix~K).
All audits and budgets for Fukaya steps therefore use the order
\[
\boxed{\ \text{for each } t\ \Longrightarrow\ \mathbf{P}_i\ \Longrightarrow\ \mathbf{T}_\tau\ \Longrightarrow\ \text{Fukaya-op compare}\ },
\]
with the same window and \(\tau\) as B-Gate\(^{+}\) (Appendix~J/G).
\end{declaration}

\begin{remark}[Budget-adjusted continuation radius]\label{O:rmk:eps-eff}
On a window \(W\) with additive \(\delta\)-ledger (Appendix~L/K), use the effective radius
\(\varepsilon_{\mathrm{eff}}:=\varepsilon+\Delta_W\) (Appendix~I) when invoking survival or matching claims.
Here \(\Delta_W\) aggregates only \emph{non-Fukaya} defects (e.g.\ Mirror/Transfer commutation bounds, A/B residuals, discretization/measurement slack).
\end{remark}

% -------------------------
\subsection*{O.3. Towers, comparison map, and diagnostics}
% -------------------------
Let \(F=(F_n)_{n\in I}\) be a directed system of geometric inputs (e.g.\ refining Hamiltonians/perturbations or nested stop systems) with colimit \(F_\infty\).
Apply \(\mathcal{F}\) and then \(\mathbf{P}_i\) to obtain a tower in \(\Pers^{\mathrm{cons}}_\Lambda\).
For \(\tau\ge 0\) consider the comparison map (Appendix~J), always formed \emph{after collapse}:
\[
\phi_{i,\tau}(F):\quad \varinjlim_n\ \mathbf{T}_\tau\!\big(\mathbf{P}_i(\mathcal{F}(F_n))\big)\ \longrightarrow\ \mathbf{T}_\tau\!\big(\mathbf{P}_i(\mathcal{F}(F_\infty))\big).
\]

\begin{definition}[Sufficient tower hypotheses (canon references)]\label{O:def:tower-conds}
Assume any one of the sufficient regimes from Appendix~D, \S~D.3 (e.g.):
\begin{enumerate}\itemsep0.2em
  \item[(S1)] truncation/colimit commutation up to vanishing error under the scope policy, after applying \(\mathbf{T}_\tau\);
  \item[(S2)] no \(\tau\)-accumulation of break times on the window (Type~IV exclusion at the chosen \(\tau\));
  \item[(S3)] a Cauchy condition in \(d_{\mathrm{int}}\) with compatible structure maps (post-collapse).
\end{enumerate}
Each regime ensures stability of barcodes under the colimit and truncation at the persistence layer.
\end{definition}

\begin{theorem}[When the comparison is an isomorphism]\label{O:thm:phi-iso}
Assume a continuation-controlled tower: there exist bounds \(\varepsilon_n\to 0\) such that
\[
d_{\mathrm{int}}\!\big(\mathbf{P}_i(\mathcal{F}(F_n)),\ \mathbf{P}_i(\mathcal{F}(F_\infty))\big)\ \le\ \varepsilon_n,
\]
and assume any of \textup{(S1)/(S2)/(S3)} from Definition~\ref{O:def:tower-conds}.
Then \(\phi_{i,\tau}(F)\) is an isomorphism for all \(\tau\ge 0\).
Consequently the tower diagnostics vanish:
\[
\mu_{i,\tau}(F)=\nu_{i,\tau}(F)=0,
\]
where \((\mu,\nu)\) are the generic fiber dimensions of \(\ker(\phi_{i,\tau})\) and \(\coker(\phi_{i,\tau})\) (Appendix~D, Remark~\ref{D:rem:generic-dim}).
\end{theorem}

\begin{corollary}[Grid \(\Rightarrow\) continuum survival (budget-aware)]\label{O:cor:g2c}
In discretization towers (mesh \(h\to 0\)) with certified continuation bounds \(\varepsilon(h)\to 0\), any bar detected in a fixed window \([0,\tau_0]\) whose \(\mathbf{T}_{\tau_0}\)-clipped length exceeds \(2\,\varepsilon_{\mathrm{eff}}(h)\) survives in the limit, where \(\varepsilon_{\mathrm{eff}}(h)=\varepsilon(h)+\Delta_W\) (Appendix~I).
\end{corollary}

% -------------------------
\subsection*{O.4. Permitted-operations table (windowed, post-collapse) and $\delta$-ledger}
% -------------------------
All comparisons follow the protocol “for each \(t\) \(\to\) persistence \(\mathbf{P}_i\) \(\to\) collapse \(\mathbf{T}_\tau\) \(\to\) compare,” on MECE, right–open windows and a fixed \(\tau\) (Appendix~G/N).
The table summarizes permitted operations, their type, quantitative contracts (post-collapse), and how to record them in the \(\delta\)-ledger.

\begin{center}
\footnotesize
\setlength{\tabcolsep}{4pt}
\renewcommand{\arraystretch}{1.1}
\begin{tabularx}{\linewidth}{@{}p{.28\linewidth} p{.16\linewidth} X @{}}
\toprule
Operation & Type & Quantitative contract after $\mathbf{T}_\tau$ and ledger entry \\
\midrule
Add stop / shrink sector & Deletion &
Deletion-type: no new bars on the window after $\mathbf{T}_\tau$; all deletion indicators nonincreasing (Appendix~E).
Ledger: $\delta^{\mathrm{alg}}=0$; record $\delta^{\mathrm{disc}},\delta^{\mathrm{meas}}$ if any. \\
Continuation (tame homotopy) & Shift &
$d_{\mathrm{int}}\le \varepsilon$ (Theorem~\ref{O:thm:cont}).
Ledger: $\delta^{\mathrm{alg}}=\varepsilon$. \\
Hamiltonian change (bounded drift) & Shift &
Reduce to continuation control: $d_{\mathrm{int}}\le \varepsilon$ post-collapse.
Ledger: $\delta^{\mathrm{alg}}=\varepsilon$. \\
Almost complex structure change (tame) & Shift &
Reduce to continuation control: $d_{\mathrm{int}}\le \varepsilon$ post-collapse.
Ledger: $\delta^{\mathrm{alg}}=\varepsilon$. \\
Regrading / Maslov shift & Bookkeeping &
Degree reindexing isometric after normalization (no metric cost).
Ledger: $\delta^{\mathrm{alg}}=0$. \\
Monotone time reparametrization & Reindex &
Isometry after reindex normalization (Appendix~G policy).
Ledger: $\delta^{\mathrm{alg}}=0$. \\
Mirror/Tropical/Transfer post-processing & External &
Accounted by Appendix~L: if commutation defect $\delta(i,\tau)$ is used, ledger it in the appropriate channel (product ledger) and collapse once by $\pi$. \\
Non-nested reflectors (if used) & External &
A/B test or soft-commuting fallback (Appendix~K/L);
ledger: $\delta^{\mathrm{alg}}=\Delta_{\mathrm{comm}}$ when fallback residual is incurred. \\
\bottomrule
\end{tabularx}
\end{center}

\begin{definition}[$\delta$–ledger and budgets (windowed, post-collapse)]\label{O:def:ledger}
Per window \(W=[u,u')\) and degree \(i\), define the aggregate budget as a commutative quantale sum (Appendix~K/L):
\[
\Sigma\delta_W(i)\ :=\ \sum_{\text{continuations/shifts}}\varepsilon\ +\ \sum_{\text{Mirror/Transfer}}\delta(i,\tau)\ +\ \sum_{\text{A/B fails}}\Delta_{\mathrm{comm}}\ +\ \sum_{\text{audits}}(\delta^{\mathrm{disc}}+\delta^{\mathrm{meas}}),
\]
where “audits” include PF/BC checks (Appendix~N), numerical tolerances, and any scope-policy return-to-constructible enforcement costs (Appendix~A/G).
All terms are measured/recorded \emph{after} applying \(\mathbf{T}_\tau\) and on the same MECE right-open windowing.
\end{definition}

\begin{remark}[Product-ledger compatibility]\label{O:rmk:product-ledger}
If the run uses a product quantale ledger (Appendix~L, \S L.3), record Fukaya continuation terms in the dedicated channel (e.g.\ \texttt{fukaya\_shift}) and collapse once via \(\pi\).
This prevents double counting across external channels (mirror/transfer/A/B/audits).
\end{remark}

% -------------------------
\subsection*{O.5. B-Gate$^{+}$, restart, and summability (window pasting)}
% -------------------------
We adopt B-Gate\(^{+}\) with a per-window safety margin \(\mathrm{gap}_\tau(i)>0\) computed after \(\mathbf{T}_\tau\).
On window \(W\) and degree \(i\), the gate \emph{passes} if
\[
\mathrm{gap}_\tau(i)\ >\ \Sigma\delta_W(i).
\]
Across consecutive windows \((W_k)_k\), assume:
(i) transitions are finite compositions of deletion-type steps and \(\varepsilon\)-continuations measured post-collapse, and
(ii) Summability holds \(\sum_k \Sigma\delta_{W_k}(i)<\infty\) (Appendix~J).
Then the Restart inequality of Appendix~J yields for some \(\kappa\in(0,1]\),
\[
\mathrm{gap}_{\tau_{k+1}}(i)\ \ge\ \kappa\Big(\mathrm{gap}_{\tau_k}(i)-\Sigma\delta_{W_k}(i)\Big),
\]
so positivity of the margin persists along the pipeline.
Per-window certificates paste to a global certificate on \(\bigcup_k W_k\) (Appendix~J, Theorem~J:\ref{J:thm:pasting}).

\begin{remark}[Choice of \(\kappa\)]\label{O:rmk:kappa}
The constant \(\kappa\) accounts for uniform losses at window transitions (e.g.\ finite alignment overheads or reindex coercions).
Under exact commutation and perfectly aligned window policies one may take \(\kappa=1\).
\end{remark}

% -------------------------
\subsection*{O.6. Windowed usage and run.yaml alignment}
% -------------------------
Record in \texttt{run.yaml} per window and degree:
\begin{itemize}\itemsep0.25em
  \item the operation sequence (stops/sector changes, continuations) with quantitative parameters (\(\varepsilon\), thresholds \(\tau\), sweep settings);
  \item the \(\delta\)–ledger entries (product vector if used) and the collapsed scalar \(\Sigma\delta_W(i)\);
  \item the B-Gate\(^{+}\) safety margin \(\mathrm{gap}_\tau(i)\) and pass/fail verdict;
  \item any external steps (Mirror/Tropical/Transfer, reflectors) with A/B policy data (\(\eta\), \(\Delta_{\mathrm{comm}}\)) (Appendix~K/L);
  \item constructibility checks: bounds on generators and event counts on the window (Appendix~H) and any enforced return-to-constructible via \(\mathbf{T}_\tau\) (Appendix~A).
\end{itemize}
All diagnostics \((\mu,\nu)\) and comparison maps \(\phi_{i,\tau}\) are computed \emph{after} truncation \(\mathbf{T}_\tau\) and logged with the same window and \(\tau\).

% -------------------------
\subsection*{O.7. Failure modes and audit checklist}
% -------------------------
\noindent\emph{Failure modes (outside our scope).}
\begin{itemize}\itemsep0.25em
  \item \textbf{Loss of filtration control.}
  Non-admissible data or bubbling may invalidate (O2); quantitative continuation bounds then fail and must not be asserted.
  \item \textbf{Near-threshold accumulation (Type IV).}
  Accumulation of bar lengths near \(\tau\) can break stabilization and invalidate tower claims without (S2)/(S3) (Appendix~D).
  \item \textbf{Inclusion-type operations.}
  Removing stops or enlarging sectors can create new features; monotonicity is not claimed. Only stability via continuation control may be used, and only post-collapse.
\end{itemize}

\noindent\emph{Audit checklist (runtime verifications).}
\begin{enumerate}\itemsep0.25em
  \item Record continuation shift bounds \(\varepsilon\) and certify \(d_{\mathrm{int}}\le\varepsilon\) post-collapse (Theorem~\ref{O:thm:cont}); ledger \(\delta^{\mathrm{alg}}=\varepsilon\).
  \item Verify constructibility on each window (finite generators/events; Appendix~H) and log per-window counts (Appendix~G).
  \item For stop additions/sector shrinkage, mark operation as deletion-type and evaluate Appendix~E deletion indicators after \(\mathbf{T}_\tau\).
  \item For towers, log \(\varepsilon_n\), verify one of (S1)/(S2)/(S3), compute \((\mu,\nu)\); \(\phi_{i,\tau}\) iso \(\Rightarrow \DiagZero\) (Appendix~D/J).
  \item If external functors are used, run Mirror/Transfer commutation audits and A/B tests (Appendix~L/K) and incorporate residuals into \(\Sigma\delta_W(i)\) using the declared ledger law.
\end{enumerate}

% -------------------------
\subsection*{O.8. Formalization stubs (Lean/Coq) [Spec]}
% -------------------------
A minimal API (cf.\ Appendix~F) includes:
\begin{itemize}\itemsep0.25em
  \item \texttt{fukaya\_realize}: returns an action-filtered chain model \(\mathcal{F}\) up to f.q.i., with constructibility on bounded windows (O1).
  \item \texttt{cont\_eps}: continuation maps with filtration increase \(\le\varepsilon\) implying \(d_{\mathrm{int}}\le\varepsilon\) after truncation (Theorem~\ref{O:thm:cont}).
  \item \texttt{stop\_delete}: deletion-type morphisms for stop addition/sector shrink post-collapse (Proposition~\ref{O:prop:stops}).
  \item \texttt{tower\_phi\_iso}: sufficient criteria ensuring \(\phi_{i,\tau}\) is an isomorphism and \(\DiagZero\) (Appendix~D/J).
  \item Budget hooks: \texttt{ledger\_add}, \texttt{gate\_plus}, and restart/summability contracts aligned with Appendix~J and Appendix~L/K product-ledger semantics.
\end{itemize}

% -------------------------
\subsection*{O.9. Examples and regression tests}
% -------------------------
\begin{example}[Exact wrapped setting with a new stop]\label{O:ex:stop}
Let \(X=T^\ast Q\) with its standard exact form and consider a wrapped setting with a stop at infinity.
Adding an additional stop supported on a Legendrian subset removes Reeb chords crossing the stop.
On any fixed action window, no new generators appear and differentials can only increase; hence the induced operation is deletion-type post-collapse (Proposition~\ref{O:prop:stops}).
Regression test: after \(\mathbf{T}_\tau\), deletion indicators (Appendix~E) weakly decrease and no new bars appear on the window.
\end{example}

\begin{example}[Continuation bound from Hamiltonian drift]\label{O:ex:ham}
Suppose \(H_0,H_1\) are cofinal Hamiltonians with \(\sup(H_1-H_0)\le \varepsilon\) on the relevant support and homotoped by a tame path.
The action change along continuation solutions is bounded by \(\varepsilon\); thus \(d_{\mathrm{int}}\le \varepsilon\) post-collapse (Theorem~\ref{O:thm:cont}).
Regression test: barcode bottleneck distance between the two collapsed persistence outputs never exceeds the certified \(\varepsilon\).
\end{example}

\begin{example}[Grid-to-continuum]\label{O:ex:g2c}
Discretize a time-dependent Floer datum with mesh \(h\) and certified continuation bound \(\varepsilon(h)\to 0\).
For any fixed \(\tau_0\), bars of \(\mathbf{T}_{\tau_0}\)-collapsed persistence with clipped length \(>2\,\varepsilon_{\mathrm{eff}}(h)\) survive in the limit (Corollary~\ref{O:cor:g2c}).
Regression test: survival rates converge after accounting for the per-window \(\Sigma\delta_W(i)\) and the B-Gate\(^{+}\) margin condition.
\end{example}

% -------------------------
\subsection*{O.10. Summary}
% -------------------------
Floer-theoretic realizations with action filtration yield constructible persistence on bounded windows (O1).
Continuation with shift \(\varepsilon\) implies \(d_{\mathrm{int}}\le \varepsilon\) at the persistence layer and remains valid after truncation (Theorem~\ref{O:thm:cont}).
Adding stops or shrinking sectors is deletion-type and hence nonincreasing for all deletion indicators after collapse (Proposition~\ref{O:prop:stops}).
\emph{Placement:} Fukaya steps are audited \emph{after collapse} on the B-side single layer (Declaration~\ref{O:dec:gate-cascade}); the same window/\(\tau\) are used for B-Gate\(^{+}\), Mirror/Transfer, and PF/BC checks.
A windowed permitted-ops table prescribes how to assign \(\delta\)-ledger entries and how to integrate external commutation/A-B residuals without double counting (Appendix~L/K).
B-Gate\(^{+}\) requires \(\mathrm{gap}_\tau>\Sigma\delta\) per window and pastes via Restart/Summability (Appendix~J).
Under standard tower hypotheses (Appendix~D), the comparison map \(\phi_{i,\tau}\) is an isomorphism and \(\DiagZero\) (Theorem~\ref{O:thm:phi-iso}); budget-aware grid-to-continuum survival follows (Appendix~I).
All items respect MECE right–open windows, half-open bars, and the canon order “\(t\to\mathbf{P}_i\to\mathbf{T}_\tau\to\)compare,” and integrate with the reproducible \texttt{run.yaml} workflow (Appendix~G).



% =========================
\section*{Appendix P. Tropical–LMHS Dictionary [Spec; after-collapse indicators only] (canon-aligned)}
% =========================
\phantomsection
\addcontentsline{toc}{section}{Appendix P. Tropical–LMHS Dictionary}
\refstepcounter{section}
\label{P:trop-lmhs}

\paragraph{Standing conventions (canon).}
We work over a coefficient field \(k\).
All persistence modules are constructible on bounded windows; filtered (co)limits are computed objectwise in \([\RR,\Vect_k]\) under the scope policy of Appendix~A and returned to the constructible range when stated (by verified constructibility on the window or by applying the exact bar-deletion reflector \(\mathbf{T}_\tau\)).
\emph{All numeric comparisons in this appendix are evaluated after collapse} in the fixed order
\[
\boxed{\ \text{for each }t\ \Longrightarrow\ \mathbf{P}_i\ \Longrightarrow\ \mathbf{T}_\tau\ \Longrightarrow\ \text{compare on } \mathbf{T}_\tau\mathbf{P}_i\ },
\]
with the same MECE, right-open windows and the same \(\tau\) as elsewhere (Appendix~G/N).
Distances/defects aggregate in a declared commutative \emph{quantale} \(V\) with sum \(\oplus\), order \(\preceq\), and scalar action \(\odot\) (Appendix~K/L/S). The interleaving metric \(d_{\mathrm{int}}\) (=\ bottleneck in the constructible \(1\)D setting) is used on \(\Pers^{\mathrm{cons}}_k\), and \(\mathbf{T}_\tau\) is exact and \(1\)-Lipschitz (Appendix~A).
Any chain-level operation \(C_\tau\) is used only up to f.q.i., and only through its induced persistence after applying \(\mathbf{P}_i\) and \(\mathbf{T}_\tau\) (Appendix~B).
The one-way bridge \(\mathrm{PH}_1\Rightarrow \Ext^1\) is used only in \(D^{\mathrm{b}}(k\text{-mod})\) and only forward (Chapter~3; Appendix~C). No converse and no new global equivalences are asserted.

% -------------------------
\subsection*{P.1. Dictionary contract (Spec) and safe use}
% -------------------------

\begin{declaration}[Tropical--LMHS dictionary (Spec; advisory, after-collapse only)]\label{P:dec:dict}
On a definable window \(W\) (Appendix~H/J), a lookup procedure maps tropical inputs
\((\mathrm{val},\mathrm{trop}_\lambda,\text{fan data})\) to coarse LMHS proxies \((W_\bullet,N,h^{p,q}_\infty)\).
These proxies are used \emph{only} to propose \emph{after-collapse} indicators on \(\mathbf{T}_\tau\mathbf{P}_i(F|_W)\) for selected degrees \(i\).
The dictionary is \emph{never} a Gate and \emph{never} a certificate: it may \emph{trigger} certified tests (Appendix~J/N/L), but cannot certify them.
\end{declaration}

\begin{remark}[Safe-use policy]\label{P:rmk:safe}
\begin{enumerate}\itemsep0.2em
  \item \textbf{Advisory only.} Dictionary proposals are advisory and may contribute at most an \emph{advisory} budget entry \(\delta^{\mathrm{spec}}\in V\) (Definition~\ref{P:def:conf}). If no evidence supports calibration, set \(\omega(W)=0\) so \(\delta^{\mathrm{spec}}=0\).
  \item \textbf{Certified decisions.} Any decision impacting PF/BC, Overlap Glue, tower diagnostics \((\mu,\nu)\), or classification must rely only on certified after-collapse statements (Appendix~N/J/L/D) and the declared window protocol (Appendix~G).
  \item \textbf{Finite verification cost.} All claims are windowed and reproducible; on bounded definable windows, event counts are finite and \v{C}ech depth is finite (Appendix~H/J), hence all required checks terminate.
\end{enumerate}
\end{remark}

% -------------------------
\subsection*{P.2. Lookup targets and after-collapse indicators}
% -------------------------

\begin{center}
\small
\setlength{\tabcolsep}{7pt}
\renewcommand{\arraystretch}{1.08}
\begin{tabularx}{\linewidth}{@{}l l Y@{}}
\toprule
Tropical input & LMHS proxy (coarse) & After-collapse indicator (advisory; non-gating) \\
\midrule
\(\mathrm{val}(\text{moduli})\) step loci
& weight jumps in \(W_\bullet\)
& candidate deletion change-points (stop/shrink \emph{hints}) to be verified as deletion-type post-collapse \\
\(\mathrm{trop}_\lambda\) slopes (\(\lambda\to 0^+\))
& \(\operatorname{rk} N=\operatorname{rk}\log T_u\)
& suggested local bins / expected smallness of windowed \(E_1\) (trigger certified tests) \\
Balanced-fan combinatorics
& split vs.\ non-split type
& additivity expectations for tower diagnostics \((\mu,\nu)\) across overlaps (to be confirmed by Overlap Glue) \\
Tropical periods/lengths
& limiting \(h^{p,q}_\infty\) pattern
& energy-bin prioritization; \(\varepsilon\)-survival \emph{hints} subject to budget-adjusted radii (Appendix~I/L) \\
\bottomrule
\end{tabularx}
\end{center}


\begin{remark}[Definable coverage]\label{P:rk:def}
Windows \(W\) are finite unions of half-open intervals definable in a fixed o-minimal structure; hence finite event decomposition and finite \v{C}ech depth hold, yielding finite verification cost for all local tests and overlap checks (Appendix~H/J).
\end{remark}

% -------------------------
\subsection*{P.3. Quantitative commutation and 2-cell bounds (after collapse)}
% -------------------------

\begin{declaration}[Tropical 2-cell bound (after-collapse)]\label{P:dec:2cell}
Let \(\mathrm{trop}_\lambda\) be a Mirror/Tropical post-processing step that is \(1\)-Lipschitz at the persistence layer and admits a controlled \(2\)-cell with collapse in the sense of Appendix~L.
On a window \(W\) and for degree \(i\), record a bound \(\delta_{\mathrm{trop}}(i,\tau;W)\in V\) such that
\[
d_V\!\Big(\mathbf{T}_\tau\mathbf{P}_i\big(\mathrm{trop}_\lambda\, C_\tau F\big)\,,\ \mathbf{T}_\tau\mathbf{P}_i\big(C_\tau\, \mathrm{trop}_\lambda F\big)\Big)\ \preceq\ \delta_{\mathrm{trop}}(i,\tau;W).
\]
Enter \(\delta_{\mathrm{trop}}(i,\tau;W)\) into the window budget (Definition~\ref{P:def:budget}) using the declared quantale sum \(\oplus\).
\end{declaration}

\begin{remark}[non-expansive measurement]\label{P:rk:nonexp}
All discrepancies are measured after \(\mathbf{T}_\tau\). Therefore any contraction in \(\mathrm{trop}_\lambda\) is preserved, and subsequent \(1\)-Lipschitz persistence post-processing cannot increase recorded bounds (Appendix~L).
\end{remark}

\begin{remark}[Product-ledger compatibility]\label{P:rk:product}
If the run uses a product quantale ledger (Appendix~L, \S L.3), record \(\delta_{\mathrm{trop}}(i,\tau;W)\) in the dedicated channel (e.g.\ \texttt{tropical\_2cell}) and collapse once via \(\pi\). This avoids double counting across channels.
\end{remark}

% -------------------------
\subsection*{P.4. Confidence-weighted advisory defect \(\delta^{\mathrm{spec}}\)}
% -------------------------

\begin{definition}[Confidence weight and advisory defect]\label{P:def:conf}
Each dictionary proposal on \(W\) carries a confidence weight \(\omega(W)\in[0,1]\) (dimensionless).
Given a raw advisory magnitude \(\widehat{\delta}_{\mathrm{spec}}(i,\tau;W)\in V\) (e.g.\ from suggested bin gaps, slope tolerances, or disagreement between proxy and measured after-collapse indicators), define the ledgered advisory term
\[
\delta^{\mathrm{spec}}(i,\tau;W)\ :=\ \omega(W)\ \odot\ \widehat{\delta}_{\mathrm{spec}}(i,\tau;W)\ \in V.
\]
Calibration of \(\omega(W)\) may use held-out windows, stability-band cross-checks, or \(\tau\)-sweeps (Appendix~J/M). Absent evidence, set \(\omega(W)=0\) (no advisory cost).
\end{definition}

\begin{remark}[Summability on countable covers]\label{P:rmk:summability}
For a locally finite countable definable cover \(\{W_j\}\) of a bounded window, \(\bigoplus_j \delta^{\mathrm{spec}}(i,\tau;W_j)\) is well-defined whenever the corresponding series converges in \(V\).
This is the \emph{T-Delta-Sum-Converges} condition, compatible with Restart/Summability (Appendix~J).
\end{remark}

% -------------------------
\subsection*{P.5. Local trigger via \(E_1\) (after-collapse)}
% -------------------------

\paragraph{Windowed \(E_1\) indicator (after collapse).}
Fix a window \(W=[u,u')\). For a degree \(i\), define the windowed \(E_1\) indicator on \(W\) by
\[
E_1^{(i)}(F;W,\tau)\ :=\ \int_{u}^{u'} \dim_k\Big(\mathbf{T}_\tau\mathbf{P}_i(F)(t)\Big)\,dt,
\]
i.e.\ the clipped Betti integral of the post-collapse persistence on \(W\) (cf.\ Appendix~M, \(\|\cdot\|_{E_1}\)).
By constructibility, this integral is a finite sum on any bounded window.

\begin{declaration}[Advisory trigger (non-gating)]\label{P:dec:trigger}
If the dictionary suggests on \(W\) that the monodromy-rank proxy vanishes (no slope/facet change), it may trigger the \emph{certified} test \(E_1^{(1)}(F;W,\tau)=0\) computed on \(\mathbf{T}_\tau\mathbf{P}_1(F|_W)\).
When verified, one may use the established after-collapse chain of implications in the approved direction:
\[
E_1^{(1)}(F;W,\tau)=0\ \Longrightarrow\ \mathbf{T}_\tau\mathbf{P}_1(F|_W)=0\ \Longrightarrow\ \mathrm{PH}_1(\mathbf{T}_\tau F|_W)=0\ \Longrightarrow\ \Ext^1(\mathcal{R}(\mathbf{T}_\tau F)|_W,k)=0,
\]
where the last implication is the one-way bridge used in \(D^b(k\text{-mod})\) (Appendix~C) and the realization \(\mathcal{R}\) is only invoked within its stated hypotheses.
Failure of the certified test records a nonzero raw advisory magnitude \(\widehat{\delta}_{\mathrm{spec}}\) (e.g.\ the smallest positive \(E_1\)-bin on the window) and hence a ledgered \(\delta^{\mathrm{spec}}\) via Definition~\ref{P:def:conf}.
\end{declaration}

\begin{remark}[Overlap Glue with finite depth]\label{P:rk:glue}
On a definable cover of a bounded window, Overlap Glue terminates after finitely many checks (Appendix~J). Any residuals are aggregated into the budget by \(\oplus\) and never override certified pass/fail decisions.
\end{remark}

% -------------------------
\subsection*{P.6. Window budget and pasting}
% -------------------------

\begin{definition}[Window budget with advisory terms]\label{P:def:budget}
For window \(W\) and degree \(i\), define the post-collapse window budget as the quantale sum
\[
\Sigma\delta_W(i)\ :=\ 
\Big(\bigoplus_{\text{Mirror/Tropical 2-cells}}\delta_{\mathrm{trop}}(i,\tau;W)\Big)
\ \oplus\
\Big(\bigoplus_{\text{A/B fails}}\Delta_{\mathrm{comm}}(i;W)\Big)
\ \oplus\
\Big(\bigoplus_{\text{audits}}(\delta^{\mathrm{disc}}+\delta^{\mathrm{meas}})\Big)
\ \oplus\
\Big(\bigoplus_{\text{advisory}}\delta^{\mathrm{spec}}(i,\tau;W)\Big),
\]
with all terms measured and recorded after \(\mathbf{T}_\tau\) and under the same MECE right-open windowing (Appendix~G/L/K/N).
B-Gate\(^{+}\) requires \(\mathrm{gap}_\tau(i)>\Sigma\delta_W(i)\) per window (Appendix~J).
\end{definition}

\begin{remark}[Countable covers and pasting]\label{P:rmk:countable}
If a locally finite countable definable cover is used, Summability plus \(\bigoplus_W \delta^{\mathrm{spec}}(i,\tau;W)\) convergent ensures pasting across windows (Appendix~J). Advisory terms are designed to be optionally zeroed (\(\omega=0\)) to preserve summability when needed.
\end{remark}

% -------------------------
\subsection*{P.7. Minimal working example (non-gating)}
% -------------------------

\noindent\emph{Single-facet advisory pass.}
On \(W\), tropical slopes are constant; the dictionary proposes \(N=0\Rightarrow E_1^{(1)}(F;W,\tau)=0\) with \(\omega(W)=0.6\).
The certified test returns \(E_1^{(1)}(F;W,\tau)=0\) \emph{true}; no advisory ledger entry is added.
If the test fails with smallest nonzero \(E_1\)-bin \(\widehat{\delta}_{\mathrm{spec}}=\varepsilon_\star\in V\), then \(\delta^{\mathrm{spec}}=0.6\odot \varepsilon_\star\) is added to \(\Sigma\delta_W(1)\).

% -------------------------
\subsection*{P.8. Reproducibility hooks (run.yaml)}
% -------------------------

\begin{center}
\small
\renewcommand{\arraystretch}{1.05}
\setlength{\tabcolsep}{7pt}
\begin{tabularx}{\linewidth}{@{}l Y@{}}
\toprule
Key & Meaning \\
\midrule
\texttt{policy:\{after\_collapse\_only\}} & Must be \texttt{true} for this appendix’s comparisons. \\
\texttt{quantale:\{name,op,unit,order,scalar\}} & Quantale \(V\), aggregation \(\oplus\), unit \(0_V\), order \(\preceq\), scalar action \(\odot\) (Appendix~K/L/S). \\
\texttt{windows:\{mece,right\_open,intervals\}} & MECE right-open window considerations and explicit intervals (Appendix~G). \\
\texttt{definable:\{structure,window\_formulae\}} & o-minimal structure and window formulas (Appendix~H/J). \\
\texttt{tropical:\{lambda,contraction\_kappa,bins\}} & Mirror/Tropical configuration and binning policy (Appendix~L/M). \\
\texttt{lmhs:\{proxies\}} & Enabled proxies: \texttt{rankN}, \texttt{weights}, \texttt{hpq\_inf}. \\
\texttt{two\_cell:\{delta\_trop\}} & Recorded \(\delta_{\mathrm{trop}}(i,\tau;W)\) bounds per window/degree (Appendix~L). \\
\texttt{dict:\{omega,raw\_spec,delta\_spec\}} & Confidence \(\omega(W)\), raw \(\widehat{\delta}_{\mathrm{spec}}\), and ledgered \(\delta^{\mathrm{spec}}\). \\
\texttt{budget:\{sum\_delta,channels\}} & Per-window budget aggregation and (optional) product-ledger channels (Appendix~L/K). \\
\bottomrule
\end{tabularx}
\end{center}


% -------------------------
\subsection*{P.9. Minimal API (pseudocode) [Spec]}
% -------------------------

\begin{verbatim}
def advisory_delta(window, degree, tau, raw_spec, omega):
    # raw_spec in V; omega in [0,1]; uses quantale scalar action ⊙
    return omega ⊙ raw_spec

def window_budget(entries):
    # entries: list of V-elements from tropical 2-cells, A/B residuals,
    # audits (disc/meas), and advisory deltas; aggregate by ⊕.
    Sigma = 0_V
    for v in entries:
        Sigma = Sigma ⊕ v
    return Sigma
\end{verbatim}

% -------------------------
\subsection*{P.10. What this appendix does \emph{not} assert (hard boundaries)}
% -------------------------
No pre-collapse control; no claims of strict commutation beyond recorded \(2\)-cell bounds; no global equivalences beyond the one-way bridge \(\mathrm{PH}_1\Rightarrow \Ext^1\) used only in its stated derived setting.
The dictionary never acts as a Gate, never overrides certified after-collapse tests, and never upgrades advisory proxies to certificates.

% -------------------------
\subsection*{P.11. Summary}
% -------------------------
A tropical\(\to\)LMHS lookup provides \emph{after-collapse} advisory signals on definable windows.
Quantitative effects are integrated through:
(i) controlled Mirror/Tropical \(2\)-cell bounds \(\delta_{\mathrm{trop}}(i,\tau;W)\) (Appendix~L), and
(ii) a confidence-weighted advisory term \(\delta^{\mathrm{spec}}=\omega\odot \widehat{\delta}_{\mathrm{spec}}\) (Definition~\ref{P:def:conf}).
All contributions enter the post-collapse window budget via the quantale sum \(\oplus\) (Definition~\ref{P:def:budget}) and are measured in the fixed order
“for each \(t\)\(\to\)\(\mathbf{P}_i\)\(\to\)\(\mathbf{T}_\tau\)\(\to\)compare.”
With Summability (Appendix~J), windowwise certificates paste across locally finite covers.
Accordingly, the dictionary remains a reproducible, non-gating aid to certified after-collapse diagnostics, aligned with Appendices~K/L/N and the global run protocol of Appendix~G.



% =========================
\section*{Appendix Q. \(p\)-adic Definable Windows (Denef--Pas) [Spec; finite event decomposition; finite \v{C}ech checks; local comparisons after collapse] (canon-aligned)}
% =========================
\phantomsection
\addcontentsline{toc}{section}{Appendix Q. \(p\)-adic Definable Windows (Denef--Pas)}
\refstepcounter{section}
\label{Q:denef-pas}

\paragraph{Standing conventions (canon).}
Let \(K\) be a non-archimedean local field with valuation \(\val:K^\times\to\ZZ\), residue field \(k\), and angular component \(\ac:K\to k\cup\{0\}\).
We work in the three-sorted Denef--Pas structure \((\VF,\RF,\VG)=(K,k,\ZZ)\) with Presburger arithmetic on \(\VG\).
This appendix is \textup{[Spec]}: it provides a \emph{definable windowing interface} on valuation-time \(\VG\) compatible with the global run protocol.

\medskip
\noindent\textbf{Global run protocol (non-negotiable).}
All numeric comparisons are evaluated \emph{after collapse} in the fixed order
\[
\boxed{\ \text{for each }t\ \Longrightarrow\ \mathbf{P}_i\ \Longrightarrow\ \mathbf{T}_\tau\ \Longrightarrow\ \text{compare on }\mathbf{T}_\tau\mathbf{P}_i\ },
\]
with the same MECE, right-open windows and the same \(\tau\) as B-Gate\(^{+}\) and the \(\delta\)-ledger (Appendix~G/J/N).
All defects live in a fixed commutative quantale \(V\) with sum \(\oplus\), order \(\preceq\), and scalar action \(\odot\) (Appendix~K/L/S), and are recorded in \texttt{run.yaml} (Appendix~G).

\medskip
\noindent\textbf{Embedding \(\VG\) into the global time axis.}
To align valuation-time with the global real parameter, fix once per run an affine embedding
\[
\iota:\VG=\ZZ\hookrightarrow \RR,\qquad \iota(n):=t_0+s\cdot n,\quad s>0,
\]
and interpret all \(\VG\)-windows on the real axis via \(\iota\).
All window endpoints remain \emph{right-open} under \(\iota\).
(Changing \((t_0,s)\) is a reindexing choice and must be logged.)

% -------------------------
\subsection*{Q.1. Denef--Pas definable windows (right-open, MECE-ready)}
% -------------------------
\begin{definition}[DP-definable right-open window on \(\VG\)]\label{Q:def:dp-window}
A \emph{right-open window} \(W\subset \VG\) is a subset of \(\ZZ\) definable in Denef--Pas whose \(\VG\)-projection admits a finite disjoint-union representation
\[
W\ =\ \bigsqcup_{m=1}^M \Big\{\,n\in\ZZ\ \Big|\ a_m \le n < b_m,\ n\equiv r_m \!\!\!\pmod{c_m}\,\Big\},
\]
with \(a_m\in\ZZ\cup\{-\infty\}\), \(b_m\in\ZZ\cup\{+\infty\}\), \(c_m\in\ZZ_{>0}\), \(r_m\in\ZZ/c_m\ZZ\).
Via \(\iota\), we regard \(W\) as a right-open subset of the global time axis.
\end{definition}

\begin{remark}[Uniform families]\label{Q:rk:param}
For a DP-definable parameter set \(\Lambda\), a family \(\{W_\lambda\}_{\lambda\in\Lambda}\) is \emph{uniformly} DP-definable if it admits a decomposition as in Definition~\ref{Q:def:dp-window} with Presburger-definable data
\(a_m(\lambda),b_m(\lambda),r_m(\lambda),c_m(\lambda)\).
All finiteness bounds below then hold uniformly in \(\lambda\) \emph{as statements about the \(\VG\)-sort}.
\end{remark}

\begin{definition}[Canonical MECE refinement on bounded valuation ranges]\label{Q:def:mece}
Fix a bounded valuation range \(I=[A,B)\subset\VG\) with \(A,B\in\ZZ\).
A \emph{canonical MECE refinement} of \(I\) is a finite partition into right-open single-step windows
\[
I\ =\ \bigsqcup_{n=A}^{B-1} [n,n{+}1),
\]
transported to \(\RR\) via \(\iota\).
All audits and budgets on DP windows are ultimately reduced to this MECE refinement unless explicitly stated otherwise.
\end{definition}

% -------------------------
\subsection*{Q.2. Finite event decomposition and finite \v{C}ech checks (windowed)}
% -------------------------
\begin{theorem}[\textnormal{\textbf{[Spec]}} DP windows on \(\VG\): finite events; finite overlap checks]\label{Q:thm:dp-finite}
Let \(W\subset\VG\) be DP-definable and \emph{bounded}, i.e.\ \(W\subset I=[A,B)\) for some integers \(A<B\).
Then:
\begin{enumerate}\itemsep0.35em
  \item \textbf{Finite event decomposition (after collapse).}
  Any \(\VG\)-indexed bookkeeping map used \emph{after collapse}---including (i) bar endpoint readouts of \(\mathbf{T}_\tau\mathbf{P}_i\), (ii) bin assignments, (iii) defect counters and ledger tags---is Presburger definable on \(\VG\).
  Hence on \(W\subset[A,B)\) it has only finitely many value changes. Equivalently, on the canonical MECE refinement (Definition~\ref{Q:def:mece}) all such readouts are determined by finitely many cells/steps.
  \item \textbf{Finite \v{C}ech checks for the \emph{run cover}.}
  Under the global window policy (Appendix~G/J), the cover used for audits on a bounded range is a finite family of right-open windows (typically the MECE refinement or a coarsening).
  Therefore the \v{C}ech nerve is finite and Overlap Glue terminates after finitely many checks (Appendix~J).
  \item \textbf{Comparators are evaluated only after collapse.}
  PF/BC/Control comparisons (Appendix~N and any applicable control appendices) are evaluated only on \(\mathbf{T}_\tau\mathbf{P}_i(F|_W)\) in the mandated order.
  Any residual slack (discretization/measurement or [Spec] commutation defects) is recorded as \(\delta\)-terms and aggregated by \(\oplus\) in \(V\).
  \item \textbf{Reproducibility.}
  Event counts (or the MECE refinement size), overlap-check counts, and all windowwise budgets \(\Sigma\delta_W(i)\) must be logged in \texttt{run.yaml} (Appendix~G).
\end{enumerate}
\end{theorem}

\begin{proof}[Proof sketch]
Presburger quantifier elimination yields piecewise-linear/constant behavior on \(\VG\); on a bounded subset of \(\ZZ\) this implies finitely many changes.
The run protocol enforces a finite window family on bounded ranges; hence the nerve is finite and Overlap Glue performs finitely many comparisons.
All comparisons are post-collapse by policy and use exactness/\(1\)-Lipschitz of \(\mathbf{T}_\tau\) (Appendix~A).
\end{proof}

\begin{remark}[What “finite \v{C}ech” means here]\label{Q:rk:cech}
We do \emph{not} assume an arbitrary DP-definable cover has small nerve dimension.
Instead, finiteness is ensured operationally by using a \emph{finite run cover} on bounded ranges (Appendix~G/J), typically derived from the MECE refinement (Definition~\ref{Q:def:mece}) and/or a finite coarsening recorded in \texttt{run.yaml}.
\end{remark}

% -------------------------
\subsection*{Q.3. \(V\)-nonexpansion and shift commutation (after collapse)}
% -------------------------
\begin{proposition}[Shift commutation and \(V\)-nonexpansion (post-collapse)]\label{Q:prop:Vshift-DP}
Let \(S^v\) be a \(V\)-Lawvere shift/reindexing operator (Chapter~2) acting on persistence by time translation/relabeling, and fix \(\tau\ge0\).
Then on any window \(W\) (viewed on \(\RR\) via \(\iota\)) we have a canonical identification
\[
\mathbf{T}_\tau\circ S^v\ \cong\ S^v\circ \mathbf{T}_\tau,
\]
and consequently, for any \(F,G\),
\[
d_V\!\Big(\mathbf{T}_\tau\mathbf{P}_i(F|_W),\ \mathbf{T}_\tau\mathbf{P}_i(G|_W)\Big)\ \preceq\
d_V\!\Big(\mathbf{P}_i(F|_W),\ \mathbf{P}_i(G|_W)\Big).
\]
All distances are measured after \(\mathbf{T}_\tau\); any subsequent \(1\)-Lipschitz post-processing on persistence cannot increase them (Appendix~A/L).
\end{proposition}

% -------------------------
\subsection*{Q.4. Local bridge trigger via \(E_1\) on DP windows (after collapse)}
% -------------------------
\begin{corollary}[Local \(E_1\) trigger on DP windows (approved direction)]\label{Q:cor:dp-bridge}
Let \(W\subset\VG\) be a bounded DP-definable window, and fix \(\tau\ge0\).
If the certified, post-collapse test \(E_1(W)=0\) holds for \(\mathbf{T}_\tau\mathbf{P}_1(F|_W)\) (Appendix~M/J), then
\[
E_1(W)=0\ \Longrightarrow\ \mathbf{T}_\tau\mathbf{P}_1(F|_W)=0
\ \Longrightarrow\ \mathrm{PH}_1(\mathbf{T}_\tau F|_W)=0
\ \Longrightarrow\ \Ext^1\!\big(\mathcal{R}(\mathbf{T}_\tau F)|_W,k\big)=0,
\]
where the last implication is the one-way bridge used only in its stated derived setting (Appendix~C).
No global converse and no additional equivalences are claimed.
\end{corollary}

% -------------------------
\subsection*{Q.5. Quantitative commutation for DP transfers (Spec; after collapse)}
% -------------------------
\begin{declaration}[DP-transfer contract with 2-cell bound \textup{[Spec]}]\label{Q:dec:dp-2cell}
Let \(\mathrm{Trans}\) be a DP-definable transfer/post-processing step (e.g.\ a \(p\)-adic tropicalization or functorial comparison) which is non-expansive up to a recorded defect on a window \(W\).
Fix \(\tau\ge0\). Record parameters \(\kappa\in(0,1]\) and \(\delta_{\mathrm{Tr}}(i,\tau;W)\in V\) such that, \emph{after} \(\mathbf{T}_\tau\),
\[
d_V\!\Big(\mathbf{T}_\tau\mathbf{P}_i(\mathrm{Trans}\,F|_W),\ \mathbf{T}_\tau\mathbf{P}_i(\mathrm{Trans}\,G|_W)\Big)
\ \preceq\
\kappa\odot d_V\!\Big(\mathbf{T}_\tau\mathbf{P}_i(F|_W),\ \mathbf{T}_\tau\mathbf{P}_i(G|_W)\Big)\ \oplus\ \delta_{\mathrm{Tr}}(i,\tau;W).
\]
Log \(\kappa\) and \(\delta_{\mathrm{Tr}}(i,\tau;W)\) in \texttt{run.yaml} and include them in the window budget \(\Sigma\delta_W(i)\) (Appendix~L/K/G).
\end{declaration}

\begin{remark}[No hidden commutation]\label{Q:rk:nohidden}
No commutation beyond the recorded defect \(\delta_{\mathrm{Tr}}\) is assumed.
If a stricter commutation statement is needed, it must be stated as a separate \(2\)-cell bound and explicitly budgeted (Appendix~L).
\end{remark}

% -------------------------
\subsection*{Q.6. P6 (Summability): \(\bigoplus_W \Sigma\delta_W\) is finite on bounded DP ranges}
% -------------------------
\begin{theorem}[T-Delta-Sum-Converges (P6) on bounded DP ranges]\label{Q:thm:P6}
Let \(I=[A,B)\subset\VG\) be bounded and let \(\{W_j\}_{j\in J}\) be a \emph{locally finite} family of DP-definable windows with \(\bigcup_{j\in J}W_j\supset I\).
Fix degree \(i\) and a pipeline whose per-window ledger yields \(\Sigma\delta_{W_j}(i)\in V\) (Appendix~G/L/K).
Then the aggregated budget over \(I\),
\[
\bigoplus_{j\in J}\Sigma\delta_{W_j}(i),
\]
is a finite \(\oplus\)-sum (hence convergent) in \(V\).
\end{theorem}

\begin{proof}
Since \(I=[A,B)\subset\ZZ\) is finite as a set of valuation levels, local finiteness implies that only finitely many windows \(W_j\) intersect \(I\) (otherwise some level \(n\in I\) would lie in infinitely many \(W_j\)).
Hence the index set \(\{j\in J: W_j\cap I\neq\emptyset\}\) is finite, and the displayed \(\oplus\)-sum is finite.
\end{proof}

\begin{corollary}[Unbounded ranges: explicit summability hypothesis \textup{[Spec]}]\label{Q:cor:P6-unbounded}
If \(I=\VG\) is unbounded, P6 is enforced by an explicit summability hypothesis recorded in \texttt{run.yaml}, e.g.:
there exist \(\rho\in(0,1)\) and \(C\in V\) such that all per-level contributions satisfy
\(\delta_{\text{type}}(n)\preceq C\odot \rho^{|n|}\),
or any other declared sufficient criterion ensuring \(\bigoplus_n \delta(n)\) converges in \(V\).
Absent a logged summability rule, no global convergence claim is made.
\end{corollary}

\begin{remark}[Uniformity across DP-families \textup{[Spec]}]\label{Q:rk:uniform}
In DP-uniform families, any claimed parameter-uniform bounds (e.g.\ uniform \(C,\rho\) for Corollary~\ref{Q:cor:P6-unbounded}) must be stated as \textup{[Spec]} and logged explicitly. No analytic uniformity (e.g.\ “uniform in \(p\)”) is assumed unless recorded with an auditable \(\delta\)-budget.
\end{remark}

% -------------------------
\subsection*{Q.7. Reproducibility (run.yaml) for DP mode}
% -------------------------
\begin{verbatim}
policy:
  after_collapse_only: true
definable:
  structure: "Denef-Pas"
  field: "Q_p"             # or finite extension; must specify
  uniformizer: "pi"
  vg_embedding:
    t0: 0.0
    scale: 1.0             # s>0; t = t0 + s*n
windows:
  mece: true
  right_open: true
  vg_windows:
    - "A <= n < B"                      # bounded valuation range I
    - "a1 <= n < b1 and n % c1 == r1"   # optional coarsenings
checks:
  finite_events: true
  finite_overlap: true
two_cell:
  dp_transfer:
    kappa: "<0<kappa<=1>"
    delta_Tr: "<rule or per-(i,tau,W) table>"
ledger:
  channels: ["disc","meas","Tr","AB","spec"]
\end{verbatim}

% -------------------------
\subsection*{Q.8. Minimal working recipe (Spec; non-gating)}
% -------------------------
\begin{enumerate}\itemsep0.35em
  \item \textbf{Fix DP windows.} Specify \(I=[A,B)\subset\VG\) and any DP-definable coarsenings \(W\subset I\) (Definition~\ref{Q:def:dp-window}); log the formulas and \(\iota\).
  \item \textbf{Reduce to MECE if needed.} Refine to the canonical MECE partition into \([n,n{+}1)\) (Definition~\ref{Q:def:mece}) whenever budgets or Overlap Glue require normalization.
  \item \textbf{Compute post-collapse readouts.} For each window \(W\) and degree \(i\), compute \(\mathbf{T}_\tau\mathbf{P}_i(F|_W)\) in the mandated order; extract event counts (Theorem~\ref{Q:thm:dp-finite}).
  \item \textbf{Run certified local tests.} If triggered, run \(E_1(W)=0\) after collapse; when verified, apply Corollary~\ref{Q:cor:dp-bridge} in the approved direction.
  \item \textbf{Apply transfers only with budgets.} If a DP-transfer is used, record \(\kappa\) and \(\delta_{\mathrm{Tr}}(i,\tau;W)\) and include them in \(\Sigma\delta_W(i)\) (Declaration~\ref{Q:dec:dp-2cell}).
  \item \textbf{P6 on bounded ranges.} On bounded \(I\), convergence of the aggregated budget is automatic under local finiteness (Theorem~\ref{Q:thm:P6}); on unbounded ranges, enforce and log an explicit summability rule (Corollary~\ref{Q:cor:P6-unbounded}).
\end{enumerate}

% -------------------------
\subsection*{Q.9. Non-claims (hard boundaries)}
% -------------------------
No pre-collapse control; no commutation beyond recorded \(2\)-cell bounds; no equivalence beyond the one-way bridge \(\mathrm{PH}_1\Rightarrow\Ext^1\) in \(D^{\mathrm{b}}(k\text{-mod})\).
No analytic uniformity across primes/fields is assumed unless stated as \textup{[Spec]} with explicit, audited \(\delta\)-budgets.

% -------------------------
\subsection*{Q.10. Integration points (scope-safe)}
% -------------------------
This appendix supplies a DP-definable \(\VG\)-window interface compatible with:
(i) the global MECE/right-open window policy and reproducible run protocol (Appendix~G),
(ii) finite Overlap Glue checks on bounded windows (Appendix~J),
(iii) post-collapse comparison discipline and budget accounting (Appendix~L/K/N),
and (iv) local bridge triggers via certified after-collapse tests (Appendix~C/M).
Any arithmetic/motivic/Langlands or other external uses remain \textup{[Spec]} and must route through explicit \(2\)-cell bounds and \(\delta\)-ledger entries; nothing in this appendix upgrades such applications to certified results.



% =========================
\section*{Appendix R. Iwasawa--AK Interface [Spec; Control $\Rightarrow$ Overlap Gate \& $\mu$-alignment guidelines] (canon-aligned)}
% =========================
\phantomsection
\addcontentsline{toc}{section}{Appendix R. Iwasawa--AK Interface}
\refstepcounter{section}
\label{R:iwasawa}

\paragraph{Standing conventions (canon).}
Fix a prime \(p\). Let \(\Gamma\simeq \ZZ_p\) be the Galois group of a cyclotomic (or other one-parameter) tower \(K_\infty/K\) with finite layers \(K_n\) and open subgroups \(\Gamma_n=\Gamma^{p^n}\).
Write \(\Lambda_{\mathrm{Iw}}=\ZZ_p\llbracket \Gamma\rrbracket\).
Let \((M_n)_{n\ge 0}\) be a compatible system of finite \(p\)-primary modules (e.g.\ class groups or Selmer quotients), and \(M_\infty=\varprojlim M_n\) a cofinitely generated \(\Lambda_{\mathrm{Iw}}\)-module.

\medskip
\noindent\textbf{Mandatory measurement order (after-collapse).}
All persistence measurements and numeric comparisons are evaluated \emph{after collapse} in the fixed order
\[
\boxed{\ \text{for each }t\ \Longrightarrow\ \mathbf{P}_i\ \Longrightarrow\ \mathbf{T}_\tau\ \Longrightarrow\ \text{compare on }\mathbf{T}_\tau\mathbf{P}_i\ },
\]
with the same MECE, right-open windows and the same \(\tau\) as B-Gate\(^{+}\) and the \(\delta\)-ledger (Chapter~1; Appendix~G/J/N).
Truncation \(\mathbf{T}_\tau\) is exact and \(1\)-Lipschitz (Appendix~A).
All defects aggregate in a fixed commutative quantale \(V\) with sum \(\oplus\), order \(\preceq\), and scalar action \(\odot\) (Appendix~K/L/S), and are recorded in \texttt{run.yaml} (Appendix~G).

\medskip
\noindent\textbf{Definable windows and finite checks.}
Windows are definable either o-minimally (real side) or in Denef--Pas (Appendix~Q). All runtime covers used for gates/audits on bounded ranges are \emph{finite} (or locally finite with explicit P6 logging), with finite Overlap Glue checks as per Appendix~J/Q.
Any statement in this appendix that depends on definability/coverage is \textup{[Spec]} and must be backed by a logged cover specification and event counts.

\medskip
\noindent\textbf{Finite means finite \(p\)-group.}
Every reference to a kernel/cokernel being \emph{finite} means a finite \(p\)-group.

% -------------------------
\subsection*{R.1. Control maps and the arithmetic Overlap Gate (post-collapse)}
% -------------------------

\begin{definition}[Control map]\label{R:def:control}
A \emph{control map} at level \(n\) is a natural comparison
\[
\Phi_n:\ (M_\infty)_{\Gamma_n}\ \longrightarrow\ M_n,
\]
with finite kernel and cokernel, where \((-)_{\Gamma_n}\) denotes coinvariants.
\end{definition}

\begin{definition}[Control defect magnitude]\label{R:def:ctrl-mag}
Define the (dimensionless) control defect magnitude
\[
e_{\mathrm{ctrl}}(n)\ :=\ v_p\!\left|\ker\Phi_n\right|\ +\ v_p\!\left|\coker\Phi_n\right|\ \in \RR_{\ge 0}.
\]
\end{definition}

\begin{definition}[Budget embedding]\label{R:def:eta}
Fix a monotone embedding \(\eta:\RR_{\ge 0}\to V\) (e.g.\ \(\eta(x)=x\) for \(V=([0,\infty],+,0)\)).
The \emph{control budget} recorded on a window \(W\) is
\[
\delta_{\mathrm{ctrl}}(n;\,W)\ :=\ \eta\!\big(e_{\mathrm{ctrl}}(n)\big)\ \in V.
\]
By default \(\delta_{\mathrm{ctrl}}(n;\,W)\) is taken constant in \(W\) (control is arithmetic-level), but a window-local version is permitted if explicitly logged and justified by the pipeline interface used.
\end{definition}

\begin{definition}[Arithmetic Overlap Gate (AK; post-collapse)]\label{R:def:overlap-gate}
Let \(\{W_\alpha\}\) be a \emph{finite} definable cover of a bounded time window (Appendix~G/J/Q), with overlap multiplicity
\[
m\ :=\ \sup_{t}\#\{\alpha:\ t\in W_\alpha\}\quad(\text{finite; logged in \texttt{run.yaml}}).
\]
An \emph{Arithmetic Overlap Gate} for a comparison \((F\Rightarrow G)\) requires that for every overlap \(W_{\alpha\beta}:=W_\alpha\cap W_\beta\),
the post-collapse comparison map
\[
\vartheta_{\alpha\beta}:\ \mathbf{T}_\tau\mathbf{P}_i\!\left(F|_{W_{\alpha\beta}}\right)\ \longrightarrow\ \mathbf{T}_\tau\mathbf{P}_i\!\left(G|_{W_{\alpha\beta}}\right)
\]
is an isomorphism in \(\Pers^{\mathrm{ft}}_k\) \emph{up to a finite \(p\)-group discrepancy} whose size is recorded as a ledger term
\(\delta^{\mathrm{alg}}(i,\tau;W_{\alpha\beta})\in V\).
All such \(\delta^{\mathrm{alg}}\) aggregate by \(\oplus\) in the window budget.
\end{definition}

\begin{remark}[Why we ledger finite discrepancies]\label{R:rk:finite-ledger}
Finite \(p\)-group kernels/cokernels are invisible to generic-dimension diagnostics (\(\mu,\nu\) in Appendix~D) but are \emph{not} ignored operationally:
they are recorded as \(\delta^{\mathrm{alg}}\) and charged to the window budget.
This separates ``Type~IV'' (invisible colimit defects) from finite arithmetic control errors: both can be invisible to \((\mu,\nu)\), but both must be budgeted when they affect comparisons.
\end{remark}

\begin{proposition}[Control \(\Rightarrow\) Arithmetic Overlap Gate \textup{[Spec]}]\label{R:prop:control-overlap}
Assume:
\begin{enumerate}\itemsep0.25em
  \item A control system \(\{\Phi_n\}\) as in Definition~\ref{R:def:control}.
  \item A definable finite cover \(\{W_\alpha\}\) on the time axis with overlap multiplicity \(m\) (Definition~\ref{R:def:overlap-gate}).
  \item An \textup{[Spec]} AK--Iwasawa interface that assigns to arithmetic objects a comparison pair \((F\Rightarrow G)\) in the pipeline such that, \emph{after} the mandated order
  \(t\to \mathbf{P}_i\to \mathbf{T}_\tau\to\)compare,
  the induced overlap discrepancies are bounded by the control defect magnitude:
  \[
  \delta^{\mathrm{alg}}(i,\tau;W_{\alpha\beta})\ \preceq\ \delta_{\mathrm{ctrl}}(n;W_{\alpha\beta})
  \qquad\text{for all overlaps }W_{\alpha\beta}.
  \]
\end{enumerate}
Then the Arithmetic Overlap Gate holds (Definition~\ref{R:def:overlap-gate}), and moreover the total algebraic budget over the cover satisfies
\[
\bigoplus_{\alpha\beta}\ \delta^{\mathrm{alg}}(i,\tau;W_{\alpha\beta})
\ \preceq\
m\ \odot\ \sup_{\alpha\beta}\delta_{\mathrm{ctrl}}(n;W_{\alpha\beta}).
\]
\end{proposition}

\begin{proof}[Proof idea]
All comparisons are post-collapse by policy; exactness of \(\mathbf{T}_\tau\) preserves the finite-defect nature of arithmetic discrepancies.
The bound \(\delta^{\mathrm{alg}}\preceq\delta_{\mathrm{ctrl}}\) is the interface axiom.
Since overlap multiplicity is \(m\), each timepoint participates in at most \(m\) overlaps; quantale monotonicity and commutativity yield the stated aggregation bound.
\end{proof}

\begin{corollary}[Finite glue with budget \textup{[Spec]}]\label{R:cor:glue}
Under Proposition~\ref{R:prop:control-overlap}, Overlap Glue (Appendix~J) performs finitely many overlap checks on the bounded run cover and yields a ledgered algebraic budget
\[
\delta_{\mathrm{alg}}^{\mathrm{tot}}(i,\tau)\ \preceq\ m\odot \sup_{\alpha\beta}\delta_{\mathrm{ctrl}}(n;W_{\alpha\beta}),
\]
which must be included in \(\Sigma\delta_W(i)\) for the relevant windows.
\end{corollary}

\begin{remark}[Cofinal tails]\label{R:rk:cofinal}
Replacing \((M_n)\) by a cofinal subsequence in \(n\) preserves the Overlap Gate verdict and the AK tower diagnostics \((\mu,\nu)\) computed after \(\mathbf{T}_\tau\) (Appendix~D/J), provided the interface and window policy are unchanged and the ledger is updated accordingly.
\end{remark}

% -------------------------
\subsection*{R.2. $\mu$--alignment guidelines (window-local; non-identification by default)}
% -------------------------

Let \(\mu_{\mathrm{class}}\) denote the classical Iwasawa \(\mu\)-invariant of the torsion \(\Lambda_{\mathrm{Iw}}\)-module \(M_\infty\) (when defined).
Let \(\mu_{\mathrm{Collapse}}\) denote the AK diagnostic \(\mu\) computed \emph{after} \(\mathbf{T}_\tau\) on windows from the tower comparison map (Appendix~D/J).

\begin{declaration}[$\mu$--alignment decision tree (Spec; window-local; verification-first)]\label{R:dec:mu-align}
Work on a definable run cover with logged overlap multiplicity \(m\) and finite checks; all quantities below are post-collapse and budgets aggregate in \(V\).
\begin{enumerate}\itemsep0.35em
  \item \textbf{Precheck (torsion).}
  If \(M_\infty\) has nonzero \(\Lambda_{\mathrm{Iw}}\)-rank, declare \(\mu_{\mathrm{class}}=\mathrm{NA}\) and do \emph{not} attempt alignment; report the free rank separately.

  \item \textbf{Control regime classification.}
  Compute/declare the control trend from \(e_{\mathrm{ctrl}}(n)\) (Definition~\ref{R:def:ctrl-mag}):
  \begin{itemize}\itemsep0.15em
    \item \emph{Bounded control:} \(\sup_n e_{\mathrm{ctrl}}(n)<\infty\).
    \item \emph{Sublinear control:} \(e_{\mathrm{ctrl}}(n)=o(p^n)\).
    \item \emph{Uncontrolled:} otherwise.
  \end{itemize}

  \item \textbf{Pipeline regime classification.}
  Classify the post-collapse pipeline steps on the window as:
  \begin{itemize}\itemsep0.15em
    \item \emph{Deletion-only:} only deletion-type steps (Appendix~E/O) and \(1\)-Lipschitz post-processing occur after collapse.
    \item \emph{General:} inclusion-type amplification may occur (no monotone claim; stability only).
  \end{itemize}

  \item \textbf{Alignment outcomes (guidelines; not automatic identifications).}
  \begin{enumerate}\itemsep0.2em
    \item \emph{Coincidence (verified):}
    If bounded control \emph{and} deletion-only pipeline, then alignment
    \(\mu_{\mathrm{Collapse}}\approx\mu_{\mathrm{class}}\) is \emph{permitted as a status} only after a cofinal-tail regression check succeeds on the run cover, with tolerance derived from the ledger (see Remark~\ref{R:rk:mu-verify}).
    \item \emph{Lower-bound mode:}
    Under bounded control alone, report \(\mu_{\mathrm{Collapse}}\) as a \textup{[Spec]} \emph{upper envelope} of persistent obstruction; do not claim equality unless verified.
    (Heuristic expectation: inclusion-type amplification can inflate \(\mu_{\mathrm{Collapse}}\), hence \(\mu_{\mathrm{class}}\preceq \mu_{\mathrm{Collapse}}\) is a plausible directional check, but remains \textup{[Spec]} unless verified.)
    \item \emph{Bracket mode:}
    Under sublinear control, define the drift rate
    \[
      \mu_{\mathrm{drift}}\ :=\ \limsup_{n\to\infty}\frac{e_{\mathrm{ctrl}}(n)}{p^n}\ \in \RR_{\ge 0},
    \]
    and report a window-local bracket statement only in the form:
    \[
      \text{``}\mu_{\mathrm{Collapse}}\ \text{is consistent with }\mu_{\mathrm{class}}\text{ up to drift }\mu_{\mathrm{drift}}\text{''},
    \]
    together with the logged \(\mu_{\mathrm{drift}}\) and the observed \(\delta_{\mathrm{ctrl}}\) trend; no numeric inequality is promoted to certified status unless verified.
    \item \emph{No-claim mode:}
    Under uncontrolled growth of \(e_{\mathrm{ctrl}}(n)\), do not align; report \(\mu_{\mathrm{Collapse}}\) together with the ledger trend and the failure reason (\texttt{mu\_decision.status = no\_claim}).
  \end{enumerate}
\end{enumerate}
All outcomes are \textbf{window-local} and are \textbf{non-identifications} unless explicitly marked \emph{Coincidence (verified)}.
\end{declaration}

\begin{remark}[Verification hook for ``Coincidence'']\label{R:rk:mu-verify}
A minimal verification routine on a cofinal tail \(\{n_j\}\) is:
(i) compute an empirical growth proxy \(\widehat{\mu}(n_j)\) from the chosen arithmetic observable (e.g.\ \(p^{-n_j}v_p|M_{n_j}|\) if available and logged);
(ii) compute \(\mu_{\mathrm{Collapse}}\) from the AK diagnostics on the same windows (after \(\mathbf{T}_\tau\));
(iii) accept ``Coincidence (verified)'' only if the discrepancy stays below a tolerance derived from the full window budget
\(\Sigma\delta_W(i)\) and the control budget \(\delta_{\mathrm{ctrl}}\), for all windows in the run cover.
The exact observable and tolerance rule must be declared in \texttt{run.yaml}.
\end{remark}

\begin{remark}[What each $\mu$ measures]\label{R:rk:mu-meaning}
\(\mu_{\mathrm{class}}\) governs \(p\)-power growth in the classical Iwasawa asymptotic (when applicable).
\(\mu_{\mathrm{Collapse}}\) is a post-collapse obstruction diagnostic derived from the AK tower comparison map after truncation (Appendix~D/J).
This appendix provides \emph{alignment guidelines} and logging requirements; it does not upgrade any arithmetic identification to a certified theorem without an explicit verification routine and budget justification.
\end{remark}

% -------------------------
\subsection*{R.3. Gate coupling and Restart/Summability (P6)}
% -------------------------

\begin{definition}[Gate coupling (Overlap Gate + B-Gate\(^{+}\))]\label{R:def:gate-couple}
Couple the Arithmetic Overlap Gate with B-Gate\(^{+}\) (Appendix~J).
On a window \(W\) and degree \(i\), B-Gate\(^{+}\) passes if
\[
\mathrm{gap}_\tau(i)\ >\ \Sigma\delta_W(i),
\]
where \(\Sigma\delta_W(i)\) includes \(\delta_{\mathrm{ctrl}}(n;W)\), all overlap-glue algebraic terms \(\delta^{\mathrm{alg}}\), and any other pipeline terms (Mirror/Transfer, A/B residuals, discretization/measurement).
\end{definition}

\begin{proposition}[Restart and P6 (window pasting)]\label{R:prop:restart-P6}
Assume consecutive windows are linked only by deletion-type steps and \(\varepsilon\)-continuations measured post-collapse, and the run satisfies P6:
\(\sum_W \Sigma\delta_W(i)<\infty\) (Appendix~J; bounded DP ranges satisfy finiteness as per Appendix~Q).
Then the Restart inequality of Appendix~J propagates a positive safety margin across windows, and windowwise certificates paste to a global certificate on \(\bigcup W\).
\end{proposition}

% -------------------------
\subsection*{R.4. Reproducibility hooks (run.yaml)}
% -------------------------
\begin{verbatim}
policy:
  after_collapse_only: true
iwasawa:
  prime: p
  tower: "cyclotomic"              # description
  gamma_presentation: "Z_p"        # generator choice logged
  levels: {start: n0, end: n1}
  control:
    e_ctrl_rule: "vp(|ker Phi_n|)+vp(|coker Phi_n|)"
    bounded: true/false
    sublinear: true/false
    eta: "<monotone embedding R>=0 -> V>"
    delta_ctrl: "<rule or table>"
overlap_gate:
  cover: "<o-minimal | Denef-Pas>"
  windows: "<MECE, right-open spec>"
  overlap_multiplicity_m: "<integer>"
  finite_checks: true
ledger:
  channels: ["ctrl","alg","disc","meas","Tr","AB","spec"]
b_gate_plus:
  tau: "<value>"
  gap_tau: "<per window/degree>"
  passed: true/false
mu_alignment:
  classical_mu: "<value | NA>"
  collapse_mu: "<value | per-window table>"
  pipeline_regime: "deletion_only | general"
  status: "NA | coincidence_verified | lower_bound | bracket | no_claim"
  drift_mu: "<value if bracket>"
  verification:
    observable: "p^{-n} vp(|M_n|)"   # or declared alternative
    tolerance_rule: "<uses Sigma_delta and delta_ctrl>"
tests:
  T_Iwasawa: true
\end{verbatim}

% -------------------------
\subsection*{R.5. Minimal decision routine (Spec; non-gating)}
% -------------------------
\begin{verbatim}
# inputs: classical_mu (or NA), e_ctrl(n), pipeline_regime
def mu_alignment(classical_mu, e_ctrl, pipeline_regime, verified):
    if classical_mu == "NA":
        return ("NA", None)
    bounded = sup_n(e_ctrl[n]) < ∞
    sublinear = limsup_n(e_ctrl[n] / p**n) == 0
    if bounded and pipeline_regime == "deletion_only" and verified:
        return ("coincidence_verified", 0)
    if bounded:
        return ("lower_bound", 0)        # guideline only unless verified
    if sublinear:
        drift = limsup_n(e_ctrl[n] / p**n)
        return ("bracket", drift)        # consistency statement only
    return ("no_claim", None)
\end{verbatim}

% -------------------------
\subsection*{R.6. Edge cases and guard-rails}
% -------------------------
\begin{itemize}\itemsep0.25em
  \item \textbf{Non-torsion modules.} If \(\mathrm{rank}_{\Lambda_{\mathrm{Iw}}}M_\infty>0\), set \(\mu_{\mathrm{class}}=\mathrm{NA}\) and do not align.
  \item \textbf{Inclusion-type steps.} Any inclusion-type amplification (post-collapse) forbids automatic ``Coincidence''; require verification or fall back to Lower-bound/Bracket/No-claim.
  \item \textbf{Overlap multiplicity.} The bound \(m\) must be computed from the \emph{actual run cover} and logged; do not assume \(m\le2\) unless the cover is explicitly MECE/right-open with the corresponding overlap structure.
  \item \textbf{Character twists.} Alignment routines may be repeated under finite character twists; twists must be logged as part of the arithmetic configuration.
  \item \textbf{Cofinal tails.} Replacing \((M_n)\) by a cofinal tail preserves window policies and diagnostics provided the ledger is recomputed and the same \(\tau\) and windows are used.
\end{itemize}

% -------------------------
\subsection*{R.7. Formalization stubs (Lean/Coq) [Spec]}
% -------------------------
\begin{verbatim}
structure ControlMap (n : ℕ) :=
  (phi : coinv Γ_n M∞ ⟶ M_n)
  (ker_fin  : finite_p (kernel phi))
  (coker_fin : finite_p (cokernel phi))

def e_ctrl (φ : ControlMap n) : ℝ≥0 :=
  vp (kernel φ).card + vp (cokernel φ).card

def δ_ctrl (η : ℝ≥0 → V) (φ : ControlMap n) : V :=
  η (e_ctrl φ)

-- Interface axiom: post-collapse algebraic defects are bounded by δ_ctrl
axiom iwasawa_AK_interface :
  ∀ (φ : ControlMap n) (W : Window) (i : ℤ) (τ : ℝ≥0),
    δ_alg(i,τ,W) ≤ δ_ctrl η φ

theorem control_implies_overlap_gate :
  ...  -- derives Proposition R:prop:control-overlap
\end{verbatim}

% -------------------------
\subsection*{R.8. Summary}
% -------------------------
Control maps with finite kernel/cokernel yield, via an explicit \textup{[Spec]} AK--Iwasawa interface, an Arithmetic Overlap Gate after collapse.
Finite discrepancies are budgeted as \(\delta^{\mathrm{alg}}\) using a monotone embedding \(\eta\) of \(p\)-exponents into the quantale \(V\), and aggregated over a finite definable run cover with logged overlap multiplicity \(m\).
Gate coupling with B-Gate\(^{+}\) is performed windowwise using the unified budget \(\Sigma\delta_W(i)\) (including \(\delta_{\mathrm{ctrl}}\)), and global certificates paste under Restart/Summability (Appendix~J; Appendix~Q for DP ranges).
Finally, \(\mu\)-alignment is treated as a \emph{window-local guideline}: identification beyond ``Coincidence (verified)'' is not asserted without an explicit regression check and ledger-based tolerance rule recorded in \texttt{run.yaml}.




% =========================
\section*{Appendix S. Quantale Catalog [Spec; selection policy for the $\delta$-ledger \& $V$-metrics] (canon-aligned)}
% =========================
\phantomsection
\addcontentsline{toc}{section}{Appendix S. Quantale Catalog}
\refstepcounter{section}
\label{S:quantale}

\paragraph{Standing conventions (canon).}
All \emph{quantitative} measurements are evaluated \emph{after collapse}, i.e.\ in the fixed order
\[
\boxed{\ \text{for each } t\ \Longrightarrow\ \mathbf{P}_i\ \Longrightarrow\ \mathbf{T}_\tau\ \Longrightarrow\ \text{compare on } \mathbf{T}_\tau\mathbf{P}_i\ },
\]
using the same MECE, right-open windows and the same \(\tau\) as elsewhere (Chapter~1; Appendix~G/J/N).
Truncation \(\mathbf{T}_\tau\) is exact and \(1\)-Lipschitz for the interleaving metric \(d_{\mathrm{int}}\) (Appendix~A).
All defects are recorded in the \(\delta\)-ledger (Appendix~G) as \(V\)-valued entries (channels such as
\(\delta^{\mathrm{alg}},\delta^{\mathrm{disc}},\delta^{\mathrm{meas}},\delta^{\mathrm{spec}}\), and typed subchannels \(\delta^{\mathrm{Tr}},\delta^{\mathrm{Fun}},\Delta_{\mathrm{comm}}\) as used in Appendices~L/M/P/Q/R).
Unless explicitly declared otherwise (see \S\ref{S:sec:change-of-base}), a \emph{single base quantale} \(V\) is fixed for the run and used for all aggregation by \(\oplus\).
Finite definable covers on bounded windows (Appendix~H/J/Q) ensure that all joins and ledger aggregations used for runtime gating are \emph{finite}; countable covers are permitted only under P6/Summability with explicit logging (Appendix~J/Q).

% -------------------------
\subsection*{S.1. Implementable-range axioms (finite commutative quantales)}\label{S:sec:axioms}
% -------------------------
A \emph{(finite) commutative quantale} for this pipeline is a tuple \((V,\oplus,0,\preceq,\vee)\) such that:
\begin{enumerate}\itemsep0.35em
  \item[(Q0)] \((V,\preceq)\) is a poset admitting \emph{finite joins} \(\bigvee\) (we do \emph{not} assume completeness).
  \item[(Q1)] \((V,\oplus,0)\) is a commutative monoid; \(a\oplus 0=a\).
  \item[(Q2)] Monotonicity: \(a\preceq a'\), \(b\preceq b'\Rightarrow a\oplus b\preceq a'\oplus b'\).
  \item[(Q3)] Finite distributivity: \(a\oplus(b\vee c)=(a\oplus b)\vee(a\oplus c)\).
  \item[(Q4)] (Optional) scalar action \(\odot\) by a commutative monoid \(R\) (typically \(R=[0,1]\) or \(\RR_{\ge0}\)) such that
  \(r\odot(a\vee b)=(r\odot a)\vee(r\odot b)\) and \(r\odot(a\oplus b)=(r\odot a)\oplus(r\odot b)\).
  This is used for contraction factors \(\kappa\in(0,1]\) and confidence weights \(\omega\in[0,1]\) (Appendices~L/M/P/Q).
\end{enumerate}

\begin{definition}[\(V\)-Lawvere distance]
A \emph{Lawvere \(V\)-distance} on a set \(X\) is a map \(d_V:X\times X\to V\) with
\(0\preceq d_V(x,x)\) and the triangle inequality
\[
d_V(x,z)\ \preceq\ d_V(x,y)\ \oplus\ d_V(y,z).
\]
A map \(f:(X,d_V)\to(Y,d'_V)\) is \emph{\(V\)-non-expansive} if
\(d'_V(fx,fy)\preceq d_V(x,y)\) for all \(x,y\).
\end{definition}

\begin{remark}[From real metrics to \(V\)]
In many modules (e.g.\ persistence comparisons), a real-valued bound \(d_{\mathrm{int}}\le \varepsilon\) is converted to a \(V\)-bound by a declared monotone embedding \(\eta:\RR_{\ge0}\to V\) (Remark~\ref{S:rk:eta}).
This keeps the \emph{after-collapse} \(1\)-Lipschitz claims of \(\mathbf{T}_\tau\) valid after change of codomain:
if \(d_V:=\eta\circ d_{\mathrm{int}}\), then \(\mathbf{T}_\tau\) remains \(V\)-non-expansive (Appendix~A).
\end{remark}

% -------------------------
\subsection*{S.2. Catalog of canonical base quantales (all \textup{[Spec]})}\label{S:sec:catalog}
% -------------------------

\paragraph{S.2.1. Additive budget (\(V_{\mathrm{add}}\)).}
\[
V_{\mathrm{add}}:=([0,\infty],\ \oplus=+,\ 0,\ \preceq=\le,\ \vee=\max),
\qquad R=\RR_{\ge0},\ \ r\odot a:=r\cdot a.
\]
\emph{Intended semantics:} ledger entries are radii/budgets that accumulate; \(\oplus\) is literal addition.
\emph{Use when:} many small contributors must be accounted for (continuation radii, discretization/measurement tolerances, algebraic totals, Restart/Summability budgets; Appendix~J).

\paragraph{S.2.2. Worst-case budget (\(V_{\max}\)).}
\[
V_{\max}:=([0,\infty],\ \oplus=\max,\ 0,\ \preceq=\le,\ \vee=\max),
\qquad R=[0,1],\ \ r\odot a:=r\cdot a.
\]
\emph{Intended semantics:} acceptance is controlled by the largest spike; aggregation retains only the worst defect.
\emph{Use when:} a single dominant defect controls pass/fail (tightest overlap, maximal commutation defect, worst bin; sup-style audits).

\paragraph{S.2.3. Magnitude \(\times\) confidence (\(V_{\mathrm{prob}}\)).}
Let \(V_{\mathrm{prob}}:=[0,\infty]\times[0,1]\) with product order
\((x,c)\preceq(x',c')\iff x\le x'\ \wedge\ c\le c'\),
finite join \(\vee\) given coordinatewise by \((\max,\max)\), and monoid law
\[
(x,c)\ \oplus\ (y,d)\ :=\ (x+y,\ c\cdot d),
\qquad 0:=(0,1).
\]
Optionally take scalar action \(r\odot(x,c):=(r x,\ c)\) for \(r\in[0,1]\).
\emph{Intended semantics:} each defect carries a magnitude and an independent confidence/success weight; magnitudes add, confidences multiply.
\emph{Use when:} advisory or estimator outputs must be tracked explicitly (e.g.\ Appendix~P confidence \(\omega\), laboratory-style regression channels).
\emph{Guard-rail:} this is \textup{[Spec]} and should be used only when the independence/multiplicativity convention is declared in \texttt{run.yaml}.

\paragraph{S.2.4. Product quantales (multi-channel bases).}
Given base quantales \(V_1,V_2\), define
\[
V_1\times V_2,\qquad
(a_1,a_2)\oplus(b_1,b_2):=(a_1\oplus_1 b_1,\ a_2\oplus_2 b_2),
\]
ordered coordinatewise; finite joins are coordinatewise.
\emph{Use when:} two audit channels must be retained without collapsing information, e.g.\
\(V_{\mathrm{add}}\times V_{\max}\) (cumulative amount \emph{and} worst spike),
or \(V_{\mathrm{add}}\times V_{\mathrm{prob}}\) (budget with explicit confidence).

\begin{remark}[Embedding of numeric budgets]\label{S:rk:eta}
Arithmetic/control numbers (e.g.\ \(v_p|\ker|+v_p|\coker|\) in Appendix~R) and real-valued radii (e.g.\ \(\varepsilon\) in continuation bounds) are injected by a fixed monotone \(\eta:\RR_{\ge0}\to V\):
\[
\eta(x)=x\ \text{ for }V_{\mathrm{add}},V_{\max};\qquad
\eta(x)=(x,1)\ \text{ for }V_{\mathrm{prob}},
\]
and componentwise for products.
The choice of \(\eta\) is part of the \emph{base quantale declaration} in \texttt{run.yaml}.
\end{remark}

% -------------------------
\subsection*{S.3. Interoperability with the pipeline (after-collapse only)}\label{S:sec:interop}
% -------------------------
\begin{itemize}\itemsep0.35em
  \item \textbf{Ledger aggregation.}
  The \(\delta\)-ledger stores \(V\)-valued entries per window/degree and aggregates by \(\oplus\) to produce \(\Sigma\delta_W(i)\) (Appendix~G/J).
  Typed entries from Mirror/Transfer/Tropical \(2\)-cells are logged as \(\delta^{\mathrm{Tr}}\), \(\delta^{\mathrm{Fun}}\), or \(\delta_{\mathrm{trop}}\) and contribute to \(\Sigma\delta\) (Appendices~L/M/P/Q).
  \item \textbf{\(V\)-metrics and shifts.}
  A \(V\)-distance \(d_V\) induces \(V\)-shifts \(S^v\) as in Chapter~2; tests \(T_{\mathrm{Vshift}}\) enforce post-collapse commutation
  \(\mathbf{T}_\tau\circ S^v\simeq S^v\circ\mathbf{T}_\tau\) and \(V\)-nonexpansion on \(\mathbf{T}_\tau\mathbf{P}_i\).
  \item \textbf{Towers and diagnostics.}
  Tower comparisons and diagnostics \((\mu,\nu)\) are computed \emph{after} \(\mathbf{T}_\tau\) (Appendix~D/J).
  The quantale governs how residuals accumulate in Overlap Glue and Restart/Summability budgets (Appendix~J).
  \item \textbf{Arithmetic layers.}
  Finite kernel/cokernel control errors are embedded into \(V\) via \(\eta\) and charged as \(\delta_{\mathrm{ctrl}}\) or \(\delta^{\mathrm{alg}}\) depending on the interface (Appendix~R).
  PF/BC residuals are entered as \(\delta^{\mathrm{disc}},\delta^{\mathrm{meas}}\) (Appendix~N).
\end{itemize}

% -------------------------
\subsection*{S.4. Selection policy (run-level base; window-local objectives)}\label{S:sec:selection}
% -------------------------
\textbf{Policy.} Choose a \emph{base quantale} \(V\) \emph{per run} (not ad hoc per window), and record it in \texttt{run.yaml.quantale}.
Window-local objectives (safety vs.\ accounting vs.\ confidence) should be handled either by (i) selecting an appropriate \emph{product base} (e.g.\ \(V_{\mathrm{add}}\times V_{\max}\)), or (ii) applying a safe change-of-base morphism (see \S\ref{S:sec:change-of-base}) and logging the coercion.

\medskip
\noindent\textbf{Guidelines (objective \(\Rightarrow\) recommended base).}
\begin{enumerate}\itemsep0.25em
  \item \textbf{Safety-margin dominated} (\emph{``did anything spike?''}): use \(V_{\max}\) or include a \(\max\)-channel via \(V_{\mathrm{add}}\times V_{\max}\).
  \item \textbf{Cumulative risk accounting} (\emph{``how much budget remains?''}): use \(V_{\mathrm{add}}\).
  \item \textbf{Confidence bookkeeping} (advisory/estimator outputs): use \(V_{\mathrm{prob}}\) or \(V_{\mathrm{add}}\times V_{\mathrm{prob}}\).
  \item \textbf{Dual-view audits} (amount \emph{and} worst spike): use \(V_{\mathrm{add}}\times V_{\max}\).
\end{enumerate}

% -------------------------
\subsection*{S.5. Change of base (safe coercions)}\label{S:sec:change-of-base}
% -------------------------
A \emph{quantale morphism} \(\phi:(V,\oplus,0,\preceq)\to(W,\boxplus,0',\preceq')\) is a monotone map satisfying
\[
\phi(0)=0',
\qquad
\phi(a\oplus b)\ \preceq'\ \phi(a)\boxplus \phi(b).
\]
Such \(\phi\) is a \emph{safe coercion}: it can only weaken (never strengthen) budget inequalities.

\begin{example}[Standard safe coercions]\label{S:ex:coercions}
\[
V_{\mathrm{add}}\to V_{\max}:\ \phi(x)=x\quad(\text{since }x+y\ge \max\{x,y\}),
\qquad
V_{\mathrm{prob}}\to V_{\mathrm{add}}:\ \phi(x,c)=x.
\]
For product tracking:
\[
V\to V\times W:\ a\mapsto (a,\phi'(a)),
\]
where \(\phi'\) is any quantale morphism into \(W\).
\end{example}

\begin{proposition}[Safe coercion monotonicity]\label{S:prop:safe-coercion}
If \(\phi\) is a quantale morphism and \(\Sigma\delta\in V\) is a ledger total, then for any gate threshold \(\mathrm{gap}\in V\),
\[
\Sigma\delta\ \preceq\ \mathrm{gap}
\quad\Longrightarrow\quad
\phi(\Sigma\delta)\ \preceq'\ \phi(\mathrm{gap})
\]
in \(W\).
\end{proposition}

\begin{remark}[How to use change-of-base operationally]
If a run fixes \(V_{\mathrm{add}}\) but a report requires worst-case semantics, apply \(\phi:V_{\mathrm{add}}\to V_{\max}\) to the \emph{already computed} ledger totals and thresholds.
Do \emph{not} recompute budgets under a different aggregation rule without declaring a new base quantale and rerunning the ledger.
All coercions used for reporting must be listed under \texttt{quantale.change\_of\_base}.
\end{remark}

% -------------------------
\subsection*{S.6. Reproducibility keys (mandatory)}\label{S:sec:yaml}
% -------------------------
\begin{verbatim}
quantale:
  name: "V_add"                # V_add | V_max | V_prob | V_addxV_max | ...
  op: "+"                      # "+", "max", "(+,·)", "product"
  unit: "0"                    # "0" or "(0,1)" etc.
  order: "<="                  # coordinatewise when product
  join: "max"                  # finite join; coordinatewise if product
  scalar_action:
    enabled: true
    monoid: "[0,1]"            # or "R>=0"
    rule: "r ⊙ a"              # declared action; per-base
  eta_embed: "x -> x"          # or "x -> (x,1)" etc.
  change_of_base:
    - {to: "V_max", phi: "x -> x"}         # example safe coercion
tests:
  T_Vshift: true
  T_Vsubadd: true              # ledger aggregation law checks
  T_ChangeOfBase: true         # verifies listed φ are morphisms
\end{verbatim}

% -------------------------
\subsection*{S.7. Non-claims (scope limits)}\label{S:sec:nonclaims}
% -------------------------
We do not assume completeness nor infinite distributivity beyond finite joins (definable finiteness suffices on bounded windows).
No pre-collapse metric statements are made: all \(V\)-nonexpansion and \(V\)-Lipschitz claims are asserted only \emph{after} \(\mathbf{T}_\tau\).

% -------------------------
\subsection*{S.8. Integration map (where $V$ is consumed)}\label{S:sec:integration}
% -------------------------
Chapter~1 fixes the after-collapse evaluation order and ledger discipline.
Chapter~2 defines \(V\)-metrics and \(V\)-shifts and supplies \(T_{\mathrm{Vshift}}\).
Appendix~D/J compute tower diagnostics and pasting under Restart/Summability in the chosen \(V\).
Appendices~L/M/P/Q/R inject \(2\)-cell and arithmetic budgets via \(\eta\) and (optional) scalar actions \(\odot\).
Appendix~N supplies PF/BC residual channels \(\delta^{\mathrm{disc}},\delta^{\mathrm{meas}}\).
Appendix~G/Chapter~12 enforce \texttt{run.yaml} compliance and test execution.



% =========================
\section*{Appendix T. Implementation Notes / Notebooks [Spec; script skeletons, Gate Cascade, Convergence Manager, counterexample hunter, \& CI demos]}
% =========================
\phantomsection
\addcontentsline{toc}{section}{Appendix T. Implementation Notes / Notebooks}
\refstepcounter{section}
\label{T:notebooks}

\paragraph{Standing conventions (canon, enforced).}
All quantitative evaluations are \emph{after collapse} in the fixed order
\[
\boxed{\ \text{for each }t\ \Longrightarrow\ \mathbf{P}_i\ \Longrightarrow\ \mathbf{T}_\tau\ \Longrightarrow\ \text{compare on }\mathbf{T}_\tau\mathbf{P}_i\ },
\]
with \emph{one} declared window partition (MECE, right-open; Appendix~G/H/J/Q) and a \emph{single} \(\tau\) per comparison bundle.
All persistence modules are treated in the constructible range on bounded windows (Appendix~A/H); filtered (co)limits are computed objectwise under the scope policy and returned to the constructible range when stated.
\(\mathbf{T}_\tau\) is exact, idempotent, and \(1\)-Lipschitz (Appendix~A).
Deletion-type operations are the \emph{only} operations granted monotone (nonincreasing) indicator claims; inclusion-type operations are \emph{never} used for monotonicity, only for stability/nonexpansion when certified (Appendix~E).
All defects/budgets are written to a fixed commutative quantale \(V\) (Appendix~S) via a \(\delta\)-ledger with canonical channels
\[
\delta^{\mathrm{alg}},\quad \delta^{\mathrm{disc}},\quad \delta^{\mathrm{meas}},\quad \delta^{\mathrm{spec}},
\]
and optional layer tags (e.g.\ \(\delta^{\mathrm{Tr}},\delta^{\mathrm{Fun}},\delta^{\mathrm{Gal}}\)) that \emph{refine} \(\delta^{\mathrm{alg}}\) (Appendix~L/Q/R/S).
Arithmetic overlaps use Control\(\Rightarrow\)Overlap Gate, with finite kernel/cokernel parts embedded into \(V\) via a fixed monotone \(\eta:\mathbb{R}_{\ge0}\to V\) (Appendix~R/S).
Notebook templates below are \textbf{[Spec]} scaffolds intended to \emph{enable} Chapter~12 tests; they do not introduce new mathematical claims.
Gate Cascade uses only after-collapse objects; the Convergence Manager enforces Restart/Summability (Appendix~J) windowwise.

\begin{remark}[Key normalization (mandatory)]
To avoid drift between LaTeX notation and \texttt{run.yaml}, we use:
\[
\texttt{delta\^alg},\ \texttt{delta\^disc},\ \texttt{delta\^meas},\ \texttt{delta\^spec}
\]
as canonical keys (Appendix~G).
Layer tags use \texttt{delta\^Tr}, \texttt{delta\^Fun}, \texttt{delta\^Gal}, etc., and must be aggregated into \texttt{delta\^alg} or a declared \texttt{delta\_total} according to the chosen quantale \(V\) (Appendix~S).
\end{remark}

% -------------------------
\subsection*{T.1. Directory layout (minimal, reproducible)}
% -------------------------
\begin{verbatim}
akhdpst/
  run.yaml                   # manifest (Appendix G; single source of truth)
  data/                      # raw inputs (tropical/LMHS/p-adic/Selmer/etc.)
  cache/                     # intermediate artifacts (per window, per τ, per i)
  logs/                      # δ-ledger snapshots, audit trails, test outputs
  notebooks/
    01_stage_tropical.ipynb
    02_stage_lmhs.ipynb
    03_stage_padic.ipynb
    04_stage_arithmetic.ipynb
    05_convergence_manager.ipynb
    06_gate_cascade.ipynb
    07_counterexample_hunter.ipynb
    demo_gl1_min.ipynb
    demo_gl2_min.ipynb
  scripts/
    tau_sweep.py
    mece_check.py
    ledger_aggregate.py
    convergence_manager.py
    gate_cascade.py
    hunter_generate.py
    demo_gl1.py
    demo_gl2.py
  tests/
    test_vshift.py
    test_definable.py
    test_control_overlap.py
    test_gate_cascade.py
    test_convergence_manager.py
    test_hunter_regressions.py
\end{verbatim}

% -------------------------
\subsection*{T.2. Manifest fragments (per stage; consistent windows/\(\tau\)/\(i\))}
% -------------------------
Record mandatory keys (Appendix~G) with stage-local fields.
\emph{Window identifiers must match across all stages} (Appendix~G/J), and all stage outputs must be indexed by \((W,i,\tau)\).

\paragraph{T.2.1. Global: quantale, definability, windows, \(\tau\)-sweep}
\begin{verbatim}
quantale:
  name: "V_addxV_max"          # Appendix S
  op: "product"
  unit: "(0,0)"
  order: "coordinatewise"
  scalar_action: true
  eta_embed: "x->(x,x)"        # example; must match Appendix S choice
definable:
  structure: "R_an,exp"        # or "Denef-Pas" (Appendix Q)
  window_formulae:
    - "W1: a1 <= t < b1"
    - "W2: b1 <= t < b2"
windows:
  base: ["W1","W2"]            # MECE target; enforced by tests
degrees:
  i_list: [0,1,2]              # explicit degrees audited
tau:
  grid: {start: 0.0, stop: 3.0, step: 0.1}
tests:
  T_Vshift: true
  T_Definable: true
  T_Vsubadd: true
\end{verbatim}

\paragraph{T.2.2. Mirror/Tropical \(\to\) advisory LMHS (Appendix~P/L)}
\begin{verbatim}
tropical:
  contraction_kappa: 0.9
  bins: [0, 0.25, 0.5, 1.0]
  export: "cache/tropical_{W}_{i}_{tau}.json"
lmhs:
  proxies: ["rankN", "weights", "h_infty"]
  omega_policy: "heldout|band|zero_default"     # Appendix P
  export: "cache/lmhs_{W}_{i}_{tau}.json"
two_cell:
  bound_delta_trop_rule: "delta^Tr <= 0.05 per (W,i,tau)"  # Appendix P/L
\end{verbatim}

\paragraph{T.2.3. p-adic (Denef--Pas) transfer and arithmetic control (Appendix~Q/R)}
\begin{verbatim}
padic:
  structure: "Denef-Pas"
  field: "Q_p"
  p: 3
  uniformizer: "p"
  vg_scale: 1
  export: "cache/padic_{W}_{i}_{tau}.json"
iwasawa:
  tower: "cyclotomic"
  levels: {start: 0, end: 8}
  control_bounds:
    ker_vp_sup: 2
    coker_vp_sup: 3
  export: "cache/control_{W}_{i}_{tau}.json"
layered_delta:
  delta^Gal_rule: "eta(vpKer+vpCoker)"          # Appendix R/S
\end{verbatim}

\paragraph{T.2.4. Convergence Manager (Restart/Summability; Appendix~J)}
\begin{verbatim}
convergence:
  restart_policy:
    kappa: 1.0
    max_restarts_per_window: 2
  summability:
    tol_add_base: 1e-6          # tolerance in a declared V_add base (Appendix S)
    max_iters: 20
  stable_band_scan:
    neighborhood: 0.02
    min_band_width: 0.05
\end{verbatim}

\paragraph{T.2.5. Gate Cascade (after-collapse only; Appendix~J/N/R)}
\begin{verbatim}
gate_cascade:
  order: ["B_GatePlus", "PF_BC", "OverlapGate"]
  B_GatePlus:
    gap_key: "gap_tau"          # measured after T_tau on the same (W,i,tau)
  PF_BC:
    enforce: true
    budget_keys: ["delta^disc", "delta^meas"]
  OverlapGate:
    enforce: true               # Appendix R: control ⇒ overlap
    budget_key: "delta^Gal"
\end{verbatim}

\paragraph{T.2.6. Counterexample hunter (regression generators; Appendix~D/E/L/Q)}
\begin{verbatim}
hunter:
  generators:
    - "typeIV_accumulation"     # near-τ pileup (Appendix D)
    - "mirror_comm_defect"      # large 2-cell delta^Tr (Appendix L/P)
    - "inclusion_spike"         # inclusion-type stress (Appendix E)
  budget_caps:
    delta^Tr: 0.2
    delta^Fun: 0.2
    delta^Gal: 5
  export: "cache/hunter_{W}_{i}_{tau}.json"
\end{verbatim}

\paragraph{T.2.7. CI demos (GL(1)/GL(2) minimal workflows; non-gating)}
\begin{verbatim}
demo:
  gl1:
    tower: "cyclotomic"
    levels: {start: 0, end: 5}
    class_module: "toy_surrogate"       # explicitly non-analytic placeholder
  gl2:
    curve: "E: y^2 = x^3 - x"
    levels: {start: 0, end: 4}
    selmer_p: 3
\end{verbatim}

% -------------------------
\subsection*{T.3. \texorpdfstring{$\tau$}{tau}-sweep driver (skeleton; canon order enforced)}
% -------------------------
\begin{verbatim}
# scripts/tau_sweep.py  (pseudocode; after-collapse policy enforced)
from akhdpst import pipeline

cfg = pipeline.load_manifest("run.yaml")

for W in cfg.windows.base:
  cells = pipeline.definable_cells(W, cfg)          # finite by App. H/J/Q
  for i in cfg.degrees.i_list:
    for tau in pipeline.grid(cfg.tau.grid):

      # (0) Initialize per-(W,i,tau) ledger frame
      pipeline.ledger.begin_frame(W=W, i=i, tau=tau)

      # (1) Mirror/Tropical stage: record only 2-cell / commutation budgets
      trop = pipeline.tropical_readout(W, i, tau, cfg)
      pipeline.ledger.add("delta^Tr", W, i, tau, trop.delta_Tr)   # in V

      # (2) LMHS proxies: advisory only (never a gate)
      lmhs = pipeline.lmhs_proxy(trop, cfg)
      # If a proposal yields a nonzero advisory term, store as delta^spec
      pipeline.ledger.add("delta^spec", W, i, tau, lmhs.delta_spec)
      pipeline.cache.save(lmhs, f"cache/lmhs_{W}_{i}_{tau}.json")

      # (3) p-adic transfer / functorial comparison (Spec): record 2-cell budget
      pad = pipeline.padic_readout(W, i, tau, cfg)
      pipeline.ledger.add("delta^Fun", W, i, tau, pad.delta_Fun)

      # (4) Canon object: compute persistence, then collapse, then compare
      #     PiT means T_tau(P_i(F|_W)) in Pers^{ft}.
      PiT = pipeline.read_Pi_then_Ttau(W, i, tau, cfg)

      # (5) Certified tests triggered by advisory hints (Appendix P/Q):
      E1 = pipeline.energy_bins_after_collapse(PiT, cfg)
      if E1.is_zero():
        pipeline.bridge_certify_after_collapse(PiT, W, i, tau, cfg)  # PH1=>Ext1

      # (6) Arithmetic control / overlap budgets (Appendix R):
      ctl = pipeline.control_bounds(W, i, tau, cfg)
      pipeline.ledger.add("delta^Gal", W, i, tau, ctl.delta_Gal)

      pipeline.ledger.end_frame(W=W, i=i, tau=tau)

pipeline.ledger.flush("logs/ledger.json")
\end{verbatim}

% -------------------------
\subsection*{T.4. Convergence Manager (Restart \& stability bands; Appendix~J)}
% -------------------------
All convergence decisions are made \emph{after collapse} and must be compatible with Restart/Summability (Appendix~J).
Because a general quantale may not carry a numeric norm, tolerances are evaluated in a declared additive base via a logged change-of-base morphism (Appendix~S).

\begin{verbatim}
# scripts/convergence_manager.py (pseudocode; Appendix J)
from akhdpst import pipeline, quantale

cfg = pipeline.load_manifest("run.yaml")

κ    = cfg.convergence.restart_policy.kappa
tol  = cfg.convergence.summability.tol_add_base
imax = cfg.convergence.summability.max_iters

# Coercion to an additive base for tolerance checks (logged in run.yaml)
phi_to_add = quantale.change_of_base(cfg.quantale, target="V_add")

def iterate_until_stable(W, i, tau0):
  tau = tau0
  prev = None

  for it in range(imax):
    PiT = pipeline.read_Pi_then_Ttau(W, i, tau, cfg)
    gap = pipeline.gap_tau(PiT, i, tau, cfg)                  # B-Gate^+ margin
    Σδ  = pipeline.ledger.window_total(W, i, tau, cfg)        # ⊕-sum in V

    if quantale.lt(Σδ, gap, cfg.quantale):                    # strict pass
      band = pipeline.find_stability_band(W, i, tau, cfg.convergence.stable_band_scan, cfg)
      return {"status":"passed", "W":W, "i":i, "tau":tau, "band":band,
              "delta_total":Σδ, "gap":gap}

    # Restart update (Appendix J): adjust tau within a declared policy
    tau = pipeline.restart_update(tau, κ, cfg)

    # Tolerance check in additive base (Spec; operational stop)
    if prev is not None:
      a_now  = phi_to_add(Σδ)
      a_prev = phi_to_add(prev)
      if abs(a_now - a_prev) <= tol:
        break
    prev = Σδ

  return {"status":"max_iters", "W":W, "i":i, "tau":tau, "delta_total":prev}

for W in cfg.windows.base:
  for i in cfg.degrees.i_list:
    info = iterate_until_stable(W, i, tau0=cfg.tau.grid.start)
    pipeline.cache.save(info, f"cache/convergence_{W}_{i}.json")
\end{verbatim}

% -------------------------
\subsection*{T.5. Gate Cascade (B-Gate$^{+}\to$ PF/BC $\to$ Overlap; after-collapse only)}
% -------------------------
The Gate Cascade is a \emph{decision pipeline} on a fixed \((W,i,\tau)\).
All checks read only \(\mathbf{T}_\tau\mathbf{P}_i\) objects and consult only ledger totals in \(V\) (Appendix~J/N/R/S).

\begin{verbatim}
# scripts/gate_cascade.py (pseudocode; Appendix J/N/R; after-collapse)
from akhdpst import pipeline, quantale

def run_cascade(W, i, tau, cfg):
  PiT = pipeline.read_Pi_then_Ttau(W, i, tau, cfg)
  gap = pipeline.gap_tau(PiT, i, tau, cfg)                     # after-collapse
  Σδ  = pipeline.ledger.window_total(W, i, tau, cfg)

  # 1) B-Gate^+ (Appendix J): accept only if gap > Σδ
  if not quantale.lt(Σδ, gap, cfg.quantale):
    return {"stage":"B_GatePlus", "pass":False, "W":W, "i":i, "tau":tau,
            "gap":gap, "delta_total":Σδ}

  # 2) PF/BC after-collapse comparators (Appendix N)
  pfbc_ok, drift = pipeline.pfbc_check_after_collapse(PiT, W, i, tau, cfg)
  pipeline.ledger.add("delta^disc", W, i, tau, drift.delta_disc)
  pipeline.ledger.add("delta^meas", W, i, tau, drift.delta_meas)
  if not pfbc_ok:
    return {"stage":"PF_BC", "pass":False, "drift":drift}

  # 3) Overlap Gate after-collapse (Appendix R): control ⇒ overlap
  ov_ok, ctl = pipeline.overlap_gate_after_collapse(W, i, tau, cfg)
  pipeline.ledger.add("delta^Gal", W, i, tau, ctl.delta_Gal)
  return {"stage":"OverlapGate", "pass":ov_ok, "control":ctl, "W":W, "i":i, "tau":tau}

# notebook cell (06_gate_cascade.ipynb):
# res = run_cascade("W1", 1, tau, cfg); save_json(res, ...)
\end{verbatim}

% -------------------------
\subsection*{T.6. Counterexample hunter (adversarial generators; regression-only)}
% -------------------------
The hunter is a \emph{stress testing} tool: it generates bounded-window instances designed to trigger known failure modes (Appendix~D/E/L/Q).
It never produces certificates; it only produces failing examples and associated ledger traces (Appendix~G/J).

\begin{verbatim}
# scripts/hunter_generate.py (pseudocode; App. D/E/L/Q)
from akhdpst import pipeline

def gen_typeIV_accumulation(W, i, tau, density=50):
  # near-τ bar-length pileup (Type IV pattern), finite on bounded windows
  return pipeline.synthetic.typeIV_pileup(W, i, tau, density=density)

def gen_mirror_comm_defect(W, i, tau, target=0.15):
  # large Mirror↔Collapse (or Mirror↔C_tau) 2-cell defect (Appendix L/P)
  return pipeline.synthetic.mirror_defect(W, i, tau, target_delta=target)

def gen_inclusion_spike(W, i, tau, factor=2.0):
  # inclusion-type amplification: should break deletion-only monotonicity guards
  return pipeline.synthetic.inclusion_spike(W, i, tau, factor=factor)

def hunt_all(W, i, tau, cfg):
  cases = []
  for name, gen in [("typeIV", gen_typeIV_accumulation),
                    ("mirror", gen_mirror_comm_defect),
                    ("incl", gen_inclusion_spike)]:
    F = gen(W, i, tau)
    PiT = pipeline.to_after_collapse(F, W, i, tau, cfg)     # Pi then T_tau
    report = pipeline.run_full_audit(PiT, W, i, tau, cfg)   # gate + ledger trace
    cases.append({"name":name, "report":report})
    pipeline.cache.save(report, f"cache/hunter_{name}_{W}_{i}_{tau}.json")
  return cases
\end{verbatim}

% -------------------------
\subsection*{T.7. MECE window coverage check (test hook; Appendix~G/H/J/Q)}
% -------------------------
\begin{verbatim}
# scripts/mece_check.py (pseudocode)
from akhdpst import windows

cfg = windows.load_manifest("run.yaml")
assert windows.is_right_open(cfg.windows.base)
assert windows.is_mece(cfg.windows.base)
assert windows.covers_target(cfg.windows.base, cfg.definable)
windows.log_partition(cfg, out="logs/windows_partition.txt")
\end{verbatim}

% -------------------------
\subsection*{T.8. \texorpdfstring{$\delta$}{delta}-aggregation (per window, per degree, per \texorpdfstring{$\tau$}{tau})}
% -------------------------
Aggregation must respect the declared quantale \(V\) (Appendix~S) and the canonical channels.
Layer tags refine \(\delta^{\mathrm{alg}}\); advisory terms remain \(\delta^{\mathrm{spec}}\).

\begin{verbatim}
# scripts/ledger_aggregate.py (pseudocode; Appendix S semantics)
from akhdpst import ledger, quantale

cfg = ledger.load_manifest("run.yaml")
V   = quantale.load(cfg.quantale)

L = ledger.load("logs/ledger.json")

for (W,i,tau) in L.frames():
  # layer tags (optional)
  dTr  = L.get("delta^Tr",  W,i,tau, default=V.zero())
  dFun = L.get("delta^Fun", W,i,tau, default=V.zero())
  dGal = L.get("delta^Gal", W,i,tau, default=V.zero())

  # canonical channels
  dDisc = L.get("delta^disc", W,i,tau, default=V.zero())
  dMeas = L.get("delta^meas", W,i,tau, default=V.zero())
  dSpec = L.get("delta^spec", W,i,tau, default=V.zero())

  # algebraic total (refinement convention)
  dAlg = V.op(dTr, V.op(dFun, dGal))

  # window total
  dTot = V.op(dAlg, V.op(dDisc, V.op(dMeas, dSpec)))
  L.store("delta_total", W,i,tau, dTot)

ledger.write(L, "logs/ledger_aggregated.json")
\end{verbatim}

% -------------------------
\subsection*{T.9. Notebook cell skeletons (per stage; after-collapse inputs only)}
% -------------------------
Each notebook cell is required to:
(i) read \texttt{run.yaml} as the source of truth,
(ii) produce cache artifacts indexed by \((W,i,\tau)\),
(iii) emit ledger entries in \(V\) using canonical keys.

\paragraph{T.9.1. 01\_stage\_tropical.ipynb}
\begin{verbatim}
cfg = load_yaml("run.yaml")
W,i,tau = "W1", 1, cfg.tau.grid.start
trop = tropical_readout(W,i,tau,cfg)
ledger.add("delta^Tr", W,i,tau, trop.delta_Tr)
save_json(trop, f"cache/tropical_{W}_{i}_{tau}.json")
\end{verbatim}

\paragraph{T.9.2. 02\_stage\_lmhs.ipynb (advisory only)}
\begin{verbatim}
trop = load_json(f"cache/tropical_{W}_{i}_{tau}.json")
lmhs = lmhs_proxy(trop, cfg.lmhs.proxies)
ledger.add("delta^spec", W,i,tau, lmhs.delta_spec)   # may be zero by policy
save_json(lmhs, f"cache/lmhs_{W}_{i}_{tau}.json")
\end{verbatim}

\paragraph{T.9.3. 03\_stage\_padic.ipynb}
\begin{verbatim}
pad = padic_readout(W,i,tau, field=cfg.padic.field, p=cfg.padic.p)
ledger.add("delta^Fun", W,i,tau, pad.delta_Fun)
save_json(pad, f"cache/padic_{W}_{i}_{tau}.json")
\end{verbatim}

\paragraph{T.9.4. 04\_stage\_arithmetic.ipynb (control / overlap)}
\begin{verbatim}
ctl = control_bounds(W,i,tau,cfg)              # Appendix R
ledger.add("delta^Gal", W,i,tau, ctl.delta_Gal)
PiT = read_Pi_then_Ttau(W,i,tau,cfg)           # after-collapse object
E1  = energy_bins_after_collapse(PiT,cfg)
if E1.is_zero(): bridge_certify_after_collapse(PiT,W,i,tau,cfg)
\end{verbatim}

\paragraph{T.9.5. 05\_convergence\_manager.ipynb}
\begin{verbatim}
info = iterate_until_stable(W,i,tau0=cfg.tau.grid.start)
save_json(info, f"cache/convergence_{W}_{i}.json")
\end{verbatim}

\paragraph{T.9.6. 06\_gate\_cascade.ipynb}
\begin{verbatim}
res = run_cascade(W,i,tau,cfg)
save_json(res, f"cache/gate_{W}_{i}_{tau}.json")
\end{verbatim}

\paragraph{T.9.7. 07\_counterexample\_hunter.ipynb}
\begin{verbatim}
cases = hunt_all(W,i,tau,cfg)
save_json(cases, f"cache/hunter_{W}_{i}_{tau}.json")
\end{verbatim}

% -------------------------
\subsection*{T.10. CI/test integration (Chapter~12)}
% -------------------------
\begin{verbatim}
tests:
  - name: "T_Vshift"
    script: "pytest tests/test_vshift.py"
  - name: "T_Definable"
    script: "python scripts/mece_check.py"
  - name: "T_Iwasawa"
    script: "pytest tests/test_control_overlap.py"
  - name: "T_GateCascade"
    script: "pytest tests/test_gate_cascade.py"
  - name: "T_Convergence"
    script: "pytest tests/test_convergence_manager.py"
  - name: "T_Hunter"
    script: "pytest tests/test_hunter_regressions.py"
  - name: "Demo_GL1"
    script: "python scripts/demo_gl1.py"
  - name: "Demo_GL2"
    script: "python scripts/demo_gl2.py"
artifacts:
  - "logs/ledger.json"
  - "logs/ledger_aggregated.json"
  - "logs/windows_partition.txt"
  - "cache/convergence_W1_1.json"
\end{verbatim}

% -------------------------
\subsection*{T.11. GL(1) / GL(2) minimal workflows (non-gating demos; [Spec])}
% -------------------------
These demos are \emph{integration} checks for the pipeline contracts (windowing, ledger, gates).
They are explicitly not claims of number-theoretic correctness beyond logged budgets and after-collapse tests.

\paragraph{T.11.1. GL(1) (cyclotomic class-module \emph{surrogate}).}
\begin{verbatim}
# scripts/demo_gl1.py (pseudocode)
from akhdpst import pipeline

cfg = pipeline.load_manifest("run.yaml")
W,i,tau = "W1", 1, cfg.tau.grid.start

ctl = pipeline.demo.gl1_control_surrogate(cfg.demo.gl1)
pipeline.ledger.add("delta^Gal", W,i,tau, ctl.delta_Gal)

PiT = pipeline.read_Pi_then_Ttau(W,i,tau,cfg)
gap = pipeline.gap_tau(PiT,i,tau,cfg)
Σδ  = pipeline.ledger.window_total(W,i,tau,cfg)

print("GL(1) demo:", "PASS" if Σδ < gap else "FAIL")
\end{verbatim}

\paragraph{T.11.2. GL(2) (toy elliptic curve workflow; [Spec]).}
\begin{verbatim}
# scripts/demo_gl2.py (pseudocode)
from akhdpst import pipeline

cfg = pipeline.load_manifest("run.yaml")
W,i,tau = "W1", 1, cfg.tau.grid.start

sel = pipeline.demo.gl2_selmer_surrogate(cfg.demo.gl2)
pipeline.ledger.add("delta^Gal", W,i,tau, sel.delta_Gal)

PiT = pipeline.read_Pi_then_Ttau(W,i,tau,cfg)
gap = pipeline.gap_tau(PiT,i,tau,cfg)
Σδ  = pipeline.ledger.window_total(W,i,tau,cfg)

print("GL(2) demo:", "PASS" if Σδ < gap else "FAIL")
\end{verbatim}

% -------------------------
\subsection*{T.12. Invariants enforced by templates (audit contract)}
% -------------------------
\begin{itemize}\itemsep3pt
  \item \textbf{After-collapse only.} No comparator reads pre-collapse metrics; all distances are computed on \(\mathbf{T}_\tau\mathbf{P}_i\).
  \item \textbf{Definable finiteness.} All loops iterate over finitely many definable cells/events on bounded windows (Appendix~H/J/Q).
  \item \textbf{Quantale compliance.} Ledger aggregation uses the declared \(\oplus\) in \(V\) with optional, logged change-of-base morphisms (Appendix~S).
  \item \textbf{Gate separation.} LMHS/tropical proxies never gate alone; they can only trigger certified after-collapse tests (Appendix~P).
  \item \textbf{Control accounting.} Finite kernel/cokernel contributions are embedded via \(\eta\) and recorded as algebraic budgets (Appendix~R/S).
  \item \textbf{Restart/Summability.} Convergence Manager applies Appendix~J (Restart + P6) and records stability bands \emph{as data} (Appendix~J/M).
\end{itemize}

% -------------------------
\subsection*{T.13. Minimal run example (end-to-end; reproducible)}
% -------------------------
\begin{verbatim}
# 1) Fill run.yaml per T.2 and choose quantale per Appendix S.
# 2) Validate MECE coverage:
python scripts/mece_check.py
# 3) Execute τ-sweep (cache + ledger):
python scripts/tau_sweep.py
# 4) Aggregate ledger totals in V:
python scripts/ledger_aggregate.py
# 5) Convergence Manager:
python scripts/convergence_manager.py
# 6) Gate Cascade per (W,i,tau):
python scripts/gate_cascade.py
# 7) (Optional) Counterexample Hunter:
python scripts/hunter_generate.py
# 8) Run CI tests (Chapter 12):
pytest -q
\end{verbatim}

% -------------------------
\subsection*{T.14. Non-claims (scope guard)}
% -------------------------
These templates assert no pre-collapse monotonicity, no commutation beyond recorded \(2\)-cell bounds, no global \(\mathrm{PH}_1\Leftrightarrow\Ext^1\), and no analytic uniformity beyond logged \(\delta\)-budgets.
All guarantees are confined to bounded definable windows with after-collapse measurements, deletion-type-only monotonicity, and the quantale semantics of Appendix~S.



% ==========================================================================
% Appendix U : AI Agent Specifications (Hunter / Mapper / Lifter)
% ==========================================================================
\section*{Appendix U. AI Agent Specifications (Hunter / Mapper / Lifter) [Spec]}
\phantomsection
\addcontentsline{toc}{section}{Appendix U. AI Agent Specifications (Hunter / Mapper / Lifter)}
\refstepcounter{section}
\label{U:agents}

% --------------------------------------------------------------------------
\subsection*{U.0. Purpose and relation to Part II (Spec-only, auditable search)}
% --------------------------------------------------------------------------
This appendix specifies the operational semantics of the AI agents introduced
in Part~II (Chapters~\ref{sec:ch14}--\ref{sec:ch17}).
All statements herein are classified as \textbf{[Spec]}: they define a
\emph{reproducible, auditable search-and-assembly protocol} for producing
candidate Proof Objects (Part~II), without extending the proven collapse
theorems of Part~I.

\paragraph{Canon constraints (mandatory).}
Throughout Part~II, all quantitative decisions are evaluated \emph{after collapse} in the fixed order
\[
\boxed{\ \text{for each }t\ \Longrightarrow\ \mathbf{P}_i\ \Longrightarrow\ \mathbf{T}_\tau\ \Longrightarrow\ \text{compare in }\mathsf{Pers}^{\mathrm{cons}}_k\ }.
\]
All time windows are right-open and MECE on the target domain (App.~G/H/J/Q).
No pre-collapse comparison is in scope.

\paragraph{\(\delta\)-ledger discipline (mandatory).}
All defects/budgets are recorded in a fixed commutative quantale \(V\) (App.~S) via the \(\delta\)-ledger
(App.~L/S/T) with the canonical decomposition
\[
\delta \ =\ \delta^{\mathrm{alg}}\ \oplus\ \delta^{\mathrm{disc}}\ \oplus\ \delta^{\mathrm{meas}}
\qquad(\text{canon; App.~L/S}).
\]
Advisory/heuristic uncertainty introduced by agents (e.g.\ gradient estimates, surrogate models) must be recorded
as a \emph{tagged} contribution within \(\delta^{\mathrm{meas}}\) (default) or \(\delta^{\mathrm{disc}}\) when appropriate,
and may additionally be reported under a non-gating label \(\delta^{\mathrm{spec}}\) for transparency.
Layer tags may refine \(\delta^{\mathrm{alg}}\) (e.g.\ \(\delta^{\mathrm{Gal}},\delta^{\mathrm{Tr}},\delta^{\mathrm{Fun}},\delta^{\mathrm{lift}}\); App.~G/L/S/T).

\paragraph{Proposer--Verifier separation (mandatory).}
Agents in Part~II act only as \emph{Proposers}. All certified acceptance/rejection is determined solely by
after-collapse Gates/tests and their logged budgets (Ch.~11/12; App.~J/N/R/T).
Advisory modules (e.g.\ tropical/LMHS proxies) are \emph{never} Gates.

\begin{remark}[Terminology: terrain cells vs.\ time windows]
To avoid collisions:
\begin{itemize}\itemsep0.25em
  \item A \emph{Terrain Cell} is a definable subset \(\mathcal{C}_\alpha\subset \mathcal{M}\) of the \emph{parameter space} \(\mathcal{M}\) navigated by agents (Part~II).
  \item A \emph{time window} is a right-open definable subset \(W\subset\mathbb{R}\) (o-minimal) or \(W\subset\mathrm{VG}\) (Denef--Pas) used for after-collapse persistence comparisons (App.~H/J/Q).
\end{itemize}
All Gate checks are evaluated on time windows \(W\) (after collapse), while the Mapper assembles coverage over terrain cells \(\mathcal{C}_\alpha\).
\end{remark}

% --------------------------------------------------------------------------
\subsection*{U.1. Agent taxonomy, state, and shared objects}
% --------------------------------------------------------------------------

\begin{definition}[Agent types and roles]\label{U:def:roles}
We distinguish three classes of autonomous agents with distinct responsibilities:
\begin{itemize}\itemsep0.25em
  \item \textbf{Hunter (H):} local navigator/optimizer on the parameter space \(\mathcal{M}\).
  It explores within a Terrain Cell \(\mathcal{C}_\alpha\) to reduce a scalar \emph{Defect Potential} \(\Phi_\tau\)
  computed from \emph{after-collapse} ledger/test readouts.
  \item \textbf{Mapper (M):} global assembler.
  It verifies local certificates produced by Hunters and stitches them into a Coverage Graph
  via Overlap Gates.
  \item \textbf{Lifter (L):} singularity handler.
  It is invoked when a Hunter encounters a Type~IV obstruction (diagnostics \((\mu,\nu)\neq(0,0)\), App.~D)
  and attempts a controlled dimensional extension subject to a \emph{lifting penalty} recorded in the \(\delta\)-ledger.
\end{itemize}
\end{definition}

\begin{definition}[Shared manifest and reproducibility anchors]\label{U:def:manifest}
All agents operate under a single manifest \texttt{run.yaml} (App.~G) which pins:
(i) the quantale \(V\) and scalarization policy \(\pi\) (App.~S),
(ii) the time-window partition (MECE, right-open) and definability mode (App.~H/J/Q),
(iii) the \(\tau\)-policy and degree list \(\{i\}\),
(iv) Gate order and enabled tests (Ch.~12; App.~J/N/R/T),
(v) deterministic numeric policy and serialization policy (App.~G).
Any agent output is valid as a Proof Object component only if it references the manifest hash
(\texttt{run\_yaml\_hash} / content hash, App.~G).
\end{definition}

\begin{definition}[Defect Potential \(\Phi_\tau\) (scalarization; Spec)]\label{U:def:phi}
Fix a monotone scalarization \(\pi:V\to \mathbb{R}_{\ge0}\cup\{\infty\}\) logged in \texttt{run.yaml}
(App.~G/S). For a parameter point \(x\in\mathcal{M}\), define \(\Phi_\tau(x)\) from after-collapse windowwise budgets as
\[
\Phi_\tau(x)\ :=\ \max_{(W,i)\in\mathcal{W}_{\mathrm{time}}\times I}\ 
\Big(\ \pi(\Sigma\delta_{W}(i;x))\ -\ \pi(\mathrm{gap}_\tau(i;W,x))\ \Big)_{+},
\]
where \(\Sigma\delta_W(i;x)\in V\) is the ledger total on \((W,i)\) at \(x\) (App.~J/S/T),
\(\mathrm{gap}_\tau(i;W,x)\in V\) is the B-Gate\(^{+}\) threshold (App.~J),
and \(\mathcal{W}_{\mathrm{time}}\) denotes the finite MECE window family declared in \texttt{run.yaml}.
Thus \(\Phi_\tau(x)=0\) implies that all required windowwise B-Gate\(^{+}\) inequalities hold (after collapse).
\end{definition}

\begin{definition}[Gradient estimates are Spec objects]\label{U:def:grad-as-spec}
Any occurrence of a “gradient” in Part~II means a \emph{logged, reproducible estimate}
\(\widehat{\nabla}\Phi_\tau(x)\) produced by the Gradient Oracle specified in Chapter~13
(Spec.~\ref{spec:13-grad-oracle}) under the manifest block \texttt{grad\_policy} (App.~G).
Unlogged gradients are invalid and must not be used for any decision trace.
\end{definition}

\begin{definition}[Hunter state tuple]\label{U:def:Hstate}
A Hunter at step \(k\) maintains the state
\[
S_k^{\mathrm{H}} \ :=\ \bigl(x_k,\ \mathcal{C}_k,\ \Phi_\tau(x_k),\ \widehat{g}_k,\ \mathsf{ctx}_k\bigr),
\]
where \(x_k\in\mathcal{M}\), \(\mathcal{C}_k\) is the containing Terrain Cell,
\(\Phi_\tau(x_k)\) is as in Def.~\ref{U:def:phi},
\(\widehat{g}_k\) is an \emph{advisory} gradient estimate or surrogate direction (Def.~\ref{U:def:grad-as-spec}),
and \(\mathsf{ctx}_k\) contains the pinned \((\tau,i)\)-bundle, window IDs, and manifest hash.
\end{definition}

\begin{definition}[Mapper state tuple]\label{U:def:Mstate}
The Mapper maintains
\[
S^{\mathrm{M}} \ :=\ \bigl(\mathcal{T}_{\mathrm{valid}},\ \mathcal{G},\ \mathsf{ctx}\bigr),
\]
where \(\mathcal{T}_{\mathrm{valid}}\) is the set of Terrain Cells \(\mathcal{C}_\alpha\) with verified certificates,
\(\mathcal{G}=(\mathcal{V},\mathcal{E})\) is the Coverage Graph with \(\mathcal{V}=\mathcal{T}_{\mathrm{valid}}\),
and \(\mathsf{ctx}\) pins the manifest hash and test policy.
\end{definition}

\begin{definition}[Lifter state tuple]\label{U:def:Lstate}
A Lifter instantiated at a singularity maintains
\[
S^{\mathrm{L}} \ :=\ \bigl(x_{\mathrm{sing}},\ \mathcal{C}_{\mathrm{sing}},\ \Phi_\tau(x_{\mathrm{sing}}),\ (\mu,\nu),\ \ell,\ \mathsf{ctx}\bigr),
\]
where \((\mu,\nu)\) are the after-collapse tower diagnostics (App.~D),
\(\ell\in\mathbb{Z}_{\ge0}\) is the lifting depth (number of auxiliary axes introduced),
and \(\mathsf{ctx}\) includes the manifest hash and the current ledger snapshot ID.
\end{definition}

% --------------------------------------------------------------------------
\subsection*{U.2. Hunter protocol (operational semantics; Spec)}
% --------------------------------------------------------------------------

\begin{specification}[Hunter action semantics]\label{U:spec:hunter}
At any state \(S_k^{\mathrm{H}}\), the Hunter must select exactly one action from the list below.
\emph{All certified decisions are based only on after-collapse readouts} (\(\mathbf{T}_\tau\mathbf{P}_i\))
and logged ledger totals (App.~J/S/T). Advisory computations must be fully logged (Sec.~\ref{sec:U-log}).

\begin{itemize}\itemsep0.35em
  \item \texttt{grad\_estimate}:
  Query the Gradient Oracle (Spec.~\ref{spec:13-grad-oracle}) at \(x_k\) and produce
  \(\widehat{g}_k=\widehat{\nabla}\Phi_\tau(x_k)\), together with an estimated variance (or certified upper bound).
  The estimation error must be charged to the \(\delta\)-ledger (default: \(\delta^{\mathrm{meas}}\) with tag \(\delta^{\mathrm{spec}}\)).
  This action is advisory and cannot certify validity.

  \item \texttt{gradient\_step}:
  If in a \emph{Ridge} regime (heuristically, \(\Phi_\tau(x_k)>0\) and a usable estimate \(\widehat{g}_k\neq 0\) exists),
  propose an update \(x_{k+1}=\mathrm{Update}(x_k,\widehat{g}_k)\) constrained to remain inside \(\mathcal{C}_k\).
  The proposal must log step size, projection/clipping to \(\mathcal{C}_k\), and the acceptance/rejection rule.
  This action is advisory.

  \item \texttt{restart}:
  If descent stagnates (e.g.\ \(\|\widehat{g}_k\|\) below a logged tolerance) while \(\Phi_\tau(x_k)>0\),
  invoke Restart Logic (App.~J; App.~T) to refine the current Terrain Cell
  \(\mathcal{C}_k\) into subcells \(\{\mathcal{C}'\}\) (definable refinement; finite subdivision policy logged).
  Restart is a search operation and does not create certificates.

  \item \texttt{validate}:
  If in a \emph{Plain} regime (\(\Phi_\tau(x_k)=0\)), request a certified Gate Cascade on the relevant
  time-window bundle \((W,i,\tau)\) (App.~T):
  \[
  \text{B-Gate}^{+}\ \to\ \text{PF/BC}\ \to\ \text{Overlap Gate}.
  \]
  If the cascade passes under the declared budgets, then the containing Terrain Cell \(\mathcal{C}_k\)
  may be flagged \texttt{valid} and packaged as a Local Certificate for the Mapper
  (Def.~\ref{U:def:localcert}).

  \item \texttt{escalate}:
  If in a \emph{Peak} regime (e.g.\ \(\Phi_\tau(x_k)\ge \lambda_{\mathrm{sing}}\) for a logged threshold)
  \emph{or} if Type~IV diagnostics are detected, i.e.
  \[
    (\mu,\nu)\neq(0,0)\qquad\text{(Type IV; App.~D; canon: nonzero means obstruction),}
  \]
  halt local navigation and invoke the Lifter (Sec.~\ref{sec:U-lifter}).
\end{itemize}

\paragraph{Logging discipline.}
Every executed action appends a linked entry to the Action Log (Sec.~\ref{sec:U-log})
including: manifest hash, RNG seed/state, \((W,i,\tau)\)-bundle IDs, ledger snapshot IDs,
diagnostics \((\mu,\nu)\) when available, and hashes of referenced artifacts/certificates (App.~G).
For \texttt{grad\_estimate}, the tuple \texttt{(method, stencil, seed, variance)} is mandatory.
\end{specification}

\begin{definition}[Local certificate (Hunter output; auditable)]\label{U:def:localcert}
A Local Certificate attached to a Terrain Cell \(\mathcal{C}_\alpha\) consists of:
\begin{enumerate}\itemsep0.25em
  \item the manifest hash and time-window partition IDs (App.~G),
  \item the list of \((W,i,\tau)\) checks performed,
  \item Gate Cascade results (pass/fail per stage) and the recorded \(\delta\)-ledger totals \(\Sigma\delta_W(i)\in V\),
  \item the diagnostics \((\mu,\nu)\) and Type~IV flags (App.~D),
  \item cryptographic hashes of artifacts used by checks (\texttt{bars/spec/ext/phi/Lambda\_len}; App.~G),
  \item a deterministic replay recipe pointer (container digest / code hash; App.~G).
\end{enumerate}
Advisory outputs (LMHS/tropical proposals) may be included only as tagged \(\delta^{\mathrm{meas}}\) contributions
(and optionally reported as \(\delta^{\mathrm{spec}}\)) and never as Gates.
\end{definition}

% --------------------------------------------------------------------------
\subsection*{U.3. Mapper protocol (Coverage Graph and assembly)}
\label{sec:U-mapper}
% --------------------------------------------------------------------------

\begin{specification}[Mapper update rule]\label{U:spec:mapper}
Upon receiving a Local Certificate for a Terrain Cell \(\mathcal{C}_\alpha\), the Mapper must:
\begin{enumerate}\itemsep0.35em
  \item \textbf{Verify certificate integrity.}
  Check the manifest hash match, required test set presence (Ch.~12),
  and ledger completeness for canonical channels (App.~L/S/T).
  If any mandatory component is missing, reject the update.

  \item \textbf{Vertex insertion.}
  Insert \(\mathcal{C}_\alpha\) into \(\mathcal{V}\) and into \(\mathcal{T}_{\mathrm{valid}}\).

  \item \textbf{Overlap verification.}
  For each existing \(\mathcal{C}_\beta\in\mathcal{T}_{\mathrm{valid}}\) with
  \(\mathcal{C}_\alpha\cap\mathcal{C}_\beta\neq\emptyset\), run the certified Overlap Gate on the overlap region,
  evaluated only after collapse on the time-window bundle:
  \begin{enumerate}\itemsep0.25em
    \item restrict parameters to \(\mathcal{C}_\alpha\cap\mathcal{C}_\beta\) and evaluate each time window \(W\);
    \item compute/read \(\mathbf{T}_\tau\mathbf{P}_i(\cdot)\) objects on overlaps;
    \item run the certified Overlap Gate (App.~R/N/J; residuals recorded in the \(\delta\)-ledger);
    \item if successful, insert an undirected edge \((\alpha,\beta)\in\mathcal{E}\).
  \end{enumerate}
  Definable finiteness (finite events, finite \v{C}ech depth) must be logged under \texttt{definable} (App.~G/H/J/Q).

  \item \textbf{Global coverage check.}
  Periodically verify whether the union of Terrain Cells in the largest connected component of \(\mathcal{G}\)
  covers the target domain \(\mathcal{M}_{\mathrm{target}}\) specified by the Proof Object.
  The coverage criterion and any tolerated uncovered residue must be explicit and logged.
\end{enumerate}
\end{specification}

% --------------------------------------------------------------------------
\subsection*{U.4. Lifter protocol (dimension management; Type IV handling)}
\label{sec:U-lifter}
% --------------------------------------------------------------------------

\begin{specification}[Lifter operational semantics]\label{U:spec:lifter}
When invoked at \(S^{\mathrm{L}}\), the Lifter attempts to resolve a Type~IV obstruction by extending the parameter space.
All lifting steps are \textbf{[Spec]} and must preserve the canon constraints: after-collapse evaluation,
ledger accounting in \(V\), and definable finiteness of the induced subproblems (App.~G/H/J/Q/S/T).

\begin{enumerate}\itemsep0.4em
  \item \textbf{Axis selection (finite catalog).}
  Select one or more auxiliary axes \(\mathcal{A}_j\) from the finite axis catalog pinned in \texttt{run.yaml}
  (implementation notes; App.~T), e.g.\ smoothing width, truncation order, arithmetic level, or model hyperparameters.
  All axes and admissible ranges must be declared in the manifest; axes are not allowed to modify Core claims.

  \item \textbf{Directional test (advisory).}
  Evaluate an advisory directional decrease test for \(\Phi_\tau\) along \(\mathcal{A}_j\) at \((x_{\mathrm{sing}},0)\) in
  \(\mathcal{M}\times\mathcal{A}_j\), using a deterministic finite-difference rule pinned by \texttt{grad\_policy}
  (App.~G; Spec.~\ref{spec:13-grad-oracle}).

  \item \textbf{Augmented gap condition (certified budget check).}
  A lift to depth \(\ell+1\) introduces a lifting penalty \(\delta^{\mathrm{lift}}(\ell+1)\in V\),
  recorded as a refinement tag of \(\delta^{\mathrm{alg}}\) (App.~L/S/T).
  The lift is permitted only if, on each required time window and degree,
  \[
  \Sigma\delta_W(i;x_{\mathrm{sing}})\ \oplus\ \delta^{\mathrm{lift}}(\ell+1)\ <\ \mathrm{gap}_\tau(i;W,x_{\mathrm{sing}}),
  \]
  i.e.\ the B-Gate\(^{+}\) inequality remains satisfied after charging the lift penalty (App.~J/S).

  \item \textbf{Commit or fail.}
  \begin{itemize}\itemsep0.25em
    \item \textbf{Commit:}
    If the directional test indicates decrease for some axis \(\mathcal{A}_j\) and the Augmented gap condition holds,
    commit the lift, charge \(\delta^{\mathrm{lift}}(\ell+1)\), and spawn a new Hunter on
    \(\mathcal{M}^{(\ell+1)}:=\mathcal{M}^{(\ell)}\times\mathcal{A}_j\), with a new Terrain Cell partition
    (definable refinement policy logged).
    \item \textbf{Fail / Terminal Barrier (candidate counterexample region):}
    Otherwise, mark \((x_{\mathrm{sing}},\mathcal{C}_{\mathrm{sing}})\) as a Terminal Barrier and halt escalation.
    This yields a counterexample \emph{candidate record} (not a theorem): diagnostics, ledger traces,
    and a reproducible failure-mode summary.
  \end{itemize}
\end{enumerate}

\paragraph{Mandatory lift trace.}
Every lift attempt must append a \texttt{lift\_attempt} entry to the Action Log including:
chosen axis, axis value(s), directional test output, penalty \(\delta^{\mathrm{lift}}\),
gap check pass/fail, and an explicit commit/reject rationale (Sec.~\ref{sec:U-log}).
\end{specification}

% --------------------------------------------------------------------------
\subsection*{U.5. Action log schema (JSON/YAML; mandatory, linked, replayable)}
\label{sec:U-log}
% --------------------------------------------------------------------------

\begin{definition}[Action log entry]\label{U:def:logentry}
A single log entry \(\mathsf{Entry}_k\) is the tuple
\[
\mathsf{Entry}_k = (k,\ \mathsf{ts},\ \mathsf{agent},\ \mathsf{seed},\ \mathsf{manifest},\ x_k,\ \mathcal{C}_k,\ \Phi_\tau(x_k),\ A_k,\ \mathsf{aux},\ \mathsf{hash}),
\]
where \(k\) is a step index, \(\mathsf{ts}\) a timestamp (optional),
\(\mathsf{agent}\) encodes agent ID/type, \(\mathsf{seed}\) encodes the RNG state,
\(\mathsf{manifest}\) is the manifest hash,
\(x_k,\mathcal{C}_k,\Phi_\tau(x_k)\) are the state components,
\(A_k\) is the action label, \(\mathsf{aux}\) stores action-specific structured data,
and \(\mathsf{hash}\) is a cryptographic digest of the canonical serialization of the entry (App.~G).
\end{definition}

\begin{specification}[Schema requirements (canonical keys)]\label{U:spec:schema}
A concrete implementation (JSON Lines recommended) MUST encode the following keys.

\paragraph{Top-level keys (every entry).}
\begin{verbatim}
k: <int>                        # step index
ts: <string|null>               # optional timestamp
agent: { id: "...", type: "H|M|L" }
manifest: { run_id: "...", run_yaml_hash: "sha256:..." }
seed: { value: <int>, rng: "...", rng_state_hash: "sha256:..." }

state:
  x: <json>                     # parameter point (typed payload)
  cell_id: "sha256:..."         # Terrain Cell identifier (hash of definable description)
  Phi: <float>                  # Phi_tau(x) (scalar)
bundle:
  tau: <float>
  degrees: [ ... ]              # list of i
  windows: [ ... ]              # list of window IDs (right-open; MECE)

ledger:
  snapshot_id: "sha256:..."
  totals:
    delta_alg: <float>
    delta_disc: <float>
    delta_meas: <float>
  tags:                          # optional refinements/tags
    delta_Gal: <float|null>
    delta_Tr:  <float|null>
    delta_Fun: <float|null>
    delta_lift:<float|null>
    delta_spec:<float|null>      # optional transparency tag (non-gating)

diagnostics:
  mu: <int|null>
  nu: <int|null>
  type_iv: <bool|null>

action: "grad_estimate|gradient_step|restart|validate|escalate|lift_attempt|map_update"
aux: { ... }                     # action-specific payload (see below)

integrity:
  refs:
    bars: "sha256:..."           # optional, if referenced in this step
    spec: "sha256:..."
    ext:  "sha256:..."
    phi:  "sha256:..."
    Lambda_len: "sha256:..."
  prev_hash: "sha256:..."        # optional hash-chain
hash: "sha256:..."               # hash of canonical serialization of this entry
\end{verbatim}

\paragraph{Action-specific payloads (mandatory when action occurs).}

\noindent\textbf{(1) \texttt{grad\_estimate}.} The following keys are mandatory:
\begin{verbatim}
aux:
  grad:
    method: "finite_difference|SPSA|surrogate"
    norm: "fro|op|custom"
    stencil: { kind: "one_sided|two_sided", eps: <float>, coords: "all|subset:[...]" }
    seed: <int>
    eval_count: <int>
    variance: <float>            # estimate or certified upper bound
    variance_mode: "estimate|upper_bound"
    delta_charge:
      channel: "delta_meas|delta_disc"
      amount: <float>
\end{verbatim}
The tuple \texttt{(method, stencil, seed, variance)} is required for replayability.

\noindent\textbf{(2) \texttt{gradient\_step}.}
\begin{verbatim}
aux:
  step:
    step_size: <float>
    rule: "projected_descent|trust_region|other"
    accept: <bool>
    reason: "<string>"
\end{verbatim}

\noindent\textbf{(3) \texttt{restart}.}
\begin{verbatim}
aux:
  restart:
    criterion: "<string>"
    refined_cells: ["sha256:...", ...]
    refinement_policy: "<string>"     # definable finite subdivision description
\end{verbatim}

\noindent\textbf{(4) \texttt{validate}.}
\begin{verbatim}
aux:
  gates:
    b_gate_plus: { passed: <bool>, slack: <float|null> }
    pfbc:        { passed: <bool|null>, residual: <float|null> }
    overlap:     { passed: <bool|null> }
    reason: "<string>"
\end{verbatim}

\noindent\textbf{(5) \texttt{lift\_attempt}.} (Lifter trace; mandatory when lifting is invoked)
\begin{verbatim}
aux:
  lift:
    depth_from: <int>
    depth_to: <int>
    axis: "<string>"                  # from finite axis catalog
    axis_value: <json>
    directional_test: { value: <float>, passed: <bool> }   # advisory
    penalty:
      delta_lift: <float>
      charged_to: "delta_alg"
    augmented_gap_check: { passed: <bool>, slack: <float|null> }
    commit: <bool>
    rationale: "<string>"
\end{verbatim}

\noindent\textbf{(6) \texttt{map\_update}.} (Mapper event)
\begin{verbatim}
aux:
  mapper:
    received_cell: "sha256:..."
    accepted: <bool>
    overlap_edges_added: [["sha256:...","sha256:..."], ...]
    reason: "<string>"
\end{verbatim}

\paragraph{Replayability condition (operational).}
A third party providing the same manifest hash, initial seed/state,
and running the trusted AK Core (Part~I) with the declared deterministic numeric policy
must reproduce identical certificates and identical serialized artifacts referenced by the log,
up to the declared serialization and tolerance policy (App.~G).
Any missing mandatory key (notably \texttt{grad.method/stencil/seed/variance} when gradients are used,
or the \texttt{lift} payload when lifting occurs) invalidates the corresponding agent-based claim.
\end{specification}

% --------------------------------------------------------------------------
\subsection*{U.6. Proof Object packaging (Spec; what counts as valid output)}
% --------------------------------------------------------------------------

\begin{definition}[Map-of-Validity Proof Object (Part II; Spec)]\label{U:def:proofobj}
A Part~II Proof Object consists of:
\begin{enumerate}\itemsep0.25em
  \item the manifest \texttt{run.yaml} and its hash (App.~G),
  \item the set of validated Terrain Cells \(\mathcal{T}_{\mathrm{valid}}\) with Local Certificates (Def.~\ref{U:def:localcert}),
  \item the Coverage Graph \(\mathcal{G}\) with Overlap Gate edge evidence (Sec.~\ref{sec:U-mapper}),
  \item the complete Action Log for all agents (Sec.~\ref{sec:U-log}),
  \item the ledger exports and aggregated totals used in B-Gate\(^{+}\) and PF/BC/Overlap checks (App.~J/N/R/S/T),
  \item any Terminal Barrier records (if present), as counterexample candidates only.
\end{enumerate}
Advisory components (LMHS/tropical suggestions) may be included only as tagged \(\delta^{\mathrm{meas}}\) contributions
(and optionally reported as \(\delta^{\mathrm{spec}}\)) and never as stand-alone validators.
\end{definition}

% --------------------------------------------------------------------------
\subsection*{U.7. Non-claims (scope guard)}
% --------------------------------------------------------------------------
No statement herein asserts new pre-collapse control, global commutation beyond recorded \(2\)-cell bounds,
or any converse beyond the one-way bridge \(\mathrm{PH}_1\Rightarrow\Ext^1\) in \(D^{\mathrm{b}}(k\text{-mod})\) (Ch.~3; App.~C).
The agents do not create mathematical truth; they create auditable search traces whose certified components
are limited to after-collapse Gates/tests under the UCC and the \(\delta\)-ledger discipline (Ch.~1/2; App.~J/N/R/S/T).




% ==========================================================================
% Appendix V : Validity Map Formalism
% ==========================================================================
% Target: AiM Standard (Topological definitions, Data structures)
% Role: Formalizes the "Map of Validity" as a stratified space and defines
%       the Global Certificate structure required for proof acceptance.
% ==========================================================================

\section*{Appendix V. Validity Map Formalism [Spec; stratified semantics and Global Certificate kernel]}
\phantomsection
\addcontentsline{toc}{section}{Appendix V. Validity Map Formalism}
\refstepcounter{section}
\label{V:validity}

\paragraph{Standing conventions (canon).}
All quantitative evaluations are \emph{after collapse} in the fixed order
\[
\boxed{\ \text{for each }t\ \Longrightarrow\ \mathbf{P}_i\ \Longrightarrow\ \mathbf{T}_\tau\ \Longrightarrow\ \text{compare on }\Pers^{\mathrm{ft}}\ } \qquad (\text{Ch.~1/2; App.~J/S/T}).
\]
Windows on the time axis are definable and right-open, and the chosen cover has finite
event counts and finite \v{C}ech depth on bounded ranges (App.~H/J/Q).
All defects/budgets are written to a fixed commutative quantale \(V\) via the \(\delta\)-ledger
(App.~S), aggregated by \(\oplus\).
Type~IV obstruction means \((\mu,\nu)\neq(0,0)\) (App.~D; canon).
PF/BC and arithmetic control (Overlap Gate) are evaluated only after collapse (App.~N/R).
All material below is \textbf{[Spec]} unless explicitly stated otherwise; it defines operational proof objects
and does not extend proven theorems of Part~I.

% --------------------------------------------------------------------------
\subsection*{V.0. Purpose and scope}
% --------------------------------------------------------------------------
This appendix formalizes the \emph{Map of Validity} introduced in Chapter~\ref{sec:ch13}.
It defines (i) the validity stratification of the parameter space \(\mathcal{M}\)
via the Defect Potential \(\Phi_\tau\), and (ii) the \emph{Global Certificate}
data structure whose verification reduces global acceptance to finite,
auditable checks under the Unified Collapse Contract (UCC; Ch.~1).

% --------------------------------------------------------------------------
\subsection*{V.1. Level-set topology of \texorpdfstring{\(\Phi_\tau\)}{Phi}}
% --------------------------------------------------------------------------

\paragraph{Definable parameter space.}
Let \(\mathcal{M}\) be a definable topological space (e.g.\ definable in an o-minimal expansion
\(\mathbb{R}_{\mathrm{an,exp}}\), or definable in a Denef--Pas structure; see App.~H/Q for definability conventions).
A \emph{Terrain Cell decomposition} \(\mathcal{T}\) is a finite or locally finite definable cover of \(\mathcal{M}\)
by definable sets (Terrain Cells) with logged overlap multiplicity and (when needed) a refinement policy
(App.~U/T).

\begin{definition}[Defect Potential (after-collapse scalarization)]\label{V:def:phi}
Fix:
(i) a collapse scale \(\tau>0\),
(ii) a finite index set of degrees \(I\),
(iii) a bundle of time windows \(\mathcal{W}_{\mathrm{time}}\) (right-open, definable; App.~H/J/Q),
(iv) a quantale \(V\) with ledger aggregation \(\oplus\) (App.~S),
and (v) a monotone scalarization \(\pi:V\to\mathbb{R}_{\ge0}\cup\{\infty\}\) logged in \texttt{run.yaml}
(App.~S/U/T).
For \(x\in\mathcal{M}\), define \(\Phi_\tau(x)\in\mathbb{R}_{\ge0}\cup\{\infty\}\) by
\[
\Phi_\tau(x)\ :=\ \max_{(W,i)\in\mathcal{W}_{\mathrm{time}}\times I}
\Big(\ \pi(\Sigma\delta_W(i;x))\ -\ \pi(\mathrm{gap}_\tau(i;W,x))\ \Big)_{+},
\]
where \(\Sigma\delta_W(i;x)\in V\) is the windowwise ledger total (App.~J/S/T) and
\(\mathrm{gap}_\tau(i;W,x)\in V\) is the B-Gate\(^{+}\) threshold (App.~J).
Thus \(\Phi_\tau(x)=0\) means that the required after-collapse inequalities pass on all declared windows/degrees.
\end{definition}

\begin{definition}[Operational strata]\label{V:def:strata}
Fix safety thresholds \(\mathrm{gap}_\tau\) (certification limit) and \(\lambda_{\mathrm{sing}}\) (singularity threshold),
both pinned by policy and logged (App.~J/U/T).
Define:
\begin{enumerate}\itemsep0.35em
  \item \textbf{Valid set}:
  \[
    Z_{\mathrm{Valid}}\ :=\ \{x\in\mathcal{M}\mid \Phi_\tau(x)=0\}.
  \]
  This is the region where the \emph{after-collapse Gate Cascade} (App.~T) succeeds under the declared \(\delta\)-budgets.

  \item \textbf{Noise set}:
  \[
    Z_{\mathrm{Noise}}\ :=\ \{x\in\mathcal{M}\mid 0<\Phi_\tau(x)<\lambda_{\mathrm{sing}}\}.
  \]
  This is the region of nonzero but non-singular defect potential (operational Types I--III). Hunter agents search here.

  \item \textbf{Singular set}:
  \[
    Z_{\mathrm{Sing}}\ :=\ \{x\in\mathcal{M}\mid \Phi_\tau(x)\ge \lambda_{\mathrm{sing}}\ \ \text{or}\ \ (\mu(x),\nu(x))\neq(0,0)\}.
  \]
  This is the region of essential obstructions (Type~IV by diagnostics and/or large potential; App.~D/U).
\end{enumerate}
\end{definition}

\begin{remark}[About ``closed/open'' claims]
We do not assume topological regularity of \(\Phi_\tau\) beyond what is logged and tested
(e.g.\ definability, finite-event behavior on time windows, optional Lipschitz controls; App.~H/J/Q/S).
Consequently, openness/closedness of the strata is not asserted unless proven in the chosen definable regime and recorded as such.
\end{remark}

\begin{definition}[Validity Map at scale \(\tau\)]\label{V:def:validitymap}
The \emph{Validity Map} at scale \(\tau\) is the tuple
\[
\mathfrak{V}_\tau\ :=\ \bigl(\mathcal{M},\ \Phi_\tau,\ \mathcal{T};\ Z_{\mathrm{Valid}},Z_{\mathrm{Noise}},Z_{\mathrm{Sing}}\bigr),
\]
where \(\mathcal{T}\) is a Terrain Cell decomposition compatible with the operational strata in the following sense:
for each \(\mathcal{W}_\alpha\in\mathcal{T}\), either
(i) \(\Phi_\tau\) has constant regime on \(\mathcal{W}_\alpha\) (all points certify / all are noise / all are singular),
or (ii) \(\Phi_\tau\) has a logged control law on \(\mathcal{W}_\alpha\) (e.g.\ a Lipschitz bound with respect to a declared metric)
sufficient for the Mapper's refinement/coverage logic (App.~U/T).
\end{definition}

\begin{specification}[Operational global acceptance condition (Spec)]\label{V:spec:accept}
Operationally, we accept \emph{global validity on a target domain} \(\mathcal{M}_{\mathrm{target}}\subseteq\mathcal{M}\)
only via a \emph{Global Certificate} (Section~\ref{V:global-cert}) that provides:
\begin{itemize}\itemsep0.35em
  \item \textbf{Coverage:} \(\bigcup_{\alpha\in A}\mathcal{W}_\alpha \supseteq \mathcal{M}_{\mathrm{target}}\), verified by a covering proof in the definable category used (o-minimal / Denef--Pas / explicit polyhedral cover).
  \item \textbf{Local validity:} for each vertex cell \(\mathcal{W}_\alpha\), the after-collapse Gate Cascade passes on the declared time-window bundle (B-Gate\(^{+}\to\)PF/BC\(\to\)Overlap Gate), under recorded \(\delta\)-budgets (App.~T; App.~J/N/R).
  \item \textbf{No certified singularities:} \(\mathcal{W}_\alpha\cap Z_{\mathrm{Sing}}=\emptyset\) for all validated cells, certified by diagnostics \((\mu,\nu)=(0,0)\) and the logged Type~IV policy (App.~D/U).
\end{itemize}
No deformation-retract or density criterion is asserted as a theorem here; any such strengthening is \textbf{[Spec]} and must be logged with explicit hypotheses and tests.
\end{specification}

% --------------------------------------------------------------------------
\subsection*{V.2. Global Certificate format}
\label{V:global-cert}
% --------------------------------------------------------------------------

\paragraph{Design principle.}
A Global Certificate is a finite, auditable object: verification is a finite traversal plus finite ledger aggregation in \(V\)
(App.~S/J/T), using only after-collapse comparators.

\begin{definition}[Global Certificate structure]\label{V:def:globalcert}
A \emph{Global Certificate} is a directed acyclic graph (DAG)
\[
\mathcal{C}_{\mathrm{global}}\ :=\ \langle \mathcal{V},\ \mathcal{E},\ \mathcal{M}_{\mathrm{meta}}\rangle
\]
with:
\begin{itemize}\itemsep0.35em
  \item \textbf{Vertices \(\mathcal{V}\):} a finite collection of validated Terrain Cells
  \(\{\mathcal{W}_\alpha\}_{\alpha\in A}\).
  Each vertex stores:
  \begin{itemize}\itemsep0.25em
    \item a definable descriptor (formula, constraints, or code-hash-defined predicate) for \(\mathcal{W}_\alpha\);
    \item the declared time-window bundle \(\mathcal{W}_{\mathrm{time},\alpha}\) and degree set \(I_\alpha\);
    \item a \emph{local validity token} consisting of:
    (i) manifest hash, (ii) Gate Cascade pass results, (iii) hashes of referenced persistence artifacts,
    (iv) windowwise ledger totals \(\Sigma\delta_W(i;\mathcal{W}_\alpha)\in V\),
    and (v) diagnostics \((\mu,\nu)\) with Type~IV flag policy (App.~D/U/T).
  \end{itemize}

  \item \textbf{Edges \(\mathcal{E}\):} verified overlaps.
  An edge \(\alpha\to\beta\) exists only if:
  \begin{itemize}\itemsep0.25em
    \item \(\mathcal{W}_\alpha\cap\mathcal{W}_\beta\neq\emptyset\), and
    \item the \emph{after-collapse Overlap Gate} passes on the overlap, with any finite kernel/cokernel parts
    recorded as \(\delta^{\mathrm{alg}}\) (App.~R; PF/BC compatibility App.~N; finite-depth glue App.~J/Q).
  \end{itemize}
  The edge record stores the overlap descriptor and the Overlap Gate evidence (hashes + budgets).

  \item \textbf{Metadata \(\mathcal{M}_{\mathrm{meta}}\):}
  \begin{itemize}\itemsep0.25em
    \item \textbf{Covering proof:} evidence that \(\bigcup_{\alpha\in A}\mathcal{W}_\alpha \supseteq \mathcal{M}_{\mathrm{target}}\),
    including the definable regime, MECE/refinement policy, and any residue tolerance (App.~G/U).
    \item \textbf{Summability / finiteness:} evidence that the aggregation performed by the verifier is finite:
    bounded definable covers on time windows (finite event/\v{C}ech depth) and P6-type summability
    where applicable (App.~J/Q/S). The verifier aggregates only finite \(\oplus\)-sums.
    \item \textbf{Type~IV clearance (within certified region):}
    evidence that for all vertices, \((\mu,\nu)=(0,0)\) on the declared checks, hence
    \(Z_{\mathrm{Sing}}\cap\big(\bigcup_{\alpha\in A}\mathcal{W}_\alpha\big)=\emptyset\)
    \emph{within the certified regime and policy}.
  \end{itemize}
\end{itemize}
\end{definition}

\begin{specification}[Verification protocol (trusted kernel)]\label{V:spec:verify}
To verify \(\mathcal{C}_{\mathrm{global}}\), a trusted kernel performs:
\begin{enumerate}\itemsep0.4em
  \item \textbf{Manifest check.}
  Confirm the manifest hash and the enabled test set (Ch.~12) match all vertex/edge records.

  \item \textbf{Local check (per vertex).}
  For each \(\mathcal{W}_\alpha\in\mathcal{V}\), re-run the after-collapse Gate Cascade on the declared time windows:
  \[
  \text{B-Gate}^{+}\ \to\ \text{PF/BC}\ \to\ \text{Overlap Gate},
  \]
  recomputing (or verifying by hash) the referenced persistence artifacts and re-aggregating the windowwise ledger totals
  \(\Sigma\delta_W(i;\mathcal{W}_\alpha)\in V\) (App.~S/T).
  Confirm that Type~IV diagnostics satisfy \((\mu,\nu)=(0,0)\) under the declared policy (App.~D).

  \item \textbf{Edge/overlap check.}
  For each \(\alpha\to\beta\in\mathcal{E}\), re-verify the overlap descriptor and the after-collapse Overlap Gate evidence,
  including any control-derived finite kernel/cokernel budget terms in \(\delta^{\mathrm{alg}}\) (App.~R).

  \item \textbf{Coverage check.}
  Verify the covering proof that \(\bigcup_{\alpha\in A}\mathcal{W}_\alpha \supseteq \mathcal{M}_{\mathrm{target}}\),
  using the definable regime declared (o-minimal / Denef--Pas / explicit combinatorial cover).

  \item \textbf{Global budget check.}
  Re-aggregate the declared global ledger total
  \(\bigoplus_{\alpha\in A}\ \Sigma\delta(\mathcal{W}_\alpha)\) using the chosen quantale \(V\),
  and confirm compliance with the global gap policy mandated by UCC (Ch.~1; App.~J/S).
\end{enumerate}
Under these checks, verification reduces to finite graph traversal plus finite aggregation in \(V\).
\end{specification}

% --------------------------------------------------------------------------
\subsection*{V.3. Reproducibility hooks (run.yaml)}
% --------------------------------------------------------------------------
\begin{verbatim}
validity_map:
  tau: "<value>"
  phi:
    quantale: "<V_name>"
    scalarization_pi: "<pi: V -> R>=0>"
    sing_threshold: "<lambda_sing>"
    gap_policy: "<gap_tau policy>"
  target_domain: "<M_target descriptor>"
  terrain:
    definable_regime: "R_an,exp | Denef-Pas | explicit"
    cells: "<list or generator + hash>"
    refinement_policy: "<restart/refine rules>"
global_certificate:
  format: "DAG"
  vertex_fields: ["cell_descriptor","windows","degrees","gate_tokens","ledger_totals","diag_mu_nu","hashes"]
  edge_fields:   ["overlap_descriptor","overlap_gate_token","delta_alg_budget","hashes"]
  checks:
    local_gate_cascade: true
    overlap_gate: true
    coverage_proof: true
    global_budget: true
tests:
  T_GateCascade: true
  T_Definable: true
  T_Vsubadd: true
\end{verbatim}

% --------------------------------------------------------------------------
\subsection*{V.4. Summary}
% --------------------------------------------------------------------------
Appendix~V formalizes the \emph{output} of the AK-HDPST framework (Part~II) as:
\begin{itemize}\itemsep0.35em
  \item a stratified semantics on \(\mathcal{M}\) induced by an after-collapse Defect Potential \(\Phi_\tau\), and
  \item a finite, auditable Global Certificate DAG whose verification is a finite traversal plus quantale aggregation.
\end{itemize}
All acceptance statements are operational and confined to the declared target domain,
the after-collapse Gate Cascade, and the logged \(\delta\)-ledger discipline under UCC.



% ==========================================================================
% Appendix W : Bridge Programs — Detailed Specifications
% ==========================================================================
% Target: AiM Standard (Quantitative inequalities, Algorithmic logic)
% Role: Runtime guard-rails for any reverse-bridge use (Ext -> PH) after collapse.
% ==========================================================================

\section*{Appendix W. Bridge Programs — Detailed Specifications [Spec; Spectral-Gap guard-rails and B1/B2/B3 state machines]}
\phantomsection
\addcontentsline{toc}{section}{Appendix W. Bridge Programs — Detailed Specifications}
\refstepcounter{section}
\label{W:bridge}

\paragraph{Standing conventions (canon).}
All quantitative evaluations are \emph{after collapse} in the fixed order
\[
\boxed{\ \text{for each }t\ \Longrightarrow\ \mathbf{P}_i\ \Longrightarrow\ \mathbf{T}_\tau\ \Longrightarrow\ \text{compare on }\Pers^{\mathrm{ft}}\ }\qquad(\text{Ch.~1/2; App.~J/S/T}).
\]
Definability is o-minimal on the real side or Denef--Pas on the \(p\)-adic side (App.~H/Q).
Finite-event and finite \v{C}ech properties on bounded windows are assumed and must be logged (App.~H/J/Q/T).
All defects/budgets live in a fixed commutative quantale \(V\) and are aggregated by \(\oplus\) via the \(\delta\)-ledger
(App.~S; entries \(\delta^{\mathrm{alg}},\delta^{\mathrm{disc}},\delta^{\mathrm{meas}},\delta^{\mathrm{spec}},\delta^{\mathrm{lift}}\)).
Type~IV obstruction is \emph{by canon} \((\mu,\nu)\neq(0,0)\) (App.~D); any appearance of Type~IV triggers the escalation rules (App.~U).
This appendix is \textbf{[Spec]} throughout: it defines operational acceptance logic and logging discipline,
and does not extend proven implications of Part~I.

\paragraph{Notation warning.}
In Part~II we use \emph{Terrain Cells} \(\mathcal{W}_\alpha\subset\mathcal{M}\) for parameter-space decomposition (App.~U/V),
and \emph{time windows} \(W\subset\mathbb{R}\) (or \(\mathrm{VG}\) in Denef--Pas mode) for after-collapse persistence readouts (App.~H/J/Q/T).
Reverse-bridge checks in this appendix are \emph{cell-local in \(\mathcal{M}\)} but evaluate \(\mathbf{P}_i(\mathbf{T}_\tau\cdot)\) on the declared time-window bundle.

% --------------------------------------------------------------------------
\subsection*{W.0. Purpose and scope}
% --------------------------------------------------------------------------
This appendix refines the Bridge Programs of Chapter~\ref{sec:ch16} into executable specifications.
Its role is \emph{strictly guard-rail}: whenever an AI agent proposes a \emph{reverse direction}
(\(\Ext^1=0 \Rightarrow \PH_1=0\)) as part of a Proof Object, the AK Core accepts such a step
\emph{only} under the Spectral-Gap Condition and the state machines B1/B2/B3 below.
The intent is to prevent numerical hallucination (false vanishing) and to force full auditability.

\textbf{Non-goal:} no global theorem \(\Ext^1\Leftrightarrow \PH_1\) is claimed.
The only proven direction remains the one-way bridge \(\PH_1\Rightarrow \Ext^1\) in \(D^{\mathrm b}(k\text{-mod})\)
under stated hypotheses (Ch.~3/11; canon).

% --------------------------------------------------------------------------
\subsection*{W.1. Spectral-Gap Condition (SGC) as an audit gate}
% --------------------------------------------------------------------------

\paragraph{Policy boundary.}
Spectral indicators are \emph{not} assumed to be filtered quasi-isomorphism invariants; they are admissible
only as a logged audit channel with explicit policy (App.~S/T; canon boundary in Ch.~2/5).
Accordingly, the Spectral-Gap Condition is an \textbf{operational acceptance condition} and must be recorded as \textbf{[Spec]}
with its computation policy, bounds, and sampling discipline.

\begin{definition}[After-collapse Laplacian profile (policy-driven)]\label{W:def:spectral-profile}
Fix a Terrain Cell \(\mathcal{W}\subset\mathcal{M}\), collapse scale \(\tau\), and a declared \emph{realization policy}
\(\mathsf{Pol}_{\mathrm{Lap}}\) (e.g.\ truncation order, discretization scheme, normalization, and any parameters such as \(\beta,M(\tau),s\)),
logged in \texttt{run.yaml} (App.~T/U/S).
For each \(x\in\mathcal{W}\), let \(L_{\tau}(x)=L_{\tau}^{\mathsf{Pol}_{\mathrm{Lap}}}(x)\) be the \emph{normalized combinatorial Laplacian}
computed from the after-collapse object (i.e.\ from \(\mathbf{P}_i(\mathbf{T}_\tau F_x)\) and the declared construction).

Define the \textbf{signal floor} and \textbf{noise ceiling} by:
\begin{enumerate}\itemsep0.35em
  \item \textbf{Signal floor} \(\gamma_{\tau,\min}(\mathcal{W})\):
  \[
    \gamma_{\tau,\min}(\mathcal{W})\ :=\ \inf_{x\in\mathcal{W}}\ \lambda_1^{+}(L_{\tau}(x)),
    \qquad
    \lambda_1^{+}(L):=\min\{\lambda\in\sigma(L)\mid \lambda>0\},
  \]
  with the convention \(\lambda_1^{+}(L)=+\infty\) if \(\sigma(L)\subseteq\{0\}\) (everywhere-zero spectrum).

  \item \textbf{Noise ceiling} \(\delta_{\max}(\mathcal{W})\):
  \[
    \delta_{\max}(\mathcal{W})\ :=\ \sup_{x\in\mathcal{W}}\ \pi\!\big(\Sigma\delta(x)\big),
  \]
  where \(\Sigma\delta(x)\in V\) is the aggregated ledger total relevant to the reverse-bridge attempt
  and \(\pi:V\to\mathbb{R}_{\ge0}\cup\{+\infty\}\) is the declared scalarization (App.~S/V).
\end{enumerate}
\end{definition}

\begin{remark}[How \(\inf/\sup\) are realized in the implementable range]\label{W:rk:finite-realization}
On bounded definable windows, event finiteness and finite \v{C}ech depth reduce all required maxima to finite checks
(App.~H/J/Q). Operationally, the AK Core \emph{does not accept} unverifiable analytic \(\inf/\sup\):
\(\gamma_{\tau,\min}\) must be supplied as a certified lower bound and \(\delta_{\max}\) as a certified upper bound,
both produced from a finite cell/event decomposition or a logged finite sampling scheme with refinement guarantees.
All such certificates are part of the Proof Object and hashed (App.~U/T/V).
\end{remark}

\begin{specification}[Spectral-Gap Condition \(\mathrm{SGC}(c)\)]\label{W:spec:SGC}
Fix a safety factor \(c>1\), logged per run.
A Terrain Cell \(\mathcal{W}\) satisfies \(\mathrm{SGC}(c)\) at scale \(\tau\) if
\[
\boxed{\ \gamma_{\tau,\min}(\mathcal{W})\ >\ c\cdot \delta_{\max}(\mathcal{W})\ }.
\]
\textbf{Default policy:} enforce \(c\ge 2\) unless a tighter factor is justified and recorded with explicit bounds and tests.
\end{specification}

\begin{remark}[Interpretation]
\(\mathrm{SGC}(c)\) enforces that the first nonzero spectral mode is separated from \(0\) by a margin larger than the worst admissible defect budget.
This is used solely to block ``near-zero'' artifacts from being misread as true vanishing.
\end{remark}

% --------------------------------------------------------------------------
\subsection*{W.2. Program B1: Local reverse logic (Ext \(\Rightarrow\) PH) with guard-rails}
% --------------------------------------------------------------------------

\paragraph{Policy boundary (reverse direction).}
The reverse implication \(\Ext^1=0\Rightarrow \PH_1=0\) is \textbf{not} a theorem of the AK Core.
Program B1 defines the \emph{only} permissible way to produce a \emph{Reverse Certificate} for a specific Terrain Cell,
and such a certificate is valid only under the logged SGC policy, gate results, and ledger budgets.

\begin{definition}[B1 status codes]\label{W:def:B1-status}
Program B1 returns one of:
\[
\texttt{Valid}\ \mid\ \texttt{Indeterminate}\ \mid\ \texttt{Reject}\ \mid\ \texttt{TypeIV}\ \mid\ \texttt{Barrier}.
\]
\texttt{Valid} means ``reverse certificate issued under SGC(c) and UCC guard-rails''.
\texttt{Indeterminate} means ``insufficient separation or insufficient bounds'' (no reverse step allowed).
\texttt{Reject} means ``violated preconditions or failed audits''.
\texttt{TypeIV} means ``\((\mu,\nu)\neq(0,0)\)'' (must escalate; App.~U).
\texttt{Barrier} means ``terminal obstruction candidate logged for B3''.
\end{definition}

\begin{specification}[Program B1 (Reverse Bridge) — executable logic]\label{W:spec:B1}
\textbf{Inputs:}
Terrain Cell \(\mathcal{W}\subset\mathcal{M}\),
scale \(\tau\),
degree \(i\) (typically \(i=1\)),
declared time-window bundle \(\mathcal{W}_{\mathrm{time}}\),
quantale \(V\) with scalarization \(\pi\),
SGC factor \(c\),
and realization policy \(\mathsf{Pol}_{\mathrm{Lap}}\).
All are read from \texttt{run.yaml} and hashed (App.~T/U/V).

\textbf{Procedure (must follow after-collapse order):}
\begin{enumerate}\itemsep0.45em
  \item \textbf{Precondition checks (hard gates).}
  \begin{itemize}\itemsep0.25em
    \item \emph{Definability:} \(\mathcal{W}\) is definable and its descriptor validates (App.~H/Q/U).
    \item \emph{After-collapse Gate Cascade:} On \(\mathcal{W}\), the declared Gate Cascade passes:
    B-Gate\(^{+}\) on \(\mathbf{P}_i(\mathbf{T}_\tau\cdot)\), then PF/BC, then Overlap Gate on declared overlaps,
    with all finite parts logged in \(\delta\)-ledger (App.~T; App.~J/N/R).
    \item \emph{Type~IV exclusion:} diagnostics satisfy \((\mu,\nu)=(0,0)\) on \(\mathcal{W}\) (App.~D/U).
    If \((\mu,\nu)\neq(0,0)\), return \texttt{TypeIV}.
  \end{itemize}

  \item \textbf{Spectral audit (SGC).}
  Compute or import certified bounds \(\gamma_{\tau,\min}(\mathcal{W})\) and \(\delta_{\max}(\mathcal{W})\)
  under the declared policies (Def.~\ref{W:def:spectral-profile}, Rem.~\ref{W:rk:finite-realization}).
  If \(\mathrm{SGC}(c)\) fails, return \texttt{Indeterminate}.

  \item \textbf{Observed Ext-bound (numerical/algorithmic check).}
  Let \(\epsilon_{\mathrm{obs}}(\mathcal{W})\) be the observed magnitude of \(\Ext^1\) under the declared computation policy
  (e.g.\ cochain-level solver tolerances and truncations), logged as a budgeted measurement channel:
  \[
    \epsilon_{\mathrm{obs}}(\mathcal{W})\ :=\ \sup_{x\in\mathcal{W}} \ \bigl\|\Ext^1(\mathcal{R}(\mathbf{T}_\tau F_x)|_{\mathcal{W}_{\mathrm{time}}},k)\bigr\|_{\mathrm{obs}}.
  \]
  Require the guarded inequality
  \[
    \epsilon_{\mathrm{obs}}(\mathcal{W})\ \le\ \delta_{\max}(\mathcal{W}).
  \]
  If it fails, return \texttt{Reject}.
  (Operational meaning: the observed nonzero Ext is larger than what the ledger allows.)

  \item \textbf{Issue Reverse Certificate (local, conditional).}
  If all checks pass, return \texttt{Valid} and output a \textbf{Reverse Certificate} token stating:
  \begin{quote}\small
  On \(\mathcal{W}\), under the declared realization policy \(\mathsf{Pol}_{\mathrm{Lap}}\),
  SGC\((c)\), the after-collapse Gate Cascade, and the recorded \(\delta\)-ledger budgets,
  the reverse step \(\Ext^1=0\Rightarrow \PH_1=0\) is accepted as an operational inference.
  \end{quote}
  The token MUST include hashes of: \(\mathcal{W}\) descriptor, \(\tau\), \(c\),
  \(\gamma_{\tau,\min}\) certificate, \(\delta_{\max}\) certificate,
  and the referenced persistence artifacts (App.~U/V).
\end{enumerate}
\end{specification}

% --------------------------------------------------------------------------
\subsection*{W.3. Program B2/B3: Global orchestration (proof mode / disproof mode)}
% --------------------------------------------------------------------------

\begin{specification}[Program B2: Global Regularity (Proof Mode; Spec)]\label{W:spec:B2}
\textbf{Goal:} Construct a Global Certificate covering \(\mathcal{M}_{\mathrm{target}}\) (App.~V),
where each certified cell supports any required reverse steps only via Program B1.

\textbf{State:}
a queue \(\mathcal{Q}\) of regions to process,
and a Coverage Graph \(\mathcal{G}\) managed by the Mapper (App.~U/V).

\textbf{Algorithm (operational):}
\begin{enumerate}\itemsep0.45em
  \item Initialize \(\mathcal{Q}\gets\{\mathcal{M}_{\mathrm{target}}\}\), \(\mathcal{G}\gets\emptyset\).
  \item While \(\mathcal{Q}\neq\emptyset\):
  \begin{itemize}\itemsep0.3em
    \item Pop a region \(R\) from \(\mathcal{Q}\).
    A Hunter proposes a definable Terrain Cell \(\mathcal{W}\subseteq R\) with a logged action trace (App.~U).
    \item Run Program B1 on \(\mathcal{W}\).
    \item If \texttt{Valid}:
      add \(\mathcal{W}\) as a vertex; invoke Mapper overlap verification (after-collapse Overlap Gate) to add edges (App.~U/R).
    \item If \texttt{Indeterminate}:
      apply Restart/Refinement policy to split \(R\) or refine \(\mathcal{W}\) into subcells
      with finite-event guarantees and updated budgets (App.~J/T/U).
    \item If \texttt{TypeIV}:
      invoke the Lifter (App.~U).
      If lift commits under the augmented gap and \(\delta^{\mathrm{lift}}\) budget, push lifted cells into \(\mathcal{Q}\);
      otherwise mark as \texttt{Barrier} and switch to B3.
    \item If \texttt{Reject}:
      refine or discard per policy; rejection reasons must be logged (App.~U/T).
  \end{itemize}
  \item Terminate with \textbf{Global Regularity (operational)} only when:
  \begin{itemize}\itemsep0.25em
    \item the largest connected component of \(\mathcal{G}\) covers \(\mathcal{M}_{\mathrm{target}}\) (covering proof), and
    \item global ledger aggregation satisfies the UCC gap policy and P6/Summability constraints as declared (App.~J/V/S).
  \end{itemize}
  Export a Global Certificate DAG (App.~V) including all B1 tokens and hashes.
\end{enumerate}
\end{specification}

\begin{specification}[Program B3: Counterexample Hunt (Disproof Mode; Spec)]\label{W:spec:B3}
\textbf{Goal:} Isolate an \emph{essential} singularity as a Certified Counterexample Candidate, with exhaustive logged lift attempts.

\textbf{Algorithm (operational):}
\begin{enumerate}\itemsep0.45em
  \item \textbf{Search.}
  A Hunter maximizes \(\Phi_\tau\) on \(\mathcal{M}_{\mathrm{target}}\) (or on unresolved regions),
  producing a peak candidate \(x^*\) with full action log (App.~U).

  \item \textbf{Verify Type~IV signature.}
  Confirm \((\mu(x^*),\nu(x^*))\neq(0,0)\) after collapse (App.~D).
  If not, the candidate is not Type~IV and must be returned to B2 refinement.

  \item \textbf{Exhaust lifts (finite catalog).}
  Enumerate the finite auxiliary-axis catalog declared for the run (App.~U/S; logged).
  For each axis \(\mathcal{A}_j\), evaluate the directional change policy
  (e.g.\ finite-difference/adjoint) for \(\partial_{\mathcal{A}_j}\Phi_\tau(x^*,0)\),
  and estimate the Lifting Penalty \(\delta^{\mathrm{lift}}(k+1)\) charged to the ledger.
  If for all \(\mathcal{A}_j\) either:
  \[
  \partial_{\mathcal{A}_j}\Phi_\tau(x^*,0)\ \ge\ 0
  \qquad\text{or}\qquad
  \Sigma\delta(x^*)\oplus \delta^{\mathrm{lift}}(k+1)\ \not<\ \mathrm{gap}_\tau,
  \]
  then declare the point \textbf{Essential} and mark it as a \textbf{Terminal Barrier} (App.~U).

  \item \textbf{Output.}
  Return \(x^*\) together with:
  the full ledger snapshot, diagnostics \((\mu,\nu)\), failed lift attempts (with hashes),
  and the precise policy bundle (quantale \(V\), scalarization \(\pi\), SGC factor \(c\), realization policy \(\mathsf{Pol}_{\mathrm{Lap}}\)).
\end{enumerate}
\end{specification}

% --------------------------------------------------------------------------
\subsection*{W.4. Reproducibility hooks (run.yaml / logs)}
% --------------------------------------------------------------------------
\begin{verbatim}
bridge_programs:
  enabled_reverse_bridge: true/false
  tau: "<value>"
  degree: 1
  time_windows: "<bundle-id>"
  phi_policy:
    quantale: "<V_name>"
    scalarization_pi: "<pi: V->R>=0>"
    gap_policy: "<gap_tau rule>"
    sing_threshold: "<lambda_sing>"
  spectral_gap:
    factor_c: 2.0
    laplacian_policy: "<Pol_Lap id>"
    gamma_cert: "<path/hash of lower-bound cert>"
    delta_max_cert: "<path/hash of upper-bound cert>"
  B1:
    require_gate_cascade: true
    require_diag_mu_nu_zero: true
    statuses: ["Valid","Indeterminate","Reject","TypeIV","Barrier"]
  B2:
    queue_policy: "<refine/restart rules>"
    stop_condition: "coverage+budget"
  B3:
    lift_catalog: "<axis list/hash>"
    max_lift_depth: "<kmax>"
logs:
  reverse_certificates: "logs/reverse_tokens.jsonl"
  bridge_audits: "logs/bridge_audits.json"
tests:
  T_GateCascade: true
  T_Vsubadd: true
  T_Definable: true
\end{verbatim}

% --------------------------------------------------------------------------
\subsection*{W.5. Non-claims}
% --------------------------------------------------------------------------
No statement here asserts a mathematical theorem \(\Ext^1=0\Rightarrow \PH_1=0\).
SGC\((c)\) is an operational audit gate only, dependent on a declared Laplacian realization policy.
No pre-collapse comparisons are permitted.
No spectral invariance beyond what is logged and tested is assumed.
All reverse certificates are cell-local, policy-local, and invalid outside the logged run configuration.

% --------------------------------------------------------------------------
\subsection*{W.6. Integration points}
% --------------------------------------------------------------------------
This appendix is consumed by Chapter~\ref{sec:ch16} (Bridge Programs),
Appendix~U (agent protocols and logging),
Appendix~V (Global Certificate format),
Appendix~T (notebook/CI skeletons),
Appendix~S (quantale/scalarization policy),
and Appendix~J (Restart/Summability constraints).
All references to overlaps and finite kernel/cokernel accounting use App.~R,
and PF/BC checks use App.~N, always after collapse.



% ==========================================================================
% Appendix NS : Navier--Stokes Case Study (Independent)
% ==========================================================================
% Target: AiM Standard (Application Template)
% Role: Demonstrates how the abstract machinery of AK-HDPST v17.0 maps
%       onto a concrete Millennium Prize Problem.
%       Defines the specific "Dissipation ECF" and "Type IV" interpretation
%       for 3D Incompressible NSE.
% ==========================================================================

% ============================================================
% Appendix NS (AK-HDPST v17.0): Navier--Stokes Case Study [Spec]
% Canon-aligned, after-collapse-only, auditable exploration mode
% ============================================================
%
% Preamble requirements (minimal):
% \usepackage{amsthm,amsmath,amssymb,hyperref,listings}
% \newtheorem{definition}{Definition}[section]
% \newtheorem{specification}{Specification}[section]
% \newtheorem{auditobligation}{Audit Obligation}[section]
%
% NOTE: This appendix is [Spec]. It does not extend Part I theorems.

\section*{Appendix NS: Case Study \textnormal{[Spec]} --- 3D Incompressible Navier--Stokes (Exploration Mode)}
\phantomsection
\addcontentsline{toc}{section}{Appendix NS. Navier--Stokes Case Study [Spec]}
\refstepcounter{section}
\label{app:NS}

% -------------------------
\subsection*{NS.0. Status, boundary, and canon}
% -------------------------

\paragraph{Status (strict).}
This appendix is \textbf{[Spec]} and belongs to the \emph{Exploration/Navigation} layer. It provides an auditable instantiation
of the Part~II machinery (windows, ledger, gates, Hunter/Mapper/Lifter) to the 3D incompressible Navier--Stokes equations (NSE).
\textbf{No Millennium-claim is made.} Any output is empirical/certificational evidence \emph{within declared search classes}
and \emph{within declared policies}. Nothing here upgrades or modifies the theorem-level core of Part~I.

\paragraph{Canon alignment (v17.0).}
All constraints below are mandatory for consistency with v17.0:
\begin{itemize}\itemsep4pt
  \item \textbf{After-collapse only.} Every metric that can influence a decision (gate pass/fail, regime label, certificate)
  is computed on the B-side: \(\mathbf{T}_\tau\mathbf{P}_i(F)\). Pre-collapse readings may be recorded, but never gate.
  \item \textbf{Type IV meaning is fixed.} ``Normal'' means \((\mu,\nu)=(0,0)\). \textbf{Type~IV obstruction} means
  \((\mu,\nu)\neq(0,0)\), computed \emph{after} \(\mathbf{T}_\tau\) under the declared tower policy.
  \item \textbf{Quantale ledger.} All budget accounting is performed in a declared commutative quantale \(\mathsf{V}\),
  with a declared scalarization/norm used \emph{only} for reporting.
  \item \textbf{Proxy separation.} Tropical/LMHS/spectral proxies (if used) are advisory only and never gate alone.
  \item \textbf{Arithmetic overlap rule.} Arithmetic control participates only through the \emph{Control\(\Rightarrow\)Overlap Gate};
  finite parts are logged as \(\delta^{\mathrm{alg}}\).
  \item \textbf{Gate order (after-collapse).} The Gate Cascade is
  \[
    \text{B-Gate}^{+}\ \longrightarrow\ \text{PF/BC}\ \longrightarrow\ \text{Overlap Gate}.
  \]
  \item \textbf{Restart/Summability.} The Convergence Manager enforces Restart/Summability windowwise and records stability bands.
  \item \textbf{Definable finiteness.} Time windows and event partitions are definable, so all enumerations are finite on bounded windows.
\end{itemize}

\paragraph{Non-claims (hard boundary).}
This appendix makes \textbf{no} claim of global regularity, no claim of equivalence \(\mathrm{PH}_1\Leftrightarrow \Ext^1\),
and no claim that any chosen realization is canonical beyond the explicitly declared policy. In particular,
any reverse-audit style step is prohibited unless guarded exactly as specified in the Bridge Programs layer (if invoked at all),
and even then remains \textbf{[Spec]}.

% -------------------------
\subsection*{NS.1. PDE arena and declared search class}
% -------------------------

\paragraph{Equation.}
We consider 3D incompressible NSE on \(\mathbb{T}^3\) (or a declared bounded domain with periodic surrogate):
\[
  \partial_t u + (u\cdot\nabla)u + \nabla p = \nu \Delta u,\qquad \nabla\cdot u=0,
\]
with viscosity \(\nu>0\) and initial data \(u(0)=u_0\).

\paragraph{Regularity class and blow-up locus (classical).}
Fix \(s>\frac{5}{2}\) and define the phase space
\[
  M_{\mathrm{NS}}^{(s)}:=\{u_0\in H^s(\mathbb{T}^3): \nabla\cdot u_0=0\}.
\]
Define the (classical) blow-up locus
\[
  L_{\mathrm{BlowUp}}:=\left\{u_0\in M_{\mathrm{NS}}^{(s)}:
  \exists T^\star<\infty,\ \limsup_{t\uparrow T^\star}\|\nabla u(t)\|_{L^\infty}=\infty\right\}.
\]

\paragraph{Declared finite-dimensional search class.}
Exploration is restricted to a parametrized family \(u_0(\theta)\) with
\[
  \theta\in\Theta\subset\mathbb{R}^d,\qquad u_0(\theta)\in M_{\mathrm{NS}}^{(s)},
\]
generated by an explicit \emph{search class generator} (e.g. vortex rings, Beltrami perturbations, multi-scale wave packets)
together with explicit constraints ensuring divergence-freeness and energy bounds.
All evidence is conditional on this declared class.

% -------------------------
\subsection*{NS.2. Windows, definability, and measurement policy}
% -------------------------

\paragraph{Time horizon and windows.}
Fix a finite horizon \([0,T]\). Partition time into windows
\[
  W_j=[t_j,t_{j+1}),\quad 0=t_0<t_1<\cdots<t_m=T,
\]
where each \(W_j\) is given by a \emph{definable formula} (o-minimal or Denef--Pas policy). This ensures finite enumeration
of events and partitions in each bounded window.

\paragraph{Discretization and measurement policy (declared).}
A run is indexed by refinement parameters
\[
  \mathfrak{r}=(\tau,\varepsilon,h,\Delta t,\{\alpha_\ell\}),
\]
where \(\tau\) is the truncation threshold, \(\varepsilon\) is optional smoothing scale, \(h\) is spatial grid scale,
\(\Delta t\) is time step, and \(\{\alpha_\ell\}\) is the filtration threshold set for persistence construction.
All consequences are conditioned on the declared refinement schedule and test suite in NS.7.

% -------------------------
\subsection*{NS.3. Realization into persistence (policy-bound)}
% -------------------------

\paragraph{Observable field.}
Let \(\omega=\nabla\times u\) be vorticity, and define a scalar field
\[
  f_t(x):=|\omega(x,t)|.
\]
Optionally apply smoothing \(f_t^{(\varepsilon)}:=\eta_\varepsilon * f_t\). Any smoothing sensitivity is charged to the ledger
as measurement error.

\paragraph{Superlevel filtration and persistence object.}
Fix filtration thresholds \(\{\alpha_\ell\}_{\ell=1}^L\) and define superlevel sets
\[
  X_{\alpha_\ell}(t):=\{x:\ f_t^{(\varepsilon)}(x)\ge \alpha_\ell\}.
\]
A \textbf{declared} realization policy produces a constructible persistence object
\[
  F(t)=\mathfrak{P}_{\mathrm{NS}}(u(t);\varepsilon,h,\{\alpha_\ell\})\ \in\ \Pers_{k}^{\mathrm{cons}}.
\]
All invariants used below are evaluated on \(\mathbf{T}_\tau F(t)\), never on \(F(t)\) directly.

\paragraph{After-collapse tower sampling.}
Choose sampling times \(0\le s_0<s_1<\cdots<s_N\le T\). Consider the truncated tower on the B-side:
\[
  \mathbf{T}_\tau F(s_0)\ \to\ \mathbf{T}_\tau F(s_1)\ \to\ \cdots\ \to\ \mathbf{T}_\tau F(s_N).
\]
The tower policy (what is compared, how colimits/terminal maps are formed) is exactly the v17.0 canon.

% -------------------------
\subsection*{NS.4. Diagnostics, Type IV rule, and defect potential (after-collapse)}
% -------------------------

\paragraph{Tower diagnostics (after \(\mathbf{T}_\tau\)).}
Compute the v17.0 tower diagnostics \((\mu_{\mathrm{NS}},\nu_{\mathrm{NS}})\) from the truncated tower.
By canon:
\[
  \text{Normal} \iff (\mu_{\mathrm{NS}},\nu_{\mathrm{NS}})=(0,0),\qquad
  \text{Type~IV} \iff (\mu_{\mathrm{NS}},\nu_{\mathrm{NS}})\neq(0,0),
\]
and the computation is performed strictly on \(\mathbf{T}_\tau F(\cdot)\).

\paragraph{Ledger accounting in a quantale.}
Let \(\mathsf{V}\) be a declared commutative quantale with operation \(\oplus\) and order \(\le_{\mathsf{V}}\).
Define ledger components (minimum set for NSE):
\begin{itemize}\itemsep2pt
  \item \(\delta^{\mathrm{meas}}\): measurement/smoothing/thresholding sensitivity,
  \item \(\delta^{\mathrm{disc}}\): discretization error (grid/time step/tower sampling),
  \item \(\delta^{\mathrm{alg}}\): algorithmic/implementation variance and finite arithmetic parts required by control tests,
  \item \(\delta^{\mathrm{lift}}\): optional lifting cost (only if Lifter commits; otherwise \(0_{\mathsf{V}}\)).
\end{itemize}
Aggregate per state/window by
\[
  \Sigma\delta:=\delta^{\mathrm{meas}}\ \oplus\ \delta^{\mathrm{disc}}\ \oplus\ \delta^{\mathrm{alg}}\ \oplus\ \delta^{\mathrm{lift}}.
\]

\paragraph{Gap and admissibility (after-collapse).}
A declared policy provides an admissible margin \(\mathrm{gap}_\tau\) (possibly windowwise):
\[
  \Sigma\delta\ <_{\mathsf{V}}\ \mathrm{gap}_\tau
  \quad\Longrightarrow\quad \text{eligible for B-Gate}^{+}\ \text{attempt}.
\]
The specific construction of \(\mathrm{gap}_\tau\) is a policy obligation (NS''.3) and must be logged.

\paragraph{Defect potential (exploration score; never theorem-level).}
Define an exploration potential \(\Phi_{\mathrm{NS},\tau}\) on the after-collapse state, e.g.
\begin{equation}\label{eq:Phi-NS}
\begin{aligned}
\Phi_{\mathrm{NS},\tau}
&:= w_{\mu} \mu_{\mathrm{NS}} \\
&\quad + w_{\nu} \nu_{\mathrm{NS}} \\
&\quad - w_{\delta} \| \Sigma\delta \|_{\mathsf{V}}
\end{aligned}
\end{equation}
with declared weights and declared reporting norm \(\|\cdot\|_{\mathsf{V}}\).
\textbf{No gate may depend solely on \(\Phi_{\mathrm{NS},\tau}\);} it is used for navigation/prioritization only.

% -------------------------
\subsection*{NS.5. Regimes and permitted outputs}
% -------------------------

\paragraph{Regime labeling (after-collapse).}
All regime labels use the B-side values:
\begin{itemize}\itemsep2pt
  \item \textbf{Plain:} \((\mu_{\mathrm{NS}},\nu_{\mathrm{NS}})=(0,0)\) and \(\Sigma\delta<_{\mathsf{V}}\mathrm{gap}_\tau\).
  \item \textbf{Noise:} \((\mu_{\mathrm{NS}},\nu_{\mathrm{NS}})=(0,0)\) but margin fails: \(\mathrm{gap}_\tau\le_{\mathsf{V}}\Sigma\delta\).
  \item \textbf{Singular (exploration label):} \((\mu_{\mathrm{NS}},\nu_{\mathrm{NS}})\neq(0,0)\) and/or declared singular score exceeded.
\end{itemize}

\paragraph{Permissible outputs (only).}
This appendix allows the following outcomes:
\begin{itemize}\itemsep3pt
  \item \textbf{Outcome A (negative evidence in class):} within declared \(\Theta\) and declared refinement sweeps,
  no robust Type~IV emergence beyond ledger inflation; all candidates fail robustness or anti-artifact tests.
  \item \textbf{Outcome B (audited candidate):} a robust Type~IV candidate \(\theta^\star\) is produced with full reproducibility bundle,
  passing anti-artifact tests and remaining Type~IV under refinement, \emph{within the declared class}.
  \item \textbf{Outcome C (barrier/terminal):} lifting attempts and restarts fail within budgets; output a terminal barrier instance
  with complete artifacts (useful for B3-style disproof search, still \textbf{[Spec]}).
\end{itemize}

% ============================================================
\section*{Appendix NS (continued): Refinement, Gates, and Reproducibility \textnormal{[Spec]}}
\phantomsection
\addcontentsline{toc}{section}{Appendix NS (continued). Refinement, Gates, and Reproducibility [Spec]}
\refstepcounter{section}
\label{app:NS-continued}

% -------------------------
\subsection*{NS.6. Refinement sweeps and robust Type IV rule}
% -------------------------

\paragraph{Refinement schedule.}
A refinement sweep is a sequence \(\{\mathfrak{r}_m\}_{m\ge 0}\) with
\[
  h_{m+1}<h_m,\quad \Delta t_{m+1}<\Delta t_m,\quad \varepsilon_{m+1}\le \varepsilon_m,
\]
and either fixed \(\tau\) or a declared \(\tau\)-grid sweep (always gating after-collapse).

\paragraph{Robust Type IV candidate (strict rule).}
A candidate \(\theta^\star\) at fixed \(\tau\) is \emph{robust} only if there exists a refinement sweep \(\mathfrak{r}_m\) such that:
\begin{itemize}\itemsep3pt
  \item (R1) \textbf{Persistence of Type IV:} \((\mu_{\mathrm{NS}},\nu_{\mathrm{NS}})(\theta^\star;\mathfrak{r}_m)\neq(0,0)\) for all sufficiently large \(m\),
  computed after \(\mathbf{T}_\tau\).
  \item (R2) \textbf{Ledger separation:} a declared normalized score separating defect growth from ledger inflation is non-decreasing
  (e.g. \(\mu_{\mathrm{NS}}/\|\Sigma\delta\|_{\mathsf{V}}\), \(\nu_{\mathrm{NS}}/\|\Sigma\delta\|_{\mathsf{V}}\), or a policy-defined alternative).
  \item (R3) \textbf{Anti-artifact tests pass:} NS.7 test suite is passed at each sufficiently refined level.
  \item (R4) \textbf{Replayability:} identical manifest + seed reproduces the candidate and its certificates.
\end{itemize}

% -------------------------
\subsection*{NS.7. Anti-artifact test suite (mandatory)}
% -------------------------

\paragraph{T1: Filtration orientation sanity.}
Compare the superlevel filtration of \(f\) with an equivalent dual construction (e.g. sublevel of \(-f\)) under the same policy.
Inconsistencies are charged to \(\delta^{\mathrm{alg}}\) and invalidate robustness.

\paragraph{T2: Smoothing sensitivity.}
Scan \(\varepsilon\) in a declared band \([\varepsilon_{\min},\varepsilon_{\max}]\). If regime labels or Type~IV status are not stable,
charge \(\delta^{\mathrm{meas}}\) and invalidate robustness.

\paragraph{T3: Grid/time convergence.}
Run a convergence check across \((h,\Delta t)\) refinement. If diagnostics change qualitatively under refinement,
charge \(\delta^{\mathrm{disc}}\) and invalidate robustness.

\paragraph{T4: Null-model controls.}
Require calibration on regimes expected to be benign within the interface:
2D NSE (global regular), 3D small-data regimes, and randomized low-vorticity controls.
Failure indicates miscalibration and invalidates the run.

% -------------------------
\subsection*{NS.8. Gates and agent actions (NSE instance; after-collapse)}
% -------------------------

\paragraph{Gate Cascade (policy, after-collapse).}
For each window \(W\) and candidate \(\theta\), apply:
\[
  \text{B-Gate}^{+}\ \to\ \text{PF/BC}\ \to\ \text{Overlap Gate}.
\]
All inputs are computed on \(\mathbf{T}_\tau F\). No pre-collapse value may enter.

\paragraph{B-Gate\(^{+}\) (eligibility).}
Attempt B-Gate\(^{+}\) only if:
\[
  (\mu_{\mathrm{NS}},\nu_{\mathrm{NS}})=(0,0)\ \ \text{and}\ \ \Sigma\delta<_{\mathsf{V}}\mathrm{gap}_\tau.
\]
Otherwise record fail reason (margin fail or Type~IV veto).

\paragraph{PF/BC (windowwise stability under policies).}
Check policy-functoriality / bar-compatibility \emph{after-collapse} (as defined in v17.0).
Any detected drift must be charged to \(\delta^{\mathrm{disc}}\oplus\delta^{\mathrm{meas}}\) and re-tested.

\paragraph{Overlap Gate (Control\(\Rightarrow\)Overlap).}
Only after PF/BC passes, verify overlap consistency across adjacent windows and enforce arithmetic control only through overlap.
Log the finite arithmetic part as \(\delta^{\mathrm{alg}}\) and record overlap certificates.

\paragraph{Restart and Summability.}
If repeated margin failures occur, invoke Restart Logic (windowwise) and enforce Summability under \(\mathsf{V}\).
Stability bands (intervals of \(\tau\) or refinement levels where decisions are constant) must be recorded.

% ============================================================
\section*{Appendix NS': Toward a Soundness Layer (Obligations) \textnormal{[Spec]}}
\phantomsection
\addcontentsline{toc}{section}{Appendix NS'. Soundness Layer (Obligations) [Spec]}
\refstepcounter{section}
\label{app:NS-prime}

\paragraph{Purpose.}
This section records theory-facing obligations required to make the NSE interface \emph{sound enough}
that exploration evidence meaningfully tracks analytic behavior. These are \textbf{obligations}, not achieved theorems.

% -------------------------
\subsection*{NS'.1. Stability obligations for the realization}
% -------------------------

\begin{auditobligation}[O1: Stable PDE-to-persistence realization]
Construct \(\mathfrak{P}_{\mathrm{NS}}\) so that a declared stability bound holds, e.g.
\[
  d_{\mathrm{int}}(\mathfrak{P}(f),\mathfrak{P}(g))\ \le\ \|f-g\|_{L^\infty},
\]
or an equivalent bound compatible with the v17.0 persistence metric and truncation policy.
\end{auditobligation}

\begin{auditobligation}[O1b: Time regularity \(\Rightarrow\) tower sampling control]
If \(\|f_t^{(\varepsilon)}-f_s^{(\varepsilon)}\|_{L^\infty}\le L|t-s|\), prove that
\(t\mapsto \mathfrak{P}_{\mathrm{NS}}(u(t))\) is Lipschitz in \(d_{\mathrm{int}}\), so sampling error is chargeable to \(\delta^{\mathrm{disc}}\).
\end{auditobligation}

% -------------------------
\subsection*{NS'.2. Analytic-to-defect bridge targets (policy goals)}
% -------------------------

\paragraph{Dissipation-to-defect target.}
For a window \(W=[t_0,t_1)\), set \(\mathcal{E}_{\mathrm{diss}}(W)=\nu\int_{t_0}^{t_1}\|\omega\|_{L^2}^2dt\).
Target: bound after-collapse defect growth by
\[
  \mu_{\mathrm{NS}}+\nu_{\mathrm{NS}}\ \le\ C(\tau)\cdot \mathcal{E}_{\mathrm{diss}}(W)\ +\ \text{(explicit ledger terms)}.
\]

\paragraph{Total persistence vs analytic norms (target).}
Define \(\mathrm{TP}_p(F):=\sum_{b\in\mathrm{Bar}(F)}\ell(b)^p\).
Target: \(\mathrm{TP}_p(\mathfrak{P}(f))\le C(p)\|\nabla f\|_{L^p}^p\) for appropriate \(p\) under declared filtrations.

\paragraph{Truncation-compatible target.}
Define \(\mathrm{TP}_p^{(\tau)}(F):=\sum_{b\in\mathrm{Bar}(\mathbf{T}_\tau F)}\ell(b)^p\).
Target: \(\mathrm{TP}_p^{(\tau)}(\mathfrak{P}(f))\le C(\tau,p)\mathcal{N}(f)\) for a stable analytic quantity \(\mathcal{N}(f)\).

\paragraph{Trimmed persistence \(\Rightarrow\) bounded tower defects (target).}
Target (policy): \(\sup_t \mathrm{TP}_p^{(\tau)}(F(t))<\infty\Rightarrow
\sup_t \mu_{\mathrm{NS}}(t;\tau),\sup_t \nu_{\mathrm{NS}}(t;\tau)<\infty\), modulo explicit ledger charges.

% ============================================================
\section*{Appendix NS'': Proof-first Program (Obligations and Loop) \textnormal{[Spec]}}
\phantomsection
\addcontentsline{toc}{section}{Appendix NS''. Proof-first Program (Obligations and Loop) [Spec]}
\refstepcounter{section}
\label{app:NS-dprime}

\subsection*{NS''.1. Core obligations (if one aims beyond exploration)}
\begin{auditobligation}[O2: Blow-up completeness within the interface]
If a strong solution loses regularity at \(T^\star\), show that for some declared \((\tau,\varepsilon)\),
\(\mu_{\mathrm{NS}}(t;\tau)\) or \(\nu_{\mathrm{NS}}(t;\tau)\) diverges as \(t\uparrow T^\star\), \emph{beyond ledger inflation}.
\end{auditobligation}

\begin{auditobligation}[O3: Analytic a priori control \(\Rightarrow\) bounded AK defects]
Using NSE inequalities and bridge targets in NS', prove uniform bounds on \(\mu_{\mathrm{NS}},\nu_{\mathrm{NS}}\) for fixed \(\tau\),
and quantify all remaining uncertainty as ledger terms.
\end{auditobligation}

\begin{auditobligation}[O4: Close the loop]
Combine O2 and O3 to rule out Type~IV within the interface, thereby excluding blow-up \emph{inside the declared framework}.
Any such result, if achieved, must be stated as a theorem in the main text, not in this appendix.
\end{auditobligation}

\subsection*{NS''.2. Calibration oracle checkpoint (mandatory for credibility)}
Require the pipeline to remain \textbf{Plain} on 2D NSE and on 3D small-data regimes within ledger budgets.
Treat failures as invalidation of the exploration configuration.

\subsection*{NS''.3. Gap estimation obligation (policy completeness)}
\begin{auditobligation}[O5: Construct \(\mathrm{gap}_\tau\)]
Provide a computable policy for \(\mathrm{gap}_\tau=\mathrm{gap}(\tau;T,\nu,\text{policy})\) from logged quantities,
controlling admissible ledger inflation for gate eligibility.
\end{auditobligation}

% ============================================================
\section*{NS Run Artifacts: Reproducibility Bundle \textnormal{[Spec]}}
\phantomsection
\addcontentsline{toc}{section}{NS Run Artifacts: Reproducibility Bundle [Spec]}
\refstepcounter{section}
\label{app:NS-run-artifacts}

\paragraph{Required directory layout.}
\begin{verbatim}
runs/
  NS-YYYYMMDD-HHMMSSZ_<shortid>/
    run.yaml
    ledger.json
    diag.csv
    proof.log
    artifacts/
      barcodes/
      diagrams/
      fields/
      checkpoints/
\end{verbatim}

\subsection*{Artifact A: run.yaml (template; canon-aligned)}
\begin{lstlisting}[language={},caption={NS Exploration Run Template: run.yaml}]
meta:
  run_id: "NS-YYYYMMDD-HHMMSSZ_ab12cd"
  created_utc: "YYYY-MM-DDTHH:MM:SSZ"
  engine: {name: "AK-HDPST", version: "v17.0", mode: "Exploration", spec_level: "[Spec]"}
  canon:
    after_collapse_only: true
    typeIV_rule: "(mu,nu)!=(0,0) after T_tau"
    gate_order: ["B_GatePlus","PF_BC","OverlapGate"]
    proxy_separation: true
    quantale_ledger: true

problem:
  equation: "3D incompressible NSE"
  domain: "T3"
  viscosity_nu: 0.001
  horizon: {t_start: 0.0, t_end: 2.0}
  function_space: {s: 3.0, divergence_free: true}

windows:
  base: ["W1","W2","W3","W4"]
  definable_formulas:
    - "W1: 0.0 <= t < 0.5"
    - "W2: 0.5 <= t < 1.0"
    - "W3: 1.0 <= t < 1.5"
    - "W4: 1.5 <= t < 2.0"

search_class:
  theta_dim: 18
  generator:
    name: "vortex_rings_collide"
    constraints: {enforce_div_free: true, energy_bound: true}

pde_solver:
  method: "pseudo_spectral"
  grid: {n: 256, aliasing: "2/3"}
  time_integration: {integrator: "RK3", dt: 1.0e-4}

observable:
  scalar_field: {definition: "|omega|"}
  smoothing: {enabled: true, eps: 0.015, ledger_tag: "delta_meas"}

persistence:
  category: "Pers_k^cons"
  filtration: {type: "superlevel", field: "f_t_eps"}
  thresholds: {count: 64}
  homology_degrees: [1,2]

truncation:
  T_tau: {enabled: true, tau: 0.02, policy: "fixed"}

tower:
  sampling: {n_times: 41}
  indicators:
    mu: {definition_policy: "v17_colim_to_terminal_cokernel", degree: 1}
    nu: {definition_policy: "v17_kernel_or_cokernel_variant", degree: 1}

ledger_policy:
  quantale: {name: "V_addxV_max", op: "product", order: "coordinatewise", report_norm: "L1"}
  components: ["delta_meas","delta_disc","delta_alg","delta_lift"]

gates:
  B_GatePlus:
    require: ["after_collapse_only","delta_total < gap_tau","(mu,nu)=(0,0)"]
  PF_BC:
    enforce: true
  OverlapGate:
    enforce: true
    control_implies_overlap: true

tests:
  anti_artifact_suite:
    T1_filtration_orientation: true
    T2_smoothing_sensitivity: true
    T3_grid_dt_convergence: true
    T4_null_model_controls: true
\end{lstlisting}

\subsection*{Artifact B: ledger.json (minimal schema; quantale)}
\begin{lstlisting}[basicstyle=\ttfamily\small,caption={Ledger Template: ledger.json}]
{
  "run_id":"NS-YYYYMMDD-HHMMSSZ_ab12cd",
  "quantale_policy":{"name":"V_addxV_max","op":"product","order":"coordinatewise","report_norm":"L1"},
  "delta_components":{
    "delta_meas":{"value":1.20,"confidence":0.7},
    "delta_disc":{"value":2.10,"confidence":0.6},
    "delta_alg":{"value":0.80,"confidence":0.8},
    "delta_lift":{"value":0.00,"confidence":1.0}
  },
  "delta_total":{"value":4.10}
}
\end{lstlisting}

\subsection*{Artifact C: diag.csv (schema; after-collapse only)}
\begin{lstlisting}[language={},caption={Diagnostics Time Series: diag.csv (schema)}]
run_id,theta_id,t,window,tau,eps,grid_n,dt,thresholds_count,mu,nu,phi_ns,delta_meas,delta_disc,delta_alg,delta_lift,delta_total,flags
\end{lstlisting}

\subsection*{Artifact D: proof.log (JSONL; anti-artifact and gate replay)}
\begin{lstlisting}[language={},caption={Test and Gate Output: proof.log (JSONL)}]
{"run_id":"...","stage":"B_GatePlus","status":"PASS","details":"..."}
{"run_id":"...","test":"T1_filtration_orientation","status":"PASS","details":"..."}
{"run_id":"...","test":"T3_grid_dt_convergence","status":"PASS","details":"..."}
{"run_id":"...","stage":"OverlapGate","status":"PASS","details":"..."}
\end{lstlisting}

\paragraph{Validity requirement.}
Any Type~IV claim (even as \textbf{[Spec]} evidence) is invalid without \texttt{run.yaml}, \texttt{ledger.json}, \texttt{diag.csv},
\texttt{proof.log}, and the referenced persisted artifacts sufficient for deterministic replay.



% -------------------------
% Concluding Remarks and Acknowledgments (v17.0, AK--HDPST + HDPS)
% Canon-aligned closing: after-collapse, Type IV meaning fixed, Core vs [Spec]
% -------------------------
\section*{Concluding Remarks and Acknowledgments (v17.0, AK--HDPST + HDPS)}
\phantomsection
\addcontentsline{toc}{section}{Concluding Remarks and Acknowledgments}

\paragraph{Standing scope, coefficients, windows, and UCC guard--rails.}
Unless stated otherwise, coefficients lie in a \emph{field} \(k\).
All \emph{Core} statements live in constructible one\hyp parameter persistence
over a field and, when realized, target \(D^{\mathrm{b}}(k\text{-mod})\).
In \textbf{[Spec]} appendices that use sheaf\hyp theoretic, arithmetic, or
Fukaya\hyp categorical realizations we write the coefficient field as
\(\Lambda\); this is only a notational change and does not alter the Core
category or its hypotheses.

Filtered (co)limits are computed \emph{objectwise} in
\([\mathbb{R},\mathsf{Vect}_\Lambda]\) under the scope policy
(Appendix~A) and returned to the constructible subcategory by verification or
by applying \(\mathbf{T}_\tau\) at the persistence layer.
All statements are made under the \emph{Unified Collapse Contract (UCC)}
(Thm.~\ref{thm:UCC}, Chapter~1): we work in constructible \(1\)D persistence,
with
\begin{itemize}
  \item a commutative Quantale \((\mathsf{V},\oplus,\le,0)\) enriching the time
  index and hosting all error budgets;
  \item \emph{right\hyp open}, MECE windows along a \(\tau\)\hyp sweep, which,
  when required, are \emph{Denef--Pas definable} to guarantee finite event sets
  and finite \v{C}ech depth (Appendix~Q);
  \item an \emph{after\hyp collapse policy}: all equalities, exactness claims,
  monotonicity statements, comparisons, and gluing are asserted only after
  applying \(\mathbf{T}_\tau\) at the persistence layer.
\end{itemize}
Within these guard\hyp rails, the \(\delta\)\hyp ledger aggregates all residuals
in \(\mathsf{V}\), is subadditive under composition, and is non\hyp increasing
under after\hyp collapse \(1\)\hyp Lipschitz post\hyp processing; this allows
\(\Sigma\delta\) to be used both as an audit budget and, in \textbf{[Spec]}
mode, as a scalar \emph{Defect Potential} \(\Phi_\tau\) for high\hyp dimensional
search (Ch.~13).

\paragraph{Canon commitments (v17.0).}
The following conventions are fixed throughout and are not overridden by any
application or search layer:
\begin{itemize}
  \item \textbf{After\hyp collapse semantics.} Any assertion that can affect a
  proof object (gate pass/fail, regime classification, certificate export) is
  evaluated on the B\hyp side \(\mathbf{T}_\tau(\mathbf{P}_i(F))\) at the
  persistence layer. Pre\hyp collapse readings may be recorded as diagnostics
  only, never as proof\hyp bearing evidence.
  \item \textbf{Type IV meaning (fixed).} ``Normal'' means
  \((\mu_{\mathrm{Collapse}},\nu_{\mathrm{Collapse}})=(0,0)\) and
  \textbf{Type~IV (invisible obstruction)} means
  \((\mu_{\mathrm{Collapse}},\nu_{\mathrm{Collapse}})\neq(0,0)\), computed
  \emph{after} \(\mathbf{T}_\tau\) under the declared tower policy.
  \item \textbf{No global equivalence.} We never assert a global equivalence
  \(\mathrm{PH}_1\Longleftrightarrow \Ext^1\). Only the one\hyp way bridge
  \(\mathrm{PH}_1\Rightarrow \Ext^1\) is Core, and any reverse direction is
  \textbf{[Spec]} and allowed only under explicit safety programs and tests.
  \item \textbf{Core vs \textbf{[Spec]} boundaries are auditable.} Any external
  realization (sheaf/arithmetic/Fukaya/spectral proxies) is either proved in
  the Core target \(D^{\mathrm b}(k\text{-mod})\) or labeled \textbf{[Spec]}
  with explicit \(\delta\)\hyp budgeting and gate restrictions.
\end{itemize}

\paragraph{What is proved (Core; F0--F6 \& P1--P10, UCC/bridge extensions).}
Within the above regime we establish machine\hyp checkable results and arrange
them for formalization.
\begin{itemize}
  \item \emph{Exact truncation and filtered lift (F1--F2).}
  The Serre reflector \(\mathbf{T}_\tau\) deletes precisely bars of length
  \(\le\tau\), is exact, idempotent, and \(1\)\hyp Lipschitz (indeed
  \(\mathsf{V}\)\hyp \(1\)\hyp Lipschitz under Quantale enrichment).
  It admits a filtered lift \(C_\tau\) unique up to f.q.i.\ with
  \(\mathbf{P}_i(C_\tau F)\cong \mathbf{T}_\tau(\mathbf{P}_iF)\)
  (Appendices~A/B).
  \item \emph{CNF and field edge (P1--P2).}
  After collapse, objects split in \(D^{\mathrm b}(k\text{-mod})\):
  \(X\simeq\bigoplus_i H^i(X)[-i]\), and
  \(\Ext^1(X,k)\cong\Hom(H^1(X),k)\).
  \item \emph{Bridge (one\hyp way) and scope of reverse.}
  Under a \(t\)\hyp exact realization of amplitude \(\le 1\),
  \(\mathrm{PH}_1(F)=0\Rightarrow \Ext^1(\mathcal{R}(F),k)=0\)
  (Appendix~C).
  Any reverse implication remains \textbf{[Spec]} and is admitted only through
  explicit safety programs with independent quantitative separation tests
  (Chapter~16; Appendix~W).
  \item \emph{Safe low\hyp pass (P4).}
  Even, mass\hyp \(1\), bandwidth \(\asymp\sqrt{\tau}\) kernels commute with
  \(\mathbf{T}_\tau\) up to a controlled \(\delta^{\mathrm{alg}}\) and keep
  \(\mathbf{T}_\tau\circ L_\tau\) \(1\)\hyp Lipschitz
  (Prop.~\ref{prop:lowpass}; Ch.~2/Appendix~E).
  \item \emph{Monotonicity vs.\ stability (P5).}
  Deletion\hyp type updates are non\hyp increasing after collapse; inclusion\hyp
  type updates are non\hyp expansive.
  \item \emph{Convergence Manager (P6).}
  For countable \emph{Denef--Pas} covers of finite \v{C}ech depth, the
  Quantale\hyp summed error satisfies \(\sum\delta<\infty\) and overlap gluing
  holds globally (Thm.~\ref{thm:dp-sum}; Appendices~J/Q).
  \item \emph{AWFS \(2\)\hyp cell additivity (P7).}
  \(2\)\hyp cell defects add subadditively in the Quantale (Ch.~5;
  Appendices~K/L).
  \item \emph{Gate calculus with cut elimination (P8).}
  The default cascade is operated as a sequent calculus with fixed ordering:
  \[
  E_1{=}0\ \Longrightarrow\ (\mu,\nu){=}(0,0)\ \Longrightarrow\
  \Ext^1{=}0\ \Longrightarrow\ \mathrm{PH}_1{=}0.
  \]
  Success later never overturns failure earlier.
  \item \emph{Stability bands (P9).}
  Open \(\tau\)\hyp intervals with \((\mu,\nu){=}(0,0)\) certify stability;
  Type\hyp IV is excluded in conjunction with P6 (Ch.~4; Appendix~D/J).
  \item \emph{Reproducibility theorem (P10).}
  From pass\hyp logs of
  \texttt{T-ExtZero-implies-PHZero},
  \texttt{T-Countable-Cover},
  \texttt{T-Delta-Sum-Converges},
  \texttt{T-Lipschitz-AfterCollapse},
  \texttt{T-Exactness-Persistence}
  (and arithmetic \texttt{T-Iwasawa-Alignment}),
  the P3/P6/P8 conclusions are mechanically reconstructed.
  \item \emph{UCC collapse nucleus and Quantale ledger (extension).}
  The Unified Collapse Contract (Thm.~\ref{thm:UCC}) upgrades
  \(\mathbf{T}_\tau\) to a \(\mathsf{V}\)\hyp nucleus and shows that the
  Quantale\hyp valued \(\delta\)\hyp ledger is subadditive, non\hyp increasing
  under after\hyp collapse post\hyp processing, and therefore sound as a
  global potential \(\Sigma\delta\) for both audit and (under \textbf{[Spec]}
  policies) navigation (Ch.~1/Appendix~S).
\end{itemize}

\paragraph{What is specified and how it is audited (\textup{[Spec]}).}
All \textbf{[Spec]} items are explicitly contracted to be \emph{non\hyp
expansive after truncation} and are audited by the windowed diagnostics
\((\mu,\nu)\) together with a Quantale\hyp valued \(\delta\)\hyp ledger.
In particular, \textbf{Type~IV is always a post\hyp \(\mathbf{T}_\tau\)
phenomenon} and never ``many short bars'' (short bars are deleted by design).
\begin{itemize}
  \item \emph{UCC search layer and dual mode.}
  The Quantale enrichment \((\mathsf{V},\oplus,\le,0)\), definable windows
  (Denef--Pas preferred), and the AWFS view
  \(\mathrm{Id}\Rightarrow L\dashv R\Rightarrow \mathrm{Id}\) with
  \(R=C_\tau\) (Ch.~1/5; Appendices~K/L) provide the \emph{Audit Mode} in which
  \(\Sigma\delta<\mathrm{gap}_\tau\) certifies validity.
  In \emph{Navigation Mode} (Part~II), the same \(\Sigma\delta\) is scalarized
  into a Defect Potential \(\Phi_\tau\) (with Type\hyp IV penalties) used by
  AI agents under Chapter~13, but never to override the audit gates.
  \item \emph{Mirror/Transfer pipelines.}
  A natural \(2\)\hyp cell
  \(\Mirror\circ C_\tau\Rightarrow C_\tau\circ\Mirror\) with uniform bounds
  \(\delta(i,\tau)\) yields \(\delta\)\hyp controlled commutation; bounds add
  along pipelines and are non\hyp increasing under \(1\)\hyp Lipschitz
  post\hyp processing (Appendix~L).
  \item \emph{Multi\hyp axis reflectors.}
  For exact reflectors from hereditary Serre subcategories, nesting is
  order\hyp independent; otherwise an A/B test with tolerance \(\eta\) and
  deterministic fallback is used.
  Residuals \(\Delta_{\mathrm{comm}}\) are recorded as \(\delta^{\mathrm{alg}}\)
  (Appendix~K).
  \item \emph{Arithmetic layers (SCTF/ECF/Iwasawa).}
  Local traces (Igusa/Tate) couple to post\hyp collapse measurements;
  discrepancies externalize to \(\delta_{\mathrm{alg}},\delta_{\mathrm{meas}}\).
  The \emph{Explicit\hyp Contract Formula} (ECF) enforces the safety\hyp side
  inequality
  \[
  \bigl|\mu_{\mathrm{Coll}}(W,\tau)
  -\langle \mathrm{Obs}(R_{\mathrm{spec}}(F),W),\varphi_\tau\rangle\bigr|
  \ \le\ \varepsilon_{\mathrm{tot}}(W,\tau),
  \]
  with RHS fully represented in the \(\delta\)\hyp ledger (Appendix~M).
  The \emph{Iwasawa Gate} aligns
  \((\mu_{\mathrm{Collapse}},\mu_{\mathrm{Iwasawa}})\) in a three\hyp state
  regime (lower bound / match / drift\hyp corrected) to suppress Type\hyp IV
  drift (Ch.~7; Appendix~R). Here \(\mu_{\mathrm{Collapse}}\) and the classical
  Iwasawa \(\mu\) are distinct invariants and are never identified.
  \item \emph{Fukaya realizations.}
  Action filtration yields constructible persistence on bounded windows;
  continuation is \(1\)\hyp Lipschitz and adding stops is deletion\hyp type
  (Appendix~O).
  \item \emph{PF/BC transport.}
  Projection formula/base change are transported only through the
  after\hyp collapse protocol; residual discretization/measurement slack is
  budgeted as \(\delta^{\mathrm{disc}},\delta^{\mathrm{meas}}\)
  (Appendix~N).
  \item \emph{HDPS engine, Validity Map, and AI agents.}
  Part~II and Appendices~U/V/W specify the \emph{High\hyp Dimensional
  Projection Search (HDPS)} layer:
  \begin{itemize}
      \item the Defect Potential \(\Phi_\tau\) and its stratification into
      Valid/Noise/Sing regimes and a terrain\hyp cell decomposition
      (Ch.~13; Appendix~V);
      \item the autonomous agents Hunter/Mapper/Lifter with formally defined
      operational semantics and the Hunter Action Log schema
      (Ch.~14--15; Appendix~U);
      \item the \emph{Validity Map} and \emph{Global Certificate} as verifiable
      graph structures (Appendix~V);
      \item the Bridge Programs B1/B2/B3 and the Spectral\hyp Gap Condition,
      which govern when a local reverse implication
      \(\Ext^1\Rightarrow \mathrm{PH}_1\) is permitted under \textbf{[Spec]}
      guard\hyp rails (Ch.~16; Appendix~W).
  \end{itemize}
  All such search\hyp side components are constrained by the UCC budget:
  they may propose paths and lifts, but the Core vetoes any step with
  \(\Sigma\delta\ge\mathrm{gap}_\tau\).
  \item \emph{Case studies (templates only).}
  Application templates---e.g.\ the Navier--Stokes case study
  (Appendix~NS)---translate classical questions into the AK\hyp HDPST language
  of truncation, tower diagnostics, Type~IV veto, and reproducible evidence.
  They remain \textbf{[Spec]} and do not alter the Core theorems.
\end{itemize}

\paragraph{Operational pipeline (end\hyp to\hyp end, canon order).}
Per window and degree: enforce \(\mathrm{B\text{--}Gate}^{+}\) with safety
margin \(\Sigma\delta<\mathrm{gap}_\tau\) and the \textbf{Type~IV veto}
\((\mu,\nu)=(0,0)\) (both evaluated after \(\mathbf{T}_\tau\)); then apply the
post\hyp collapse \emph{PF/BC} checks; and finally apply the \emph{Overlap
Gate} (post\hyp collapse equivalence up to budget, \v{C}ech control, and
alignment constraints).
Across windows, Restart/Summability (\(\kappa\)\hyp restart and
\(\sum\delta<\infty\)) paste local certificates into global ones
(Appendix~J), yielding a verifiable global object (Appendix~V).
A single Quantale\hyp sum \(\oplus\) aggregates pipeline budgets:
Mirror--Collapse bounds \(\delta(i,\tau)\), A/B residuals
\(\Delta_{\mathrm{comm}}\), discretization and measurement terms, and, in HDPS
mode, lifting penalties \(\delta^{\mathrm{lift}}\).

\paragraph{Reproducibility, formalization, and tests (\texttt{run.yaml} v17.0).}
Appendix~G specifies the manifest \texttt{run.yaml} (Quantale, definable window
formulae, AWFS/\(2\)\hyp cell bounds, \(\tau\)\hyp sweeps, spectral and
lifting policy, search strategy) and artifact schemas with canonical
serialization and cross\hyp linked hashes.
Appendix~F outlines a Lean/Coq pathway for Core components (Serre
localization; \(1\)\hyp Lipschitz; comparison maps and \((\mu,\nu)\); CNF and
field edge; the one\hyp way bridge; and API stubs for budgets and controlled
commutation).
Appendix~U prescribes the Hunter Action Log and Proposer/Verifier split;
Appendix~V formalizes the Global Certificate; Appendix~W the Bridge Programs.
Chapter~12 provides tests for \(\mathsf{V}\)\hyp Lipschitz laws, definable
coverage/finite \v{C}ech depth, deletion\hyp type monotonicity, filtered\hyp
colimit behavior, Mirror/tropical pipelines, A/B soft\hyp commuting,
Restart/Summability, arithmetic alignment
(\texttt{T-Iwasawa-Alignment}), and---in HDPS mode---stability of
\(\Phi_\tau\) under resolution changes.
Optional tests include \texttt{T-PFBC-AfterCollapse} and
\texttt{T-}\(\Lambda_{\!len}\).

\paragraph{Limitations and guard\hyp rails.}
All claims are confined to the implementable persistence/spectral/categorical
layers \emph{after collapse}.
No claim of a global equivalence \(\mathrm{PH}_1\Leftrightarrow \Ext^1\) is
made; only the one\hyp way implication under (B1)--(B3) and a locally
certified reverse under the Spectral\hyp Gap Condition are used, and then only
within definable windows and explicit budgets.
Spectral indicators are not f.q.i.\ invariants; they are evaluated under fixed
policies with deletion\hyp type monotonicity and general non\hyp expansiveness.
The collapse diagnostic \(\mu_{\mathrm{Collapse}}\) is distinct from the
classical Iwasawa \(\mu\).
The HDPS/AI layer (Hunter/Mapper/Lifter, Bridge Programs, Validity Maps, case
studies) is entirely \textbf{[Spec]}: it can \emph{propose} search trajectories
and certificates, but only the Core/UCC can \emph{certify} them.

\paragraph{Outlook.}
Future work will refine quantitative links between persistence energies and
spectral tails, broaden verifiable criteria for stability bands and
\(\DiagZero\), and extend formal libraries (shift/interleaving; PF/BC
transport; budget calculus) together with domain templates
(arithmetic/Langlands/PDE/Fukaya) equipped with auditable \(\delta\)\hyp
controls.
On the HDPS side, further development of Bridge Programs, Validity Maps, and
domain\hyp specific case studies (NSE, BSD, RH, Langlands) may clarify when
AI\hyp assisted exploration can be safely promoted to Core\hyp level proofs.

% =========================================================================
% References (minimal; standard facts)
% =========================================================================
\addcontentsline{toc}{section}{References}
\begin{thebibliography}{99}

\bibitem{EdelsbrunnerHarer}
H.~Edelsbrunner and J.~Harer,
\emph{Computational Topology: An Introduction},
American Mathematical Society, 2010.

\bibitem{ChazalBook}
F.~Chazal, V.~de~Silva, M.~Glisse, and S.~Oudot,
\emph{The Structure and Stability of Persistence Modules},
SpringerBriefs in Mathematics, Springer, 2016.

\bibitem{LesnickInterleaving}
M.~Lesnick,
The Theory of the Interleaving Distance on Multidimensional Persistence Modules,
\emph{Foundations of Computational Mathematics} \textbf{15} (2015), no.~3, 613--650.

\bibitem{WeibelHA}
C.~A.~Weibel,
\emph{An Introduction to Homological Algebra},
Cambridge University Press, 1994.

\bibitem{GabrielZisman}
P.~Gabriel and M.~Zisman,
\emph{Calculus of Fractions and Homotopy Theory},
Springer, 1967.

\bibitem{vanDenDries}
L.~van~den~Dries,
\emph{Tame Topology and o-minimal Structures},
Cambridge University Press, 1998.

\bibitem{RosenthalQuantales}
K.~I.~Rosenthal,
\emph{Quantales and Their Applications},
Pitman Research Notes in Mathematics Series 234, Longman, 1990.

\bibitem{GarnerSOA}
R.~Garner,
Understanding the Small Object Argument,
\emph{Applied Categorical Structures} \textbf{17} (2009), no.~3, 247--285.

\end{thebibliography}

\paragraph{Acknowledgments.}
The manuscript was prepared by the author, with assistance from an AI tool
(ChatGPT) through an iterative workflow; any remaining errors are the author’s
responsibility.

\medskip
\noindent\textbf{Final note.}
The separation between the provable \emph{Core} and auditable \textbf{[Spec]}
contracts, together with the after\hyp collapse order, Quantale\hyp aggregated
\(\delta\)\hyp budgets, Restart/Summability, and the HDPS engine, provides a
conservative and extensible methodology for cross\hyp domain reuse within the
guard\hyp rails of \textbf{v17.0 (AK--HDPST + HDPS)}.




\end{document}