\RequirePackage[2023-11-01]{latexrelease}

\documentclass[11pt]{article}

% --- (NEW) Math alphabet reservation cut (must be BEFORE ams/bm) ---
\newcommand\hmmax{0}
\newcommand\bmmax{0}

% --- Encoding & Language ---
% [XeLaTeX] utf8 はネイティブ対応なので inputenc は不要・非推奨
% \usepackage[utf8]{inputenc} % <-- 削除
\usepackage[T1]{fontenc}
\usepackage[english]{babel}
\usepackage{geometry}
\geometry{margin=1in}

% --- Core Math & Utilities (AMS first) ---
\usepackage{amsmath,amssymb,amsthm,amsfonts,mathtools}
\usepackage{mathrsfs}
\usepackage{bm}
\usepackage{stmaryrd}
\usepackage{changepage}
\usepackage{amscd}
\usepackage{textcomp}
\usepackage{etoolbox} % for \forcsvlist
\usepackage{enumitem}
\usepackage{array}

% --- Fonts (XeLaTeX でも利用可能な pdfLaTeX 系 Times セット) ---
\usepackage{newtxtext,newtxmath}  % Times-like text & math
\usepackage{inconsolata}          % Monospace for listings
\DeclareTextFontCommand{\textttit}{\ttfamily\slshape} % monospace "italic-like"

% --- Diagrams ---
\usepackage{tikz, tikz-cd}
\usetikzlibrary{
  matrix, arrows.meta, cd, calc, positioning, quotes,
  decorations.pathmorphing, decorations.markings,
  shapes.misc, shapes.geometric, arrows, fit, backgrounds, fadings
}
\usepackage{pgfplots}
\pgfplotsset{compat=1.18}
\usepackage[all,cmtip]{xy}
\numberwithin{equation}{section}
\theoremstyle{plain}
\theoremstyle{definition}
\theoremstyle{remark}

% --- Tables ---
\usepackage{array,tabularx,booktabs}
\newcolumntype{L}[1]{>{\raggedright\arraybackslash}p{#1}}
\newcolumntype{C}[1]{>{\centering\arraybackslash}p{#1}}
\newcolumntype{R}[1]{>{\raggedleft\arraybackslash}p{#1}}
\newcolumntype{Y}{>{\raggedright\arraybackslash}X}

% --- Monospace token helper (safe in math) ---
\newcommand{\fname}[1]{\text{\texttt{#1}}}

% --- Listings / Code ---
\usepackage{xcolor}
\usepackage{listings}
\usepackage{float}
\lstset{
  basicstyle=\ttfamily\small,
  keywordstyle=\color{blue},
  commentstyle=\color{gray},
  stringstyle=\color{orange},
  frame=single,
  breaklines=true,
  showstringspaces=false,
  captionpos=b,
  xleftmargin=1em,
  columns=fullflexible,
  upquote=true,
  mathescape=false,
  literate=*%
    {∀}{{$\forall$}}1 {∃}{{$\exists$}}1
    {→}{{$\to$}}1 {←}{{$\leftarrow$}}1 {⟶}{{$\longrightarrow$}}1 {⇒}{{$\Rightarrow$}}1 {⇔}{{$\Leftrightarrow$}}1 {↦}{{$\mapsto$}}1 {⥤}{{$\to$}}1
    {α}{{$\alpha$}}1 {β}{{$\beta$}}1 {γ}{{$\gamma$}}1 {δ}{{$\delta$}}1 {ε}{{$\varepsilon$}}1
    {λ}{{$\lambda$}}1 {μ}{{$\mu$}}1 {ν}{{$\nu$}}1 {π}{{$\pi$}}1 {τ}{{$\tau$}}1
    {κ}{{$\kappa$}}1 {θ}{{$\theta$}}1 {η}{{$\eta$}}1 {φ}{{$\varphi$}}1
    {ι}{{$\iota$}}1 {ρ}{{$\rho$}}1 {χ}{{$\chi$}}1 {ω}{{$\omega$}}1
    {Γ}{{$\Gamma$}}1 {Δ}{{$\Delta$}}1 {Π}{{$\Pi$}}1 {Σ}{{$\Sigma$}}1 {Ω}{{$\Omega$}}1 {Λ}{{$\Lambda$}}1 {Φ}{{$\Phi$}}1 {Ψ}{{$\Psi$}}1
    {ℕ}{{$\mathbb{N}$}}1 {ℤ}{{$\mathbb{Z}$}}1 {ℚ}{{$\mathbb{Q}$}}1 {ℝ}{{$\mathbb{R}$}}1
    {≤}{{$\le$}}1 {≥}{{$\ge$}}1 {≠}{{$\ne$}}1 {∈}{{$\in$}}1 {⊆}{{$\subseteq$}}1
    {∧}{{$\land$}}1 {∨}{{$\lor$}}1
    {⪯}{{$\preceq$}}1 {⪰}{{$\succeq$}}1 {≡}{{$\equiv$}}1 {≅}{{$\cong$}}1 {⋙}{{$\ggg$}}1
    {⊕}{{$\oplus$}}1 {⊗}{{$\otimes$}}1 {⊙}{{$\odot$}}1 {⊣}{{$\dashv$}}1 {⊢}{{$\vdash$}}1 {⊥}{{$\perp$}}1
    {·}{{$\cdot$}}1 {∙}{{$\cdot$}}1 {⋅}{{$\cdot$}}1 {•}{{$\bullet$}}1
    {・}{{$\cdot$}}1 {‧}{{$\cdot$}}1
    {×}{{$\times$}}1 {⨯}{{$\times$}}1
    {…}{{$\ldots$}}1 {–}{{--}}1 {—}{{---}}1 {∞}{{$\infty$}}1
    {₁}{{$_1$}}1 {⁺}{{${}^{+}$}}1
}

% --- Math Operators ---
\DeclareMathOperator{\Ext}{Ext}
\DeclareMathOperator{\Hom}{Hom}
\DeclareMathOperator{\Spec}{Spec}
\DeclareMathOperator{\colim}{colim}
\DeclareMathOperator{\PH}{PH}
\DeclareMathOperator{\Tor}{Tor}
\DeclareMathOperator{\rank}{rank}
\DeclareMathOperator{\im}{im}
\DeclareMathOperator{\id}{id}
\DeclareMathOperator{\Ker}{Ker}
\DeclareMathOperator{\Coker}{Coker}
\DeclareMathOperator{\Collapse}{Collapse}
\DeclareMathOperator{\Mot}{Mot}
\DeclareMathOperator{\GL}{GL}
\DeclareMathOperator{\RHom}{RHom}
\DeclareMathOperator{\HT}{HT}
\DeclareMathOperator{\gdim}{gdim}

% --- Minimal hyphenat compatibility (no package) ---
\DeclareRobustCommand{\hyp}{\nobreakdash-} % "one\hyp way"
\providecommand{\NoHyphens}[1]{\mbox{#1}}   % light-weight shim
\providecommand{\nohyphens}[1]{\mbox{#1}}

% --- General Macros ---
\newcommand{\QQ}{\mathbb{Q}}
\newcommand{\RR}{\mathbb{R}}
\newcommand{\CC}{\mathbb{C}}
\newcommand{\ZZ}{\mathbb{Z}}
\newcommand{\TT}{\mathbb{T}}
\newcommand{\CollapseZone}{\mathfrak{C}}
\newcommand{\cF}{\mathcal{F}}
\newcommand{\cG}{\mathcal{G}}
\newcommand{\cE}{\mathcal{E}}
\newcommand{\cO}{\mathcal{O}}
\newcommand{\cD}{\mathcal{D}}
\newcommand{\cH}{\mathcal{H}}
\newcommand{\into}{\hookrightarrow}
\newcommand{\onto}{\twoheadrightarrow}
\newcommand{\eps}{\varepsilon}
\newcommand{\Sha}{\mathcal{X}}
\newcommand{\CollapseCompatible}{\mathsf{CollapseCompatible}}
\newcommand{\CollapseTypeOf}[1]{\operatorname{CollapseType}(#1)}
\newcommand{\CollapseImage}{\operatorname{CollapseImage}}
\newcommand{\CollapseEnergy}{\mathcal{E}_{\mathrm{Coll}}}
\newcommand{\ptorsionfree}[1]{\ensuremath{#1}\nobreakdash-torsion-free}
\newcommand{\Pers}{\mathsf{Pers}}
\newcommand{\Vect}{\mathsf{Vect}}
\newcommand{\Ho}{\mathrm{Ho}}
\newcommand{\dq}{\textquotedbl}
\newcommand{\rH}{\mathrm{H}}
\newcommand{\timevar}{t}
\newcommand{\T}{\mathsf{T}}
\newcommand{\C}{\mathsf{C}}
\newcommand{\Coll}{\mathsf{C}}
\newcommand{\Rfun}{\mathcal{R}}

\theoremstyle{plain}
\theoremstyle{definition}

\newcommand{\kk}{k}
\newcommand{\xto}{\xrightarrow}
\newcommand{\xtoH}{\xRightarrow{\ \simeq\ }}
\newcommand{\xtoM}{\xRightarrow{\ \mono\ }}
\newcommand{\mono}{\mathrm{mono}}
\newcommand{\epi}{\mathrm{epi}}
\newcommand{\DD}{\mathcal{D}}
\newcommand{\cC}{\mathcal{C}}
\newcommand{\supp}{\mathrm{supp}}
\newcommand{\PE}{\mathrm{PE}}
\newcommand{\betti}{\beta}
\newcommand{\CT}{C_{\tau}}
\newcommand{\dint}{d_{\mathrm{int}}}

% --- Theorem-like Environments ---
\numberwithin{equation}{section}
\newtheorem{theorem}{Theorem}[section]
\newtheorem{proposition}[theorem]{Proposition}
\newtheorem{lemma}[theorem]{Lemma}
\newtheorem{corollary}[theorem]{Corollary}
\newtheorem{axiom}{Axiom}[section]
\newtheorem{conjecture}{Conjecture}[section]
\theoremstyle{definition}
\newtheorem{definition}[theorem]{Definition}
\newtheorem{example}[theorem]{Example}
\newtheorem{remark}[theorem]{Remark}
\newtheorem{specification}[theorem]{Specification}
\newtheorem{declaration}[theorem]{Declaration}
\newtheorem{interpretation}[theorem]{Interpretation}
% --- Unified notation ---
\DeclareRobustCommand{\FiltCh}[1]{\mathsf{FiltCh}(#1)}
\DeclareRobustCommand{\Perskft}{\Pers^{\mathrm{cons}}_{k}}
\DeclareRobustCommand{\Pk}{\mathbf{P}}
\DeclareRobustCommand{\Ttau}{\texorpdfstring{\ensuremath{\mathbf{T}_{\tau}}}{T\_\tau}}
\DeclareRobustCommand{\Ctau}{\texorpdfstring{\ensuremath{C_{\tau}}}{C\_\tau}}
\DeclareRobustCommand{\Dbplain}{D^{\mathrm{b}}}
\DeclareRobustCommand{\Dbkmod}{D^{\mathrm{b}}(k\text{-mod})}
\DeclareRobustCommand{\Db}{D^{\mathrm{b}}(k\text{-mod})}
\DeclareRobustCommand{\muc}{\mu_{\mathrm{Collapse}}}
\DeclareRobustCommand{\nuc}{\nu_{\mathrm{Collapse}}}
\DeclareRobustCommand{\fqi}{\text{f.q.i.}}
\DeclareRobustCommand{\LC}{\texorpdfstring{\ensuremath{\mathrm{(LC)}}}{(LC)}}
\DeclareRobustCommand{\Qtest}{\{\,k[0]\,\}}
\DeclareRobustCommand{\pos}[1]{\left(#1\right)_{+}}
\DeclareRobustCommand{\Trop}{\texorpdfstring{\ensuremath{\operatorname{Trop}}}{Trop}}
\DeclareRobustCommand{\Mirror}{\texorpdfstring{\ensuremath{\operatorname{Mirror}}}{Mirror}}
\newcommand{\ProofBadge}{\textsf{\footnotesize[Proof]}}
\newcommand{\SpecBadge}{\textsf{\footnotesize[Spec]}}
\DeclarePairedDelimiter{\norm}{\lVert}{\rVert}
\numberwithin{equation}{section}
\theoremstyle{plain}
\theoremstyle{definition}
\theoremstyle{remark}

% Macros
\newcommand{\bbR}{\mathbb{R}}
\newcommand{\bbN}{\mathbb{N}}
\newcommand{\Vectk}{\mathsf{Vect}_k}
\newcommand{\Dbk}{D^{\mathrm{b}}(k\text{-mod})}
\newcommand{\Perscons}{\Pers^{\mathrm{cons}}_k}
\newcommand{\Persft}{\Pers^{\mathrm{ft}}_k}
\newcommand{\Crop}{\mathbf{W}}
\newcommand{\Res}{\mathrm{Res}}
\newcommand{\intdist}{d_{\mathrm{int}}}
\newcommand{\Ecat}{\mathsf{E}_\tau}
\newcommand{\Orth}{(\mathsf{E}_\tau)^{\perp}}
\newcommand{\Len}{\Lambda_{\mathrm{len}}}
\newcommand{\qand}{\quad\text{and}\quad}
\newcommand{\qtext}[1]{\quad\text{#1}\quad}
\newcommand{\ang}[1]{\langle #1\rangle}
\newcommand{\ol}[1]{\overline{#1}}
\newcommand{\Trans}{\mathsf{Trans}}
\newcommand{\Funct}{\mathsf{Funct}}
\newcommand{\loc}{\mathrm{loc}}
\newcommand{\MECE}{\mathrm{MECE}}
\newcommand{\clip}{\mathrm{clip}}
\newcommand{\len}{\mathrm{len}}
\newcommand{\NN}{\mathbb{N}}
\newcommand{\EE}{\mathbb{E}}
\newcommand{\op}{\mathrm{op}}
\newcommand{\Pfun}{\mathbf{P}}
\newcommand{\Ufun}{\mathcal{U}}
\newcommand{\win}{\ensuremath{\mathrm{win}}}
\newcommand{\spec}{\ensuremath{\mathrm{spec}}}
\newcommand{\disc}{\ensuremath{\mathrm{disc}}}
\newcommand{\meas}{\ensuremath{\mathrm{meas}}}
\newcommand{\tot}{\ensuremath{\mathrm{tot}}}

% 注意:\int は「積分記号」
\newcommand{\inter}{\ensuremath{\mathrm{int}}}
\providecommand{\Cfun}[1]{\mathsf{C}_{#1}}
\providecommand{\Tfun}[1]{\mathbf{T}_{#1}}
\providecommand{\Ctau}{\Cfun{\tau}}
\providecommand{\Csigma}{\Cfun{\sigma}}
\providecommand{\Ttau}{\Tfun{\tau}}
\providecommand{\Tsigma}{\Tfun{\sigma}}
\providecommand{\intdist}{d_{\mathrm{int}}}
\DeclareMathOperator{\coker}{coker}

% --- safe stubs (preamble) ---
\usepackage{amsmath,amssymb,mathtools}
\providecommand{\LC}{\mathsf{LC}}
\providecommand{\Gap}{\mathrm{Gap}}
\providecommand{\Ttau}{\mathbf{T}_{\tau}}
\providecommand{\Ctau}{\mathsf{C}_{\tau}}
\providecommand{\Rfun}{\mathcal{R}}
\providecommand{\Mirror}{\mathsf{Mirror}}
\providecommand{\Funct}{\mathsf{Fun}}
\providecommand{\muc}{\mu_{\mathrm{Collapse}}}
\providecommand{\nuc}{\nu_{\mathrm{Collapse}}}
\providecommand{\Gal}{\mathrm{Gal}}
\providecommand{\Tr}{\mathrm{Tr}}
\providecommand{\Fun}{\mathrm{Fun}}
\providecommand{\alg}{\mathrm{alg}}
\providecommand{\disc}{\mathrm{disc}}
\providecommand{\meas}{\mathrm{meas}}
\providecommand{\Defect}{\mathsf{Defect}}

% --- Over/underfull mitigation ---
\usepackage{microtype}
\emergencystretch=2em

% --- Hyperlinks (load ONCE, near the end) ---
\usepackage[colorlinks=true, linkcolor=blue, citecolor=blue, urlcolor=blue]{hyperref}
\usepackage{xurl}
\urlstyle{tt}
\def\UrlBreaks{\do\/\do\_\do\.\do\-}

\usepackage{tabularx,booktabs,array}
\newcolumntype{L}[1]{>{\raggedright\arraybackslash}p{#1}}
\newcolumntype{Y}{>{\raggedright\arraybackslash}X}

% --- Clever references (after hyperref) ---
\usepackage[capitalise,nameinlink,noabbrev]{cleveref}

% --- Float / page control ---
\usepackage{placeins}

% ======================================================
% === Paste-artifact guard: robust \n handling (HEAVY) ===
% ======================================================
\providecommand{\n}{\unskip\space}
\makeatletter
\newcommand{\patchn}[1]{\expandafter\def\csname n#1\endcsname{\unskip\space #1}}
\forcsvlist{\patchn}{Hence,Therefore,Thus,Then,Proof,Remark,Theorem,Proposition,Lemma,Corollary,
Definition,Example,Declaration,Specification,Spec,Appendix,Section,Subsection,We,In,Let,Under,
Given,Assuming,Assumption,Fix,If,When,Where,While,For,From,By,On,Off,Also,Moreover,Conversely,
Equivalently,Finally,Next,First,Second,Third,Note,Notation,Convention,Claim,Case,Step,Algorithm,
Figure,Table,Thm,Prop,Lem,Cor,Def,Ex,Rem,Eq,Eqn,Ch,Chap,PF,BC,AK,NS,HT,PH}
\forcsvlist{\patchn}{hence,therefore,thus,then,proof,remark,theorem,proposition,lemma,corollary,
definition,example,declaration,specification,spec,appendix,section,subsection,we,in,let,under,
given,assuming,assumption,fix,if,when,where,while,for,from,by,on,off,also,moreover,conversely,
equivalently,finally,next,first,second,third,note,notation,convention,claim,case,step,algorithm,
figure,table,thm,prop,lem,cor,def,ex,rem,eq,eqn,ch,chap,pf,bc,ak,ns,ht,ph}
\patchn{Proof)}\patchn{Remark)}\patchn{Theorem)}\patchn{Lemma)}\patchn{Corollary)}
\newcommand{\patchnletter}[1]{\expandafter\def\csname n#1\endcsname{\unskip\space #1}}
\@tfor\@tempa:=ABCDEFGHIJKLMNOPQRSTUVWXYZabcdefghijklmnopqrstuvwxyz0123456789\do{%
  \expandafter\patchnletter\@tempa}
\makeatother

% ---------- Font fallback ----------
\makeatletter
\AtBeginDocument{%
  \@ifpackageloaded{stmaryrd}{\SetSymbolFont{stmry}{bold}{U}{stmry}{m}{n}}{}%
}
\makeatother


% ---------- 本文・見出し・しおりの Unicode → TeX 変換 ----------
\usepackage{newunicodechar}
\newunicodechar{Σ}{\texorpdfstring{\ensuremath{\Sigma}}{Sigma}}
\newunicodechar{Γ}{\texorpdfstring{\ensuremath{\Gamma}}{Gamma}}
\newunicodechar{Δ}{\texorpdfstring{\ensuremath{\Delta}}{Delta}}
\newunicodechar{Π}{\texorpdfstring{\ensuremath{\Pi}}{Pi}}
\newunicodechar{τ}{\texorpdfstring{\ensuremath{\tau}}{tau}}
\newunicodechar{ν}{\texorpdfstring{\ensuremath{\nu}}{nu}}
\newunicodechar{κ}{\texorpdfstring{\ensuremath{\kappa}}{kappa}}
\newunicodechar{θ}{\texorpdfstring{\ensuremath{\theta}}{theta}}
\newunicodechar{η}{\texorpdfstring{\ensuremath{\eta}}{eta}}
\newunicodechar{φ}{\texorpdfstring{\ensuremath{\varphi}}{phi}}
\newunicodechar{π}{\texorpdfstring{\ensuremath{\pi}}{pi}}
\newunicodechar{ε}{\texorpdfstring{\ensuremath{\varepsilon}}{epsilon}}
\newunicodechar{δ}{\texorpdfstring{\ensuremath{\delta}}{delta}}
\newunicodechar{ℝ}{\texorpdfstring{\ensuremath{\mathbb{R}}}{R}}
\newunicodechar{ℤ}{\texorpdfstring{\ensuremath{\mathbb{Z}}}{Z}}
\newunicodechar{ℕ}{\texorpdfstring{\ensuremath{\mathbb{N}}}{N}}
\newunicodechar{≤}{\texorpdfstring{\ensuremath{\le}}{<=}}
\newunicodechar{≥}{\texorpdfstring{\ensuremath{\ge}}{>=}}
\newunicodechar{≠}{\texorpdfstring{\ensuremath{\ne}}{\string!=}}
\newunicodechar{∈}{\texorpdfstring{\ensuremath{\in}}{in}}
\newunicodechar{⊆}{\texorpdfstring{\ensuremath{\subseteq}}{subseteq}}
\newunicodechar{∨}{\texorpdfstring{\ensuremath{\lor}}{or}}
\newunicodechar{∧}{\texorpdfstring{\ensuremath{\land}}{and}}
\newunicodechar{⪯}{\texorpdfstring{\ensuremath{\preceq}}{preceq}}
\newunicodechar{⪰}{\texorpdfstring{\ensuremath{\succeq}}{succeq}}
\newunicodechar{≡}{\texorpdfstring{\ensuremath{\equiv}}{equiv}}
\newunicodechar{≅}{\texorpdfstring{\ensuremath{\cong}}{cong}}
\newunicodechar{⊣}{\texorpdfstring{\ensuremath{\dashv}}{dashv}}
\newunicodechar{⊥}{\texorpdfstring{\ensuremath{\perp}}{perp}}
\newunicodechar{⊕}{\texorpdfstring{\ensuremath{\oplus}}{oplus}}
\newunicodechar{⊙}{\texorpdfstring{\ensuremath{\odot}}{odot}}
\newunicodechar{⋅}{\texorpdfstring{\ensuremath{\cdot}}{.}}
\newunicodechar{⟶}{\texorpdfstring{\ensuremath{\longrightarrow}}{->}}
\newunicodechar{⥤}{\texorpdfstring{\ensuremath{\to}}{->}}
\newunicodechar{⋙}{\texorpdfstring{\ensuremath{\ggg}}{>>>}}
\newunicodechar{∞}{\texorpdfstring{\ensuremath{\infty}}{infty}}
\newunicodechar{∀}{\texorpdfstring{\ensuremath{\forall}}{forall}}
\newunicodechar{∃}{\texorpdfstring{\ensuremath{\exists}}{exists}}
\newunicodechar{₁}{\texorpdfstring{$_1$}{_1}}
\newunicodechar{⁺}{\texorpdfstring{\textsuperscript{+}}{+}}
\newunicodechar{Č}{\v C}
\newunicodechar{č}{\v c}
\newunicodechar{Š}{\v S}
\newunicodechar{š}{\v s}
\newunicodechar{Ž}{\v Z}
\newunicodechar{ž}{\v z}
\newunicodechar{–}{--}
\newunicodechar{—}{---}
\newunicodechar{§}{\S}
\newunicodechar{’}{'}
\newunicodechar{“}{``}
\newunicodechar{”}{''}

\title{AK–Arithmetic Programs v1.0\\
\Large Version 17.0.1: Collapse-Based Calibration for Weil RH over Finite Fields and Fermat's Last Theorem}
\author{\textbf{Atsushi Kobayashi} \quad {\small (with ChatGPT Research Partner)}}
\date{December 2025}
\begin{document}

\maketitle



\begin{abstract}
This document presents the \textbf{Arithmetic Calibration Program} for the AK High-Dimensional Projection Structural Theory (AK--HDPST v17.0).
Before deploying the High-Dimensional Projection Search (HDPS) engine on open problems such as the Navier--Stokes equations, we must validate its diagnostic machinery against established mathematical truths.
To this end, we construct formal realizations of two ``Closed World'' arithmetic landscapes: the moduli space of varieties over finite fields ($M_{\mathrm{Weil}}$) and the deformation space of semistable elliptic curves over $\mathbb{Q}$ ($M_{\mathrm{FLT}}$).

Incorporating the classical theorems of Deligne and Wiles (together with their extensions) as \textbf{Core-Input Axioms}, we demonstrate that the AK \emph{Unified Collapse Contract (UCC)} faithfully reproduces the topological structure of these validity spaces.
Specifically, the system identifies the Finite-Field Riemann Hypothesis as a global \textbf{``Plain of Truth''} (vanishing defect potential) and Fermat's Last Theorem as a \textbf{``Global Obstruction''} (a structurally empty Frey Locus detected via Type~IV failures).
These results serve as an \emph{epistemic certification} of the $\delta$-ledger and the Type~IV detector, thereby formally authorizing the transition of the AK--HDPST platform from arithmetic calibration to ``Open World'' exploration of the Millennium Prize Problems.
\end{abstract}



% ------------------------------------------------------------
% Part I : Arithmetic AK Core
% ------------------------------------------------------------

\section{Chapter 1: Arithmetic AK Perspective and Problem Statements}
\label{ch:arith-perspective}

% Safe macro: usable in text and math
\newcommand{\AKverdict}{\textsf{AKVerdict}}

This chapter fixes the basic perspective, notation, and problem statements
for the arithmetic part of AK--HDPST v17.0.
We explain how the existing collapse machinery is used as a calibration
device for two deep and already-resolved arithmetic problems:
the finite-field Riemann hypothesis (the Weil conjectures in the form
proved by Deligne) and Fermat's Last Theorem (FLT).
Throughout, we are careful to separate three levels:

\begin{itemize}[leftmargin=2.5em]
  \item \emph{Classical truth} in arithmetic geometry and number theory
        (Weil/Deligne, Wiles and successors).
  \item The \emph{AK collapse core}, which is the same constructible
        persistence and derived-category infrastructure as in the global
        v17.0 document.
  \item The \emph{\AKverdict{}}, i.e.\ the output of an AK-based
        decision procedure once we have embedded an arithmetic problem
        into the collapse framework.
\end{itemize}

The purpose of this part is not to reprove classical theorems, but to
check that, once the arithmetic data are embedded into AK--HDPST,
the resulting \AKverdict{} is compatible with those theorems.
This gives a ``true-side calibration'' for later speculative applications
(such as the Navier--Stokes case study).

\subsection{AK--HDPST v17.0 in the background}

We briefly recall the standing scope of AK--HDPST v17.0 used in this part.
The full details are given in Part~I of the main document; here we record
only what is needed to make arithmetic sense.

\begin{declaration}[Standing scope]\label{dec:scope}
Unless explicitly stated otherwise, the following assumptions are in force.
\begin{enumerate}[label=(S\arabic*), leftmargin=2.5em]
  \item \textbf{Coefficients.}
  We work over a fixed field \(k\) of characteristic \(0\) (or of
  sufficiently large positive characteristic when explicitly allowed),
  and all persistence modules take values in \(\Vect_k\).

  \item \textbf{Persistence layer.}
  The collapse core operates in the one-parameter constructible
  persistence category \(\Perscons\).
  All equalities and stability estimates at the persistence level are
  formulated inside this category.

  \item \textbf{Realization layer.}
  When required, persistence data are realized in the bounded derived
  category \(\Dbk\) by a fixed \(t\)-exact realization functor of
  amplitude \(\leq 1\).
  We never leave \(\Dbk\) at the level of provable statements.

  \item \textbf{Collapse machinery.}
  We use the same exact bar-deletion reflector \(\Ttau\) on persistence
  and its filtered lift \(\Ctau\) (defined up to filtered quasi-isomorphism),
  the Unified Collapse Contract (UCC), the tower diagnostics
  \(\muc\) and \(\nuc\), and the failure landscape (Types~I--IV) as in
  the v17.0 core.

  \item \textbf{Definable parameter spaces.}
  Arithmetic parameter spaces such as \(M_{\mathrm{Weil}}\) and
  \(M_{\mathrm{FLT}}\) are treated as definable sets in a suitable
  Denef--Pas language, as in Appendix~Q.
  All windows and policies used in UCC are required to respect these
  definability guard-rails.
\end{enumerate}
\end{declaration}

In particular, all \emph{Core} statements in this arithmetic part remain
within the same categorical and homological framework; we only change the
\emph{input data} and the interpretation of the collapse diagnostics.
Any claims about arithmetic objects that go beyond this scope are marked
as \SpecBadge{} and treated as part of the Spec layer.

\subsection{Two classical arithmetic problems}

We now state the arithmetic problems that we will use to calibrate AK.
We only need the rough form of the statements; detailed background is
deferred to Appendices~A and~B.

\subsubsection*{Finite-field Riemann hypothesis (Weil/Deligne)}

Let \(X\) be a smooth projective variety over a finite field
\(\mathbb{F}_q\).
Its zeta function is
\[
  Z(X,t)
  = \exp\Bigl(
      \sum_{n\ge 1}
      \frac{\#X(\mathbb{F}_{q^n})}{n} t^n
    \Bigr).
\]
The Weil conjectures, proved in full generality by Deligne, assert that
\(Z(X,t)\) is a rational function and that its reciprocal zeros and poles
have absolute values \(q^{-i/2}\) in degree \(i\).
For the purposes of this part we isolate the finite-field Riemann
hypothesis component.

\begin{declaration}[Finite-field RH, classical version]\label{dec:ff-rh}
Let \(X/\mathbb{F}_q\) be smooth and projective.
For each integer \(i\ge 0\), the eigenvalues \(\alpha\) of the geometric
Frobenius on the \(i\)th \(\ell\)-adic cohomology group
\(H^i_{\textnormal{\textup{\'et}}}(X_{\overline{\mathbb{F}}_q},\mathbb{Q}_\ell)\)
satisfy \(\lvert \alpha\rvert = q^{i/2}\) under any complex embedding.
We refer to this property as \emph{finite-field RH for
\((X,i)\)}.
\end{declaration}

We will collectively denote by \(M_{\mathrm{Weil}}\) the parameter space
consisting of isomorphism classes of such pairs \((X,i)\), or suitable
variants (e.g.\ restricted to curves) when convenient.

\subsubsection*{Fermat's Last Theorem}

Fermat's Last Theorem (FLT), now a theorem due to Wiles and subsequent
work, states that for any integer exponent \(n \ge 3\) there is no
non-trivial solution to the Diophantine equation
\[
  x^n + y^n = z^n
\]
in non-zero integers \(x,y,z\).
From the AK perspective, the key point is that the Wiles proof proceeds
through the modularity of certain elliptic curves.

\begin{declaration}[Fermat's Last Theorem, classical version]\label{dec:flt}
For every integer \(n \ge 3\), there are no non-zero integers
\(x,y,z\) satisfying \(x^n + y^n = z^n\).
Equivalently, the Diophantine equation \(x^n+y^n=z^n\) has no non-trivial
integral solutions.
\end{declaration}

We will denote by \(M_{\mathrm{FLT}}\) a parameter space encoding the
relevant arithmetic data for FLT (exponent, associated elliptic curves,
levels, conductors, and so on); a precise definition is given in
Chapter~\ref{ch:param}.

\subsection{Embedding arithmetic data into AK}

The arithmetic programs in this part are based on the following principle:
to each parameter \(\theta\) in an arithmetic parameter space
\(M_{\mathrm{arith}}\) we associate a persistence object
\(F_\theta \in \Perscons\),
constructed from cohomological or Galois data, and then apply the usual
collapse machinery to \(F_\theta\).

\begin{definition}[Arithmetic AK configuration]\label{def:arith-config}
An \emph{arithmetic AK configuration} consists of:
\begin{enumerate}[label=(C\arabic*), leftmargin=2.5em]
  \item A parameter \(\theta\) in a fixed arithmetic parameter space
        \(M_{\mathrm{arith}}\) (e.g.\ \(\theta=(X,i)\) or \(\theta\) in
        \(M_{\mathrm{FLT}}\)).
  \item A functorial assignment of a persistence object
        \(F_\theta \in \Perscons\), constructed from arithmetic data
        (cohomology, Galois representations, etc.).
  \item A choice of collapse threshold \(\tau > 0\),
        a quantale-valued defect functional
        \(\Defect_\tau(F_\theta)\in V\),
        and a UCC policy (windows, gates, and after-collapse comparison).
        In arithmetic applications, \(\Defect_\tau\) typically accounts for
        spectral-radius deviations, numerical approximation errors for
        Frobenius eigenvalues, and truncation errors along arithmetic towers,
        with these contributions recorded in a \(\delta\)-ledger as in the
        v17.0 core.
\end{enumerate}
We write \((\theta,F_\theta,\tau,V,\Defect_\tau)\) for such a
configuration when we want to emphasize all components.
\end{definition}

% ------------------------------------------------------------
% v17.0.1 compatibility: naming reservation (B-Gate^+, cell vs window)
% ------------------------------------------------------------
\begin{declaration}[Naming reservation for gates and windows (v17.0.1 compatibility)]
\label{dec:arith-naming-reservation}
Throughout this document we reserve the term \emph{B-Gate$^{+}$} for the
\emph{after-collapse gate of AK--HDPST v17.0.1} (cf.\ v17.0.1, Definition~\texttt{def:bgate-plus}).
Concretely, whenever we write
\(\text{B-Gate}^{+}(\theta;W,\tau)\),
it is understood as the v17 Core predicate applied to the derived persistence
object \(F_\theta:=\mathfrak{P}(\theta)\) on a \emph{right-open filtration window}
\(W=[u,u')\subset\mathbb{R}\) at deletion threshold \(\tau>0\),
with all decisions taken \emph{after} applying \(\mathbf{T}_\tau\) (equivalently via \(C_\tau\)).

Problem-specific numerical checks used in the arithmetic calibrations
(e.g.\ Frobenius spectral defect potentials, modularity potentials)
are treated as \emph{pre-gates} (input/measurement sanity checks and ledger budget checks);
they do \emph{not} redefine the v17 Core predicate B-Gate$^{+}$.

To avoid ambiguity we use:
(i) \(U\subset M_{\mathrm{arith}}\) for \emph{parameter cells} (regions in moduli/parameter space), and
(ii) \(W\subset\mathbb{R}\) (right-open) for \emph{filtration windows}.
\end{declaration}


The \emph{AK program} for a given arithmetic problem is then a
prescription that, for each admissible parameter \(\theta\),
constructs an arithmetic AK configuration and applies the collapse
diagnostics to produce a decision.

\subsection{The notion of an AK verdict}

To state the main calibration statements, we need a precise notion of
what it means for AK to ``judge'' an arithmetic instance.

\begin{definition}[AK verdict]\label{def:ak-verdict}
Fix an arithmetic parameter space \(M_{\mathrm{arith}}\) and an AK
program as in Definition~\ref{def:arith-config}.
An \emph{AK verdict} for this program is a function
\[
  \AKverdict \colon M_{\mathrm{arith}}
  \longrightarrow \{\mathsf{Valid},\ \mathsf{Obstructed},\ \mathsf{Inconclusive}\}
\]
with the following properties.
\begin{enumerate}[label=(V\arabic*), leftmargin=2.5em]
  \item (\emph{Valid.})
  We set \(\AKverdict(\theta)=\mathsf{Valid}\) if the collapse diagnostics
  applied to \(F_\theta\) satisfy the UCC constraints with no Type~IV
  failure and with defect potential \(\Defect_\tau(F_\theta)\) below a
  fixed acceptance threshold.
  Intuitively, \(\theta\) lies in a \emph{Plain} region of the validity
  map.

  \item (\emph{Obstructed.})
  We set \(\AKverdict(\theta)=\mathsf{Obstructed}\) if there
  is a certified Type~IV failure or a collapse obstruction that cannot be
  attributed to numerical noise as accounted for in the \(\delta\)-ledger.
  Intuitively, \(\theta\) lies in a \emph{Peak} region.

  \item (\emph{Inconclusive.})
  In all remaining cases (e.g.\ when the diagnostics do not terminate, or
  the defect potential sits in a gray zone that cannot be resolved within
  the chosen policy), we set
  \(\AKverdict(\theta)=\mathsf{Inconclusive}\).
  Intuitively, \(\theta\) lies in a \emph{Noise} region or in an
  unresolved ridge.
\end{enumerate}
\end{definition}

In this part we will design AK programs for the finite-field RH and FLT
such that, \emph{assuming} the classical results (Weil/Deligne and Wiles),
the AK verdict matches the known truth on the relevant parameter spaces.
We stress that this is a \emph{calibration exercise}: we do not attempt to
derive Weil/Deligne or Wiles from the AK axioms alone.

\subsection{Calibration goals and structure of the arithmetic part}

We can now formulate the two main calibration goals of this part.
They are deliberately stated as \emph{AK--Perspective} declarations,
not as new theorems in arithmetic.

\begin{declaration}[AK--Perspective: finite-field RH calibration]\label{dec:ak-weil-goal}
Let \(M_{\mathrm{Weil}}\) be a parameter space for smooth projective
varieties (or curves) over finite fields, and let
\(\AKverdict_{\mathrm{Weil}}\) be the AK verdict associated
to the Weil program defined in Part~II.
Assume the finite-field Riemann hypothesis in the classical sense
(Declaration~\ref{dec:ff-rh}) for all parameters in \(M_{\mathrm{Weil}}\).
Then the AK program can be arranged so that
\[
  \AKverdict_{\mathrm{Weil}}(\theta)
  = \mathsf{Valid}
  \quad\text{for all }\theta\in M_{\mathrm{Weil}}.
\]
In other words, under classical finite-field RH, the AK validity map on
\(M_{\mathrm{Weil}}\) has no obstructed peaks: all points lie in the Plain
region.
\end{declaration}

\begin{declaration}[AK--Perspective: FLT calibration]\label{dec:ak-flt-goal}
Let \(M_{\mathrm{FLT}}\) be a parameter space encoding the relevant data
for Fermat-type equations and their associated elliptic curves, and let
\(\AKverdict_{\mathrm{FLT}}\) be the AK verdict associated
to the FLT program defined in Part~III.
Assume the classical Fermat's Last Theorem (Declaration~\ref{dec:flt}) and
the needed modularity results as external input.
Then the AK program can be arranged so that
\[
  \AKverdict_{\mathrm{FLT}}(\theta)
  \in \{\mathsf{Obstructed},\ \mathsf{Inconclusive}\}
  \quad\text{for all }\theta\in M_{\mathrm{FLT}},
\]
and, more precisely, every hypothetical counterexample parameter is classified as
\(\mathsf{Obstructed}\).
In particular, an \(\mathsf{Obstructed}\) outcome reflects a structural
incompatibility with the modularity and Galois constraints encoded in the
arithmetic AK configuration (and ultimately inherited from the classical
modularity theorems), rather than a numerical failure or truncation artefact.
Equivalently, the AK validity map on \(M_{\mathrm{FLT}}\) has no Plain
cell corresponding to a genuine FLT counterexample.
\end{declaration}

These declarations are the targets of Parts~II and~III.
They express the idea that, once the arithmetic objects are embedded into
\(\Perscons\) and processed by the collapse machinery, the AK picture of
the arithmetic world is consistent with the known theorems: the finite-field
RH looks uniformly regular (no peaks), while FLT looks uniformly obstructed
(no Plain solutions).

\begin{remark}[Role of AK in arithmetic calibration]
Declarations~\ref{dec:ak-weil-goal} and~\ref{dec:ak-flt-goal} should be
read carefully.
They do \emph{not} claim new arithmetic results; rather, they specify how
AK--HDPST v17.0 behaves when fed arithmetic data whose classical behavior
is already known.
The point is that the collapse infrastructure is sensitive enough to
distinguish between a ``globally regular'' situation (finite-field RH) and
a ``no-solution'' situation (FLT) on purely structural grounds.
This calibration justifies, at least conceptually, the later use of the
same machinery in settings where the classical truth is not yet known.
\end{remark}

\subsection{Outline of the arithmetic AK programs}

For the reader's convenience we summarize the content of the subsequent
chapters.

\begin{itemize}[leftmargin=2.5em]
  \item Chapter~2 defines the parameter spaces \(M_{\mathrm{Weil}}\) and
        \(M_{\mathrm{FLT}}\) and the associated one-parameter filtrations,
        in a way compatible with the Denef--Pas and UCC guard-rails.

  \item Chapter~3 describes the arithmetic version of the Unified Collapse
        Contract, including the choice of quantale \(V\), the
        \(\delta\)-ledger in the arithmetic context, and the interpretation
        of Type~I--IV failures in towers arising from field extensions or
        modular levels.

  \item Chapter~4 explains how the tower diagnostics \(\muc,\nuc\) behave
        in arithmetic towers, and how they will be used in the Weil and
        FLT programs.

  \item Part~II (Chapters~5--8) develops the AK program for finite-field
        RH: realization of Frobenius spectra, the corresponding collapse
        contract, the validity map on \(M_{\mathrm{Weil}}\), and the
        behavior of the HDPS engine on this space.

  \item Part~III (Chapters~9--11) develops the AK program for FLT:
        realization via elliptic curves and Galois representations,
        diagnostics for hypothetical counterexamples, and the resulting
        validity map on \(M_{\mathrm{FLT}}\).

  \item Appendices~A--E collect the classical arithmetic background and
        the technical details of the arithmetic realizations and the
        relation between AK verdicts and mathematical truth.
\end{itemize}

This completes the conceptual setup needed for the arithmetic calibration
of AK--HDPST v17.0.



% ------------------------------------------------------------
% Part I : Arithmetic AK Core
% ------------------------------------------------------------

\section{Chapter 2: Arithmetic Parameter Spaces and Filtrations}
\label{ch:arith-param}

This chapter defines the geometric and arithmetic domains over which the
AK--HDPST engine operates.
To apply the collapse machinery of the v17.0 core, we must rigorously
specify the \emph{parameter spaces} \(\mathcal{M}\) (where the Hunter
agents search) and the \emph{filtrations} (which define the time axis
\(t\) for persistence).

We adopt the Denef--Pas formalism to ensure that these spaces are
compatible with the definability guard-rails established in
Appendix~Q of the v17.0 core.
This guarantees that event counts are finite on definable windows and
that the overlap and gate policies used in the UCC can be implemented as
decidable procedures.

% ------------------------------------------------------------
\subsection{2.1. The Weil Parameter Space \texorpdfstring{\(M_{\mathrm{Weil}}\)}{M\_Weil}}
\label{sec:M-weil}

The target of the finite-field Riemann hypothesis calibration is a
moduli-type parameter space of smooth projective varieties over finite
fields.

\begin{definition}[Weil parameter space \(M_{\mathrm{Weil}}\)]
The parameter space \(M_{\mathrm{Weil}}\) is the disjoint union over
prime powers \(q = p^r\) and dimensions \(d \ge 1\) of isomorphism
classes of pairs
\[
  M_{\mathrm{Weil}}
  \;:=\;
  \bigsqcup_{q, d}
  \Bigl\{
    (X,\mathbb{F}_q)
    \;\Big|\;
    X \text{ is a smooth projective variety of dimension } d
    \text{ over } \mathbb{F}_q
  \Bigr\}.
\]
For operational purposes (Hunter traversal), we stratify
\(M_{\mathrm{Weil}}\) by invariants such as genus \(g\) (for curves) or
Betti numbers (for higher dimensions), and we view discrete parameters
(like coefficients of defining equations in an affine patch) as elements
in the residue field sort of an underlying Denef--Pas structure.
\end{definition}

\begin{specification}[Definability of \(M_{\mathrm{Weil}}\)]
We fix a Denef--Pas structure
\(\mathfrak{S} = (\mathrm{VF},\mathrm{RF},\mathrm{VG})\)
where \(\mathrm{RF}\) represents a finite field (or its algebraic
closure) and \(\mathrm{VG}\) encodes discrete valuations.
For any bounded subset of \(M_{\mathrm{Weil}}\) (for example, curves of
fixed genus \(g\) over \(\mathbb{F}_{p^k}\) with \(k \le K\)), we choose
coordinates that identify this subset with a definable set in
\(\mathrm{RF}^N \times \mathrm{VG}^M\).
In particular, any constructible function on such a bounded subset (such
as the defect potential \(\Defect_\tau\) or a spectral indicator) has
only finitely many level sets within any UCC window.
\end{specification}

The detailed arithmetic and geometric background for
\(M_{\mathrm{Weil}}\) (including possible restrictions to curves or to
fixed dimension) is collected in Appendix~A.

% ------------------------------------------------------------
\subsection{2.2. The FLT Parameter Space \texorpdfstring{\(M_{\mathrm{FLT}}\)}{M\_FLT}}
\label{sec:M-flt}

For Fermat's Last Theorem, the parameter space must capture both the
underlying Diophantine data and the associated geometric objects
(Frey curves and their Galois representations).
It is convenient to separate a \SpecBadge{}--level configuration space
for \emph{hypothetical} counterexamples from a more canonical geometric
moduli space.

\begin{specification}[Hypothetical FLT configuration]\label{spec:flt-config}
A \emph{hypothetical FLT configuration} is a tuple
\(\theta = (p,A,B,C,E)\) with the following components:
\begin{itemize}[leftmargin=2.5em]
  \item \(p \ge 3\) is a prime exponent.
  \item \((A,B,C) \in \mathbb{Z}^3\setminus\{(0,0,0)\}\) is a triple of
        integers satisfying \(A^p + B^p + C^p = 0\) and \(\gcd(A,B,C)=1\).
  \item \(E\) is the associated Frey elliptic curve over \(\mathbb{Q}\),
        e.g.
        \[
          E_{A,B,C} \colon
          y^2 = x(x-A^p)(x+B^p).
        \]
\end{itemize}
From the classical point of view (assuming Fermat's Last Theorem),
no such integer triple exists for \(p\ge 3\), so this configuration
space is empty.
In the AK perspective, however, it is useful as a \emph{formal} target
for the Hunter in ``counterexample-search mode'': any hypothetical
counterexample would determine such a configuration and hence a point in
the geometric parameter space described below.
\end{specification}

\begin{definition}[FLT parameter space \(M_{\mathrm{FLT}}\)]
The parameter space \(M_{\mathrm{FLT}}\) is defined as a moduli-type
space of semistable elliptic curves \(E/\mathbb{Q}\) equipped with a
mod-\(p\) Galois representation \(\bar{\rho}_{E,p}\) satisfying the
local and global conditions that appear in the Wiles--Ribet strategy
(unramified outside a prescribed finite set of primes and with ``too
little'' ramification at \(p\)).
Formally, we view \(M_{\mathrm{FLT}}\) as a disjoint union
\[
  M_{\mathrm{FLT}}
  \;\simeq\;
  \bigsqcup_{p \ge 3}
  \bigl\{
    (E,\bar{\rho}_{E,p}) \text{ semistable, modular of weight }2
    \text{ with the prescribed local behavior}
  \bigr\}/\!\sim,
\]
and we require that each hypothetical FLT configuration
\(\theta = (p,A,B,C,E_{A,B,C})\) as in
Specification~\ref{spec:flt-config}, when it exists, determines a point
of \(M_{\mathrm{FLT}}\) via the associated Frey curve.
\end{definition}

\begin{specification}[Definability of \(M_{\mathrm{FLT}}\)]
Within a fixed bounded range of conductors, levels, and primes \(p\),
the space \(M_{\mathrm{FLT}}\) can be modeled as a definable subset of a
Denef--Pas structure by coding the coefficients of Weierstrass models
and local Galois data into residue-field and value-group sorts.
As in the Weil case, this ensures that the UCC windows and gate
conditions are compatible with definability and that the Hunter's
search over \(M_{\mathrm{FLT}}\) can be organized into finitely many
cells on any bounded region.
\end{specification}

\begin{remark}[Search mode and hypothetical obstructions]
In the AK framework, \(M_{\mathrm{FLT}}\) is treated as a space of
\emph{hypothetical obstructions}.
The Hunter's goal, in counterexample search mode, is to locate a point
\(\theta\in M_{\mathrm{FLT}}\) that passes the basic arithmetic
well-posedness checks (B--Gate\(^{+}\)) but triggers Type~IV diagnostics
when modularity and Galois constraints are imposed.
Assuming the classical modularity theorems and FLT, no such point exists,
and the AK verdict for every hypothetical counterexample parameter is
\(\mathsf{Obstructed}\) (as formulated in
Declaration~\ref{dec:ak-flt-goal} of Chapter~\ref{ch:arith-perspective}).
\end{remark}

The detailed arithmetic background for \(M_{\mathrm{FLT}}\), including
its link to modular forms and Iwasawa theory, is summarized in
Appendix~B.

% ------------------------------------------------------------
\subsection{2.3. Filtration strategies (defining ``time'')}
\label{sec:arith-filtrations}

To embed these static arithmetic objects into the dynamic world of
persistence \(\Perscons\), we must define a filtration parameter
\(t \in \mathbb{R}\).
We describe \SpecBadge{}--level filtration schemes for both
\(M_{\mathrm{Weil}}\) and \(M_{\mathrm{FLT}}\), chosen to be compatible
with the UCC and the deletion-type policies emphasized in the v17.0 core.

\subsubsection{Weil: Frobenius-orbit filtration}

For \((X,\mathbb{F}_q)\in M_{\mathrm{Weil}}\), the underlying topology
of \(X\) is static, but the arithmetic action of Frobenius
\(\mathrm{Fr}_q\) induces a natural discrete dynamical system.
We use this to define a filtration that encodes the Frobenius orbit
structure in a persistence-friendly way.

\begin{definition}[Frobenius filtration schema]\label{def:frob-filtration}
Let \((X,\mathbb{F}_q)\in M_{\mathrm{Weil}}\), and fix an
\(\ell\)-adic sheaf \(\mathcal{E}\) on \(X\) (for example, the constant
sheaf \(\mathbb{Q}_\ell\)).
We informally write \(H^i(X)\) for the \(i\)-th \(\ell\)-adic cohomology
group \(H^i_{\mathrm{\acute{e}t}}(X_{\overline{\mathbb{F}}_q},\mathbb{Q}_\ell)\),
equipped with the Frobenius endomorphism
\(\phi = \mathrm{Fr}_q\).
The Frobenius filtration schema associates to \((X,\mathbb{F}_q)\) a
filtered object \(F_{\mathrm{Weil}}\) whose persistence in degree \(i\)
encodes the behavior of the iterates \(\phi^k\) for \(k\ge 1\).

Concretely, one convenient model is given by the mapping-torus
construction:
\begin{align*}
  V_t
  \;:=\;
  \bigoplus_{0 \le k \le \lfloor t \rfloor} H^i(X)\cdot T^k
  \Big/
  \bigl\langle v\cdot T - \phi(v) \mid v \in H^i(X) \bigr\rangle,
  \qquad t \ge 0,
\end{align*}
with structure maps \(V_s \to V_t\) for \(s\le t\) given by inclusion of
summands.
The resulting persistence module
\(\mathbf{P}_i(F_{\mathrm{Weil}})\) lies in \(\Perscons\) and is
constructed functorially in \((X,\mathbb{F}_q)\).
\end{definition}

\begin{specification}[Spectral interpretation]
In this schema, the eigenvalues \(\alpha\) of \(\phi\) on \(H^i(X)\) are
encoded in the bar decomposition of \(\mathbf{P}_i(F_{\mathrm{Weil}})\).
Roughly speaking, eigenvalues with \(|\alpha| = q^{w/2}\) correspond to
controlled ``oscillatory'' behavior whose persistence signature is
captured by the spectral indicators introduced in the v17.0 core.
This is the mechanism by which the finite-field Riemann hypothesis is
translated into a collapse-friendly regularity statement for
\(\mathbf{P}_i(F_{\mathrm{Weil}})\).
\end{specification}

\subsubsection{FLT: Iwasawa-tower filtration}

For \(E \in M_{\mathrm{FLT}}\), a natural time axis is given by the depth
in a cyclotomic \(\mathbb{Z}_p\)-extension, aligning with the Iwasawa
interface in Appendix~R of the v17.0 core.

\begin{definition}[Iwasawa filtration schema]\label{def:iwasawa-filtration}
Fix a prime \(p\ge 3\) and let \(\{K_n\}_{n\ge 0}\) denote the layers of
the cyclotomic \(\mathbb{Z}_p\)-extension of \(\mathbb{Q}\).
For an elliptic curve \(E\in M_{\mathrm{FLT}}\), we define a filtered
complex \(F_{\mathrm{FLT}}\) by setting, for \(t\ge 0\),
\[
  F_{\mathrm{FLT}}^t
  \;:=\;
  \text{Selmer complex of \(E\) over } K_{\lfloor t \rfloor},
\]
with transition maps induced by corestriction (norm) as we pass from
\(K_{n'}\) down to \(K_n\) for \(n' > n\).
The resulting persistence modules
\(\mathbf{P}_i(F_{\mathrm{FLT}})\) provide the input to the collapse
machinery in the FLT program.
\end{definition}

\begin{specification}[Deletion-type behavior in the Iwasawa tower]
From the AK point of view, we treat the transition maps
\(F_{\mathrm{FLT}}^{t'} \to F_{\mathrm{FLT}}^t\) for \(t' > t\) as
\emph{deletion-type} operations in the sense of the v17.0 core:
they can annihilate Selmer or class-group classes (shortening bars) but
are not allowed to create new stable classes ex nihilo inside a fixed
UCC window.
This monotonicity assumption is encoded at the Spec layer and is crucial
for bringing the Convergence Manager (Appendix~J of the core) to bear on
arithmetic towers.
\end{specification}

% ------------------------------------------------------------
\subsection{2.4. The arithmetic realization functor \texorpdfstring{\(\mathfrak{P}\)}{P}}
\label{sec:embedding-functor}

We consolidate the above constructions into a single realization functor
from arithmetic parameter spaces to the filtered-chain world used by the
collapse core.

\begin{specification}[Arithmetic realization functor \(\mathfrak{P}\)]
\label{spec:arith-realization}
There exists a realization functor
\[
  \mathfrak{P} \colon
  M_{\mathrm{arith}}
  \longrightarrow
  \FiltCh{k},
\]
defined on a disjoint union
\(M_{\mathrm{arith}} := M_{\mathrm{Weil}} \sqcup M_{\mathrm{FLT}}\),
with the following properties, compatible with the Unified Collapse
Contract (UCC):

\begin{enumerate}[label=(R\arabic*), leftmargin=2.5em]
  \item \textbf{Constructibility.}
  For any \(\theta \in M_{\mathrm{arith}}\), the associated persistence
  modules \(\mathbf{P}_i(\mathfrak{P}(\theta))\) lie in \(\Perscons\).
  In particular, they have only finitely many critical values in any
    bounded right-open filtration window \(W=[u,u')\subset\mathbb{R}\).

  \item \textbf{Lipschitz stability.}
  Small deformations of \(\theta\) in \(M_{\mathrm{arith}}\) (for example,
  \(p\)-adic perturbations of coefficients of defining equations or
  Weierstrass models, within a fixed definable cell) induce
  \(d_{\mathrm{int}}\)-small deformations of the corresponding
  persistence modules.
  This ensures that the UCC low-pass and window policies apply without
  modification.

  \item \textbf{\(\tau\)-exactness.}
  After realization into \(\Dbk\), the complexes
  \(\mathfrak{P}(\theta)\) satisfy the amplitude and \(t\)-exactness
  assumptions required for the PH\(_1\Rightarrow\Ext^1\) bridge in the
  v17.0 core on each fixed UCC window.
\end{enumerate}
\end{specification}

\begin{remark}[Operational usage in \texttt{run.yaml}]
In a practical run, the arithmetic AK mode is specified in the
\texttt{run.yaml} manifest, for example:
\begin{lstlisting}[language={},basicstyle=\ttfamily\small]
realization:
  mode: "arith"
  type: "Weil_Frobenius"   # or "FLT_Iwasawa"
  parameters:
    q: ...
    dim: ...
    p: ...
\end{lstlisting}
The AK core then invokes \(\mathfrak{P}\) to construct the filtered
complex \(F\), applies the collapse \(\Ttau\), updates the
\(\delta\)-ledger, and finally produces an \AKverdict{} as in
Definition~\ref{def:ak-verdict}.
\end{remark}

This completes the description of the arithmetic parameter spaces and
filtration schemes.
In the next chapters we specialize these constructions to the finite-field
Riemann hypothesis calibration (Part~II) and to the FLT calibration
(Part~III), tracking carefully which parts remain in the Core and which
reside in the Spec layer.



\section{Chapter 3: Collapse Contract in the Arithmetic Setting}
\label{ch:arith-collapse}

This chapter refines the Unified Collapse Contract (UCC) of the v17.0
core to the arithmetic domain.
Starting from the realization functor
\(\mathfrak{P} : M_{\mathrm{arith}} \to \FiltCh{k}\) of
Chapter~\ref{ch:arith-param}, we specify how the generic collapse
machinery \((\Ttau,\Defect_\tau,V)\) is instantiated to handle
arithmetic noise.

The guiding philosophy is that \emph{arithmetic noise}---such as finite
submodules in Iwasawa theory, or non-dominant spectral terms in zeta
functions---should correspond to \emph{short bars} in the associated
persistence modules.
The reflector \(\Ttau\) then acts as a rigorous low-pass filter that
separates this noise from the essential arithmetic content (ranks,
characteristic ideals, weights, spectral radii).

% ------------------------------------------------------------
\subsection{3.1. Arithmetic UCC policy}
\label{sec:arith-ucc}

We do \emph{not} change the definition of \(\Ttau\) or the UCC at the
core level.
Instead, we specify how \(\Ttau\) is \emph{interpreted} in the two
arithmetic modes.

\begin{specification}[Arithmetic interpretation of the collapse \(\Ttau\)]
\label{spec:arith-Ttau}
Let \(\theta \in M_{\mathrm{arith}}\) and
\(F = \mathfrak{P}(\theta) \in \FiltCh{k}\) be its realization.
The global collapse operator \(\Ttau\) of the v17.0 core acts on the
persistence modules \(\mathbf{P}_i(F)\in\Perscons\).
In arithmetic mode we interpret its effect as follows:
\begin{itemize}[leftmargin=2.5em]
  \item \textbf{FLT / Iwasawa mode.}
  Bars of length \(\le \tau\) in \(\mathbf{P}_i(F_{\mathrm{FLT}})\) are
  treated as ``short-lived'' classes that fail to survive \(\tau\)
  steps in the Iwasawa tower.
  Heuristically, these are expected to correspond to finite
  \(p\)-primary torsion submodules or transient Selmer classes.
  By choosing \(\tau\) above the expected noise scale, the contribution
  of such classes to collapse diagnostics is removed.

  \item \textbf{Weil / Frobenius mode.}
  For \((X,\mathbb{F}_q)\in M_{\mathrm{Weil}}\) realized via the
  Frobenius mapping-torus schema
  (Definition~\ref{def:frob-filtration}), bars whose behavior is
  controlled by unstable or transient eigen-components (e.g.\ small
  Jordan blocks, rapidly decaying modes) are shortened below \(\tau\)
  and hence deleted by \(\Ttau\).
  Persistent contributions from eigenvalues on the expected RH circle
  \(|\alpha| = q^{w/2}\) are designed to survive the truncation.
\end{itemize}
In both modes, the meaning of ``short'' is governed by the same UCC
window policy as in Part~I; only the arithmetic interpretation of
short-lived bars changes.
\end{specification}

\begin{specification}[Arithmetic after-collapse policy]
\label{spec:arith-after-collapse}
In the arithmetic AK programs, all higher-level arithmetic invariants
used for decision-making are computed only \emph{after} collapse.
More precisely, More precisely, for a fixed degree \(i\), a right-open filtration window \(W=[u,u')\subset\mathbb{R}\),
and a threshold \(\tau>0\), any
invariant of interest is evaluated on \(\Ttau \mathbf{P}_i(F)\), not on
\(\mathbf{P}_i(F)\) itself.
Symbolically, for a measurement functor \(\mathrm{Meas}\) we write
\begin{equation}
  \mathrm{ArithInv}_i(\theta;\tau)
  \;:=\;
  \mathrm{Meas}\!\left(
    \Ttau \bigl( \mathbf{P}_i(\mathfrak{P}(\theta)) \bigr)
  \right).
\end{equation}
Examples of such invariants (depending on the program) include:
ranks of Selmer-type groups, Iwasawa \(\lambda,\mu\)-indicators,
Newton polygon slopes, or quantale-valued defect norms
\(\Defect_\tau(\theta)\in V\).
This enforces the ``after-collapse only'' policy of Chapter~1 in the
arithmetic setting, and prevents finite-layer arithmetic noise from
triggering spurious Type~IV failures.
\end{specification}

% ------------------------------------------------------------
\subsection{3.2. Choice of quantale \texorpdfstring{\(V\)}{V}}
\label{sec:arith-quantale}

The quantale \(V\) underlying the defect potential must reflect the
structure of arithmetic discrepancies.
In this chapter we describe two standard choices, tailored to the FLT /
Iwasawa and Weil modes respectively; in implementation these can be
combined via the product construction of Appendix~S.

\subsubsection{Valuation quantale (FLT / Iwasawa mode)}

In the \(p\)-adic setting, errors arising in control theorems are finite
abelian \(p\)-groups, and their ``size'' is naturally measured by
\(p\)-adic valuation or \(\mathbb{Z}_p\)-length.

\begin{definition}[Valuation quantale \(V_{\mathrm{val}}\)]
\label{def:V-val}
Let
\[
  V_{\mathrm{val}} := \bigl([0,\infty], +, 0, \le \bigr)
\]
with the usual order and monoidal operation given by addition.
Given a finite \(p\)-primary \(\mathbb{Z}_p\)-module \(M\), we define
\[
  \mathrm{size}_p(M)
  \;:=\;
  v_p(\# M)
  \;=\;
  \mathrm{length}_{\mathbb{Z}_p}(M)
  \;\in\; [0,\infty].
\]
For infinite \(p\)-primary modules we set \(\mathrm{size}_p(M) = \infty\),
which is interpreted as an immediate gate failure in the AK diagnostics.
Extensions of finite \(p\)-primary modules correspond to additive
accumulation of length in \(V_{\mathrm{val}}\).
\end{definition}

\subsubsection{Spectral quantale (Weil mode)}

In the complex-analytic setting, discrepancies are measured in terms of
deviation of Frobenius eigenvalues from the expected RH circle and from
equidistribution.

\begin{definition}[Spectral quantale \(V_{\mathrm{spec}}\)]
\label{def:V-spec}
Let
\[
  V_{\mathrm{spec}}
  :=
  \bigl(
    [0,\infty] \times [0,\pi],\ \oplus,\ (0,0),\ \preceq
  \bigr),
\]
with:
\begin{itemize}[leftmargin=2.5em]
  \item monoidal operation
  \(\ (a_1,\theta_1)\oplus(a_2,\theta_2)
      := (a_1 + a_2,\ \min\{\pi,\theta_1+\theta_2\})\),
  \item product order
  \((a_1,\theta_1)\preceq(a_2,\theta_2)\) if and only if
  \(a_1\le a_2\) and \(\theta_1\le \theta_2\).
\end{itemize}
For a Frobenius eigenvalue \(\alpha\) on \(H^i(X)\), with conjectural
weight \(w\) and norm \(q\), we define:
\begin{align*}
  \delta_r(\alpha)
  &:= \bigl|\log|\alpha| - \tfrac{w}{2}\log q\bigr|,\\
  \delta_\theta(\alpha)
  &:= \text{an angle deviation in }[0,\pi]\text{ associated with }\alpha.
\end{align*}
The pair \((\delta_r(\alpha),\delta_\theta(\alpha))\in V_{\mathrm{spec}}\)
is then used as the basic spectral defect associated to \(\alpha\).
Aggregation of defects over all eigenvalues in a given window proceeds
via the monoidal operation \(\oplus\).
\end{definition}

\begin{remark}[Product quantales and mixed modes]
\label{rem:product-quantale}
For programs that simultaneously monitor \(p\)-adic and spectral
discrepancies (for example, where a Weil calibration and an Iwasawa
calibration run jointly), we use the product quantale
\(V = V_{\mathrm{val}}\times V_{\mathrm{spec}}\) with componentwise
operations.
This is compatible with the general quantale product construction of
Appendix~S and with the global defect potential
\(\Defect_\tau : M_{\mathrm{arith}} \to V\).
\end{remark}

% ------------------------------------------------------------
\subsection{3.3. Arithmetic \texorpdfstring{\(\delta\)}{delta}-ledger}
\label{sec:arith-ledger}

We now describe how standard arithmetic operations are mapped into the
\(\delta\)-ledger components
\((\delta^{\alg},\delta^{\disc},\delta^{\meas})\) of the v17.0 core.
This gives the precise interface between arithmetic error sources and
the AK auditor.

\begin{declaration}[Arithmetic \(\delta\)-ledger schema]
\label{dec:arith-ledger}
Let \(U\) be a single step in an arithmetic pipeline in Proof or Hunter
mode (for example, level raising, descent, passage along an Iwasawa
tower, Frobenius iteration).
We associate to \(U\) a triple
\((\delta^{\alg},\delta^{\disc},\delta^{\meas})\in V^3\) as follows.

\begin{enumerate}[label=(L\arabic*), leftmargin=2.5em]
  \item \textbf{Algebraic defect \(\delta^{\alg}\): control-theorem term.}\\
  In Iwasawa-theoretic transitions between an infinite module \(X_\infty\)
  and finite layers \(X_n\), control theorems provide maps with finite
  kernel and cokernel.
  We record these deviations explicitly:
  \[
    \delta^{\alg}(U)
    \;:=\;
    \mathrm{size}_p(\ker U)
    \;+\;
    \mathrm{size}_p(\mathrm{coker}\,U)
    \;\in\; V_{\mathrm{val}},
  \]
  and embed this into the full quantale \(V\) via the canonical
  inclusion \(V_{\mathrm{val}}\hookrightarrow V\).

  \item \textbf{Discretization defect \(\delta^{\disc}\): truncation and approximation.}\\
  This component records deterministic approximation errors, such as:
  \begin{itemize}
    \item finite-degree truncation of \(p\)-adic power series
          (e.g.\ computations modulo \(p^N\)),
    \item truncation of Euler products or zeta integrals at a finite
          cutoff,
    \item finite-dimensional approximations to cohomology or spectral
          decompositions.
  \end{itemize}
  Bounds on \(\delta^{\disc}\) are derived from explicit analytic or
  algebraic error estimates and are encoded as elements of \(V\).

  \item \textbf{Measurement defect \(\delta^{\meas}\): heuristic noise.}\\
  This term is reserved for inherently probabilistic or heuristic
  components, such as:
  \begin{itemize}
    \item probabilistic primality tests,
    \item heuristic class number estimates,
    \item sampling-based spectral statistics.
  \end{itemize}
  In \emph{Proof mode}, the policy is \(\delta^{\meas}\equiv 0\):
  such steps are forbidden.
  In \emph{Hunter / exploration mode}, \(\delta^{\meas}\) may be
  non-zero but must remain bounded by a user-specified budget in
  \texttt{run.yaml}, and cannot be used to certify core
  theorems---only to guide the search.
\end{enumerate}
The total defect attached to a composite pipeline is then accumulated
via the monoidal structure on \(V\), as in Appendix~S.
\end{declaration}

% ------------------------------------------------------------
\subsection{3.4. Bridge policy and arithmetic interpretation}
\label{sec:arith-bridge-policy}

The core bridge \(\PH_1 \Rightarrow \Ext^1\) (Appendix~C) admits a
natural arithmetic interpretation when applied to filtrations arising
from elliptic curves, Selmer complexes, and Frobenius actions.
In this section we record the interpretation and the safety policies
governing its use.

\begin{specification}[Arithmetic bridge interpretation (\SpecBadge)]
\label{spec:arith-bridge}
Let \(\theta \in M_{\mathrm{arith}}\), and let
\(F_\theta = \mathfrak{P}(\theta)\) be its realization.
On any fixed right-open filtration window \(W=[u,u')\subset\mathbb{R}\) and threshold \(\tau>0\)
where the hypotheses of
Appendix~C (constructibility, amplitude \(\le 1\), \(t\)-exactness) are
satisfied, we use the following dictionary:

\begin{itemize}[leftmargin=2.5em]
  \item In FLT/Iwasawa mode, the condition
        \(\PH_1(\Ctau F_\theta) \approx 0\) is interpreted as
        ``no persistent Selmer-type classes'' in the given window:
        after collapse, all candidate extension classes die within the
        noise scale~\(\tau\).

  \item In the same setting, the vanishing
        \(\Ext^1(\Rfun(\Ctau F_\theta),k) \approx 0\) is read as
        ``no cohomological obstructions in the corresponding
        Galois/Selmer complex'' on that window
        (e.g.\ no non-trivial classes in an appropriate
        \(H^1\)-group).

  \item In Weil mode, the disappearance of \(\PH_1(\Ctau F_\theta)\)
        is interpreted as the absence of persistent topological cycles
        induced by Frobenius, while \(\Ext^1\)-vanishing reflects the
        triviality of certain extension data in the realized
        cohomological categories.
\end{itemize}
These interpretations belong to the \SpecBadge{} layer: they provide a
conceptual dictionary between the topological collapse diagnostics and
arithmetic invariants, but they do not assert new arithmetic theorems
beyond the one-way bridge proved in the v17.0 core.
\end{specification}

\begin{specification}[Safety restrictions on bridge usage]
\label{spec:arith-bridge-safety}
In order to avoid overreaching beyond the v17.0 core, the arithmetic
programs obey the following rules when using the bridge:

\begin{enumerate}[label=(B\arabic*), leftmargin=2.5em]
  \item \textbf{Forward usage only in Core.}\\
  The one-way implication
  \[
    \PH_1(\Ctau F_\theta) = 0
    \;\Longrightarrow\;
    \Ext^1(\Rfun(\Ctau F_\theta),k) = 0
  \]
  proved in the core (under the hypotheses of Appendix~C) is always
  admissible in the Core layer.

  \item \textbf{Reverse usage requires explicit certification.}\\
  Any use of a reverse implication
  \(\Ext^1 = 0 \Rightarrow \PH_1 = 0\) must be explicitly marked as
  \SpecBadge{} and justified by an additional certification of the
  window (for example, \(E_1\)-degeneracy or a spectral sequence
  collapse proven independently).
  Such steps do not enter the Core proof ledger.

  \item \textbf{Motivic extrapolation is Spec-only.}\\
  Interpretations involving motivic cohomology, higher \(K\)-groups, or
  non-abelian extensions lie completely in the \SpecBadge{} layer and
  are isolated from the UCC-based Core logic.
  They may guide Hunter search and program design, but cannot be used to
  close a Core-level theorem.
\end{enumerate}
In particular, while conjectures such as the Iwasawa Main Conjecture
suggest deeper two-way bridges, AK--HDPST v17.0 deliberately restricts
itself to the proven one-way implications at the Core level.
\end{specification}



\section{Chapter 4: Tower Diagnostics and Failure Types in Arithmetic Towers}
\label{ch:arith-tower}

This chapter defines how the AK--HDPST engine diagnoses the asymptotic behavior of arithmetic systems.
While Chapter~3 dealt with individual objects, arithmetic proofs (both Weil and FLT) fundamentally rely on the behavior of objects through infinite extensions---\emph{arithmetic towers}.

We apply the tower machinery of the v17.0 core (Appendix~D and~J) to these settings.
The central tool is the \emph{Comparison Map} $\phi_{i,\tau}$ evaluated \emph{after collapse}, and its obstruction indices $(\mu_{\mathrm{Collapse}}, \nu_{\mathrm{Collapse}})$.
This formalism unifies the treatment of ``asymptotic stability'' (Weil) and ``control theorems'' (Iwasawa/FLT).

% ------------------------------------------------------------
\subsection{4.1. Arithmetic Towers and Their Limits}
\label{sec:arith-towers}

We specify the concrete directed systems $(F_n)_{n \in \mathbb{N}}$ and their limits $F_\infty$ for our two calibration targets.

\begin{definition}[The Weil Tower: Extension of Scalars]
Let $(X, \mathbb{F}_q) \in M_{\mathrm{Weil}}$. The natural tower is formed by the constant field extensions $\mathbb{F}_{q^n}$.
\begin{itemize}
    \item \textbf{Finite Layers ($F_n$):} The realization of $X$ over $\mathbb{F}_{q^n}$. The Frobenius action $\mathrm{Fr}_q$ iterates $n$ times.
    \item \textbf{Limit ($F_\infty$):} The realization of $X$ over the algebraic closure $\overline{\mathbb{F}}_q$.
    \item \textbf{AK Perspective:} This is a \emph{colimit} system in persistence. The comparison map $\phi_{i,\tau}$ asks: ``Is the persistence of the geometric limit completely determined by the persistence of the finite layers (after truncation)?''
\end{itemize}
\end{definition}

\begin{definition}[The FLT Tower: Iwasawa Levels]
Let $E \in M_{\mathrm{FLT}}$. The tower corresponds to the layers $K_n$ of the cyclotomic $\mathbb{Z}_p$-extension $\mathbb{Q}_\infty / \mathbb{Q}$.
\begin{itemize}
    \item \textbf{Finite Layers ($F_n$):} Selmer complexes over $K_n$.
    \item \textbf{Limit ($F_\infty$):} The Selmer complex over $K_\infty$, understood operationally as the \emph{direct limit of the Pontryagin duals} of the finite-level Selmer groups, so that the tower is presented to the AK Core as a filtered colimit.
    \item \textbf{AK Perspective:} This is typically an \emph{inverse limit} system in arithmetic algebra, which we pass to a directed system of duals (or an equivalent filtered colimit of cohomology) via Pontryagin duality. The comparison map $\phi_{i,\tau}$ then checks the validity of \emph{Control Theorems} in this colimit presentation.
\end{itemize}
\end{definition}


\begin{specification}[Standard Comparison Map $\phi_{i,\tau}$]
For any arithmetic tower $\mathcal{T} = \{F_n \to F_\infty\}$, the AK Core computes the canonical map (Appendix~D):
\[
  \phi_{i,\tau}(\mathcal{T}) : \quad
  \varinjlim_{n} \mathbf{T}_\tau \big( \mathbf{P}_i(F_n) \big)
  \ \longrightarrow \
  \mathbf{T}_\tau \big( \mathbf{P}_i(F_\infty) \big).
\]
\textbf{Policy.} All terms are truncated by $\mathbf{T}_\tau$ \emph{before} comparison. This ensures that finite arithmetic noise (which may grow with $n$) does not cloud the asymptotic structural diagnosis.
\end{specification}

% ------------------------------------------------------------
\subsection{4.2. Interpreting \texorpdfstring{$(\mu, \nu)$}{(mu, nu)} in Arithmetic}
\label{sec:arith-mu-nu}

The kernel and cokernel of $\phi_{i,\tau}$ yield the diagnostics $(\mu_{\mathrm{Collapse}}, \nu_{\mathrm{Collapse}})$.
We map these topological defects to arithmetic concepts.

\begin{remark}[Meaning of $\mu_{\mathrm{Collapse}}$: The Ghost Kernel]
A value $\mu_{\mathrm{Collapse}} > 0$ means there are persistent classes in the finite layers (surviving $\tau$-collapse) that \emph{vanish} or become invisible in the limit $F_\infty$.
\begin{itemize}
    \item \textbf{Iwasawa Context:} This detects a failure (or weakening) of a Control Theorem. If finite layers contain systematic torsion that is killed in the limit (kernel of restriction), and if this torsion is ``long-lived'' enough to survive $\mathbf{T}_\tau$, it registers as $\mu_{\mathrm{Collapse}} > 0$.
    \item \textbf{Policy:} Under standard Iwasawa Control Theorem hypotheses, the kernel is finite. Thus, for $\tau$ sufficiently large (or via the $\delta$-ledger accounting of Appendix~R), we expect $\mu_{\mathrm{Collapse}} \to 0$ in the calibrated regimes.
\end{itemize}
\end{remark}

\begin{remark}[Meaning of $\nu_{\mathrm{Collapse}}$: The Phantom Cokernel]
A value $\nu_{\mathrm{Collapse}} > 0$ means the limit $F_\infty$ contains persistent features that are \emph{not} generated by any finite layer $F_n$.
\begin{itemize}
    \item \textbf{Weil Context:} Heuristically, this would correspond to a ``transcendental'' cycle in $X_{\overline{\mathbb{F}}_q}$ that is never defined over any finite extension $\mathbb{F}_{q^n}$. In the classical theory of smooth projective varieties over finite fields (rationality of zeta functions, comparison isomorphisms), such phenomena are not expected to occur.
    \item \textbf{FLT Context:} A ``phantom'' Selmer class that exists only at infinity but restricts to zero at every finite level; in practice, standard control statements aim to rule this out in the calibration regime.
\end{itemize}
\end{remark}

\begin{remark}[Separation from the Classical Iwasawa $\mu$]
We reiterate the warning from Appendix~R: $\mu_{\mathrm{Collapse}}$ is \textbf{not} the classical Iwasawa $\mu$-invariant (which measures the size of a module). The index $\mu_{\mathrm{Collapse}}$ measures the \emph{defect of the limit transition}.
A tower can have a large classical Iwasawa $\mu$-invariant but perfect control ($\mu_{\mathrm{Collapse}} = 0$), and conversely.
\end{remark}

% ------------------------------------------------------------
\subsection{4.3. The Failure Landscape (Type I--IV)}
\label{sec:arith-failure}

We classify the possible outcomes of an AK audit on an arithmetic parameter $\theta$.
The labels mirror those of the global v17.0 Core, specialized to the arithmetic setting.

\begin{definition}[Arithmetic Failure Types]
When B-Gate$^{+}$ or the Overlap Gate fails for $\theta$, we classify the failure as follows:

\begin{itemize}
    \item \textbf{Type I (Topological): $\mathrm{PH}_1 \neq 0$.}
    A robust bar persists after collapse.
    \begin{itemize}
        \item \emph{Weil (Interpretation).} A Type~I failure would correspond, in a Weil calibration thought experiment, to the detection of an eigenvalue $\alpha$ strictly off the circle $|\alpha| = q^{w/2}$---i.e.\ a configuration that would behave like a counterexample to the finite-field Riemann Hypothesis.
        \item \emph{FLT (Interpretation).} Similarly, a Type~I failure in the FLT program would behave like a non-trivial persistent Selmer class associated with a hypothetical solution $(A,B,C)$---a configuration that would act as a counterexample to FLT if it were realized in the classical setting.
    \end{itemize}

    \item \textbf{Type II (Categorical): $\Ext^1 \neq 0$.}
    The object fails to split or trivialize in the derived category, even if $\mathrm{PH}_1 = 0$. In the arithmetic calibration programs, this would signal a subtle extension-level obstruction beyond the reach of purely homological persistence.

    \item \textbf{Type III (Instability): Budget Overflow.}
    Numerical noise ($\delta^{\mathrm{disc}}$) or commutation errors ($\Delta_{\mathrm{comm}}$) exceed the safety gap. This indicates that the chosen resolution/precision is insufficient. It is a failure of the computational pipeline, not necessarily evidence for or against the underlying arithmetic conjecture.

    \item \textbf{Type IV (Essential Singularity): $(\mu_{\mathrm{Collapse}}, \nu_{\mathrm{Collapse}}) \neq (0,0)$.}
    \textbf{The critical tower failure.} The finite layers suggest validity (collapse) in each bounded window, but the limit object refuses to collapse.
    \begin{itemize}
        \item This represents an ``infinite descent'' or ``blow-up'' scenario where the arithmetic object degenerates asymptotically in the tower.
        \item In the context of FLT, finding a Type~IV failure in a hypothetical search regime would correspond to a system of points that locally mimics a modular form at all finite levels but fails to assemble into a global modular object (a failure of modularity lifting) in the limit.
    \end{itemize}
\end{itemize}
\end{definition}

% ------------------------------------------------------------
\subsection{4.4. Conclusion: The Common Foundation}
\label{sec:arith-tower-conclusion}

Chapters~\ref{ch:arith-param}, \ref{ch:arith-collapse}, and~\ref{ch:arith-tower} have established the \textbf{Arithmetic AK Core}.
We have:
\begin{enumerate}
    \item Defined the space and time (arithmetic parameter spaces and filtrations).
    \item Defined the laws of motion (collapse contract, quantales, and the arithmetic $\delta$-ledger).
    \item Defined the instrumentation (tower diagnostics and failure types).
\end{enumerate}

With this common foundation, Part~II can now run the machinery on the Weil calibration program, demonstrating that the AK--verdict aligns with the known \emph{Valid} outcome.
This serves as a system-level calibration before turning to the hypothetical counterexample space for FLT.



% ------------------------------------------------------------
% Part II : AK Program for Weil Conjectures / Finite-Field RH
% ------------------------------------------------------------
\part{AK Program for Finite-Field Riemann Hypothesis (Calibration)}
\label{part:weil}

\section{Chapter 5: Zeta Functions, Frobenius Spectra, and AK Realizations}
\label{ch:weil-realization}

This chapter bridges the classical arithmetic geometry of varieties over finite fields with the AK--HDPST topological machinery.
Our goal is to construct the \emph{AK Realization} of the Weil Conjectures, transforming the spectral properties of the Frobenius endomorphism into a Persistence/Collapse problem.

We rely on the parameter space $M_{\mathrm{Weil}}$ (Chapter~2) and the Spectral Quantale $V_{\mathrm{spec}}$ (Chapter~3).
The central object constructed here is the \textbf{Frobenius Defect Potential} $\Phi^{\mathrm{Weil}}_\tau$, which serves as the objective function for the Hunter agents in the subsequent calibration.

% ------------------------------------------------------------
\subsection{5.1. The Weil Zeta Function (Classical Core)}
\label{sec:weil-zeta}

We briefly review the classical definitions to fix the target for our realization functor $\mathfrak{P}$.

\begin{definition}[Zeta Function]
Let $X$ be a smooth projective variety of dimension $d$ over $\mathbb{F}_q$. The zeta function is defined as the formal power series:
\[
  Z(X, t) \ := \ \exp\left( \sum_{n=1}^{\infty} \frac{\# X(\mathbb{F}_{q^n})}{n} t^n \right).
\]
\end{definition}

\begin{theorem}[Rationality and Spectral Decomposition (Dwork, Grothendieck)]
The zeta function is a rational function. Specifically, via $\ell$-adic cohomology, it admits the factorization:
\[
  Z(X, t) \ = \ \prod_{i=0}^{2d} P_i(t)^{(-1)^{i+1}}, \qquad P_i(t) \ = \ \det\left( \mathrm{Id} - t \cdot \mathrm{Fr}_q^* \mid H^i_{\text{\'{e}t}}(X_{\overline{\mathbb{F}}_q}, \mathbb{Q}_\ell) \right).
\]
Here, $P_i(t) \in \mathbb{Z}[t]$ (independent of $\ell$) and factorizes over $\mathbb{C}$ as $\prod_{j=1}^{b_i} (1 - \alpha_{ij}t)$.
\end{theorem}

\begin{declaration}[The Target: Finite-Field RH]
The AK calibration targets the third Weil conjecture (Deligne's Theorem):
\[
  |\alpha_{ij}| \ = \ q^{i/2} \quad \text{for all } 1 \le j \le b_i.
\]
In the AK framework, any deviation from this equality is interpreted as a \textbf{Type~I Topological Failure} (a persistent feature with the "wrong" lifetime scaling).
\end{declaration}

% ------------------------------------------------------------
\subsection{5.2. AK Realization: From Frobenius to Persistence}
\label{sec:weil-ak-realization}

We specify the functor $\mathfrak{P}: M_{\mathrm{Weil}} \to \FiltCh{k}$ defined conceptually in Chapter~2, focusing on how it encodes the spectrum.

\begin{specification}[Frobenius Mapping Torus Realization]
Let $V^i = H^i_{\text{\'{e}t}}(X, \mathbb{Q}_\ell)$ and $\phi = \mathrm{Fr}_q^*$.
We construct a filtered chain complex $F_{\mathrm{Weil}}$ whose homology in degree $i$ approximates the \emph{polylogarithmic filtration} of the Frobenius action.

Conceptually, the persistence module $\mathbf{P}_i(F_{\mathrm{Weil}})$ tracks the eigenspaces of $\phi$.
For a time parameter $s \in \mathbb{R}_{\ge 0}$, we define the filtration such that a class $v \in V^i$ with $\phi(v) = \alpha v$ persists with a ``lifetime'' or ``scaling factor'' related to $\log_q |\alpha|$.

\textbf{Operational Construction (for numerical/AI auditing):}
Instead of literally implementing infinite-dimensional mapping tori, the AK Core uses the \textbf{Spectral Indicators} (Chapter~11) directly on the linear operator $\phi$.
\begin{enumerate}
    \item \textbf{Input:} The matrix representation of $\mathrm{Fr}_q^*$ on $H^i(X)$ (computed via point counting or cohomological algorithms).
    \item \textbf{AK Object:} A persistence surrogate where the ``bar length'' $\beta(\alpha)$ for an eigenvalue $\alpha$ is normalized to:
    \[
      \beta(\alpha) \ := \ \frac{2 \log |\alpha|}{\log q}.
    \]
    \item \textbf{Target:} Under RH, we expect $\beta(\alpha) \equiv i$ (the cohomological degree).
\end{enumerate}
\end{specification}

\begin{remark}[After-Collapse Policy consistency]
By applying $\mathbf{T}_\tau$ (with $\tau$ adapted to the numerical precision of the eigenvalue computation), we filter out:
\begin{itemize}
    \item Numerical noise in the eigenvalue approximation ($\delta^{\mathrm{meas}}$).
    \item Artifacts from finite-field extensions if approximating $H^i$ via bounded towers.
\end{itemize}
Only the stable eigenvalues $\alpha_{ij}$ constitute the ``persistent homology'' of this arithmetic setup.
\end{remark}

% ------------------------------------------------------------
\subsection{5.3. The Defect Potential \texorpdfstring{$\Phi^{\mathrm{Weil}}_\tau$}{Phi\_Weil}}
\label{sec:weil-defect}

We define the scalar field that the Hunter agents will explore. In this calibration setting, we know theoretically that the potential is zero everywhere, but the system must verify this.

\begin{definition}[Frobenius Defect Potential]
For a parameter $\theta = (X, \mathbb{F}_q) \in M_{\mathrm{Weil}}$, let $\{\alpha_{ij}\}$ be the set of Frobenius eigenvalues in degree $i$ recovered from $\mathbf{T}_\tau \mathbf{P}_i(\mathfrak{P}(\theta))$.
Using the Spectral Quantale $V_{\mathrm{spec}}$ (Chapter~3), we define the defect:
\[
  \Phi^{\mathrm{Weil}}_\tau(\theta) \ := \ \sum_{i=0}^{2d} \sum_{j=1}^{b_i} \left| \log |\alpha_{ij}| - \frac{i}{2} \log q \right|^2.
\]
(Alternatively, using the $V_{\mathrm{spec}}$ norm notation: $\|\Sigma\delta_{\mathrm{spec}}\|_V$.)
\end{definition}

\begin{interpretation}[Regime Classification]
Based on $\Phi^{\mathrm{Weil}}_\tau(\theta)$, the AK Core classifies the point $\theta$:
\begin{itemize}
    \item \textbf{Plain (Valid):} $\Phi \le \delta^{\mathrm{disc}}$ (pure numerical noise). This corresponds to RH holding true.
    \item \textbf{Ridge (Noise):} $\delta^{\mathrm{disc}} < \Phi < \lambda_{\mathrm{sing}}$. (Not expected in Weil context if algorithms are correct; indicates potential code error or extreme numerical instability).
    \item \textbf{Peak (Counterexample):} $\Phi \ge \lambda_{\mathrm{sing}}$. A definitive violation of the Riemann Hypothesis weights.
\end{itemize}
\end{interpretation}

% ------------------------------------------------------------
\subsection{5.4. Integration with the Gate Cascade}
\label{sec:weil-gate}

The validation of a terrain cell $U \subset M_{\mathrm{Weil}}$ proceeds via the standard Gate Cascade (Chapter~1/9), specialized for spectral data.

\begin{specification}[Weil--Spectral Pre-Gate (input/measurement sanity check)]
\label{spec:weil-pregate}
Let \(U\subset M_{\mathrm{Weil}}\) be a parameter cell (e.g.\ a definable family of curves/varieties over \(\mathbb{F}_q\)).
For each sampled \(\theta\in U\), the program computes the defect potential \(\Phi^{\mathrm{Weil}}_\tau(\theta)\)
from the \emph{after-collapse} spectral data.

The pre-gate \textsf{Weil-PreGate} checks:
\begin{enumerate}
  \item \textbf{Parameter stability:} the expected discrete invariants (e.g.\ Betti numbers \(b_i\)) are constant across \(U\).
  \item \textbf{Spectral budget:} for all sampled \(\theta \in U\),
  \(\Phi^{\mathrm{Weil}}_\tau(\theta)\) stays within the declared \(\delta\)-ledger budget (see \S6.2).
\end{enumerate}
If passed, the program issues an \emph{input/measurement-layer certificate} that \(U\) is \(\mathsf{Valid}\)
for the calibration task.

\emph{Note.} This is a pre-gate: it does not redefine the v17 Core predicate \emph{B-Gate$^{+}$};
it only certifies that the embedded arithmetic data are consistent with the classical boundary condition (Deligne)
up to the declared ledger budget.
\end{specification}


\begin{remark}[The Calibration Logic]
In the subsequent chapters, we will simulate the Hunter traversing $M_{\mathrm{Weil}}$.
Because Deligne's theorem is true, the Hunter will find \emph{no} descent directions (gradients of $\Phi$ will be noise-dominated) and \emph{no} Peaks.
The Map of Validity will consist of a single, global ``Plain of Truth.''
This trivial behavior is exactly the \textbf{calibration signal} we seek: it confirms that the AK diagnostics $(\mu, \nu, \Phi)$ correctly identify a regular, obstruction-free arithmetic world.
\end{remark}



% ------------------------------------------------------------
% Part II : AK Program for Weil Conjectures / Finite-Field RH
% ------------------------------------------------------------
\part{AK Program for Finite-Field Riemann Hypothesis (Calibration)}
\label{part:weil}

\section{Chapter 6: Collapse Contract and the Finite-Field Riemann Hypothesis}
\label{ch:weil-contract}

This chapter operationalizes the Unified Collapse Contract (UCC) for the Weil Conjectures.
Having defined the defect potential $\Phi^{\mathrm{Weil}}_\tau$ in Chapter~\ref{ch:weil-realization}, we now specify the acceptance criteria---the \emph{Spectral UCC}---that the AK Core uses to certify a terrain cell $U \subset M_{\mathrm{Weil}}$ as \emph{Valid}.

This is a \emph{calibration exercise}: we utilize the known truth of Deligne's theorem to tune the sensitivity of the AK $\delta$-ledger.
If the system is correctly calibrated, the ``Plain of Truth'' (Valid region) must cover the entire parameter space $M_{\mathrm{Weil}}$.

% ------------------------------------------------------------
\subsection{6.1. The Spectral UCC: Filtering via \texorpdfstring{$\mathbf{T}_\tau$}{T\_tau}}
\label{sec:weil-spectral-ucc}

In the standard AK theory, $\mathbf{T}_\tau$ deletes short bars.
In the spectral realization of the Weil conjectures, this translates to filtering out eigenvalues that violate the weight constraints or arise from numerical noise.

\begin{definition}[Spectral Truncation $\mathbf{T}_\tau^{\mathrm{spec}}$]
Let $F$ be the Frobenius realization of a variety $X$.
The operator $\mathbf{T}_\tau^{\mathrm{spec}}$ acts on the detected spectrum $\sigma(\mathrm{Fr}_q)$ as follows:
\begin{itemize}
  \item \textbf{Input:} A set of approximate eigenvalues $\{\tilde{\alpha}_j\}$ and a tolerance width $\varepsilon(\tau)$ derived from the $\tau$-scale.
  \item \textbf{Filter:} An eigenvalue $\tilde{\alpha}$ is \emph{retained} (considered a persistent feature) if and only if it lies within distance $\varepsilon(\tau)$ of the circle
  \[
    |z| \;=\; q^{w/2}
  \]
  for some integer weight $w \in \{0,\dots,2d\}$.
  \item \textbf{Collapse:} Eigenvalues falling into the ``spectral gap'' (far from any integer-weight circle) are treated as transient noise and annihilated (mapped to zero persistence).
\end{itemize}
\end{definition}

\begin{specification}[The Spectral Collapse Condition]
A parameter $\theta \in M_{\mathrm{Weil}}$ satisfies the \emph{Spectral UCC} at scale $\tau$ if, after applying $\mathbf{T}_\tau^{\mathrm{spec}}$:
\begin{enumerate}
  \item \textbf{Betti Stability (No Spurious Contribution):}
  The surviving spectrum yields Betti numbers that agree with those of $X$;
  in particular, no spurious Betti contribution is created by numerical noise or truncation.
  \item \textbf{Defect Vanishing:}
  The renormalized defect potential vanishes within the allowed budget:
  \[
    \mathbf{T}_\tau^{\mathrm{spec}}\bigl(\Phi^{\mathrm{Weil}}_\tau(\theta)\bigr)
    \;\approx\; 0
    \quad \text{(within the total budget $\Sigma\delta$).}
  \]
\end{enumerate}
\end{specification}


% ------------------------------------------------------------
\subsection{6.2. The Arithmetic \texorpdfstring{$\delta$}{delta}-Ledger}
\label{sec:weil-ledger}

Since we cannot compute with infinite precision, the condition $\Phi^{\mathrm{Weil}}_\tau = 0$ is never met exactly in a digital simulation.
We map the specific error sources of arithmetic geometry into the AK $\delta$-ledger (Appendix~G/S) to distinguish ``numerical non-zero'' from a genuine ``topological counterexample.''

\begin{declaration}[Weil Ledger Entries]
For a computation on a parameter cell $U \subset M_{\mathrm{Weil}}$:

\begin{enumerate}
  \item \textbf{Discretization Defect ($\delta^{\mathrm{disc}}$):}
  Arises from approximating the cohomology $H^i(X)$ or the zeta function $Z(X,t)$.
  \begin{itemize}
    \item \emph{Source:} Finite truncation of the point-counting series
    \(
      N_r = \#X(\mathbb{F}_{q^r})
    \)
    (e.g.\ using only $r \le R$).
    \item \emph{Bound:} Analytic number theory gives tail bounds of the form
    \(
      |\text{tail}| \le C \cdot q^{-R/2}
    \)
    for an explicit constant $C$ depending on $X$.
    \item \emph{Entry:} We record a bound
    \(
      \delta^{\mathrm{disc}} \simeq C \cdot q^{-R/2}.
    \)
  \end{itemize}

  \item \textbf{Measurement Defect ($\delta^{\mathrm{meas}}$):}
  Arises from the floating-point or $p$-adic precision of the eigensolver.
  \begin{itemize}
    \item \emph{Source:} Computing eigenvalues of a large sparse matrix over $\mathbb{C}$ or $\mathbb{C}_p$.
    \item \emph{Entry:} The solver tolerance (e.g.\ $10^{-12}$) or the $p$-adic truncation order.
  \end{itemize}

  \item \textbf{Algebraic Defect ($\delta^{\mathrm{alg}}$):}
  \begin{itemize}
    \item \emph{Source:} None in the smooth projective case: we assume $X$ is smooth and projective, so no extra algebraic correction terms appear from resolutions.
    \item \emph{Entry:} $\delta^{\mathrm{alg}} := 0$ for smooth $X$.
  \end{itemize}
\end{enumerate}

\textbf{Pass Condition:} \textsf{Weil-PreGate} accepts the point $\theta$ if
\[
  \Phi^{\mathrm{Weil}}_\tau(\theta)
  \;\le\;
  \delta^{\mathrm{disc}} \oplus \delta^{\mathrm{meas}},
\]
where $\oplus$ is the aggregation operation in the Spectral Quantale $V_{\mathrm{spec}}$.
\end{declaration}

% ------------------------------------------------------------
\subsection{6.3. The Calibration Theorem (AK--Perspective)}
\label{sec:weil-calibration-thm}

We now formalize the goal of this Part: establishing that the AK machinery correctly identifies the ``truth'' when provided with correct data.

\begin{theorem}[AK--Weil Consistency]\label{thm:ak-weil-consistency}
Assume the classical theorems of Weil (1949), Dwork (1960), Grothendieck et al.\ (1960s), and Deligne (1974).
Construct the AK realization $\mathfrak{P}(\theta)$ for any $\theta \in M_{\mathrm{Weil}}$ as per Chapter~\ref{ch:weil-realization}.
Then:
\begin{enumerate}
  \item \textbf{Universal Validity:}
  For any scale $\tau > 0$ and any window $W$, the defect potential satisfies
  \[
    \Phi^{\mathrm{Weil}}_\tau(\theta) \;=\; 0
    \qquad \text{(mathematically, i.e.\ before numerical truncation).}
  \]
  \item \textbf{No Type IV Failures:}
  For the extension tower $\mathbb{F}_{q^n}$,
  the tower diagnostics satisfy
  \[
    \bigl(\mu_{\mathrm{Collapse}}, \nu_{\mathrm{Collapse}}\bigr)
    \;=\;
    (0,0).
  \]
  \item \textbf{AK Verdict:}
  In the ideal (non-numerical) regime, the system outputs
  \[
    \AKverdict(\theta) \;=\; \mathsf{Valid}
    \quad \text{for all } \theta \in M_{\mathrm{Weil}}.
  \]
\end{enumerate}
\end{theorem}

\begin{proof}[Logic of the Calibration]
This is not a proof of the Weil conjectures.
It is a proof that \emph{AK--HDPST is a faithful observer} when the input arithmetic world already satisfies the conjectures.

\begin{itemize}
  \item Deligne's theorem implies that all eigenvalues $\alpha_{ij}$ lie exactly on the circles
  \(
    |\alpha_{ij}| = q^{i/2}.
  \)
  Hence the radial deviation term in $\Phi^{\mathrm{Weil}}_\tau$ is identically zero.
  \item Rationality (Dwork/Grothendieck) and the cohomological formalism imply that the Betti numbers are already realized over finite extensions, and no new persistent features appear at the algebraic closure.
  Thus the comparison map for the tower has trivial kernel and cokernel after collapse, giving
  \(
    \mu_{\mathrm{Collapse}} = \nu_{\mathrm{Collapse}} = 0.
  \)
  \item Therefore, the AK gate cascade, operating strictly on $\mathbf{T}_\tau$-collapsed data and with an error budget dominated by $(\delta^{\mathrm{disc}},\delta^{\mathrm{meas}})$, must pass every point $\theta$ when numerical defects are abstracted away.
\end{itemize}
\end{proof}

% ------------------------------------------------------------
\subsection{6.4. Epistemological Status: Calibration vs.\ Proof}
\label{sec:weil-epistemology}

It is vital to distinguish the role of this chapter from the Navier--Stokes application (Appendix~NS).

\begin{remark}[The Calibration Paradigm]
\begin{itemize}
  \item \textbf{In Navier--Stokes (Future):}
  We do not know if regularity holds.
  We use AK--HDPST to \emph{discover} the Map of Validity.
  The system may output $\mathsf{Obstructed}$ (Type~IV) on some regions.
  \item \textbf{In Weil (Here):}
  We know regularity (finite-field RH) holds.
  We use the known truth to \emph{debug and calibrate} the AK--HDPST parameters (e.g.\ quantale sensitivity, choice of $\tau$, window size, and ledger thresholds).
\end{itemize}
If the AK Core were to report a ``Peak'' (counterexample) in $M_{\mathrm{Weil}}$ under this assumption, it would signify a bug in the implementation of the realization functor $\mathfrak{P}$ or in the numerical solver, not a disproof of the finite-field Riemann Hypothesis.
This asymmetry is the defining feature of a calibration test.
\end{remark}



\section{Chapter 7: Validity Map on \texorpdfstring{$M_{\mathrm{Weil}}$}{M\_Weil}}
\label{ch:weil-validity-map}

In this chapter, we assemble the local diagnostics of the previous chapters into a global geometric structure: the \emph{Validity Map}.
For the Weil calibration, this map serves as a ``control experiment.'' 
Since the Finite-Field Riemann Hypothesis is a proven theorem, the Validity Map must exhibit a specific, trivial topology: it should be devoid of essential singularities.

This chapter defines the terrain regions---\emph{Plain}, \emph{Ridge}, and \emph{Peak}---specifically for the spectral data of Frobenius, establishing the reference standard for what a ``solved'' arithmetic problem looks like to the AK--HDPST engine.

% ------------------------------------------------------------
\subsection{7.1. Definition of the Map \texorpdfstring{$\mathcal{V}_\tau$}{V\_tau}}
\label{sec:weil-map-def}

The Validity Map is not just a function but a stratification of the parameter space based on the AK verdict.

\begin{definition}[The Spectral Validity Map]
Fix a collapse scale $\tau > 0$ and a $\delta$-ledger budget $\mathcal{B}$.
The \emph{Validity Map} on $M_{\mathrm{Weil}}$ is the assignment
\[
  \mathcal{V}_\tau : M_{\mathrm{Weil}} \longrightarrow \{\mathsf{Plain}, \mathsf{Ridge}, \mathsf{Peak}\}
\]
determined by the value of the defect potential $\Phi^{\mathrm{Weil}}_\tau(\theta)$ and the tower diagnostics $(\mu_{\mathrm{Collapse}}, \nu_{\mathrm{Collapse}})$ relative to the budget.
\end{definition}

\begin{specification}[Classification Criteria]
For a parameter $\theta \in M_{\mathrm{Weil}}$ with defect $\Phi = \Phi^{\mathrm{Weil}}_\tau(\theta)$:

\begin{enumerate}
  \item \textbf{Plain (The Valid Region $Z_{\mathrm{Valid}}$):}
  \[
    \Phi \le \delta^{\mathrm{disc}} \oplus \delta^{\mathrm{meas}}
    \quad \text{AND} \quad
    (\mu_{\mathrm{Collapse}}, \nu_{\mathrm{Collapse}}) = (0, 0).
  \]
  The point satisfies the Spectral UCC within the allocated noise budget.

  \item \textbf{Ridge (The Noise Region $Z_{\mathrm{Noise}}$):}
  \[
    \delta^{\mathrm{disc}} \oplus \delta^{\mathrm{meas}} < \Phi < \lambda_{\mathrm{sing}}
    \quad \text{OR} \quad
    (\mu_{\mathrm{Collapse}}, \nu_{\mathrm{Collapse}}) \text{ indeterminate/unstable}.
  \]
  The point is ambiguous. In the Weil context, this signifies numerical instability (e.g.\ insufficient $p$-adic or floating-point precision) rather than a genuine counterexample.

  \item \textbf{Peak (The Singular Region $Z_{\mathrm{Sing}}$):}
  \[
    \Phi \ge \lambda_{\mathrm{sing}}
    \quad \text{OR} \quad
    (\mu_{\mathrm{Collapse}}, \nu_{\mathrm{Collapse}}) \neq (0, 0) \text{ robustly}.
  \]
  The point is a certified obstruction: a Type~I or Type~IV failure.
\end{enumerate}
\end{specification}

% ------------------------------------------------------------
\subsection{7.2. Topography of the Calibration Space}
\label{sec:weil-topography}

We now state what the map \emph{looks like} under the assumption of classical truth.

\begin{theorem}[The Plain of Truth]\label{thm:weil-plain}
Assume the validity of Deligne's Theorem (Riemann Hypothesis for varieties over finite fields).
Then, for any ``ideal'' AK implementation (where $\delta^{\mathrm{meas}} \to 0$ and $\delta^{\mathrm{disc}}$ is handled by exact analytic bounds),
\[
  Z_{\mathrm{Valid}} \ = \ M_{\mathrm{Weil}},
  \qquad
  Z_{\mathrm{Sing}} \ = \ \emptyset.
\]
Geometrically, the Validity Map is a single connected component of type $\mathsf{Plain}$.
\end{theorem}

\begin{corollary}[Empty Peak Condition]
Under the same assumptions, there are no ``mountains'' in $M_{\mathrm{Weil}}$.
The Defect Potential $\Phi^{\mathrm{Weil}}_\tau$ is globally minimized at (structurally) zero everywhere.
Any observed gradient $\nabla \Phi$ in a numerical simulation is purely an artifact of the variation in $\delta^{\mathrm{disc}}$ (for instance, bounds getting tighter or looser as dimension or genus changes), not an intrinsic arithmetic feature.
\end{corollary}

% ------------------------------------------------------------
\subsection{7.3. The Global Certificate}
\label{sec:weil-certificate}

Since the Peak region is empty, the Mapper agent can construct a trivial covering.

\begin{definition}[Global Certificate for Weil]
A \emph{Global Certificate} is a directed acyclic graph (DAG) of verified cells covering $M_{\mathrm{Weil}}$.
For the Weil program, this graph is structurally trivial:
\begin{itemize}
  \item \textbf{Nodes:} Cells $W_\alpha$ covering $M_{\mathrm{Weil}}$ (for example, stratified by genus or dimension).
  \item \textbf{Edges:} Overlap passes showing consistency of Betti numbers between neighboring cells.
  \item \textbf{State:} All nodes are colored \textcolor{green!50!black}{\textbf{GREEN}} (Valid).
\end{itemize}
\end{definition}

\begin{remark}[Contrast with FLT]
In Part~III (FLT), the certificate will look radically different.
We expect $Z_{\mathrm{Valid}}$ to coincide with the locus of modular elliptic curves, while the ``hypothetical'' part of $M_{\mathrm{FLT}}$ corresponding to Frey curves of putative solutions is mapped to $Z_{\mathrm{Sing}}$ (or excluded entirely by \textsf{Weil-PreGate}).
\end{remark}

% ------------------------------------------------------------
\subsection{7.4. AK--Perspective: The Ideal Model for NSE}
\label{sec:weil-nse-link}

This chapter establishes the ``Gold Standard'' for the Navier--Stokes investigation (Appendix~NS).

\begin{remark}[The Calibration Lesson]
The Weil Calibration teaches us that:
\begin{enumerate}
  \item \textbf{Regularity manifests as flatness.}
  A problem with global regularity (like the Weil setting) appears to the AK engine as a featureless plain where $\Phi \approx 0$ everywhere.

  \item \textbf{Budgets matter.}
  Even on the Plain of Truth, $\Phi$ is never exactly zero in a computer.
  The distinction between $Z_{\mathrm{Valid}}$ and $Z_{\mathrm{Sing}}$ relies entirely on the rigorous derivation of the budget
  $\delta^{\mathrm{disc}} \oplus \delta^{\mathrm{meas}}$ (Chapter~6).
\end{enumerate}
When we tackle Navier--Stokes, we will search for the same kind of flatness.
If we find a region where $\Phi$ rises above the budget and cannot be flattened by the Lifter, we have found a singularity.
If we find only flatness (up to budget), we have evidence in favor of regularity.
\end{remark}



\section{Chapter 8: HDPS Hunter/Mapper on the Weil Side (Conceptual Execution)}
\label{ch:weil-execution}

This chapter describes the dynamic execution of the AK--HDPST agents on the Weil parameter space.
Unlike static proofs, the HDPS engine operates as a \emph{search process}.
In this calibration setting, we simulate a ``Hostile Audit'' where the Hunter agent actively tries (and fails) to disprove the Riemann Hypothesis.

The inability of the Hunter to find a gradient of ascent in the Defect Potential $\Phi^{\mathrm{Weil}}_\tau$ serves as the \textbf{empirical validation} of the entire AK framework.

% ------------------------------------------------------------
\subsection{8.1. The Hunter's Protocol: Seeking Gradients}
\label{sec:weil-hunter-protocol}

The Hunter agent is configured with an aggressive optimization policy.

\begin{specification}[Hunter Policy $\mathcal{H}_{\mathrm{Weil}}$]
\begin{itemize}
  \item \textbf{Objective Function:} Maximize the defect potential $\Phi^{\mathrm{Weil}}_\tau(\theta)$.
  \item \textbf{Action Space:}
  \begin{itemize}
    \item \emph{Continuous:} Perturb coefficients of the defining polynomial equations in $\mathbb{Z}_p$ or finite field lifts.
    \item \emph{Discrete:} Jump between fiber dimensions, genera, or primes $q$.
  \end{itemize}
  \item \textbf{Stopping Condition:}
  \begin{itemize}
    \item \emph{Success (Counterexample):} Find $\theta$ where $\Phi^{\mathrm{Weil}}_\tau(\theta) > \lambda_{\mathrm{sing}}$.
    \item \emph{Failure (Regularity):} Explore $N$ steps without finding any direction where
    \[
      \|\nabla \Phi^{\mathrm{Weil}}_\tau(\theta)\| \cdot \Delta \theta \;>\; \delta^{\mathrm{friction}}
    \]
    for some fixed friction threshold $\delta^{\mathrm{friction}}$ in the error budget.
  \end{itemize}
\end{itemize}
\end{specification}

\begin{definition}[Quantale Friction]
The $\delta$-ledger budget imposes an effective ``friction'' on the search landscape.
If the estimated gain in defect potential is dominated by numerical noise, i.e.
\[
  \|\nabla \Phi^{\mathrm{Weil}}_\tau(\theta)\| \cdot \Delta \theta \ \le \ \delta^{\mathrm{meas}},
\]
the Hunter considers the terrain locally flat and refuses to move.
This prevents the AI from chasing ``phantom gradients'' created by floating-point or $p$-adic rounding errors.
\end{definition}

% ------------------------------------------------------------
\subsection{8.2. Execution Log: The Null Gradient}
\label{sec:weil-execution-log}

We present a conceptual trace of the Hunter operating on $M_{\mathrm{Weil}}$ under the assumption of classical truth (Deligne).

\begin{example}[Hunter Trace: Project Weil-Audit]
\begin{verbatim}
[INIT]  Target: Weil Conjectures (Finite Fields)
[CONF]  Quantale: V_spec | Tau: 1e-4 | Ledger: Conservative
[START] Spawning Hunter-01 at random seed (Genus=5, q=49) ...

[STEP 1] Computing Zeta Function Z(X, t)...
         > Eigenvalues: All within 1e-12 of |z| = q^{w/2}.
         > Phi_tau: 1.2e-13 (<< delta_disc).
         > Status: PLAIN.

[STEP 2] Attempting Gradient Ascent (Deforming coefficients)...
         > Perturbation: c_0 -> c_0 + epsilon.
         > New Phi_tau: 1.3e-13.
         > Gradient: ~0 (Masked by friction).
         > Action: REJECT move (No ascent detected).

[STEP 3] Jump Strategy (Hyper-elliptic locus)...
         > Checking singular fibers...
         > (Resolution of singularities applied automatically via P_fun)
         > Phi_tau: 5.5e-14.
         > Status: PLAIN.

... [10,000 steps later] ...

[END]   Hunter-01 stalled. No ascent direction found.
        Max Phi observed: 2.1e-13 (Budget: 1.0e-6).
        Coverage: 98% of strata sampled.
        Peaks found: 0.
\end{verbatim}
\end{example}

\begin{remark}[The Meaning of Silence]
The Hunter's failure to move significantly away from $\Phi^{\mathrm{Weil}}_\tau \approx 0$ confirms that the manifold $M_{\mathrm{Weil}}$ is ``hydrostatically stable'' with respect to the Unified Collapse Contract.
There are no pockets of high defect energy for the Hunter to exploit.
\end{remark}

% ------------------------------------------------------------
\subsection{8.3. The Mapper's Job: Trivial Covering}
\label{sec:weil-mapper}

While the Hunter searches for peaks, the Mapper aggregates the validated paths into the Global Certificate.

\begin{theorem}[Structure of the Weil Map]
Under the calibration assumption, the output of the Mapper is a \emph{Trivial Homotopy Type} certificate.
\begin{enumerate}
  \item \textbf{Connectivity:} All sampled cells $\{W_\alpha\}$ are connected via overlap passes.
  \item \textbf{Uniformity:} Every cell label is $\mathsf{Valid}$.
  \item \textbf{No Type IV Boundaries:} There are no internal boundaries where the tower diagnostics $(\mu_{\mathrm{Collapse}}, \nu_{\mathrm{Collapse}})$ jump from $(0,0)$ to non-zero values.
\end{enumerate}
This structure is the AK--HDPST definition of a \emph{Solved Regular Problem}.
\end{theorem}

% ------------------------------------------------------------
\subsection{8.4. Calibration Sign-off}
\label{sec:weil-signoff}

This concludes the Weil Calibration Program (Part~II).

\begin{declaration}[System Readiness]
The successful reconstruction of the ``Plain of Truth'' on $M_{\mathrm{Weil}}$ confirms:
\begin{enumerate}
  \item \textbf{$\mathfrak{P}$ is Faithful:} The realization functor captures the spectral rigidity of Frobenius.
  \item \textbf{$\delta$-Ledger is Tuned:} The thresholds for $\delta^{\mathrm{disc}}$ and $\delta^{\mathrm{meas}}$ are loose enough to absorb numerical noise, but tight enough to not mask potential (hypothetical) $O(1)$ violations.
  \item \textbf{UCC is Robust:} The collapse operator $\mathbf{T}_\tau$ correctly identifies the stable Betti numbers despite noise and discretization.
\end{enumerate}
\textbf{Verdict:} The AK--HDPST engine is calibrated. We are authorized to proceed to Part~III (Fermat's Last Theorem), where we expect to encounter a strictly different topology (Global Obstruction).
\end{declaration}



% ------------------------------------------------------------
% Part III : AK Program for Fermat's Last Theorem (FLT)
% ------------------------------------------------------------
\part{AK Program for Fermat's Last Theorem (Global Obstruction)}
\label{part:flt}

\section{Chapter 9: FLT, Elliptic Curves, and Modularity -- AK Setup}
\label{ch:flt-setup}

This chapter initiates the final major calibration target: the framework established by Wiles and Taylor for proving Fermat's Last Theorem (FLT).
Our objective is not to reprove FLT, but to translate the classical proof structure---especially the mechanism that rules out counterexamples---into the language of the AK Core.
This calibration establishes the AK system's ability to diagnose a \emph{Global Obstruction}, where a seemingly valid region is shown to be structurally empty.

% ------------------------------------------------------------
\subsection{9.1. The Classical FLT Statement and the Frey Connection}
\label{sec:flt-classical}

\begin{theorem}[Fermat's Last Theorem]\label{thm:FLT-classical}
There are no positive integers \(A, B, C\) satisfying the equation
\[
  A^n + B^n = C^n
\]
for any integer exponent \(n \ge 3\).
\end{theorem}

\begin{remark}[The Wiles Strategy]\label{rmk:wiles-strategy}
The proof, completed by Wiles and Taylor, relies on linking a hypothetical solution \((A, B, C)\) to a structural impossibility: the existence of a \emph{Frey elliptic curve} \(E_{A,B,C}\) that is simultaneously forced to be modular and non-modular.

At a high level, three classical components are combined:
\begin{enumerate}
    \item \textbf{Frey/Ribet:}
    From a would-be solution \((A,B,C,n)\) one constructs a Frey curve \(E_{A,B,C}\) whose associated Galois representation has properties incompatible with modularity at the target level (Ribet's theorem).
    \item \textbf{Modularity:}
    Every elliptic curve over \(\mathbb{Q}\) is modular (Taniyama--Shimura--Weil / Wiles--Taylor).
    \item \textbf{Contradiction:}
    The same object \(E_{A,B,C}\) is forced, by construction, to be non-modular and, by the global theorem, to be modular.
\end{enumerate}
The AK program rephrases this ``no-place-to-live'' phenomenon as a global obstruction in the parameter space.
\end{remark}

% ------------------------------------------------------------
\subsection{9.2. The FLT Parameter Space \texorpdfstring{$M_{\mathrm{FLT}}$}{M\_FLT}}
\label{sec:flt-param-space}

We now recall and refine the FLT parameter space introduced in Chapter~\ref{sec:M-flt}.
There, \(M_{\mathrm{FLT}}\) was defined in terms of semistable elliptic curves over \(\mathbb{Q}\) with restricted ramification and specific mod-\(p\) representations.
For the AK program, it is convenient to work with an augmented version that remembers the exponent.

\begin{definition}[FLT Parameter Space \(M_{\mathrm{FLT}}\)]
We let \(M_{\mathrm{FLT}}\) denote the space of parameters \(\theta = (E, n)\), where:
\begin{itemize}
    \item \(n \ge 3\) is an integer exponent;
    \item \(E\) is an elliptic curve over \(\mathbb{Q}\), equipped with its structural invariants (minimal discriminant, conductor \(N\), and the associated \(\ell\)-adic representations).
\end{itemize}
Abusing notation slightly, we view \(E\) as ranging over the moduli described in Chapter~\ref{sec:M-flt}, and we simply attach the tag \(n\) as an additional discrete coordinate.
\end{definition}

\begin{definition}[Frey Locus \(L_{\mathrm{Frey}}\)]
The \emph{Frey locus} \(L_{\mathrm{Frey}} \subset M_{\mathrm{FLT}}\) is the subset of parameters
\[
  \theta = (E_{A,B,C}, n)
\]
that arise from hypothetical solutions \((A,B,C,n)\) of the Fermat equation via the Frey construction.
\end{definition}

\begin{remark}[AK Perspective on \(L_{\mathrm{Frey}}\)]
The set \(L_{\mathrm{Frey}}\) represents the \emph{singular target region} that the Hunter must ultimately certify as empty.
A point \(\theta \in L_{\mathrm{Frey}}\) is conjecturally characterized by a deep structural defect: the associated Galois representation \(\rho_{E,n}\) would have to exhibit a specific Type~IV obstruction (failure of modularity lifting) that prevents its persistence data from collapsing in the way mandated by the Unified Collapse Contract.
\end{remark}

% ------------------------------------------------------------
\subsection{9.3. AK Realization: Galois Representation to Persistence}
\label{sec:flt-ak-realization}

We now describe the realization functor
\[
  \mathfrak{P}_{\mathrm{FLT}} : M_{\mathrm{FLT}} \longrightarrow \mathsf{FiltCh}(k)
\]
which maps arithmetic data to the persistent homology framework.

\begin{specification}[Galois Representation Realization \(\mathfrak{P}_{\mathrm{FLT}}\)]
Let \(\theta = (E, n) \in M_{\mathrm{FLT}}\).
The AK realization is based on the \(\ell\)-adic Galois representations attached to \(E\):
\[
  \rho_{E,\ell} : \mathrm{Gal}(\overline{\mathbb{Q}}/\mathbb{Q}) \longrightarrow \mathrm{GL}_2(\mathbb{Z}_\ell).
\]
The construction proceeds schematically as follows:
\begin{enumerate}
    \item \textbf{Input Complex:}
    Start from the \(\ell\)-adic Tate module \(T_\ell(E)\), together with the Galois action and the local decomposition of inertia at primes dividing the conductor \(N\).
    This data is assembled into a cochain complex whose cohomology computes the relevant Selmer groups.
    \item \textbf{Persistence Module \(\mathbf{P}(\theta)\):}
    Consider the cyclotomic tower \(K_n \subset K_\infty\) used in Iwasawa theory.
    The \(\ell\)-adic Selmer group \(\mathrm{Sel}_\ell(E/K_\infty)\) is obtained as an inverse limit of finite-level cohomology groups.
    Passing to the Pontryagin duals yields compact \(\Lambda\)-modules for which the inverse limit becomes a direct limit.
    The \emph{Iwasawa Alignment} (Appendix~D/R) then converts this dualized direct-limit structure into a directed system whose image under \(\mathfrak{P}_{\mathrm{FLT}}\) lies in \(\Perscons\).
    \item \textbf{Bridge Target:}
    On this persistent object, the key diagnostic is the \emph{extension obstruction index} \(\mu_{\mathrm{Ext}}\), extracted from an \(\Ext^1\)-group that controls the deformation theory of \(\rho_{E,\ell}\) (cf.\ deformation ring \(R\) and Hecke algebra \(\mathbb{T}\)).
\end{enumerate}
The detailed cochain-level implementation is deferred to Appendix~\ref{appendix:ak-flt-realization}.
\end{specification}

\begin{remark}[Iwasawa Alignment and Tower Diagnostics]
The tower structure associated with \(\theta\) is the cyclotomic \(\mathbb{Z}_p\)-extension \(K_n \to K_\infty\) considered in Chapter~\ref{ch:arith-tower}.
The persistence module \(\mathbf{P}(\theta)\) records the size and growth behavior of \(\mathrm{Sel}_\ell(E/K_\infty)\), classically analyzed via the Iwasawa \(\mu\)- and \(\lambda\)-invariants.
In the AK framework, the same information is probed by the tower diagnostics \((\mu_{\mathrm{Collapse}}, \nu_{\mathrm{Collapse}})\) applied to the comparison map \(\phi_{i,\tau}\) of the Selmer complexes, checking the validity of the Control Theorems after collapse.
\end{remark}

% ------------------------------------------------------------
\subsection{9.4. The Ext/PH Bridge and Modularity}
\label{sec:flt-bridge}

The core of the Wiles--Taylor argument is the alignment
\[
  R \ \cong \ \mathbb{T},
\]
between a deformation ring \(R\) (Galois side) and a Hecke algebra \(\mathbb{T}\) (modular side).
AK--HDPST encodes this alignment as a bridge between topological persistence and extension groups.

\begin{definition}[Extension Obstruction \(\mu_{\mathrm{Ext}}\)]
Let \(C_\theta\) be a deformation complex associated with \(\rho_{E,\ell}\), and let
\[
  E^1_\theta \ := \ \Ext^1\bigl( \mathcal{R}(C_\theta), k \bigr)
\]
be the corresponding first extension group in the realization category.
We define the \emph{extension obstruction index} \(\mu_{\mathrm{Ext}}(\theta)\) as a numerical invariant extracted from \(E^1_\theta\).
For the Core v17.0 implementation, one may take, for example,
\[
  \mu_{\mathrm{Ext}}(\theta)
  \ := \
  \dim_k E^1_\theta
\]
whenever \(E^1_\theta\) is finite-dimensional, or more generally the length of \(E^1_\theta\) as a module over the relevant Iwasawa algebra.
Refinements of this choice, and the precise normalization used in the \(\delta\)-ledger, are specified in Appendix~\ref{appendix:ak-flt-realization}.
\end{definition}

\begin{declaration}[Bridge Condition for Modularity]\label{dec:bridge-modularity}
In the AK design, we \emph{encode} modularity of \(E\) by the bridge condition
\[
  \mu_{\mathrm{Ext}}(\theta) \ \approx \ \mu_{\mathrm{Collapse}}(\theta) \ \approx \ 0,
\]
evaluated in a suitable error budget.
Conceptually:
\begin{itemize}
    \item \(\mu_{\mathrm{Collapse}}(\theta) \approx 0\) expresses that the tower diagnostics for the Selmer complexes exhibit no Type~IV failure (good control in the Iwasawa sense).
    \item \(\mu_{\mathrm{Ext}}(\theta) \approx 0\) expresses that the deformation problem has no essential extension obstruction: the map \(R \to \mathbb{T}\) behaves as an isomorphism within the prescribed tolerances.
\end{itemize}
This bridge condition is part of the \textbf{[Spec]} layer: it formalizes, inside AK--HDPST, how the classical \(R \cong \mathbb{T}\) alignment is reflected in the persistence/Ext diagnostics; it does not assert any new theorem beyond the established modularity results.
\end{declaration}

\begin{remark}[FLT as a Global Obstruction Scenario]
If a solution \((A,B,C,n)\) to Fermat's equation existed, the associated Frey curve \(E_{A,B,C}\) would yield a parameter \(\theta \in L_{\mathrm{Frey}}\) for which the classical strategy predicts an incompatibility between the Galois and modular descriptions.
In the AK picture, this incompatibility is interpreted as the \emph{requirement} that some diagnostic (such as \(\mu_{\mathrm{Collapse}}\) or \(\mu_{\mathrm{Ext}}\)) must register a robust failure.
The subsequent chapters analyze this scenario abstractly: under the assumption of the classical modularity theorems, the AK verdict forces the Frey locus to be structurally empty.
\end{remark}



\section{Chapter 10: Collapse Diagnostics and Would-Be Solutions of FLT}
\label{ch:flt-diagnostics}

In this chapter, we define the diagnostic protocol for checking hypothetical solutions to the Fermat equation.
The AK--HDPST engine operates in \emph{Counterexample Search Mode} (Hunter).
We describe how a hypothetical Frey curve $E_{A,B,C}$ would manifest as a specific topological anomaly—a \textbf{Type~IV Essential Singularity}—and how the Wiles--Taylor theorem, imported as a Core-Input, certifies that such anomalies cannot exist in the arithmetic landscape.

% ------------------------------------------------------------
\subsection{10.1. Anatomy of a Would-Be Solution (The Frey Obstruction)}
\label{sec:flt-would-be}

Suppose, for the sake of the search algorithm, that a parameter $\theta = (E_{A,B,C}, n) \in L_{\mathrm{Frey}}$ exists.
How does the AK auditor perceive this object?

\begin{definition}[The Frey Profile]
A \emph{Would-Be Solution} is a persistence object $P_\theta = \mathfrak{P}_{\mathrm{FLT}}(\theta)$ satisfying:
\begin{enumerate}
    \item \textbf{Local Validity (Type I/II Pass):}
    It is semistable (square-free conductor $N$) and has good reduction properties locally.
    The local M-Gate$^+$ checks pass: $\Phi_{\mathrm{local}} \approx 0$.
    \item \textbf{Global Discrepancy (Type IV Fail):}
    It exhibits a \emph{Modularity Gap}.
    The Galois representation $\rho_\theta$ constructed from $P_\theta$ cannot be matched to any Hecke eigenform $f$ of weight~$2$ and the level predicted by Ribet's level-lowering (in particular, level~$2$ in the classical FLT scenario).
\end{enumerate}
\end{definition}

\begin{remark}[Type~IV Diagnosis via $\mu_{\mathrm{Ext}}$]
In the language of Chapter~\ref{ch:arith-tower}, this discrepancy manifests as a divergence in a bridge index.
Ribet's theorem implies that if $E_{A,B,C}$ exists, its associated deformation ring $R$ is ``too large'' compared to the relevant Hecke algebra $\mathbb{T}$ at level~$2$ (whose cusp-form sector is trivial).
At the \textbf{[Spec]} layer, we summarize this by a scalar index
\[
  \mu_{\mathrm{Ext}}(\theta)
  \ := \
  \dim_k \ker\!\bigl(R \longrightarrow \mathbb{T}_{\mathrm{level}=2}\bigr)
  \ \gg \ 0,
\]
where $k$ is a fixed residue field.
Conceptually, this represents an infinite tower of deformations that finds no modular ground to land on—a model Type~IV singularity in the FLT setting.
Precise implementation details of $\mu_{\mathrm{Ext}}$ are deferred to Appendix~D.
\end{remark}

% ------------------------------------------------------------
\subsection{10.2. The UCC Protocol on the Frey Locus}
\label{sec:flt-ucc-check}

We formalize the check performed by the AK Core when the Hunter proposes a candidate $\theta$.

\begin{specification}[The Modularity Gate Check]
For a candidate $\theta \in M_{\mathrm{FLT}}$, the system computes the \emph{Modularity Potential} $\Psi(\theta)$:
\[
  \Psi(\theta) \ := \
  \inf_{f \in \mathcal{S}_2(\Gamma_0(N))}
  \text{dist}_{\mathrm{Galois}}\bigl(\rho_\theta, \rho_f\bigr),
\]
where the distance is measured in the quantized $\delta$-ledger metric (matching traces of Frobenius at a controlled set of primes).

\textbf{The Ribet Constraint.}
If $\theta \in L_{\mathrm{Frey}}$, classical theory (Ribet) forces $\Psi(\theta)$ to be evaluated against forms of level~$2$.
Since $\mathcal{S}_2(\Gamma_0(2)) = 0$, there are no such forms, and we interpret this as
\[
  \Psi(\theta) \ \longrightarrow \ \infty
  \qquad\text{(structural mismatch).}
\]
Thus, any candidate in the Frey Locus triggers a massive violation of the Unified Collapse Contract (UCC) once modularity is enforced.
\end{specification}

% ------------------------------------------------------------
\subsection{10.3. Wiles--Taylor as External Core-Input}
\label{sec:flt-core-input}

The AK system does not prove Wiles' theorem from scratch.
It accepts it as a boundary condition for the validity map.

\begin{declaration}[External Input: The Modularity Theorem]
The AK Core accepts the following statement as a Core-Input for the FLT calibration:
\begin{quote}
    For every semistable elliptic curve $E/\mathbb{Q}$, the Modularity Potential $\Psi(E)$ is zero (i.e., $E$ is modular).
\end{quote}
Formally, this implies that the bridge isomorphism $R \cong \mathbb{T}$ holds for all valid parameters in the sense of the Modularity Program.
\end{declaration}

\begin{theorem}[Collapse of the Frey Locus]
Under the External Input (Wiles--Taylor) and the Ribet Constraint:
\begin{enumerate}
    \item Any valid geometric object $E$ must satisfy $\Psi(E) = 0$.
    \item Any object in the Frey Locus $L_{\mathrm{Frey}}$ must satisfy $\Psi(E) = \infty$.
\end{enumerate}
Consequently, the intersection of the set of valid geometric objects and the Frey Locus is empty:
\[
  Z_{\mathrm{Valid}} \cap L_{\mathrm{Frey}} \ = \ \emptyset.
\]
\end{theorem}

% ------------------------------------------------------------
\subsection{10.4. The AK Verdict: Structural Emptiness}
\label{sec:flt-verdict}

We summarize the system's output for the FLT program.

\begin{definition}[AK Verdict for FLT]
The validity map $\mathcal{V}_\tau$ on $M_{\mathrm{FLT}}$ assigns the status $\mathsf{Obstructed}$ to any parameter that fits the profile of a solution in the Frey Locus:
\[
  \theta \in L_{\mathrm{Frey}}
  \quad\Longrightarrow\quad
  \AKverdict(\theta) \ = \ \mathsf{Obstructed}.
\]
\end{definition}

\begin{corollary}[No Valid Cells on the Frey Locus]
When the Mapper agent attempts to build a Global Certificate for FLT counterexamples:
\begin{itemize}
    \item It queries the Frey Locus $L_{\mathrm{Frey}}$.
    \item For every hypothetical point $\theta$, the M-Gate$^+$ (Modularity Check) returns \texttt{FAIL: Type~IV Obstruction}.
    \item The set of valid cells covering $L_{\mathrm{Frey}}$ is the empty set.
\end{itemize}
Therefore, the AK Verdict at the global level is summarized as
\[
  \AKverdict(\mathrm{FLT}) \ = \ \mathsf{No\_Solution\_Exists},
\]
where this verdict is understood as a \emph{structural restatement} of the classical Wiles--Taylor theorem, not a new proof.
\end{corollary}

\begin{remark}[Comparison with Navier--Stokes]
In the FLT case, the ``emptiness'' of the singular region is enforced by a theorem (Wiles--Taylor).
In the Navier--Stokes case (Appendix~NS), the emptiness of the singular region (blow-up locus) is \emph{conjectural}.
The AK--HDPST engine uses the same diagnostic machinery ($\mu, \nu$, Type~IV) for both:
\begin{itemize}
    \item FLT: We \emph{know} Type~IV is impossible for valid curves $\Rightarrow$ no solutions.
    \item NSE: We \emph{test} whether Type~IV is realized for fluid flows $\Rightarrow$ regularity versus blow-up.
\end{itemize}
This highlights the utility of the FLT program as a calibration of the Type~IV detector.
\end{remark}



\section{Chapter 11: Validity Map on \texorpdfstring{$M_{\mathrm{FLT}}$}{M\_FLT}}
\label{ch:flt-validity-map}

This chapter concludes Part~III by assembling the global Validity Map $\mathcal{V}_\tau$ for the FLT parameter space.
Unlike the Weil calibration (Part~II), where the map was a uniform ``Plain of Truth'', the FLT map is defined by a sharp \textbf{structural dichotomy}.

We visualize the clash between the \emph{Frey Locus} (where solutions must live) and the \emph{Modular World} (where valid curves must live).
The AK--HDPST engine confirms that, under the laws of the Unified Collapse Contract and the Wiles--Taylor Core-Input, these two territories are disjoint.
The resulting map certifies the non-existence of solutions as a topological necessity of the calibration, rather than as a new proof.

% ------------------------------------------------------------
\subsection{11.1. Topography of the Exclusion}
\label{sec:flt-map-def}

We define the regions of the Validity Map based on the diagnostics of Chapter~\ref{ch:flt-diagnostics}.

\begin{definition}[The FLT Validity Stratification]
The map
\[
  \mathcal{V}_\tau : M_{\mathrm{FLT}} \longrightarrow
  \{\mathsf{Valid}, \mathsf{Obstructed}\}
\]
partitions the space into:

\begin{enumerate}
    \item \textbf{The Modular Plain ($Z_{\mathrm{Valid}}$):}
    The set of parameters $\theta = (E, n)$ where the Modularity Potential vanishes (within the $\delta$-ledger budget),
\[
  Z_{\mathrm{Valid}}
  \ := \
  \bigl\{ \theta \in M_{\mathrm{FLT}}
    \mid \text{\emph{M-Gate}}^{+}(\theta) = \mathsf{PASS} \bigr\}.
\]

    By the Wiles--Taylor Core-Input, this contains all semistable elliptic curves over $\mathbb{Q}$.

    \item \textbf{The Singular Target ($Z_{\mathrm{Target}} = L_{\mathrm{Frey}}$):}
    The set of parameters derived from hypothetical Fermat solutions.
    By the Ribet Constraint, every point here has $\Psi(\theta) = \infty$, and thus carries an active Type~IV obstruction:
    \[
      L_{\mathrm{Frey}}
      \ \subset \
      \{\theta \in M_{\mathrm{FLT}}
        \mid \text{Type~IV Failure is active at }\theta\}.
    \]
\end{enumerate}
\end{definition}

\begin{theorem}[The Disjointness Certificate]
Under the Wiles--Taylor Modularity Input and the Ribet Constraint, the central output of the FLT program is the emptiness of the intersection
\[
  Z_{\mathrm{Valid}} \cap L_{\mathrm{Frey}} \ = \ \emptyset.
\]
In the language of the Validity Map: the terrain in which a counterexample could theoretically exist is structurally designated as \textbf{Obstructed Territory}.
\end{theorem}

\begin{remark}
This theorem is a structural restatement of the classical Wiles--Taylor argument in the language of AK--HDPST.
It does not constitute a new proof of FLT, but rather a calibration statement: assuming Modularity and Ribet, the AK engine reconstructs the same exclusion picture at the level of its collapse diagnostics.
\end{remark}

% ------------------------------------------------------------
\subsection{11.2. Hunter Execution: The Null Search}
\label{sec:flt-hunter-exec}

We now describe the behaviour of the Hunter agent when it attempts to locate a valid point within the Frey Locus.

\paragraph{Simulation (Hunter Trace: Project FLT-Counterexample).}

\begin{verbatim}
[INIT]  Target: Fermat's Last Theorem (Counterexample Search)
[CONF]  Constraint: A^n + B^n = C^n (n >= 3)
[START] Spawning Hunter-X...

[ATTEMPT 1] Proposing solution (A,B,C) = (small integers)...
         > Check: Arithmetic? FAIL (A^n + B^n != C^n).
         > Action: REJECT (Pre-filter).

[ATTEMPT k] Proposing hypothetical parameter theta_k in L_Frey...
         > (Assume arithmetic equality holds for simulation)
         > Constructing Frey curve E_k...
         > Computing Modularity Potential Psi(E_k)...
         > RIBET_ALARM: Level lowering requires weight-2 form at level 2.
         > DATABASE_LOOKUP: dim S_2(Gamma_0(2)) = 0.
         > Result: Psi(E_k) -> INFINITY.
         > M-Gate Status: BLOCKED (Type IV Obstruction).
         > Action: REJECT.

[... Iterating over definable families of (A,B,C) ...]

[END]   Hunter-X terminated.
        Total Candidates Proposed: N (symbolic sweep).
        Total Validated Frey Curves: 0.
        Verdict: Search space is structurally empty.
\end{verbatim}

\begin{remark}
The Hunter does not fail because it cannot find the ``right'' numbers.
It fails because the \emph{gate} to the valid region is permanently locked against objects with the Frey profile.
In AK terms, the Frey Locus is a region where Type~IV obstruction is hard-wired by the Modularity constraints.
\end{remark}

% ------------------------------------------------------------
\subsection{11.3. AK--Perspective: From Calibration to Exploration}
\label{sec:flt-perspective}

This concludes the Arithmetic Calibration of AK--HDPST v17.0.
We have demonstrated that the core machinery---persistence, collapse, $\delta$-ledgers, and gate diagnostics---can be aligned with the logical structure of two monumental achievements in mathematics:

\begin{itemize}
    \item \textbf{Weil Program (Part~II):} Validated the ability to recognize \emph{Global Regularity} (the Plain of Truth on $M_{\mathrm{Weil}}$).
    \item \textbf{FLT Program (Part~III):} Validated the ability to recognize \emph{Global Obstruction} (the structurally empty Frey Locus on $M_{\mathrm{FLT}}$).
\end{itemize}

From the AK perspective, these two calibrations anchor the extremes of the Validity Map:
one where every point is structurally valid, and one where the would-be singular region is structurally empty.

% ------------------------------------------------------------
\subsection{11.4. Grand Finale: The Bridge to Navier--Stokes}
\label{sec:flt-nse-link}

We now turn our gaze from the solved past to the open future.
The apparatus that has been calibrated on the rigid crystalline structures of Number Theory is about to be deployed into the turbulent fluid of the Navier--Stokes Equations (Appendix~NS).

\begin{declaration}[The Transition to Speculative Science]
In the upcoming Appendix~NS (Navier--Stokes Case Study), the epistemic status of the AK--HDPST engine shifts to the \textbf{[Spec]} layer:

\begin{enumerate}
    \item \textbf{No External Truth:}
    We no longer have a Deligne or Wiles theorem to guide us.
    The map is \emph{terra incognita}.

    \item \textbf{The Type~IV Question:}
    The central question becomes:
    \begin{quote}
        Does the 3D Euler/Navier--Stokes flow admit a Type~IV singularity (blow-up)
        analogous to the Frey obstruction, or is it protected by a Global Regularity
        analogous to the Weil Conjectures?
    \end{quote}

    \item \textbf{The Energy--Collapse Hypothesis:}
    To address this, the Modularity Potential is replaced by a Defect Potential
    $\Phi_{\mathrm{NS}}$ derived from an Energy--Collapse inequality on suitable
    persistence realizations of the flow.
\end{enumerate}

With the logic gates verified and the failure modes calibrated in arithmetic, the AK--HDPST v17.0 platform is now ready, at the \textbf{[Spec]} layer, to audit the Navier--Stokes Millennium Problem.
\end{declaration}



\section{Appendix A: Classical Weil/Deligne Background (Core-Input)}
\label{app:weil-core}

This appendix specifies the \textbf{Core-Input} from classical arithmetic geometry used to calibrate the AK system in Part~II.
Within the AK--HDPST framework, these statements are treated not as conjectures to be proven, but as \emph{boundary conditions} that define the “Plain of Truth.”
Any deviation from these laws observed by the Hunter agent is interpreted strictly as a bug in the implementation (e.g.\ numerical precision failure), not as a mathematical counterexample.

% ------------------------------------------------------------
\subsection{A.1. The Zeta Function of a Variety}
\label{app:weil-zeta}

Let $X$ be a smooth projective variety of dimension $d$ defined over a finite field $\mathbb{F}_q$ with $q$ elements.
Let $N_n = \# X(\mathbb{F}_{q^n})$ be the number of rational points over the degree-$n$ extension.

\begin{definition}[Weil Zeta Function]
The zeta function $Z(X, t)$ is defined as the formal power series
\[
  Z(X, t)
  \ := \
  \exp\!\left( \sum_{n=1}^{\infty} N_n \frac{t^n}{n} \right)
  \ \in \ \mathbb{Q}[[t]].
\]
\end{definition}

This function encodes the Diophantine information of $X$ into a generating function.
The Weil conjectures (1949) predicted its global structure.

% ------------------------------------------------------------
\subsection{A.2. Rationality and Cohomological Interpretation}
\label{app:weil-rationality}

The first major input is the rationality of the zeta function, proven by Dwork using $p$-adic analysis and interpreted cohomologically by Grothendieck and his school.

\begin{theorem}[Rationality (Core-Input I)]\label{thm:weil-rationality}
The function $Z(X, t)$ is a rational function of $t$.
More precisely, there exist polynomials $P_i(t) \in \mathbb{Z}[t]$ such that
\[
  Z(X, t)
  \ = \
  \prod_{i=0}^{2d} P_i(t)^{(-1)^{i+1}},
  \qquad
  P_i(t)
  \ = \
  \det\!\bigl(\mathrm{Id} - t \cdot \mathrm{Fr}_q^* \mid H^i_{\text{\emph{\'{e}t}}}(X_{\overline{\mathbb{F}}_q}, \mathbb{Q}_\ell)\bigr).
\]
\end{theorem}

\begin{interpretation}[AK–Mapping: Type~IV Stability (Heuristic)]
In the AK framework, rationality is read as a \emph{finiteness constraint} on the associated persistence module.
Informally:
\begin{itemize}
  \item If $Z(X,t)$ were to exhibit a natural boundary or essential singularity, the induced persistence module would display an unbounded cascade of new bars at finer and finer scales, corresponding to a \textbf{Type~IV (Essential Singularity)} failure.
  \item The classical rationality theorem therefore functions, for AK, as a guarantee that the tower diagnostics $(\mu_{\mathrm{Collapse}}, \nu_{\mathrm{Collapse}})$ vanish for any smooth projective $X$:
  \[
    \mu_{\mathrm{Collapse}}(X) = \nu_{\mathrm{Collapse}}(X) = 0.
  \]
\end{itemize}
This is an \emph{interpretive bridge} rather than a new theorem: AK--HDPST is calibrated so that rationality corresponds to the absence of Type~IV failures.
\end{interpretation}

% ------------------------------------------------------------
\subsection{A.3. The Riemann Hypothesis for Finite Fields (Deligne's Theorem)}
\label{app:weil-rh}

The deepest input, proven by Deligne, concerns the weights of the roots of the polynomials $P_i(t)$.

\begin{theorem}[Riemann Hypothesis / Purity of Weights (Core-Input II)]\label{thm:weil-deligne}
Factor the polynomial $P_i(t)$ over $\mathbb{C}$ as
\[
  P_i(t)
  \ = \
  \prod_{j=1}^{b_i} (1 - \alpha_{ij} t).
\]
Then for every $i \in \{0, \dots, 2d\}$ and every $j \in \{1, \dots, b_i\}$, the inverse root $\alpha_{ij}$ is an algebraic integer satisfying
\[
  |\alpha_{ij}| \ = \ q^{i/2}
\]
for any complex embedding $\iota : \overline{\mathbb{Q}} \hookrightarrow \mathbb{C}$.
\end{theorem}

\begin{interpretation}[AK–Mapping: Type~I Stability]
This theorem specifies the \textbf{target spectrum} for the Hunter.
\begin{itemize}
  \item The “Plain of Truth” for degree $i$ is defined by the condition that all spectral features reconstructed from persistence behave as if their eigenvalues satisfy $|\alpha| = q^{i/2}$.
  \item Any persistent feature whose effective scaling deviates from $q^{i/2}$ is registered as a \textbf{Type~I (Topological) Failure}.
  \item Deligne's theorem guarantees that, in the classical setting, the set of such Type~I failures is empty.
\end{itemize}
Thus, the AK spectral defect potential $\Phi^{\mathrm{Weil}}_\tau$ is \emph{mathematically} zero on all of $M_{\mathrm{Weil}}$.
\end{interpretation}

% ------------------------------------------------------------
\subsection{A.4. The Trace Formula and Betti Numbers}
\label{app:weil-trace}

To connect point-count data with spectral information, the AK system relies on the Lefschetz trace formula.

\begin{theorem}[Grothendieck Trace Formula]\label{thm:trace-formula}
The point counts $N_n$ are related to the eigenvalues of the Frobenius endomorphism $\mathrm{Fr}_q$ acting on the $\ell$-adic cohomology groups $H^i_{\text{\emph{\'{e}t}}}(X_{\overline{\mathbb{F}}_q}, \mathbb{Q}_\ell)$ by
\[
  \# X(\mathbb{F}_{q^n})
  \ = \
  \sum_{i=0}^{2d} (-1)^i \,
  \mathrm{Tr}\bigl(\mathrm{Fr}_q^n \mid H^i_{\text{\emph{\'{e}t}}}(X_{\overline{\mathbb{F}}_q}, \mathbb{Q}_\ell)\bigr)
  \ = \
  \sum_{i=0}^{2d} (-1)^i \sum_{j=1}^{b_i} \alpha_{ij}^n.
\]
\end{theorem}

\begin{specification}[AK–Input Data]
The Hunter agent does not “see’’ the cohomology groups directly.
Operationally it observes a finite prefix of the integer sequence
\[
  (N_1, N_2, \dots, N_K),
\]
from which it performs \emph{spectral recovery} (for example via Hankel-type constructions or lattice reduction) to approximate the eigenvalues $\{\alpha_{ij}\}$ and, hence, the polynomials $P_i(t)$.
These recovered spectra are then fed into the spectral indicators and the defect potential $\Phi^{\mathrm{Weil}}_\tau$ defined in Part~II.
\end{specification}

% ------------------------------------------------------------
\subsection{A.5. Summary of Core Constraints}
\label{app:weil-constraints}

For the purpose of AK--HDPST v17.0, Part~II, we package the ``classical truth'' into a set of immutable constraints recorded in the $\delta$-ledger.

\begin{declaration}[Axioms of the Weil Calibration]
For any parameter $\theta \in M_{\mathrm{Weil}}$ corresponding to a smooth projective variety:
\begin{enumerate}
  \item \textbf{No Singularity:}
    The associated persistence tower is finite-rank and exhibits no Type~IV failures:
    \[
      \mu_{\mathrm{Collapse}}(\theta) = 0, \qquad
      \nu_{\mathrm{Collapse}}(\theta) = 0.
    \]
  \item \textbf{No Drift:}
    The spectral defect potential is mathematically zero, and any non-zero value observed in computation must be accounted for by the $\delta$-ledger:
    \[
      \Phi^{\mathrm{Weil}}_\tau(\theta)
      \ \le \
      \delta^{\mathrm{disc}}(\theta) \oplus \delta^{\mathrm{meas}}(\theta).
    \]
  \item \textbf{Hard Integrality:}
    The characteristic polynomials $P_i(t)$ have coefficients in $\mathbb{Z}$; no $p$-adic or floating drift is allowed in the Core-Input.
\end{enumerate}
Under these axioms, the Mapper is justified in coloring the entire parameter space $M_{\mathrm{Weil}}$ as \textsf{Valid} (green) \emph{a priori}; Part~II then serves to verify that the implemented AK pipeline indeed reproduces this trivial Validity Map.
\end{declaration}



\section{Appendix B: Classical FLT/Modularity Background (Core-Input)}
\label{app:flt-core}

This appendix specifies the arithmetic machinery underlying the FLT Calibration (Part~III).
In the AK--HDPST framework, the proof of Fermat's Last Theorem is not reconstructed from scratch; rather, the logical structure of the Wiles--Taylor and Ribet theorems is imported as a set of \textbf{Core-Input Constraints} that define the topology of the Validity Map.

The interaction between these inputs creates a “hard-wired obstruction’’ that forces the Hunter to reject any candidate parameter in the Frey Locus.

% ------------------------------------------------------------
\subsection{B.1. The Frey Curve Construction}
\label{app:flt-frey}

Let $p \ge 3$ be a prime (it suffices to prove FLT for prime exponents).
Suppose there exists a non-trivial solution to the Fermat equation:
\[
  A^p + B^p = C^p,
  \qquad
  A, B, C \in \mathbb{Z} \setminus \{0\}, \quad \gcd(A,B,C)=1.
\]
Equivalently, one may write $A^p + B^p + C^p = 0$ with $C^p := - (A^p + B^p)$.
After standard normalizations, we may assume
\[
  A \equiv -1 \pmod{4},
  \qquad
  B \equiv 0 \pmod{2}.
\]

\begin{definition}[Frey Curve]
The \emph{Frey elliptic curve} associated to the solution $(A,B,C)$ is defined by the Weierstrass equation
\[
  E_{A,B,C}
  : \quad
  y^2 = x(x-A^p)(x+B^p).
\]
\end{definition}

\begin{specification}[Arithmetic Profile (Core-Input)]
Classical calculations establish the following invariants for $E_{A,B,C}$, up to powers of $p$ that do not affect the conductor:
\begin{enumerate}
  \item \textbf{Minimal Discriminant:}
    \(
      \Delta_{\min} \sim 2^{-8} (ABC)^{2p}.
    \)
  \item \textbf{Conductor:}
    \(
      N_{E} = 2 \prod_{q \mid ABC} q,
    \)
    in particular $N_E$ is square-free.
  \item \textbf{Semistability:}
    Since $N_E$ is square-free, $E_{A,B,C}$ is semistable.
  \item \textbf{Torsion / Flatness at $p$:}
    The $p$-torsion $E[p]$ arises from a finite flat group scheme over $\mathbb{Z}_p$, giving the “minimal at $p$’’ ramification profile used in Ribet’s theorem.
\end{enumerate}
This profile defines the \textbf{Frey Locus} $L_{\mathrm{Frey}}$ within the AK parameter space $M_{\mathrm{FLT}}$.
\end{specification}

% ------------------------------------------------------------
\subsection{B.2. Ribet's Theorem: The Structural Trap}
\label{app:flt-ribet}

The mechanism that prevents the Frey curve from existing inside the modular world is Ribet's level-lowering theorem.

\begin{theorem}[Ribet's Theorem / Epsilon Conjecture (Core-Input III)]\label{thm:ribet}
Let
\[
  \rho_{E,p} :
  \mathrm{Gal}(\overline{\mathbb{Q}}/\mathbb{Q})
  \longrightarrow
  \mathrm{GL}_2(\mathbb{F}_p)
\]
be the mod-$p$ Galois representation associated to the Frey curve $E_{A,B,C}$.
If $E_{A,B,C}$ is modular of weight~$2$ and level~$N_E$, and if $\rho_{E,p}$ satisfies the standard local conditions (minimal at $p$), then $\rho_{E,p}$ must arise from a modular form $f$ of weight~$2$ and level
\[
  N_{\mathrm{reduced}} \ = \ 2.
\]
\end{theorem}

\begin{interpretation}[AK--Mapping: The Vacuum Target]
Ribet's theorem directs the Modularity Potential $\Psi(\theta)$ to verify compatibility against the space of cusp forms $S_2(\Gamma_0(2))$.
However, an explicit dimension computation (via the genus of the modular curve $X_0(2)$) shows that
\[
  \dim_{\mathbb{C}} S_2(\Gamma_0(2)) \ = \ 0.
\]
Thus the target space for the “level-lowered’’ representation is the \textbf{empty set}.
In AK terms, this creates an infinite barrier ($\Psi(\theta) \to \infty$) for any point in the Frey Locus that attempts to enter the Valid Region.
\end{interpretation}

% ------------------------------------------------------------
\subsection{B.3. The Modularity Theorem (Wiles et al.)}
\label{app:flt-wiles}

The closure of the trap is the theorem that semistable elliptic curves \emph{must} be modular.

\begin{theorem}[Modularity Lifting (Core-Input IV)]\label{thm:wiles-lifting}
Let $E$ be a semistable elliptic curve over $\mathbb{Q}$, and let
\(
  \rho :
  \mathrm{Gal}(\overline{\mathbb{Q}}/\mathbb{Q})
  \to \mathrm{GL}_2(\mathbb{Z}_\ell)
\)
be the associated $\ell$-adic Galois representation.
Then $E$ is modular.
More formally, there is a surjective ring homomorphism
\[
  R_{\rho} \twoheadrightarrow \mathbb{T}_{N_E},
\]
which is an isomorphism (complete intersection property), identifying the universal deformation ring of $\rho$ with the Hecke algebra at level $N_E$.
\end{theorem}

\begin{specification}[AK--Mapping: The Plain of Validity]
This theorem is entered into the AK Core as the definition of the \textsf{Valid} state for semistable parameters:
\[
  \theta \in M_{\mathrm{FLT}}^{\mathrm{semistable}}
  \ \implies \
  \text{M-Gate}^+(\theta) = \mathsf{PASS}
  \qquad
  (\text{provided } \theta \notin L_{\mathrm{Frey}}).
\]
\end{specification}

% ------------------------------------------------------------
\subsection{B.4. Summary of Core Constraints for FLT}
\label{app:flt-constraints}

We synthesize these inputs into the $\delta$-ledger rules for Part~III.

\begin{declaration}[Axioms of the FLT Calibration]
For any parameter $\theta = (E, n) \in M_{\mathrm{FLT}}$:
\begin{enumerate}
  \item \textbf{Global Modularity (Wiles Input):}
    If $E$ is semistable, it is modular.
    In AK notation, the extension obstruction index satisfies
    \[
      \mu_{\mathrm{Ext}}(\theta) \ \approx \ 0
      \qquad\text{for semistable } E.
    \]

  \item \textbf{Frey Incompatibility (Ribet + Level 2 Vacuum):}
    If $\theta \in L_{\mathrm{Frey}}$ (i.e.\ $E$ is a Frey curve derived from a putative solution), then level-lowering forces
    \[
      \text{Target}(L_{\mathrm{Frey}})
      \ \subset \
      S_2(\Gamma_0(2))
      \ = \ \{0\}.
    \]
    Thus the Modularity Potential is “infinitely large’’:
    \(
      \Psi(\theta) = \infty
    \)
    for any such $\theta$.

  \item \textbf{Empty Intersection:}
    Combining (1) and (2): a Frey curve is semistable, so it \emph{must} be modular (by Wiles), but it \emph{cannot} be modular (by Ribet plus the emptiness of level~2).
    Therefore the parameter $\theta$ cannot exist as a valid arithmetic object.
\end{enumerate}

The AK verdict
\[
  \AKverdict(\theta) = \mathsf{Obstructed}
  \qquad
  \text{for } \theta \in L_{\mathrm{Frey}}
\]
is the system’s operational expression of this logical contradiction inside the Validity Map.
\end{declaration}



\section{Appendix C: AK--Weil Realization Details}
\label{app:weil-realization-details}

This appendix provides the technical specifications for the realization functor
$\mathfrak{P}_{\mathrm{Weil}}$ and the associated spectral metrics used in Part~II.
It bridges the gap between the abstract cohomological definitions (Appendix~A)
and the numerical computations performed by the Hunter agent.

% ------------------------------------------------------------
\subsection{C.1. The Spectral Surrogate Construction}
\label{app:weil-surrogate}

While the conceptual realization involves a mapping torus of the Frobenius action,
the operational implementation uses a \textbf{Spectral Surrogate}.
This allows the AK Core to treat eigenvalues directly as persistence features.

\begin{definition}[Spectral Surrogate]
Let $\theta = (X, \mathbb{F}_q) \in M_{\mathrm{Weil}}$.
Suppose the spectral recovery algorithm (from point counts $N_n$) yields a set of
eigenvalues $\sigma = \{ \alpha_{ij} \}$ for the cohomology groups $H^i(X)$.
We construct a \emph{Surrogate Persistence Module} $\mathbb{M}(\theta)$ consisting of a
disjoint union of intervals (bars) $I_{ij}$:
\[
  \text{Barcode}(\mathbb{M}(\theta))
  \ := \
  \bigsqcup_{i=0}^{2d} \bigsqcup_{j=1}^{b_i} I_{ij},
\]
where the interval $I_{ij}$ is defined by
\[
  I_{ij} \ = \ [0, L_{ij}), \qquad
  L_{ij} \ := \
  \frac{1}{
    \varepsilon + \bigl|
      \log_q |\alpha_{ij}| - \tfrac{i}{2}
    \bigr|
  }.
\]
Here $\varepsilon > 0$ is a small regularization parameter.
\end{definition}

\begin{interpretation}[Persistence as Stability]
\begin{itemize}
  \item If $|\alpha_{ij}| = q^{i/2}$ (RH holds), the denominator becomes
    $\varepsilon$, and the bar length $L_{ij} \approx 1/\varepsilon$ becomes very
    large (high persistence).
  \item If $|\alpha_{ij}| \neq q^{i/2}$ (RH violation), the denominator grows and
    $L_{ij}$ becomes small (low persistence).
  \item \textbf{AK Logic:}
    The collapse operator $\mathbf{T}_\tau$ removes short bars. Thus,
    “surviving collapse’’ in this surrogate model is equivalent to “being close
    to the RH circle’’ in the sense of the spectral defect.
\end{itemize}
\end{interpretation}

% ------------------------------------------------------------
\subsection{C.2. The Spectral Quantale \texorpdfstring{$V_{\mathrm{spec}}$}{V\_spec}}
\label{app:weil-quantale-details}

We detail the internal structure of the quantale used to aggregate spectral defects.

\begin{definition}[Quantale Structure]
We work with
\[
  V_{\mathrm{spec}} \ = \ [0, \infty) \times [0, 2\pi],
\]
equipped with the order and aggregation used by the $\delta$-ledger.
For an element $\delta = (\delta_r, \delta_\theta) \in V_{\mathrm{spec}}$:
\begin{enumerate}
  \item \textbf{Norm:}
    \[
      \|\delta\|_V
      \ := \
      \sqrt{\delta_r^2 + \omega \cdot \delta_\theta^2},
    \]
    where $\omega \ge 0$ is a weight tuning the sensitivity to angular
    distribution (typically $\omega = 0$ for pure RH checks,
    $\omega = 1$ for Sato--Tate–type statistics).
  \item \textbf{Aggregation ($\oplus$):}
    $\delta_1 \oplus \delta_2$ is defined component-wise, or by $L^2$-summation
    of norms, depending on the \texttt{run.yaml} setting
    (conservative vs.\ average case). In either policy, $\oplus$ is
    commutative, associative, and monotone with respect to the underlying order.
\end{enumerate}
\end{definition}

\begin{definition}[Defect Potential Functional]
The global potential $\Phi^{\mathrm{Weil}}_\tau(\theta)$ used in Part~II can be
written explicitly as
\[
  \Phi^{\mathrm{Weil}}_\tau(\theta)
  \ := \
  \sum_{i,j}
  \max\!\left(
    0,\,
    \Bigl|
      \log_q |\alpha_{ij}| - \frac{i}{2}
    \Bigr|
    - \delta^{\mathrm{disc}}_i
  \right)^2.
\]
The subtraction of $\delta^{\mathrm{disc}}_i$ represents the \textbf{Safe Zone}:
radial deviations smaller than the discretization error are masked to zero and
do not contribute to the potential.
\end{definition}

% ------------------------------------------------------------
\subsection{C.3. Stability and Threshold Policies}
\label{app:weil-thresholds}

The calibration requires precise definitions of the “Safe Zone’’ boundaries.

\begin{specification}[Discretization Error Bounds ($\delta^{\mathrm{disc}}$)]
When recovering eigenvalues from $K$ point counts $(N_1, \dots, N_K)$, we use
the following heuristic policy:
\begin{enumerate}
  \item \textbf{Unknown Count:}
    The unknowns are (ultimately) the roots of the polynomials $P_i(t)$, whose
    total degree is
    \(
      B := \sum_i b_i.
    \)
  \item \textbf{Sampling Requirement (Nyquist-Type):}
    We require $K \ge B$ for algebraic identification of the zeta factors.
  \item \textbf{Precision Propagation:}
    If computation is done with $M$ bits (or decimal digits) of precision,
    the error in roots propagates at scale $\sim 10^{-M/B}$.
  \item \textbf{Policy (Heuristic Bound):}
    \[
      \delta^{\mathrm{disc}}(\theta)
      \ := \
      \frac{C \cdot q^{(d+1)K}}{10^{M}}
      \qquad \text{[Spec: heuristic upper bound]},
    \]
    for a fixed constant $C$ chosen in the \texttt{run.yaml} profile.
    If the observed defect stays below this threshold, the status of
    $\theta$ is \textsf{Plain}.
\end{enumerate}
\end{specification}

\begin{specification}[Singularity Threshold ($\lambda_{\mathrm{sing}}$)]
To distinguish a numerical artifact from a “Counterexample’’ (Type~I Failure),
we set
\[
  \lambda_{\mathrm{sing}}
  \ := \
  100 \cdot \delta^{\mathrm{disc}}(\theta).
\]
A defect $\Phi^{\mathrm{Weil}}_\tau(\theta)$ exceeding this value triggers a
\textbf{Peak Alert}.
Under the Core-Input (Deligne’s theorem), such a state is unreachable for
mathematically valid data and therefore indicates an implementation or
precision bug.
\end{specification}

% ------------------------------------------------------------
\subsection{C.4. Operational Protocol for the Hunter}
\label{app:weil-hunter-ops}

The Hunter agent executes the following loop for a parameter $\theta$:

\begin{enumerate}
  \item \textbf{Compute:}
    Calculate $N_r$ for $r = 1, \dots, K$.
  \item \textbf{Solve:}
    Recover candidate zeta factors and their roots $\{\alpha_{ij}\}$.
  \item \textbf{Filter:}
    Apply $\mathbf{T}_\tau^{\mathrm{spec}}$ (Definition~\ref{sec:weil-spectral-ucc}
    in Chapter~6) to remove spectrally unstable eigenvalues.
  \item \textbf{Measure:}
    Compute the defect
    $\Phi = \Phi^{\mathrm{Weil}}_\tau(\theta)$ as above.
  \item \textbf{Check:}
    \begin{itemize}
      \item If $\Phi \le \delta^{\mathrm{disc}}(\theta)$:
        return \texttt{VALID}.
      \item If $\Phi > \delta^{\mathrm{disc}}(\theta)$:
        re-run with higher precision (increase $M$ or $K$).
      \item If $\Phi$ persists above $\delta^{\mathrm{disc}}(\theta)$ despite
        increased precision:
        flag the run as \texttt{BUG} (given the Core-Input that Deligne’s
        theorem holds).
    \end{itemize}
\end{enumerate}



\section{Appendix D: AK--FLT Realization Details}
\label{app:flt-realization-details}

This appendix provides the technical specifications for the realization functor
$\mathfrak{P}_{\mathrm{FLT}}$ and the bridge diagnostics used in Part~III.
It details how the infinite towers of Iwasawa theory are mapped into the
finite-type persistence modules of the AK Core, and how the “Modularity Gap’’
is quantified.

% ------------------------------------------------------------
\subsection{D.1. From Iwasawa Modules to Persistence}
\label{app:flt-iwasawa-map}

Classical Iwasawa theory studies modules over the Iwasawa algebra
$\Lambda \cong \mathbb{Z}_p[[T]]$.
The AK Core, however, requires a filtration of vector spaces (or modules)
indexed by a real parameter $t$ or a discrete level $k$.

\begin{definition}[Pontryagin Duality Realization]
Let $\theta = (E, n) \in M_{\mathrm{FLT}}$ and let
$K_\infty / \mathbb{Q}$ be the cyclotomic $\mathbb{Z}_p$-extension.
The primary object is the $p$-primary Selmer group
$\mathrm{Sel}_p(E/K_\infty)$.
This is a discrete module.
To obtain a compact module suitable for structure theory, we take the
Pontryagin dual:
\[
  X_\infty(E)
  \ := \
  \mathrm{Hom}_{\mathrm{cont}}\!\bigl(
    \mathrm{Sel}_p(E/K_\infty),\ \mathbb{Q}_p/\mathbb{Z}_p
  \bigr).
\]
Then $X_\infty(E)$ is a finitely generated $\Lambda$-module.
\end{definition}

\begin{specification}[The Functor $\mathfrak{P}_{\mathrm{FLT}}$]
The realization functor constructs a persistence module $\mathbb{M}(\theta)$
indexed by the tower level $k \in \mathbb{N}$
(smoothed to $t \in \mathbb{R}_{\ge 0}$ in the AK engine):
\begin{enumerate}
  \item \textbf{Filtration:}
    The filtration is defined by the duals of the restriction maps in the
    Iwasawa tower.
    For each $k \ge 0$ set
    \[
      \omega_k \ := \ (1+T)^{p^k} - 1, \qquad
      X_k \ := \ X_\infty(E) / \omega_k X_\infty(E).
    \]
    The persistence module tracks the evolution of invariants such as
    $\dim_{\mathbb{F}_p}(X_k \otimes \mathbb{F}_p)$ as $k \to \infty$.
    Pontryagin dualization turns the inverse limit on the Selmer side into a
    direct system $\{X_k\}_{k \ge 0}$ suitable for the AK colimit-based
    tower diagnostics.

  \item \textbf{Barcode Interpretation (Structure Theorem):}
    Classically, the structure theorem gives an isomorphism of $\Lambda$-modules
    \[
      X_\infty(E)
      \ \sim \
      \Lambda^{\lambda}
      \ \oplus \
      \bigoplus_j \Lambda/(p^{\mu_j})
      \ \oplus \
      \bigoplus_i \Lambda/(f_i(T)).
    \]
    In the AK persistence language:
    \begin{itemize}
      \item The \textbf{$\lambda$-invariant} corresponds to the number of
        \emph{infinite bars} in the persistence diagram (stable rank).
      \item The \textbf{$\mu$-invariant} corresponds to bars that persist
        “vertically’’ in the $p$-adic direction (torsion with unbounded growth).
      \item The factors $\Lambda/(f_i(T))$ correspond to \emph{finite bars} of
        length controlled by $\deg(f_i)$ (and the valuations of their roots).
    \end{itemize}
\end{enumerate}
\end{specification}

\begin{interpretation}[The Control Theorem as Collapse]
Mazur’s Control Theorem asserts that the natural maps
\[
  X_\infty(E) / \omega_k X_\infty(E)
  \longrightarrow
  \mathrm{Sel}_p(E/K_k)^\vee
\]
have finite kernel and cokernel.
In AK terms, this means that the comparison maps $\phi_{i,\tau}$ associated
with the Iwasawa tower have trivial diagnostics
$(\mu_{\mathrm{Collapse}}, \nu_{\mathrm{Collapse}}) = (0,0)$
\emph{after} the collapse operator $\mathbf{T}_\tau$ removes the finite
error terms recorded in the $\delta$-ledger.
Finite $p$-power deviations are interpreted as “short bars’’ and are deleted
by $\mathbf{T}_\tau$.
\end{interpretation}

% ------------------------------------------------------------
\subsection{D.2. The Extension Bridge Index \texorpdfstring{$\mu_{\mathrm{Ext}}$}{mu\_Ext}}
\label{app:flt-bridge-index}

This index quantifies the “Wiles Gap’’—the failure of the Galois deformation
ring to align with the Hecke algebra.

\begin{definition}[Tangent Space Obstruction]
Let $\bar{\rho} = \rho_{E,p}$ be the residual representation.
Let $R_{\Sigma}$ be the universal deformation ring (Galois side) and
$\mathbb{T}_{\Sigma}$ the corresponding Hecke algebra (modular side).
There is a canonical surjection
$\pi : R_{\Sigma} \twoheadrightarrow \mathbb{T}_{\Sigma}$.
Let $\mathfrak{m}_R$ and $\mathfrak{m}_{\mathbb{T}}$ denote the maximal ideals.
We define the AK bridge index $\mu_{\mathrm{Ext}}$ via the tangent spaces:
\[
  \mu_{\mathrm{Ext}}(\theta)
  \ := \
  \dim_k
    \frac{\mathfrak{m}_R}{\mathfrak{m}_R^2 + (p)}
  \ - \
  \dim_k
    \frac{\mathfrak{m}_{\mathbb{T}}}{\mathfrak{m}_{\mathbb{T}}^2},
\]
where $k$ is the residue field.
Informally, $\mu_{\mathrm{Ext}}$ measures the “extra directions’’ in the
Galois deformation space that are not accounted for by modular forms.
\end{definition}

\begin{specification}[Operational Check]
In the Hunter simulation:
\begin{enumerate}
  \item If $\theta \in Z_{\mathrm{Valid}}$ (modular case), then by the
    Wiles--Taylor Core-Input we have $R_{\Sigma} \cong \mathbb{T}_{\Sigma}$,
    so the tangent spaces agree and
    \[
      \mu_{\mathrm{Ext}}(\theta) \ \approx \ 0
    \]
    (up to the numerical tolerances recorded in the $\delta$-ledger).

  \item If $\theta \in L_{\mathrm{Frey}}$ (Frey profile), then
    the level-lowered Hecke algebra at level $2$ collapses to $0$
    (Ribet + $\dim S_2(\Gamma_0(2)) = 0$), while the deformation ring
    $R_{\Sigma}$ is still non-trivial.
    At the level of the scalar summary, this is recorded as
    \[
      \mu_{\mathrm{Ext}}(\theta)
      \ \gg \
      \lambda_{\mathrm{sing}},
    \]
    i.e.\ it exceeds the singularity threshold in the quantale.
    In the implementation log this may be tagged as
    “\texttt{BRIDGE\_GAP := SATURATED}’’ rather than a literal infinity.
\end{enumerate}
\end{specification}

% ------------------------------------------------------------
\subsection{D.3. Stability and Threshold Policies for FLT}
\label{app:flt-thresholds}

Unlike the Weil calibration, where errors are analytic tails, FLT errors are
discrete algebraic objects (finite $p$-groups, tangent-space dimensions).

\begin{declaration}[The Discrete Ledger $\delta^{\mathrm{alg}}$]
\begin{enumerate}
  \item \textbf{Quantale:}
    For algebraic defects we use a valuation-type quantale $V_{\mathrm{val}}$,
    for example the ordered set of non-negative integers with addition.
  \item \textbf{Control Error:}
    The kernel and cokernel of the Control Theorem maps are recorded as
    $\delta^{\mathrm{alg}}$.
    For semistable curves with good ordinary reduction at $p$, these defects
    are bounded constants independent of the tower level $k$.
  \item \textbf{Threshold:}
    We choose the Singularity Threshold $\lambda_{\mathrm{sing}}$ so that
    \[
      \text{finite $p$-groups}
      \ \ll \
      \lambda_{\mathrm{sing}}
      \ \ll \
      \text{genuinely unbounded growth (e.g. effective $\mu > 0$)}.
    \]
    This ensures that standard Iwasawa finite submodules are filtered out by
    $\mathbf{T}_\tau$, while genuine structural failures (such as a non-zero
    collapse index or a saturated bridge gap $\mu_{\mathrm{Ext}}$) trigger
    the gate.
\end{enumerate}
\end{declaration}

% ------------------------------------------------------------
\subsection{D.4. Operational Protocol for the Hunter (FLT Mode)}
\label{app:flt-hunter-ops}

The Hunter agent executes the following loop when probing the Frey Locus:

\begin{enumerate}
  \item \textbf{Propose:}
    Generate a triple $(A,B,C)$ and, when the arithmetic constraint
    $A^n + B^n = C^n$ is (hypothetically) satisfied, construct the Frey curve
    $E_{A,B,C}$.

  \item \textbf{Realize:}
    Compute the invariants of $\bar{\rho}_{E,p}$ (conductor $N$, traces $a_\ell$,
    local conditions) and build the associated deformation problem.

  \item \textbf{Target:}
    Identify the target space of modular forms $S_2(\Gamma_0(N_{\mathrm{reduced}}))$
    dictated by level lowering (Ribet Constraint).

  \item \textbf{Bridge Check:}
    \begin{itemize}
      \item Estimate the bridge index $\mu_{\mathrm{Ext}}(\theta)$ (gap
        between the Galois deformation ring and the Hecke algebra).
      \item For Frey curves, $N_{\mathrm{reduced}} = 2$, so the target space
        of cusp forms is zero and the effective gap exceeds
        $\lambda_{\mathrm{sing}}$.
      \item Result: the Modularity Gate reports a Type~IV obstruction for
        every such $\theta$.
    \end{itemize}

  \item \textbf{Verdict:}
    Record the path as \texttt{OBSTRUCTED} and prevent the corresponding cell
    from entering the Validity Map as a \textsf{Valid} region.
\end{enumerate}



\section{Appendix E: AK--Verdict and Mathematical Truth}
\label{app:ak-verdict-truth}

This appendix concludes the Arithmetic Calibration program by defining the
epistemological status of the ``AK Verdict.'' We address the fundamental
question: \emph{What does it mean for the AK--HDPST engine to declare a
parameter ``Valid'' or ``Obstructed''?}

This establishes the logical foundation for transferring the machinery from the
\textbf{Closed World} of Arithmetic Geometry (where Truth is known via Oracle)
to the \textbf{Open World} of Partial Differential Equations (where Truth is
unknown).

% ------------------------------------------------------------
\subsection{E.1. Two Modes of Operation: Calibration vs.\ Exploration}
\label{app:modes-of-op}

The AK--HDPST platform operates in two distinct epistemic modes.
The distinction lies in the source of the ``Ground Truth.''

\begin{definition}[Mode I: Calibration (Closed World)]
\textbf{Context:} Part II (Weil) and Part III (FLT).
\begin{itemize}
  \item \textbf{Oracle:} We possess external theorems (Deligne, Wiles, Taylor,
        Ribet) that serve as Core-Inputs.
  \item \textbf{Goal:} To minimize the discrepancy between
        $\AKverdict(\theta)$ and the oracle truth-value for $\theta$.
  \item \textbf{Failure Meaning:} If AK reports a Peak where the Oracle says
        Plain, it is a \textbf{System Bug} (precision error, logic flaw, or
        incorrect realization).
  \item \textbf{Success Criteria:} The reproduction of the known map topology
        (a global Plain for Weil, structural exclusion of the Frey locus for
        FLT) certifies the instrument's sensitivity.
\end{itemize}
\end{definition}

\begin{definition}[Mode II: Exploration (Open World)]
\textbf{Context:} Appendix~NS (Navier--Stokes) and future work (BSD, Langlands).
\begin{itemize}
  \item \textbf{Oracle:} None. The map is \emph{terra incognita}.
  \item \textbf{Goal:} To generate the Validity Map $\mathcal{V}_\tau$ and
        identify stable regions ($Z_{\mathrm{Valid}}$) versus singularity
        candidates ($Z_{\mathrm{Sing}}$).
  \item \textbf{Failure Meaning:} If AK reports a Peak (Type~IV), it is treated
        as a \textbf{Discovery}: a candidate obstruction to be scrutinized by
        rigorous analysis.
  \item \textbf{Success Criteria:} The production of either a
        \emph{Global Certificate} (evidence for regularity) or a
        \emph{Certified Obstruction} (evidence for blow-up) that withstands
        external mathematical checking.
\end{itemize}
\end{definition}

% ------------------------------------------------------------
\subsection{E.2. The Anatomy of an AK Verdict}
\label{app:verdict-anatomy}

We now state precisely what the output
$\AKverdict(\theta) = \mathsf{Valid}$ signifies mathematically.

\begin{declaration}[The White-Box Guarantee]
An AK Verdict is not a ``black-box'' prediction (as in a generic classifier).
It is a \textbf{computational certificate} composed of three verifiable traces:
\begin{enumerate}
  \item \textbf{The Persistence Trace:}
    A record that the persistence barcode $B(\mathfrak{P}(\theta))$ contains
    no bars violating the Collapse Contract (after $\mathbf{T}_\tau$
    truncation).

  \item \textbf{The $\delta$-Ledger Trace:}
    A strict accounting showing that all deleted information had norm
    $\|\cdot\|_V$ strictly below the allocated budget
    (Discretization + Measurement + Algebraic defects).

  \item \textbf{The Tower Trace:}
    Evidence that the comparison maps $\phi_{i,\tau}$ in the defining tower
    (Frobenius or Iwasawa) have trivial kernel and cokernel after collapse,
    i.e.\ $(\mu_{\mathrm{Collapse}}, \nu_{\mathrm{Collapse}}) = (0,0)$ within
    the $\delta$-ledger tolerances.
\end{enumerate}
Therefore, $\AKverdict(\theta) = \mathsf{Valid}$ means:
\emph{Within resolution $\tau$ and budget $\delta$, no structural obstruction
is detected by the AK Core.}
\end{declaration}

% ------------------------------------------------------------
\subsection{E.3. The Justification for Navier--Stokes}
\label{app:nse-justification}

Why does the success in Number Theory justify the application to Fluid
Dynamics? The link is the universality of the
\textbf{Type~IV Singularity}.

\begin{interpretation}[The Universal Structure of Failure]
\begin{itemize}
  \item In \textbf{FLT}, a counterexample would require a Galois
        representation that is locally valid everywhere but fails to assemble
        globally into a modular form.
        This is a \textbf{coherence failure} across the Iwasawa tower and the
        deformation/Hecke bridge.

  \item In \textbf{NSE} (Navier--Stokes), a blow-up would require a velocity
        field that is smooth on every finite scale but fails to extend
        continuously to a limit time $T^*$.
        This is a \textbf{continuation failure} along the time-direction tower.
\end{itemize}
The AK Core diagnoses both phenomena using the same algebraic tool:
a non-vanishing limit defect, recorded as
$\mu_{\mathrm{Collapse}} > 0$ (or a saturated bridge index).
By calibrating the Type~IV detector on the hard-wired obstructions of FLT,
we verify that the engine is capable of detecting subtle coherence failures.

Thus, if the engine scans the Navier--Stokes parameter space and finds
\emph{no} Type~IV failures above the $\delta$-budget, the evidence for
regularity is structurally stronger than that of isolated numerical
simulations: it is a statement about the \emph{global topology} of the
Validity Map.
\end{interpretation}

% ------------------------------------------------------------
\subsection{E.4. Final Authorization}
\label{app:final-auth}

\begin{declaration}[System Transition]
The Arithmetic Calibration Programs (Part~II and Part~III) are hereby marked
\texttt{COMPLETE}. The internal logic gates of the AK--HDPST v17.0 engine are
certified against the Weil and FLT oracles.

\textbf{Next Step:} The system is authorized to engage the
\textbf{Navier--Stokes Case Study (Appendix NS)}.
\begin{itemize}
  \item \textbf{Active Conjecture:} Navier--Stokes Global Regularity.
  \item \textbf{Search Strategy:} Disproof Mode (Hunter seeks violations of
        the Energy--Collapse inequality / Type~IV signatures).
  \item \textbf{Status:} \textsf{[Spec]} layer active (no external oracle).
\end{itemize}
\end{declaration}





\end{document}