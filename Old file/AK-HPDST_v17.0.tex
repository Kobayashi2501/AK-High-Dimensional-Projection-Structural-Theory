% ===========================
% AK High-Dimensional Projection Structural Theory v17.0 (XeLaTeX)
% ===========================
\RequirePackage[2023-11-01]{latexrelease}

\documentclass[11pt]{article}

% --- (NEW) Math alphabet reservation cut (must be BEFORE ams/bm) ---
\newcommand\hmmax{0}
\newcommand\bmmax{0}

% --- Encoding & Language ---
% [XeLaTeX] utf8 はネイティブ対応なので inputenc は不要・非推奨
% \usepackage[utf8]{inputenc} % <-- 削除
\usepackage[T1]{fontenc}
\usepackage[english]{babel}
\usepackage{geometry}
\geometry{margin=1in}

% --- Core Math & Utilities (AMS first) ---
\usepackage{amsmath,amssymb,amsthm,amsfonts,mathtools}
\usepackage{mathrsfs}
\usepackage{bm}
\usepackage{stmaryrd}
\usepackage{changepage}
\usepackage{amscd}
\usepackage{textcomp}
\usepackage{etoolbox} % for \forcsvlist
\usepackage{enumitem}
\usepackage{array}

% --- Fonts (XeLaTeX でも利用可能な pdfLaTeX 系 Times セット) ---
\usepackage{newtxtext,newtxmath}  % Times-like text & math
\usepackage{inconsolata}          % Monospace for listings
\DeclareTextFontCommand{\textttit}{\ttfamily\slshape} % monospace "italic-like"

% --- Diagrams ---
\usepackage{tikz, tikz-cd}
\usetikzlibrary{
  matrix, arrows.meta, cd, calc, positioning, quotes,
  decorations.pathmorphing, decorations.markings,
  shapes.misc, shapes.geometric, arrows, fit, backgrounds, fadings
}
\usepackage{pgfplots}
\pgfplotsset{compat=1.18}
\usepackage[all,cmtip]{xy}
\numberwithin{equation}{section}
\theoremstyle{plain}
\theoremstyle{definition}
\theoremstyle{remark}

% --- Tables ---
\usepackage{array,tabularx,booktabs}
\newcolumntype{L}[1]{>{\raggedright\arraybackslash}p{#1}}
\newcolumntype{C}[1]{>{\centering\arraybackslash}p{#1}}
\newcolumntype{R}[1]{>{\raggedleft\arraybackslash}p{#1}}
\newcolumntype{Y}{>{\raggedright\arraybackslash}X}

% --- Monospace token helper (safe in math) ---
\newcommand{\fname}[1]{\text{\texttt{#1}}}

% --- Listings / Code ---
\usepackage{xcolor}
\usepackage{listings}
\usepackage{float}
\lstset{
  basicstyle=\ttfamily\small,
  keywordstyle=\color{blue},
  commentstyle=\color{gray},
  stringstyle=\color{orange},
  frame=single,
  breaklines=true,
  showstringspaces=false,
  captionpos=b,
  xleftmargin=1em,
  columns=fullflexible,
  upquote=true,
  mathescape=false,
  literate=*%
    {∀}{{$\forall$}}1 {∃}{{$\exists$}}1
    {→}{{$\to$}}1 {←}{{$\leftarrow$}}1 {⟶}{{$\longrightarrow$}}1 {⇒}{{$\Rightarrow$}}1 {⇔}{{$\Leftrightarrow$}}1 {↦}{{$\mapsto$}}1 {⥤}{{$\to$}}1
    {α}{{$\alpha$}}1 {β}{{$\beta$}}1 {γ}{{$\gamma$}}1 {δ}{{$\delta$}}1 {ε}{{$\varepsilon$}}1
    {λ}{{$\lambda$}}1 {μ}{{$\mu$}}1 {ν}{{$\nu$}}1 {π}{{$\pi$}}1 {τ}{{$\tau$}}1
    {κ}{{$\kappa$}}1 {θ}{{$\theta$}}1 {η}{{$\eta$}}1 {φ}{{$\varphi$}}1
    {ι}{{$\iota$}}1 {ρ}{{$\rho$}}1 {χ}{{$\chi$}}1 {ω}{{$\omega$}}1
    {Γ}{{$\Gamma$}}1 {Δ}{{$\Delta$}}1 {Π}{{$\Pi$}}1 {Σ}{{$\Sigma$}}1 {Ω}{{$\Omega$}}1 {Λ}{{$\Lambda$}}1 {Φ}{{$\Phi$}}1 {Ψ}{{$\Psi$}}1
    {ℕ}{{$\mathbb{N}$}}1 {ℤ}{{$\mathbb{Z}$}}1 {ℚ}{{$\mathbb{Q}$}}1 {ℝ}{{$\mathbb{R}$}}1
    {≤}{{$\le$}}1 {≥}{{$\ge$}}1 {≠}{{$\ne$}}1 {∈}{{$\in$}}1 {⊆}{{$\subseteq$}}1
    {∧}{{$\land$}}1 {∨}{{$\lor$}}1
    {⪯}{{$\preceq$}}1 {⪰}{{$\succeq$}}1 {≡}{{$\equiv$}}1 {≅}{{$\cong$}}1 {⋙}{{$\ggg$}}1
    {⊕}{{$\oplus$}}1 {⊗}{{$\otimes$}}1 {⊙}{{$\odot$}}1 {⊣}{{$\dashv$}}1 {⊢}{{$\vdash$}}1 {⊥}{{$\perp$}}1
    {·}{{$\cdot$}}1 {∙}{{$\cdot$}}1 {⋅}{{$\cdot$}}1 {•}{{$\bullet$}}1
    {・}{{$\cdot$}}1 {‧}{{$\cdot$}}1
    {×}{{$\times$}}1 {⨯}{{$\times$}}1
    {…}{{$\ldots$}}1 {–}{{--}}1 {—}{{---}}1 {∞}{{$\infty$}}1
    {₁}{{$_1$}}1 {⁺}{{${}^{+}$}}1
}

% --- Math Operators ---
\DeclareMathOperator{\Ext}{Ext}
\DeclareMathOperator{\Hom}{Hom}
\DeclareMathOperator{\Spec}{Spec}
\DeclareMathOperator{\colim}{colim}
\DeclareMathOperator{\PH}{PH}
\DeclareMathOperator{\Tor}{Tor}
\DeclareMathOperator{\rank}{rank}
\DeclareMathOperator{\im}{im}
\DeclareMathOperator{\id}{id}
\DeclareMathOperator{\Ker}{Ker}
\DeclareMathOperator{\Coker}{Coker}
\DeclareMathOperator{\Collapse}{Collapse}
\DeclareMathOperator{\Mot}{Mot}
\DeclareMathOperator{\GL}{GL}
\DeclareMathOperator{\RHom}{RHom}
\DeclareMathOperator{\HT}{HT}
\DeclareMathOperator{\gdim}{gdim}

% --- Minimal hyphenat compatibility (no package) ---
\DeclareRobustCommand{\hyp}{\nobreakdash-} % "one\hyp way"
\providecommand{\NoHyphens}[1]{\mbox{#1}}   % light-weight shim
\providecommand{\nohyphens}[1]{\mbox{#1}}

% --- General Macros ---
\newcommand{\QQ}{\mathbb{Q}}
\newcommand{\RR}{\mathbb{R}}
\newcommand{\CC}{\mathbb{C}}
\newcommand{\ZZ}{\mathbb{Z}}
\newcommand{\TT}{\mathbb{T}}
\newcommand{\CollapseZone}{\mathfrak{C}}
\newcommand{\cF}{\mathcal{F}}
\newcommand{\cG}{\mathcal{G}}
\newcommand{\cE}{\mathcal{E}}
\newcommand{\cO}{\mathcal{O}}
\newcommand{\cD}{\mathcal{D}}
\newcommand{\cH}{\mathcal{H}}
\newcommand{\into}{\hookrightarrow}
\newcommand{\onto}{\twoheadrightarrow}
\newcommand{\eps}{\varepsilon}
\newcommand{\Sha}{\mathcal{X}}
\newcommand{\CollapseCompatible}{\mathsf{CollapseCompatible}}
\newcommand{\CollapseTypeOf}[1]{\operatorname{CollapseType}(#1)}
\newcommand{\CollapseImage}{\operatorname{CollapseImage}}
\newcommand{\CollapseEnergy}{\mathcal{E}_{\mathrm{Coll}}}
\newcommand{\ptorsionfree}[1]{\ensuremath{#1}\nobreakdash-torsion-free}
\newcommand{\Pers}{\mathsf{Pers}}
\newcommand{\Vect}{\mathsf{Vect}}
\newcommand{\Ho}{\mathrm{Ho}}
\newcommand{\dq}{\textquotedbl}
\newcommand{\rH}{\mathrm{H}}
\newcommand{\timevar}{t}
\newcommand{\T}{\mathsf{T}}
\newcommand{\C}{\mathsf{C}}
\newcommand{\Coll}{\mathsf{C}}
\newcommand{\Rfun}{\mathcal{R}}

\theoremstyle{plain}
\theoremstyle{definition}

\newcommand{\kk}{k}
\newcommand{\xto}{\xrightarrow}
\newcommand{\xtoH}{\xRightarrow{\ \simeq\ }}
\newcommand{\xtoM}{\xRightarrow{\ \mono\ }}
\newcommand{\mono}{\mathrm{mono}}
\newcommand{\epi}{\mathrm{epi}}
\newcommand{\DD}{\mathcal{D}}
\newcommand{\cC}{\mathcal{C}}
\newcommand{\supp}{\mathrm{supp}}
\newcommand{\PE}{\mathrm{PE}}
\newcommand{\betti}{\beta}
\newcommand{\CT}{C_{\tau}}
\newcommand{\dint}{d_{\mathrm{int}}}

% --- Theorem-like Environments ---
\numberwithin{equation}{section}
\newtheorem{theorem}{Theorem}[section]
\newtheorem{proposition}[theorem]{Proposition}
\newtheorem{lemma}[theorem]{Lemma}
\newtheorem{corollary}[theorem]{Corollary}
\newtheorem{axiom}{Axiom}[section]
\newtheorem{conjecture}{Conjecture}[section]
\theoremstyle{definition}
\newtheorem{definition}[theorem]{Definition}
\newtheorem{example}[theorem]{Example}
\newtheorem{remark}[theorem]{Remark}
\newtheorem{specification}[theorem]{Specification}
\newtheorem{declaration}[theorem]{Declaration}

% --- Unified notation ---
\DeclareRobustCommand{\FiltCh}[1]{\mathsf{FiltCh}(#1)}
\DeclareRobustCommand{\Perskft}{\Pers^{\mathrm{cons}}_{k}}
\DeclareRobustCommand{\Pk}{\mathbf{P}}
\DeclareRobustCommand{\Ttau}{\texorpdfstring{\ensuremath{\mathbf{T}_{\tau}}}{T\_\tau}}
\DeclareRobustCommand{\Ctau}{\texorpdfstring{\ensuremath{C_{\tau}}}{C\_\tau}}
\DeclareRobustCommand{\Dbplain}{D^{\mathrm{b}}}
\DeclareRobustCommand{\Dbkmod}{D^{\mathrm{b}}(k\text{-mod})}
\DeclareRobustCommand{\Db}{D^{\mathrm{b}}(k\text{-mod})}
\DeclareRobustCommand{\muc}{\mu_{\mathrm{Collapse}}}
\DeclareRobustCommand{\nuc}{\nu_{\mathrm{Collapse}}}
\DeclareRobustCommand{\fqi}{\text{f.q.i.}}
\DeclareRobustCommand{\LC}{\texorpdfstring{\ensuremath{\mathrm{(LC)}}}{(LC)}}
\DeclareRobustCommand{\Qtest}{\{\,k[0]\,\}}
\DeclareRobustCommand{\pos}[1]{\left(#1\right)_{+}}
\DeclareRobustCommand{\Trop}{\texorpdfstring{\ensuremath{\operatorname{Trop}}}{Trop}}
\DeclareRobustCommand{\Mirror}{\texorpdfstring{\ensuremath{\operatorname{Mirror}}}{Mirror}}
\newcommand{\ProofBadge}{\textsf{\footnotesize[Proof]}}
\newcommand{\SpecBadge}{\textsf{\footnotesize[Spec]}}
\DeclarePairedDelimiter{\norm}{\lVert}{\rVert}
\numberwithin{equation}{section}
\theoremstyle{plain}
\theoremstyle{definition}
\theoremstyle{remark}

% Macros
\newcommand{\bbR}{\mathbb{R}}
\newcommand{\bbN}{\mathbb{N}}
\newcommand{\Vectk}{\mathsf{Vect}_k}
\newcommand{\Dbk}{D^{\mathrm{b}}(k\text{-mod})}
\newcommand{\Perscons}{\Pers^{\mathrm{cons}}_k}
\newcommand{\Persft}{\Pers^{\mathrm{ft}}_k}
\newcommand{\Crop}{\mathbf{W}}
\newcommand{\Res}{\mathrm{Res}}
\newcommand{\intdist}{d_{\mathrm{int}}}
\newcommand{\Ecat}{\mathsf{E}_\tau}
\newcommand{\Orth}{(\mathsf{E}_\tau)^{\perp}}
\newcommand{\Len}{\Lambda_{\mathrm{len}}}
\newcommand{\qand}{\quad\text{and}\quad}
\newcommand{\qtext}[1]{\quad\text{#1}\quad}
\newcommand{\ang}[1]{\langle #1\rangle}
\newcommand{\ol}[1]{\overline{#1}}
\newcommand{\Trans}{\mathsf{Trans}}
\newcommand{\Funct}{\mathsf{Funct}}
\newcommand{\loc}{\mathrm{loc}}
\newcommand{\MECE}{\mathrm{MECE}}
\newcommand{\clip}{\mathrm{clip}}
\newcommand{\len}{\mathrm{len}}
\newcommand{\NN}{\mathbb{N}}
\newcommand{\EE}{\mathbb{E}}
\newcommand{\op}{\mathrm{op}}
\newcommand{\Pfun}{\mathbf{P}}
\newcommand{\Ufun}{\mathcal{U}}
\newcommand{\win}{\ensuremath{\mathrm{win}}}
\newcommand{\spec}{\ensuremath{\mathrm{spec}}}
\newcommand{\disc}{\ensuremath{\mathrm{disc}}}
\newcommand{\meas}{\ensuremath{\mathrm{meas}}}
\newcommand{\tot}{\ensuremath{\mathrm{tot}}}

% 注意:\int は「積分記号」
\newcommand{\inter}{\ensuremath{\mathrm{int}}}
\providecommand{\Cfun}[1]{\mathsf{C}_{#1}}
\providecommand{\Tfun}[1]{\mathbf{T}_{#1}}
\providecommand{\Ctau}{\Cfun{\tau}}
\providecommand{\Csigma}{\Cfun{\sigma}}
\providecommand{\Ttau}{\Tfun{\tau}}
\providecommand{\Tsigma}{\Tfun{\sigma}}
\providecommand{\intdist}{d_{\mathrm{int}}}
\DeclareMathOperator{\coker}{coker}

% --- safe stubs (preamble) ---
\usepackage{amsmath,amssymb,mathtools}
\providecommand{\LC}{\mathsf{LC}}
\providecommand{\Gap}{\mathrm{Gap}}
\providecommand{\Ttau}{\mathbf{T}_{\tau}}
\providecommand{\Ctau}{\mathsf{C}_{\tau}}
\providecommand{\Rfun}{\mathcal{R}}
\providecommand{\Mirror}{\mathsf{Mirror}}
\providecommand{\Funct}{\mathsf{Fun}}
\providecommand{\muc}{\mu_{\mathrm{Collapse}}}
\providecommand{\nuc}{\nu_{\mathrm{Collapse}}}
\providecommand{\Gal}{\mathrm{Gal}}
\providecommand{\Tr}{\mathrm{Tr}}
\providecommand{\Fun}{\mathrm{Fun}}
\providecommand{\alg}{\mathrm{alg}}
\providecommand{\disc}{\mathrm{disc}}
\providecommand{\meas}{\mathrm{meas}}
\providecommand{\Defect}{\mathsf{Defect}}

% --- Over/underfull mitigation ---
\usepackage{microtype}
\emergencystretch=2em

% --- Hyperlinks (load ONCE, near the end) ---
\usepackage[colorlinks=true, linkcolor=blue, citecolor=blue, urlcolor=blue]{hyperref}
\usepackage{xurl}
\urlstyle{tt}
\def\UrlBreaks{\do\/\do\_\do\.\do\-}

\usepackage{tabularx,booktabs,array}
\newcolumntype{L}[1]{>{\raggedright\arraybackslash}p{#1}}
\newcolumntype{Y}{>{\raggedright\arraybackslash}X}

% --- Clever references (after hyperref) ---
\usepackage[capitalise,nameinlink,noabbrev]{cleveref}

% --- Float / page control ---
\usepackage{placeins}

% ======================================================
% === Paste-artifact guard: robust \n handling (HEAVY) ===
% ======================================================
\providecommand{\n}{\unskip\space}
\makeatletter
\newcommand{\patchn}[1]{\expandafter\def\csname n#1\endcsname{\unskip\space #1}}
\forcsvlist{\patchn}{Hence,Therefore,Thus,Then,Proof,Remark,Theorem,Proposition,Lemma,Corollary,
Definition,Example,Declaration,Specification,Spec,Appendix,Section,Subsection,We,In,Let,Under,
Given,Assuming,Assumption,Fix,If,When,Where,While,For,From,By,On,Off,Also,Moreover,Conversely,
Equivalently,Finally,Next,First,Second,Third,Note,Notation,Convention,Claim,Case,Step,Algorithm,
Figure,Table,Thm,Prop,Lem,Cor,Def,Ex,Rem,Eq,Eqn,Ch,Chap,PF,BC,AK,NS,HT,PH}
\forcsvlist{\patchn}{hence,therefore,thus,then,proof,remark,theorem,proposition,lemma,corollary,
definition,example,declaration,specification,spec,appendix,section,subsection,we,in,let,under,
given,assuming,assumption,fix,if,when,where,while,for,from,by,on,off,also,moreover,conversely,
equivalently,finally,next,first,second,third,note,notation,convention,claim,case,step,algorithm,
figure,table,thm,prop,lem,cor,def,ex,rem,eq,eqn,ch,chap,pf,bc,ak,ns,ht,ph}
\patchn{Proof)}\patchn{Remark)}\patchn{Theorem)}\patchn{Lemma)}\patchn{Corollary)}
\newcommand{\patchnletter}[1]{\expandafter\def\csname n#1\endcsname{\unskip\space #1}}
\@tfor\@tempa:=ABCDEFGHIJKLMNOPQRSTUVWXYZabcdefghijklmnopqrstuvwxyz0123456789\do{%
  \expandafter\patchnletter\@tempa}
\makeatother

% ---------- Font fallback ----------
\makeatletter
\AtBeginDocument{%
  \@ifpackageloaded{stmaryrd}{\SetSymbolFont{stmry}{bold}{U}{stmry}{m}{n}}{}%
}
\makeatother


% ---------- 本文・見出し・しおりの Unicode → TeX 変換 ----------
\usepackage{newunicodechar}
\newunicodechar{Σ}{\texorpdfstring{\ensuremath{\Sigma}}{Sigma}}
\newunicodechar{Γ}{\texorpdfstring{\ensuremath{\Gamma}}{Gamma}}
\newunicodechar{Δ}{\texorpdfstring{\ensuremath{\Delta}}{Delta}}
\newunicodechar{Π}{\texorpdfstring{\ensuremath{\Pi}}{Pi}}
\newunicodechar{τ}{\texorpdfstring{\ensuremath{\tau}}{tau}}
\newunicodechar{ν}{\texorpdfstring{\ensuremath{\nu}}{nu}}
\newunicodechar{κ}{\texorpdfstring{\ensuremath{\kappa}}{kappa}}
\newunicodechar{θ}{\texorpdfstring{\ensuremath{\theta}}{theta}}
\newunicodechar{η}{\texorpdfstring{\ensuremath{\eta}}{eta}}
\newunicodechar{φ}{\texorpdfstring{\ensuremath{\varphi}}{phi}}
\newunicodechar{π}{\texorpdfstring{\ensuremath{\pi}}{pi}}
\newunicodechar{ε}{\texorpdfstring{\ensuremath{\varepsilon}}{epsilon}}
\newunicodechar{δ}{\texorpdfstring{\ensuremath{\delta}}{delta}}
\newunicodechar{ℝ}{\texorpdfstring{\ensuremath{\mathbb{R}}}{R}}
\newunicodechar{ℤ}{\texorpdfstring{\ensuremath{\mathbb{Z}}}{Z}}
\newunicodechar{ℕ}{\texorpdfstring{\ensuremath{\mathbb{N}}}{N}}
\newunicodechar{≤}{\texorpdfstring{\ensuremath{\le}}{<=}}
\newunicodechar{≥}{\texorpdfstring{\ensuremath{\ge}}{>=}}
\newunicodechar{≠}{\texorpdfstring{\ensuremath{\ne}}{\string!=}}
\newunicodechar{∈}{\texorpdfstring{\ensuremath{\in}}{in}}
\newunicodechar{⊆}{\texorpdfstring{\ensuremath{\subseteq}}{subseteq}}
\newunicodechar{∨}{\texorpdfstring{\ensuremath{\lor}}{or}}
\newunicodechar{∧}{\texorpdfstring{\ensuremath{\land}}{and}}
\newunicodechar{⪯}{\texorpdfstring{\ensuremath{\preceq}}{preceq}}
\newunicodechar{⪰}{\texorpdfstring{\ensuremath{\succeq}}{succeq}}
\newunicodechar{≡}{\texorpdfstring{\ensuremath{\equiv}}{equiv}}
\newunicodechar{≅}{\texorpdfstring{\ensuremath{\cong}}{cong}}
\newunicodechar{⊣}{\texorpdfstring{\ensuremath{\dashv}}{dashv}}
\newunicodechar{⊥}{\texorpdfstring{\ensuremath{\perp}}{perp}}
\newunicodechar{⊕}{\texorpdfstring{\ensuremath{\oplus}}{oplus}}
\newunicodechar{⊙}{\texorpdfstring{\ensuremath{\odot}}{odot}}
\newunicodechar{⋅}{\texorpdfstring{\ensuremath{\cdot}}{.}}
\newunicodechar{⟶}{\texorpdfstring{\ensuremath{\longrightarrow}}{->}}
\newunicodechar{⥤}{\texorpdfstring{\ensuremath{\to}}{->}}
\newunicodechar{⋙}{\texorpdfstring{\ensuremath{\ggg}}{>>>}}
\newunicodechar{∞}{\texorpdfstring{\ensuremath{\infty}}{infty}}
\newunicodechar{∀}{\texorpdfstring{\ensuremath{\forall}}{forall}}
\newunicodechar{∃}{\texorpdfstring{\ensuremath{\exists}}{exists}}
\newunicodechar{₁}{\texorpdfstring{$_1$}{_1}}
\newunicodechar{⁺}{\texorpdfstring{\textsuperscript{+}}{+}}
\newunicodechar{Č}{\v C}
\newunicodechar{č}{\v c}
\newunicodechar{Š}{\v S}
\newunicodechar{š}{\v s}
\newunicodechar{Ž}{\v Z}
\newunicodechar{ž}{\v z}
\newunicodechar{–}{--}
\newunicodechar{—}{---}
\newunicodechar{§}{\S}
\newunicodechar{’}{'}
\newunicodechar{“}{``}
\newunicodechar{”}{''}


% === Document Metadata ===
\title{AK High-Dimensional Projection Structural Theory\\
\Large Version 17.0: Collapse Structures, Group Simplification, and Persistent Projection Geometry} 
\author{\textbf{Atsushi Kobayashi} \quad {\small (with ChatGPT Research Partner)}}
\date{December 2025}

% === Document Start ===
\begin{document}

\maketitle



\begin{abstract}
We present \textbf{AK--HDPST v17.0}, a comprehensive framework that unifies rigorous topological auditing with AI-driven \emph{High-Dimensional Projection Search (HDPS)}. By redefining the \(\delta\)-ledger of v16.5 as a scalar \textbf{Defect Potential} \(\Phi\), we transform the passive diagnosis of mathematical conjectures into an active navigation problem on the parameter space \(\mathcal{M}\).

\textbf{Part I: Core Theory (The Auditor).}
Retaining the \emph{Universal Control Contract (UCC)} of v16.5, we work with constructible one-parameter persistence over a field. The exact bar-deletion reflector \(\mathbf{T}_\tau\) (collapse at scale \(\tau\)) is \(1\)-Lipschitz and idempotent on persistence and provides the foundational lens for all diagnostics. On \textbf{Denef--Pas definable} windows, we specify a \textbf{Local Reverse Bridge} at the search layer,
\[
\mathrm{Ext}^1 \approx 0 \ \wedge \ \text{Spectral-Gap Condition} \ \Longrightarrow \ \mathrm{PH}_1 = 0,
\]
which authorizes a controlled translation from categorical vanishing to topological regularity under audited spectral separation, without asserting any unconditional global equivalence \(\mathrm{Ext}^1 \Leftrightarrow \mathrm{PH}_1\).

\textbf{Part II: HDPS Engine (The Navigator).}
We partition \(\mathcal{M}\) into \emph{Terrain Cells} and deploy autonomous agents:
\begin{itemize}
    \item The \textbf{Hunter} minimizes \(\Phi\) via regime-aware gradient descent to locate valid regions (\(\Phi < \mathrm{gap}_\tau\)).
    \item The \textbf{Lifter} resolves essential singularities (Type~IV) by dimensional extension, subject to a strict \textbf{Lifting Penalty} recorded in the quantale ledger.
    \item The \textbf{Mapper} glues local certificates into a global \textbf{Map of Validity}.
\end{itemize}
This architecture replaces black-box AI predictions with reproducible, white-box computational proofs. The resulting \textbf{AK Structural Regularity Theorem} states that, under global boundedness of \(\Phi\), absence of Type~IV obstructions, and certified coverage of \(\mathcal{M}\), the underlying realization (e.g.\ Navier--Stokes flows, families of elliptic curves) collapses to the trivial object after \(\mathbf{T}_\tau\), yielding the desired regularity or equality statements. All protocols are machine-verifiable via the \texttt{run.yaml} Proof Object and the associated audit logs.
\end{abstract}



% =========================================================================
% Chapter 1 : Collapse — Operational Definition and Scope (v17.0 / UCC+HDPS)
% =========================================================================
\section{Chapter 1: Collapse — Operational Definition and Scope}
\addcontentsline{toc}{section}{Collapse — Operational Definition and Scope}

%------------------------------------------
% Scope Box (Global Guard-Rails; UCC-enhanced)
%------------------------------------------
\noindent\fbox{%
\parbox{\textwidth}{%
\textbf{Scope Box (UCC guard-rails; after-collapse \& search-ready).}
All statements in this chapter (and throughout) are made under the following guard-rails, collectively referred to as the \emph{UCC layer} (Index/Collapse/Audit/Search):
\begin{itemize}[leftmargin=1.25em]
  \item \textbf{Constructible 1D over a field.}
        We work in constructible one-parameter persistence over a \emph{field}.
        Filtered (co)limits are computed objectwise in \([\mathbb{R},\mathsf{Vect}]\) and then \emph{returned} to the constructible subcategory by verification or by applying \(\mathbf{T}_\tau\) (Appendix~A).

  \item \textbf{Index/Quantale enrichment; definable windows.}
        The time index \((\mathbb{R},\le)\) is enriched over a \emph{commutative quantale} \((\mathsf{V},\otimes,\mathbb{I},\le)\) (e.g.\ \([0,\infty],+\!,0\)).
        This quantale serves dual roles: as a metric for \emph{audit} and as a cost space for \emph{search}.
        Windows are \emph{right-open} intervals \([u,u')\) and definable (o-minimal or Denef--Pas), ensuring finite event sets and finite Čech depth.

  \item \textbf{After-collapse policy.}
        \emph{All equalities, exactness claims, monotonicity, comparisons, and gluing} are asserted only \emph{after collapse} at the persistence layer; concretely, we evaluate by the protocol
        \[
          \boxed{\ \text{for each } t\ \Longrightarrow\ \mathbf{P}_i\ \Longrightarrow\ \mathbf{T}_\tau\ \Longrightarrow\ \text{compare in }\Pers^{\mathrm{cons}}_k\ }.
        \]

  \item \textbf{Bridge policy (one-way globally; local reverse under Safety).}
        The forward bridge \(\mathrm{PH}_1\Rightarrow \Ext^1\) is established in Chapter~3 under hypotheses (B1)--(B3).
        A \emph{local reverse bridge} \(\Ext^1\Rightarrow\mathrm{PH}_1\) is \emph{only} authorized on windows satisfying the \emph{Spectral-Gap Condition} and the Collapse-Consistent Conditions (CCC) of Chapter~16.
        No global equivalence \(\mathrm{PH}_1\Leftrightarrow\Ext^1\) is claimed.

  \item \textbf{Dual Mode: Audit \& Navigation.}
  \begin{itemize}
      \item \textbf{Audit Mode:} \(\Sigma\delta < \mathrm{gap}_\tau\) certifies validity (proof) on a window via B-Gate\(^{+}\).
      \item \textbf{Navigation Mode:} \(\Sigma\delta\) is scalarized to a potential \(\Phi\) and minimized by AI agents (HDPS).
  \end{itemize}
  Non-commutation and implementation errors are externalized in the \(\delta\)-ledger, which aggregates in \(\mathsf{V}\) and supports both modes.
\end{itemize}
Appendices detail implementable ranges (PF/BC in Appendix~N; Mirror/Transfer in Appendix~L; AWFS in Appendix~K; reproducibility in Appendix~G; Definability in Appendix~Q; Iwasawa in Appendix~R; AI Agents in Appendix~U).%
}%
}

\medskip

%------------------------------------------
% Unified Collapse Contract (UCC)
%------------------------------------------
\begin{theorem}[Unified Collapse Contract (UCC)]\label{thm:UCC}
Fix a commutative quantale \(\mathsf{V}\) for distances/budgets, and right-open windows (optionally definable).
Then:
\begin{enumerate}[label=(\roman*),leftmargin=1.25em]
  \item \textbf{\(\mathbf{T}_\tau\) is a \(\mathsf{V}\)-nucleus.}
        For each \(\tau>0\), the Serre reflector \(\mathbf{T}_\tau\) is idempotent, \(\mathsf{V}\)-\(1\)-Lipschitz for \(d_{\mathrm{int}}\), and a closure/nucleus on the \(\tau\)-local subcategory (Appendix~A).

  \item \textbf{\(C_\tau\) is idempotent up to f.q.i.\ in \(\Ho(\mathsf{FiltCh})\).}
        Any filtered lift \(C_\tau\) is an idempotent comonad in \(\Ho(\mathsf{FiltCh})\) \emph{up to filtered quasi-isomorphism} (Appendix~B/K).

  \item \textbf{Deletion vs.\ inclusion after-collapse.}
        After applying \(\mathbf{T}_\tau\), deletion-type steps are \emph{monotone} (energy non-increasing), and inclusion-type steps are \emph{non-expansive} (Appendix~E).

  \item \textbf{Quantale \(\delta\)-ledger as Potential.}
        The \(\delta\)-ledger aggregates all residuals additively in \(\mathsf{V}\).
        It is \emph{subadditive} under composition and \emph{non-increasing} under after-collapse \(1\)-Lipschitz post-processing.
        This allows \(\Sigma\delta\) to function as a stable Defect Potential \(\Phi\) for high-dimensional search (Chapter~13).
\end{enumerate}
\end{theorem}

\noindent\emph{Proof sketch.}
Unchanged from v16.5, with (iv) extended to support the potential interpretation via Appendix~S.

\medskip

%------------------------------------------
% Windowed policy + UCC terminology
%------------------------------------------
\subsection*{1.0. Windowed proof policy, UCC-Contract, Overlap Gate, and AI Integration}

\begin{definition}[UCC-Contract]\label{def:UCC-Contract}
A \emph{UCC-Contract} on a run consists of:
\begin{itemize}[leftmargin=1.25em]
  \item \textbf{Index layer.}
        A quantale \(\mathsf{V}\) and a MECE family of definable windows \([u,u')\).

  \item \textbf{Collapse layer.}
        The exact reflector \(\mathbf{T}_\tau\) and a filtered lift \(C_\tau\); plus the after-collapse protocol
        \(\mathbf{P}_i \to \mathbf{T}_\tau \to \text{compare in }\Pers^{\mathrm{cons}}_k\).

  \item \textbf{Audit/Search layer.}
        A \(\delta\)-ledger with values in \(\mathsf{V}\).
        \begin{itemize}
            \item In \textbf{Audit Mode}, it enforces the inequality \(\Sigma\delta < \mathrm{gap}_\tau\).
            \item In \textbf{Search Mode}, its scalarization provides the potential \(\Phi\) and (discrete) gradients \(\nabla\Phi\) for Hunter agents (Chapter~14).
        \end{itemize}
\end{itemize}
The run manifest \texttt{run.yaml} (Appendix~G) records the mode and all budgets.
\end{definition}

\begin{definition}[Windows (MECE) and \(\delta\)-ledger]\label{def:windows-delta}
A \emph{domain window} is a right-open interval \(W=[u,u')\).
A windowing is \emph{MECE} (Mutually Exclusive, Collectively Exhaustive) if these windows partition the time axis.
For per-step budgets \(\delta_j\in\mathsf{V}\), the \emph{pipeline budget} is
\[
\Sigma\delta(i,\{\tau_j\})\ :=\ \bigoplus\nolimits_j \delta_j(i,\tau_j).
\]
In Search Mode this sum serves as the \textbf{Defect Potential} \(\Phi\).
\end{definition}

\begin{definition}[Window restriction (cropping) functor]\label{def:cropping}
For a window \(W=[u,u')\), define
\(\Crop_{W}:\Pers^{\mathrm{cons}}_k\to \Pers^{\mathrm{cons}}_k\) by restricting bars to \(W\).
The functor \(\Crop_{W}\) is exact and \(1\)-Lipschitz for the interleaving distance \(\intdist\).
\end{definition}

\begin{definition}[Overlap Gate]\label{def:overlap-gate}
For overlapping windows \(W_\alpha, W_\beta\), the \emph{Overlap Gate} passes if:
\begin{enumerate}[label=(\roman*),leftmargin=1.25em]
\item \(\intdist\big(\Crop_{W_\alpha\cap W_\beta}\mathcal{B}_{\alpha,i},\,\Crop_{W_\alpha\cap W_\beta}\mathcal{B}_{\beta,i}\big)\le \Sigma\delta_{\alpha\beta}\);
\item the safety margin satisfies \(\mathrm{gap}_\tau > \Sigma\delta_{\alpha\beta}\);
\item tower diagnostics satisfy \((\mu_{\mathrm{Collapse}},\nu_{\mathrm{Collapse}})=(0,0)\).
\end{enumerate}
Global passing pastes local certificates into a coherent Map of Validity (Mapper Protocol, Chapter~14).
\end{definition}

\begin{definition}[B-Gate\(^{+}\) (after-collapse gate)]\label{def:bgate-plus}
On a window \(W\) at threshold \(\tau\), \emph{B-Gate\(^{+}\) passes} if:
\begin{enumerate}[label=(\arabic*),leftmargin=1.25em]
\item \(\PH_1(C_\tau F|_W)=0\);
\item \(\Ext^1(\mathcal{R}(C_\tau F|_W),k)=0\) (eligibility checked);
\item \((\mu_{\mathrm{Collapse}},\nu_{\mathrm{Collapse}})=(0,0)\);
\item \(\mathrm{gap}_\tau > \Sigma\delta\) (Budget Check).
\end{enumerate}
In Search Mode, failure of (4) triggers the Hunter; failure of (3) triggers the Lifter (Chapter~14).
\end{definition}

%------------------------------------------
% Terminology and notation (succinct)
%------------------------------------------
\subsection*{1.1. Terminology and notation}
Fix a base field \(k\).
\(\Vect_k\) denotes finite-dimensional \(k\)-vector spaces.
\(\FiltCh\) denotes finite-type filtered chain complexes.
\(\mathbf{P}_i\) is degreewise persistence.
\(\intdist\) is the interleaving distance.
\textbf{Convention:} All filtered (co)limits are objectwise in \([\mathbb{R},\Vect_k]\) and validated for constructibility (Appendix~A).

%------------------------------------------
% Operational collapse
%------------------------------------------
\subsection*{1.2. Collapse (operational)}

\begin{definition}[Exact truncation \(\mathbf{T}_\tau\) and lift \(C_\tau\)]\label{def:Ttau-Ctau}
\(\mathbf{T}_\tau:\Pers^{\mathrm{cons}}_k\to\Pers^{\mathrm{cons}}_k\) is the exact reflective localization deleting bars of length \(\le \tau\).
A filtered lift \(C_\tau:\FiltCh\to\FiltCh\) is chosen so that
\(\mathbf{P}_i \circ C_\tau \cong \mathbf{T}_\tau \circ \mathbf{P}_i\) up to filtered quasi-isomorphism.
\end{definition}

\begin{definition}[Collapse Zone \(\mathfrak{C}\)]
\[
\mathfrak{C} := \{F \mid \PH_1(F)=0 \ \wedge\ \Ext^1(\mathcal{R}(F),k)=0\}.
\]
Under the UCC, PH-collapse implies Ext-collapse (Chapter~3).
A local reverse bridge \(\Ext^1\Rightarrow\PH_1\) is available only under the Spectral-Gap and CCC hypotheses of Chapter~16 (marked \emph{[Spec]}).
\end{definition}

%------------------------------------------
% Failure landscape and tower diagnostics
%------------------------------------------
\subsection*{1.3. Failure landscape and the invisible obstruction \texorpdfstring{\((\mu,\nu)\)}{(mu, nu)}}
We retain the failure classification from v16.5:
\begin{itemize}[leftmargin=1.25em]
  \item \textbf{Type I--III:} Observable defects (topological, categorical, instability).
        These correspond to non-vanishing \(\PH_1\), \(\Ext^1\), or violation of monotonicity, and can be targeted directly by Hunter agents via \(\Phi\)-minimization.

  \item \textbf{Type IV (Invisible):}
        Tower limits \(\phi_i\) fail to be isomorphisms, yielding \((\mu_{\mathrm{Collapse}},\nu_{\mathrm{Collapse}}) \neq (0,0)\).
        This indicates an \emph{essential singularity} in the parameter space, treated in the HDPS layer by Dimensional Lifting (Lifter Agent, Chapter~14).
\end{itemize}

%------------------------------------------
% Convergence manager (definable cover; summability)
%------------------------------------------
\subsection*{1.5. Convergence manager (definable cover; summability)}

\begin{theorem}[Definable countable cover \(\Rightarrow\ \Sigma\delta<\infty\)]\label{thm:dp-sum}
Let \(\{W_n\}\) be a countable family of right-open \emph{definable} windows with finite Čech depth \(K\).
If \(\sum_n \delta(W_n)<\infty\) in \(\mathsf{V}\), then the global overlap error is bounded, and gluing via Overlap Gates holds on the covered domain.
\end{theorem}

\medskip

\noindent\textit{References and provenance.}
Existence, exactness, and interval decomposition in the constructible \(1\)D range follow standard sources (e.g.\ Crawley--Boevey; Chazal--de Silva--Glisse--Oudot).
Derived/sheaf-theoretic and Fukaya-category realizations are \textbf{[Spec]} (Appendix~N/O); they do not enlarge the proven bridge.
AWFS/2-cell commutation and quantitative soft-commuting are \textbf{[Spec]} (Appendix~K/L).
Denef--Pas definability is detailed in Appendix~Q; Iwasawa alignment in Appendix~R; Restart/Summability in Appendix~J.

%------------------------------------------
% Manifest requirements and Badges
%------------------------------------------
\subsection*{1.7. Manifest requirements and Badges}

\paragraph{run.yaml (v17.0 schema).}
\[
\texttt{quantale/\ definable/\ awfs/\ diagnostics/\ gates/\ cover/\ delta\_budget/\ search\_strategy}
\]
New keys: \texttt{search\_strategy} (Hunter configuration), \texttt{lifting\_penalty} (Lifter cost policy).

\paragraph{Badges.}
\emph{Proof}: UCC (Theorem~\ref{thm:UCC}), Convergence (Theorem~\ref{thm:dp-sum}).
\emph{Spec}: AI Agent Protocols (Chapter~14), Defect Potential \(\Phi\) (Chapter~13), Reverse Bridge Programs (Chapter~16).




% ===========================
% Chapter 2 : Concrete Model — Finite-Type Filtered Chain Complexes and Thresholded Collapse (v17 / UCC)
% IMRN/AiM style: concise, proof-first where canonical; [Spec] marks implementable-but-abstract claims.
% ===========================
\section{Chapter 2: Concrete Model — Finite-Type Filtered Chain Complexes and Thresholded Collapse}\label{sec:ch2}
\addcontentsline{toc}{section}{Concrete Model — Finite-Type Filtered Chain Complexes and Thresholded Collapse}

\subsection*{2.1. The category \texorpdfstring{$\FiltCh$}{FiltCh(k)} and persistence modules}\label{subsec:ch2-cat}
Fix a field \(k\). Let \(\mathsf{Ch}^{b}(k)\) be the category of bounded chain complexes of finite-dimensional \(k\)-vector spaces. A \emph{finite-type filtered chain complex} is a pair
\[
F=(C_\bullet,\{F^{t}C_\bullet\}_{t\in\bbR})
\]
where \(F^{t}C_\bullet\subseteq F^{t'}C_\bullet\) for \(t\le t'\), the filtration is exhaustive and left-bounded, and, for each \(i\), the persistence module
\[
\mathrm{H}_i(F):\ \bbR\longrightarrow \Vectk,\qquad t\longmapsto \mathrm{H}_i(F^{t}C_\bullet)
\]
is pointwise finite-dimensional with finitely many critical parameters on compact intervals. Denote by \(\FiltCh\) the category of such \(F\) with filtration-preserving chain maps. For each \(i\), let
\[
\mathbf{P}_i:\ \FiltCh\longrightarrow \Perscons
\]
be the functor sending \(F\mapsto \mathrm{H}_i(F)\). We write \(\PH_i(F)\) for the barcode (multiset of intervals) of \(\mathbf{P}_i(F)\). Throughout, the interleaving (equivalently, bottleneck) distance on persistence modules is denoted \(\intdist\).

\paragraph{Standing convention (constructible range and notation).}
We work inside the constructible subcategory
\(\Perscons\subset\Pers\) (finite critical set on bounded intervals, equivalently p.f.d.\ with finitely many changes on compacts).
We identify \(\Persft\) with \(\Perscons\) by convention.
\(\Perscons\) is abelian, admits interval decompositions, and carries a well-defined length. All uses of abelianity, Serre subcategories, and exact localizations are made within \(\Perscons\); see Appendix~A for details.
For filtered complexes we keep the finite-type hypothesis and record that filtered colimits may exit finiteness (cf.\ Appendix~A). Filtered (co)limits, when used, are computed objectwise in \([\bbR,\Vectk]\) and then verified to return to \(\Perscons\); no claim is made outside this regime.

\begin{remark}[Constructible abelian setting]\label{rem:constructible-abelian}
Within \(\Perscons\), kernels and cokernels are computed pointwise and preserve finiteness; thus \(\Perscons\) is abelian with interval decompositions and Serre localizations by bar length. For each \(\tau>0\), the full subcategory \(\Ecat\) generated by interval modules of length \(\le \tau\) is a hereditary Serre (localizing) subcategory in this constructible \(1\)D setting.
\end{remark}

\begin{remark}[Windowed proof policy; right-open endpoints]\label{rem:ch2-mece}
Statements are applied \emph{per window} (cf.\ Chapter~1). A \emph{domain window} is a right-open interval \([u_k,u_{k+1})\). A windowing is \emph{MECE} if \(\bigsqcup_k [u_k,u_{k+1})=[u_0,U)\) and adjacent windows meet only at endpoints. Coverage checks: (i) \(\sum_k (u_{k+1}-u_k)=U-u_0\); (ii) event counts (births/deaths, with multiplicity) add over windows up to rounding tolerance. Thresholds and spectral bins are \emph{fixed per window} and used only \emph{after collapse}.
\end{remark}

\subsection*{2.2. Thresholded collapse: Serre localization on persistence and filtered lift}\label{subsec:ch2-collapse}
We recall truncation on persistence, give a \(V\)-enriched interleaving view, then lift to filtered complexes.

\paragraph{Lawvere \(V\)-distance and \(V\)-shifts.}\label{def:Vlawvere}
Let \((V,\le,\otimes,\mathbb{I})\) be a commutative unital quantale. A \emph{\(V\)-Lawvere metric} on \(\Perscons\) is encoded by a system of endofunctors \(\{S^{v}\}_{v\in V}\) with coherences:
\begin{enumerate}[leftmargin=1.25em]
    \item \(S^{\mathbb{I}}\cong \mathrm{Id}\), \(S^{v}\circ S^{w}\cong S^{v\otimes w}\), and if \(v\le w\) then \(S^v\Rightarrow S^w\).
    \item For intervals \(I[a,b)\), \(S^v\) preserves bar lengths (classically \(S^\varepsilon\) is the \(\varepsilon\)-shift).
\end{enumerate}
Two objects \(M,N\) are \(v\)-interleaved if there are \(f:M\!\to\! S^vN\), \(g:N\!\to\! S^vM\) closing to the units; put \(d_V(M,N):=\inf\{v\mid M,N\ \text{are }v\text{-interleaved}\}\). For \(V=([0,\infty],\ge,+,0)\), \(d_V=\intdist\).

\paragraph{Ephemeral part and localization (constructible \(1\)D).}
Let \(\Ecat\subset\Perscons\) be the Serre subcategory generated by intervals of length \(\le\tau\).
The reflector
\[
\Ttau:\ \Perscons\longrightarrow \Orth
\]
is the exact localization at \(\Ecat\) and is \(1\)-Lipschitz for \(\intdist\) (in fact \(V\)-\(1\)-Lipschitz for \(d_V\); Lemma~\ref{lem:Vshift}). Here \(\Orth\) is the \(\tau\)-local (orthogonal) subcategory. Concretely,
\(\Ttau(M)=M/E_\tau(M)\) with \(E_\tau\) the maximal \(\tau\)-ephemeral subobject.

\begin{remark}[Endpoint independence]\label{rk:2-endpoints}
\(\Ttau\) deletes precisely finite bars of length \(\le\tau\); infinite bars are invariant. Open/closed endpoint conventions do not affect \(\Ttau\) (bar lengths are interleaving invariants).
\end{remark}

\begin{lemma}[\(V\)-shift commutation and \(V\)-\(1\)-Lipschitz]\label{lem:Vshift}
For any \(v\in V\), \(\Ttau\circ S^{v}\cong S^{v}\circ \Ttau\). Hence \(\Ttau\) preserves \(v\)-interleavings and is \(V\)-\(1\)-Lipschitz.
\emph{Sketch.} \(S^v\) preserves \(\Ecat\) and descends to the Serre quotient; use the universal property of localization. \qed
\end{lemma}

\paragraph{Lifting to filtered complexes.}
Fix a functor \(\mathcal{U}:\Perscons\to\FiltCh\) realizing interval modules by elementary filtered complexes. For \(F\in\FiltCh\) define \(C_\tau(F)\) by filtered quasi-isomorphisms
\[
\mathbf{P}_i\big(\Ctau(F)\big)\ \xrightarrow{\ \cong\ }\ \Ttau\!\big(\mathbf{P}_i(F)\big)\qquad\text{for all }i.
\]
Any functorial choice (up to f.q.i.) is a \emph{thresholded collapse}.

\paragraph{Stability and calculus (persistence layer).}
\begin{enumerate}[leftmargin=1.25em]
  \item \emph{Non-expansiveness:} \(\intdist(\mathbf{P}_i(\Ctau F),\mathbf{P}_i(\Ctau G)) \le \intdist(\mathbf{P}_i(F),\mathbf{P}_i(G))\).
  \item \emph{Monotonicity/idempotence:} If \(\tau\le\sigma\) then \(\Ttau\circ\mathbf{T}_\sigma=\mathbf{T}_{\max\{\tau,\sigma\}}\), hence \(\Ctau\circ C_\sigma\simeq C_{\max\{\tau,\sigma\}}\simeq C_\sigma\circ\Ctau\) up to f.q.i.
  \item \emph{Exactness:} \(\Ttau\) is exact (Serre localization), preserving finite limits/colimits.
\end{enumerate}

% --- NEW: CNF (P1) and Ext-identification (P2) ---
\subsection*{2.2.1. Collapse Normal Form and the Ext–Hom edge}\label{subsec:CNF}
\begin{theorem}[Collapse Normal Form (CNF)]\label{thm:CNF}
In \(D^{\mathrm b}(k\text{-mod})\), every object \(X\) is canonically isomorphic to the direct sum of its cohomology objects placed in degrees:
\[
X\ \cong\ \bigoplus_{i\in\mathbb{Z}} H^i(X)[-i].
\]
\emph{Proof.} Over a field, short exact sequences of \(k\)-vector spaces split; by induction on the amplitude and splicing standard triangles, all extensions split in \(D^{\mathrm b}(k\text{-mod})\). Hence the stated decomposition. \qed
\end{theorem}

\begin{corollary}[Ext–Hom edge (degree \(1\))]\label{cor:ExtHom}
For any \(X\in D^{\mathrm b}(k\text{-mod})\),
\[
\Ext^1(X,k)\ \cong\ \Hom\!\big(H^1(X),k\big).
\]
\emph{Proof.} Apply \Cref{thm:CNF} and use \(\Ext^1\) on a direct sum of shifts; only the \(H^1(X)[-1]\) summand contributes. \qed
\end{corollary}

\begin{remark}[Use in gates]
Applied to \(X=\mathcal{R}(C_\tau F|_W)\) (amplitude \(\le 1\)), \Cref{cor:ExtHom} identifies \(\Ext^1\) with the edge \(H^1\); on \(E_1\)-degenerate windows \(H^1\cong \PH_1\), yielding the local reverse bridge (Chapter~3).
\end{remark}

\subsection*{2.3. Operator toolkit at window scale}\label{subsec:toolkit}
We collect the window-level operators and their basic interactions.

\paragraph{Cropping on persistence.}
For a right-open window \(W=[u,u')\), let
\[
\Crop_W:\ \Perscons\to\Perscons
\]
restrict bars to \(W\) (precompose with \(W\hookrightarrow\RR\), extend by \(0\)). \(\Crop_W\) is exact and \(1\)-Lipschitz for \(\intdist\).

\begin{lemma}[Commutation with truncation]\label{lem:CropT}
\(\Crop_W\circ \mathbf{T}_\tau\ \cong\ \mathbf{T}_\tau\circ \Crop_W\) for all \(W,\tau\).
\emph{Proof.} \(\Crop_W\) preserves bar lengths and the Serre subcategory \(\Ecat\); argue as in Lemma~\ref{lem:Vshift}. \qed
\end{lemma}

\paragraph{Window lift on filtered complexes.}
Write \(W_{\mathrm{clip}}:\FiltCh\to\FiltCh\) for any filtered functor whose persistence equals \(\Crop_W\) degreewise; identities below are asserted at the persistence layer via \(\mathbf{P}_i\circ W_{\mathrm{clip}}\simeq \Crop_W\circ \mathbf{P}_i\).

\subsection*{2.3.1. Safe low-pass (optional; P4)}\label{subsec:lowpass}
Let \(L_\tau\) be a linear smoothing operator on the measured side (signal/filtration axis) with kernel \(\varphi_\tau\) satisfying:
\[
\text{(LP1) even},\qquad \text{(LP2) mass }1,\qquad \text{(LP3) bandwidth }\asymp \sqrt{\tau}.
\]
We only measure \emph{after} applying \(\Ctau\) (persistence layer).

\begin{proposition}[Safe low-pass: non-expansive after collapse]\label{prop:lowpass}
Under \textnormal{(LP1)–(LP3)}, there exists a budget \(\delta^{\mathrm{alg}}_{\mathrm{LP}}(i,\tau)\) such that
\[
\mathbf{T}_\tau\circ \mathbf{P}_i\circ L_\tau\ \cong\ \mathbf{T}_\tau\circ \mathbf{P}_i\quad\text{up to }\delta^{\mathrm{alg}}_{\mathrm{LP}}(i,\tau),
\]
and \(\mathbf{T}_\tau\circ \mathbf{P}_i\circ L_\tau\) is \(1\)-Lipschitz for \(\intdist\).
\emph{Sketch.} (LP1)–(LP3) preserve bar lengths up to sub-threshold deformation; commutation with \(\Ttau\) follows by the same Serre-localization argument, with deviation accounted for in \(\delta^{\mathrm{alg}}_{\mathrm{LP}}\). \qed
\end{proposition}

\begin{definition}[Test \texttt{T-Lipschitz-AfterCollapse}]
Accept \(L_\tau\) on a run iff the empirical Lipschitz constant of \(M\mapsto \mathbf{T}_\tau\mathbf{P}_i(L_\tau M)\) is \(\le 1\) within tolerance and the commutation defect with \(\Ttau\) stays \(\le \delta^{\mathrm{alg}}_{\mathrm{LP}}\) on the declared sample. Logs are recorded in the \(\delta\)-ledger.
\end{definition}

\subsection*{2.3.2. Orthogonal operator form (triad) and \texorpdfstring{$\delta$}{delta}-commutation}\label{subsec:triad}
\begin{definition}[Operator triad and normal form]\label{def:triad}
On each window \(W\) and threshold \(\tau\), we use the triad
\[
\boxed{\ C_\tau\quad\longrightarrow\quad W_{\mathrm{clip}}\quad\longrightarrow\quad L_\tau\ (\text{optional})\ }
\]
with all measurements made on \(\mathbf{T}_\tau\mathbf{P}_i\). We call this the \emph{orthogonal form}.
\end{definition}

\begin{proposition}[Pairwise commutation up to \texorpdfstring{$\delta$}{delta}]\label{prop:pairwise-delta}
For each \(i\), there exist budgets \(\delta^{\mathrm{alg}}_{\mathrm{CW}}(i,\tau;W)\), \(\delta^{\mathrm{alg}}_{\mathrm{WL}}(i,\tau;W)\), \(\delta^{\mathrm{alg}}_{\mathrm{CL}}(i,\tau)\) in the chosen quantale such that
\[
\begin{aligned}
\mathbf{P}_i(C_\tau W_{\mathrm{clip}}F)&\ \cong\ \mathbf{P}_i(W_{\mathrm{clip}} C_\tau F)&&\text{up to }\delta^{\mathrm{alg}}_{\mathrm{CW}},\\
\mathbf{T}_\tau\mathbf{P}_i(W_{\mathrm{clip}} L_\tau F)&\ \cong\ \mathbf{T}_\tau\mathbf{P}_i(L_\tau W_{\mathrm{clip}} F)&&\text{up to }\delta^{\mathrm{alg}}_{\mathrm{WL}},\\
\mathbf{T}_\tau\mathbf{P}_i(C_\tau L_\tau F)&\ \cong\ \mathbf{T}_\tau\mathbf{P}_i(L_\tau C_\tau F)&&\text{up to }\delta^{\mathrm{alg}}_{\mathrm{CL}}.
\end{aligned}
\]
All three maps are \(1\)-Lipschitz after applying \(\mathbf{T}_\tau\).
\emph{Sketch.} Use \Cref{lem:CropT}, \Cref{prop:lowpass}, and the fact that all claims are asserted \emph{after collapse}. \qed
\end{proposition}

\begin{remark}[Recording and acceptance]
Budgets in \Cref{prop:pairwise-delta} are recorded in the \(\delta\)-ledger; if \(\sum\delta<\infty\) on a definable cover of finite Čech depth, Overlap Glue holds globally (Chapter~1; Appendix~J).
\end{remark}

\subsection*{2.4. Windowing (MECE), \texorpdfstring{$\tau$}{tau}-adaptation, and spectral bins}\label{subsec:ch2-window}
\begin{definition}[Domain windows (MECE) and coverage]\label{def:ch2-mece}
A \emph{domain windowing} is a finite or countable family \(\{[u_k,u_{k+1})\}_k\) with \(\bigsqcup_k [u_k,u_{k+1})=[u_0,U)\) and \(u_k<u_{k+1}\). Coverage checks:
\[
\sum_k (u_{k+1}-u_k)=U-u_0,\qquad \#\mathrm{Events}([u_0,U))=\sum_k \#\mathrm{Events}([u_k,u_{k+1}))\ (\pm\ \text{rounding}).
\]
\end{definition}

\begin{definition}[Collapse thresholds and \(\tau\)-sweep]\label{def:tau-adapt}
Fix \(\tau>0\) per window, e.g.\ \(\tau=\alpha\cdot \max\{\Delta t,\Delta x\}\) (\(\alpha>0\) fixed per run). A \emph{\(\tau\)-sweep} is a discrete set \(\{\tau_\ell\}\) on which \((\mu_{\mathrm{Collapse}},\nu_{\mathrm{Collapse}})\) and B-Gate\(^{+}\) are evaluated. A \emph{stable band} is a contiguous range of \(\tau\) with \((\mu,\nu)=(0,0)\).
\end{definition}

\begin{definition}[Spectral bins and aux-bars]\label{def:aux-bars}
For spectrum \((\lambda_m)_{m\ge 1}\), fix \(\beta>0\) and \([a,b]\). Bins \(I_r=[a+r\beta,a+(r+1)\beta)\) collect counts \(E_r\).
Along a discrete index, runs where \(E_r(j)>0\) define \emph{auxiliary spectral bars} (lifetimes measured in that index). After applying \(\Ctau\), these are monotone under deletion-type steps and stable under \(\varepsilon\)-continuations. They never replace the B-side gate.
\end{definition}

\subsection*{2.5. Collapse admissibility and robust variants}\label{subsec:ch2-admissibility}
Let \(\mathcal{R}:\FiltCh\to\Dbk\) be \(t\)-exact of amplitude \(\le 1\); fix \(\mathcal{Q}=\{k[0]\}\).

\begin{definition}[Admissibility]
\[
\texttt{CollapseAdmissible}(F)\ :\iff\ \PH_1(F)=0\ \ \wedge\ \ \Ext^1\!\big(\mathcal{R}(F),k\big)=0.
\]
Under the bridge (Chapter~3), \(\PH_1(F)=0\Rightarrow \Ext^1(\mathcal{R}(F),k)=0\) in \(\Dbk\).
\end{definition}

\begin{definition}[Robust admissibility at scale \(\varepsilon\)]
\(F\) is \emph{\(\varepsilon\)-robustly collapse-admissible} if
\(\PH_1\!\big(C_\varepsilon(F)\big)=0\) and \(\Ext^1\!\big(\mathcal{R}(C_\varepsilon(F)),k\big)=0\).
\end{definition}

\subsection*{2.6. Local equivalence on saturation windows}\label{subsec:ch2-local-equiv}
\begin{definition}[Saturation window]\label{def:saturation}
Fix \(i=1\), a window \(W=[u,u')\), and \(\tau>0\). \(W\) is a \emph{saturation window} for \(F\) if: (i) event stability holds on \(W\); (ii) the maximal finite bar length in \(W\)\(\le \tau-\eta\) for some \(\eta>0\); (iii) no bar lengths in \(W\) increase to \(\tau\). Also require tail isomorphism: \(\mu_{\mathrm{Collapse}}=\nu_{\mathrm{Collapse}}=0\) on \(W\).
\end{definition}

\begin{theorem}[Local PH–Ext equivalence on saturation windows]\label{thm:local-equiv}
Assume: (1) \(t\)-exact \(\mathcal{R}\) of amplitude \(\le 1\); (2) \(W\) is a saturation window at \(\tau\); (3) tail isomorphism at \(\tau\) on \(W\).
Then, on \(W\) at threshold \(\tau\),
\[
\PH_1\!\big(\Ctau F\big)=0\quad\Longleftrightarrow\quad \Ext^1\!\big(\mathcal{R}(\Ctau F),k\big)=0.
\]
\emph{Sketch.} Use \Cref{cor:ExtHom} on \(\mathcal{R}(C_\tau F|_W)\) and the identification \(H^1\!\cong \PH_1\) on \(E_1\)-windows; saturation excludes accumulation at \(\tau\). \qed
\end{theorem}

\subsection*{2.7. Length spectrum operator \texorpdfstring{$\Len$}{Lambda\_len} (windowed) and its invariance}\label{subsec:ch2-len}
\begin{definition}[Windowed length spectrum]\label{def:len-operator}
If \(M\cong \bigoplus_{j} I[b_j,d_j)\), define for \(W=[u,u')\) the clipped length \(\ell_W(I[b_j,d_j)):=\max\{0,\min\{d_j,u'\}-\max\{b_j,u\}\}\).
Let \(\Len(M;W)\) be the diagonal endomorphism on \(\bigoplus_j k\cdot e_j\) with eigenvalues \(\{\ell_W(I[b_j,d_j))\}_j\).
\end{definition}

\begin{proposition}[Invariance]\label{prop:length-spectrum}
The multiset of eigenvalues of \(\Len(M;W)\) equals \(\{\ell_W(I[b_j,d_j))\}_j\) and is invariant under isomorphisms \(M\simeq M'\).
\end{proposition}

\begin{remark}[First-length functional and Chapter~11]\label{rem:E1-crosslink}
\(E_1(M;W)=\mathrm{tr}(\Len(M;W))=\sum_j \ell_W(I[b_j,d_j))\). Stability for \(E_1\) follows from \(\Ttau\)'s \(1\)-Lipschitzness.
\end{remark}

\subsection*{2.8. \texorpdfstring{$\varepsilon$}{epsilon}-survival and robustness}\label{subsec:epsilon-survival}
\begin{lemma}[\(\varepsilon\)-survival under interleavings]\label{lem:epsilon-survival}
If \(\intdist(\mathbf{P}_i(F),\mathbf{P}_i(G))\le \varepsilon\), then any bar \(b\) of \(\mathbf{P}_i(F)\) with \([0,\tau_0]\)-clipped length \(\ell_{\tau_0}(b)>2\varepsilon\) has a counterpart in \(\mathbf{T}_{\tau_0}(\mathbf{P}_i(G))\) with clipped length at least \(\ell_{\tau_0}(b)-2\varepsilon\).
\end{lemma}

\subsection*{2.9. Operating summary (Chapter~2)}\label{subsec:ch2-summary}
Per right-open window \([u_k,u_{k+1})\):
\begin{itemize}[leftmargin=1.25em]
  \item Fix \(\tau\) (resolution-adapted) and, if used, spectral binning \((\beta,[a,b])\).
  \item Apply A-side steps (deletion-type or \(\varepsilon\)-continuation), then \(\Ctau\); \emph{measure only on} \(\mathbf{T}_\tau\mathbf{P}_i\).
  \item Use the operator triad \(C_\tau\to W_{\mathrm{clip}}\to L_\tau\) (optional) in orthogonal form; record \(\delta\)-commutation budgets from \Cref{prop:pairwise-delta}.
  \item Verify B-Gate\(^{+}\) and, for covers, the Overlap Gate; paste certificates using Restart/Summability (Chapter~4).
  \item If \Cref{thm:local-equiv} applies, use the window-local PH–Ext equivalence; otherwise keep the forward bridge only.
\end{itemize}

\paragraph{Core labels met in this chapter (cf.\ Chapter~1).}
\emph{P1 (CNF):} \Cref{thm:CNF}. \emph{P2 (Ext–Hom):} \Cref{cor:ExtHom}. \emph{P4 (Low-pass safety):} \Cref{prop:lowpass} with \texttt{T-Lipschitz-AfterCollapse}. Operator normal form \(C_\tau/W_{\mathrm{clip}}/L_\tau\) and \(\delta\)-commutation: \Cref{prop:pairwise-delta}.

\medskip
\noindent\textit{References and provenance.}
Serre localization, barcode abelianity, and interleaving stability are standard in the constructible \(1\)D setting. The CNF for \(D^{\mathrm b}(k\text{-mod})\) and the Ext–Hom edge are classical over fields. Cropping and truncation commute by preservation of the Serre subcategory. Safe low-pass is adopted with explicit acceptance tests; all smoothing is audited \emph{after collapse} and never used as a sole gate.



% ===========================
% Chapter 3 : A One-Way Bridge PH1⇒Ext1 and the Hypothesis Scheme (v17 / UCC)
% IMRN/AiM style: proof-first; [Spec] marks implementable-but-abstract items.
% ===========================
\section{Chapter 3: A One-Way Bridge \texorpdfstring{$\mathrm{PH}_1\!\Rightarrow\!\mathrm{Ext}^1$}{PH1⇒Ext1} and the Hypothesis Scheme}
\addcontentsline{toc}{section}{A One-Way Bridge $\mathrm{PH}_1\Rightarrow\mathrm{Ext}^1$ and the Hypothesis Scheme}

\noindent\textbf{Scope note (reinforced windowed policy).}
All statements lie in the constructible \(1\)D regime of Chapter~2 with field coefficients.
The implication \(\mathrm{PH}_1\Rightarrow \Ext^1\) is proved only in \(D^{\mathrm b}(k\text{-mod})\) under \textup{(B1)}–\textup{(B3)}.
Every claim is issued per right-open domain window \(W=[u,u')\) and fixed threshold \(\tau>0\); gate decisions are taken only on the B-side \emph{after collapse}, i.e.\ on single-layer objects \(\mathbf{T}_\tau\mathbf{P}_i\) (equivalently \(\mathbf{P}_i(C_\tau-)\)).
Equalities for filtered complexes hold up to filtered quasi-isomorphism (f.q.i.).

\subsection*{3.0. Windowed usage, \texorpdfstring{$E_1$}{E1}-first policy, and gate integration}
This chapter supplies the \(\mathrm{PH}_1\!\Rightarrow\!\Ext^1\) bridge used by B-Gate\(^{+}\) (Chapter~1) after applying \(C_\tau\) on \((W,\tau)\).
We adopt an \emph{\(E_1\)-first policy}: evaluate the windowed first-length functional \(\,E_1\) \emph{after collapse} as the primary determinant, then discharge Ext via the bridge.
Formally:
\begin{itemize}[leftmargin=1.25em]
  \item If \(E_1(C_\tau F;W)=0\), then \(\mathrm{PH}_1(C_\tau F|_W)=0\) (by definition of \(E_1\); cf.\ Chapter~2, \S2.7); if moreover \((\mu,\nu)=(0,0)\) and the safety margin satisfies \(\mathrm{gap}_\tau>\Sigma\delta\), B-Gate\(^{+}\) passes.
  \item If \(E_1(C_\tau F;W)>0\), the Ext-part cannot discharge the PH-part; B-Gate\(^{+}\) may still fail due to \(\mathrm{PH}_1>0\) or \((\mu,\nu)\neq(0,0)\).
  \item Independently, under \textup{(B1)}–\textup{(B3)} we always have the one-way implication \(\mathrm{PH}_1\!\Rightarrow\!\Ext^1\) on \((W,\tau)\).
\end{itemize}
Cross-domain comparisons (PF/BC, Mirror/Transfer) are performed \emph{after} \(C_\tau\); all non-commutations are recorded in the \(\delta\)-ledger.

\subsection*{3.1. Bridging Hypotheses (B1–B3)}
Fix the notation of Chapter~2. In particular, \(k\) is a field, \(\FiltCh(k)\) denotes finite-type filtered chain complexes, \(\mathbf{P}_i(F)\) is the degree-\(i\) persistence with barcode \(\mathrm{PH}_i(F)\), and \(\mathcal{R}:\FiltCh(k)\to D^{\mathrm b}(k\text{-mod})\) is a \(t\)-exact realization.

\begin{description}
\item[\normalfont (B1) Finite-type over a field.]
\(F\in\FiltCh(k)\) with pointwise finite-dimensional persistence. Filtered (co)limits are computed objectwise in \([\RR,\Vect_k]\) and used only within the constructible scope (Appendix~A).

\item[\normalfont (B2) Amplitude \(\boldsymbol{\le 1}\) and identification of the \(\boldsymbol{H^{-1}}\)-edge.]
There is a two-term model
\[
\mathcal{R}(F)\ \simeq\ \bigl[\ \varinjlim_{t} H_1(F^{t}C_\bullet)\ \xrightarrow{\,d\,}\ \varinjlim_{t} H_0(F^{t}C_\bullet)\ \bigr]\in D^{[-1,0]}(k\text{-mod}),
\]
natural in \(F\).

\item[\normalfont (B3) Edge identification for degree \(1\) with \(Q=k\).]
For any \(A\in D^{[-1,0]}(k\text{-mod})\),
\[
\Ext^1(A,k)\ \cong\ \Hom\!\big(H^{-1}(A),k\big),
\]
naturally in \(A\).
\end{description}

\begin{remark}[On (B2) and the edge identification]\label{rk:B2-edge}
By (B2), \(H^{-1}(\mathcal{R}(F))\cong \varinjlim_t H_1(F^tC_\bullet)\); over a field, (B3) gives \(\Ext^1(\mathcal{R}(F),k)\cong \Hom(H^{-1}(\mathcal{R}(F)),k)\). All uses remain in \(D^{\mathrm b}(k\text{-mod})\).
\end{remark}

\subsection*{3.2. One-way bridge and \texorpdfstring{$E_1$}{E1}-local strengthening}
\begin{theorem}[One-way bridge]\label{thm:PH1-to-Ext1}
Assume \textup{(B1)}–\textup{(B3)}. If \(\mathrm{PH}_1(F)=0\), then
\(\Ext^1\!\big(\mathcal{R}(F),k\big)=0\).
\end{theorem}

\begin{proof}
\(\mathrm{PH}_1(F)=0\) implies \(\varinjlim_t H_1(F^tC_\bullet)=0\); apply (B2) and (B3).
\end{proof}

\begin{theorem}[Local Reverse under \(\boldsymbol{E_1{=}0}\) (P3)]\label{thm:local-reverse}
Let \(W\) be a right-open window and \(\tau>0\). Assume after-collapse amplitude \(\le 1\) and tail isomorphism \((\mu,\nu)=(0,0)\) on \(W\).
If \(E_1(C_\tau F;W)=0\), then for any \(F\),
\[
\Ext^1\!\big(\mathcal{R}(C_\tau F|_W),k\big)=0\ \Longrightarrow\ \mathrm{PH}_1(C_\tau F|_W)=0.
\]
\end{theorem}

\begin{proof}
By Chapter~2, \S2.7, \(E_1(C_\tau F;W)=0\) iff the clipped-length multiset on \(W\) is identically \(0\), hence \(\mathrm{PH}_1(C_\tau F|_W)=0\).
On the other hand, by Chapter~2, \S2.2.1 (CNF) and Cor.~2.2.1, \(\Ext^1(X,k)\cong \Hom(H^1(X),k)\) for \(X\in D^{\mathrm b}(k\text{-mod})\).
Amplitude \(\le 1\) and tail isomorphism ensure the \(H^1\)-edge matches the stabilized degree-\(1\) persistence on \(W\).
Thus the stated implication holds (indeed, the premise on \(\Ext^1\) is superfluous once \(E_1=0\) is known; we keep it to match gate logging).
\end{proof}

\begin{remark}[Position relative to Theorem~\ref{thm:PH1-to-Ext1} and local equivalence]
Theorem~\ref{thm:local-reverse} is a reverse fragment valid on \emph{\(E_1\)-degenerate} windows.
A stronger window-local equivalence (PH\(\Leftrightarrow\)Ext) under definability, amplitude, and tail isomorphism appears in Theorem~\ref{thm:E1-local} below; globally we retain only the one-way bridge Theorem~\ref{thm:PH1-to-Ext1}.
\end{remark}

\subsection*{3.2 bis. E\(_1\)-local equivalence on definable windows}
\begin{theorem}[E\(_1\)-local equivalence on definable windows]\label{thm:E1-local}
Let \(W\) be right-open and \emph{o-minimal definable}, \(\tau>0\), and assume:
(i) all quantities are evaluated on \(\mathbf{T}_\tau\mathbf{P}_1(F|_W)\);
(ii) \(\mathcal{R}(C_\tau F)\in D^{[-1,0]}(k\text{-mod})\);
(iii) tail isomorphism \((\mu,\nu)=(0,0)\) on \(W\).
Then
\[
E_1(F;W,\tau)=0\quad\Longleftrightarrow\quad \mathrm{PH}_1(C_\tau F|_{W})=0\quad\Longleftrightarrow\quad \Ext^1\!\big(\mathcal{R}(C_\tau F)|_{W},k\big)=0.
\]
\end{theorem}

\begin{proof}[Proof sketch]
The first equivalence is by definition of \(E_1\) after collapse (Chapter~2, \S2.7).
The second follows from the two-term realization and Leray descent on a finite Čech nerve (definability), together with tail isomorphism.
\end{proof}

\subsection*{3.3. Quantitative primitives on a fixed window}
For a window \(W\) and threshold \(\tau\), define the residual-length energy
\(
E_i(F;W,\tau)=\sum_{J\in\mathcal{B}_i(F;W)} \max\{|J|-\tau,0\}
\)
and tail counts \(C_{i,r}(F;W,\tau)\) as in Chapter~3, \S3.1 bis; they are finite, piecewise-linear in \(\tau\), monotone, and stable for \(\intdist\).
After collapse,
\(
\sum_{J\in \mathcal{B}_i(C_\tau F;W)} |J|\le E_i(F;W,\tau)
\),
and \(E_i\), \(C_{i,r}\) are nonexpansive along \(\tau\)-continuations.

\subsection*{3.4. Survival lemma and safety margins}
\begin{lemma}[ε-survival]\label{lem:epsilon-survival-CH3}
Fix \(W\), \(\tau>0\), and \(\mathrm{gap}_\tau>\Sigma\delta\). If \(F,G\) are \(\varepsilon\)-interleaved on \(W\) and some \(J\in\mathcal{B}_1(F;W)\) satisfies \(|J|_\tau\ge \varepsilon+\mathrm{gap}_\tau\), then a corresponding \(J'\in\mathcal{B}_1(G;W)\) has \(|J'|_\tau\ge \mathrm{gap}_\tau\) and survives collapse on \(G\).
\end{lemma}

\subsection*{3.5. Gate indicators and quantitative linkage}
On \((W,\tau)\), the PH-indicator is \(\mathrm{PH}_1(C_\tau F)\); the Ext-indicator is \(\Ext^1(\mathcal{R}(C_\tau F),k)\); collapse indicators are \((\mu,\nu)\); spectral indicators are \(\{C_r\}_{r\ge 0}\) and \(E_1\).
If \(\mathrm{PH}_1(C_\tau F)=0\) and \((\mu,\nu)=(0,0)\), then by Theorem~\ref{thm:PH1-to-Ext1} the Ext-part discharges and B-Gate\(^{+}\) passes.
If \(C_r(F;W,\tau)=0\) for some \(r>\varepsilon+\Sigma\delta\), then \(\mathrm{PH}_1(C_\tau G)=0\) for every \(G\) \(\varepsilon\)-interleaved with \(F\) on \(W\) (Lemma~\ref{lem:epsilon-survival-CH3}), hence Ext discharges.

\subsection*{3.6. Test \texttt{T-ExtZero-implies-PHZero} (window-local; audit-ready)}
\begin{definition}[Test specification]\label{def:T-ExtZero-PHZero}
On a right-open window \(W\) and threshold \(\tau\), the test \texttt{T-ExtZero-implies-PHZero} \emph{passes} for \(F\) if
\[
\Bigl(\ E_1(C_\tau F;W)=0\ \wedge\ (\mu,\nu)=(0,0)\ \wedge\ \Ext^1(\mathcal{R}(C_\tau F|_W),k)=0\ \Bigr)\ \Longrightarrow\ \PH_1(C_\tau F|_W)=0.
\]
\emph{Run manifest.} The outcome, premises, and any violation are logged with keys
\noindent
\path{tests.T-ExtZero-implies-PHZero.{window, \tau, E1, mu, nu, Ext, PH, pass}} (Appendix~G).
Violations are tagged \path{counterexample.local_reverse}.
\end{definition}


\begin{remark}[Minimality and redundancy]
By Theorem~\ref{thm:local-reverse}, once \(E_1=0\) and \((\mu,\nu)=(0,0)\) hold, the conclusion \(\PH_1=0\) is forced; the explicit \(\Ext^1=0\) premise is recorded to align with audit trails and to surface any unexpected Ext anomalies under numerical or modeling noise.
\end{remark}

\subsection*{3.7. Naturality, stability, and windowed gate usage}
The edge identifications in (B2)–(B3) are natural in \(F\); \(\mathbf{T}_\tau\) is \(1\)-Lipschitz for \(d_{\mathrm{int}}\); thus \(\PH_1(C_\varepsilon(F))=0\) is metrically stable, and gate outcomes are invariant under functorial choices of \(C_\varepsilon\).
Quantitative primitives \(E_1\) and \(C_{1,r}\) provide monotone, stable diagnostics compatible with the gate.

\subsection*{3.8. Interaction with PF/BC, Mirror, and the \texorpdfstring{$\delta$}{delta}-ledger}
PF/BC transport is applied per \(t\), then collapsed; Mirror/Transfer comparisons are performed only after \(C_\tau\).
All non-commutations are externalized in the \(\delta\)-ledger; Ext checks are confined to \(D^{\mathrm b}(k\text{-mod})\).

\subsection*{3.9. Scope and limitations}
No claim is made that \(\Ext^1(\mathcal{R}(F),k)=0\Rightarrow \mathrm{PH}_1(F)=0\) globally.
Failure modes (including Type~IV/tower artifacts) are detected by \((\mu,\nu)\) (Appendix~D). Window-local equivalence requires definability, amplitude, and tail isomorphism (Theorem~\ref{thm:E1-local}).

\subsection*{3.10. Formalizability}
(B1)–(B3) and Theorem~\ref{thm:PH1-to-Ext1} are formalizable: (B2) via a two-term interface for \(\mathcal{R}\), (B3) via truncation and the long exact sequence.
\(E_1\), \(C_r\) are barcode-level primitives; their stability reduces to bottleneck stability (Appendix~H).
Windowed usage (B-Gate\(^{+}\)), MECE policy, and \(\delta\)-ledger appear as operational axioms in Appendix~F; proofs remain on the persistence layer and in \(D^{\mathrm b}(k\text{-mod})\).

\medskip
\noindent\textit{References and provenance.}
Serre localization, barcode abelianity, interleaving stability, and CNF/Ext–Hom (Chapter~2) are classical over fields.
The \(E_1\)-first policy is operational (Chapter~2, \S2.7; Chapter~11) and never replaces PH/Ext gates; it prioritizes a measurable determinant that is exact after collapse.



\section{Chapter 4: Failure Lattice, Local PH–Ext Equivalence, Čech–Ext Gluing, and the Tower-Sensitivity Invariant $\mu_{\mathrm{Collapse}}$}
\addcontentsline{toc}{section}{Failure Lattice, Local PH–Ext Equivalence, Čech–Ext Gluing, and the Tower-Sensitivity Invariant $\mu_{\mathrm{Collapse}}$}

\paragraph{Standing hypotheses and scope.}
We work in the constructible (finite-type) persistence range (Chapter~2, §2.1), adopt the bridging hypotheses \textup{(B1)–(B3)} from Chapter~3 with the minimal test family $\mathcal{Q}=\{k[0]\}$, and fix a $t$-exact realization $\mathcal{R}:\mathsf{FiltCh}(k)\to D^{\mathrm{b}}(k\text{-mod})$ of amplitude $\le1$.
All filtered (co)limit statements are asserted \emph{at the persistence layer} (Appendix~A).
Endpoints and infinite bars follow Chapter~2, Remark~\ref{rk:2-endpoints}.
Monotonicity for indicators applies only to \emph{deletion-type} updates; inclusion-type updates are \emph{stability-only} (Appendix~E).
Every claim is \emph{windowed} (Chapter~1, Def.~1.0; Chapter~2, §2.4), and all gate decisions are taken \emph{only} on the B-side after collapse (single layer $\mathbf{T}_\tau\mathbf{P}_i$).
We assert \emph{only} the one-way core bridge $\mathrm{PH}_1\Rightarrow \Ext^1$ in $D^{\mathrm{b}}(k\text{-mod})$ under \textup{(B1)–(B3)} (Chapter~3; Appendix~C).

\medskip
\noindent\emph{(B2) (edge identification, recall).}
There is a natural isomorphism $H^{-1}(\mathcal{R}(F))\cong \varinjlim_t H_1(F^tC_\bullet)$ and $\mathcal{R}(F)\in D^{[-1,0]}$; see Appendix~C.

\subsection*{4.1. Failure lattice and observable vs.\ invisible modes}
We organize collapse failures (cf.\ Chapter~1, §1.4):
\begin{itemize}[leftmargin=1.25em]
  \item \textbf{Type I (Topological):} $\mathrm{PH}_1(F)\neq 0$.
  \item \textbf{Type II (Categorical):} $\Ext^1(\mathcal{R}(F),k)\neq 0$ (tested against $\mathcal{Q}=\{k[0]\}$).
  \item \textbf{Type III (Functorial/[Spec]):} admissibility unstable under a prescribed operation (e.g.\ a given pullback/filtered colimit) at finite level.
  \item \textbf{Type IV (Invisible/tower-level):} all finite layers appear admissible while the limit is not; detected by the tower-sensitivity invariants below.
\end{itemize}
Types~I–II are \emph{observable}; Type~III is \emph{specification-level}; Type~IV is \emph{invisible} at finite layers and requires tower diagnostics.
We emphasize again: \emph{no} global equivalence $\mathrm{PH}_1\Leftrightarrow \Ext^1$ is claimed; only $\mathrm{PH}_1\Rightarrow \Ext^1$ under \textup{(B1)–(B3)} (Chapter~3; Appendix~C).

\begin{remark}[Specification-level failures and functorial calculus]
Type~III uses Chapter~2, §2.3: non-expansiveness, shift–commutation (Lemma~\ref{lem:shift}), and persistence-layer (co)limit/pullback compatibilities in Proposition~\ref{prop:stability}\,(4)(5). Filtered-level statements are \textbf{[Spec]} and used only up to f.q.i. (Appendix~B).
\end{remark}

\begin{remark}[Model towers]
Pure-kernel, pure-cokernel, and mixed toy towers, together with vanishing regimes under constructible filtered colimits, appear in Appendix~D (D.1–D.3). Counterexamples to the converse $\Ext^1{=}0\Rightarrow\mathrm{PH}_1{=}0$ are in D.4 (see also Appendix~C).
\end{remark}

\subsection*{4.2. The tower-sensitivity invariants $\mu_{\mathrm{Collapse}}$, $\nu_{\mathrm{Collapse}}$, and the Defect functor}
\emph{Generic fiber dimension.} As in Chapter~1, §1.4 and Appendix~D, Remark~\ref{rem:D-generic-dim}, for $M\in\Pers^{\mathrm{cons}}_k$ the \emph{generic fiber dimension} is $\mathrm{gdim}(M)=\lim_{t\to+\infty}\dim_k M(t)$; after $\mathbf{T}_\tau$, it equals the multiplicity of the infinite bar $I[0,\infty)$.

\medskip
\noindent\textbf{Comparison map.}
Fix $\tau>0$. Let $\{F_n\}_{n\in\bbN}$ be a directed system in $\mathsf{FiltCh}(k)$ with colimit $F_\infty$. For each degree $i$, define
\[
\phi_{i,\tau}:\ \varinjlim_{n}\ \mathbf{T}_\tau\!\big(\mathbf{P}_i(F_n)\big)\ \longrightarrow\ \mathbf{T}_\tau\!\big(\mathbf{P}_i(F_\infty)\big).
\]

\begin{definition}[Defect objects and tower-sensitivity invariants]\label{def:defect}
In $\Pers^{\mathrm{cons}}_k$ set
\[
\Defect_{i,\tau}^{\ker}:=\ker(\phi_{i,\tau}),\qquad \Defect_{i,\tau}^{\coker}:=\mathrm{coker}(\phi_{i,\tau}).
\]
Then
\[
\mu_{i,\tau}:=\mathrm{gdim}\big(\Defect_{i,\tau}^{\ker}\big),\quad \nu_{i,\tau}:=\mathrm{gdim}\big(\Defect_{i,\tau}^{\coker}\big),\quad
\mu_{\mathrm{Collapse}}:=\sum_i\mu_{i,\tau},\ \nu_{\mathrm{Collapse}}:=\sum_i\nu_{i,\tau}.
\]
Finite homological range and constructibility give $\mu_{\mathrm{Collapse}},\nu_{\mathrm{Collapse}}<\infty$ on bounded $\tau$-windows.
\end{definition}

\begin{proposition}[Generic dimension equals infinite-bar multiplicity]\label{prop:generic-dimension-barcode}
For any morphism $\psi:M\to N$ in $\Pers^{\mathrm{cons}}_k$, $\mathrm{gdim}\ker(\psi)$ (resp.\ $\mathrm{gdim}\mathrm{coker}(\psi)$) equals the multiplicity of $I[0,\infty)$ in the barcode of $\ker(\psi)$ (resp.\ $\mathrm{coker}(\psi)$). The same holds after $\mathbf{T}_\tau$.
\end{proposition}

\begin{remark}[Functoriality, invariance, and calculus]
The maps $\phi_{i,\tau}$ are natural in the tower, independent of filtered representatives, and invariant under cofinal reindexing (Appendix~J). Hence $\Defect_{i,\tau}^{\ker/\coker}$ and $(\mu_{i,\tau},\nu_{i,\tau})$ are invariant under f.q.i. Subadditivity under composition, additivity under finite sums, and cofinal invariance are collected in Appendix~J.
\end{remark}

\begin{definition}[V-distance and control]\label{def:V-metric}
A \emph{V-distance} on towers assigns to each pair of towers a value in $[0,\infty]$ and satisfies: (V1) compatibility with cropping and $\mathbf{T}_\tau$; (V2) non-expansiveness under filtered colimits and cofinal reindexing; (V3) triangle inequality for composable controlled morphisms; (V4) stability under finite sums. We write $V\big((F_\bullet,\phi_{i,\tau}),(\tilde F_\bullet,\tilde\phi_{i,\tau})\big)\le \varepsilon$ to mean the towers and their comparison maps are $\varepsilon$-controlled on the window.
\end{definition}

\begin{proposition}[V-subadditivity/additivity/cofinal invariance and stability]\label{prop:V-subadd}
Fix $i$ and $\tau>0$.
\begin{enumerate}[leftmargin=1.25em]
  \item \emph{Subadditivity.} For $A\xrightarrow{\phi}B\xrightarrow{\psi}C$ (arising from towers via $\mathbf{T}_\tau\mathbf{P}_i$),
  \[
  \mu(\psi\!\circ\!\phi)\le \mu(\phi)+\mu(\psi),\qquad \nu(\psi\!\circ\!\phi)\le \nu(\phi)+\nu(\psi).
  \]
  \item \emph{Additivity on finite sums.} $\mu(\phi\oplus\phi')=\mu(\phi)+\mu(\phi')$ and $\nu(\phi\oplus\phi')=\nu(\phi)+\nu(\phi')$.
  \item \emph{Cofinal invariance.} Cofinal reindexing preserves $\mu_{i,\tau},\nu_{i,\tau}$.
  \item \emph{V-stability on stable bands.} If $B$ is a stable band on a window (Def.~\ref{def:stable-band}) and $V\le\varepsilon$ uniformly on $\tau\in B$ with band margin $>\varepsilon$, then $\mu_{i,\tau},\nu_{i,\tau}$ agree for the two towers on $B$. In particular, $\mu_{i,\tau},\nu_{i,\tau}$ are upper semicontinuous in $V$.
\end{enumerate}
\end{proposition}

\begin{proposition}[Deletion-type monotonicity]\label{prop:deletion-mono}
Along any pipeline consisting only of deletion-type updates and $\varepsilon$-continuations, $(\mu_{i,\tau},\nu_{i,\tau})$ after collapse are nonincreasing in deletion steps and $1$-Lipschitz in $\varepsilon$-continuations (on stable bands). If $(\mu_{i,\tau},\nu_{i,\tau})=(0,0)$ at some stage on a stable band, it remains $(0,0)$ under further deletion-type updates within the same band.
\end{proposition}

\subsection*{4.3. Local PH–Ext equivalence on saturation windows (core)}
\begin{theorem}[Local PH–Ext equivalence on saturation windows]\label{thm:ch4-local-equiv}
Let $F\in\mathsf{FiltCh}(k)$, $W=[u,u')$ right-open, and $\tau>0$. Assume:
\begin{enumerate}[leftmargin=1.25em]
  \item \emph{(Amplitude)} $\mathcal{R}(C_\tau F)\in D^{[-1,0]}(k\text{-mod})$.
  \item \emph{(Saturation/gap)} $W$ is a saturation window at $\tau$ (Chapter~2, Def.~2.6).
  \item \emph{(Tail isomorphism)} $\phi_{1,\tau}$ is an isomorphism on $W$, i.e.\ $(\mu_{\mathrm{Collapse}},\nu_{\mathrm{Collapse}})=(0,0)$ in degree $1$.
\end{enumerate}
Then, on $W$ at threshold $\tau$,
\[
\mathrm{PH}_1\!\big(C_\tau F\big)=0\quad\Longleftrightarrow\quad \Ext^1\!\big(\mathcal{R}(C_\tau F),k\big)=0.
\]
\end{theorem}

\begin{remark}[Core vs.\ global]
The equivalence is \emph{window-local}. Globally, we keep the one-way policy $\mathrm{PH}_1\Rightarrow \Ext^1$ (Chapter~3).
\end{remark}

\paragraph{Definable covers and finite Čech depth.}
\begin{lemma}[Definable Čech finiteness]\label{lem:definable-Cech-finite}
Let $\{X_\alpha\}$ be a definably locally finite cover in a fixed o-minimal expansion with finite overlap multiplicity $\le r$. Then all $(r{+}1)$–fold intersections are empty, the Čech nerve $N(\mathcal{U})$ has dimension $\le r{-}1$, and the Čech complex stops in degree $\le r{-}1$. Consequently, Overlap Glue (after collapse) reduces to finitely many overlap checks of order $\le r{-}1$.
\end{lemma}

\subsection*{4.4. Čech–Ext\texorpdfstring{$^1$}{1} gluing and the Overlap Gate}
\begin{definition}[Čech nerve and local data after collapse]\label{def:cech}
Given a cover $\{X_\alpha\}$ and right-open windows $\{W_\alpha\}$, set for $i$ and $\tau>0$
\[
\mathcal{B}_{\alpha,i}\ :=\ \mathbf{T}_\tau\,\Crop_{W_\alpha}\big(\mathbf{P}_i(F|_{X_\alpha})\big)\in \Pers^{\mathrm{cons}}_k.
\]
On overlaps use restrictions $\mathcal{B}_{\alpha_0\cdots\alpha_p,i}$.
\end{definition}

\begin{definition}[Čech–Ext\textsuperscript{1}–acyclicity (after collapse)]\label{def:cech-acyclic}
The collapsed local data are \emph{Čech–Ext\textsuperscript{1}–acyclic} in degree $1$ if $\Ext^1(\mathcal{R}(C_\tau(F|_{X_{\alpha_0\cdots\alpha_p}})),k)=0$ for all nonempty overlaps and the Čech differentials land in zero $\Ext^1$ on each overlap (equivalently, the $\Ext^1$–row of the Čech $E_1$–page vanishes).
\end{definition}

\begin{theorem}[Overlap Gate with Čech–Ext\textsuperscript{1}: local-to-global gluing]\label{thm:cech-glue}
Fix degree $i=1$, a right-open windowing, and $\tau>0$. Assume:
\begin{enumerate}[leftmargin=1.25em]
  \item \emph{(Local gates)} On each $X_\alpha\times W_\alpha$, B–Gate$^{+}$ passes: $\mathrm{PH}_1(C_\tau F|_{X_\alpha})=0$, $\Ext^1(\mathcal{R}(C_\tau F|_{X_\alpha}),k)=0$, $(\mu,\nu)=(0,0)$, with safety margin $\mathrm{gap}_\tau>\Sigma\delta_\alpha$.
  \item \emph{(Overlap Gate)} For each $(\alpha,\beta)$ with nonempty overlap, the collapsed restrictions agree up to the recorded budget; the safety margin dominates the overlap budget; $(\mu,\nu)=(0,0)$ on overlaps. \emph{All discrepancies are recorded as $\delta_{\alpha\beta}^{\mathrm{alg}}$ in the ledger.}
  \item \emph{(Čech–Ext\textsuperscript{1}–acyclicity)} As in Definition~\ref{def:cech-acyclic}.
\end{enumerate}
Then B–Gate$^{+}$ passes globally on $\bigcup_\alpha X_\alpha\times W_\alpha$ at threshold $\tau$:
\[
\mathrm{PH}_1(C_\tau F)=0,\qquad \Ext^1(\mathcal{R}(C_\tau F),k)=0,\qquad (\mu,\nu)=(0,0),
\]
with a global safety margin equal to the minimum of local margins minus recorded overlap budgets.
\end{theorem}

\begin{remark}[Practical check and logging]
In practice one checks $\Ext^1(\mathcal{R}(C_\tau F|_{X_\alpha}),k)=0$ and the Mayer–Vietoris row ($p=1$), records $\delta^{\mathrm{alg}}$ on overlaps in the $\delta$-ledger, and audits margins against \texttt{run.yaml} (Appendix~G). All checks are B-side (after collapse).
\end{remark}

\subsection*{4.4 bis. Convergence Manager on definable covers (quantale summability)}
\begin{theorem}[Countable DP-cover $\Rightarrow$ global Overlap Glue under $\Sigma\delta<\infty$]\label{thm:dp-sum}
Let $\{W_n\}_{n\ge1}$ be a countable family of right-open \emph{Denef–Pas definable} windows covering a bounded range, with Čech depth $\le K$. Let $\mathsf{V}$ be the fixed commutative quantale for budgets (Chapter~1). If
\[
\sum_{n=1}^{\infty}\ \delta(W_n)\ <\ \infty\qquad\text{in }\mathsf{V},
\]
then, for every point, the total overlap error is bounded by $K\cdot\sum_n\delta(W_n)$ in $\mathsf{V}$, and the Overlap Glue from Theorem~\ref{thm:cech-glue} holds globally. In particular, B–Gate$^{+}$ certificates paste to a global certificate on $\bigcup_n W_n$ whenever per-window margins dominate the local budgets and the above sum is finite.
\end{theorem}

\begin{proof}[Proof sketch]
Denef–Pas definability yields finite Čech depth $K$ on any bounded subrange; quantale subadditivity and right-open MECE refinement control the cumulative budget on each $p$–fold overlap by $\sum_n\delta(W_n)$. Summing over $p\le K{-}1$ gives the bound $K\cdot\sum_n\delta(W_n)$. With margins dominating local budgets, Overlap Gate constraints are satisfied at each overlap level, hence gluing holds globally.
\end{proof}

\subsection*{4.5. Type IV: finite admissibility need not pass to the limit}
\begin{proposition}[Type IV: finite-level admissibility may fail at the limit]\label{prop:type4}
There exists a tower $\{F_n\}$ and $\tau>0$ such that
\[
\forall n:\ \ \mathrm{PH}_1\!\big(C_\tau(F_n)\big)=0\ \text{ and }\ \Ext^1\!\big(\mathcal{R}(C_\tau(F_n)),k\big)=0,\qquad
\mathrm{but}\quad \mathrm{PH}_1\!\big(C_\tau(F_\infty)\big)\neq 0,
\]
hence $\Ext^1\!\big(\mathcal{R}(C_\tau(F_\infty)),k\big)\neq 0$. One can arrange $\mu_{\mathrm{Collapse}}=0$ and $\nu_{\mathrm{Collapse}}>0$ (pure cokernel type).
\end{proposition}

\begin{figure}[t]
\centering
\begin{tikzcd}[row sep=1.0em, column sep=2.0em]
I[0,\tau-\tfrac{1}{1}) \arrow[r, hook] \arrow[dr] &
I[0,\tau-\tfrac{1}{2}) \arrow[r, hook] 
& \cdots \arrow[r, hook] \arrow[dr] &
I[0,\tau-\tfrac{1}{n}) \arrow[r, hook] 
& \cdots \arrow[r, hook] \arrow[dr] &
I[0,\tau) \arrow[r, hook] \arrow[dr] &
I[0,\infty) \\\n& \mathbf{T}_\tau(\cdot)=0 & & \mathbf{T}_\tau(\cdot)=0 & &
\mathbf{T}_\tau(\cdot)=0 & \mathbf{T}_\tau(\cdot)=I[0,\infty)
\end{tikzcd}
\caption{Type~IV intuition (after $\mathbf{T}_\tau$): all finite layers vanish, while the apex produces an infinite bar.}
\label{fig:typeIV-intuition}
\end{figure}

\subsection*{4.6. A natural refinement-limit example (pure cokernel type)}
\begin{example}[Resolution refinement producing a limit infinite bar]
Fix $\tau>0$. Let degree-$1$ persistence at level $n$ have one bar $[0,\tau-\delta_n)$ with $\delta_n\downarrow 0$. Then $\mathbf{T}_\tau(\mathbf{P}_1(F_n))=0$ for all $n$, while $\mathbf{T}_\tau(\mathbf{P}_1(F_\infty))\cong I[0,\infty)$. This realizes a \emph{pure cokernel} Type~IV failure ($\mu_{\mathrm{Collapse}}=0$, $\nu_{\mathrm{Collapse}}>0$).
\end{example}

\begin{remark}[When invisible failure is excluded]
Under the hypotheses of Proposition~\ref{prop:stability}\,(4) (Chapter~2) and the tower conditions of Proposition~\ref{prop:mu-vanishing} (Appendix~J), each $\phi_{i,\tau}$ is an isomorphism, so no Type~IV occurs and finite-level admissibility propagates to the limit.
\end{remark}

\subsection*{4.7. Restart lemma, summability, and pasting of windowed certificates}
\begin{definition}[Per-window safety margin and pipeline budget]\label{def:window-budget}
For a MECE partition $\{[u_k,u_{k+1})\}_k$, threshold $\tau_k>0$, and degree $i$, set
\[
\Sigma\delta_k(i)\ :=\ \sum_{j\in J_k}\!\bigl(\delta^{\mathrm{alg}}_{j}(i,\tau_k)+\delta^{\mathrm{disc}}_{j}(i,\tau_k)+\delta^{\mathrm{meas}}_{j}(i,\tau_k)\bigr),
\]
where $J_k$ indexes the A-side steps before the B-side gate on $W_k$. The \emph{safety margin} $\mathrm{gap}_{\tau_k}>0$ is the admissible slack for the gate on $W_k$ (Chapter~1).
\end{definition}

\begin{lemma}[Restart lemma (window-to-window inheritance)]\label{lem:restart}
If B–Gate$^{+}$ passes on $W_k$ with $\mathrm{gap}_{\tau_k}>\Sigma\delta_k(i)$ and $W_{k+1}$ is reached via deletion-type steps and/or $\varepsilon$-continuations followed by $C_{\tau_{k+1}}$, then there exists $\kappa\in(0,1]$ (depending only on the admissible step class and the $\tau$-adaptation policy) with
\[
\mathrm{gap}_{\tau_{k+1}}\ \ge\ \kappa\ \bigl(\mathrm{gap}_{\tau_k}-\Sigma\delta_k(i)\bigr).
\]
Thus positive margin propagates provided the new budget $\Sigma\delta_{k+1}(i)$ is small enough.
\end{lemma}

\begin{definition}[Summability policy]\label{def:summability}
A run satisfies \emph{summability} if
\[
\sum_k \Sigma\delta_k(i)\ <\ \infty
\]
for the monitored degrees $i$ on a MECE partition. A sufficient design pattern is geometric damping of step sizes and/or continuation strengths, recorded in \texttt{run.yaml} (Appendix~G).
\end{definition}

\begin{theorem}[Restart–Summability (Convergence Manager, v17)]\label{thm:pasting}
If B–Gate$^{+}$ passes on each $W_k$ with $\mathrm{gap}_{\tau_k}>\Sigma\delta_k(i)$, the summability policy holds (Def.~\ref{def:summability}), and Lemma~\ref{lem:restart} applies at each transition, then windowed certificates paste to a global certificate on $\bigcup_k [u_k,u_{k+1})$ for the monitored degrees $i$.
\end{theorem}

\begin{remark}[Manifest integration]
Threshold adaptation, ledger aggregation law (quantale), Čech depth bound, and restart/summability parameters \emph{must} be recorded in \texttt{run.yaml} (Appendix~G). Pipelines read these values to audit Overlap Gate and Convergence Manager decisions.
\end{remark}

\subsection*{4.8. Stable bands and $\tau$-sweeps}
\begin{definition}[Stable band]\label{def:stable-band}
For a fixed window $W$ and degree $i$, a \emph{stable band} $B\subset (0,\infty)$ is a contiguous range such that for all $\tau\in B$ the comparison maps $\phi_{i,\tau}$ are isomorphisms; hence $(\mu_{i,\tau},\nu_{i,\tau})=(0,0)$. A \emph{$\tau$-sweep} is a discrete set $\{\tau_\ell\}$ used to probe $(\mu_{i,\tau_\ell},\nu_{i,\tau_\ell})$; a band is declared stable when a consecutive subarray reports $(0,0)$ and persists under refinement.
\end{definition}

\begin{proposition}[Sparse sweep on gap-protected bands]\label{prop:sparse-sweep}
Let $B\subset (0,\infty)$ be compact and \emph{gap-protected}: there exists $\eta>0$ such that no bar endpoint lies within distance $\eta$ of any $\tau\in B$ (on $W$). If a sweep $\{\tau_\ell\}\subset B$ with mesh $<\eta/3$ yields $(\mu_{i,\tau_\ell},\nu_{i,\tau_\ell})=(0,0)$ for all $\ell$, then $(\mu_{i,\tau},\nu_{i,\tau})=(0,0)$ for all $\tau\in B$; hence $B$ is stable.
\end{proposition}

\begin{theorem}[Stability-band detection via drift threshold]\label{thm:band-detect}
Fix $W$, degree $i$, and constants $\eta>0$ (gap) and $\Delta>0$ (drift threshold). Suppose a sweep $\{\tau_\ell\}$ in an open interval $I$ satisfies:
\begin{enumerate}[leftmargin=1.25em]
  \item $(\mu_{i,\tau_\ell},\nu_{i,\tau_\ell})=(0,0)$ for all sampled $\tau_\ell$;
  \item for every consecutive pair $\tau_\ell<\tau_{\ell+1}$ and any admissible continuation between the two gates, the interleaving drift of $\mathbf{T}_{\tau_\ell}\mathbf{P}_i$ to $\mathbf{T}_{\tau_{\ell+1}}\mathbf{P}_i$ is $<\Delta$ (measured in $d_{\mathrm{int}}$);
  \item $\min\{\tau_{\ell+1}-\tau_\ell\}\ge 3\eta$ and the band is $\eta$-gap-protected.
\end{enumerate}
Then there exists a nonempty open subinterval $B\subset I$ on which $\phi_{i,\tau}$ is an isomorphism and $(\mu_{i,\tau},\nu_{i,\tau})=(0,0)$ for all $\tau\in B$.
\end{theorem}

\begin{proof}[Proof sketch]
Gap protection implies local constancy of $\mathbf{T}_\tau$ on subintervals; small drift ensures no creation of $I[0,\infty)$ in kernels/cokernels between samples. Compactness of $I$ and mesh $\ge3\eta$ yield an open subinterval $B$ where $\mathbf{T}_\tau$ and the comparison maps are constant, hence $(\mu_{i,\tau},\nu_{i,\tau})=(0,0)$.
\end{proof}

\subsection*{4.9. Summary}
The failure lattice separates observable (Types~I–II), specification-level (Type~III), and invisible tower effects (Type~IV). The Defect objects $\Defect_{i,\tau}^{\ker/\coker}$ and the invariants $(\mu_{\mathrm{Collapse}},\nu_{\mathrm{Collapse}})$ provide principled tower diagnostics with subadditivity, additivity, and cofinal invariance. The \emph{only} core bridge is $\mathrm{PH}_1\Rightarrow \Ext^1$ in $D^{\mathrm{b}}(k\text{-mod})$; nevertheless, on \emph{saturation windows} with tail isomorphism we obtain a window-local equivalence $\mathrm{PH}_1(C_\tau F)=0 \Leftrightarrow \Ext^1(\mathcal{R}(C_\tau F),k)=0$ (Theorem~\ref{thm:ch4-local-equiv}). For gluing, the Overlap Gate with Čech–Ext$^1$–acyclicity gives local-to-global propagation (Theorem~\ref{thm:cech-glue}); the Convergence Manager (Theorems~\ref{thm:dp-sum} and \ref{thm:pasting}) ensures pasting under quantale-summable budgets on definable covers. Stable bands (Definition~\ref{def:stable-band}), sparse sweeps (Proposition~\ref{prop:sparse-sweep}), and the drift-threshold detector (Theorem~\ref{thm:band-detect}) guide $\tau$-selection for tower audits. Our $\mu_{\mathrm{Collapse}}$ is a persistence–theoretic audit invariant (Remark~\ref{rem:Iwasawa-mu-vs-collapse-mu}); it is distinct from classical Iwasawa $\mu$.

\begin{remark}[Iwasawa $\mu$ vs.\ $\mu_{\mathrm{Collapse}}$]\label{rem:Iwasawa-mu-vs-collapse-mu}
Both quantify \emph{defect accumulation along towers}, but in different categories and with different laws: $\mu_{\mathrm{Collapse}}$ depends on degree and threshold, is window-local, and is additive on sums/subadditive under composition; Iwasawa $\mu$ is prime- and tower-global via characteristic ideals. No direct implication is intended.
\end{remark}

\subsection*{4.10. Mandatory tests (operational)}
The following tests are part of the core suite; all are evaluated \emph{after collapse} and logged with budgets/margins.
\begin{description}[leftmargin=1.25em]
\item[\textbf{T-Lipschitz-AfterCollapse}.] Verify that for each monitored degree $i$ and adjacent pipeline steps, $\mathbf{T}_\tau$ preserves interleavings: $d_{\mathrm{int}}(\mathbf{T}_\tau\mathbf{P}_i(F),\mathbf{T}_\tau\mathbf{P}_i(G))\le d_{\mathrm{int}}(\mathbf{P}_i(F),\mathbf{P}_i(G))$ (Chapter~2, Lemma~\ref{lem:shift}).
\item[\textbf{T-Countable-Cover}.] On a Denef–Pas definable right-open cover, record the Čech depth bound $K$ (Lemma~\ref{lem:definable-Cech-finite}) and verify finiteness of overlap indices used by the Overlap Gate.
\item[\textbf{T-Delta-Sum-Converges}.] Check quantale-summability $\sum_n\delta(W_n)<\infty$ for the run’s cover and thresholds; then apply Theorem~\ref{thm:dp-sum} to assert global Overlap Glue under recorded margins.
\end{description}
All three tests are parameterized by \texttt{run.yaml} keys \texttt{quantale/*}, \texttt{diagnostics/*}, \texttt{gates/*}, \texttt{cover/*}, and \texttt{delta\_budget/*} (Appendix~G).



% NOTE (out-of-band): Some previously uploaded files appear to have expired. If you want me to load them again, please re-upload them. — This line is a comment and harmless for pdflatex.

\section{Chapter 5: Functoriality, Set-Theoretic Coherence, and Formalization Specifications (Proof/Spec)}
\addcontentsline{toc}{section}{Functoriality, Set-Theoretic Coherence, and Formalization Specifications (Proof/Spec)}

All adjunction and (co)limit statements in this chapter are made in the \emph{implementable range} and inside
\(\mathrm{Ho}(\mathsf{FiltCh}(k))\), \emph{up to filtered quasi\hyp isomorphism} (Appendix~B). Equalities are asserted \emph{at the persistence layer}. We retain the standing conventions of Chapters~1–4: constructible range, field coefficients, \(t\)\hyp exact realization, and the after\hyp truncation policy. Endpoint conventions and infinite bars are as in Chapter~2, Remark~\ref{rk:2-endpoints}. \emph{Monotonicity claims apply only to deletion\hyp type updates; inclusion\hyp type updates are stability\hyp only} (Appendix~E). All statements are \emph{windowed} and gates are evaluated \emph{after collapse} (B\hyp side single layer).

\subsection*{5.1. Exactness, (Co)Limit Behavior, and a Right\hyp Adjoint Collapse in the Implementable Range (up to f.q.i.)}

Let \(k\) be a field. Recall: \(\mathsf{FiltCh}(k)\) is the category of finite\hyp type (constructible) filtered chain complexes; \(\mathbf{P}_i\) is degreewise persistence; \(\mathbf{T}_\tau\) is the exact truncation deleting all bars of length \(\le\tau\) (Chapter~2, §2.2); \(C_\tau\) is any filtered lift of \(\mathbf{T}_\tau\) (Chapter~2, §§2.2–2.3; always \emph{up to f.q.i.}); and \(\mathcal{R}:\mathsf{FiltCh}(k)\to D^{\mathrm{b}}(k\text{-mod})\) is \(t\)\hyp exact. We also keep the minimal test family \(\mathcal{Q}=\{k[0]\}\).

\paragraph{Persistence\hyp level (reflective) adjunction.}
Let \(\mathsf{Pers}^{\mathrm{cons}}_k\) be the abelian category of constructible \(k\)\hyp persistence modules and \(\mathsf{Pers}^{\mathrm{cons}}_{k,\tau\text{-tf}}\subset \mathsf{Pers}^{\mathrm{cons}}_k\) the full subcategory of \(\tau\)\hyp torsion\hyp free objects (no composition factors of length \(\le\tau\)). As in Chapter~2, §§2.2–2.3:
\begin{itemize}
  \item \(\mathsf{E}_\tau\subset \mathsf{Pers}^{\mathrm{cons}}_k\) (generated by interval modules of length \(\le\tau\)) is hereditary Serre (localizing).
  \item The reflector
  \[
  \mathbf{T}_\tau:\ \mathsf{Pers}^{\mathrm{cons}}_k\longrightarrow \mathsf{Pers}^{\mathrm{cons}}_{k,\tau\text{-tf}}
  \]
  is exact and exhibits a \emph{reflective} adjunction \(\mathbf{T}_\tau\dashv \iota_\tau\) with the inclusion
  \(\iota_\tau:\mathsf{Pers}^{\mathrm{cons}}_{k,\tau\text{-tf}}\hookrightarrow \mathsf{Pers}^{\mathrm{cons}}_k\).
  Consequently, \(\mathbf{T}_\tau\) preserves finite limits and colimits and is \(1\)\hyp Lipschitz for \(d_{\mathrm{int}}\)
  (Lemma~\ref{lem:shift}, Proposition~\ref{prop:stability}(1),(3)).
\end{itemize}

\paragraph{Filtered\hyp complex level (operational coreflection; implementable range, up to f.q.i.).}
Define the full subcategory
\[
\mathsf{S}_\tau\ :=\ \Big\{\,F\in \mathsf{FiltCh}(k)\ \Big|\ \forall i,\ \mathbf{P}_i(F)\ \text{is \(\tau\)\hyp torsion\hyp free and }\ \Ext^1\!\big(\mathcal{R}(F),k\big)=0\,\Big\},
\]
and let \(\mathsf{S}_\tau^{\mathrm{h}}\) be its image in \(\mathrm{Ho}(\mathsf{FiltCh}(k))\).
For later comparison, recall the “trivial\hyp at\hyp \(\tau\)” gate
\[
\mathsf{Triv}_\tau\ :=\ \Big\{\,F\ \Big|\ \mathrm{PH}_1\big(C_\tau(F)\big)=0\ \ \text{and}\ \ \mathrm{Ext}^1\!\big(\mathcal{R}(F),k\big)=0\,\Big\},\qquad \mathsf{Triv}_\tau^{\mathrm{h}}\subset \mathsf{S}_\tau^{\mathrm{h}}.
\]

\begin{proposition}[Operational collapse as a right adjoint (implementable range; up to f.q.i.; Proof/Spec)]\label{prop:operational-coreflection}
Assume \textup{(B1)–(B3)} (Chapter~3) and the lifting–coherence hypothesis (Appendix~B).
Then there exists a functor
\[
\mathsf{C}_\tau^{\mathrm{comb}}:\ \mathrm{Ho}(\mathsf{FiltCh}(k))\longrightarrow \mathsf{S}_\tau^{\mathrm{h}}
\]
and a natural transformation \(\eta:\mathrm{Id}\Rightarrow \iota\,\mathsf{C}_\tau^{\mathrm{comb}}\) (with \(\iota:\mathsf{S}_\tau^{\mathrm{h}}\hookrightarrow \mathrm{Ho}(\mathsf{FiltCh}(k))\) the inclusion) such that, \emph{in this regime},
\begin{enumerate}
  \item \textup{(Adjunction)} \(\mathsf{C}_\tau^{\mathrm{comb}}\) is right adjoint to \(\iota\):
  \[
  \mathrm{Hom}\!\big(\iota(G),F\big)\ \cong\ \mathrm{Hom}\!\big(G,\mathsf{C}_\tau^{\mathrm{comb}}(F)\big)\qquad(G\in\mathsf{S}_\tau^{\mathrm{h}}).
  \]
  \item \textup{(Compatibility; persistence layer)} For each \(i\), \(\mathbf{P}_i\!\big(\mathsf{C}_\tau^{\mathrm{comb}}(F)\big)\cong \mathbf{T}_\tau\!\big(\mathbf{P}_i(F)\big)\) in \(\mathsf{Pers}^{\mathrm{cons}}_k\).
  \item \textup{(Compatibility; realization layer)} \(\mathcal{R}\!\big(\mathsf{C}_\tau^{\mathrm{comb}}(F)\big)\cong \tau_{\ge 0}\,\mathcal{R}(F)\) in \(D^{\mathrm{b}}(k\text{-mod})\).
  \item \textup{(Soundness)} \(\mathsf{C}_\tau^{\mathrm{comb}}(F)\in \mathsf{S}_\tau^{\mathrm{h}}\) for all \(F\), and
  \[
  \mathsf{C}_\tau^{\mathrm{comb}}(F)\in \mathsf{Triv}_\tau^{\mathrm{h}}\quad\Longleftrightarrow\quad \mathbf{T}_\tau\!\big(\mathbf{P}_1(F)\big)=0\ \ \big(\text{i.e.\ } \mathrm{PH}_1(C_\tau(F))=0\big).
  \]
\end{enumerate}
\emph{All equalities above are asserted at the persistence layer, and all filtered\hyp complex statements are in \(\mathrm{Ho}(\mathsf{FiltCh}(k))\) up to f.q.i.\ only.}
\end{proposition}

\begin{declaration}[AWFS for collapse (operational, up to f.q.i.)]\label{dec:awfs}
Fix \(\tau>0\). There exist endofunctors
\[
L_\tau,\ R_\tau:\ \mathrm{Ho}(\mathsf{FiltCh}(k))\longrightarrow \mathrm{Ho}(\mathsf{FiltCh}(k))
\]
together with structure maps
\[
\varepsilon:\ L_\tau\Rightarrow \mathrm{Id},\quad \delta:\ L_\tau\Rightarrow L_\tau L_\tau,\qquad\n\eta:\ \mathrm{Id}\Rightarrow R_\tau,\quad \mu:\ R_\tau R_\tau\Rightarrow R_\tau,
\]
and a natural \(2\)\hyp cell (distributive law)
\[
\lambda:\ L_\tau R_\tau\ \Rightarrow\ R_\tau L_\tau
\]
such that:
\begin{enumerate}
  \item \(R_\tau=\iota\circ \mathsf{C}_\tau^{\mathrm{comb}}\) is an idempotent comonad up to f.q.i.\ (Proposition~\ref{prop:comonad}); dually, \(L_\tau\) is an idempotent monad up to f.q.i.\ modeling admissible \emph{pre\hyp processing} (e.g.\ normalization, calibration) that is \(1\)\hyp Lipschitz at persistence and preserves constructibility.
  \item \((L_\tau,R_\tau,\lambda)\) forms an \emph{operational algebraic weak factorization system} (AWFS) up to f.q.i.: on the pipeline class of maps (Appendix~E), every morphism functorially factors as an \(L_\tau\)\hyp step followed by an \(R_\tau\)\hyp step; the AWFS triangles and coherence hold up to f.q.i.\ and are quantitatively controlled in the \(\delta\)\hyp ledger (Specification~\ref{spec:delta-ledger}).
\end{enumerate}
At the persistence layer, the left/right structures are strict: \(L_\tau\) is exact and \(1\)\hyp Lipschitz; \(R_\tau\) realizes \(\mathbf{M}_\tau:=\iota_\tau\circ \mathbf{T}_\tau\).
\end{declaration}

\begin{corollary}[Limit/(co)limit behavior of \(\mathsf{C}_\tau^{\mathrm{comb}}\) (persistence layer)]
Under Proposition~\ref{prop:stability}(4),(5), \(\mathsf{C}_\tau^{\mathrm{comb}}\) preserves finite limits
(in particular, finite pullbacks) \emph{up to f.q.i.} at the filtered\hyp complex level; at the
\emph{persistence layer} one has, for every degree \(i\),
\[
\mathbf{P}_i\!\big(\mathsf{C}_\tau^{\mathrm{comb}}(\varinjlim_\Lambda F_\lambda)\big)\ \cong\ \varinjlim_\Lambda\, \mathbf{P}_i\!\big(\mathsf{C}_\tau^{\mathrm{comb}}(F_\lambda)\big).
\]
Hence \(\mathsf{C}_\tau^{\mathrm{comb}}\) is \(1\)\hyp Lipschitz at the persistence layer and inherits exactness via \(\mathbf{T}_\tau\) (Chapter~2, §2.3). All equalities are stated at the persistence layer; no additional metric statement is made in \(\mathrm{Ho}(\mathsf{FiltCh}(k))\) beyond up to f.q.i.\ compatibility.
\end{corollary}

\begin{remark}[Scope of colimit claims]
Since \(\tau_{\ge 0}\) is a right adjoint, it need not commute with filtered colimits; all colimit statements are therefore restricted to the \emph{persistence layer} (via \(\mathbf{T}_\tau\) and \(\mathbf{P}_i\)).
\end{remark}

\subsection*{5.1 bis. Idempotent monad/comonad and AWFS triangles: strictness at persistence, up to f.q.i.\ on \(\mathrm{Ho}\)}

\begin{proposition}[Idempotent monad at the persistence layer]\label{prop:monad}
Let \(\mathbf{M}_\tau:=\iota_\tau\circ \mathbf{T}_\tau:\mathsf{Pers}^{\mathrm{cons}}_k\to \mathsf{Pers}^{\mathrm{cons}}_k\). With unit \(\eta:\mathrm{Id}\Rightarrow \mathbf{M}_\tau\) and multiplication induced by the counit on \(\mathsf{Pers}^{\mathrm{cons}}_{k,\tau\text{-tf}}\), \((\mathbf{M}_\tau,\eta,\mu)\) is an \emph{idempotent monad}. It is exact and \(1\)\hyp Lipschitz (Appendix~A; see also Appendix~K).
\end{proposition}

\begin{proposition}[Idempotent comonad on Ho up to f.q.i.]\label{prop:comonad}
Let \(\mathbf{G}_{\tau}:=\iota\circ \mathsf{C}_\tau^{\mathrm{comb}}:\mathrm{Ho}(\mathsf{FiltCh}(k))\to \mathrm{Ho}(\mathsf{FiltCh}(k))\), 
with counit \(\varepsilon:\mathbf{G}_{\tau}\Rightarrow \mathrm{Id}\) and comultiplication \(\delta\) induced by the unit of the adjunction in Proposition~\ref{prop:operational-coreflection}. 
Then \((\mathbf{G}_\tau,\varepsilon,\delta)\) is an idempotent comonad in \(\mathrm{Ho}\) up to f.q.i.; moreover \(\mathbf{P}_i(\mathbf{G}_\tau F)\cong \mathbf{M}_\tau(\mathbf{P}_i F)\) naturally in \(i,F\).
\end{proposition}

\begin{theorem}[AWFS triangles and order control]\label{thm:awfs-triangle}
In the setting of Declaration~\ref{dec:awfs}, the following hold:
\begin{enumerate}
  \item \textup{(Idempotence)} \(C_\tau\circ C_\tau\ \simeq\ C_\tau\) in \(\mathrm{Ho}\) up to f.q.i., and \(\mathbf{T}_\tau\circ \mathbf{T}_\tau=\mathbf{T}_\tau\) strictly at persistence.
  \item \textup{(Distributive \(2\)\hyp cell)} There is a natural \(2\)\hyp cell \(\lambda:L_\tau R_\tau\Rightarrow R_\tau L_\tau\) whose persistence\hyp level image is an isomorphism; any non\hyp commutation on \(\mathrm{Ho}\) is bounded by \(\delta^{\mathrm{alg}}(\tau)\) and recorded in the \(\delta\)\hyp ledger (Definition~\ref{def:delta-2cell}, Specification~\ref{spec:delta-ledger}).
  \item \textup{(Triangle identities)} The AWFS triangles for \((L_\tau,R_\tau)\) hold up to f.q.i.; defects compose according to the Quantale structure of the \(\delta\)\hyp ledger (Specification~\ref{spec:delta-ledger}).
\end{enumerate}
\end{theorem}

\begin{remark}[Strictness vs.\ implementability]
The monad is \emph{strict} in \(\mathsf{Pers}^{\mathrm{cons}}_k\); the comonad and the AWFS are \emph{operational} in \(\mathrm{Ho}\) up to f.q.i.\ only. This realizes “strict at persistence, up to f.q.i.\ after lifting”.
\end{remark}

\subsection*{5.1 ter. Product–ledger Quantale and tolerance profile \texorpdfstring{\(\boldsymbol{\eta}\)}{η} (P7)}

We standardize the \emph{product–ledger} for multi\hyp axis budgeting.

\begin{definition}[Product–ledger quantale and tolerance]\label{def:product-ledger}
Fix axes \(\mathcal{A}=\{\text{alg},\text{disc},\text{meas}\}\). For each \(a\in\mathcal{A}\), let \((\mathsf{Q}_a,\otimes_a,\mathbf{1}_a,\le_a)\) be a commutative unital quantale (e.g.\ \(\overline{\mathbb{R}}_{\ge0},+\!,0,\le\)). Define the \emph{product–ledger quantale}
\[
\mathsf{Q}\ :=\ \prod_{a\in\mathcal{A}}\mathsf{Q}_a,\qquad 
(\delta_a)_{a}\ \otimes\ (\delta'_a)_{a}\ :=\ (\delta_a\otimes_a \delta'_a)_{a},\qquad
(\delta_a)_a\ \le\ (\delta'_a)_a\iff \forall a:\ \delta_a\le_a \delta'_a.
\]
A \emph{tolerance profile} is a vector \(\eta=(\eta_a)_{a\in\mathcal{A}}\in\mathsf{Q}\).
Acceptance on a window uses the \emph{componentwise} test \((\Sigma\delta)_a\le_a \eta_a\) for all \(a\).
When a scalar guard is needed (e.g.\ for logging), use any fixed \emph{monotone scalarization} \(\norm{-}:\mathsf{Q}\to \overline{\mathbb{R}}_{\ge0}\) (e.g.\ \(\ell^\infty\) or \(\ell^1\)) recorded in \texttt{run.yaml}.
\end{definition}

\begin{theorem}[AWFS \(2\)\hyp cell additivity on the product–ledger (P7)]\label{thm:2cell-additivity}
Let \(\epsilon_{i,\tau}:\Mirror\!\circ C_\tau \Rightarrow C_\tau\!\circ\Mirror\) be the natural \(2\)\hyp cell with bound \(\delta_{i,\tau}\in\mathsf{Q}\) (Definition~\ref{def:delta-2cell}). Then:
\begin{enumerate}
  \item \textbf{Vertical composition (series).} For composable \(2\)\hyp cells with bounds \(\delta,\delta'\in\mathsf{Q}\), the composite has bound \(\delta\otimes \delta'\) (componentwise aggregation).
  \item \textbf{Horizontal composition (parallel).} For independent branches with bounds \(\delta,\delta'\in\mathsf{Q}\), the product diagram has bound \(\delta\otimes \delta'\).
  \item \textbf{Tolerance filter.} If \(\delta\le \eta\) and \(\delta'\le \eta\), then any finite composite satisfies \(\bigotimes_j \delta^{(j)}\le \eta\); hence acceptance is preserved under pipeline composition.
\end{enumerate}
\emph{Proof sketch.} (1)–(2) are the monoidal laws in \(\mathsf{Q}=\prod_a \mathsf{Q}_a\). (3) follows by componentwise monotonicity and associativity of \(\otimes_a\).
\end{theorem}

\begin{remark}[Manifest fields (required)]
\texttt{run.yaml} must record: \texttt{ledger.axes}, \texttt{quantale.op} (per axis), \texttt{tolerance.eta} (vector), and \texttt{aggregation.scalarization}. This ensures reproducible acceptance with explicit \(\eta\) and scalar guard.
\end{remark}

\subsection*{5.2. [Spec] Coq/Lean Contracts: Stability, (Co)Limits, Bridge, AWFS, and Product–ledger \(\delta\)\hyp Commutation}

\noindent Identifiers are indicative; concrete names may follow local conventions (e.g.\ Lean/mathlib namespaces). Appendix~F lists one naming scheme. All equalities are asserted at the persistence layer; filtered\hyp level objects are considered in \(\mathrm{Ho}\) up to f.q.i.

\begin{specification}[Persistence truncation]\label{spec:pers-Ttau}
\begin{itemize}
  \item \texttt{pers\_Ttau\_exact}: exactness on short exact sequences.
  \item \texttt{pers\_Ttau\_lipschitz}: \(d_{\mathrm{int}}(\mathbf{T}_\tau M,\mathbf{T}_\tau N)\le d_{\mathrm{int}}(M,N)\).
  \item \texttt{pers\_Ttau\_pres\_colim\_pullback}: filtered colimits and finite limits preserved (constructible range).
  \item \texttt{pers\_Ttau\_compose}: \(\mathbf{T}_\tau\circ \mathbf{T}_\sigma=\mathbf{T}_{\max\{\tau,\sigma\}}\).
\end{itemize}
\end{specification}

\begin{specification}[Filtered\hyp complex level]\label{spec:filtered-level}
\begin{itemize}
  \item \texttt{Ctau\_lift}: \(\mathbf{P}_i(C_\tau F)\cong \mathbf{T}_\tau(\mathbf{P}_i(F))\).
  \item \texttt{Ctau\_colim}: \(\mathbf{P}_i(C_\tau(\varinjlim F_\lambda))\cong \varinjlim \mathbf{P}_i(C_\tau(F_\lambda))\).
  \item \texttt{Ctau\_pullback}: \(\mathbf{P}_i(C_\tau(F\times_H G))\cong \mathbf{P}_i(C_\tau(F)\times_{C_\tau(H)}C_\tau(G))\).
\end{itemize}
\end{specification}

\begin{specification}[AWFS contracts]\label{spec:awfs}
\begin{itemize}
  \item \texttt{awfs\_R\_comonad}: \(R_\tau\) is an idempotent comonad up to f.q.i.; persistence\hyp level image equals \(\mathbf{M}_\tau\).
  \item \texttt{awfs\_L\_monad}: \(L_\tau\) is an idempotent monad up to f.q.i.; exact and \(1\)\hyp Lipschitz at persistence.
  \item \texttt{awfs\_dist\_law}: a natural \(2\)\hyp cell \(\lambda:L_\tau R_\tau\Rightarrow R_\tau L_\tau\) with \(\delta\)\hyp bound.
  \item \texttt{awfs\_triangle}: AWFS triangle identities hold up to f.q.i.; defects aggregate in the product–ledger \(\mathsf{Q}\).
  \item \texttt{awfs\_factorization}: pipeline morphisms factor functorially as \(L_\tau\)\hyp step then \(R_\tau\)\hyp step.
\end{itemize}
\end{specification}

\begin{specification}[\(\delta\)\hyp ledger: product–ledger and tolerance]\label{spec:delta-ledger}
Fix \(\mathsf{Q}=\prod_{a\in\mathcal{A}}\mathsf{Q}_a\) as in Definition~\ref{def:product-ledger} and a tolerance profile \(\eta\in\mathsf{Q}\).
\begin{itemize}
  \item \texttt{delta\_quantale\_product}: bounds live in \(\mathsf{Q}\); aggregation is componentwise \(\otimes\).
  \item \texttt{delta\_2cell\_mirror\_collapse}: natural \(2\)\hyp cell \(\epsilon_{i,\tau}\) with bound \(\delta_{i,\tau}\in\mathsf{Q}\).
  \item \texttt{delta\_compose\_vertical/horizontal}: vertical and horizontal compositions aggregate via \(\otimes\) (Theorem~\ref{thm:2cell-additivity}).
  \item \texttt{delta\_tolerance}: acceptance if \((\Sigma\delta)_a\le_a \eta_a\) for all axes; optional scalarization \(\norm{-}\) is monotone.
  \item \texttt{delta\_lipschitz\_post}: any \(1\)\hyp Lipschitz post\hyp processing is \(\mathsf{Q}\)\hyp monotone (non\hyp increasing).
\end{itemize}
\end{specification}

\begin{specification}[Bridge and admissibility]\label{spec:bridge}
\begin{itemize}
  \item \texttt{PH1\_to\_Ext1\_under\_B}: under (B1)–(B3), \(\mathrm{PH}_1(F)=0\Rightarrow \Ext^1(\mathcal{R}(F),k)=0\).
  \item \texttt{admissible\_robust\_eps}: if \(\mathrm{PH}_1(C_\varepsilon F)=0\), then \(\Ext^1(\mathcal{R}(C_\varepsilon F),k)=0\).
\end{itemize}
\end{specification}

\begin{specification}[Tower diagnostics]\label{spec:mu-nu}
\begin{itemize}
  \item \texttt{mu\_def}: \(\mu^{\,i}=\mathrm{gdim}\,\ker(\phi_{i,\tau})\), \ \texttt{nu\_def}: \(\nu^{\,i}=\mathrm{gdim}\,\mathrm{coker}(\phi_{i,\tau})\).
  \item \texttt{mu\_nu\_finite}: finiteness (bounded degrees).
  \item \texttt{mu\_nu\_vanish}: under constructible filtered colimits, \(\phi_{i,\tau}\) is an isomorphism; hence \(\muc=\nuc=0\).
\end{itemize}
\end{specification}

\begin{specification}[Combined collapse coreflection]\label{spec:combined}
\begin{itemize}
  \item \texttt{Ccomb\_adjunction}: inclusion \(\iota:\mathsf{S}_\tau^{\mathrm{h}}\hookrightarrow\mathrm{Ho}(\mathsf{FiltCh}(k))\) has right adjoint \(\mathsf{C}_\tau^{\mathrm{comb}}\) (implementable range).
  \item \texttt{Ccomb\_compat}: \(\mathbf{P}_i(\mathsf{C}_\tau^{\mathrm{comb}}F)\cong \mathbf{T}_\tau(\mathbf{P}_iF)\) and \(\mathcal{R}(\mathsf{C}_\tau^{\mathrm{comb}}F)\cong \tau_{\ge 0}\mathcal{R}(F)\).
  \item \texttt{Ccomb\_lipschitz\_pers}: \(d_{\mathrm{int}}(\mathbf{P}_i(\mathsf{C}_\tau^{\mathrm{comb}}F),\mathbf{P}_i(\mathsf{C}_\tau^{\mathrm{comb}}G))\le d_{\mathrm{int}}(\mathbf{P}_i(F),\mathbf{P}_i(G))\).
\end{itemize}
\end{specification}

\subsection*{5.3. Minimal Foundations: ZFC and Dependent Type Theory}

\paragraph{ZFC assumptions (minimal).}
(S1)–(S5) as in the draft hold; in particular, \(\mathsf{Pers}^{\mathrm{cons}}_k\) is abelian and admits Serre localization, and \(\mathsf{FiltCh}(k)\) supplies bounded, finite\hyp type models.

\paragraph{Dependent type theory (Coq/Lean).}
(T1)–(T6) as in the draft hold; in particular, a relative\hyp category treatment of \(\mathrm{Ho}(\mathsf{FiltCh}(k))\) supports right adjoints up to f.q.i.

\paragraph{Set\hyp theoretic coherence.}
At persistence level, \(\mathbf{T}_\tau\dashv \iota_\tau\) is a reflection; at realization level, \(\tau_{\ge 0}\) is the right adjoint truncation. Proposition~\ref{prop:operational-coreflection} aggregates them into a right adjoint collapse in \(\mathrm{Ho}\) (up to f.q.i.), consistent with stability and (co)limit behavior in Chapter~2.

\subsection*{5.4. Declarations for External Realizations and Operational Recipe (\textbf{[Spec]})}

\begin{declaration}[Spec–Derived realizations]\label{spec:derived-geo}
We may use \(\mathcal{R}_{\mathrm{coh}}:\mathsf{FiltCh}(k)\to D^{\mathrm{b}}\mathrm{Coh}(X)\) or
\(\mathcal{R}_{\acute{e}t}:\mathsf{FiltCh}(k)\to D^{\mathrm{b}}_{\mathrm{c}}(X_{\acute{e}t},\Lambda)\) with \emph{field} \(\Lambda\) as specifications.
Projection formula and base change are invoked as in Appendix~N. The bridge \(\mathrm{PH}_1\Rightarrow\Ext^1\) is proved only in \(D^{\mathrm{b}}(k\text{-mod})\); external realizations do not extend the proven bridge.
\end{declaration}

\begin{declaration}[Spec–Operational recipe]\label{spec:operational-coreflection}
We operate with
\[
F\ \longmapsto\ \big(C_\tau(F)\ \text{at persistence}\big)\quad\text{and}\quad\big(\tau_{\ge 0}\mathcal{R}(F)\ \text{at realization}\big),
\]
using right\hyp adjoint phrasing only at [Spec]; coherence and limits are in Appendix~B. When present, admissible \(L_\tau\)\hyp preprocessing precedes collapse; the \(\delta\)\hyp ledger aggregates both sides.
\end{declaration}

\subsection*{5.5. \(\delta\)\hyp Budget Naturalities and the Pipeline Error Budget}

\begin{definition}[Natural \(2\)\hyp cell and \(\delta\)\hyp ledger]\label{def:delta-2cell}
For each \(\tau>0\) and \(i\), a natural \(2\)\hyp cell \(\epsilon_{i,\tau}:\Mirror\circ C_\tau\Rightarrow C_\tau\circ\Mirror\) carries a bound \(\delta_{i,\tau}\in\mathsf{Q}\) (Specification~\ref{spec:delta-ledger}):
\[
d_{\mathrm{int}}\!\Big(\mathbf{T}_\tau \mathbf{P}_i(\Mirror(C_\tau F)),\ \mathbf{T}_\tau \mathbf{P}_i(C_\tau(\Mirror F))\Big)\ \le\ \norm{\delta_{i,\tau}},
\]
monotone in any chosen scalarization \(\norm{-}\). Decompose \(\delta=(\delta_a)_{a\in\mathcal{A}}\) and record in the product–ledger.
\end{definition}

\begin{proposition}[Pipeline error budget]\label{prop:pipeline-budget}
Let \(U_m,\dots,U_1\) be A\hyp side steps with collapses \(C_{\tau_j}\) and bounds \(\delta_j(i,\tau_j)\in\mathsf{Q}\). Then for fixed \(\tau\),
\[
d_{\mathrm{int}}\!\Big(\mathbf{T}_\tau \mathbf{P}_i(\Mirror(C_{\tau_m}U_m\!\cdots C_{\tau_1}U_1F)),\ \mathbf{T}_\tau \mathbf{P}_i(C_{\tau_m}U_m\!\cdots C_{\tau_1}U_1\Mirror F)\Big)\ \le\ \norm{\bigotimes_{j=1}^m \delta_j(i,\tau_j)},
\]
and any \(1\)\hyp Lipschitz post\hyp processing is non\hyp increasing for the bound.
\end{proposition}

\begin{remark}[Safety margin with tolerance]
Per window \(W\) and \(\tau\), B\hyp Gate\({}^{+}\) uses a tolerance profile \(\eta\) and the product–ledger sum \(\Sigma\delta\); accept if \((\Sigma\delta)_a\le_a \eta_a\) on all axes (and, optionally, \(\norm{\Sigma\delta}\le \norm{\eta}\)).
\end{remark}

\subsection*{5.6. Commutable Torsion: Adoption Policy, A/B Soft\hyp Commuting Priorities, and Accounting}

\begin{definition}[Torsion reflectors and nesting]\label{def:torsion-reflectors}
Let \(T_A,T_B\) be exact reflectors on \(\mathsf{Pers}^{\mathrm{cons}}_k\) from hereditary Serre subcategories \(E_A,E_B\). Say \(T_A,T_B\) are \emph{nested} if \(E_A\subseteq E_B\) or \(E_B\subseteq E_A\).
\end{definition}

\begin{proposition}[Order independence under nesting]\label{prop:nested-commute}
If nested, then \(T_A\circ T_B=T_B\circ T_A=T_{A\vee B}\), where \(E_{A\vee B}\) is the Serre subcategory generated by \(E_A\cup E_B\). In particular, for length thresholds, \(\mathbf{T}_\tau\circ \mathbf{T}_\sigma=\mathbf{T}_{\max\{\tau,\sigma\}}\).
\end{proposition}

\begin{definition}[A/B soft\hyp commuting: priorities, fallback, and \(\Delta_{\mathrm{comm}}\) accounting]\label{def:soft-commute}
Given \(T_A,T_B\) not known to be nested:
\begin{enumerate}
  \item \textbf{Priority rule.} Prefer \emph{index\hyp local} reflectors first (e.g.\ cropping/birth\hyp window) then \emph{length\hyp type} reflectors; if incomparable, choose the order minimizing a pilot bound of \(\Delta_{\mathrm{comm}}:=d_{\mathrm{int}}(T_AT_BM,T_BT_AM)\) on a calibration set.
  \item \textbf{Tolerance test.} If \(\Delta_{\mathrm{comm}}\le \norm{\eta}\) (or componentwise \(\Delta_{\mathrm{comm}}\le \eta\)), accept soft\hyp commuting and do not reorder; else \emph{fix} the priority order and proceed deterministically.
  \item \textbf{Ledger rule.} Log \(\Delta_{\mathrm{comm}}\) as \(\delta^{\mathrm{alg}}\) in the product–ledger and include it in \(\Sigma\delta\); subsequent \(1\)\hyp Lipschitz post\hyp processing cannot increase it.
\end{enumerate}
\end{definition}

\begin{remark}[Micro example]
For \(T^{\mathrm{len}}_\tau\) (length) and \(T^{\mathrm{birth}}_{[u,u')}\) (birth\hyp window), apply \(T^{\mathrm{birth}}\) then \(T^{\mathrm{len}}\); if \(\Delta_{\mathrm{comm}}>\norm{\eta}\), keep this order and record \(\Delta_{\mathrm{comm}}\) into \(\delta^{\mathrm{alg}}\).
\end{remark}

\subsection*{5.7. Worked micro\hyp example (policy illustration)}
Let \(T^{\mathrm{len}}_\tau\) be length threshold and \(T^{\mathrm{birth}}_{[u,u')}\) birth\hyp window deletion. Measure \(\Delta_{\mathrm{comm}}(M;\mathrm{len},\mathrm{birth})\). If \(\le\norm{\eta}\), adopt soft\hyp commuting; otherwise fix the order \(T^{\mathrm{birth}}\) then \(T^{\mathrm{len}}\) and record \(\Delta_{\mathrm{comm}}\) in \(\delta^{\mathrm{alg}}\).

\subsection*{5.8. Overlap Gate (Functorial Gluing): collapse compatibility, soft commuting, Čech–Ext\textsuperscript{1}, stable bands}

We formalize a \emph{functorial} Overlap Gate that lifts the operational Overlap Gate (Chapter~1) to a typed, gluing\hyp ready interface.

\begin{definition}[Window Stack (WinFib) and Čech nerve]\label{def:winfib}
Let \(\mathsf{Win}\) be the category of pairs \((\alpha,W_\alpha)\) with \(W_\alpha\) right\hyp open, morphisms induced by inclusions \(X_\alpha\cap W_\alpha\hookrightarrow X_\beta\cap W_\beta\). For fixed degree \(i\) and \(\tau>0\), define a pseudo\hyp functor
\[
\mathcal{S}_{i,\tau}(-):\ \mathsf{Win}^{\mathrm{op}}\longrightarrow \mathrm{Cat},\qquad (\alpha,W_\alpha)\ \longmapsto\ \Big\{\mathbf{T}_\tau\,\Crop_{W_\alpha}\big(\mathbf{P}_i(F|_{X_\alpha})\big)\Big\}\subset\mathsf{Pers}^{\mathrm{cons}}_k.
\]
Its Grothendieck fibration \(\pi:\int\mathcal{S}_{i,\tau}(-)\to \mathsf{Win}\) is the \emph{Window Stack (WinFib)}. Let \(N(\mathcal{U})\) be the Čech nerve of the domain cover \(\{X_\alpha\}\).
\end{definition}

\begin{definition}[Overlap Gate \(\mathrm{OG}^{\mathrm{funct}}\)]\label{def:og-funct}
Fix \((i,\tau)\) and a windowed cover \(\{X_\alpha,W_\alpha\}\). We say \(\mathrm{OG}^{\mathrm{funct}}(i,\tau)\) \emph{passes} if:
\begin{enumerate}
  \item \textbf{Collapse compatibility} (after\hyp collapse \(1\)\hyp Lipschitz defect): for all overlaps, the objects in \(\int\mathcal{S}_{i,\tau}(-)\) agree up to the declared \(\delta\)\hyp budget, and the safety margin dominates the budget on overlaps.
  \item \textbf{Soft commuting (A/B)}: any pair of non\hyp nested reflectors used on overlaps passes the tolerance test with profile \(\eta\) (Definition~\ref{def:soft-commute}); otherwise a deterministic order is fixed and \(\Delta_{\mathrm{comm}}\) is accounted in \(\delta^{\mathrm{alg}}\).
  \item \textbf{Čech–Ext\textsuperscript{1}–acyclicity} (degree \(1\)): \(\Ext^1(\mathcal{R}(C_\tau F|_{X_{\alpha_0\cdots\alpha_p}}),k)=0\) for \(p=0,1\) on overlaps, and the Čech differential in \(\Ext^1\) vanishes.
  \item \textbf{Stable band \& no\hyp accumulation} (tower diagnostics): on each window and overlap, the tail comparison is an isomorphism \((\mu,\nu)=(0,0)\) on a stability band of \(\tau\)’s, with near\hyp \(\tau\) non\hyp accumulation.
\end{enumerate}
\end{definition}

\begin{theorem}[Functorial gluing via \(\mathrm{OG}^{\mathrm{funct}}\)]\label{thm:og-funct}
If \(\mathrm{OG}^{\mathrm{funct}}(i,\tau)\) passes for degree \(i=1\) on a windowed cover, then:
\begin{enumerate}
  \item (Existence) The local collapsed objects glue to a global object in \(\mathsf{Pers}^{\mathrm{cons}}_k\) (persistence layer) that is unique up to isomorphism on windows.
  \item (Gate propagation) The global B–Gate\({}^{+}\) passes on the union \(\bigcup_\alpha X_\alpha\times W_\alpha\): specifically, \(\mathrm{PH}_1(C_\tau F)=0\), \(\Ext^1(\mathcal{R}(C_\tau F),k)=0\), \((\mu,\nu)=(0,0)\).
  \item (Budget control) The global safety margin is bounded below by the minimum of local margins minus the overlap budgets (including A/B residuals), with product–ledger accounting.
\end{enumerate}
\end{theorem}

\begin{remark}[IMRN/AiM readiness]
The acceptance criteria (1)–(4) are checkable on the Čech nerve of the windowed cover, wholly \emph{after collapse} on the B\hyp side single layer. All budgets and tolerances are recorded per window; proofs use only exactness/Lipschitzness at persistence and amplitude \(\le 1\) at realization. The AWFS view (Declaration~\ref{dec:awfs}, Theorem~\ref{thm:awfs-triangle}) packages pre\hyp processing/collapse coherently and quantifies residual non\hyp commutation via the product–ledger \(\mathsf{Q}\).
\end{remark}

\subsection*{5.9. Window Stack (WinFib): typed acceptance on the nerve and auditability}

\begin{definition}[Typed acceptance predicate on the nerve]\label{def:typed-accept}
For each simplex \(\sigma=\{\alpha_0,\dots,\alpha_p\}\) in \(N(\mathcal{U})\) and a window \(W_\sigma=\bigcap_j W_{\alpha_j}\), define a \emph{typed acceptance} datum
\[
\mathbf{Acc}(\sigma; i,\tau)\ :=\ \Big(\mathrm{iso\_after\_collapse},\ \mathrm{AB\_soft\_commute},\ \mathrm{Cech\_Ext^1\_zero},\ \mathrm{stable\_band\_ok}\Big),
\]
with booleans and product–ledger budgets. We say the nerve passes if \(\mathbf{Acc}(\sigma;i,\tau)\) holds for all \(\sigma\) up to edges and vertices, and higher\hyp dimensional diagonals impose no extra constraints beyond Čech\(^1\)\hyp acyclicity.
\end{definition}

\begin{proposition}[Nerve acceptance implies \(\mathrm{OG}^{\mathrm{funct}}\)]\label{prop:nerve-accept}
If \(\mathbf{Acc}(\sigma;1,\tau)\) holds for all \(\sigma\) up to edges and vertices, then \(\mathrm{OG}^{\mathrm{funct}}(1,\tau)\) passes. Consequently Theorem~\ref{thm:og-funct} applies.
\end{proposition}

\begin{remark}[Machine\hyp checkable audit]
The quadruple \(\mathbf{Acc}\) is a minimal, machine\hyp checkable record per nerve simplex; together with the product–ledger and tolerance profile \(\eta\), it yields a complete audit trail for local\hyp to\hyp global collapse decisions. A Lean/Coq stub can represent \(\mathbf{Acc}\) as a structure with fields and proofs (Appendix~F).
\end{remark}

\subsection*{5.10. Summary}
We established a coherent functorial core for collapse within the constructible regime: a persistence\hyp level exact reflector \(\mathbf{T}_\tau\), an operational right adjoint collapse in \(\mathrm{Ho}\) (up to f.q.i.), and formal contracts for stability and bridge usage. An \emph{operational AWFS} \((L_\tau,R_\tau,\lambda)\) captures pre\hyp processing and collapse with idempotence and triangle laws up to f.q.i., while the \emph{product–ledger} makes the \(\delta\)\hyp budget \emph{natural} and \emph{additive} for both vertical and horizontal compositions with a \emph{tolerance profile} \(\eta\). For torsion reflectors, \emph{order independence} is guaranteed under \emph{nesting}; otherwise an A/B \emph{soft\hyp commuting} policy specifies priorities, a tolerance test, and a deterministic fallback with explicit \(\Delta_{\mathrm{comm}}\) logging. We formalized a \emph{functorial Overlap Gate} packaging collapse compatibility, soft commuting, Čech–Ext\({}^{1}\)–acyclicity, and stability bands; together with the \emph{Window Stack (WinFib)}, this yields a typed, nerve\hyp level acceptance test ensuring local\hyp to\hyp global gluing after collapse with fully logged budgets. All gate decisions remain on the B\hyp side after collapse, within a reproducible, windowed, and metrically stable framework that aligns with the tower diagnostics of Chapter~4 and the realization bridge of Chapter~3.


% NOTE TO READER: some previously uploaded auxiliary files appear to have expired on the platform; if you need them reloaded for cross-checks, please re-upload. This comment does not affect compilation.
\section{Chapter 6: Geometric Collapse (Program/Spec)}
\addcontentsline{toc}{section}{Geometric Collapse (Program/Spec)}
\label{sec:ch6}

\noindent\textbf{Monotonicity policy (after truncation).}
Deletion\hyp type updates are \emph{non\hyp increasing} for windowed persistence energies and spectral indicators; inclusion\hyp type updates are \emph{stability\hyp only} (non\hyp expansive). See Appendix~E for sufficient conditions and counterexamples. All comparisons, equalities, and gate decisions are made only after applying the truncation \(\mathbf{T}_\tau\) at the persistence layer:
\[
\boxed{\ \text{for each } t\ \Longrightarrow\ \mathbf{P}_i\ \Longrightarrow\ \mathbf{T}_\tau\ \Longrightarrow\ \text{compare in }\Pers^{\mathrm{ft}}\ }.
\]

\subsection*{6.0. Standing hypotheses and admissible geometric realization}
We work over a fixed field \(k\) and adopt the notation and hypotheses of Part~I. In particular,
\(\mathsf{FiltCh}(k)\) denotes finite\hyp type filtered chain complexes over \(k\),
\(\mathbf{P}_i:\mathsf{FiltCh}(k)\to\Pers^{\mathrm{ft}}_k\) the degreewise persistence functor,
and we write \(\mathbf{T}_\tau\) for the exact bar\hyp deletion (Serre) localization at scale \(\tau\ge 0\) (allowing \(\mathbf{T}_\tau=\mathrm{Id}\) when \(\tau=0\)).
Its filtered lift \(C_\tau\) is used \emph{up to filtered quasi\hyp isomorphism} (Chapter~2, §§2.2–2.3; Appendix~B).
The realization \(\mathcal{R}:\mathsf{FiltCh}(k)\to D^{\mathrm{b}}(k\text{-mod})\) is \(t\)\hyp exact.
All statements in this chapter lie in the constructible range (we identify \(\Pers^{\mathrm{ft}}_k\) with the constructible subcategory).
Unless explicitly marked \textbf{[Spec]}, \emph{equalities and Lipschitz claims are asserted only at the persistence layer}; identities at the filtered\hyp complex layer hold \emph{up to filtered quasi\hyp isomorphism}.
Kernel/cokernel diagnostics \((\muc,\nuc)\) are computed from the comparison maps
\[
  \phi_{i,\tau}:\ \varinjlim\nolimits_\lambda \mathbf{T}_\tau\!\big(\mathbf{P}_i(F_\lambda)\big)\ \longrightarrow\ \mathbf{T}_\tau\!\big(\mathbf{P}_i(\varinjlim\nolimits_\lambda F_\lambda)\big),
\]
with \(\dim_k\) interpreted as the \emph{generic\hyp fiber} dimension after truncation (multiplicity of \(I[0,\infty)\)); see Appendix~D, Remark~\ref{rem:D-generic-dim}.
Windows are MECE and right\hyp open by default. When stated, windows are \emph{definable} in a fixed o\hyp minimal expansion to guarantee finite event sets and finite Čech depth (Appendix~H/J; used below).

\begin{definition}[Admissible geometric realization]\label{def:geom-real}
Let \(\mathsf{Geom}\) be a geometric input category (e.g.\ metric or metric\hyp measure spaces with \(1\)\hyp Lipschitz maps; triangulated manifolds with mesh\hyp refinement maps; weighted graphs with contraction/sparsification maps).
An \emph{admissible geometric realization} is a functor
\[
  \mathcal{G}:\ \mathsf{Geom}\longrightarrow \mathsf{FiltCh}(k)
\]
such that:
\emph{(i)} \(\mathcal{G}\) is functorial and sends non\hyp expansive maps to filtered chain maps whose images under each \(\mathbf{P}_i\) are \(1\)\hyp Lipschitz for the interleaving distance;
\emph{(ii)} degreewise finite\hyp type is preserved;
\emph{(iii)} subsampling/refinement maps are carried to filtered maps that, for each fixed \(\tau\), induce filtered quasi\hyp isomorphisms after applying \(C_\tau\).
\end{definition}

\begin{remark}[Program posture and bridges]\label{rk:LC}
All specifications are asserted within the \emph{implementable range} of Part~I: (co)limit and stability statements are restricted to the persistence layer; the lifting–coherence hypothesis \(\mathrm{(LC)}\) is assumed for comparing \(C_\tau\) on \(\mathsf{FiltCh}(k)\) with effects after realization \(\mathcal{R}\).
No equivalence \(\mathrm{PH}_1\Leftrightarrow\Ext^1\) is claimed; only the one\hyp way bridge under \textup{(B1)–(B3)} from Part~I is used.
The obstruction \(\muc\) is \emph{distinct} from the classical Iwasawa \(\mu\)\hyp invariant.
\end{remark}

\begin{remark}[Stability vs.\ monotonicity; spectral policy]\label{rem:stability-vs-monotonicity}
Non\hyp expansive maps ensure stability (non\hyp expansiveness) of all indicators.
Under \emph{deletion\hyp type} updates satisfying Appendix~E (Dirichlet restriction, principal submatrices/Schur complements, Loewner contractions, and—in the symplectic setting—stop additions/Liouville contractions), spectral tails and windowed energies are \emph{non\hyp increasing}.
Inclusion\hyp type updates guarantee only \emph{stability}.
Spectral indicators are \emph{not} f.q.i.\ invariants; throughout we treat them as \emph{stable under a fixed normalization policy} and evaluate them on \(L(C_\tau F)\) (see Chapter~11).
\end{remark}

All monotonicity claims are interpreted after truncation by \(\mathbf{T}_\tau\).

\subsection*{6.0bis. Pipeline normal form and safe low\hyp pass (C\texorpdfstring{\(_\tau\)}{tau}\(\rightarrow\)W\(_{\mathrm{clip}}\)\(\rightarrow\)LP\(_\tau\))}
\begin{definition}[Window clipping]\label{def:Wclip}
For a right\hyp open window \(W=[u,u')\), the \emph{window clip} \(W_{\mathrm{clip}}\) acts on a persistence module \(M\) by restriction and extension by zero: \(M\mapsto M|_W\) viewed inside \(\Pers^{\mathrm{ft}}_k\). At the filtered level we implement \(W_{\mathrm{clip}}\) by cropping the filtration (up to f.q.i.).
\end{definition}

\begin{definition}[Safe low\hyp pass at scale \(\tau\)]\label{def:LP}
A \emph{safe low\hyp pass} operator \(\mathrm{LP}_\tau\) (optional) is a post\hyp collapse, window\hyp local endomorphism acting on \(L(C_\tau F|_W)\) (or equivalently its heat semigroup) such that:
\begin{enumerate}
  \item \emph{Even kernel, unit mass, band \(\tau\):} the time/scale kernel \(k_\tau\) is even, has unit mass, and support (or effective bandwidth) controlled at scale \(\tau\).
  \item \emph{After\hyp collapse Lipschitz:} pulling back to persistence through the Hodge functor, the map induced by \(\mathrm{LP}_\tau\) on \(\mathbf{T}_\tau\mathbf{P}_i(F|_W)\) is \(1\)\hyp Lipschitz for \(d_{\mathrm{int}}\).
\end{enumerate}
\end{definition}

\begin{theorem}[T\hyp Lipschitz\hyp After\hyp Collapse adoption test]\label{thm:LP-adopt}
Adopt the pipeline
\[
\boxed{\,C_\tau\ \longrightarrow\ W_{\mathrm{clip}}\ \longrightarrow\ \mathrm{LP}_\tau\,}
\]
\emph{only if} the after\hyp collapse Lipschitz condition of Definition~\ref{def:LP}\,(2) is verified on the window \(W\) within the declared tolerance. Otherwise, set \(\mathrm{LP}_\tau=\mathrm{Id}\). Under adoption, deletion\hyp type monotonicity (Remark~\ref{rem:stability-vs-monotonicity}) is preserved.
\end{theorem}

\subsection*{6.1. Monitored indicators and energies}
Fix an admissible \(\mathcal{G}\) and write \(F=\mathcal{G}(X)\in \mathsf{FiltCh}(k)\).

\begin{definition}[Persistence energies]\label{def:PE}
Let \(\mathcal{B}_i(F)\) be the multiset of intervals of \(\mathbf{P}_i(F)\).
For \(\alpha>0\) (default \(\alpha=1\)) and a truncation window \([0,\tau]\), define
\begin{equation}\label{eq:PE-alpha-trunc}
\begin{aligned}
\mathrm{PE}_{i,\alpha}^{\le\tau}(F)
&:= \sum_{[b,d)\in \mathcal{B}_i(F)}
   \bigl(\min\{d,\tau\}-\min\{b,\tau\}\bigr)_{+}^{\alpha},\\
(x)_{+}&:=\max\{x,0\}.
\end{aligned}
\end{equation}
By default \(\alpha=1\), and \(\mathrm{PE}^{\le\tau}(F):=\sum_i \mathrm{PE}_{i}^{\le\tau}(F)\).
\emph{All energies are evaluated on the truncated barcode: \(\mathrm{PE}_{i,\alpha}^{\le\tau}(F)=\mathrm{PE}_{i,\alpha}^{\le\tau}(C_\tau F)\) with \(\mathbf{T}_\tau\mathbf{P}_i(F)=\mathbf{P}_i(C_\tau F)\).}
\end{definition}

\begin{definition}[Spectral indicators]\label{def:spectral}
Let \(L(C_\tau F)\) be a combinatorial Hodge Laplacian on the truncated complex \(C_\tau F\) (normalized, with the Euclidean inner product on chains).
Denote the non\hyp decreasing spectrum by \((\lambda_m(C_\tau F))_{m\ge 0}\).
For \(\beta>0\) and an \emph{integer} cutoff \(M(\tau)\in\mathbb{N}\), define the spectral tail
\[
  \mathrm{ST}_{\beta}^{\ge M(\tau)}(F)\ :=\ \sum_{m\ge M(\tau)} \lambda_m(C_\tau F)^{-\beta},\qquad
  \mathrm{HT}(t;F)\ :=\ \mathrm{Tr}\big(e^{-tL(C_\tau F)}\big)\quad (t>0),
\]
with zero modes excluded (or replaced by the Moore–Penrose pseudoinverse). Qualitative specifications are invariant under these standard choices; the policy \((\beta,M(\tau),t)\) is fixed across a run (Appendix~G; Chapter~11).
\end{definition}

\begin{remark}[Convergence, parameterization, and logging]\label{rk:ST-conv}
Choose \(\beta\) and \(M(\tau)\) to ensure convergence (typical \(\beta\in\{1,2\}\), \(M(\tau)=\lfloor c\,\tau^{\gamma}\rfloor\) with \(c>0\), \(\gamma\in(0,2]\)).
When sweeping \(\tau\), take \(M(\tau)\) non\hyp decreasing to avoid artificial discontinuities.
Normalization, zero\hyp mode handling, and the window policy are fixed and logged with \((\beta,M(\tau),t)\).
\end{remark}

\begin{definition}[\(\Ext^1\)\hyp collapse at scale]\label{def:ext-collapse}
Writing \(\mathcal{R}(F)\in D^{\mathrm{b}}(k\text{-mod})\), we say \emph{\(\Ext^1\)\hyp collapse holds at scale \(\tau\)} if, for all \(Q\in\mathcal{Q}:=\{k[0]\}\),
\[
  \Ext^1\!\big(\mathcal{R}(C_\tau F),\, Q\big)\ =\ 0.
\]
\end{definition}

\subsection*{6.2. Stability under filtered colimits (geometry level)}
Let \(L_i(C_\tau F)\) denote the normalized combinatorial Hodge Laplacian in degree \(i\) on \(C_\tau F\), with nondecreasing positive spectrum \(\bigl(\lambda_{i,m}(C_\tau F)\bigr)_{m\ge 0}\). For brevity we suppress \(i\) and write \(L(C_\tau F)\), \(\mathrm{ST}_{\beta}^{\ge M(\tau)}(F)\), \(\mathrm{HT}(t;F)\) when the degree is clear from context.
\begin{declaration}[Specification: Stability under filtered colimits in geometry]\label{spec:geom-colim}
Assume a filtered diagram \(\{F_\lambda\}\) in \(\mathsf{FiltCh}(k)\) remains degreewise finite\hyp type; filtered (co)limits are computed objectwise in \([\mathbb{R},\mathsf{Vect}_k]\) and used only under the scope policy of Appendix~A (compute in the functor category and verify return to \(\Pers^{\mathrm{cons}}_k\)).
Then, for each fixed \(\tau\), the induced maps
\[
  \phi_{i,\tau}:\ \varinjlim\nolimits_\lambda \mathbf{T}_\tau\!\big(\mathbf{P}_i(F_\lambda)\big)\ \xrightarrow{\ \cong\ }\ \mathbf{T}_\tau\!\big(\mathbf{P}_i(\varinjlim\nolimits_\lambda F_\lambda)\big)
\]
are isomorphisms; hence \(\muc=\nuc=0\) at that scale. The conclusion holds pointwise along any discrete \(\tau\)\hyp sweep.
\end{declaration}

\begin{remark}[Endpoints and infinite bars]\label{rk:endpoints-ch6}
Endpoint conventions (open/closed) and the treatment of infinite bars are as in Chapter~2, Remark~\ref{rk:2-endpoints}; \(\mathbf{T}_\tau\) deletes only finite bars of length \(\le\tau\).
\end{remark}

\subsection*{6.3. Joint monitoring and programmatic guarantees}
\begin{declaration}[Specification: Geometric collapse indicators]\label{spec:geom-indicators}
Under \(\mathrm{(LC)}\) and within the implementable range, along geometric degenerations \emph{compute and record}:
\begin{enumerate}
  \item \(\mathbf{T}_\tau\mathbf{P}_i(F)\) and the truncated energies \(\mathrm{PE}_i^{\le\tau}\) on \(\mathbf{T}_\tau\mathbf{P}_i(F)=\mathbf{P}_i(C_\tau F)\);
  \item spectral indicators \(\mathrm{ST}_{\beta}^{\ge M(\tau)}\) or \(\mathrm{HT}(t;\cdot)\) on \(L(C_\tau F)\) (parameters as in Remark~\ref{rk:ST-conv});
  \item the \(\Ext^1\)\hyp check \(\Ext^1\big(\mathcal{R}(C_\tau F),Q\big)=0\) for \(Q\in\mathcal{Q}\).
\end{enumerate}
The \emph{stable regime} is declared where \((\muc,\nuc)=(0,0)\) and (1)–(3) hold jointly.
\end{declaration}

\begin{remark}[Saturation gate (reference; see Chapter~11)]\label{rk:sat-gate-ch6}
We follow the Chapter~11 policy for a window \([0,\tau^\ast]\): \emph{(i)} eventually the maximal finite bar length in \(\mathbf{T}_{\tau^\ast}\mathbf{P}_i(F_t)\) is \(\le \eta\); \emph{(ii)} eventually \(d_{\mathrm{int}}\!\big(\mathbf{T}_{\tau^\ast}\mathbf{P}_i(F_t),\mathbf{T}_{\tau^\ast}\mathbf{P}_i(F_{t'})\big)\le \eta\); \emph{(iii)} the edge gap \(\delta:=\tau^\ast-\max\{b_r<\tau^\ast\}\) satisfies \(\delta>\eta\).
This chapter \emph{uses the gate only as a reference}; the quantitative policy and its verification are centralized in Chapter~11.
\end{remark}

\subsection*{6.3bis. Gate Cascade (P8): \(E_1\rightarrow (\mu,\nu)\rightarrow \Ext^1\rightarrow \mathrm{PH}_1\)}
\begin{theorem}[Gate Cascade rule (windowed)]\label{thm:cascade}
Fix a definable right\hyp open window \(W\subset\mathbb{R}\) and \(\tau>0\).
Assume: \emph{(a)} after\hyp collapse evaluation; \emph{(b)} amplitude \(\le 1\) and tail\hyp isomorphism on \(W\); \emph{(c)} stable band for \(\phi_{1,\tau}\).
Then, on \(W\),
\[
E_1(F;W,\tau)=0\ \Longrightarrow\ (\mu_{1,\tau},\nu_{1,\tau})=(0,0)\ \Longrightarrow\ \Ext^1\!\big(\mathcal{R}(C_\tau F)|_W,k\big)=0\ \Longrightarrow\ \mathrm{PH}_1(C_\tau F|_W)=0.
\]
The first arrow uses that \(E_1{=}0\) leaves no residual positive\hyp length bars after \(\mathbf{T}_\tau\); the second uses amplitude/tail\hyp isomorphism and the one\hyp way bridge; the last is tautological under \(E_1{=}0\) (or follows from Theorem~3.2 when invoked).
\end{theorem}

\subsection*{6.4. Scope, definable windows, and design patterns}\label{subsec:ch6-definable}
\begin{declaration}[Specification: Scope of admissible degenerations]\label{spec:scope}
The program encompasses: \emph{(a)} metric(\hyp measure) collapses modeled by subsampling and \(1\)\hyp Lipschitz retractions; \emph{(b)} simplicial refinements with bounded local degree; \emph{(c)} graph sparsifications preserving the normalized Laplacian construction and the \(1\)\hyp Lipschitz property of \(\mathcal{G}\), thereby keeping each \(\mathbf{P}_i\) non\hyp expansive under these maps.
Each case is functorially embedded by an admissible \(\mathcal{G}\).
\end{declaration}

\begin{remark}[Definable windows and finite Čech depth]\label{rem:definable-windows}
When windows \([u,u')\) (and, if present, domain covers) are definable in a fixed o\hyp minimal expansion of \((\mathbb{R},+,\cdot)\), one has: (i) only finitely many events (births/deaths) occur on each bounded window; (ii) the Čech nerve has finite depth; hence Overlap Gate checks reduce to finitely many overlaps and are fully auditable. This applies verbatim to geometric realizations, and integrates with the E\(_1\)\,–\,local gate (Chapter~3, Theorem~3.2) on definable windows.
\end{remark}

\begin{center}
\begin{tikzcd}[
  row sep=1.6em, column sep=3.4em,
  cells={nodes={font=\footnotesize}},
  every label/.append style={font=\scriptsize},
  scale=0.92, transform shape
]
{\text{Geometric input } X} \arrow[r, "\mathcal{G}"] 
  & {F\in \mathsf{FiltCh}(k)} \arrow[r, "\mathbf{P}_i"] 
    & {\mathbf{P}_i(F)} \arrow[r, "\mathbf{T}_\tau"] \arrow[d, dashed, "\mathrm{PE}_i^{\le\tau}"']
      & {\mathbf{T}_\tau\mathbf{P}_i(F)} \arrow[d, dashed, "\mathrm{PE}_i^{\le\tau}"] \\
{} & {} & {\text{Spectral } L(C_\tau F)}
      \arrow[r, dashed, "\text{heat trace / tails on }L(C_\tau F)"] 
      & {{\mathrm{HT},\,\mathrm{ST}_{\beta}}} \\
{} & {\mathcal{R}(F)}
      \arrow[from=3-2, to=1-3, bend left=20, dashed, "\mathrm{(LC)}"]
      \arrow[from=3-2, to=3-4, dashed, "{\begin{matrix}C_\tau\text{ then }\\ \Ext^1(-,Q)\end{matrix}}"]
  & {}
  & {{\Ext^1\big(\mathcal{R}(C_\tau F),Q\big)=0\ \ (\text{check})}}
\end{tikzcd}
\end{center}

\subsection*{6.5. Failure geometry and diagnostics}
\begin{definition}[Geometric failure types at scale]\label{def:geom-failure}
Within the monitored window, a sample is \emph{Type~IV at scale \(\tau\)} if \(\mathrm{PE}^{\le\tau}\) and spectral indicators decay while \((\muc,\nuc)\neq(0,0)\).
The \emph{pure cokernel type} denotes \(\muc=0\) and \(\nuc>0\).
\end{definition}

\begin{declaration}[Specification: Diagnostic actions]\label{spec:diagnostic}
When \((\muc,\nuc)\neq(0,0)\), refine the index diagram or adjust \(\tau\)\hyp sweep granularity until either \emph{(a)} the obstruction vanishes, or \emph{(b)} the failure persists across refinements, in which case the regime is recorded as non\hyp collapsible at the monitored scale.
\end{declaration}

\subsection*{6.6. Symplectic hook: Fukaya realization (\textbf{[Spec]})}
\begin{declaration}[Spec--Fukaya realization]\label{spec:fukaya}
Let \(\mathsf{Symp}^{\mathrm{adm}}\) be exact/monotone Liouville domains or sectors with stops.
\(\mathcal{G}_{\mathrm{Fuk}}\) assigns action\hyp filtered Floer complexes on a fixed window \([a,\tau]\) with \(a\le \tau\) over a field.
Assume:
\emph{(F1)} finite action spectrum in \([a,\tau]\);
\emph{(F2)} continuation maps shift actions by \(\le\varepsilon\) uniformly (hence are \(1\)\hyp Lipschitz for interleavings);
\emph{(F3)} stop additions/Liouville contractions are deletion\hyp type (Appendix~E).
Then for each degree \(i\) and scale \(\tau\) the comparison maps \(\phi_{i,\tau}\) are isomorphisms, hence \((\mu,\nu)=(0,0)\) on the monitored window.
Proof sketches and scope limits appear in Appendix~O.
\end{declaration}

\begin{remark}[Scope and bridge domain]
The specification above does \emph{not} extend the proved bridge beyond \(D^{\mathrm{b}}(k\text{-mod})\);
it provides a stable geometric hook whose persistence\hyp level behavior feeds the Part~I pipeline.
\end{remark}

\subsection*{6.7. Permitted operations catalog and \(\delta\)\hyp ledger (reinforced policy)}\label{subsec:ops-delta}
We record the admissible A\hyp side operations, their expected persistence\hyp level behavior \emph{after collapse}, and the mandatory \(\delta\) logging.

\begin{definition}[Permitted operations]\label{def:ops-catalog}
Each A\hyp side step \(U\) is labeled:
\begin{itemize}
  \item \emph{Deletion\hyp type (monotone; P5).} Examples: stop addition / sector shrinking (symplectic), mollification (low\hyp pass filtering), viscosity increment (PDE), threshold lowering, filter upper\hyp cap. \emph{Guarantee:} after applying \(C_\tau\) (and, if adopted, \(\mathrm{LP}_\tau\)), windowed persistence energies and spectral auxiliaries (aux\hyp bars) are \emph{non\hyp increasing}.
  \item \emph{\(\varepsilon\)\hyp continuation (non\hyp expansive).} Examples: small Hamiltonian continuation; micro time\hyp step; minor stop shift. \emph{Guarantee:} \(d_{\mathrm{int}}(\mathbf{P}_i(F),\mathbf{P}_i(UF))\le \varepsilon\); after \(C_\tau\), indicators are \emph{stable} up to the prescribed \(\varepsilon\).
  \item \emph{Inclusion\hyp type (stable only).} Examples: domain enlargement, inclusion maps not covered by the deletion\hyp type list. \emph{Guarantee:} no monotonicity claim; only stability (non\hyp expansiveness) if the induced map is \(1\)\hyp Lipschitz on persistence.
\end{itemize}
\end{definition}

\begin{declaration}[Mandatory \(\delta\)\hyp ledger]\label{dec:delta-ledger-ch6}
For each step \(U\) with collapse \(C_{\tau}\) and a fixed degree \(i\), record a three\hyp part non\hyp commutation budget
\[
\delta(i,\tau)\ =\ \delta^{\mathrm{alg}}(i,\tau)\ +\ \delta^{\mathrm{disc}}(i,\tau)\ +\ \delta^{\mathrm{meas}}(i,\tau),
\]
where \(\delta^{\mathrm{alg}}\) is the theoretical Mirror/Transfer–Collapse mismatch, \(\delta^{\mathrm{disc}}\) the discretization error, and \(\delta^{\mathrm{meas}}\) the numerical/estimation error. The per\hyp window pipeline budget is \(\Sigma\delta(i)=\sum_{U\in W}\delta(i,\tau)\) and must satisfy \(\mathrm{gap}_\tau>\Sigma\delta(i)\) to pass B\hyp Gate\({}^{+}\) (Chapter~1).
\end{declaration}

\subsection*{6.8. Gate template (per step, per window) and saturation usage}\label{subsec:gate-template}
The following operational template is used for each A\hyp side step within a fixed domain window \(W=[u,u')\) and a fixed collapse threshold \(\tau>0\):
\begin{enumerate}
  \item \emph{Apply step \(U\) and collapse.} Execute \(U\) (labeled as in Definition~\ref{def:ops-catalog}), then apply \(C_\tau\); clip to \(W\) and, only if Theorem~\ref{thm:LP-adopt} holds, apply \(\mathrm{LP}_\tau\).
  \item \emph{Measure on B\hyp side single layer.} Compute \(\mathbf{T}_\tau\mathbf{P}_i(F)\), \(\mathrm{PE}_i^{\le\tau}\), spectral indicators on \(L(C_\tau F)\) under the fixed policy, and (if in scope) \(\Ext^1(\mathcal{R}(C_\tau F),k)\).
  \item \emph{Record \(\delta\).} Append \(\delta^{\mathrm{alg}},\delta^{\mathrm{disc}},\delta^{\mathrm{meas}}\) for this step to the per\hyp window ledger and update \(\Sigma\delta(i)\).
  \item \emph{Evaluate B\hyp Gate\({}^{+}\).} Use the Cascade rule (Theorem~\ref{thm:cascade}) with the windowed safety margin \(\mathrm{gap}_\tau\); require \(\mathrm{gap}_\tau>\Sigma\delta(i)\).
  \item \emph{Log verdict.} If all pass, issue a windowed certificate; otherwise, classify failure (Type~I–IV) and proceed with diagnostics (Declaration~\ref{spec:diagnostic}).
\end{enumerate}
On windows declared \emph{saturated} in the sense of Chapter~11, one may use the saturation gate as a reference to shorten step (4) (remain within its quantitative policy).

\subsection*{6.9. Windowed workflow and logging (MECE enforcement)}\label{subsec:window-workflow}
Let \(\{[u_k,u_{k+1})\}_k\) be a MECE partition (Chapter~2, Def.~\ref{def:ch2-mece}). For each window:
\begin{itemize}
  \item Fix \(\tau\) by the adaptation rule (Chapter~2, Def.~\ref{def:tau-adapt}); if spectral auxiliaries are used, fix \((\beta,[a,b])\).
  \item Run the gate template (Subsection~\ref{subsec:gate-template}) for each step; aggregate \(\Sigma\delta(i)\) and evaluate B\hyp Gate\({}^{+}\).
  \item Record coverage checks (sum of lengths; sum of events) and all parameters in the manifest (Appendix~G).
\end{itemize}
Global claims are obtained by pasting windowed certificates via Restart (Appendix~J, Lemma~J.\ref{J:lem:restart}) and Summability (Appendix~J, Definition~J.\ref{J:def:summability}); \(\tau\) is selected inside stable bands (Appendix~J, Definition~J.\ref{J:def:stab-band}). When multiple torsion reflectors are used (e.g.\ length plus birth window), apply the soft\hyp commuting policy (Chapter~5, Definition~\ref{def:soft-commute}); otherwise fix a deterministic order and record the commutation defect in \(\delta^{\mathrm{alg}}\).

\subsection*{6.10. Compliance checklist (per run)}\label{subsec:compliance}
\begin{enumerate}
  \item MECE windows recorded; coverage checks pass; if windows are definable, include their formulas and the o\hyp minimal structure used.
  \item Collapse threshold \(\tau\) adapted to resolution; spectral bin policy fixed and logged.
  \item Each step labeled (deletion/\(\varepsilon\)/inclusion) with \(1\)\hyp Lipschitz rationale; \(\delta^{\mathrm{alg}},\delta^{\mathrm{disc}},\delta^{\mathrm{meas}}\) recorded.
  \item Indicators computed on B\hyp side single layer only; B\hyp Gate\({}^{+}\) evaluated (Cascade rule, safety margin).
  \item Tower audit \((\mu,\nu)=(0,0)\) on the window; stable band identified for \(\tau\); if applicable, E\(_1\)–local gate used on definable windows (Chapter~3, Theorem~3.2).
  \item Verdict (accept/reject) and failure type logged; Restart/Summability plan updated for the next window.
\end{enumerate}

\subsection*{6.11. Summary}
This chapter specifies the operational program for geometric collapse in the implementable range. Admissible realizations (Definition~\ref{def:geom-real}) feed the persistence layer, where collapse \(C_\tau\) is applied and all indicators are computed on the B\hyp side single layer. The pipeline is standardized as
\[
C_\tau\ \to\ W_{\mathrm{clip}}\ \to\ \mathrm{LP}_\tau\ \ (\text{optional, only if Theorem~\ref{thm:LP-adopt} holds}).
\]
Deletion\hyp type steps are \emph{non\hyp increasing} after collapse (and optional safe low\hyp pass); \(\varepsilon\)–continuations are \emph{stable}. Mirror/Transfer non\hyp commutation with collapse is \emph{externalized} via a \(\delta\)–ledger and accumulated additively along pipelines in a fixed commutative quantale. Windowed certificates are issued per MECE window by B\hyp Gate\({}^{+}\); the Gate Cascade (Theorem~\ref{thm:cascade}) organizes decisions as \(E_1\to(\mu,\nu)\to\Ext^1\to\mathrm{PH}_1\). Global claims are obtained by pasting certificates using Restart and Summability, with \(\tau\) selected inside stable bands. The soft\hyp commuting policy (Chapter~5) governs multi\hyp axis torsions. All assertions remain confined to the persistence layer and respect the one\hyp way bridge (Chapter~3) and the tower calculus (Chapter~4).

\subsection*{6.12. Tropical Mirror/Transfer: natural 2\hyp cell and energy monotonicity (\textbf{[Spec]})}
We now specify a tropical endofunctor and its quantitative commutation with collapse; this is used solely as a \emph{post\hyp collapse comparator} and an \emph{energy monotonicity} trigger.

\begin{definition}[Tropical functor and 2\hyp cell bound]\label{def:trop-2cell}
Let \(\Trop_\lambda:\mathsf{Geom}\to \mathsf{Geom}\) be a family of endofunctors indexed by \(\lambda\in(0,1]\) such that:
\begin{enumerate}
  \item \textup{(Non\hyp expansiveness)} For each degree \(i\), \(d_{\mathrm{int}}\big(\mathbf{P}_i(\mathcal{G}\circ \Trop_\lambda X),\mathbf{P}_i(\mathcal{G}\circ \Trop_\lambda Y)\big)\le d_{\mathrm{int}}\big(\mathbf{P}_i(\mathcal{G}X),\mathbf{P}_i(\mathcal{G}Y)\big)\).
  \item \textup{(Shortening proxy)} There exists \(\kappa(\lambda)\in(0,1]\) such that on each window \([0,\tau]\) and degree \(i\) the truncated bars of \(\mathbf{T}_\tau\mathbf{P}_i(\mathcal{G}\circ \Trop_\lambda X)\) are obtained from those of \(\mathbf{T}_\tau\mathbf{P}_i(\mathcal{G}X)\) by endpoint shifts bounded by a uniform \(2\varepsilon\) and a multiplicative shortening by at most \(\kappa(\lambda)\), up to f.q.i.\ (cf.\ Appendix~M).
  \item \textup{(2\hyp cell with collapse)} There is a natural \(2\)\hyp cell
  \[
  \theta_{i,\tau}:\ \mathbf{P}_i\big(C_\tau(\mathcal{G}\circ \Trop_\lambda X)\big)\ \Longrightarrow\ \mathbf{P}_i\big(\mathcal{G}\circ \Trop_\lambda(C_\tau X)\big)
  \]
  whose effect on \(\mathbf{T}_\tau\) is bounded by a uniform \(\delta_\mathrm{trop}(i,\tau)\) in the chosen quantale (Appendix~L).
\end{enumerate}
We call \((\Trop_\lambda,\kappa,\delta_\mathrm{trop})\) a \emph{windowed tropical comparator}.
\end{definition}

\begin{theorem}[After\hyp collapse energy non\hyp increase under tropical shortening]\label{thm:trop-energy}
Under Definition~\ref{def:trop-2cell}, for each degree \(i\), window \([0,\tau]\), and \(\alpha>0\),
\[
\mathrm{PE}_{i,\alpha}^{\le \tau}\!\big(C_\tau(\mathcal{G}\circ \Trop_{\lambda'} X)\big)\ \le\ \mathrm{PE}_{i,\alpha}^{\le \tau}\!\big(C_\tau(\mathcal{G}\circ \Trop_{\lambda} X)\big)\qquad(\lambda'\le \lambda),
\]
with strict decrease whenever \(\kappa(\lambda',\lambda)<1\) acts on a positive\hyp mass subset of clipped bars. Moreover, the comparison is \(\delta_\mathrm{trop}\)\hyp controlled at the persistence layer in the sense of Appendix~L.
\end{theorem}

\begin{proof}[Proof sketch]
Energy non\hyp increase follows from the shortening proxy (Appendix~M, Theorems~M.\ref{thm:conv}–M.\ref{thm:mono-dominance}) and the fact that \(\mathbf{T}_\tau\) is \(1\)\hyp Lipschitz. The \(2\)\hyp cell bound provides a quantitative defect \(\delta_\mathrm{trop}(i,\tau)\) to be recorded in the \(\delta\)\hyp ledger.
\end{proof}

\subsection*{6.13. PF/BC after\hyp collapse comparison protocol (arithmetic comparator)}
We promote the projection\hyp formula/base\hyp change (\textbf{PF/BC}) comparison to an after\hyp collapse arithmetic protocol at fixed windows and thresholds; see Chapter~6, §§6.12–6.17 of the main text for full details, and Appendix~N for the PF/BC transport contracts. The post\hyp collapse metric drift (if any) is logged in \(\delta^{\mathrm{disc}},\delta^{\mathrm{meas}}\) for the window’s budget (Appendix~G).

\subsection*{6.14. Collapse classification and the Defect functor (Iwasawa\hyp style notation)}
As in Chapter~6, §§6.13–6.17 and Appendix~D/J, the windowed verdict \(\mathrm{Verdict}(W,\tau)\) classifies invisible failures via \((\mu_{i,\tau},\nu_{i,\tau})\), aggregates budgets in the chosen quantale, and enforces \(\mathrm{gap}_\tau>\Sigma\delta\).

\subsection*{6.15. run.yaml augmentation (synchronization with Appendix~G)}
To make tropical, definable, and quantale choices audit\hyp ready, the manifest must include:

\begin{verbatim}
quantale:
  name: "R_plus"            # e.g., R_plus, R_max, product
  op: "add"                 # add|max|product
  unit: 0.0
  order: "le"               # <= (Lawvere orientation)

definable:
  o_minimal_structure: "R_an,exp"
  window_formulae:
    - id: "W01"
      expr: "0 <= t < 1.0"
    - id: "W02"
      expr: "1.0 <= t < 2.0"

tropical:
  bins: { a: 0.0, beta: 0.02, bins: 96, boundary: "right-open" }
  kappa: { lambda_prime: 0.4, lambda: 0.7, value: 0.85 }
  two_cell_bound:
    degree: 1
    tau: 0.25
    delta: 0.010
\end{verbatim}

All other fields (windows/coverage checks, operations, persistence verdicts, spectral policies, \(\delta\)\hyp ledger) remain as specified in Appendix~G.

\noindent\emph{Policy.} The tropical comparator is a [Spec]-only, post-collapse comparator and is never used as a gate; any $2$-cell defect contributes solely to the quantale-valued $\delta$-ledger.



\section{Chapter 7: Arithmetic Layers and Iwasawa Refinement (Design)}
\addcontentsline{toc}{section}{Arithmetic Layers and Iwasawa Refinement (Design)}

\noindent\textbf{Index separation.}
The collapse obstruction \(\muc\) used in this chapter is a persistence--level diagnostic and is \emph{unrelated} to the classical Iwasawa \(\mu\)–invariant; no identity or implication between them is asserted (see also §\ref{rk:separation}).

\begin{remark}[Monotonicity convention]
Throughout this chapter we adopt the corrected monotonicity convention of
Chapter~6, Remark~\ref{rem:stability-vs-monotonicity}:
\emph{deletion--type} updates are non–increasing for spectral tails and windowed energies,
while \emph{inclusion--type} updates are only stable (non–expansive);
see Appendix~E for sufficient conditions and counterexamples.
\end{remark}

\subsection*{7.0. Standing hypotheses and admissible arithmetic realization}
All statements in this chapter are made within the \emph{constructible range}
(we identify \(\Pers^{\mathrm{ft}}_k\) with the constructible subcategory as in Chapters~2 and~6).
Fix a base field \(k\) and adopt the notation and posture of Part~I:
\(\FiltCh{k}\) denotes finite–type filtered chain complexes,
\(\mathbf{P}_i:\FiltCh{k}\to\Pers^{\mathrm{cons}}_k\) the degreewise persistence functor,
and we write \(\Ttau:=\mathbf{T}_\tau\) for the Serre (bar–deletion) reflector at scale \(\tau\ge 0\) (with \(\Ttau=\mathrm{Id}\) at \(\tau=0\)).
Its filtered lift \(C_\tau\) is used \emph{up to filtered quasi–isomorphism} (Chapter~2, §§2.2–2.3).
A fixed realization \(\Rfun:\FiltCh{k}\to D^{\mathrm{b}}(k\text{-mod})\) is \(t\)–exact.
Unless explicitly marked \textbf{[Spec]}, \emph{equalities and Lipschitz claims are asserted only at the persistence layer};
at the filtered–complex layer they hold \emph{up to filtered quasi–isomorphism}.
Endpoint conventions and the treatment of infinite bars are as in Chapter~2, Remark~\ref{rk:2-endpoints}.

Arithmetic input is organized as towers
\[
  \mathbb{T}\ :=\ \{X_t\}_{t\in I}\ \longrightarrow\ X_\infty,
\]
indexed by a directed set \(I\cup\{\infty\}\) with transition maps \(X_{t'}\to X_t\) for \(t'\ge t\) (e.g.\ norm/corestriction, specialization, level–lowering).
Typical instances include cyclotomic/ray–class towers of number fields, modular–level towers, or Selmer–complex towers.
Filtered (co)limits, when used, are computed objectwise in \([\mathbb{R},\mathsf{Vect}_k]\) and used only under the scope policy of Appendix~A (compute in the functor category and verify return to \(\Pers^{\mathrm{cons}}_k\)); no claim is made outside this regime.

\begin{remark}[Denef–Pas windows]
In arithmetic sections we \emph{adopt Denef–Pas definable height windows} (right–open, MECE) whenever possible; see Appendix~Q for the Denef–Pas framework and quantifier–elimination tools used to ensure finiteness of event sets and finite Čech depth on height slices.
All window–local audits and gates below are meant to operate on such definable windows when declared. \label{rk:DP-windows}
\end{remark}

\begin{definition}[Iwasawa tower \(\Rightarrow\) persistence; height/local–intensity pattern]\label{def:Iwasawa-tower->pers}
A \emph{classical Iwasawa tower} consists of a directed system \(\{K_t\}_{t\in I}\) of global fields (e.g.\ \(t=n\) for \(\mathbb{Z}_p\)–extensions), together with arithmetic objects \(\{A_t\}\) (e.g.\ class groups, Selmer groups, cohomology complexes with local conditions \(\mathcal{L}_t\)). We encode this data into the persistence pipeline via:
\begin{enumerate}
  \item \textbf{Filtered realization.} Choose a filtered chain model \(F_t\in\FiltCh{k}\) for each \(t\), functorially in \((K_t,A_t,\mathcal{L}_t)\), such that:
  \begin{itemize}
    \item the \emph{height} filtration \(F_t(\,\cdot\,)\) is non–decreasing in a height parameter \(h\) (conductor/level/weight);
    \item \emph{local intensity} (imposed by \(\mathcal{L}_t\)) is implemented by Serre–class reflectors on \(F_t\) (pre–collapse).
  \end{itemize}
  \item \textbf{Transitions.} Corestriction/norm/specialization maps \(A_{t'}\to A_t\) (\(t'\ge t\)) induce filtered maps \(F_{t'}\to F_t\) that are non–expansive after applying \(\mathbf{P}_i\) (interleaving metric), up to filtered quasi–isomorphism.
  \item \textbf{Collapse and persistence.} Apply \(C_\tau\) and then \(\mathbf{P}_i\) to obtain truncated persistence modules \(\Ttau\mathbf{P}_i(F_t)\) and barcodes; energies and spectra are computed on \(C_\tau F_t\).
\end{enumerate}
Thus the tower pattern is
\[
\text{Iwasawa tower } (K_t,A_t,\mathcal{L}_t)\ \longmapsto\ F_t\ \xrightarrow{\,\mathbf{P}_i\,}\ \mathbf{P}_i(F_t)\ \xrightarrow{\,\Ttau\,}\ \Ttau\mathbf{P}_i(F_t),
\]
which we call the \emph{Iwasawa\(\to\)persistence} pattern. All claims are at the persistence layer, with filtered–level statements interpreted up to filtered quasi–isomorphism.
\end{definition}

\begin{definition}[Admissible arithmetic realization]\label{def:arith-real}
An \emph{admissible arithmetic realization} is a functor
\[
\begin{aligned}\n\mathcal{A}\colon\ & \mathsf{ArithTower}\longrightarrow \FiltCh{k},\\[-0.25em]\n& \mathbb{T}\longmapsto F_\bullet=\{F_t\}_{t\in I\cup\{\infty\}},\n\end{aligned}
\]
subject to: (1) functorial non–expansiveness at persistence (with interleaving bounds \(\varepsilon_{t',t}\)); (2) finite–type preservation and objectwise filtered (co)limits; (3) realization coherence \(\Rfun(C_\tau F_t)\simeq \tau_{\ge 0}\Rfun(F_t)\) up to f.q.i.; (4) endpoint policy of Part~I.
\end{definition}

\begin{remark}[Cone extension for the tower]\label{rk:cone-extension}
We work in the filtered index category \(I\cup\{\infty\}\) with \(t\le \infty\) and \emph{cone maps} \(X_t\to X_\infty\).
The realization \(\mathcal{A}\) carries these to filtered maps \(F_t\to F_\infty\), yielding the comparison maps in Definition~\ref{def:phi-munu}, mirroring Chapter~4.
\end{remark}

% ---- §7.1–§7.17: retained with minor edits for DP windows cross-ref (omitted here for brevity in this excerpt) ----
% The following subsections reproduce the user's approved §7.1–§7.17 content verbatim,
% except for harmless cross-references to Remark~\ref{rk:DP-windows} where “definable windows” are invoked.

\subsection*{7.1. Class/Selmer visualization at the persistence layer}
\begin{definition}[Arithmetic visualization data]\label{def:vis}
Given \(\mathbb{T}\mapsto F_\bullet\) via \(\mathcal{A}\), define for each \(t\in I\) and degree \(i\):
\[
  \mathcal{B}_i(F_t):=\mathrm{bars}\big(\mathbf{P}_i(F_t)\big),\qquad
  \mathrm{PE}_i^{\le\tau}(F_t)\ \text{ as in §6.1 (evaluated on }\Ttau\mathbf{P}_i(F_t)\text{)}.
\]
\end{definition}

\begin{remark}[Spectral layer and \(\Ext^1\)–check]\label{rk:spectral-ext}
Form the normalized Hodge Laplacian \(L_i(C_\tau F_t)\) and record spectral tails/heat traces as in §6.1.
At the categorical layer, check \(\Ext^1\!\big(\Rfun(C_\tau F_t),Q\big)=0\) for \(Q\in\Qtest=\{k[0]\}\).
\end{remark}

\subsection*{7.2. Tower diagnostics and obstructions}
\begin{definition}[Tower comparison and obstruction indices]\label{def:phi-munu}
For each degree \(i\) and scale \(\tau\), the comparison map
\[
  \phi_{i,\tau}:\ \varinjlim_{t\in I}\ \Ttau\!\big(\mathbf{P}_i(F_t)\big)\ \longrightarrow\ \Ttau\!\big(\mathbf{P}_i(F_\infty)\big)
\]
yields obstruction counts
\(
  \mu_{i,\tau}:=\dim_k\ker\phi_{i,\tau},\quad
  \nu_{i,\tau}:=\dim_k\,\mathrm{coker}\,\phi_{i,\tau}
\),
with \(\muc=\sum_i\mu_{i,\tau}\), \(\nuc=\sum_i\nu_{i,\tau}\), where \(\dim_k\) denotes the \emph{generic–fiber} dimension after truncation.
\end{definition}

\begin{declaration}[Spec–Arithmetic towers (non–expansion)]\label{spec:arith-nonexp}
Index transitions are non–expansive (interleaving sense), uniformly controlled.
Under finite–type and objectwise filtered colimits (Appendix~A), each \(\phi_{i,\tau}\) is an isomorphism, hence \((\muc,\nuc)=(0,0)\).
\end{declaration}

\begin{declaration}[Specification: Tower stability at the persistence layer]\label{spec:tower-stability}
For each fixed \(\tau\) and all \(i\),
\(
  \varinjlim_{t} \Ttau\mathbf{P}_i(F_t)\xrightarrow{\ \cong\ }\Ttau\mathbf{P}_i(F_\infty)
\);
thus \((\muc,\nuc)=(0,0)\) at scale \(\tau\).
\end{declaration}

\begin{remark}[Excluding Type~IV under tower stability]\label{rk:exclude-typeIV}
Under Declaration~\ref{spec:tower-stability} we have \((\muc,\nuc)=(0,0)\); hence Type~IV cannot occur at that scale.
\end{remark}

\begin{remark}[Failure patterns]\label{rk:fail}
If \((\muc,\nuc)\neq(0,0)\), record: \emph{pure cokernel} \((\muc=0,\nuc>0)\), \emph{pure kernel} \((\muc>0,\nuc=0)\), or \emph{mixed}.
\end{remark}

\begin{example}[Toy towers at the persistence layer]\label{ex:toy-towers}
(As in the approved draft; omitted here for space.)
\end{example}

\subsection*{7.3. Non–identity with classical Iwasawa \(\mu\)}
\begin{remark}[Separation of indices]\label{rk:separation}
The persistence obstruction \(\muc\) is extracted from kernels/cokernels of \(\phi_{i,\tau}\) between \emph{truncated} persistence modules, whereas the classical Iwasawa \(\mu\) measures \(p\)–primary growth of \(\Lambda=\mathbb{Z}_p[[T]]\)–modules.
No identity or implication is asserted; any relation, if present, is programmatic and confined to \textbf{[Conjecture]} statements.
\end{remark}

\begin{remark}[Alignment conditions for \(\mu_{\mathrm{Collapse}}\) and classical \(\mu\)]\label{rem:mu-align}
Programmatic alignment may be arranged on selected windows under:
(1) windowed torsion reflection;
(2) deletion–dominance (or uniformly bounded non–expansive shifts);
(3) stability of local conditions;
see the approved draft for details.
\end{remark}


\subsection*{7.4. Program specifications for arithmetic towers}
\begin{declaration}[Specification: Admissible indexings and maps]\label{spec:indexing}
An indexing of the tower by conductor, level, or height that renders the transition maps non\hyp expansive (hence \(1\)\hyp Lipschitz under each \(\mathbf{P}_i\) in the interleaving sense of Definition~\ref{def:arith-real}(1)) is \emph{admissible}.
Under such indexings, energy and spectral indicators are stable (non\hyp expansive) in general and \emph{non\hyp increasing for deletion\hyp type steps} (Appendix~E), up to f.q.i.; no non\hyp increase is claimed for inclusion\hyp type updates.
\end{declaration}

\subsection*{7.5. Conjectural propagation along arithmetic towers}
\begin{conjecture}[AK–Arithmetic tower propagation]\label{conj:arith-prop}
Assume an admissible arithmetic realization $\mathcal{A}$ and \LC.
If, along a non\hyp expansive tower segment and for a scale interval in $\tau$, we have $(\muc,\nuc)=(0,0)$ and the persistence energies (deletion\hyp type: non\hyp increasing; general: stable) together with the spectral indicators are controlled as above, then the arithmetic visualization stabilizes at that scale:
the proxies registered by persistence/spectral layers remain bounded, and the categorical check
\(\Ext^1\big(\Rfun(C_\tau F_t),Q\big)=0\) persists along the segment.
No number\hyp theoretic identity, and no identification with the classical Iwasawa invariants, is asserted.
\end{conjecture}

\subsection*{7.6. Diagram and data flow}
\begin{center}
\begin{tikzcd}[
  ampersand replacement=\&,
  column sep=1.8em, row sep=1.8em,
  cells={nodes={font=\footnotesize}},
  every label/.append style={font=\scriptsize},
  scale=0.92, transform shape
]
{\text{Arithmetic tower } \mathbb{T}} \arrow[r, "\mathcal{A}"]
  \& {\{F_t\}\subset \FiltCh{k}} \arrow[r, "\mathbf{P}_i"]
    \& {\mathbf{P}_i(F_t)} \arrow[r, "\Ttau"]
       \arrow[from=1-3, to=2-3, dashed, "{\mathrm{PE}_i^{\le \tau}}"']
      \& {\Ttau\mathbf{P}_i(F_t)}
       \arrow[from=1-4, to=2-4, dashed, "{\mathrm{PE}_i^{\le \tau}}"] \\\n{\phantom{\cdot}} \& {\phantom{\cdot}}
  \& {\text{Spectral } L(C_\tau F_t)}
    \arrow[r, dashed, "{\text{heat trace on }L(C_\tau F_t)}"]
  \& {\mathrm{HT},\ \mathrm{ST}_{\beta}} \\\n{\phantom{\cdot}} \& {\Rfun(F_t)}
  \arrow[from=3-2, to=1-3, bend left=40, dashed, "\LC"]
  \arrow[from=3-2, to=3-4, dashed, "{C_\tau\ \text{ then }\Ext^1(-,Q)}"]
  \& {\phantom{\cdot}} \& {\Ext^1(\Rfun(C_\tau F_t),Q)=0} \\\n{\phantom{\cdot}} \& {\varinjlim_{t}\, \Ttau\mathbf{P}_i(F_t)}
  \arrow[from=4-2, to=4-4, "\phi_{i,\tau}"]
  \& {\phantom{\cdot}} \& {\Ttau\mathbf{P}_i(F_\infty)}
    \arrow[from=4-4, to=5-2, bend right=25, dashed, swap, "{\text{log }(\muc,\nuc)}"] \\\n{\phantom{\cdot}} \& {\text{failure log (pure/mixed)}} \& {\phantom{\cdot}} \& {\phantom{\cdot}}
\end{tikzcd}
\end{center}

\subsection*{7.7. Minimal assumptions per arithmetic class (design templates)}
\begin{declaration}[Specification: Template hypotheses]\label{spec:templates}
For practical deployment, the following minimal templates ensure admissibility (one\hyp line concrete instances shown):
\begin{itemize}
  \item \textbf{(MM spaces from arithmetic data).} Index by conductor/level; realize transitions as \(1\)\hyp Lipschitz retractions between metric(\hyp measure) models (e.g.\ modular curves under level\hyp lowering with Gromov–Hausdorff \(1\)\hyp Lipschitz maps); preserve finite\hyp type per degree.
  \item \textbf{(Simplicial/complex models).} Use bounded\hyp degree subdivisions for level changes (e.g.\ barycentric refinement at fixed depth); ensure objectwise degreewise colimits; non\hyp expansiveness under each \(\mathbf{P}_i\).
  \item \textbf{(Graphs/quotients).} Sparsify while preserving normalized Laplacians and the \(1\)\hyp Lipschitz property of \(\mathcal{G}\)/\(\mathcal{A}\) (e.g.\ degree\hyp bounded sparsification of Cayley graphs); compute spectra on \(C_\tau F_t\).
\end{itemize}
\end{declaration}

\subsection*{7.8. Reproducibility and logs}
\begin{remark}[Run logs and parameters]\label{rk:logs}
For each run, log: the tower index range \(t\in[t_{\min},t_{\max}]\), the scale sweep \(\tau\in[\tau_{\min},\tau_{\max}]\) with step, spectral parameters \((\beta,M(\tau),t_{\mathrm{HT}})\), and the obstruction tuple \((\muc,\nuc)\) per \(\tau\) (with failure type).
Record also the degree set used for aggregation (per\hyp degree vs.\ summed across \(i\)) to ensure consistent replays.
These logs are part of the program specification and enable exact reruns.
For end\hyp to\hyp end validation scripts and datasets (class group and Selmer rank \(0/1\) scenarios), see Chapter~12 (Test Benches), which binds the logging format here with the executable test harness.
\end{remark}

\subsection*{7.9. Final guard-rails}
\begin{remark}[Scope and non-claims]\label{rk:nonclaims}
This chapter provides a design blueprint at the persistence/spectral/categorical layers for arithmetic towers.
It does \emph{not} assert number\hyp theoretic identities or decide deep conjectures; all forward\hyp looking statements are explicitly labeled \textbf{[Conjecture]} and rely on the implementable range and \LC.
No claim of \(\mathrm{PH}_1\Leftrightarrow\Ext^1\) is made; only the one\hyp way bridge under \textup{(B1)–(B3)} from Part~I is used.
\end{remark}

% -------------------- Reinforcements: height windows, PF/BC, commutativity, δ-ledger, gates --------------------

\subsection*{7.10. Height windows (MECE), PF/BC audit, and $\delta$-naturality}
\begin{definition}[Height windows and MECE partition]\label{def:height-windows}
Let the index set \(I\) carry a \emph{height} function \(h:I\to \mathbb{R}\) (e.g.\ conductor/level/weight) that is non\hyp decreasing along transitions. A \emph{height windowing} is a MECE partition \(\{W_k=[u_k,u_{k+1})\}_k\) of the height range such that the subdiagram of indices \(\{t\in I:\ h(t)\in W_k\}\) is filtered. All audits, gates, and certificates are performed \emph{per window}.
\end{definition}

\begin{declaration}[PF/BC audit after collapse]\label{dec:pf-bc-audit}
For external comparison functors (Projection Formula/Base Change) denoted \(\mathrm{PF},\mathrm{BC}\) at the arithmetic layer, we \emph{first} pass to persistence and \emph{then} collapse:
\[
X_t\ \xrightarrow{\ \mathcal{A}\ }\ F_t\ \xrightarrow{\ \mathbf{P}_i\ }\ \mathbf{P}_i(F_t)\ \xrightarrow{\ \Ttau\ }\ \Ttau\mathbf{P}_i(F_t).
\]
Pseudonaturality and PF/BC equalities are checked \emph{after} \(\Ttau\), i.e.\ on \(\Ttau\mathbf{P}_i(F_t)\), uniformly on each window. Any mismatch is recorded in the \(\delta\)\hyp ledger as
\[
\delta^{\mathrm{alg}}_{\mathrm{PF/BC}}(i,\tau;W_k)\ :=\ d_{\mathrm{int}}\!\Big(\Ttau \mathbf{P}_i\big(\mathrm{PF/BC}\circ \mathcal{A}\big),\ \Ttau \mathbf{P}_i\big(\mathcal{A}\circ \mathrm{PF/BC}\big)\Big).
\]
Discretization and measurement contributions are added as \(\delta^{\mathrm{disc}},\delta^{\mathrm{meas}}\); the per\hyp window budget is \(\Sigma\delta(i)\) (cf.\ Chapter~5, Specification~\ref{spec:delta-ledger} and Chapter~6, Declaration~\ref{dec:delta-ledger-ch6}).
\end{declaration}

\begin{remark}[Mirror/level transfer and $\delta$-ledger]
Let \(\Mirror\) denote level transfer (e.g.\ norm, corestriction, specialization). We measure the $2$-cell defect \(\epsilon_{i,\tau}:\Mirror\circ C_\tau\Rightarrow C_\tau\circ \Mirror\) after collapse with bound \(\delta(i,\tau)\) as in Chapter~5, Definition~\ref{def:delta-2cell}, and add it additively to the window ledger. Pipeline additivity and $1$-Lipschitz post\hyp processing follow from Proposition~\ref{prop:pipeline-budget}.
\end{remark}

\subsection*{7.11. Commutativity and pseudonaturality tests after collapse}
\begin{declaration}[Pseudonaturality verification policy]\label{dec:pseudonat}
All naturality/compatibility diagrams involving level transfer, PF/BC, or auxiliary reflectors are verified \emph{on the collapsed persistence layer} (\(\Ttau\mathbf{P}_i(F_t)\)). This avoids pre\hyp collapse torsion noise and aligns the audit with the gate posture (B\hyp side single layer).
\end{declaration}

\begin{definition}[A/B commutativity test and fallback]\label{def:ab-test}
Given two persistence\hyp level reflectors \(T_A,T_B\) (e.g.\ length\hyp threshold and birth\hyp window) we define
\[
\Delta_{\mathrm{comm}}(M;A,B)\ :=\ d_{\mathrm{int}}\!\big(T_A T_B M,\ T_B T_A M\big).
\]
On each height window \(W_k\) we run the A/B test on \(M=\Ttau\mathbf{P}_i(F_t)\). If \(\Delta_{\mathrm{comm}}\le \eta\) (tolerance), we accept \emph{soft\hyp commuting} (Chapter~5, Definition~\ref{def:soft-commute}); otherwise we fix a deterministic order (e.g.\ \(T_B\circ T_A\)), and record \(\Delta_{\mathrm{comm}}\) into \(\delta^{\mathrm{alg}}\).
\end{definition}

\begin{remark}[Nested torsions and order independence]
If the Serre classes are nested, order independence holds and no A/B test is required (Chapter~5, Proposition~\ref{prop:nested-commute}); otherwise soft\hyp commuting governs adoption.
\end{remark}

\subsection*{7.12. Gate template for arithmetic windows and saturation usage}
\begin{enumerate}
  \item \textbf{Window selection.} Choose a height window \(W_k=[u_k,u_{k+1})\) (Definition~\ref{def:height-windows}); fix \(\tau\) inside a stable band (Chapter~4, Definition~\ref{def:stable-band}).
  \item \textbf{Collapse then measure.} For each \(t\) with \(h(t)\in W_k\), compute \(\Ttau\mathbf{P}_i(F_t)\), energies \(\mathrm{PE}_i^{\le\tau}\), spectral indicators on \(L(C_\tau F_t)\), and (if in scope) \(\Ext^1(\Rfun(C_\tau F_t),k)\).
  \item \textbf{$\delta$ logging.} Audit PF/BC and Mirror transfer after collapse (Declaration~\ref{dec:pf-bc-audit}); run A/B tests (Definition~\ref{def:ab-test}); accumulate \(\Sigma\delta(i)\).
  \item \textbf{B\hyp Gate$^{+}$.} Require: \(\mathrm{PH}_1(C_\tau F_t)=0\), \((\mu,\nu)=(0,0)\) per window (Declarations~\ref{spec:tower-stability}), Ext\(^1\) pass (if checked), and safety margin \(\mathrm{gap}_\tau>\Sigma\delta(i)\).
  \item \textbf{Certificate \& paste.} Issue the window certificate; paste across windows via Restart and Summability (Chapter~4, Lemma~\ref{lem:restart}, Definition~\ref{def:summability}).
\end{enumerate}
On windows declared \emph{saturated} (Chapter~11), the gate may reference the saturation criteria directly.

\subsection*{7.13. Compliance checklist (arithmetic run)}
\begin{enumerate}
  \item Height windows form a MECE partition; coverage log recorded.
  \item \(\tau\)\hyp sweep and stable bands documented; spectral parameters fixed.
  \item PF/BC and Mirror audits executed \emph{after collapse}; \(\delta^{\mathrm{alg}},\delta^{\mathrm{disc}},\delta^{\mathrm{meas}}\) ledger complete.
  \item A/B commutativity tests run per window; soft\hyp commuting adopted or deterministic order fixed with \(\Delta_{\mathrm{comm}}\) logged.
  \item B\hyp side only measurements; B\hyp Gate$^{+}$ passed with safety margin; \((\muc,\nuc)=(0,0)\) per window.
  \item Certificates issued and pasted with Restart/Summability; failure types logged if any.
\end{enumerate}


\subsection*{7.14. Mirror/Transfer on arithmetic towers: 2\hyp cell defect, Control\(\to\)Overlap Gate, additivity, and non\hyp increase (IMRN/AiM)}

\begin{definition}[Mirror/Transfer on arithmetic towers]\label{def:arith-mirror}
Let \(\Mirror:\FiltCh{k}\to\FiltCh{k}\) be a functor representing a level transfer (norm/corestriction/specialization) on arithmetic towers. We assume:
\begin{itemize}
  \item \textbf{(M1) Non\hyp expansiveness at persistence:} for all \(i\), \(d_{\mathrm{int}}(\mathbf{P}_i(\Mirror F),\mathbf{P}_i(\Mirror G))\le d_{\mathrm{int}}(\mathbf{P}_i(F),\mathbf{P}_i(G))\).
  \item \textbf{(M2) 2\hyp cell after collapse:} there exists a natural \(2\)\hyp cell \(\epsilon_{i,\tau}:\Mirror\circ C_\tau\Rightarrow C_\tau\circ \Mirror\) with uniform bound \(\delta(i,\tau)\ge 0\) in \(d_{\mathrm{int}}\), invariant under f.q.i.
\end{itemize}
\end{definition}

\begin{proposition}[Mirror × Collapse: additivity and non\hyp increase]\label{prop:arith-mirror-pipeline}
Let \(U_m,\dots,U_1\) be A\hyp side steps (each deletion\hyp type or \(\varepsilon\)\hyp continuation), interlaced with collapses \(C_{\tau_j}\). Under \textup{(M1)}–\textup{(M2)} and Chapter~5, Proposition~\ref{prop:pipeline-budget}, for any fixed \(\tau\) and degree \(i\),
\[
d_{\mathrm{int}}\!\Big(\Ttau\mathbf{P}_i(\Mirror(C_{\tau_m}U_m\cdots C_{\tau_1}U_1F)),\ \Ttau\mathbf{P}_i(C_{\tau_m}U_m\cdots C_{\tau_1}U_1\Mirror F)\Big)\ \le\ \sum_{j=1}^m \delta_j(i,\tau_j),
\]
and $1$\hyp Lipschitz post\hyp processing (including PF/BC comparators of §7.10) does not increase the right\hyp hand side.
\end{proposition}

\begin{proof}[Proof sketch]
Compose the natural \(2\)\hyp cells and apply the triangle inequality; use $1$\hyp Lipschitzness of \(\Ttau\) and post\hyp processors as in Chapter~5.
\end{proof}

\begin{remark}[Overlap Gate (persistence layer), recall]
We use the \emph{Overlap Gate} from Chapter~5 as the acceptance criterion that two persistence\hyp level pipelines agree up to a controlled finite defect after collapse, i.e.\ their outputs are isomorphic modulo a finite number of bars of length \(\le \tau\), with the total defect charged to the \(\delta\)\nobreakdash ledger.
\end{remark}

\begin{proposition}[Control\(\Rightarrow\)Overlap Gate; finite defect recorded]\label{prop:control-overlap-gate}
Assume a classical arithmetic Control Theorem on a tower segment (e.g.\ Mazur/Greenberg\hyp style) providing that the transfer map on arithmetic objects \(A_{t'}\to A_t\) is an isomorphism modulo finite kernel/cokernel uniformly on the segment. For an admissible realization \(\mathcal{A}\) and fixed \(\tau\),
the induced persistence comparison
\[
\phi_{i,\tau}:\ \varinjlim_{t}\ \Ttau\big(\mathbf{P}_i(F_t)\big)\ \longrightarrow\ \Ttau\big(\mathbf{P}_i(F_\infty)\big)
\]
is an isomorphism up to a \emph{finite} kernel/cokernel consisting of bars of length \(\le \tau\). Consequently, the Overlap Gate accepts the segment, and the total finite defect is recorded as an algebraic budget
\[
\delta^{\mathrm{alg}}_{\mathrm{Ctrl}}(i,\tau)\ :=\ \dim_k\ker\phi_{i,\tau}\ +\ \dim_k\mathrm{coker}\,\phi_{i,\tau}.
\]
The quantity \(\delta^{\mathrm{alg}}_{\mathrm{Ctrl}}(i,\tau)\) is stable under $1$\hyp Lipschitz post\hyp processing and invariant under f.q.i.\ of the filtered models.
\end{proposition}

\begin{proof}[Proof sketch]
Finite kernel/cokernel at the arithmetic layer is carried by \(\mathcal{A}\) to finite\hyp rank changes in \(F_t\). After collapse, these manifest as a finite multiset of bars of length \(\le \tau\); all other (generic\hyp fiber) summands match. The Overlap Gate accepts by definition; stability follows from $1$\hyp Lipschitzness and f.q.i.\ invariance.
\end{proof}

\begin{remark}[Window arithmetic comparator with Mirror]
Combine Proposition~\ref{prop:arith-mirror-pipeline} with Proposition~\ref{prop:control-overlap-gate} and the PF/BC audit (Declaration~\ref{dec:pf-bc-audit}) to certify window\hyp level comparators \emph{after collapse}; aggregate the budgets in the \(\delta\)\nobreakdash ledger.
\end{remark}

\subsection*{7.15. Tropical shortening at the arithmetic layer (\textbf{[Spec]})}
We encode a \emph{tropical} base contraction on arithmetic heights/regulators as a window\hyp level barcode shortener.

\begin{definition}[Tropical base contraction (\textbf{[Spec]})]\label{def:tropical}
Let \(\Trop_\lambda:\mathsf{ArithTower}\to\mathsf{ArithTower}\) be an endofunctor with parameter \(\lambda\in(0,1]\) such that the induced filtered map on \(\FiltCh{k}\) is non\hyp expansive under \(\mathbf{P}_i\) and, on each window \(W\) and threshold \(\tau\), \emph{uniformly shortens} degreewise barcodes by factor \(\kappa(\lambda',\lambda)\le 1\) (Definition~8.1 in spirit), up to f.q.i., after applying \(\Ttau\).
\end{definition}

\begin{proposition}[Window energy non\hyp increase under tropical shortening (\textbf{[Spec]})]\label{prop:tropical-energy}
Assume Definition~\ref{def:tropical}. Then for each degree \(i\) and window \(W\),
\[
\mathrm{PE}_i^{\le\tau}\big(C_\tau (\mathcal{A}\circ \Trop_{\lambda'})(X)\big)\ \le\ \mathrm{PE}_i^{\le\tau}\big(C_\tau (\mathcal{A}\circ \Trop_{\lambda})(X)\big)\qquad (\lambda'\le \lambda),
\]
with strict decrease whenever a positive portion of clipped bars are shortened by a factor \(<1\).
\end{proposition}

\begin{proof}[Proof sketch]
Shortening reduces clipped lengths inside \([0,\tau]\) up to f.q.i.; sum of clipped lengths (Definition~\ref{def:PE}) therefore decreases, cf.\ Theorem~8.1 in the geometric setting and Chapter~6, §6.1.
\end{proof}

\begin{remark}[Scope]
Tropical shortening is a \textbf{[Spec]} design tool: no number\hyp theoretic identity is invoked; it supplies a proxy to enforce monotone decay of window energies under controlled base contractions (e.g.\ level pruning).
\end{remark}

\subsection*{7.16. Weak group collapse (linear proxy) at fixed windows}
We define a window\hyp level \emph{weak group collapse} proxy that can be tested on arithmetic symmetry/transfer actions.

\begin{definition}[Barcode space and linearization]\label{def:barcode-space}
Fix a window \(W\) and threshold \(\tau\). For degree \(i\), write \(\Ttau\mathbf{P}_i(F)\cong \bigoplus_{b\in \mathcal{B}_{i,\tau}(F;W)} I_b\), and let
\[
V_{i,\tau}(W)\ :=\ \bigoplus_{b\in \mathcal{B}_{i,\tau}(F;W)} k\cdot e_b.
\]
For a groupoid \(\mathsf{Aut}(F)\) of filtered self\hyp maps at arithmetic level, any \(g\in \mathsf{Aut}(F)\) induces (after \(\Ttau\)) a linear map on \(V_{i,\tau}(W)\), well\hyp defined up to conjugacy.
\end{definition}

\begin{definition}[Weak group collapse (window\hyp level gate)]\label{def:weak-group}
Fix a finite set \(S\subset \mathsf{Aut}(F)\). We say \emph{weak group collapse holds at \((W,\tau)\)} if:
\begin{itemize}
  \item (Semi\hyp contraction) \(\mathrm{spr}(\rho_{i,\tau}(g))\le 1\) for all \(g\in S\), all \(i\) (spectral radius bound over an algebraic closure).
  \item (Bounded unipotent length) There is \(m\in\mathbb{N}\) with \((\rho_{i,\tau}(g)-I)^m=0\) for all \(g\in S\), uniformly in \(i\).
\end{itemize}
\end{definition}

\begin{proposition}[Acceptance criterion and stability]\label{prop:weak-group-accept}
If weak group collapse holds at \((W,\tau)\) and the tower diagnostics vanish \((\muc,\nuc)=(0,0)\), then the group action is semi\hyp contractive on the bar basis \emph{after collapse}, uniformly on \(W\), and B\hyp Gate$^{+}$ may adopt weak\hyp group\hyp collapse as an \emph{auxiliary} acceptance tag. The property is invariant under f.q.i.\ and stable under $1$\hyp Lipschitz post\hyp processing.
\end{proposition}

\begin{proof}[Proof sketch]
The linear proxy abstracts window\hyp level action on truncated bars; semi\hyp contraction and bounded unipotent length ensure no growth modes survive after collapse. Tower stability excludes Type~IV across the window. Invariance and stability follow from persistence\hyp level functoriality (Chapter~5) and the $1$\hyp Lipschitz policy.
\end{proof}

\begin{remark}[IMRN/AiM posture]
Weak group collapse does \emph{not} assert group trivialization; it is a linear, persistence\hyp level proxy on a fixed window and threshold, compatible with the budgeted pipeline and local gates. It fits the acceptance\hyp test toolbox, not a global equivalence claim.
\end{remark}

\subsection*{7.17. Summary (Mirror/Tropical/Langlands extensions, budgeted and windowed)}
We have consolidated the Mirror × Collapse \(2\)\hyp cell (with additive, non\hyp increasing bounds), a window\hyp level tropical shortening \textbf{[Spec]} that enforces energy non\hyp increase at fixed \(\tau\), and a weak group collapse proxy (semi\hyp contraction + bounded unipotent length) as auxiliary gate criteria. Each tool is \emph{windowed} and \emph{after collapse}, integrated with the PF/BC comparator (§7.10), the tower Defect calculus (§7.2), and the typed verdict of Chapter~6. All deviations are recorded in the \(\delta\)\hyp ledger; arithmetic decisions remain entirely at the persistence layer and within the constructible range, with the one\hyp way bridge only (Chapter~3). This unified policy delivers an IMRN/AiM\hyp ready, auditable program for arithmetic layers and Iwasawa\hyp style refinement that requires no further reinforcement beyond implementation details in Appendices~K–O.

\subsection*{7.18. The Iwasawa Gate: tri–state alignment between \(\mu_{\mathrm{Collapse}}\) and classical \(\mu\)}
We introduce a \emph{window–local} acceptance predicate that compares the persistence obstruction \(\mu_{\mathrm{Collapse}}\) with a classical Iwasawa \(\mu\)–proxy, without asserting any identity. The gate returns one of three outcomes.

\begin{definition}[Iwasawa Gate (tri–state)]\label{def:iwasawa-gate}
Fix a Denef–Pas height window \(W=[u,u')\) (Remark~\ref{rk:DP-windows}) and a threshold \(\tau\) in a stable band (Chapter~4, Definition~\ref{def:stable-band}).
Let \(\widehat{\mu}_{\mathrm{Iw}}(W)\) denote a \emph{classical \(\mu\)–proxy} extracted from the arithmetic tower on \(W\)
(e.g.\ the \(\mathbb{Z}_p\)–length growth rate of a chosen \(\Lambda\)–torsion module restricted to \(W\), computed by standard control on the segment; no identity with \(\muc\) is assumed).
Define:
\[
\mathrm{Gate}_{\mathrm{Iw}}(W,\tau)\ \in\ \{\mathrm{Aligned},\ \mathrm{Inconclusive},\ \mathrm{Misaligned}\},
\]
by the following budgeted, post–collapse rules:
\begin{itemize}
  \item \textbf{Aligned.} On \(W\), the sign pattern and window–wise monotonic trend of \(\muc(\tau)\) match those of \(\widehat{\mu}_{\mathrm{Iw}}(W)\) within the declared tolerance \(\eta_{\mathrm{Iw}}\), and the tower diagnostics vanish: \((\muc,\nuc)=(0,0)\).
  \item \textbf{Inconclusive.} Either \((\muc,\nuc)\neq(0,0)\) but can be driven to zero by mesh refinement/\(\tau\)–refinement within the existing \(\delta\)–budget, or the proxy is undefined on a subwindow due to missing arithmetic inputs.
  \item \textbf{Misaligned.} The sign/monotone pattern conflicts beyond tolerance after budgeted post–processing, or Type~IV persists on \(W\).
\end{itemize}
All comparisons are made \emph{after collapse} on \(\Ttau\mathbf{P}_i(F_t)\), with all drifts accounted for in the quantale–valued \(\delta\)–ledger (Chapter~5, Specification~\ref{spec:delta-ledger}; Chapter~6, Declaration~\ref{dec:delta-ledger-ch6}).
\end{definition}

\begin{figure}[t]
\centering
\begin{tikzpicture}[>=latex, node distance=12mm]
\tikzstyle{box}=[draw, rounded corners, align=center, inner sep=3pt, font=\small]
\node[box] (start) {Pick Denef--Pas window \(W\)\\ and \(\tau\) in a stable band};
\node[box, below=of start] (diag0) {Compute \((\muc,\nuc)\)\\ and \(\widehat{\mu}_{\mathrm{Iw}}(W)\)};
\node[box, below left=12mm and 20mm of diag0] (aligned) {\textbf{Aligned}\\ patterns match within \(\eta_{\mathrm{Iw}}\)\\ and \((\muc,\nuc)=(0,0)\)};
\node[box, below=12mm of diag0] (inc) {\textbf{Inconclusive}\\ refine mesh/\(\tau\) or inputs};
\node[box, below right=12mm and 20mm of diag0] (mis) {\textbf{Misaligned}\\ conflict beyond tolerance\\ or persistent Type~IV};
\draw[->] (start) -- (diag0);
\draw[->] (diag0.west) -- ++(-13mm,0) |- (aligned.north);
\draw[->] (diag0) -- (inc);
\draw[->] (diag0.east) -- ++(13mm,0) |- (mis.north);
\end{tikzpicture}
\caption{Iwasawa Gate: tri–state alignment on a Denef–Pas window, post–collapse and budgeted.}
\label{fig:iwasawa-gate}
\end{figure}

\begin{remark}[Scope and guarantees]
The gate is \emph{diagnostic}: it does not assert any equality \(\muc=\widehat{\mu}_{\mathrm{Iw}}\).
``Aligned'' certifies window–level agreement of \emph{trends} under stable–band selection and vanishing tower defects; ``Misaligned'' flags a genuine arithmetic/persistence discrepancy or unresolved Type~IV; ``Inconclusive'' directs refinement or input completion.
\end{remark}

\subsection*{7.19. Denef–Pas windows: adoption and cross–reference}
We \emph{adopt} Denef–Pas definable height windows whenever arithmetic parameters admit such descriptions (Appendix~Q).
This yields: (i) finite event sets per bounded window; (ii) finite Čech depth for window–wise gluing; (iii) a uniform setting for PF/BC and Mirror $2$–cells \emph{after collapse}.
All invocations of ``definable windows'' in §7 are to be read in this Denef–Pas sense unless explicitly stated otherwise.

\subsection*{7.20. Test \texttt{T–Iwasawa–Alignment} (new)}
We add a machine–checkable test that implements Definition~\ref{def:iwasawa-gate}.

\begin{specification}[\texttt{T–Iwasawa–Alignment}]\label{spec:T-Iwasawa}
\emph{Inputs:}
a height window \(W=[u,u')\) (Denef–Pas definable), a threshold \(\tau\) in a stable band, a tolerance \(\eta_{\mathrm{Iw}}\in \mathbb{R}_{\ge 0}\), and a classical \(\mu\)–proxy \(\widehat{\mu}_{\mathrm{Iw}}(W)\) (if available).

\emph{Checks (all after collapse):}
\begin{enumerate}
  \item \textbf{Tower stability:} \((\muc,\nuc)=(0,0)\) on \(W\) (pointwise in \(\tau\) if sweeping).
  \item \textbf{Trend comparison:} the monotone trend (non–increase/non–decrease) and sign pattern of \(\muc(\tau)\) match those of \(\widehat{\mu}_{\mathrm{Iw}}(W)\) within \(\eta_{\mathrm{Iw}}\).
  \item \textbf{Budget dominance:} safety margin \(\mathrm{gap}_\tau>\Sigma\delta(i)\) for the monitored degrees; PF/BC and Mirror $2$–cell defects included.
\end{enumerate}

\emph{Output:}
\(\mathrm{Gate}_{\mathrm{Iw}}(W,\tau)\in\{\mathrm{Aligned},\mathrm{Inconclusive},\mathrm{Misaligned}\}\) as in Definition~\ref{def:iwasawa-gate}, with the \(\delta\)–ledger excerpt and the stable–band certificate attached.

\emph{Usage:}
attach \texttt{T–Iwasawa–Alignment} to the windowed checklist (§7.13): if ``Aligned'', annotate the window certificate with an ``Iw–aligned'' tag; if ``Inconclusive'', trigger refinement; if ``Misaligned'', raise an arithmetic comparator warning.
\end{specification}

\subsection*{7.21. Amendments to the windowed checklist}
In §7.13 (Compliance checklist), insert:
``(7) If a Denef–Pas window and a \(\mu\)–proxy are declared, run \texttt{T–Iwasawa–Alignment} (Spec.~\ref{spec:T-Iwasawa}) and record the tri–state outcome and budgets.''

\subsection*{7.22. Summary addendum}
The Iwasawa Gate provides a \emph{tri–state}, window–local diagnostic linking \(\mu_{\mathrm{Collapse}}\) to a classical \(\mu\)–proxy without asserting any identity. Denef–Pas windows (Appendix~Q) supply the definability and finiteness needed for auditable, nerve–level checks; the new test \texttt{T–Iwasawa–Alignment} integrates with the existing PF/BC/Mirror comparators, stable–band selection, and the quantale \(\delta\)–ledger. All decisions remain after collapse on the B–side single layer and within the constructible range, respecting the one–way bridge of Part~I.



\section{Chapter 8: Mirror/Tropical Collapse (Weak Group Collapse)}
\addcontentsline{toc}{section}{Mirror/Tropical Collapse (Weak Group Collapse)}
\label{sec:ch8}

\noindent\textbf{Windowed policy and B-side judgement.}
All decisions in this chapter are made \emph{after collapse} on the B-side, i.e.\ on single-layer objects
\(\Ttau\mathbf{P}_i\) (equivalently \(\mathbf{P}_i(C_\tau-)\)).
Filtered-complex statements are always taken \emph{up to filtered quasi-isomorphism} (f.q.i.; Appendix~B).
All equalities and Lipschitz bounds are asserted \emph{only at the persistence layer} in the constructible range.

\begin{remark}[Monotonicity convention]\label{rk:8-mono}
We adopt Chapter~6, Remark~\ref{rem:stability-vs-monotonicity}:
\emph{deletion-type} updates are non-increasing for spectral tails and windowed energies;
\emph{inclusion-type} updates are stable (non-expansive).
See Appendix~E.
\end{remark}

\subsection*{8.0. Standing hypotheses and post-collapse policy}
Fix a field \(k\) and work within the \emph{implementable range} of Part~I.
Identify \(\Pers^{\mathrm{ft}}_k\) with the constructible subcategory (Chapter~6).
Let \(\FiltCh{k}\) be finite-type filtered chain complexes, and \(\mathbf{P}_i:\FiltCh{k}\to\Pers^{\mathrm{cons}}_k\) the degreewise persistence functor.
Write \(\Ttau:=\mathbf{T}_\tau\) for the Serre (bar-deletion) reflector at scale \(\tau\ge 0\); its filtered lift \(\Ctau\) is used up to f.q.i.\ (Chapter~2, §§2.2–2.3).
A fixed \(t\)-exact realization \(\Rfun:\FiltCh{k}\to D^{\mathrm{b}}(k\text{-mod})\) is retained, and \LC\ holds whenever \(\Ctau\) is compared with \(\tau_{\ge 0}\!\circ\!\Rfun\).
Endpoint conventions and infinite bars follow Chapter~2, Remark~\ref{rk:2-endpoints}.
Kernel/cokernel diagnostics \((\mu_{\mathrm{Collapse}},\nu_{\mathrm{Collapse}})\) at scale \(\tau\) are computed from comparison maps as in Chapter~4, §4.2, with \(\dim_k\) interpreted as generic-fiber dimension after truncation (multiplicity of \(I[0,\infty)\)); see Appendix~D, Remark~\ref{rem:D-generic-dim}.

\medskip
\noindent\textbf{Quantitative commutation and the product-ledger quantale (P7).}
We fix a commutative unital quantale \((\mathsf{Q},\otimes,\mathbf{1},\le)\) to aggregate \(2\)-cell defects.
The \emph{product-ledger} policy is standard:
budgets \(\delta\in\mathsf{Q}\) compose multiplicatively via \(\otimes\) along pipelines and are compared using \(\le\).
Default: \(\mathsf{Q}=\overline{\mathbb{R}}_{\ge 0}\) with \(\otimes=+\), \(\mathbf{1}=0\) (Lawvere order).
A tolerance \(\eta\in\mathsf{Q}\) is fixed per window.

\begin{declaration}[Post-collapse Non-expansion Policy (P4)]\label{dec:post-collapse-policy}
All comparisons, gates, and error budgets are evaluated \emph{after} applying \(\Ttau\).
Let \(L:\FiltCh{k}\to\FiltCh{k}\) be non-expansive degreewise for each \(\mathbf{P}_i\) and admit a natural \(2\)-cell
\(\theta^L:L\circ C_\tau\Rightarrow C_\tau\circ L\) (up to f.q.i.).
Then, for all degrees \(i\),
\[
  d_{\mathrm{int}}\!\big(\Ttau\mathbf{P}_i(LF),\,\Ttau\mathbf{P}_i(LG)\big)\ \le\ d_{\mathrm{int}}\!\big(\mathbf{P}_i(F),\,\mathbf{P}_i(G)\big),
\]
and for the same \(F\),
\[
  d_{\mathrm{int}}\!\big(\Ttau\mathbf{P}_i\big((L\!\circ\! C_\tau)F\big),\,\Ttau\mathbf{P}_i\big((C_\tau\!\circ\! L)F\big)\big)\ \le\ \delta^L(i,\tau)\in\mathsf{Q}.
\]
All bounds are one-sided ``\(\le\)'' (safe-side only). All equalities are at persistence; filtered-level statements hold up to f.q.i.
\end{declaration}

\begin{remark}[No pre-collapse observation]\label{rk:8-no-pre}
Intermediate pre-collapse observations are not used for decisions.
Only post-collapse quantities \(\Ttau\mathbf{P}_i(-)\) and their indicators enter gates and budgets.
\end{remark}

We consider admissible realizations (Chapter~6, Definition~\ref{def:geom-real}; Chapter~7, Definition~\ref{def:arith-real})
\[
  \mathsf{Geom}_A \xrightarrow{\ \mathcal{G}_A\ } \FiltCh{k},\qquad
  \mathsf{Geom}_B \xrightarrow{\ \mathcal{G}_B\ } \FiltCh{k}.
\]
A \emph{tropical base contraction} at parameter \(\lambda\in(0,1]\) is an endofunctor
\(\Trop_\lambda:\mathsf{Geom}_A\to \mathsf{Geom}_A\)
whose induced filtered map on \(F:=\mathcal{G}_A(X)\) is non-expansive under each \(\mathbf{P}_i\) and monotone (deletion-type) as \(\lambda\searrow 0\).
A \emph{mirror transfer} is a functor \(\Mirror:\FiltCh{k}\to\FiltCh{k}\) that is non-expansive for each \(\mathbf{P}_i\), compatible with \(\Ctau\) up to f.q.i., and subject to \LC\ for comparisons after realization \(\Rfun\).

\begin{remark}[Endpoints and infinite bars]\label{rk:8-endpoints}
All statements are insensitive to open/closed endpoints. Infinite bars are not removed by \(\Ttau\); windowed indicators clip their contributions (Chapter~6).
\end{remark}

\begin{remark}[Cone extension for the tropical flow]\label{rk:8-cone}
For a directed parameter set \(\Lambda\subset(0,1]\) with \(\lambda'\le \lambda\), adjoin a terminal element \(\lambda_\ast\) (representing \(\lambda\to 0\)) and cone maps \(\Trop_{\lambda}\Rightarrow \Trop_{\lambda_\ast}\).
Under \(\mathcal{G}_A\), these induce filtered maps \(F_\lambda\to F_{\lambda_\ast}\), providing the comparison maps used to compute \((\mu_{\mathrm{Collapse}},\nu_{\mathrm{Collapse}})\) at fixed \(\tau\) along the \(\lambda\)-tower (Chapter~4).
\end{remark}

\subsection*{8.1. Tropical contraction and barcode shortening}
\begin{definition}[Uniform shortening proxy \textbf{[Spec]}]\label{def:shorten}
Let \(F\in\FiltCh{k}\) and fix \(\tau\ge 0\).
A filtered map \(F\to F'\) \emph{uniformly shortens} degreewise barcodes at factor \(\kappa\in(0,1]\) \emph{up to f.q.i.} if, for every \(i\), the multiset of lengths in \(\Ttau\mathbf{P}_i(F')\) is obtained from that of \(\Ttau\mathbf{P}_i(F)\) by multiplying lengths by \(\le \kappa\) and possibly deleting some bars, modulo f.q.i.
Infinite bars are unaffected; shortening is enforced only within the monitored window \([0,\tau]\) after \(\Ttau\).
The factor may depend on \(i,\tau\) (write \(\kappa_i(\tau)\)).
\end{definition}

\begin{declaration}[Specification: Tropical reduction vs.\ barcode shortening]\label{spec:trop-short}
Within the implementable range, \(\Trop_{\lambda'\le\lambda}\) induces filtered maps
\[
  \mathcal{G}_A(\Trop_{\lambda'}X)\longrightarrow \mathcal{G}_A(\Trop_{\lambda}X)
\]
that uniformly shorten degreewise barcodes at a factor \(\kappa(\lambda',\lambda)\le 1\) up to f.q.i.
Consequently, for fixed \(\tau\), the truncated energies \(\mathrm{PE}_i^{\le\tau}\) are non-increasing along \(\lambda\searrow 0\), and strictly decrease whenever \(\kappa(\lambda',\lambda)<1\) on a subset of bars whose cumulative length within the \(\tau\)-window has positive proportion of the total windowed bar length.
\end{declaration}

\subsection*{8.2. Weak group collapse: linear proxies on automorphism groupoids}
\begin{definition}[Automorphism groupoid and linear proxies]\label{def:group-proxy}
Let \(\mathsf{Aut}(F)\) be the groupoid of filtered self-maps of \(F\) in \(\FiltCh{k}\).
Fix \(\tau\) and \(i\).
Choose an interval-decomposition (up to f.q.i.) of \(\Ttau\mathbf{P}_i(F)\cong\bigoplus_{b\in \mathcal{B}_{i,\tau}(F)} I_b\).
Define the barcode vector space \(V_{i,\tau}:=\bigoplus_{b\in \mathcal{B}_{i,\tau}(F)} k\cdot e_b\).
Any \(g\in \mathsf{Aut}(F)\) induces an automorphism of \(\Ttau\mathbf{P}_i(F)\), hence (after choosing a decomposition) a linear map on \(V_{i,\tau}\), well-defined up to conjugacy:
\[
  \rho_{i,\tau}:\ \mathsf{Aut}(F)\to \mathsf{GL}(V_{i,\tau}).
\]
For a finite generating set \(S\subset \mathsf{Aut}(F)\) set
\[
  \rho_{\max,i,\tau}(S):=\sup_{g\in S}\ \mathrm{spr}\big(\rho_{i,\tau}(g)\big),\qquad
  \mathrm{nilp}_{i,\tau}(S):=\min\{m\ge 0\mid (\rho_{i,\tau}(g)-I)^m=0\ \forall g\in S\}.
\]
We say \(F\) has \emph{weak group collapse (WGC) at scale \(\tau\)} if for some finite \(S\),
\(\rho_{\max,i,\tau}(S)\le 1\) for all \(i\) and \(\sup_i \mathrm{nilp}_{i,\tau}(S)<\infty\).
\end{definition}

\begin{remark}[Meaning and invariance]\label{rk:group-meaning}
“Non-expansive’’ means the induced maps on \(\Ttau\mathbf{P}_i(\cdot)\) are \(1\)-Lipschitz for \(d_{\mathrm{int}}\) at the fixed \(\tau\).
The numbers \(\rho_{\max,i,\tau}(S),\mathrm{nilp}_{i,\tau}(S)\) are conjugacy invariants, well-defined up to f.q.i.
\end{remark}

\begin{remark}[Auxiliary tag]\label{rk:wgc-aux}
WGC is a \emph{persistence-level linear proxy} and used \emph{only as an auxiliary tag} on a fixed window \([0,\tau]\).
It is \emph{not} a gate criterion.
\end{remark}

\subsection*{8.3. Post-collapse transfer, Mirror, and δ-budget}
\begin{declaration}[Non-expansion and 2-cell bounds (post-collapse)]\label{spec:mirror}
Let \(L:\FiltCh{k}\to\FiltCh{k}\) be non-expansive degreewise for each \(\mathbf{P}_i\) and admit a natural transformation \(\epsilon_{i,\tau}:L\circ C_\tau \Rightarrow C_\tau\circ L\) up to f.q.i., with persistence-level bound \(\delta^L(i,\tau)\in\mathsf{Q}\).
Then for all \(i\),
\[
  d_{\mathrm{int}}\!\Big(\Ttau\mathbf{P}_i\big((L\!\circ\! C_\tau)F\big),\ \Ttau\mathbf{P}_i\big((C_\tau\!\circ\! L)F\big)\Big)\ \le\ \delta^L(i,\tau).
\]
Post-processing by \(1\)-Lipschitz persistence maps (degree projections \(\mathbf{P}_i\), shifts \(S^\varepsilon\), further truncations \(\mathbf{T}_{\tau'}\)) does not increase this bound.
\end{declaration}

\begin{theorem}[Pipeline error budget (product-ledger, P7)]\label{thm:pipe-budget-8}
Let \(U_m,\dots,U_1\) be A-side steps (each deletion-type or \(\varepsilon\)-continuation) with interleaved collapses \(C_{\tau_j}\).
Let \(L\) be non-expansive with \(2\)-cells \(\epsilon_{i,\tau_j}\) bounded by \(\delta^L(i,\tau_j)\in\mathsf{Q}\).
Fix a B-side threshold \(\tau\) and degree \(i\).
Then
\[
d_{\mathrm{int}}\!\Big(\Ttau \mathbf{P}_i\big(L(C_{\tau_m}U_m\cdots C_{\tau_1}U_1F)\big),\ \Ttau \mathbf{P}_i\big(C_{\tau_m}U_m\cdots C_{\tau_1}U_1(LF)\big)\Big)\ \le\ \bigotimes_{j=1}^m \delta^L(i,\tau_j).
\]
Under the default \(\mathsf{Q}=\overline{\mathbb{R}}_{\ge 0}\) this reads ``\(\le \sum_{j}\delta^L(i,\tau_j)\)''.
\end{theorem}

\begin{remark}[Soft-commuting and \(\delta^{\mathrm{alg}}\) accounting]\label{rk:soft-comm-8}
For reflectors \(T_A,T_B\), test
\(\Delta_{\mathrm{comm}}:=d_{\mathrm{int}}(T_AT_BM,\ T_BT_AM)\) on \(M=\Ttau\mathbf{P}_i(F)\) per window/degree.
If \(\Delta_{\mathrm{comm}}\le \eta\), adopt soft-commuting; else fix an order and add \(\Delta_{\mathrm{comm}}\) to \(\delta^{\mathrm{alg}}(i,\tau)\in\mathsf{Q}\).
\end{remark}

\begin{remark}[Mirror bounds (safe-side, P4)]\label{rk:mirror-quant}
For \(L=\Mirror\) with \(2\)-cell bound \(\delta^{\mathrm{Fun}}(i,\tau)\in\mathsf{Q}\),
\[
  d_{\mathrm{int}}\!\big(\Ttau\mathbf{P}_i((\Mirror\!\circ\! C_\tau)F),\ \Ttau\mathbf{P}_i((C_\tau\!\circ\!\Mirror)F)\big)\ \le\ \delta^{\mathrm{Fun}}(i,\tau).
\]
For \(F,G\),
\[
  d_{\mathrm{int}}\!\big(\Ttau\mathbf{P}_i((\Mirror\!\circ\! C_\tau)F),\ \Ttau\mathbf{P}_i((C_\tau\!\circ\!\Mirror)G)\big)
  \ \le\ d_{\mathrm{int}}\!\big(\mathbf{P}_i(F),\,\mathbf{P}_i(G)\big) \ \oplus\ \delta^{\mathrm{Fun}}(i,\tau),
\]
where \(\oplus\) denotes \(\otimes\) in \(\mathsf{Q}\) (sum under the default).
\end{remark}

\begin{conjecture}[Mirror correspondences under collapse monitoring]\label{conj:mirror}
Assuming \((\mu_{\mathrm{Collapse}},\nu_{\mathrm{Collapse}})=(0,0)\) and \LC, mirror correspondences preserve the monitored indicators across \(\Mirror\) on the same \(\tau\)-range.
WGC (Definition~\ref{def:group-proxy}) propagates as an \emph{auxiliary tag}.
\end{conjecture}

\subsection*{8.3.1.\ Spec–Saturation gate (tropical/mirror)}
\begin{declaration}[Saturation gate \textbf{[Spec]} (tropical/mirror)]\label{gate:8-saturation}
Fix \(\tau^\ast>0\), tolerance \(\eta>0\), and edge gap
\(\mathrm{Gap}:=\tau^\ast-\max\{\, b_r<\tau^\ast \,\}>0\).
On \([0,\tau^\ast]\), assume:
(i) eventually the maximal \emph{finite} bar length in \(\T_{\tau^\ast}\mathbf{P}_i(F_\lambda)\) is \(\le \eta\);
(ii) eventually \(d_{\mathrm{int}}(\T_{\tau^\ast}\mathbf{P}_i(F_\lambda),\T_{\tau^\ast}\mathbf{P}_i(F_{\lambda'}))\le \eta\);
(iii) \(\mathrm{Gap}>\eta\).
Then, \textbf{within this window only}, temporarily adopt
\[
\mathrm{PH}_1(\C_{\tau^\ast}F_\lambda)=0\quad\Longleftrightarrow\quad \Ext^1(\Rfun(\C_{\tau^\ast}F_\lambda),k)=0.
\]
Quantitative verification and usage are centralized in Chapter~11.
\end{declaration}

\subsection*{8.4. Monitoring protocol for tropical/mirror flows}
\begin{declaration}[Specification: Monitoring protocol]\label{spec:prot}
Fix a sweep \(\lambda\searrow 0\) and finite scales \(\tau\).
For each sample:
\begin{enumerate}
  \item Compute \(\Ttau\mathbf{P}_i(\mathcal{G}_A(\Trop_\lambda X))\) and \(\mathrm{PE}_i^{\le\tau}\) on truncated barcodes (equivalently on \(\Ctau\)).
  \item Compute spectral indicators \(\mathrm{ST}_{\beta}^{\ge M(\tau)}\), \(\mathrm{HT}(t;\cdot)\) on \(L(\Ctau(\mathcal{G}_A(\Trop_\lambda X)))\) with fixed \((\beta,M(\tau),t)\).
  \item Check \(\Ext^1(\Rfun(\Ctau{-}),Q)=0\) for \(Q\in\Qtest\).
  \item Evaluate \((\mu_{\mathrm{Collapse}},\nu_{\mathrm{Collapse}})\) from the \(\lambda\)-tower at \(\tau\) (Remark~\ref{rk:8-cone}).
  \item (Auxiliary) Choose finite \(S\subset\mathsf{Aut}(\mathcal{G}_A(\Trop_\lambda X))\) and record \(\rho_{\max,i,\tau}(S)\), \(\mathrm{nilp}_{i,\tau}(S)\) on \(V_{i,\tau}\) (WGC-tag).
  \item Apply \(\Mirror\) and repeat (1)–(5); compare via the \(\delta^{\mathrm{Fun}}(i,\tau)\) budget.
\end{enumerate}
The \emph{stable regime} at \(\tau\) is where \((\mu_{\mathrm{Collapse}},\nu_{\mathrm{Collapse}})=(0,0)\) and (1)–(3) are non-increasing along \(\lambda\searrow 0\).
\end{declaration}

\subsection*{8.5. Diagram (post-collapse indicators and mirror transfer)}
\begin{center}
% spacing knobs
\def\hA{5.0em}\def\hC{5.0em}\def\hB{5.0em}\def\vA{5.0em}\def\vB{9.0em}
\begin{tikzpicture}[baseline,>=latex]
\tikzset{box/.style={draw,rounded corners,inner sep=4pt},
         widebox/.style={box,inner xsep=6pt,inner ysep=4pt,minimum width=8em,align=center},
         lblUp/.style={midway,above=0.45em},
         lblDn/.style={midway,below=0.45em}}
\node (X) {$X$};
\node (F)       [right=\hA of X,box] {$F=\mathcal{G}_A(X)$};
\node (PiRaw)   [right=\hA of F,box] {$\mathbf{P}_i(F)$};
\node (PiF)     [right=\hC of PiRaw,box] {$\Ttau\mathbf{P}_i(F)$};
\node (MirPiF)  [right=\hB of PiF,box] {$\Ttau\mathbf{P}_i(\Mirror F)$};
\node (PEF)     [below=\vA of PiF,widebox]   {$\mathrm{PE}_i^{\le\tau}$\\$\mathrm{ST}_\beta,\ \mathrm{HT}$};
\node (PEMir)   [below=\vA of MirPiF,widebox]{$\mathrm{PE}_i^{\le\tau}$\\$\mathrm{ST}_\beta,\ \mathrm{HT}$};
\node (Ext)     [below=\vB of PiF,box] {$\Ext^1(\Rfun(\Ctau F),Q)$};
\draw[->,shorten >=0.2em,shorten <=0.2em] (X) -- node[lblUp] {$\mathcal{G}_A$} (F);
\draw[->,shorten >=0.2em,shorten <=0.2em] (F) -- node[lblUp] {$\mathbf{P}_i$} (PiRaw);
\draw[->,shorten >=0.2em,shorten <=0.2em] (PiRaw) -- node[lblDn] {$\Ttau$} (PiF);
\draw[->,shorten >=0.2em,shorten <=0.2em] (PiF) -- node[lblUp] {$\Mirror$} (MirPiF);
\draw[->,dashed,shorten >=0.2em,shorten <=0.2em] (PiF) -- (PEF);
\draw[->,dashed,shorten >=0.2em,shorten <=0.2em] (MirPiF) -- (PEMir);
\draw[->,dashed,shorten >=0.2em,shorten <=0.2em] (F) to[bend right=18] node[left] {$\Ctau$} (Ext);
\end{tikzpicture}
\end{center}

\subsection*{8.6. Toy instance (persistence layer)}
\begin{example}[Uniform shortening under tropical scaling]\label{ex:trop}
Let \(\mathbf{P}_i(F)\) have lengths \(\{\ell_j\}_j\) in \([0,\tau]\).
Suppose \(\Trop_\lambda\) induces \(\ell_j\mapsto \ell'_j\le \kappa\ell_j\) for a fixed \(\kappa<1\) on a subset \(S\) whose cumulative \(\tau\)-clipped length is positive, and possibly deletes other bars.
Then \(\mathrm{PE}_i^{\le\tau}\) strictly decreases.
If additionally \((\mu_{\mathrm{Collapse}},\nu_{\mathrm{Collapse}})=(0,0)\) at \(\tau\), one may annotate with a WGC-tag (Definition~\ref{def:group-proxy}); this tag is not used for gating.
\end{example}

\subsection*{8.7. Final guard-rails}
\begin{remark}[Scope and non-claims]\label{rk:8-guard}
All claims are at the persistence/spectral/categorical layers in the implementable range, with \LC\ when comparing after realization.
No group-theoretic trivialization is asserted; WGC is an auxiliary tag.
No claim of \(\mathrm{PH}_1\Leftrightarrow\Ext^1\) is made; only the one-way bridge of Part~I is used.
The obstruction \(\mu_{\mathrm{Collapse}}\) is unrelated to the classical Iwasawa \(\mu\)-invariant.
\end{remark}

\subsection*{8.8. Langlands tri-layer gates (Galois $\to$ Transfer $\to$ Functorial)}
We formalize gate criteria across a three-layer pipeline, each verified \emph{after collapse} on \(\Ttau\mathbf{P}_i(-)\), window by window.

\begin{definition}[Layer functors and objects]\label{def:layers}
Let \(F\in\FiltCh{k}\) and fix \(\tau\).
\begin{enumerate}
  \item Galois layer: a subgroup \(G\subset \mathsf{Aut}(F)\) acts by filtered maps. After collapse, \(G\) acts on \(V_{i,\tau}\) via \(\rho_{i,\tau}\) (Definition~\ref{def:group-proxy}).
  \item Transfer layer: a finite family \(\Trans=\{T_a:\FiltCh{k}\to\FiltCh{k}\}_a\) of non-expansive transfers (norm, corestriction, Hecke, BC/PF adapters), each with a \(2\)-cell \(T_a\circ C_\tau\Rightarrow C_\tau\circ T_a\) bounded by \(\delta^{\mathrm{Tr}}_a(i,\tau)\in\mathsf{Q}\).
  \item Functorial layer: a non-expansive \(\Funct:\FiltCh{k}\to\FiltCh{k}\) (e.g.\ Mirror, Langlands lift) with a \(2\)-cell \(\Funct\circ C_\tau\Rightarrow C_\tau\circ \Funct\) bounded by \(\delta^{\mathrm{Fun}}(i,\tau)\in\mathsf{Q}\).
\end{enumerate}
\end{definition}

\begin{definition}[Layer collapse maps and kernels]\label{def:layer-kernels}
For each \(i,\tau\):
\begin{itemize}
  \item Galois kernel: for \(S\subset G\) finite,
  \(
  \phi^{\mathrm{Gal}}_{i,\tau}(g):=\Ttau\mathbf{P}_i(g)-\mathrm{Id},
  \quad
  \mu^{\mathrm{Gal}}_{i,\tau}(S):=\dim_k\bigcap_{g\in S}\ker\phi^{\mathrm{Gal}}_{i,\tau}(g).
  \)
  \item Transfer kernel: for each \(T_a\),
  \(
  \phi^{\mathrm{Tr}}_{i,\tau}(a):\Ttau\mathbf{P}_i(F)\to \Ttau\mathbf{P}_i(T_a F),\ 
  \mu^{\mathrm{Tr}}_{i,\tau}(a):=\dim_k\ker\phi^{\mathrm{Tr}}_{i,\tau}(a).
  \)
  \item Functorial kernel:
  \(
  \phi^{\mathrm{Fun}}_{i,\tau}:\Ttau\mathbf{P}_i(F)\to \Ttau\mathbf{P}_i(\Funct F),\ 
  \mu^{\mathrm{Fun}}_{i,\tau}:=\dim_k\ker\phi^{\mathrm{Fun}}_{i,\tau}.
  \)
\end{itemize}
Define cokernels by \(\nu^{L}_{i,\tau}=\dim_k\mathrm{coker}(\phi^L_{i,\tau})\), \(L\in\{\mathrm{Gal},\mathrm{Tr},\mathrm{Fun}\}\).
All counts are invariant under f.q.i.\ and cofinal reindexing.
\end{definition}

\begin{declaration}[Layer gates (windowed)]\label{gate:tri-layer}
Fix a window and \(\tau\).
We accept a layer when:
\begin{itemize}
  \item Galois gate: \(\mu^{\mathrm{Gal}}_{i,\tau}(S)=0\) for all monitored \(i\).
        (WGC, if present, is \emph{recorded} but not required.)
  \item Transfer gate: for all \(T_a\), \(\mu^{\mathrm{Tr}}_{i,\tau}(a)=0\) and \(\bigotimes_a \delta^{\mathrm{Tr}}_a(i,\tau)\ \le\ \eta\).
  \item Functorial gate: \(\mu^{\mathrm{Fun}}_{i,\tau}=0\) and \(\delta^{\mathrm{Fun}}(i,\tau)\le \eta\).
\end{itemize}
The \emph{tri-layer gate} passes when all three pass and \((\mu_{\mathrm{Collapse}},\nu_{\mathrm{Collapse}})=(0,0)\) at \(\tau\); energies/spectral tails are non-increasing within the window.
Under the default \(\mathsf{Q}\), ``\(\le \eta\)'' reads ``sum of bounds \(<\) edge gap'' (Chapter~11).
\end{declaration}

\begin{remark}[Rationale]\label{rk:tri-rationale}
The Galois gate forbids nontrivial fixed sectors after collapse;
the transfer gate forbids losses invisible to energy/spectral proxies but detected by \(\ker\phi^{\mathrm{Tr}}\);
the functorial gate controls changes under functorial lifts via \(\ker\phi^{\mathrm{Fun}}\) with budget control.
\end{remark}

\subsection*{8.9. Type IV failure (μ\_Collapse–based) — layerwise visibility}
\begin{definition}[Visibility and Type IV codes]\label{def:type-iv}
At fixed \(\tau\) and window:
\begin{itemize}
  \item Visible failure if any energy/spectral indicator exceeds tolerance beyond the \(\delta\)-budget.
  \item Invisible failure if indicators stay within budget but some \(\mu^L_{i,\tau}+\nu^L_{i,\tau}>0\), \(L\in\{\mathrm{Gal},\mathrm{Tr},\mathrm{Fun}\}\).
\end{itemize}
Record layerwise codes \(\mathrm{IV}\text{-}L[\mathrm{vis}/\mathrm{inv}]\), \(L\in\{\mathrm{Gal},\mathrm{Tr},\mathrm{Fun}\}\).
Set a global Type~IV flag at \(\tau\) if any layer has \(\mu^L+\nu^L>0\).
\end{definition}

\begin{remark}[Transfer–kernel trigger]\label{rk:transfer-trigger}
The transfer collapse kernel is gate-decisive: \(\mu^{\mathrm{Tr}}_{i,\tau}(a)=0\) for all \(a\) is necessary to avoid \(\mathrm{IV}\text{-}\mathrm{Tr}[\mathrm{inv}]\).
\end{remark}

\subsection*{8.10. δ-ledger split by layers and cross-layer soft-commuting}
\begin{definition}[Layered product-ledger]\label{def:layer-delta}
We refine the ledger by
\[
\delta(i,\tau)\ =\ \delta^{\mathrm{alg}}(i,\tau)\ \otimes\ \delta^{\mathrm{disc}}(i,\tau)\ \otimes\ \delta^{\mathrm{meas}}(i,\tau),
\]
with the algebraic bin factorized as
\[
\delta^{\mathrm{alg}}(i,\tau)\ =\ \delta^{\mathrm{Gal}}(i,\tau)\ \otimes\ \Big(\bigotimes_{a}\delta^{\mathrm{Tr}}_a(i,\tau)\Big)\ \otimes\ \delta^{\mathrm{Fun}}(i,\tau)\ \in\ \mathsf{Q}.
\]
Under the default \(\mathsf{Q}\) this is ordinary addition split by layers.
\end{definition}

\begin{declaration}[Cross-layer soft-commuting]\label{dec:cross-soft}
For reflectors \(T_A,T_B\) and a layer functor \(L\in\{g\in G,\ T_a,\ \Funct\}\),
test
\[
\Delta_{\mathrm{comm}}(M;A,B|L)\ :=\ d_{\mathrm{int}}\!\big(T_A T_B (L\cdot M),\ T_B T_A (L\cdot M)\big)
\]
on \(M=\Ttau\mathbf{P}_i(F)\).
If \(\Delta_{\mathrm{comm}}\le \eta\), adopt soft-commuting; else fix an order and ledger \(\Delta_{\mathrm{comm}}\) into the corresponding layer bin of \(\delta^{\mathrm{alg}}\).
\end{declaration}

\subsection*{8.11. Tri-layer diagram and gate placement}
\begin{center}
\def\hstep{4.0em}\def\vstep{4.0em}\def\padL{1.0em}\def\padR{2.0em}
\begin{tikzpicture}[baseline,>=latex,node distance=2.0em]
\tikzset{box/.style={draw,rounded corners,inner sep=3pt,minimum width=10em,align=center}, ok/.style={inner sep=2pt}}
\node (M0) [box] {$\Ttau\mathbf{P}_i(F)$};
\node (G1) [right=\hstep of M0,box] {$\Ttau\mathbf{P}_i(F)$};
\node (G2) [right=\hstep of G1,box] {$\Ttau\mathbf{P}_i(F)$};
\node (Gok) [right=\hstep of G2,ok] {$\checkmark_{\mathrm{Gal}}$};
\draw[->,dashed,shorten >=0.2em,shorten <=0.2em] (M0) -- node[above] {$g\in G$} (G1);
\draw[->,dashed,shorten >=0.2em,shorten <=0.2em] (G1) -- node[above] {$\phi^{\mathrm{Gal}}_{i,\tau}(g)$} (G2);
\draw[->,dashed,shorten >=0.2em,shorten <=0.2em] (G2) -- node[above] {Gate: \(\mu^{\mathrm{Gal}}=0\)} (Gok);
\node (T1) [below=\vstep of G1,box] {$\Ttau\mathbf{P}_i(T_a F)$};
\node (T2) [right=\hstep of T1,box] {$\Ttau\mathbf{P}_i(T_a F)$};
\node (Tok) [right=\hstep of T2,ok] {$\checkmark_{\mathrm{Tr}}$};
\draw[->,dashed,shorten >=0.2em,shorten <=0.2em] (M0) -- node[left] {$T_a$} (T1);
\draw[->,dashed,shorten >=0.2em,shorten <=0.2em] (T1) -- node[above] {$\phi^{\mathrm{Tr}}_{i,\tau}(a)$} (T2);
\draw[->,dashed,shorten >=0.2em,shorten <=0.2em] (T2) -- node[above] {Gate: \(\mu^{\mathrm{Tr}}=0\)} (Tok);
\node (F2) [below=\vstep of T2,box] {$\Ttau\mathbf{P}_i(\Funct F)$};
\node (Fok) [right=\hstep of F2,ok] {$\checkmark_{\mathrm{Fun}}$};
\draw[->,dashed,shorten >=0.2em,shorten <=0.2em] (M0) to[bend right=10] node[below] {$\Funct$} (F2);
\draw[->,dashed,shorten >=0.2em,shorten <=0.2em] (F2) -- node[above] {Gate: \(\mu^{\mathrm{Fun}}=0\)} (Fok);
\end{tikzpicture}
\end{center}

\subsection*{8.12. Integration with tropical shortening}
\begin{proposition}[Shortening, stability, and tri-layer acceptance]\label{prop:short-trilayer}
Assume on a fixed window and \(\tau\):
(i) \((\mu_{\mathrm{Collapse}},\nu_{\mathrm{Collapse}})=(0,0)\);
(ii) tropical shortening with factor \(\kappa<1\) on a positive-mass subset (Definition~\ref{def:shorten});
(iii) all layer functors are non-expansive with \(2\)-cell budgets recorded via \(\otimes\) in \(\mathsf{Q}\).
Then indicators are non-increasing; tri-layer acceptance (Declaration~\ref{gate:tri-layer}) holds \emph{iff} all layer kernels vanish and the ledgered budgets stay within tolerance~\(\eta\).
Any WGC evidence may be recorded as an auxiliary tag but is not required for acceptance.
\end{proposition}

\begin{proof}[Proof sketch]
(i) ensures baseline stability under truncation.
(ii) yields strict decrease of windowed energy when shortening affects a positive-mass subset.
(iii) ensures post-collapse non-expansion; acceptance reduces to kernel vanishing plus ledger tolerance.
\end{proof}

\subsection*{8.13. Operational checklist (tri-layer, windowed)}
\begin{enumerate}
  \item Fix MECE windows and a \(\tau\)-sweep; set spectral parameters and tolerances \(\eta\); initialize the layered product-ledger in \(\mathsf{Q}\).
  \item Run tropical flow \(\lambda\searrow 0\); per sample compute \(\Ttau\mathbf{P}_i\), energies, spectra, \(\Ext^1\).
  \item Compute \((\mu_{\mathrm{Collapse}},\nu_{\mathrm{Collapse}})\) at \(\tau\) along the \(\lambda\)-tower.
  \item Galois layer: evaluate \(\mu^{\mathrm{Gal}}_{i,\tau}(S)\); run soft-commute tests; optionally record WGC-tag via \(\rho_{\max},\mathrm{nilp}\).
  \item Transfer layer: for each \(T_a\), evaluate \(\mu^{\mathrm{Tr}}_{i,\tau}(a)\); record \(\delta^{\mathrm{Tr}}_a\) in \(\mathsf{Q}\); run soft-commute tests.
  \item Functorial layer: evaluate \(\mu^{\mathrm{Fun}}_{i,\tau}\); record \(\delta^{\mathrm{Fun}}\); run soft-commute tests.
  \item Gate: accept layers; accept tri-layer if all pass and \((\mu,\nu)=(0,0)\); otherwise log Type~IV codes (Definition~\ref{def:type-iv}).
\end{enumerate}

\subsection*{8.14. Notes on sheaf proxies and towers (optional \textbf{[Spec]})}
\begin{remark}[Sheaf proxies]\label{rk:8-sheaf}
If a constructible sheaf model \(\mathcal{F}\) on a geometric carrier is available, compute windowed barcodes of \(\mathrm{R}\Gamma(\mathcal{F})\) and run the same gates on \(\Ttau\mathbf{P}_i(\mathrm{Sing}(\mathcal{F}))\), remaining at the persistence layer; no new identities are claimed.
\end{remark}

\begin{remark}[Tower compatibility]\label{rk:8-tower}
All layer kernels and Type~IV codes are invariant under f.q.i.\ and cofinal reindexing (Appendix~J). This ensures pasteability across windows and levels.
\end{remark}

\subsection*{8.15. Compliance summary (IMRN/AiM posture)}
\begin{enumerate}
  \item All decisions are B-side, windowed, and ledgered; equalities only at persistence; filtered-level statements up to f.q.i.
  \item Non-expansive policies are one-sided (\(\le\)) and \(2\)-cell budgets are aggregated by the product-ledger quantale (P7).
  \item Langlands tri-layer gates expose failure loci; transfer collapse kernel passes iff zero.
  \item Type~IV classification is layerwise with visibility labels and \(\mu/\nu\) diagnostics.
  \item WGC is an auxiliary persistence-level tag; it is never a gate criterion.
\end{enumerate}



\section{Chapter 9: Langlands Collapse (Three Layers)}
\addcontentsline{toc}{section}{Langlands Collapse (Three Layers)}
\label{sec:chapter9}

\noindent\textbf{Scope note (after–collapse policy, PF/BC discipline, and comparison order).}
We work in the constructible range over a fixed field \(k\), with degreewise finite-type filtered chain complexes
\(\FiltCh{k}\), degreewise persistence \(\mathbf{P}_i:\FiltCh{k}\to\Perskft\), the exact Serre reflector \(\Ttau:=\mathbf{T}_\tau\) (bar-deletion at scale \(\tau\ge 0\)), and a filtered lift \(\Ctau\) \emph{up to f.q.i.} as in Part~I.
All \emph{comparisons, metrics, and indicators are evaluated only after collapse}: the standard operating order is
\[
\text{for each }t\quad\Longrightarrow\quad \mathbf{P}_i\quad\Longrightarrow\quad \Ttau\quad\Longrightarrow\quad \text{compare in }\Perskft.
\]
Projection Formula / Base Change (PF/BC) is used as tabulated in Appendix~N \emph{objectwise in \(t\)}, transported to persistence, and \emph{all PF/BC comparators are measured only on the collapsed layer \(\Ttau\mathbf{P}_i(-)\)}.
Residuals (non-commutation, discretization, numerics) are externalized in the \(\delta\)-ledger and, unless explicitly algebraic, are charged to \(\delta_{\disc}\) and \(\delta_{\meas}\) (see Remark~\ref{rk:9-delta-policy} and Spec.~\ref{spec:9-T-PFBC-after}).
Equalities and Lipschitz statements are asserted at persistence after truncation; at the filtered level they hold up to filtered quasi-isomorphism (f.q.i.). Endpoint conventions and infinite bars follow Chapter~2, Remark~\ref{rk:2-endpoints}.

\begin{remark}[Monotonicity convention]
\label{rem:9-monotonicity}
We adopt Chapter~6, Remark~\ref{rem:stability-vs-monotonicity}: \emph{deletion-type} updates are non-increasing for windowed energies and spectral tails after truncation, whereas \emph{inclusion-type} updates are stability-only (non-expansive). See Appendix~E.
\end{remark}

\subsection*{9.0. Standing hypotheses, Gate Cascade alignment, and definable cover}
\emph{Implementable range.} We identify \(\Perskft\) with the constructible subcategory as in Chapter~6. Fix a \(t\)-exact realization \(\Rfun:\FiltCh{k}\to D^{\mathrm{b}}(k\text{-mod})\) (amplitude \(\le 1\) in use) and the lifting–coherence hypothesis \(\LC\) when comparing \(\Ctau\) with \(\tau_{\ge 0}\!\circ\!\Rfun\).

\emph{Three data layers.}
\[
\mathsf{Gal}\ \xrightarrow{\ \mathsf{Trans}\ }\ \mathsf{Par}\ \xrightarrow{\ \mathsf{Funct}\ }\ \mathsf{Aut},
\]
heuristically “Galois \(\to\) Transfer \(\to\) Functoriality”.
An admissible Langlands triple consists of functors
\(\mathcal{L}_{\mathrm{Gal}},\mathcal{L}_{\mathrm{Tr}},\mathcal{L}_{\mathrm{Aut}}:\FiltCh{k}\to\FiltCh{k}\) satisfying:
\begin{itemize}
  \item \textbf{Non-expansiveness.} Each layer induces filtered maps whose images under every \(\mathbf{P}_i\) are \(1\)-Lipschitz for interleavings; deletion-type steps make windowed indicators non-increasing after truncation; inclusion-type steps are stability-only.
  \item \textbf{\(\Ctau\)-compatibility.} For each \(i\): \(\mathbf{P}_i(\Ctau{-})\cong \Ttau\,\mathbf{P}_i(-)\) in \(\Perskft\).
  \item \textbf{Finite-type \& (co)limits.} Degreewise finite-type outputs; degreewise filtered (co)limits computed objectwise in \([\mathbb{R},\mathsf{Vect}_k]\) and used under the scope policy of Appendix~A.
  \item \textbf{Realization coherence.} Under \(\LC\), functorially up to f.q.i., \(\Rfun(\Ctau F)\simeq \tau_{\ge 0}\,\Rfun(F)\).
\end{itemize}

\begin{definition}[Langlands Gate Cascade and tri-layer gate]
\label{def:9-gate-cascade}
Fix a right-open window \(W\) definable in an o-minimal structure (Appendix~G/H/J), a threshold \(\tau\), and a working quantale \(V\) with operation \(\oplus\).
The \emph{Gate Cascade} on \(W\) at scale \(\tau\) is:
\[
\text{(GC1) B–Gate+ (single layer, Ch.~1)}\ \Rightarrow\ \text{(GC2) Overlap Gate (gluing, Ch.~1/5)}\ \Rightarrow\ \text{(GC3) Tri-layer Gate (this chapter)}.
\]
\emph{Tri-layer Gate} passes on \(W\) if, after truncation and on the same window/\(\tau\):
\begin{enumerate}
  \item \textbf{Layerwise acceptance:} Each \(\ast\in\{\mathrm{Gal},\mathrm{Tr},\mathrm{Aut}\}\) passes B–Gate+ in monitored degrees.
  \item \textbf{T–PFBC–AfterCollapse:} All PF/BC comparators used between layers are checked only on \(\Ttau\mathbf{P}_i(-)\) (Spec.~\ref{spec:9-T-PFBC-after}); total residuals are within budget.
  \item \textbf{Pseudonaturality after truncation:} The inter-layer comparisons of Remark~\ref{rk:9-compare} become isomorphisms in \(\Perskft\) up to the aggregated tolerance \(\boldsymbol{\delta}_{\mathrm{tot}}\) (Remark~\ref{rk:9-delta-policy}).
\end{enumerate}
\end{definition}

\begin{remark}[Run manifest (mandatory \texttt{run.yaml} fields)]
\label{rk:9-runyaml}
\texttt{quantale:\{name,op,unit,order\}}, \texttt{definable:\{structure,window\_formulae\}}, \texttt{layered\_\(\delta\):\{\(\delta^{\Gal}\),\(\delta^{\Tr}\),\(\delta^{\Fun}\)\}}, a fixed \(\tau\), window convention, and 2-cell bounds if AWFS/2-cell auditing is enabled.
\end{remark}

\begin{remark}[Operational order (PF/BC and collapse)]
\label{rk:9-operational}
Every PF/BC step (Appendix~N) is evaluated via
\[
\text{(i) objectwise in \(t\)}\ \Longrightarrow\ \text{(ii) }\mathbf{P}_i\ \Longrightarrow\ \text{(iii) }\Ttau\ \Longrightarrow\ \text{(iv) compare in }\Perskft,
\]
and \emph{never} before collapse in metrics/energies. Same window/\(\tau\) and same \(\delta\)-policy across all checks.
\end{remark}

\begin{declaration}[Spec–derived realizations and non-expansive transfers]
\label{spec:9-derived}
Besides \(\Rfun:\FiltCh{k}\!\to\!D^{\mathrm{b}}(k\text{-mod})\), we may use
\(\mathsf{R}_{\mathrm{coh}}:\FiltCh{k}\to D^{\mathrm{b}}\!\operatorname{Coh}(\mathfrak{X})\) and
\(\mathsf{R}_{\acute{e}t}:\FiltCh{k}\to D^{\mathrm{b}}_{\mathrm{c}}(\mathfrak{Y}_{\acute{e}t},\Lambda)\) with field \(\Lambda\).
PF/BC is assumed as in Appendix~N. Normalized transfers (degree-normalized pull/push, kernel/Hecke, parabolic induction/Jacquet) are \emph{non-expansive after truncation}:
\[
d_{\mathrm{int}}\!\big(\Ttau\mathbf{P}_i(\Phi F),\Ttau\mathbf{P}_i(\Phi G)\big)\ \le\ d_{\mathrm{int}}\!\big(\Ttau\mathbf{P}_i(F),\Ttau\mathbf{P}_i(G)\big).
\]
All claims reside at persistence after truncation; the bridge \(\mathrm{PH}_1\Rightarrow\Ext^1\) is used only in \(D^{\mathrm{b}}(k\text{-mod})\).
\end{declaration}

\begin{declaration}[Deletion-type operations (PDE)]
\label{spec:9-pde}
Maps implemented by the PDE repertoire of Appendix~E (Dirichlet restriction/absorbing boundaries, p.s.d.\ Loewner contractions, principal submatrices/Schur complements) are \emph{deletion-type} and make windowed energies and spectral tails \emph{non-increasing} after truncation. Inclusion-type steps are stability-only.
\end{declaration}

\begin{remark}[Endpoints and infinite bars]
\label{rk:9-endpoints}
Open/closed endpoint conventions are immaterial; \(\Ttau\) never removes infinite bars; windowed indicators clip them (Chapter~6).
\end{remark}

\begin{remark}[Indexing and cone extension]
\label{rk:9-cone}
For a directed index \(I\), adjoin a terminal \(\infty\) and cone maps \(t\to\infty\). Under realizations these yield filtered maps \(F_t\to F_\infty\) per layer/degree and provide the comparison maps \((\mu_{\mathrm{Collapse}},\nu_{\mathrm{Collapse}})\) at fixed~\(\tau\) (Chapter~4).
\end{remark}

\begin{remark}[Inter-layer comparison and pseudonaturality]
\label{rk:9-compare}
Fix natural transformations
\[
\alpha:\ \mathcal{L}_{\mathrm{Tr}}\!\circ\!\mathsf{Trans}\Rightarrow\mathcal{L}_{\mathrm{Gal}},\qquad
\beta:\ \mathcal{L}_{\mathrm{Aut}}\!\circ\!\mathsf{Funct}\Rightarrow\mathcal{L}_{\mathrm{Tr}},
\]
which become filtered quasi-isomorphisms degreewise after applying \(\Ctau\).
Equivalently, at persistence there are natural isomorphisms
\[
\Ttau\mathbf{P}_i(\mathcal{L}_{\mathrm{Tr}}\!\circ\!\mathsf{Trans}(-))\ \cong\ \Ttau\mathbf{P}_i(\mathcal{L}_{\mathrm{Gal}}(-)),\quad
\Ttau\mathbf{P}_i(\mathcal{L}_{\mathrm{Aut}}\!\circ\!\mathsf{Funct}(-))\ \cong\ \Ttau\mathbf{P}_i(\mathcal{L}_{\mathrm{Tr}}(-)),
\]
assembling to \emph{pseudonatural equivalences after truncation}.
\end{remark}

\begin{remark}[Unified \(\delta\)-policy and \(\delta\)-ledger]
\label{rk:9-delta-policy}
Fix \(\boldsymbol{\delta}=(\delta_{\mathrm{int}},\delta_{\mathrm{win}},\delta_{\mathrm{spec}})\) at scale \(\tau\). Layered budgets \(\delta^{\Gal},\delta^{\Tr},\delta^{\Fun}\in V\) aggregate as
\(\delta_{\mathrm{tot}}=\delta^{\Gal}\oplus\delta^{\Tr}\oplus\delta^{\Fun}\).
The \(\delta\)-ledger decomposes residuals as
\[
\delta = \delta_{\alg}\ \oplus\ \delta_{\disc}\ \oplus\ \delta_{\meas},
\]
where \(\delta_{\alg}\) logs provable algebraic non-commutation (A/B order, non-nested reflectors), while \(\delta_{\disc}\) and \(\delta_{\meas}\) log discretization and numerical tolerances (PF/BC transport, sampling, solvers). \emph{All PF/BC residuals in this chapter are charged to \(\delta_{\disc}\oplus \delta_{\meas}\)} unless stated algebraic. When \(\boldsymbol{\delta}_{\mathrm{tot}}=\mathbf{0}\), equalities are taken in \(\Perskft\) (up to iso).
\end{remark}

\subsection*{9.1. Persistence-layer interface for the three layers}
For \(F=\mathcal{L}_{\ast}(x)\) in layer \(\ast\) and degree \(i\), we monitor
\[
\Ttau\,\mathbf{P}_i(F),\qquad
\mathrm{PE}_i^{\le \tau}(F),\qquad
\mathrm{ST}_{\beta_{\mathrm{spec}}}^{\ge M(\tau)}(F),\ \mathrm{HT}(s;F),\qquad
\Ext^1\big(\Rfun(\Ctau F),Q\big)=0\ (Q\in\Qtest).
\]
All metrics/energies/indicators are evaluated after truncation, on a fixed window/\(\tau\), and within \(\boldsymbol{\delta}\).

\subsection*{9.2. Diagnostics along the index \(I\)}
For fixed layer and degree \(i\), at scale \(\tau\) define
\[
\phi_{i,\tau}:\ \varinjlim_{t\in I}\ \Ttau\,\mathbf{P}_i(F_t)\ \longrightarrow\ \Ttau\,\mathbf{P}_i(F_\infty),
\]
\[
\mu_{i,\tau}:=\dim_k\ker\phi_{i,\tau},\qquad
\nu_{i,\tau}:=\dim_k\mathrm{coker}(\phi_{i,\tau}),\qquad
\muc:=\sum_i\mu_{i,\tau},\ \nuc:=\sum_i\nu_{i,\tau}.
\]
All computed after truncation and on the same window/\(\tau\) (Remark~\ref{rk:9-alignment}).

\begin{remark}[Window/scale/metric alignment]
\label{rk:9-alignment}
Colimit and target of \(\phi_{i,\tau}\) use the same \(\tau\), window, and \(\boldsymbol{\delta}\)-policy.
\end{remark}

\subsection*{9.3. Propagation across the three layers}
\begin{declaration}[Propagation under \((\muc,\nuc)=(0,0)\)]
\label{spec:9-prop}
Assume \((\muc,\nuc)=(0,0)\) at a fixed \(\tau\) across the three layers, \(\LC\), and the inter-layer data of Remark~\ref{rk:9-compare}.
Then
\[
\mathsf{Gal}\ \xrightarrow{\ \mathsf{Trans}\ }\ \mathsf{Par}\ \xrightarrow{\ \mathsf{Funct}\ }\ \mathsf{Aut}
\]
commutes after truncation, up to isomorphism in \(\Perskft\) in each degree \(i\). Any obstruction is detected by \((\muc,\nuc)\). Residual slack is bounded by \(\boldsymbol{\delta}_{\mathrm{tot}}\); when \(\boldsymbol{\delta}_{\mathrm{tot}}=\mathbf{0}\), comparisons are strict (up to iso).
\end{declaration}

\subsection*{9.4. Monitoring protocol (Langlands three-layer)}
\begin{declaration}[Joint monitoring protocol]
\label{spec:9-protocol}
Fix \(\tau\in[\tau_{\min},\tau_{\max}]\), a window convention, an index range \(I\), and a \(\boldsymbol{\delta}\)-policy with \(\delta\)-ledger split \(\delta_{\alg}\oplus\delta_{\disc}\oplus\delta_{\meas}\).
For each layer \(\ast\in\{\mathrm{Gal},\mathrm{Tr},\mathrm{Aut}\}\) and sample \(t\):
\begin{enumerate}
  \item Record \(\Ttau\,\mathbf{P}_i(F^{(\ast)}_t)\) and \(\mathrm{PE}_i^{\le \tau}(F^{(\ast)}_t)\) on \(\Ttau\,\mathbf{P}_i\) (or \(\Ctau\)), within \(\delta_{\win}\).
  \item Record spectral indicators on \(L(\Ctau F^{(\ast)}_t)\) with fixed \((\beta_{\mathrm{spec}},M(\tau),s_{\mathrm{HT}})\) within \(\delta_{\spec}\).
  \item Check \(\Ext^1\big(\Rfun(\Ctau F^{(\ast)}_t),Q\big)=0\) for \(Q\in\Qtest\).
  \item Evaluate \((\muc,\nuc)\) via \(\phi_{i,\tau}\) along \(I\); log failure types (pure/mixed). Compare distances within \(\delta_{\int}\).
  \item Cross-layer check: non-expansiveness and \(\Ctau\)-compatibility up to f.q.i.; test pseudonaturality after truncation under \(\boldsymbol{\delta}_{\mathrm{tot}}\).
  \item PF/BC comparators \emph{only after collapse}: apply Spec.~\ref{spec:9-T-PFBC-after}; charge residuals to \(\delta_{\disc}\oplus\delta_{\meas}\) unless algebraic.
\end{enumerate}
Declare the collapse-stable regime where (1)–(3) hold jointly and \((\muc,\nuc)=(0,0)\) across layers.
\end{declaration}

\subsection*{9.5. Diagram (three layers, indicators, obstructions)}
\label{sec:9.5}
\begin{center}
\begin{tikzcd}[
  column sep=5.5em, row sep=1.8em,
  cells={nodes={font=\footnotesize}},
  every label/.append style={font=\scriptsize}
]
\text{Galois} \arrow[r, "\mathsf{Trans}"]
  & \text{Transfer} \arrow[r, "\mathsf{Funct}"]
    & \text{Automorphic} \\
\mathcal{L}_{\mathrm{Gal}} \arrow[r, dashed, "{\text{non-expansive},\ \Ctau\text{-compat}}"']
  & \mathcal{L}_{\mathrm{Tr}} \arrow[r, dashed, "{\text{non-expansive},\ \Ctau\text{-compat}}"']
    & \mathcal{L}_{\mathrm{Aut}} \\
\mathbf{P}_i \arrow[r]
  & \mathbf{P}_i \arrow[r]
    & \mathbf{P}_i \\
\Ttau\,\mathbf{P}_i \arrow[r, dashed, "{\cong\ \text{in }\Perskft\ \text{after truncation}}"]
  & \Ttau\,\mathbf{P}_i \arrow[r, dashed, "{\cong\ \text{in }\Perskft\ \text{after truncation}}"]
    & \Ttau\,\mathbf{P}_i \arrow[d, dashed, "\mathrm{PE}_i^{\le \tau}"] \\
& & \text{spectra on }L(\Ctau F^{(\ast)}_t),\ \ \Ext^1(\Rfun(\Ctau F^{(\ast)}_t),k)=0 \arrow[d, dashed] \\
\varinjlim_{t}\,\Ttau\,\mathbf{P}_i(F_t) \arrow[rr, "\phi_{i,\tau}"]
  & & \Ttau\,\mathbf{P}_i(F_\infty) \arrow[dl, bend right=25, dashed, "{(\muc,\nuc)\ \text{log}}"'] \\
& \text{failure log (pure/mixed)} &
\end{tikzcd}
\end{center}

\subsection*{9.6. Stability, non-expansiveness, and \(\delta\)-aggregation}
\begin{declaration}[Layerwise non-expansiveness with \(\delta\)-aggregation]
\label{spec:9-delta-agg}
Let \(\Phi_1:\mathsf{Gal}\!\to\!\mathsf{Par}\) and \(\Phi_2:\mathsf{Par}\!\to\!\mathsf{Aut}\) be admissible layer maps. For each \(i,\tau\),
\[
d_{\mathrm{int}}\!\Big(\Ttau\mathbf{P}_i(\Phi_2\!\circ\!\Phi_1(F)),\ \Ttau\mathbf{P}_i(\Phi_2\!\circ\!\Phi_1(G))\Big)\ \le\ d_{\mathrm{int}}\!\big(\Ttau\mathbf{P}_i(F),\ \Ttau\mathbf{P}_i(G)\big),
\]
and empirical/normalization slack is bounded by \(\delta_{\mathrm{int,tot}}=\delta_{\mathrm{int}}^{(\Phi_1)}\oplus\delta_{\mathrm{int}}^{(\Phi_2)}\).
Deletion-type steps make windowed energies/spectral tails non-increasing after truncation; inclusion-type steps are stable within \(\delta_{\mathrm{win,tot}},\delta_{\mathrm{spec,tot}}\).
\end{declaration}

\subsection*{9.7. Conjectural stability of functorial transfer}
\begin{conjecture}[Stability of functorial transfer under collapse]
\label{conj:9-stability}
Within the implementable range, assume non-expansive layer maps, \(\LC\), and \((\muc,\nuc)=(0,0)\) on a \(\tau\)-interval.
Then functorial transfer is stabilized at that scale: deletion-type steps do not increase persistence energies; spectral indicators do not grow; and \(\Ext^1\big(\Rfun(\Ctau -),Q\big)=0\) persists across the three layers.
All comparisons are performed after truncation with the same window/\(\tau\) and the same \(\boldsymbol{\delta}\)-policy; \(\boldsymbol{\delta}_{\mathrm{tot}}\) is the sum of per-step budgets.
No number-theoretic identity is asserted.
\end{conjecture}

\subsection*{9.8. Guard-rails, A/B testing, and non-claims}
\begin{remark}[A/B pseudonaturality test after collapse]
\label{rk:9-AB}
For two composites \(\gamma_A,\gamma_B\) through the three layers, test after truncation:
\[
d_{\mathrm{int}}\!\Big(\Ttau\mathbf{P}_i(\gamma_A(F)),\ \Ttau\mathbf{P}_i(\gamma_B(F))\Big)\ \le\ \delta_{\mathrm{int,tot}},\quad
\big|\mathrm{PE}_i^{\le\tau}(\gamma_A)-\mathrm{PE}_i^{\le\tau}(\gamma_B)\big|\ \le\ \delta_{\mathrm{win,tot}},
\]
with spectral/heat discrepancies \(\le\delta_{\mathrm{spec,tot}}\), all on the same window/\(\tau\).
Excess beyond budget is logged as Type~III (spec-mismatch) unless explained by \((\muc,\nuc)\).
\end{remark}

\begin{remark}[Scope and non-claims]
\label{rk:9-guard}
All statements are persistence/spectral/categorical under (B1)–(B3) of Part~I; no claim of \(\mathrm{PH}_1\Leftrightarrow\Ext^1\).
Obstructions are recorded by \((\muc,\nuc)\) and are unrelated to classical Iwasawa \(\mu\).
Binary saturation gates (PH\(_1\)\(\Leftrightarrow\)\(\Ext^1\) policies) are organized in Chapter~11.
This chapter provides a design/specification blueprint; it does not decide Langlands correspondence.
\end{remark}

\subsection*{9.9. Boundary models: geometric vs.\ arithmetic regions and the region map}
\begin{definition}[Region map and boundary model]
\label{def:region}
Let \(\mathsf{Dom}\) be the ambient index/parameter space. A \emph{region map} is
\(\mathrm{Reg}:\mathsf{Dom}\to\{\mathrm{Geom},\mathrm{Arith}\}\)
piecewise constant (finitely many jumps per window, recorded in the manifest).
\end{definition}

\begin{remark}[Usage]
\(W\in\mathrm{Geom}\): audit with Chapter~6 (energies/spectra, monotonicity/stability, \((\muc,\nuc)\)).
\(W\in\mathrm{Arith}\): audit with PF/BC comparators, transfer kernels, and tri-layer gates (Ch.~7–8).
Mixed windows are refined to a MECE partition where \(\mathrm{Reg}\) is constant.
\end{remark}

\subsection*{9.10. Window predicates: \(\mathrm{Ext\_trivial}\Rightarrow\mathrm{WeakGroup\_collapse}\) (typed)}
\begin{definition}[Typed window predicates]
\label{def:predicates}
Fix right-open \(W\), scale \(\tau\), degree set \(\mathcal{I}\).
\begin{itemize}
  \item \(\mathrm{Ext\_trivial}(W,\tau)\) iff \(\Ext^1(\Rfun(\Ctau F|_W),k)=0\).
  \item \(\mathrm{Tower\_stable}(W,\tau)\) iff \((\muc,\nuc)=(0,0)\) on \(W\) for all \(i\in\mathcal{I}\).
  \item \(\mathrm{PF/BC\_ok}(W,\tau)\) iff all PF/BC comparators pass \emph{after collapse} within \(\delta_{\int}\) and residuals are booked to \(\delta_{\disc}\oplus\delta_{\meas}\).
  \item \(\mathrm{Transfer\_ker\_zero}(W,\tau)\) iff each transfer at \(\tau\) has \(\mu^{\mathrm{Tr}}_{i,\tau}=0\) for all \(i\in\mathcal{I}\).
  \item \(\mathrm{WeakGroup\_collapse}(W,\tau)\) iff Def.~\ref{def:group-proxy} holds for fixed finite \(S\subset\mathsf{Aut}(F|_W)\).
\end{itemize}
All predicates are computed on the B-side single layer and logged in the manifest.
\end{definition}

\begin{theorem}[Predicate schema]
\label{thm:pred-schema}
Assume on \(W,\tau\): \(\mathrm{Ext\_trivial}(W,\tau)\), \(\mathrm{Tower\_stable}(W,\tau)\), and either
\emph{(a)} tropical shortening with factor \(\kappa<1\) on a positive-mass bar subset, or
\emph{(b)} deletion-type regime with strictly decreasing window energies (Ch.~6).
Then \(\mathrm{WeakGroup\_collapse}(W,\tau)\) holds for any non-expansive finite \(S\subset\mathsf{Aut}(F|_W)\).
\end{theorem}

\begin{remark}[Region-aware instantiation]
On \(\mathrm{Geom}\) windows, (b) is typical (deletion-type smoothing). On \(\mathrm{Arith}\) windows, (a) is supplied by tropical proxies (Ch.~7). The implication is persistence-layer, windowed, and budgeted.
\end{remark}

\subsection*{9.11. Region-aware diagnostics and acceptance}
\begin{definition}[Region-specific acceptance]
\label{def:region-accept}
Fix \(W,\tau\).
\begin{itemize}
  \item \(\mathrm{Accept}_{\mathrm{Geom}}(W,\tau)\): \(\mathrm{Tower\_stable}(W,\tau)\) and deletion-type indicators non-increasing; if also \(\mathrm{Ext\_trivial}(W,\tau)\) then declare \(\mathrm{WeakGroup\_collapse}(W,\tau)\).
  \item \(\mathrm{Accept}_{\mathrm{Arith}}(W,\tau)\): \(\mathrm{Tower\_stable}(W,\tau)\), \(\mathrm{PF/BC\_ok}(W,\tau)\), and \(\mathrm{Transfer\_ker\_zero}(W,\tau)\); if also \(\mathrm{Ext\_trivial}(W,\tau)\) then declare \(\mathrm{WeakGroup\_collapse}(W,\tau)\).
\end{itemize}
The global window verdict is \texttt{Valid} iff the corresponding regional predicate holds and the \(\delta\)-budget is dominated by the edge gap.
\end{definition}

\begin{remark}[Boundary jumps]
If \(\mathrm{Reg}\) jumps inside a coarse window, refine to a MECE partition; evaluate per refined window and paste via Restart/Summability (Chapter~4).
\end{remark}

\subsection*{9.12. Examples (boundary map and predicates)}
\begin{example}[Geometric region]
If \(W\in\mathrm{Geom}\) with viscosity ramping (deletion-type), then \(\mathrm{Tower\_stable}(W,\tau)\) and monotone \(\mathrm{PE}^{\le\tau}\) hold; if \(\mathrm{Ext\_trivial}(W,\tau)\), Theorem~\ref{thm:pred-schema} yields \(\mathrm{WeakGroup\_collapse}(W,\tau)\).
\end{example}

\begin{example}[Arithmetic region]
If \(W\in\mathrm{Arith}\) with PF/BC-admissible transfers and tropical shortening \(\kappa<1\), then \(\mathrm{PF/BC\_ok}(W,\tau)\), \(\mathrm{Transfer\_ker\_zero}(W,\tau)\), and \(\mathrm{Tower\_stable}(W,\tau)\) certify \(\mathrm{Accept}_{\mathrm{Arith}}(W,\tau)\).
If also \(\mathrm{Ext\_trivial}(W,\tau)\), conclude \(\mathrm{WeakGroup\_collapse}(W,\tau)\).
\end{example}

\subsection*{9.13. Summary (boundary models and predicates)}
We separated \(\mathrm{Geom}\) and \(\mathrm{Arith}\) windows via a region map \(\mathrm{Reg}\), introduced typed predicates that formalize “collapse after truncation’’ decisions at fixed \(\tau\), and used the core schema
\(\mathrm{Ext\_trivial}\wedge\mathrm{Tower\_stable}\Rightarrow\mathrm{WeakGroup\_collapse}\)
(under tropical or deletion-type compression).
Together with tri-layer gates and PF/BC comparators \emph{checked only after collapse}, this boundary view makes explicit \emph{where} collapse holds and \emph{which} layer blocks it, with Type~IV failures reported via \((\muc,\nuc)\). All acceptance criteria are windowed and budgeted (\(\delta_{\alg}\oplus\delta_{\disc}\oplus\delta_{\meas}\)).

\subsection*{9.14. T–PFBC–AfterCollapse (explicit specification)}
\begin{declaration}[T–PFBC–AfterCollapse]
\label{spec:9-T-PFBC-after}
Let \(\Psi\) be any PF/BC move admissible in Appendix~N (projection formula, base change, push–pull along Cartesian squares, base-change for kernels/Hecke, etc.). For each \(i\) and window \(W\) at threshold \(\tau\):
\begin{enumerate}
  \item \textbf{Transport to the collapsed layer.} Comparators for \(\Psi\) are formed and tested \emph{only} on \(\Ttau W\,\mathbf{P}_i(-)\). No metric/indicator is read before truncation.
  \item \textbf{Non-expansive baseline.} There is a canonical map (iso in the ideal PF/BC setting)
  \[
  \Ttau W\,\mathbf{P}_i(\Psi F)\ \longrightarrow\ \Ttau W\,\mathbf{P}_i(\Psi G)
  \]
  compatible with the corresponding map \(\Ttau W\,\mathbf{P}_i(F)\to\Ttau W\,\mathbf{P}_i(G)\) and \(1\)-Lipschitz in \(d_{\mathrm{int}}\).
  \item \textbf{Residual accounting.} Any deviation of PF/BC comparators from isomorphism after truncation is charged to \(\delta_{\disc}\oplus\delta_{\meas}\) on \(W\) (unless an algebraic obstruction is identified, in which case it is charged to \(\delta_{\alg}\)). The tri-layer and A/B tests use only these after-collapse residuals.
\end{enumerate}
In particular, all PF/BC-based \emph{distance} or \emph{energy} comparisons in this chapter are meaningful \emph{only} after applying \(\Ttau\) (equivalently, on \(\Ctau\)).
\end{declaration}



\section{Chapter 10: Application Program (PDE / BSD rank 0/1 / RH up to T)}
\addcontentsline{toc}{section}{Application Program (PDE / BSD rank 0/1 / RH up to T)}
\label{sec:chapter10}

\begin{remark}[Stability vs.\ monotonicity; corrected]
For non-expansive maps, indicators are stable (non-expansive).
Deletion-type operations satisfying Appendix~E (e.g.\ Dirichlet restriction, principal submatrices/Schur complements, positive-semidefinite Loewner contractions)
make spectral tails and windowed energies \emph{non-increasing}.
Inclusion-type updates generally do not guarantee non-increase; we only claim stability.
\end{remark}

\subsection*{10.0. Standing hypotheses and admissible realizations}
We fix a field \(k\) and work within the \emph{implementable range} of Part~I.
\emph{All statements in this chapter are made within the constructible range}
(we identify \(\Perskft\) with the constructible subcategory as in Chapter~6).
Let \(\FiltCh{k}\) be finite-type filtered chain complexes, and \(\mathbf{P}_i:\FiltCh{k}\to\Perskft\) the degreewise persistence functor.
We write \(\Ttau:=\mathbf{T}_\tau\) for the Serre (bar-deletion) reflector at scale \(\tau\ge 0\), and use its filtered lift \(\Ctau\) \emph{up to filtered quasi-isomorphism} (Chapter~2, §§2.2–2.3).
A fixed \(t\)-exact realization \(\Rfun:\FiltCh{k}\to D^{\mathrm{b}}(k\text{-mod})\) is retained; the lifting–coherence hypothesis \(\LC\) is assumed when comparing \(\Ctau\) with \(\tau_{\ge 0}\!\circ\!\Rfun\).
\emph{Equalities and Lipschitz claims are asserted only at the persistence layer; at the filtered-complex layer they hold up to filtered quasi-isomorphism.}
Endpoint conventions and infinite bars follow Chapter~2, Remark~\ref{rk:2-endpoints}.

Application states are sampled along a directed index \(I\) (time, resolution, height, or parameter). An \emph{admissible realization} is a functor
\[
  \mathsf{State}\ \xrightarrow{\ \mathcal{P}\ }\ \FiltCh{k},\qquad U\longmapsto F=\mathcal{P}(U),
\]
satisfying:
\begin{itemize}
  \item \textbf{Finite-type and (co)limits:} \(F\) is degreewise finite-type; degreewise filtered (co)limits in \(\FiltCh{k}\) are computed objectwise in \([\mathbb{R},\mathsf{Vect}_k]\) and used only under the scope policy of Appendix~A (compute in the functor category and verify return to \(\Pers^{\mathrm{cons}}_k\)).
  \item \textbf{Non-expansiveness under persistence:} along each directed update (e.g.\ time step, parameter step, height step, down-/up-sampling), the induced filtered map is non-expansive degreewise under \(\mathbf{P}_i\);
  in \emph{deletion-type} steps (Appendix~E) indicators are non-increasing up to f.q.i., while inclusion-type updates guarantee only stability.
  \item \textbf{Compatibility with truncation:} for each \(i\), naturally in \(\Perskft\),
  \(
    \mathbf{P}_i(\Ctau F)\ \cong\ \Ttau\,\mathbf{P}_i(F).
  \)
  \item \textbf{Realization coherence:} \(\Rfun\) is \(t\)-exact and compatible with \(\LC\), so functorially up to f.q.i.,
  \(\Rfun(\Ctau F)\simeq \tau_{\ge 0}\,\Rfun(F)\).
\end{itemize}

\subsection*{10.0a. Window certificates, manifests, and \(\delta\)-ledgers (generic)}
A \emph{window} is an interval \(W=[u,u')\subset\mathbb{R}\) in the index axis (time/resolution/height/parameter).
Fix \(\tau>0\) (resolution-adapted; Chapter~2). A \emph{window certificate} at \((W,\tau)\) records:
\begin{itemize}
  \item the single-layer objects \(\Ttau\,\mathbf{P}_i(F_s)\) for \(s\in W\cap I\),
  \item windowed persistence energies \(\mathrm{PE}_i^{\le \tau}\), spectral tails \(\mathrm{ST}_\beta^{\ge M(\tau)}\), and heat traces \(\mathrm{HT}(t;\cdot)\) computed on \(L(\Ctau F_s)\),
  \item the obstruction counts \((\muc,\nuc)\) computed via \(\phi_{i,\tau}\) (cf.\ §\ref{subsec:10.3}),
  \item the categorical check \(\Ext^1\big(\Rfun(\Ctau F_s),Q\big)=0\) for \(Q\in\Qtest\),
  \item a \emph{manifest} (run log) including discretization/sampling controls and thresholds,
  \item a \(\delta\)-ledger with the decomposition
  \(
    \delta(i,\tau)=\delta^{\mathrm{alg}}(i,\tau)+\delta^{\mathrm{disc}}(i,\tau)+\delta^{\mathrm{meas}}(i,\tau),
  \)
  aggregated as \(\Sigma\delta=\sum_{U\in W}\delta_U\).
\end{itemize}
Passing the gate (§\ref{subsec:10.8}) with safety margin \(\mathrm{gap}_\tau>\Sigma\delta\) produces a \emph{window certificate} for \((W,\tau)\).
Window pasting (Restart/Summability; §\ref{subsec:10.5}) aggregates certificates into global coverage.

\begin{remark}[Triggers (generic)]
A \emph{trigger} is a domain-specific necessary condition for gate failure within \(W\); it does not replace the gate but augments diagnostics. We use three canonical categories:
\begin{itemize}
  \item \textbf{Blow-up signs:} sustained growth in high-frequency/height/complexity channels after \(\Ctau\).
  \item \textbf{Tower accumulation:} repeated kernel/cokernel obstructions \((\muc,\nuc)\ne(0,0)\) or aux-bar persistence across windows.
  \item \textbf{PF/BC deviations:} violations of domain-specific physical/arithmetic fidelity or sampling/contour budgets (e.g.\ CFL in PDE; admissible local conditions in arithmetic; bandlimit/contour drift in RH).
\end{itemize}
All triggers are logged with parameters and timestamps in the manifest.
\end{remark}

\subsection*{10.0b. Definable windows and the \texorpdfstring{\(E_1\)}{E1} trigger}
\label{subsec:10.0b}
Work on right-open windows \(W\) that are definable in a fixed o-minimal structure (Appendix~G/H/J).
Then \emph{finite-event} and \emph{finite-Čech-depth} properties hold, and Chapter~3, Theorem~\textup{\ref{thm:E1-local}} (see also Appendix~C) yields, on \(W\),
\[
E_1(W)=0\ \Longleftrightarrow\ \PH_1(\Ctau F|_W)=0\ \Longleftrightarrow\ \Ext^1\!\big(\Rfun(\Ctau F|_W),k\big)=0.
\]
Thus, on definable windows, a vanishing \(E_1(W)\) \emph{short-circuits} the B-side gate: the PH/Ext checks become equivalent at window level (global equivalence is \emph{not} asserted).

\subsection*{10.1. Permitted operations and NS-specific examples (with CFL/CN controls)}
Each A-side step \(U\) is labeled and immediately followed by collapse \(\Ctau\); all measurements and gate decisions are taken on the B-side single layer \(\Ttau\mathbf{P}_i\) (Chapter~1, B-Gate\(^{+}\)). The \emph{Courant number} \(\mathrm{CN}\) and \emph{CFL} condition are recorded in the run manifest (Appendix~G) and justify quantitative non-expansiveness (\(\varepsilon\)-interleavings) for time stepping.

\paragraph{Operation labels and NS examples.}
\begin{itemize}
  \item \emph{Deletion-type (monotone):} low-pass mollification (filter width \(\sigma\)), viscosity increment \(\nu\mapsto \nu+\delta\nu\), threshold lowering in levelset filtrations, Dirichlet/absorbing boundary introduction, conservative averaging, Schur complements on blocks of the discrete operators. After \(\Ctau\), windowed persistence energies and spectral tails/heat traces on \(L(\Ctau F)\) are \emph{non-increasing} (Appendix~E).
  \item \emph{\(\varepsilon\)-continuation (non-expansive):} small time step \(\Delta t\) respecting CFL (e.g.\ \(\mathrm{CN}=\frac{u\,\Delta t}{\Delta x}\le \mathrm{CN}_{\max}\)), small parameter drifts (forcing amplitude, boundary condition perturbations), micro-updates of numerical flux limiters. Collapse-after stability holds with interleaving drift \(\varepsilon\!\sim\!C\Delta t\) (recorded).
  \item \emph{Inclusion-type (stable only):} domain enlargement, mesh refinement without smoothing, addition of couplings/sources (as long as the induced filtered map is 1-Lipschitz for \(\mathbf{P}_i\)). No monotonicity is claimed; stability only.
\end{itemize}
For each step, record in the \emph{\(\delta\)-ledger} (Chapter~5; Appendix~L) the decomposition
\(\delta=\delta^{\mathrm{alg}}+\delta^{\mathrm{disc}}+\delta^{\mathrm{meas}}\).

\begin{declaration}[Deletion-type operations (PDE)]
\label{spec:10-pde-del}
Operations covered by Appendix~E (Dirichlet restriction/absorbing boundaries, positive-semidefinite Loewner contractions with trace monotonicity, principal submatrices and Schur complements, conservative averaging) are treated as \emph{deletion-type}.
After truncation they are non-expansive for each \(\mathbf{P}_i\), and windowed energies \(\mathrm{PE}_i^{\le\tau}\) as well as spectral tails/heat traces on \(L(\Ctau F)\) are \emph{non-increasing}.
Inclusion-type updates are asserted only to be stable (non-expansive).
\end{declaration}

\begin{remark}[Quantitative non-expansiveness]
\label{rk:10-epsilon}
Let \(d_{\mathrm{int}}\) denote the interleaving distance on degreewise persistence. Along an update \(F_{s+1}\to F_s\), assume
\(
  d_{\mathrm{int}}\big(\mathbf{P}_i(F_{s+1}),\mathbf{P}_i(F_s)\big)\le \varepsilon_s\ (\varepsilon_s\ge 0).
\)
If \(\sup_s\varepsilon_s\le \varepsilon\), we call the tower \emph{\(\varepsilon\)-Lipschitz}. In deletion-type steps typically \(\varepsilon_s=0\); inclusion-type need not be zero. Time stepping under a CFL bound provides a concrete \(\varepsilon_s\sim C\Delta t\), recorded in the manifest.
\end{remark}

\begin{remark}[Truncation is \(1\)-Lipschitz]
Since \(\Ttau\) is \(1\)-Lipschitz for \(d_{\mathrm{int}}\), the same \(\varepsilon\)-Lipschitz control holds after truncation:
\(
d_{\mathrm{int}}\!\big(\Ttau\,\mathbf{P}_i(F_{s+1}),\ \Ttau\,\mathbf{P}_i(F_s)\big)\le \varepsilon_s.
\)
\end{remark}

\begin{remark}[Endpoints and infinite bars]
\label{rk:10-endpoints}
Open/closed endpoint choices are immaterial; infinite bars are not removed by \(\Ttau\) and are clipped by windowed indicators (as in Chapter~6).
\end{remark}

\begin{remark}[Index set and cone extension]
\label{rk:10-cone}
Work in the \emph{filtered index category} \(I\cup\{\infty\}\) with cone apex \(\infty\): for \(s\in I\), adjoin cone maps \(s\to\infty\).
Under \(\mathcal{P}\), these yield filtered maps \(F_s\to F_\infty\) used to define the comparison morphisms \(\phi_{i,\tau}\) at fixed~\(\tau\) (cf.\ Chapter~4).
\end{remark}

\subsection*{10.2. Construction principles for \texorpdfstring{$\mathcal{P}$}{P} (PDE)}
We list domain-agnostic templates; any one suffices for admissibility.
\begin{itemize}
  \item \textbf{Scalar-field cubical pipeline.} From a field \(q\) (e.g.\ vorticity magnitude, enstrophy density, \(Q\)-criterion) on a grid, build a cubical filtration by superlevel/sublevel sets; chains are \(k\)-valued on cubes.
  \item \textbf{Graph/simplicial pipeline.} From point samples, build Vietoris–Rips/alpha complexes with scale \(\varepsilon\); chains are \(k\)-valued on simplices.
  \item \textbf{Hybrid pipeline.} Combine topology of coherent structures with connectivity of level sets; filtration is vectorized but evaluated degreewise.
\end{itemize}
All three preserve finite-type per degree and admit non-expansive updates for standard PDE integrators (viscous steps are smoothing; down-sampling is deletion-type).
\emph{Spectral proxies are computed on \(L(\Ctau F)\) (positive eigenvalues; zero modes excluded or via pseudoinverse).}

\begin{remark}[Normalization and logging]
Normalization (graph vs.\ Hodge, symmetric vs.\ random-walk), zero-mode handling, and the window policy are fixed throughout a run and recorded alongside \((\beta,M(\tau),t)\) (Appendix~G).
All spectral indicators are computed on \(L(\Ctau F)\) to align with the truncation window.
\emph{Spectral indicators are not f.q.i.\ invariants; we only claim stability under a fixed policy \((\beta,M(\tau),t)\) on \(L(\Ctau F)\) (cf.\ Chapter~11).}
\end{remark}

\subsection*{10.2a. Construction principles for arithmetic and RH realizations}
\paragraph{BSD rank 0/1 (arithmetic).}
Let \(E/\mathbb{Q}\) be an elliptic curve; \(A\) denotes an \emph{arithmetic state} (e.g.\ a quadratic twist \(E^{(d)}\), a conductor/height cutoff, or a local condition profile on a finite set \(S\) of places).
We construct filtered complexes by any of:
\begin{itemize}
  \item \emph{Selmer filtration:} complexes whose chains encode \(p\)-Selmer data filtered by local condition strength or height; arrows reflect tightening/loosening local conditions.
  \item \emph{Descent graph pipeline:} graphs whose vertices are local condition classes; edges encode compatibility constraints; build a filtration by penalty thresholds.
  \item \emph{Hybrid pipeline:} combine Selmer layers with isogeny factors or visibility relations; evaluate degreewise.
\end{itemize}
Deletion-type updates include restriction to a smaller \(S\), tightening a local condition, or projecting along an isogeny with positive-semidefinite trace contraction on the chosen Laplacian model (Appendix~E analogues).
\(\varepsilon\)-continuation steps include small changes in a twist parameter \(d\) within a controlled family and height cutoffs; inclusion-type includes enlarging \(S\) or adding local conditions. Spectral proxies are computed on \(L(\Ctau F)\).

\paragraph{RH up to \(T\) (analytic).}
Let the \emph{state} encode samples of \(\xi(1/2+it)\), argument \(S(t)\), or zero counts \(N(t)\) on a window of heights. Build filtered complexes via:
\begin{itemize}
  \item \emph{Gram-graph pipeline:} nodes at Gram points/mesh points; edges connect near neighbors; filtration by magnitude thresholds or discrepancy of the argument from expected trends.
  \item \emph{Bandlimited scalar pipeline:} sub/superlevel filtrations of smoothed \(|\zeta(1/2+it)|\), \(|\xi|\), or of explicit-formula residuals; smoothing widths serve as deletion-type operations.
  \item \emph{Hybrid pipeline:} combine zero-locator events with discrepancy fields from the explicit formula.
\end{itemize}
Deletion-type updates include convolution smoothing (Gaussian/Fejér), restriction to subwindows, or projection onto bandlimited subspaces; \(\varepsilon\)-continuation includes small height increments \(\Delta t\) under Nyquist/bandlimit controls; inclusion includes window enlargement or resolution increase. Spectral proxies are computed on \(L(\Ctau F)\).

\subsection*{10.3. Indicators and diagnostics}
\label{subsec:10.3}
For each sample \(s\in I\) and degree \(i\) we monitor:
\[
  \Ttau\,\mathbf{P}_i(F_s),\qquad
  \mathrm{PE}_i^{\le \tau}(F_s)\ \text{(truncated energies on }\Ttau\,\mathbf{P}_i(F_s)\text{)},\qquad
  \mathrm{ST}_\beta^{\ge M(\tau)}(F_s),\ \mathrm{HT}(t;F_s)\ \text{on }L(\Ctau F_s),
\]
together with the categorical check \(\Ext^1\big(\Rfun(\Ctau F_s),Q\big)=0\) for \(Q\in\Qtest\).
For fixed \(\tau\), define the comparison map
\[
  \phi_{i,\tau}:\ \varinjlim_{s\in I}\ \Ttau\,\mathbf{P}_i(F_s)\ \longrightarrow\ \Ttau\,\mathbf{P}_i(F_\infty),
\]
and obstruction counts
\(
\mu_{i,\tau}=\dim_k\ker\phi_{i,\tau},\quad
\nu_{i,\tau}=\dim_k\mathrm{coker}\,\phi_{i,\tau},
\)
with \(\muc=\sum_i\mu_{i,\tau}\), \(\nuc=\sum_i\nu_{i,\tau}\) (finite by bounded degrees).
\emph{The obstructions \((\muc,\nuc)\) are invariant under filtered quasi-isomorphisms and under cofinal reindexing of the tower} (Appendix~J).

\begin{declaration}[Specification: Tower stability at the persistence layer]
\label{spec:10-stab}
Under the finite-type and objectwise degreewise-colimit hypotheses, for each fixed \(\tau\) and all \(i\) the map
\(
  \phi_{i,\tau}:\ \varinjlim_{s}\ \Ttau\,\mathbf{P}_i(F_s)\ \xrightarrow{\ \cong\ }\ \Ttau\,\mathbf{P}_i(F_\infty)
\)
is an isomorphism; hence \((\muc,\nuc)=(0,0)\) at that scale and Type~IV is excluded at \(\tau\).
\end{declaration}

\subsection*{10.4. Trigger pack ([Spec], domain-restricted necessary conditions)}
\paragraph{PDE (Navier–Stokes).}
\begin{itemize}
  \item \textbf{High-frequency surge:} sustained growth of enstrophy or high-wavenumber density in \(W\) \(\Rightarrow\) aux-bars (Chapter~11) persist \(>0\) after \(\Ctau\) or \(\mu>0\) is detected at \(\tau\).
  \item \textbf{Under-resolved advection:} CFL violation or \(\mathrm{CN}>\mathrm{CN}_{\max}\) \(\Rightarrow\) \(\varepsilon\)-continuation drift \(\varepsilon\) exceeds the safety margin \(\mathrm{gap}_\tau\) and B-Gate\(^{+}\) fails.
  \item \textbf{Unbalanced dissipation:} lack of smoothing under nominally viscous steps \(\Rightarrow\) non-decrease of \(\mathrm{PE}_i^{\le\tau}\) or spectral tails; repeated violations within \(W\) mark the window as non-regularizing.
  \item \textbf{PF/BC deviations:} boundary condition mismatches or energy budget imbalances beyond tolerance \(\Rightarrow\) flag window as suspect.
\end{itemize}

\paragraph{BSD rank 0/1.}
\begin{itemize}
  \item \textbf{Rank-proxy surge:} persistent increase of \(p\)-Selmer size proxies or regulator surges under deletion-type tightening \(\Rightarrow\) aux-bars persist \(>0\) or \(\mu>0\).
  \item \textbf{Local inconsistency:} repeated flips of local condition satisfaction under small parameter moves \(\Rightarrow\) \(\varepsilon\)-drift exceeds \(\mathrm{gap}_\tau\).
  \item \textbf{PF/BC deviations:} admissibility violations for chosen local models (e.g.\ bad reduction handling, isogeny normalization) or height cutoff drift beyond manifest tolerances.
\end{itemize}

\paragraph{RH up to \(T\).}
\begin{itemize}
  \item \textbf{Argument anomaly:} excursions of \(S(t)\) or explicit-formula residual discrepancies exceeding tolerance within \(W\) \(\Rightarrow\) aux-bars persist or \(\mu>0\).
  \item \textbf{Sampling under-resolution:} Nyquist/bandlimit violation for the chosen smoothing/kernel parameters \(\Rightarrow\) \(\varepsilon\)-drift exceeds \(\mathrm{gap}_\tau\).
  \item \textbf{PF/BC deviations:} contour/normalization policies (e.g.\ Gram grid misalignment) outside manifest tolerances.
\end{itemize}
All triggers are logged (Appendix~G) with quantitative thresholds and do \emph{not} replace B-Gate\(^{+}\); they augment diagnostics.

\subsection*{10.5. Window pasting: Restart and Summability}
\label{subsec:10.5}
Let \(\{W_k=[u_k,u_{k+1})\}_k\) be a MECE partition (Chapter~2). On each \(W_k\), fix \(\tau_k\) (resolution-adapted; Chapter~2) and compute the pipeline budget \(\Sigma\delta_k(i)=\sum_{U\in W_k}\delta_U(i,\tau_k)\). If B-Gate\(^{+}\) passes with a safety margin \(\mathrm{gap}_{\tau_k}>\Sigma\delta_k(i)\), the Restart lemma (Chapter~4) yields
\[
\mathrm{gap}_{\tau_{k+1}}\ \ge\ \kappa\ \bigl(\mathrm{gap}_{\tau_k}-\Sigma\delta_k(i)\bigr)\quad(\kappa\in(0,1]).
\]
If moreover \(\sum_k\Sigma\delta_k(i)<\infty\) (Summability; e.g.\ geometric decay of \(\tau_k,\beta_k\)), windowed certificates paste to a global one (Chapter~4).

\subsection*{10.6. Persistence-guided regularization ([Spec])}
\begin{declaration}[Specification: Persistence-guided regularization]
\label{spec:10-PGR}
A numerical or data-analytic regime is \emph{persistence-regularizing at scale \(\tau\)} if, along \(s\in I\),
\begin{enumerate}
  \item \(\mathrm{PE}_i^{\le \tau}\) are non-increasing (strictly decreasing on steps with genuine deletion-type smoothing),
  \item spectral indicators \(\mathrm{ST}_\beta^{\ge M(\tau)}\), \(\mathrm{HT}(t;\cdot)\) on \(\Ctau F_s\) are non-increasing (stability in general, monotone decrease in smoothing steps),
  \item \((\muc,\nuc)=(0,0)\),
  \item \(\Ext^1\big(\Rfun(\Ctau F_s),Q\big)=0\) for \(Q\in\Qtest\).
\end{enumerate}
When these hold across a \(\tau\)-interval, the regime aligns with established regularization/verification frameworks at that scale (domain-specific hypotheses to be listed separately). No analytic identity is claimed.
\end{declaration}

\subsection*{10.7. AK--NS hypothesis (programmatic)}
\begin{conjecture}[AK--NS hypothesis]
\label{conj:10-AKNS}
For Navier–Stokes-type flows, under an admissible realization \(\mathcal{P}\) and \(\LC\), if a monitored segment satisfies Declaration~\ref{spec:10-PGR} across a \(\tau\)-interval, then the designed persistence structure \emph{collapses} at that scale (bars shorten/vanish in aggregate), spectral tails decay, and the categorical check persists.
Programmatically, this corresponds to convergence toward known regularity scenarios at that scale.
No equivalence \(\mathrm{PH}_1\Leftrightarrow\Ext^1\) is asserted; only the one-way bridge under (B1)–(B3) is used.
\end{conjecture}

\subsection*{10.8. Gate template (PDE)}
\label{subsec:10.8}
On a fixed window \(W=[u,u')\), collapse threshold \(\tau>0\), and degree \(i\):
\begin{enumerate}
  \item Apply step \(U\) (labeled as above), then collapse \(\Ctau\).
  \item Measure on the B-side single layer: \(\Ttau\mathbf{P}_i\), \(\mathrm{PE}_i^{\le\tau}\), spectral auxiliaries (aux-bars; Chapter~11), and (in scope) \(\Ext^1\).
  \item Record \(\delta^{\mathrm{alg}},\delta^{\mathrm{disc}},\delta^{\mathrm{meas}}\) and update \(\Sigma\delta\).
  \item Evaluate B-Gate\(^{+}\): require \(\PH_1=0\), (in scope) \(\Ext^1=0\), \((\mu,\nu)=(0,0)\) after \(\Ttau\), and \(\mathrm{gap}_\tau>\Sigma\delta\).
  \item Log verdict; issue a windowed certificate on success; otherwise classify failure (Type I–IV).
\end{enumerate}

\subsection*{10.9. Monitoring protocol (PDE)}
\begin{declaration}[Specification: Joint monitoring protocol]
\label{spec:10-protocol}
Fix scales \(\tau\in[\tau_{\min},\tau_{\max}]\) and an index set \(I\) (time/resolution/parameter).
For each sample \(s\in I\):
\begin{enumerate}
  \item \emph{Compute and record} \(\Ttau\,\mathbf{P}_i(F_s)\) and \(\mathrm{PE}_i^{\le \tau}\) on \(\Ttau\,\mathbf{P}_i(F_s)\) (equivalently on \(\Ctau F_s\)).
  \item \emph{Compute and record} spectral indicators \(\mathrm{ST}_\beta^{\ge M(\tau)}\), \(\mathrm{HT}(t;F_s)\) on \(L(\Ctau F_s)\) with a fixed \((\beta,M(\tau),t)\) policy.
  \item \emph{Check} \(\Ext^1\big(\Rfun(\Ctau F_s),Q\big)=0\) for \(Q\in\Qtest\).
  \item \emph{Evaluate} \((\muc,\nuc)\) via \(\phi_{i,\tau}\) along \(s\in I\) and log failure types \emph{(pure kernel/cokernel/mixed)} if present.
  \item \emph{Stability declaration:} declare the \emph{persistence-regularizing regime} where (1)–(4) hold across the monitored \(\tau\)-range.
\end{enumerate}
\end{declaration}

\subsection*{10.10. Diagram (PDE pipeline and diagnostics)}
\begin{center}
\begin{tikzcd}[
  ampersand replacement=\&,
  column sep=2.8em, row sep=2.2em,
  cells={nodes={font=\footnotesize}},
  every label/.append style={font=\scriptsize},
  scale=0.9, transform shape
]
% --- Row 1 ---
{\text{PDE state }U} \arrow[r, "\mathcal{P}"] \&
{F\in \FiltCh{k}} \arrow[r, "\mathbf{P}_i"] \&
{\mathbf{P}_i(F)} \arrow[r, "\Ttau"] \&
{\Ttau\,\mathbf{P}_i(F)} \\
% --- Row 2 ---
\& \& {L(\Ctau F_s)} \arrow[r, dashed, "{\text{heat trace on }L(\Ctau F_s)}"] \&
{\mathrm{HT},\ \mathrm{ST}_{\beta}} \\
% --- Row 3 ---
\& {\Rfun(F_s)} \& \& {\Ext^1(\Rfun(\Ctau F_s),k)=0} \\
% --- Row 4 ---
\& {\varinjlim_{s}\,\Ttau\,\mathbf{P}_i(F_s)} \arrow[rr, "{\phi_{i,\tau}}"] \& \&
{\Ttau\,\mathbf{P}_i(F_\infty)} \\
% --- Row 5 ---
\& {\text{failure log (pure kernel/cokernel/mixed)}} \& \&
%
% ===== extra arrows with absolute coordinates =====
\arrow[from=1-3, to=2-3, dashed, "{\mathrm{PE}_i^{\le \tau}}"']
\arrow[from=1-4, to=2-4, dashed, "{\mathrm{PE}_i^{\le \tau}}"]
\arrow[from=3-2, to=1-2, bend left=40, dashed, "\LC"]
\arrow[from=3-2, to=3-4, dashed, "{\Ctau\ \text{ then }\Ext^1(-,k)\ \text{test}}"]
\arrow[from=4-4, to=5-2, bend right=25, dashed, swap,
       "{(\muc,\nuc)\ \text{log (pure kernel/cokernel/mixed)}}"]
\end{tikzcd}
\end{center}

\subsection*{10.11. Toy instances (persistence layer)}
\begin{example}[Viscous smoothing]
\label{ex:10-visc}
Let \(s\mapsto U_s\) be viscous steps for which the induced maps are deletion-type on the filtration.
Then bar lengths within the \(\tau\)-window decrease (or vanish), \(\mathrm{PE}_i^{\le \tau}\) strictly decreases, and \((\muc,\nuc)=(0,0)\) at fixed \(\tau\) by Declaration~\ref{spec:10-stab}.
Spectral proxies are evaluated as tails/heat traces of \(L(\Ctau F_s)\).
\end{example}

\begin{example}[Refinement/averaging pair]
A refinement \(F\to F'\) (inclusion-type) followed by conservative averaging \(F'\to \bar{F}\) (deletion-type) yields a non-expansive two-step update.
Under stability, \(\mathrm{PE}_i^{\le \tau}\) is non-increasing; failure logs isolate kernel/cokernel imbalance when present.
Spectral indicators are computed on \(L(\Ctau \bar{F})\).
\end{example}

\subsection*{10.12. Reproducibility (PDE)}
\begin{remark}[Run logs and parameters]
\label{rk:10-logs}
For each run, log: index range \(s\in[s_{\min},s_{\max}]\) (e.g.\ time), scales \(\tau\in[\tau_{\min},\tau_{\max}]\) (step width), spectral parameters \((\beta,M(\tau),t)\), discretization choices (cubical/simplicial/hybrid), CFL/CN numbers, \emph{a barcode-matching seed for reproducible vineyard tracking} (Appendix~G), \((\muc,\nuc)\) per \(\tau\) with failure types, and the \(\delta\)-ledger decomposition at step level.
These logs enable exact reruns and pipeline audits.
\end{remark}

\noindent A minimal \texttt{run.yaml} PDE block:

\begin{verbatim}
windows:
  domain: [[0,1), [1,2), [2,3)]
  collapse_tau: 0.08
  spectral_bins: {a: 0.0, beta: 0.02, bins: 96, boundary: "right-open"}
coverage_check:
  length_sum: 3.0
  length_target: 3.0
  events_sum_equals_global: true
cfl:
  courant_number_max: 0.5
  courant_number_measured: 0.32
operations:
  - U: mollify; type: deletion; tau: 0.08; delta: {alg:0.004, disc:0.003, meas:0.001}
  - U: timestep; type: epsilon;  tau: 0.08; eps: 0.006; delta: {alg:0.000, disc:0.002, meas:0.001}
persistence:
  PH1_zero: true
  Ext1_zero: true
  mu: 0
  nu: 0
  phi_iso_tail: true
spectral:
  aux_bars_remaining: 0
budget:
  sum_delta: 0.011
  safety_margin: 0.025
gate:
  accept: true
\end{verbatim}

\subsection*{10.13. Guard-rails and non-claims}
\begin{remark}[Scope and non-claims]
\label{rk:10-guard}
All statements operate at the persistence/spectral/categorical layers in the implementable range.
No analytic regularity theorem is proved; the AK–NS hypothesis is programmatic.
No claim of \(\mathrm{PH}_1\Leftrightarrow\Ext^1\) is made; only the one-way bridge under (B1)–(B3) is used.
The obstruction \(\muc\) is unrelated to classical Iwasawa \(\mu\).
\end{remark}

\subsection*{10.14. Completion note}
\begin{remark}[No further supplementation required]
This chapter integrates: (i) MECE windowing and resolution-adapted \(\tau\) with stability bands (via Chapters~2 and 4), (ii) the permitted operations catalog with NS-specific examples under CFL/CN controls and \(\delta\)-ledger accounting (Chapter~5; Appendix~L), (iii) B-side single-layer gate B-Gate\(^{+}\) with \(\PH_1/\Ext^1/(\mu,\nu)/\)safety-margin, (iv) triggers as \textbf{[Spec]} with a complete monitoring protocol, (v) Restart/Summability for window pasting, and (vi) reproducibility (run.yaml) with audit fields. All claims remain within the v16.0 guard-rails and cross-reference the proven core.
\end{remark}

\subsection*{10.15. Application II: BSD rank 0/1 (definable windows, E\texorpdfstring{\(_1\)}{1}-trigger, Iwasawa interface)}
\label{subsec:10.15}
We monitor families where analytic/algebraic rank is expected to be \(0\) or \(1\) (e.g.\ twists \(E^{(d)}\), isogeny classes, conductor windows) using the same gate/certificate format. \emph{No BSD assertion is made.} We provide a persistence/spectral protocol with reproducible manifests and window short-circuiting via \(E_1\).

\paragraph{Admissible realization \(\mathcal{P}_{\mathrm{BSD}}\).}
Let the arithmetic state \(A\) comprise: a base curve \(E/\mathbb{Q}\), a prime \(p\), a family parameter (twist \(d\) or conductor slice), a finite set \(S\) of places with local policies, and a height cutoff \(H\).
Define \(F=\mathcal{P}_{\mathrm{BSD}}(A)\) by one of:
\begin{itemize}
  \item \emph{Selmer complex filtration:} degrees encode \(p\)-Selmer cochains filtered by local-condition penalties; deletion-type steps tighten local conditions or decrease \(H\).
  \item \emph{Descent graph:} vertices are local symbols/classes; edges capture compatibility; filtration by cumulative penalty; deletion-type is edge/vertex contraction under verified dominance (Appendix~E analogues).
  \item \emph{Hybrid:} combine isogeny pushforwards with Selmer layers; normalize Laplacians as per a fixed policy recorded in the manifest.
\end{itemize}

\paragraph{Definable windows and \(E_1\)-short-circuit.}
Fix a right-open, o-minimal definable window \(W\) (Archimedean) or Denef–Pas definable window (non-Archimedean; Appendix~Q).
Then Chapter~3, Theorem~\ref{thm:E1-local} implies, \emph{on \(W\)},
\[
E_1(W)=0\ \Longleftrightarrow\ \PH_1(\Ctau F|_W)=0\ \Longleftrightarrow\ \Ext^1\!\big(\Rfun(\Ctau F|_W),k\big)=0,
\]
so B-Gate\(^{+}\) reduces to verifying \(E_1(W)=0\) plus tower stability \((\muc,\nuc)=(0,0)\) and budget dominance \(\mathrm{gap}_\tau>\Sigma\delta\).
This is a \emph{window-local} equivalence; global equivalence is not asserted.

\paragraph{Control \(\Rightarrow\) Overlap Gate (Iwasawa interface).}
Using Chapter~7, Proposition~\textup{\ref{prop:control-overlap-gate}} and Appendix~R, control theorems translate to the Overlap Gate: finite kernel/cokernel contributions are absorbed into \(\delta^{\mathrm{alg}}\) in the ledger, with explicit bounds recorded as \texttt{control\_finite\_bounds}. Layered \(\delta\)-boxes \((\delta^{\Gal},\delta^{\Tr},\delta^{\Fun})\) are mandatory (Chapter~9).

\paragraph{Indicators and gate.}
Compute \(\Ttau\,\mathbf{P}_i(F_s)\), \(\mathrm{PE}_i^{\le\tau}\), \(\mathrm{ST}_\beta^{\ge M(\tau)}\), \(\mathrm{HT}(t;\cdot)\) on \(L(\Ctau F_s)\) per window.
Gate requires (after collapse, on \(W\)): \(\PH_1=0\), \(\Ext^1=0\) (by \(E_1=0\)), \((\mu,\nu)=(0,0)\), and \(\mathrm{gap}_\tau>\Sigma\delta\).

\paragraph{Triggers ([Spec]).}
\begin{itemize}
  \item \emph{Rank-proxy surge:} persistent increase of rank proxies under deletion-type updates.
  \item \emph{Local inconsistency:} instability of local conditions under small parameter moves.
  \item \emph{PF/BC deviations:} policy violations in bad reduction handling or height normalization.
\end{itemize}

\paragraph{Window certificate (BSD).}
A certificate for \((W,\tau)\) contains: single-layer persistence objects, spectral proxies on \(L(\Ctau F_s)\), the obstruction log with failure types, Ext checks (by \(E_1=0\) on \(W\)), and the \(\delta\)-ledger.
The manifest includes: prime \(p\), family parameters, local policy for \(S\), height cutoff \(H\), Laplacian normalization, layered \(\delta\)-boxes, and seeds for deterministic matching.

\noindent A minimal \texttt{run.yaml} BSD/Iwasawa block (IMRN/AiM-ready):

\begin{verbatim}
quantale:
  name: "[0,inf]_plus"
  op: "+"
  unit: 0.0
  order: "<="
definable:
  o_minimal_structure: "R_an,exp"
  window_formulae:
    - "u <= t < u'"
  p_adic:
    structure: "Denef-Pas"
    window_formulae:
      - "val(x) in [a,b) and ac_n(x)=c"
layered_delta:
  deltaGal: 0.004
  deltaTr:  0.003
  deltaFun: 0.002
iwasawa:
  tower_level: ["N=1","N=2","N=4","N=8"]
  control_finite_bounds:
    kernel_leq: 2
    cokernel_leq: 2
windows:
  domain: [[1e3,2e3), [2e3,3e3)]
  collapse_tau: 0.12
  spectral_bins: {a: 0.0, beta: 0.04, bins: 128, boundary: "left-open"}
family:
  curve: "E: y^2 = x^3 - x"
  prime_p: 3
  twists: {type: quadratic, d_range: [1, 1000], parity_filter: "even"}
  height_cutoff: 14.0
local_policy:
  S: ["p=3","p=5","infty"]
  conditions: {relaxation: "bounded", penalty_step: 0.5}
awfs_2cell:
  awfs_enabled: true
  two_cell_bounds: {delta_upper: 0.006}
operations:
  - U: tighten_local; type: deletion; tau: 0.12; delta: {alg:0.002, disc:0.001, meas:0.001}
  - U: twist_step;    type: epsilon;  tau: 0.12; eps: 0.005; delta: {alg:0.000, disc:0.002, meas:0.001}
persistence:
  E1_zero_window: true
  PH1_zero: true
  Ext1_zero: true
  mu: 0
  nu: 0
  phi_iso_tail: true
spectral:
  aux_bars_remaining: 0
budget:
  delta_total: 0.009
  safety_margin: 0.021
gate:
  accept: true
\end{verbatim}

\begin{remark}[Guard-rails for BSD]
The protocol monitors persistence-layer stability under fixed policies and logs reproducible manifests.
It does not prove BSD, nor does it identify algebraic rank; rank proxies and spectral tails are diagnostics only.
All claims remain persistence/spectral/categorical and policy-dependent; PH/Ext equivalence is window-local via \(E_1(W)=0\).
\end{remark}

\subsection*{10.16. Application III: RH up to \texorpdfstring{$T$}{T} (template)}
We provide a verification\hyp style template to monitor windows in height and produce reproducible certificates.\ \emph{No RH claim is made}; zero\hyp locating or discrepancy detection is carried out at the persistence/spectral layer under fixed sampling/smoothing policies.

\paragraph{Admissible realization \(\mathcal{P}_{\mathrm{RH}}\).}
Let the state record: a height window \(W=[u,u')\), sampling mesh \(\Delta t\), smoothing kernel and bandwidth, and a normalization policy for \(\xi\), \(S(t)\), or explicit\hyp formula residuals. Construct \(F=\mathcal{P}_{\mathrm{RH}}(W)\) via:
\begin{itemize}
  \item \emph{Gram\hyp graph pipeline:} nodes at mesh/Gram points; edges connect neighbors; filtration by discrepancy thresholds \(|S(t)-S_{\mathrm{ref}}(t)|\).
  \item \emph{Scalar pipeline:} sub/superlevel filtrations of smoothed \(|\zeta(1/2+it)|\), \(|\xi|\), or residual fields.
  \item \emph{Hybrid:} couple zero\hyp candidate events with discrepancy fields; evaluate degreewise.
\end{itemize}
Deletion\hyp type: convolution smoothing; restriction to subwindows; projection to bandlimited subspaces. Epsilon\hyp continuation: small height steps under Nyquist control; inclusion: window enlargement or mesh refinement.

\paragraph{Indicators and gate.}
Compute \(\Ttau\,\mathbf{P}_i(F_s)\), \(\mathrm{PE}_i^{\le\tau}\), \(\mathrm{ST}_\beta^{\ge M(\tau)}\), \(\mathrm{HT}(t;\cdot)\) on \(L(\Ctau F_s)\) per window, with fixed normalization and bandlimit policies.
Gate: PH1\(=0\), Ext1\(=0\) (for test objects reflecting normalization checks), \((\mu,\nu)=(0,0)\), and \(\mathrm{gap}_\tau>\Sigma\delta\).

\paragraph{Triggers ([Spec]).}
\begin{itemize}
  \item \emph{Argument anomaly:} excursions of \(S(t)\) beyond tolerances, or explicit\hyp formula residual spikes.
  \item \emph{Sampling under\hyp resolution:} Nyquist/bandlimit violations for the chosen kernel/bandwidth.
  \item \emph{PF/BC deviations:} contour/grid normalization mismatches (e.g.\ Gram grid drift) or policy inconsistencies.
\end{itemize}

\paragraph{Window certificate (RH).}
A certificate for \((W,\tau)\) contains: single\hyp layer persistence, spectral proxies on \(L(\Ctau F_s)\), obstruction counts and failure types, Ext checks, and a \(\delta\)\hyp ledger. The manifest includes: mesh \(\Delta t\), kernel/bandwidth, bandlimit/Nyquist checks, normalization policy, and deterministic seeds.

\noindent A minimal \texttt{run.yaml} RH block:

\begin{verbatim}
windows:
  domain: [[1e9,1e9+5e5), [1e9+5e5, 1e9+1e6)]
  collapse_tau: 0.06
  spectral_bins: {a: 0.0, beta: 0.03, bins: 256, boundary: "right-open"}
sampling:
  dt: 1.0e-3
  bandlimit: 3000.0
  nyquist_check: true
smoothing:
  kernel: "gaussian"
  bandwidth: 2.5e-3
operations:
  - U: smooth;     type: deletion; tau: 0.06; delta: {alg:0.001, disc:0.002, meas:0.001}
  - U: heightstep; type: epsilon;  tau: 0.06; eps: 0.004; delta: {alg:0.000, disc:0.002, meas:0.001}
persistence:
  PH1_zero: true
  Ext1_zero: true
  mu: 0
  nu: 0
  phi_iso_tail: true
spectral:
  aux_bars_remaining: 0
budget:
  sum_delta: 0.006
  safety_margin: 0.018
gate:
  accept: true
\end{verbatim}

\begin{remark}[Guard\hyp rails for RH]
The protocol verifies stability of persistence/spectral indicators under fixed sampling/smoothing and normalization policies and produces reproducible manifests.
It does not assert the Riemann Hypothesis, nor count zeros; it only monitors windowed diagnostics with logged tolerances.
\end{remark}

% ============================================================
% Cross-application integration and auditability
% ============================================================

\subsection*{10.17. Cross\hyp application gate, triggers, and pasting}
All three applications (PDE, BSD rank~0/1, RH up to \(T\)) share:
\begin{itemize}
  \item \textbf{Gate B\hyp Gate\(^{+}\):} single\hyp layer decisions on \(\Ttau\mathbf{P}_i\) with PH1/Ext1/(\(\mu,\nu\))/safety\hyp margin.
  \item \textbf{Triggers:} blow\hyp up signs, tower accumulation, PF/BC deviations (domain\hyp specific specializations).
  \item \textbf{Restart/Summability:} window pasting with budgeted \(\Sigma\delta\) and geometric decay options for \(\tau,\beta\).
  \item \textbf{Reproducibility:} unified manifest keys (window domain, collapse\_tau, spectral bins, operations with \(\delta\)\hyp ledger, persistence verdicts, spectral auxiliaries, budget, gate).
\end{itemize}
Domain\hyp specific policies (normalization, local conditions, sampling/bandlimit) are fixed per run and logged; spectral indicators are always computed on \(L(\Ctau F)\).

\subsection*{10.18. Effect and operational readiness}
The templates make \emph{immediate operational deployment} possible: each application ships with (i) a gate specification, (ii) a trigger pack, (iii) a run manifest schema, and (iv) a window certificate format, yielding clear outcomes (certificate + reproducible logs).
Relative to v16.0, the deliverables are explicit and auditable.

\subsection*{10.19. Final guard\hyp rails (IMRN/AiM\hyp style)}
\begin{remark}[Non\hyp equivalences and scope]
All interleaving/Lipschitz/monotonicity claims are asserted at the persistence layer and, when stated for spectral proxies, under a fixed normalization policy on \(L(\Ctau F)\). Ext tests are scope\hyp restricted to a finite \(\Qtest\).
No analytic equivalences (e.g.\ BSD, RH, PDE regularity) are claimed or used. The program provides certificate\hyp style diagnostics with reproducible manifests and budgeted stability, suitable for audit and re\hyp execution.
\end{remark}

% ============================================================
% PoC reinforcements requested (PDE Burgers/NS, BSD Overlap, RH explicit formula)
% ============================================================

\subsection*{10.20. PoC I (PDE): Burgers / 2D--NS and a dissipation certificate}
\label{subsec:10.20}
\begin{definition}[Dissipation ECF and collapse witness]
\label{def:10-ecf}
For a window \(W=[u,u')\) and scale \(\tau\), define the \emph{energy--cumulative dissipation} (ECF)
\[
\mathrm{ECF}_\tau(W)\ :=\ \int_{u}^{u'} \mathcal{D}_\tau(s)\,ds,\qquad
\mathcal{D}_\tau(s)\ \coloneqq\ \mathrm{PE}_i^{\le\tau}(F_{s^-})-\mathrm{PE}_i^{\le\tau}(F_{s^+})\ \ge 0,
\]
where \(s^-,s^+\) denote instants immediately before/after a deletion\hyp type step at \(s\).
The \emph{collapse witness} is
\(
\mathrm{Col}_\tau(W):=\sum_i \bigl(\mathrm{bars}_{i,\tau}(u)-\mathrm{bars}_{i,\tau}(u')\bigr)\ge 0.
\)
\end{definition}

\begin{declaration}[PoC inequality (Spec)]
\label{spec:10-ecf-mu}
For Burgers/2D\hyp NS pipelines under the admissible operations of §10.1 and fixed normalization on \(L(\Ctau F)\), there exists a constant \(C(\tau)\ge 0\) (policy\hyp dependent) such that on any window \(W\)
\[
\muc(W)\ \le\ C(\tau)\,\mathrm{ECF}_\tau(W)\qquad\text{and}\qquad
\mathrm{Col}_\tau(W)\ \le\ C(\tau)\,\mathrm{ECF}_\tau(W).
\]
This is a \textbf{[Spec]} certificate: it is logged and verified numerically; it is not claimed as an analytic identity.
\end{declaration}

\begin{example}[Burgers shock--smoothing window]
On \(W\) containing a mollify\(\rightarrow\)advect cycle with CFL below bound, \(\mathrm{ECF}_\tau(W)>0\) and \(\mathrm{Col}_\tau(W)>0\).
If \(\sum_k\mathrm{ECF}_\tau(W_k)<\infty\) over a MECE cover, Restart/Summability (Chapter~4) yields global bar shortening at scale \(\tau\).
\end{example}

\subsection*{10.21. PoC II (BSD rank 0/1): Overlap globalization of the \(E_1\)-bridge}
\label{subsec:10.21}
\begin{proposition}[Overlap globalization of the \(E_1\)-bridge (rank \(0/1\) windows)]
\label{prop:10-overlap-global}
Let \(\{W_k\}\) be a MECE cover by definable windows with \(E_1(W_k)=0\) for all \(k\).
Assume the Overlap Gate holds on pairwise overlaps and \(\sum_k \Sigma\delta_k<\infty\).
Then on \(\bigcup_k W_k\) we have \(\PH_1(\Ctau F)=0\) and \(\Ext^1(\Rfun(\Ctau F),k)=0\) at scale \(\tau\).
\end{proposition}

\begin{proof}[Proof sketch]
By §10.0b, \(E_1(W_k)=0\) implies \(\PH_1(\Ctau F|_{W_k})=0\) iff \(\Ext^1(\Rfun(\Ctau F|_{W_k}),k)=0\).
The Overlap Gate identifies tails on overlaps up to the ledger budget; Restart/Summability pastes window certificates, yielding global vanishing at \(\tau\).
\end{proof}

\subsection*{10.22. PoC III (RH): explicit formula as trace, window\hyp local Weil positivity}
\label{subsec:10.22}
\begin{declaration}[Explicit\hyp formula comparator (T–PFBC–AfterCollapse)]
\label{spec:10-rh-pfbc}
Let \(\Phi_{\mathrm{EF}}\) denote the explicit\hyp formula transform and \(\Phi_{\mathrm{Tr}}\) the trace comparator under a fixed normalization policy.
All distance/energy comparisons between \(\Phi_{\mathrm{EF}}\) and \(\Phi_{\mathrm{Tr}}\) are evaluated \emph{only after} \(\Ttau\) (cf.\ Chap.~9, Spec.~\ref{spec:9-T-PFBC-after}); residuals are booked in \(\delta_{\disc}\oplus\delta_{\meas}\).
\end{declaration}

\begin{remark}[Weil\hyp type positivity, window\hyp local]
Fix a test function \(\varphi\) in the allowed class; the quadratic form \(Q(\varphi)\ge 0\) is verified \emph{window\hyp locally} by checking the corresponding spectral proxy on \(L(\Ctau F)\) and recording residuals in the ledger.
No global spectral statement is asserted.
\end{remark}

\subsection*{10.23. p\hyp adic interface (GL(1)\(\to\)GL(2))}
\label{subsec:10.23}
\paragraph{Local kernels and AK measurement.}
For GL(1), use Tate integrals/Igusa local zeta to define local kernels \(K_p\) that act as deletion\hyp type or \(\varepsilon\)\hyp continuation steps depending on cutoffs; their effects are measured on \(\Ttau\mathbf{P}_i\) and charged to the \(\delta\)\hyp ledger.
For GL(2) at small conductor, introduce local test functions stagewise; PF/BC comparators with Hecke kernels are evaluated only after collapse (T–PFBC–AfterCollapse).

\noindent A minimal \texttt{run.yaml} p\hyp adic block:

\begin{verbatim}
padic:
  primes: [3,5,7]
  local_kernels:
    - {p: 3, type: "Tate", cutoff: 9, action: "deletion"}
    - {p: 5, type: "Igusa", cutoff: 7, action: "epsilon"}
gl2:
  conductor: 64
  hecke_normalization: "unitary"
  stage_intro: ["T_p", "T_p^2", "U_p"]
operations:
  - U: local_Tate_p3; type: deletion; tau: 0.10; delta: {alg:0.001, disc:0.002, meas:0.001}
  - U: local_Igusa_p5; type: epsilon;  tau: 0.10; eps: 0.003; delta: {alg:0.000, disc:0.001, meas:0.001}
persistence:
  PH1_zero: true
  Ext1_zero: true
  mu: 0
  nu: 0
gate:
  accept: true
\end{verbatim}

\subsection*{10.24. Chapter summary (PoC addendum)}
The PoC additions (PDE ECF certificate, BSD Overlap globalization, RH explicit\hyp formula comparator, p\hyp adic local kernels) are fully \emph{after\hyp collapse}, windowed, and budgeted by the \(\delta\)\hyp ledger.
They integrate with B\hyp Gate\(^{+}\), Restart/Summability, and T–PFBC–AfterCollapse, with no analytic number theory or PDE regularity claims beyond the persistence/spectral/categorical guard\hyp rails of v16.0.



\section{Chapter 11: Collapse Energy, Spectral Indicators, and TDA Notes}
\addcontentsline{toc}{section}{Collapse Energy, Spectral Indicators, and TDA Notes}
\label{sec:ch11}

\subsection*{11.0. Scope, standing hypotheses, and notation}
We work within the \emph{implementable range} of Part~I, using realizations into $\FiltCh{k}$ as in Chs.~6–10. All persistence quantities are computed degreewise \emph{after} truncation by $\Ctau$; spectral indicators are computed on the normalized combinatorial Hodge Laplacian of $\Ctau F$; categorical checks use a fixed $t$-exact $\Rfun:\FiltCh{k}\to\Db(k\text{-mod})$ compatible with \LC. Filtered (co)limits, when used, are computed objectwise in $[\mathbb{R},\mathsf{Vect}_k]$ and only under the scope policy of Appendix~A (compute in the functor category and verify return to $\Pers^{\mathrm{cons}}_k$). No claim of $\mathrm{PH}_1\Leftrightarrow\Ext^1$ is made; only the one-way bridge under (B1)–(B3) is used. The obstruction pair $(\muc,\nuc)$ is the \emph{collapse} diagnostic and is unrelated to the classical Iwasawa $\mu$. We adopt the global $\delta$-policy $\boldsymbol{\delta}=(\delta_{\mathrm{int}},\delta_{\mathrm{win}},\delta_{\mathrm{spec}})$ and the additive ledger on a fixed commutative quantale $V$ (Appendix~S); layered boxes $(\delta^{\Gal},\delta^{\Tr},\delta^{\Fun})$ are mandatory (Ch.~9).

\begin{remark}[Endpoints and infinite bars]\label{rk:11-endpoints}
Open/closed endpoint conventions are immaterial; infinite bars are not removed by $\Ttau$ and are clipped by the window $\tau$ in all windowed quantities (cf.\ Ch.~6).
\end{remark}

\subsection*{11.0+. First-class determinant \(E_1\); definable windows and finite events}
On right-open windows $W$ definable in a fixed o-minimal (Archimedean) or Denef–Pas (non-Archimedean) structure, Betti integrals are piecewise constant with finitely many jumps and Čech depth is finite; see Appendices~H and~J. In this regime, the Page-$E_1$ term is our \emph{first-class determinant}:

\begin{theorem}[Local bridge on definable windows; reprise of Ch.~3]\label{thm:11-E1-local-reprise}
Let $W$ be definable and right-open. Then
\[
E_1(W)=0 \ \Longleftrightarrow\ \PH_1(\Ctau F|_W)=0 \ \Longleftrightarrow\ \Ext^1\!\big(\Rfun(\Ctau F|_W),k\big)=0.
\]
The equivalence is \emph{window-local} and \emph{after collapse}; no global equivalence is asserted.
\end{theorem}

\begin{remark}[Unique comparison order]\label{rk:11-order}
All measurements obey the \emph{single} order
\[
\boxed{\ \text{for each } t\ \Longrightarrow\ \mathbf{P}_i \ \Longrightarrow\ \mathbf{T}_\tau \ \Longrightarrow\ \text{compare in }\Pers^{\mathrm{cons}}_k\ }.
\]
Pre-collapse comparisons are out of scope. Overlap checks use the same order (Overlap Gate; Chs.~1 \& 5).
\end{remark}

\subsection*{11.0++.\ \(\Lambda_{\mathrm{len}}\) audit (default) and T–PFBC–AfterCollapse}
Throughout this chapter we \emph{adopt by default} the \emph{length-spectrum audit} $\Lambda_{\mathrm{len}}$ (Def.~\ref{def:11-len-operator}) for stability-band diagnostics. Any PF/BC-based comparator or transfer (Appendix~N) is evaluated only \emph{after} truncation (\textbf{T–PFBC–AfterCollapse}; cf.\ Ch.~9), and residuals are charged to $\delta_{\disc}\oplus\delta_{\meas}$ in the ledger (Appendix~L).

\subsection*{11.1. Length spectrum and invariance}
Let $\Lambda_{\mathrm{len}}(M;W)$ denote the length-spectrum operator of Ch.~2, Definition~\ref{def:len-operator} (we also write $\Len(M;W)$).

\begin{proposition}[Length spectrum equals clipped bar lengths; invariance]\label{prop:11-len}
If $M\simeq \bigoplus_j I[b_j,d_j)$ in $\Pers^{\mathrm{cons}}_k$ and $W=[u,u')$, then the eigenvalue multiset of $\Lambda_{\mathrm{len}}(M;W)$ is $\{\ell_W(I[b_j,d_j))\}_j$, with $\ell_W$ the Lebesgue length of $I[b_j,d_j)\cap W$. Hence the total collapse energy $\mathrm{PE}^{\le\tau}_i$ (Definition~\ref{def:11-PE}) equals the $L^1$-mass of $\Lambda_{\mathrm{len}}\!\big(\Ttau\mathbf{P}_i(F);[0,\tau]\big)$ and is invariant under isomorphisms in $\Pers^{\mathrm{cons}}_k$.
\end{proposition}

\begin{definition}[Stability bands via \(\Lambda_{\mathrm{len}}\)]\label{def:11-stability-band}
Fix $(i,\tau)$ and a window $W$. The \emph{$\Lambda_{\mathrm{len}}$-stability band} on $W$ is the maximal subunion $B\subseteq W$ such that
\[
\big\|\Lambda_{\mathrm{len}}(\Ttau\mathbf{P}_i(F_t);[0,\tau])-\Lambda_{\mathrm{len}}(\Ttau\mathbf{P}_i(F_{t'});[0,\tau])\big\|_{1}\ \le\ \delta_{\mathrm{win}}
\quad\text{for all }t,t'\in B.
\]
Here $\|\cdot\|_1$ is the trace (sum of eigenvalues) norm; the policy fixes the norm choice in the manifest.
\end{definition}

\subsection*{11.2. Collapse energy (windowed persistence energies)}
Fix degree $i$, scale $\tau\ge 0$, and exponent $\alpha>0$ (default $\alpha=1$). For a bar $b=[b_\ell,b_r)$ in the barcode of $\mathbf{P}_i(F)$ set
\[
\ell_\tau(b)=\bigl(\min\{b_r,\tau\}-\min\{b_\ell,\tau\}\bigr)_+,\qquad (x)_+=\max\{x,0\}.
\]
With weights $w_i:\mathcal{B}_i(F)\to[0,\infty)$ (default $1$):
\begin{definition}[Windowed energies]\label{def:11-PE}
\[
\mathrm{PE}^{\le\tau}_i(F;w_i,\alpha)=\sum_{b\in\mathcal{B}_i(F)}w_i(b)\,\big(\ell_\tau(b)\big)^\alpha,\qquad
\mathbf{CE}^{\le\tau}(F)=\big(\mathrm{PE}^{\le\tau}_i(F)\big)_i,\ \ \|\mathbf{CE}^{\le\tau}(F)\|_1=\sum_i\mathrm{PE}^{\le\tau}_i(F).
\]
All quantities are computed on $\Ttau\mathbf{P}_i(F)$ (equivalently on $\Ctau F$).
\end{definition}

\begin{remark}[Stability and deletion-type monotonicity]\label{rk:11-PE-stab}
$1$-Lipschitz updates under $\mathbf{P}_i$ yield non-expansive changes of $\mathrm{PE}^{\le\tau}_i$; \emph{deletion-type} updates (Appendix~E) make $\mathrm{PE}^{\le\tau}_i$ \emph{non-increasing} up to \fqi. Since $\Ttau$ is $1$-Lipschitz for interleaving distance, the same bounds hold after truncation. Under the tower hypotheses (Ch.~4), $(\muc,\nuc)=(0,0)$ at fixed $\tau$ excludes Type~IV.
\end{remark}

\subsection*{11.3. Spectral indicators on \(L(\Ctau F)\)}
Let $L(\Ctau F)$ be the normalized combinatorial Hodge Laplacian (per degree). Write $\{\lambda_j\}_{j\ge1}$ for its positive eigenvalues (zero modes omitted or handled by the Moore–Penrose pseudoinverse).

\begin{definition}[Spectral tails and heat traces]\label{def:11-spectral}
Fix $\beta>0$ and a cutoff policy $M(\tau)\in\mathbb{N}$. Set
\[
\mathrm{ST}^{\ge M(\tau)}_\beta(F)=\sum_{j\ge M(\tau)}\lambda_j^{-\beta},\qquad
\mathrm{HT}(t;F)=\sum_{j\ge1}e^{-t\lambda_j},\ \ t>0,
\]
with policies such as: (i) $M(\tau)=\lfloor c\,\tau^\gamma\rfloor$ ($c>0$, $\gamma\in[0,2]$); (ii) $t\in[c_1\tau^{-2},c_2\tau^{-2}]$ ($0<c_1\le c_2$).
\end{definition}

\begin{remark}[Spectral proof obligations; App.~E only]\label{rk:11-spectral-proof}
We rely solely on Appendix~E: deletion-type steps imply \emph{non-increase} of $\mathrm{ST}_\beta^{\ge M(\tau)}$ and $\mathrm{HT}(t;-)$; $\varepsilon$-continuations imply \emph{stability}. No further spectral claims are used.
\end{remark}

\begin{remark}[Mandatory ordering and norms]\label{rk:11-spectral-order}
Eigenvalues are stored in \emph{ascending} order; the matrix norm for tolerances is logged as \texttt{norm}$\in\{\texttt{op},\texttt{fro}\}$ (Appendix~G).
\end{remark}

\subsection*{11.4. Auxiliary spectral bars (aux-bars)}
Fix a spectral window $[a,b]$ and bin width $\beta>0$; bins are right-open $I_r=[a+r\beta,a+(r+1)\beta)$, $r=0,\dots,R-1$.

\begin{definition}[Occupancies and aux-bars]\label{def:11-aux}
For sample index $j$ (e.g.\ time) let $\{\lambda_m(j)\}_{m\ge1}$ be the positive spectrum of $L(\Ctau F_j)$ and $E_r(j)=\#\{m:\lambda_m(j)\in I_r\}$, with under/overflow $E_{<a},E_{\ge b}$ recorded. For each $r$, an \emph{aux-bar} is a maximal consecutive run $J$ with $E_r(j)>0$ for all $j\in J$; its lifetime is $|J|$ (or a rescaling).
\end{definition}

\begin{proposition}[Cumulative profile monotonicity and stability]\label{prop:11-aux}
For $C_r(j)=\sum_{s=r}^{R-1}E_s(j)$:
\begin{enumerate}
\item (\emph{Deletion-type}) $C_r(j\!+\!1)\le C_r(j)$ for all $r$.
\item (\emph{$\varepsilon$-continuation}) If $\|A_{j+1}-A_j\|_{\mathrm{op}}\le\varepsilon$, then
$C_{r+q}(j+1)\le C_r(j)\le C_{\max\{0,r-q\}}(j+1)$ with $q=\lceil\varepsilon/\beta\rceil$.
\end{enumerate}
(See Appendix~E.)
\end{proposition}

\begin{remark}[Bin policy]\label{rk:11-bin}
Aux-bars are computed \emph{after} collapse and with a fixed policy $(a,b,\beta)$ per window; boundary is right-open; under/overflow are part of the manifest (Appendix~G).
\end{remark}

\subsection*{11.5. Categorical check (one-way bridge)}
With $\Qtest=\{k[0]\}$, we monitor
\[
\Ext^1\big(\Rfun(\Ctau F),Q\big)=0\qquad(Q\in\Qtest),
\]
performed \emph{after truncation} and only in the one-way direction under (B1)–(B3).

\subsection*{11.6. Collapse diagnostics along towers}
For index category $I$ with cone apex $\infty$,
\[
\phi_{i,\tau}:\ \varinjlim_{t\in I}\ \Ttau\mathbf{P}_i(F_t)\ \longrightarrow\ \Ttau\mathbf{P}_i(F_\infty),
\]
\[
\mu_{i,\tau}=\dim_k\ker\phi_{i,\tau}\footnote{Here $\dim_k$ denotes the \emph{generic fiber} dimension after truncation, i.e.\ the multiplicity of $I[0,\infty)$ summands; see Appendix~D, Remark~\ref{rem:D-generic-dim}.},\quad
\nu_{i,\tau}=\dim_k\mathrm{coker}\,\phi_{i,\tau},\quad
\muc=\sum_i\mu_{i,\tau},\ \nuc=\sum_i\nu_{i,\tau}.
\]
Under the hypotheses of Ch.~4, each $\phi_{i,\tau}$ is an isomorphism and $(\muc,\nuc)=(0,0)$.

\subsection*{11.7. Overlap Gate and measurement}
\begin{declaration}[Overlap-aware measurement policy]\label{dec:11-overlap}
For a right-open cover $\{W_\alpha\}$ and fixed $(i,\tau)$:
\begin{enumerate}
\item \textbf{Local:} compute $\Ttau\mathbf{P}_i(F|_{W_\alpha})$ and all indicators \emph{after} truncation.
\item \textbf{Overlaps:} require collapse-compatibility within the $\delta$-budget (Appendix~L), Čech–$\Ext^1$-acyclicity in degree $1$ after truncation, and $(\mu,\nu)=(0,0)$ with near-$\tau$ non-accumulation.
\item \textbf{Global:} when overlaps pass, glue to a global truncated object (Ch.~5) and audit \emph{after} truncation.
\end{enumerate}
The manifest (Remark~\ref{rk:11-impl}) must log overlap checks (boolean), Čech–$\Ext^1$ status, and overlap $\delta$-budgets.
\end{declaration}

\subsection*{11.8. Joint monitoring protocol}
\begin{declaration}[Specification: joint monitoring]\label{spec:11-monitor}
Fix a finite sweep $\tau\in[\tau_{\min},\tau_{\max}]$ and a policy $(\alpha,w_i;\ \beta,M(\tau),t)$.
For each sample $t\in I$ and degree $i$:
\begin{enumerate}
\item \emph{Compute \& record} $\Ttau\mathbf{P}_i(F_t)$; evaluate $\mathrm{PE}^{\le\tau}_i(F_t)$ on $\Ttau\mathbf{P}_i(F_t)$ (equivalently on $\Ctau F_t$).
\item \emph{Compute \& record} $\mathrm{ST}_\beta^{\ge M(\tau)}(F_t)$ and $\mathrm{HT}(t;\Ctau F_t)$ using $L(\Ctau F_t)$; compute aux-bars via Definition~\ref{def:11-aux} with fixed $(a,b,\beta)$.
\item \emph{Check} $\Ext^1\big(\Rfun(\Ctau F_t),Q\big)=0$ for $Q\in\Qtest$.
\item \emph{Evaluate} $(\muc,\nuc)$ at each $\tau$ via $\phi_{i,\tau}$; log failure type (pure kernel/cokernel/mixed) if $(\muc,\nuc)\neq(0,0)$.
\item \emph{Declare stable} at $\tau$ when (1)–(3) hold jointly and $(\muc,\nuc)=(0,0)$. On definable windows, $E_1(W)=0$ short-circuits the PH/Ext checks (Theorem~\ref{thm:11-E1-local-reprise}).
\end{enumerate}
All persistence-layer statements are \fqi-invariant by construction; spectral/aux-bar steps are \emph{stable} under the fixed policy on $L(\Ctau F_t)$ (Appendix~E).
\end{declaration}

\subsection*{11.9. Noise, discretization, and $\delta$-ledger}
\begin{declaration}[Specification: noise/discretization policy]\label{spec:11-noise}
Let $\varepsilon>0$ be the noise scale.
\begin{itemize}
\item \textbf{Barcode denoising:} remove bars of length $\le\varepsilon$ within the $\tau$-window (\emph{$\varepsilon$-clipping}); bottleneck perturbations $\le\varepsilon$ preserve f.q.i.~invariants.
\item \textbf{Energy stability:} there exists $C_{i,\tau,\alpha}$ with
$\big|\mathrm{PE}^{\le\tau}_i(F)-\mathrm{PE}^{\le\tau}_i(\tilde F)\big|\le C_{i,\tau,\alpha}\,\varepsilon^{\min\{1,\alpha\}}$
whenever $d_{\mathrm{int}}(\Ttau\mathbf{P}_i(F),\Ttau\mathbf{P}_i(\tilde F))\le\varepsilon$.
\item \textbf{Spectral stabilization:} compute spectra on $L(\Ctau F)$; keep $(\beta,M(\tau),t)$ fixed. Averaging over $N$ runs reduces variance as $N^{-1/2}$. Ignore aux-bar lifetimes $\le 2$ frames if declared in the manifest.
\item \textbf{Resolution rule:} minimal resolvable feature length $\ge 3$ grid steps; sweep $\tau$ on a lattice $\Delta\tau\le\frac12$ of the minimal resolvable bar length.
\item \textbf{$\delta$-ledger:} decompose per-step $\delta(i,\tau)=\delta^{\mathrm{alg}}+\delta^{\mathrm{disc}}+\delta^{\mathrm{meas}}$ in $V$ and aggregate additively; with layered boxes $(\delta^{\mathrm{Gal}},\delta^{\mathrm{Tr}},\delta^{\mathrm{Fun}})$ for cross-layer runs.
\end{itemize}
\end{declaration}

\subsection*{11.10. Saturation gate \textbf{[Spec]}}
\begin{declaration}[Window-local saturation]\label{gate:11-sat}
Fix $\tau^\ast>0$ and parameters $\eta,\delta>0$. On $[0,\tau^\ast]$, assume:
(i) eventually the maximal \emph{finite} bar length in $\Ttau^\ast\mathbf{P}_i(F_t)$ is $\le\eta$;
(ii) eventually $d_{\mathrm{int}}\!\big(\Ttau^\ast\mathbf{P}_i(F_t),\Ttau^\ast\mathbf{P}_i(F_{t'})\big)\le\eta$;
(iii) the edge gap $\delta=\tau^\ast-\max\{b_r<\tau^\ast\}$ satisfies $\delta>\eta$.
Then, \textbf{within this window only}, adopt the temporary binary policy
\[
\PH_1(\C_{\tau^\ast}F)=0\quad\Longleftrightarrow\quad \Ext^1(\Rfun(\C_{\tau^\ast}F),k)=0.
\]
\end{declaration}

\subsection*{11.11. Artifacts, manifests, and minimal schema}
\begin{remark}[Implementation notes]\label{rk:11-impl}
\emph{Artifacts.} (i) \texttt{bars.json/h5}: records $\langle i,b_\ell,b_r,w\rangle$; (ii) \texttt{spec.json/h5}: positive eigenvalues of $L(\Ctau F)$ per degree; (iii) \texttt{aux.json/h5}: occupancies $E_r(j)$ and bin metadata; (iv) \texttt{ext.json}: boolean for $\Ext^1(\Rfun(\Ctau F),Q)$ with minimal witness; (v) \texttt{phi.json}: ranks of $\phi_{i,\tau}$ and $(\mu_{i,\tau},\nu_{i,\tau})$. \emph{Run log.} Store: sweep $\tau_{\min}$:$\Delta\tau$:$\tau_{\max}$; $(\alpha,w_i;\beta,M(\tau),t)$; discretization (grid/complex, step sizes); seeds; software versions; $\delta$-ledger per step; bin window $[a,b]$, width $\beta$, under/overflow; mandatory spectral fields \texttt{order=ascending}, \texttt{norm}$\in\{\texttt{op},\texttt{fro}\}$; overlap checks; Čech–$\Ext^1$ status; tail-isomorphism flag; optional length-spectrum summary.
\end{remark}

\noindent Minimal \texttt{run.yaml} block (augmented):
\begin{verbatim}
quantale:
  name: "[0,inf]_plus"
  op: "+"
  unit: 0.0
  order: "<="
layered_delta: {deltaGal: 0.002, deltaTr: 0.003, deltaFun: 0.002}
windows:
  domain: [[0,1), [1,2), [2,3)]
  collapse_tau: 0.08
  spectral_bins: {a: 0.0, beta: 0.02, bins: 96, boundary: "right-open"}
coverage_check: {length_sum: 3.0, length_target: 3.0, events_sum_equals_global: true}
overlap_checks: {local_equal_after_collapse: true, cech_ext1_ok: true, stability_band_ok: true}
spectral_policy: {order: "ascending", norm: "op"}
operations:
  - U: mollify; type: deletion; tau: 0.08; delta: {alg:0.004, disc:0.003, meas:0.001}
  - U: timestep; type: epsilon;  tau: 0.08; eps:0.006; delta: {alg:0.000, disc:0.002, meas:0.001}
persistence:
  E1_zero_window: true
  PH1_zero: true
  Ext1_zero: true
  mu: 0
  nu: 0
  phi_iso_tail: true
length_spectrum: {degree: 1, tau: 0.08, eigenvalues: [0.24, 0.51, 0.78]}
spectral:
  ST_beta: 2
  ST_M_of_tau: "floor(0.5 * tau^1.5)"
  HT_t: [0.5*tau^-2, 1.0*tau^-2]
  aux_bars_remaining: 0
budget: {sum_delta: 0.011, safety_margin: 0.025}
gate: {accept: true}
\end{verbatim}

\subsection*{11.12. Compliance checks and unit tests}
\begin{declaration}[Minimal test suite]\label{spec:11-tests}
Every deployment must pass:
\begin{itemize}
\item \textbf{Stability test.} Under synthetic $\varepsilon$-perturbations, verify non-expansiveness of $\mathrm{PE}^{\le\tau}_i$ and stability of spectral indicators and aux-bars under the fixed policy.
\item \textbf{Monotone-update test.} For a deletion-type update, confirm non-increase of $\mathrm{PE}^{\le\tau}_i$, spectral tails, and active-bin mass; record $(\muc,\nuc)=(0,0)$ at fixed $\tau$.
\item \textbf{Cone-extension test.} Verify that $\phi_{i,\tau}$ is an isomorphism on a model tower (hence $(\muc,\nuc)=(0,0)$) and that Type~IV is excluded at that $\tau$.
\item \textbf{Categorical check.} On a curated sample, confirm stability of $\Ext^1(\Rfun(\Ctau F),Q)=0$ under admissible \fqi-updates.
\item \textbf{T–$\Lambda_{\mathrm{len}}$–Consistency (optional).} Check that $\big\|\Lambda_{\mathrm{len}}(\Ttau\mathbf{P}_i(F_{t});[0,\tau])-\Lambda_{\mathrm{len}}(\Ttau\mathbf{P}_i(F_{t'});[0,\tau])\big\|_{1}\le \delta_{\mathrm{win}}$ along declared stability bands (Def.~\ref{def:11-stability-band}), and that
$\mathrm{PE}^{\le\tau}_i(F_{t})=\mathrm{trace}\big(\Lambda_{\mathrm{len}}(\Ttau\mathbf{P}_i(F_{t});[0,\tau])\big)$ holds to numerical tolerance.
\end{itemize}
\end{declaration}

\subsection*{11.13. Diagram (pipeline and logs)}
\begin{center}
\begin{tikzcd}[
  ampersand replacement=\&,
  column sep=2.4em, row sep=2.0em,
  cells={nodes={font=\footnotesize}},
  every label/.append style={font=\scriptsize},
  scale=0.9, transform shape
]
\text{Input }F \arrow[r, "\mathbf{P}_i"]
  \& \mathbf{P}_i(F) \arrow[r, "\Ttau"] \arrow[d, dashed, "{\mathrm{PE}_i^{\le \tau}}"]
    \& \Ttau\mathbf{P}_i(F) \arrow[d, dashed, "{\mathrm{PE}_i^{\le \tau}}"] \\
\& L(\Ctau F) \arrow[r, dashed, "{\text{heat/aux on }L(\Ctau F)}"]
    \& \mathrm{HT},\ \mathrm{ST}_{\beta},\ \text{aux-bars} \\
\Rfun(F) \arrow[uu, bend left=40, dashed, "\LC"]
  \arrow[rr, dashed, "{\Ctau\ \text{ then }\Ext^1(-,Q)}"]
    \&\& \Ext^1(\Rfun(\Ctau F),Q)=0 \\
\varinjlim_{t}\ \Ttau\mathbf{P}_i(F_t) \arrow[rr, "{\phi_{i,\tau}}"]
  \&\& \Ttau\mathbf{P}_i(F_\infty) \arrow[lld, bend right=25, swap, dashed, "{(\muc,\nuc)\ \text{log}}"] \\
\text{failure log (pure kernel/cokernel/mixed)} \&\&
\end{tikzcd}
\end{center}

\subsection*{11.14. Completion note}
\begin{remark}[No further supplementation required]\label{rk:11-done}
This chapter fully integrates: (i) $E_1$ as first-class determinant on definable windows with finite-event/finite-Čech guarantees (App.~H/J) and a reprise of the local bridge (Thm.~\ref{thm:11-E1-local-reprise}); (ii) default $\Lambda_{\mathrm{len}}$ audit for stability-band diagnostics with invariance via Prop.~\ref{prop:11-len}; (iii) spectral auxiliaries restricted to deletion-type non-increase and $\varepsilon$-stability with proofs deferred to App.~E; (iv) Overlap Gate enforcement and the unique comparison order; (v) tower diagnostics $(\muc,\nuc)$ and the joint monitoring protocol; (vi) noise/discretization rules and a quantale $\delta$-ledger (with layered boxes); (vii) artifacts/manifests with mandatory audit fields; (viii) a minimal, IMRN/AiM-ready test suite including the optional T–$\Lambda_{\mathrm{len}}$–Consistency check. All claims remain in the B-side single-layer scope; no additional supplementation is needed for operational use.
\end{remark}

\subsection*{11.15. Guard-rails}
\begin{remark}[Scope and non-claims]\label{rk:11-guard}
This chapter specifies measurement protocols and auxiliaries at the persistence/spectral/categorical layers. No analytic regularity, group trivialization, or number-theoretic identity is asserted. All statements respect the guard-rails of Part~I; in particular, no claim of $\mathrm{PH}_1\Leftrightarrow\Ext^1$ is made, and $\muc$ differs from the classical Iwasawa $\mu$.
\end{remark}



\section{Chapter 12: Formal Test Suite and Open Problems}
\addcontentsline{toc}{section}{Formal Test Suite and Open Problems}
\label{sec:ch12}

% --- Badge policy (explicit at chapter head) ---
\noindent\textbf{Badge policy.}
\begin{itemize}
  \item \textbf{[Prop]}: mathematics proved in Part~I (core results; cite exact source).
  \item \textbf{[Declaration]}: programmatic specification in the implementable range, verifiable by the test suite in this chapter.
  \item \textbf{[Conjecture]}: forward-looking statement; no claim beyond the stated scope.
\end{itemize}

% --- Notation & Conventions block for the whole chapter ---
\subsection*{12.0. Notation \& conventions}
\begin{itemize}
  \item \textbf{Constructible range.} We identify $\Perskft$ with the constructible subcategory of $[\mathbb{R},\mathsf{Vect}_k]$ and use $\Perskft$ uniformly.
  \item \textbf{Truncation phrase.} “after applying $\mathbf{T}_\tau$; equivalently on $\Ctau F$’’ indicates computation at the persistence layer after truncation (hence equivalently on the filtered lift $\Ctau F$).
  \item \textbf{Comparison order (unique).} All PF/BC and metric/spectral comparisons use the \emph{single} order
  \[
  \text{for each } t\ \Longrightarrow\ \mathbf{P}_i\ \Longrightarrow\ \mathbf{T}_\tau\ \Longrightarrow\ \text{compare in }\Perskft,
  \]
  i.e.\ \textbf{T–PFBC–AfterCollapse}. Pre\hyp collapse comparisons are out of scope (Chs.~1, 9, 11).
  \item \textbf{Generic–fiber dimension.} For a comparison map $\phi_{i,\tau}$ at fixed $\tau$, $\dim_k$ denotes the \emph{generic–fiber dimension after truncation}, i.e.\ the multiplicity of $I[0,\infty)$ summands in $\Ttau\mathbf{P}_i(-)$; informally, the $t\to\infty$ stable rank within the $\tau$–window.
  \item \textbf{Spectral ordering and norms.} Positive eigenvalues of $L(\Ctau F)$ are listed in ascending order $\lambda_1\le \lambda_2\le\cdots$. Matrix/operator norms are $\|\cdot\|_{\mathrm{op}}$ and $\|\cdot\|_{\mathrm{fro}}$; each test declares and logs its choice.
  \item \textbf{Obstruction totals and macros.} $\mu_{i,\tau}=\dim_k\ker\phi_{i,\tau}$, $\nu_{i,\tau}=\dim_k\mathrm{coker}\,\phi_{i,\tau}$, totals $\muc=\sum_i\mu_{i,\tau}$, $\nuc=\sum_i\nu_{i,\tau}$ (finite by bounded degree).
  \item \textbf{Endpoints.} Endpoint conventions and infinite bars follow the global policy (Appendix~A); infinite bars are not removed by $\Ttau$ and are clipped by the window in all windowed quantities.
  \item \textbf{Non\hyp expansiveness.} We use the spelling “non\hyp expansive’’/“non\hyp expansiveness’’ uniformly.
  \item \textbf{Quantale \& $V$-enrichment.} All $\delta$-budgets live in a fixed commutative quantale $V$ (Appendix~S); $V$-Lipschitz and $V$-nucleus properties are tested in~T16. Layered boxes $(\delta^{\Gal},\delta^{\Tr},\delta^{\Fun})$ are mandatory (Ch.~9).
  \item \textbf{Definable windows.} Windows $W$ are definable in a fixed o\hyp minimal (Archimedean) or Denef–Pas (non\hyp Archimedean) structure; definability and finite Čech depth are tested in~T17.
\end{itemize}

\subsection*{12.1. Badge inventory (representative items)}
\begin{center}
\begingroup
\renewcommand{\arraystretch}{1.12}
\begin{tabular}{@{}l >{\raggedright\arraybackslash}p{0.74\textwidth}@{}}
\toprule
\textbf{Badge} & \textbf{Representative items (label / location)}\\
\midrule
\textbf{[Prop]} &
  Stability, idempotence, and exactness of $\Ttau$ (Prop.~\ref{prop:stability}, Ch.~2);\\
& Shift–commutation / $1$\nobreakdash-Lipschitz for $\Ttau$ (Lemma~\ref{lem:shift}, Ch.~2);\\
& Operational coreflection $\mathsf{C}_\tau^{\mathrm{comb}}$ on the implementable range
  (Prop.~\ref{prop:operational-coreflection}, Ch.~5);\\
& Tower diagnosis: $(\muc,\nuc)$ via cone extension; isomorphism criterion excluding Type~IV (Prop.~\ref{prop:mu-vanishing}, Ch.~4).\\
\textbf{[Thm]} &
  One\hyp way bridge: $\mathrm{PH}_1(F)=0 \Rightarrow \Ext^1(\mathcal{R}(F),k)=0$ under (B1)–(B3) (Thm.~\ref{thm:PH1-to-Ext1}, Ch.~3);\\
& Local bridge on definable windows (Thm.~\ref{thm:E1-local}, Ch.~3; reprise Ch.~11).\\
\textbf{[Declaration]} &
  Ch.~2: (co)limit and pullback compatibility \emph{at the persistence layer only} (after $\Ttau$);\\
& Ch.~6: filtered\hyp colimit stability in geometry; joint indicators and protocol (after truncation);\\
& Ch.~7: arithmetic tower stability; non\hyp identity of $\muc$ with Iwasawa $\mu$;\\
& Ch.~8: tropical shortening $\Rightarrow$ weak group collapse; mirror transfer \emph{non\hyp expansive after truncation};\\
& Ch.~9: three\hyp layer (Gal$\to$Trans$\to$Funct) compatibility as isomorphisms in $\Perskft$ \emph{after} $\Ttau\mathbf{P}_i$;\\
& Ch.~10: persistence\hyp guided regularization; AK–NS hypothesis (programmatic);\\
& Ch.~11: joint monitoring, noise/discretization policy, minimal test suite; Saturation gate.\\
\textbf{[Conjecture]} &
  Cross\hyp domain collapse propagation (Chs.~6–10); AK–NS (Ch.~10); mirror\hyp side propagation (Ch.~8);\\
& Functorial transfer stability (Ch.~9).\\
\bottomrule
\end{tabular}
\endgroup
\end{center}

\subsection*{12.2. Formal test suite (unit / integration / regression)}
All tests operate at the truncated persistence, spectral (on $L(\Ctau F)$), and categorical layers and are \fqi\ invariant at the persistence layer. A test \emph{passes} iff all stated pass\hyp criteria are met and logs are complete. Pass\hyp criteria must state whether indicators are evaluated \emph{per degree} or \emph{aggregated} across degrees; the choice must be fixed and logged for the run. Spectra use ascending order $\lambda_1\le\lambda_2\le\cdots$, and the chosen norm $\|\cdot\|_{\mathrm{op}}$ or $\|\cdot\|_{\mathrm{fro}}$ must be declared and logged.

\paragraph{(T1) Stability under non\hyp expansive updates [Unit].}
\emph{Input:} pairs $F\to F'$ with $d_{\mathrm{int}}(\mathbf{P}_i(F),\mathbf{P}_i(F'))\le \varepsilon$.\\
\emph{Assertions:} $|\mathrm{PE}^{\le\tau}_i(F)-\mathrm{PE}^{\le\tau}_i(F')|\le C_{i,\tau,\alpha}\varepsilon^{\min\{1,\alpha\}}$ (after $\Ttau$; equivalently on $\Ctau$); spectra of $L(\Ctau F)$ vs.\ $L(\Ctau F')$ satisfy the fixed $(\beta,M(\tau),t)$\hyp policy stability bounds in the declared norm; $\Ext^1(\Rfun(\Ctau-),Q)$ is stable under admissible \fqi\ updates ($Q\in\Qtest$).\\
\emph{Artifacts:} \texttt{bars.json}, \texttt{spec.json}, \texttt{ext.json}; \texttt{run.yaml} (norm and spectral policy recorded).

\paragraph{(T2) Monotone update (deletion–/inclusion–type) [Unit].}
\emph{Input:} $F\to F'$ monotone.\\
\emph{Assertions:} \emph{Deletion–type:} $\mathrm{PE}^{\le\tau}_i$ and spectral indicators are non\hyp increasing (after $\Ttau$). In the two\hyp term cone tower the $\phi_{i,\tau}$ map is an isomorphism, hence $(\muc,\nuc)=(0,0)$ at fixed $\tau$. \emph{Inclusion–type:} stability only.\\
\emph{Artifacts:} \texttt{bars.json}, \texttt{spec.json}, \texttt{ext.json}, \texttt{phi.json}; \texttt{run.yaml}.

\paragraph{(T3) Filtered\hyp colimit stability [Integration].}
\emph{Input:} tower $\{F_\lambda\}_\lambda$.\\
\emph{Assertions:} for fixed $\tau$, $\phi_{i,\tau}:\varinjlim_\lambda \Ttau\mathbf{P}_i(F_\lambda)\xrightarrow{\cong}\Ttau\mathbf{P}_i(F_{\lambda_\ast})$; thus $(\muc,\nuc)=(0,0)$ at that scale. Terminal symbol consistency is logged.\\
\emph{Artifacts:} \texttt{phi.json}, \texttt{run.yaml}.

\paragraph{(T4) Mirror/tropical pipeline [Integration].}
\emph{Input:} $X$, tropical flow $\Trop_\lambda$, realization $F_\lambda$, mirror functor $\Mirror$.\\
\emph{Assertions:} shortening factor $\kappa\le 1$ implies non\hyp increase of $\mathrm{PE}^{\le\tau}_i$ (after $\Ttau$); mirror transfer is non\hyp expansive after truncation; group proxies (if used) meet weak\hyp collapse thresholds (Ch.~8). Spectral eigenvalues ascending; norm declared.\\
\emph{Artifacts:} per\hyp $\lambda$ \texttt{bars/spec/ext/phi.json}; \texttt{run.yaml}.

\paragraph{(T5) Three\hyp layer compatibility [Integration].}
\emph{Input:} Gal$\to$Trans$\to$Funct data with comparison natural transformations (Ch.~9).\\
\emph{Assertions:} after $\Ctau$ and $\mathbf{P}_i$, commutativity holds up to isomorphism in $\Perskft$ per degree; indicators consistent; failures typed and logged.\\
\emph{Artifacts:} per\hyp layer \texttt{bars/spec/ext.json}; global \texttt{phi.json}; \texttt{run.yaml}.

\paragraph{(T6) PDE monitoring loop [Regression].}
\emph{Input:} index set $I$ (time/resolution/parameter), realization $\mathcal{P}$.\\
\emph{Assertions:} Ch.~11 protocol holds; stable regime flags match logs of $\mathrm{PE}^{\le\tau}$, spectral indicators \& aux\hyp bars, $\Ext^1$, $(\muc,\nuc)$; reporting choice (per degree vs.\ aggregated) fixed and logged; spectra ascending; norm declared.\\
\emph{Artifacts:} \texttt{bars/spec/aux/ext/phi.json} over $I$; \texttt{run.yaml}.

\paragraph{(T7) Saturation gate verification [Integration].}
\emph{Input:} window $[0,\tau^\ast]$ with candidate saturation (Ch.~11).\\
\emph{Assertions:} verify saturation parameters $(\eta,\delta)$ and record the window\hyp local binary policy $\PH_1(\C_{\tau^\ast}F)=0 \Leftrightarrow \Ext^1(\Rfun(\C_{\tau^\ast}F),k)=0$ (within the window only). Spectral reports ascending; norm declared.\\
\emph{Artifacts:} \texttt{bars.json}, \texttt{ext.json}; \texttt{run.yaml}.

\paragraph{(T8) $\varepsilon$\hyp clipping regression [Unit].}
\emph{Input:} paired runs with unclipped vs.\ $\varepsilon$\hyp clipped $\Ttau\mathbf{P}_i$ (Ch.~11).\\
\emph{Assertions:} energy stability bound; $(\muc,\nuc)$ computed on unclipped data and identical across the pair; logs distinguish clipping.\\
\emph{Artifacts:} \texttt{bars.json} (both), \texttt{phi.json}; \texttt{run.yaml}.

\paragraph{(T9) MECE window coverage \& event accounting [Unit].}
\emph{Input:} MECE windowing $\{[u_k,u_{k+1})\}_k$ and global range $[u_0,U)$.\\
\emph{Assertions:} coverage equality; event counts add up (up to tolerance); uniform $\tau$ and bin policy unless justified and logged.\\
\emph{Artifacts:} \texttt{run.yaml} (coverage\_check), per\hyp window \texttt{bars.json}.

\paragraph{(T10) A/B commutativity test for reflectors [Unit/Integration].}
\emph{Input:} two persistence\hyp level reflectors $T_A,T_B$, tolerance $\eta\ge 0$.\\
\emph{Assertions:} $\Delta_{\mathrm{comm}}(M;A,B)=d_{\mathrm{int}}(T_AT_BM,\ T_BT_AM)$; pass if $\Delta_{\mathrm{comm}}\le\eta$, else fallback order is used and defect added to $\delta^{\alg}$ (Appendix~L).\\
\emph{Artifacts:} \texttt{run.yaml} (A/B policy), \texttt{bars.json} before/after.

\paragraph{(T11) Restart \& Summability [Integration/Regression].}
\emph{Input:} windows with thresholds $\tau_k$, budgets $\Sigma\delta_k(i)$, margins $\mathrm{gap}_{\tau_k}$.\\
\emph{Assertions:} Restart: $\mathrm{gap}_{\tau_{k+1}}\ge \kappa(\mathrm{gap}_{\tau_k}-\Sigma\delta_k(i))$ (record $\kappa$). Summability: $\sum_k \Sigma\delta_k(i)<\infty$. Certificates paste.\\
\emph{Artifacts:} \texttt{run.yaml} (restart/summability), gate logs, global certificate.

\paragraph{(T12) Trigger pack verification (domain\hyp restricted) [Integration].}
\emph{Input:} declared triggers (e.g.\ PDE, Ch.~10).\\
\emph{Assertions:} each trigger implies a B\hyp Gate$^{+}$ failure on the window; detection rate and false positives logged. Triggers are \textbf{[Spec]} and complement (not replace) B\hyp Gate$^{+}$.\\
\emph{Artifacts:} \texttt{run.yaml} (thresholds), \texttt{aux.json}, \texttt{phi.json}, gate verdicts.

\paragraph{(T13) $\delta$-ledger additivity \& pipeline budget [Integration].}
\emph{Input:} steps $U_m,\dots,U_1$ with per\hyp step collapses $C_{\tau_j}$ and bounds $\delta_j(i,\tau_j)$.\\
\emph{Assertions:} verify
\[
d_{\mathrm{int}}\!\Big(\Ttau \mathbf{P}_i\big(\Mirror(C_{\tau_m}\!\cdots C_{\tau_1}F)\big),\ \Ttau \mathbf{P}_i\big(C_{\tau_m}\!\cdots C_{\tau_1}\Mirror F\big)\Big)\ \le\ \sum_{j=1}^m \delta_j(i,\tau_j),
\]
and that post\hyp processing by $1$\hyp Lipschitz maps does not increase the bound (Appendix~L).\\
\emph{Artifacts:} \texttt{run.yaml} (\,$\delta$-ledger), \texttt{bars.json}, distance logs.

\paragraph{(T14) Overlap Gate gluing test [Integration].}
\emph{Input:} charts $X_1,X_2$ with windows $W_1,W_2$ and overlap $X_{12}$; fixed $\tau$; reflectors $T_A,T_B$.\\
\emph{Assertions:} after $\mathbf{P}_i$ and $\mathbf{T}_\tau$, verify post\hyp collapse local equivalence on $X_{12}$ within budget, Čech–$\Ext^1$ vanishing, A/B soft\hyp commuting or logged fallback; construct glued truncated object and confirm B\hyp Gate$^{+}$.\\
\emph{Artifacts:} \texttt{run.yaml} (overlap/A\!/\!B/\,$\delta$), per\hyp chart/overlap \texttt{bars/ext.json}; global verdict.

\paragraph{(T15) Length spectrum audit [Unit].}
\emph{Input:} $\Ttau\mathbf{P}_i(F)$ and $\Len(\Ttau\mathbf{P}_i(F);[0,\tau])$.\\
\emph{Assertions:} eigenvalue multiset equals clipped bar\hyp length multiset (up to permutation); $L^1$-mass equals $\mathrm{PE}^{\le\tau}_i(F)$. Hash or canonical ordering recorded.\\
\emph{Artifacts:} \texttt{bars.json}, \texttt{Lambda\_len.json} (optional list/hash), \texttt{run.yaml}.

\paragraph{(T16) \textbf{V\hyp shift (Quantale) test} [Unit].}
\emph{Input:} a fixed commutative quantale $V$ with operation $\oplus$, unit $e$, order $\preceq$; Lawvere $V$-distance and $V$-shift operators $S^{v}$ ($v\in V$); truncation $\Ttau$.\\
\emph{Assertions:} (i) $V$-Lipschitz: $d_V(\Ttau S^{v}M,\Ttau S^{v}N)\preceq d_V(\Ttau M,\Ttau N)$ for all $v$ (Ch.~2, Lemma~“$V$-shift’’); (ii) commutation: $\Ttau\circ S^{v}\cong S^{v}\circ \Ttau$ at the persistence layer; (iii) composition: $S^{v_2}\circ S^{v_1}\cong S^{v_1\oplus v_2}$ and quantitative defect (if any) is recorded in $\delta^{\alg}$.\\
\emph{Artifacts:} \texttt{run.yaml} (\texttt{quantale:\{name, op, unit, order, mode\}}), \texttt{bars.json} before/after $S^{v}$, distance logs.

\paragraph{(T17) \textbf{Definable coverage \& Čech finiteness} [Integration].}
\emph{Input:} a window cover by formulas $\{\varphi_\alpha(x)\}$ in a fixed o\hyp minimal or Denef–Pas structure; right\hyp open windows $W_\alpha=\{x:\varphi_\alpha(x)\}$.\\
\emph{Assertions:} (i) definability check passes for each $\varphi_\alpha$; (ii) finite event count on each $W_\alpha$ (piecewise constant Betti integrals; Appendix~H); (iii) finite Čech depth on the cover (Appendix~J); (iv) Overlap Gate checks pass with logged $\delta$-budgets; (v) optional confirmation of Ch.~3 local bridge when $E_1(W_\alpha)=0$.\\
\emph{Artifacts:} \texttt{run.yaml} (\texttt{definable:\{structure, window\_formulae\}}), per\hyp window \texttt{bars.json}, overlap logs.

\paragraph{(T18) \textbf{Iwasawa control $\Rightarrow$ Overlap Gate} [Integration].}
\emph{Input:} arithmetic tower (Ch.~7) with a Control theorem yielding finite kernel/cokernel on comparison maps; Overlap Gate configuration.\\
\emph{Assertions:} (i) finite kernel/cokernel are absorbed into $\delta^{\alg}$ as per policy; (ii) post\hyp collapse comparison maps satisfy $\phi_{i,\tau}$ isomorphism on windows where the control bound holds, hence $(\muc,\nuc)=(0,0)$ at fixed $\tau$; (iii) Overlap Gate passes with recorded $\delta$; (iv) explicit bounds \texttt{control\_finite\_bounds} are logged.\\
\emph{Artifacts:} \texttt{run.yaml} (\texttt{iwasawa:\{tower\_level, control\_finite\_bounds\}}), \texttt{phi.json}, gate logs.

% ---------------- Mandatory named tests (aliases and additions) ----------------
\subsection*{12.2a. Mandatory named tests (aliases/additions)}
The following \emph{named} tests are \textbf{mandatory}. Each comes with a canonical alias into (T1)–(T18) for reporting.

\paragraph{(T–ExtZero$\Rightarrow$PHZero) \quad [Integration].}
\emph{Scope:} Only on definable windows with $E_1(W)=0$ (Thm.~\ref{thm:E1-local}) \emph{or} under the Saturation Gate (Ch.~11, Decl.~\ref{gate:11-sat}).\\
\emph{Assertions:} $\Ext^1(\Rfun(\Ctau F|_W),k)=0 \Rightarrow \PH_1(\Ctau F|_W)=0$ (window\hyp local). Outside this scope, the test is marked \texttt{inapplicable}.\\
\emph{Alias:} T7 (saturation) and T17 (definable). Artifacts as in T7/T17.

\paragraph{(T–Countable–Cover) \quad [Integration].}
\emph{Input:} a countable MECE cover $W=\bigsqcup_{n\ge 1}W_n$ with local finiteness on compact subintervals.\\
\emph{Assertions:} (i) coverage/equality checks of T9 extend to the countable case; (ii) Overlap Gate passes on each finite subcover; (iii) certificates paste by Restart/Summability (T11).\\
\emph{Alias:} T9$+$T11$+$T14. Artifacts as in those tests.

\paragraph{(T–Delta–Sum–Converges) \quad [Regression].}
\emph{Assertions:} $\sum_k \Sigma\delta_k(i)<\infty$ with logged tail bounds; global certificate exists.\\
\emph{Alias:} T11 (Summability). Artifacts: restart/summability block.

\paragraph{(T–Lipschitz–AfterCollapse) \quad [Unit].}
\emph{Assertions:} (i) $\Ttau$ is $1$–Lipschitz: $d_{\mathrm{int}}(\Ttau M,\Ttau N)\le d_{\mathrm{int}}(M,N)$; (ii) each declared update is $1$–Lipschitz \emph{after} $\Ttau$ within recorded $\varepsilon$; (iii) deletion–type steps achieve non\hyp increase of $\mathrm{PE}^{\le\tau}_i$.\\
\emph{Alias:} T1$+$T2. Artifacts: as in T1/T2.

\paragraph{(T–Exactness–Persistence) \quad [Unit].}
\emph{Input:} short exact sequences in $\Perskft$ (implementable range).\\
\emph{Assertions:} $\Ttau$ is exact on these sequences; induced maps on barcodes respect subquotients; equality verified after collapse.\\
\emph{Alias:} supports [Prop] “exactness of $\Ttau$’’ and audits via T10 (A/B) when multiple reflectors are present. Artifacts: before/after \texttt{bars.json}.

\paragraph{(T–Iwasawa–Alignment) \quad [Integration].}
\emph{Assertions:} control\hyp theorem comparators align with $\delta$–ledger: finite kernel/cokernel absorbed into $\delta^{\alg}$; Overlap Gate passes; $(\muc,\nuc)=(0,0)$ at fixed $\tau$.\\
\emph{Alias:} T18. Artifacts: as in T18.

\paragraph{(T–PFBC–AfterCollapse) \quad [Unit/Integration, optional].}
\emph{Assertions:} PF/BC steps computed objectwise in $t$ then compared only \emph{after} $\Ttau$; non\hyp expansiveness verified post\hyp truncation; any discretization/sampling residuals are charged to $\delta_{\disc}\oplus\delta_{\meas}$.\\
\emph{Alias:} extends T5/T13 with PF/BC flags. Artifacts: PF/BC comparator logs.

\paragraph{(T–$\Lambda_{\mathrm{len}}$) \quad [Unit, optional].}
\emph{Assertions:} identical to T15; provide canonical hash for eigenvalue multiset of $\Len(\Ttau\mathbf{P}_i(F);[0,\tau])$.\\
\emph{Alias:} T15. Artifacts: \texttt{Lambda\_len.json}.

% ---------------- Reproducibility & logs ----------------
\subsection*{12.3. Reproducibility and logs}
Every run ships with a manifest \texttt{run.yaml} declaring: sweep $\tau_{\min}\!:\Delta\tau\!:\tau_{\max}$; spectral policy $(\beta,M(\tau),t)$; discretization (grid/complex, steps); seeds; software versions; tower index set and cone extension (including the terminal symbol); pass\hyp criteria (per\hyp degree vs.\ aggregated); norm choice $\|\cdot\|_{\mathrm{op}}$ or $\|\cdot\|_{\mathrm{fro}}$; A/B tolerance $\eta$; Restart constants ($\kappa$) and Summability evidence; Overlap Gate status; file pointers to \texttt{bars/spec/aux/ext/phi.json} (optionally \texttt{.h5}). Persistence quantities are computed after $\Ttau$; equivalently on $\Ctau F$; and remain invariant under \fqi\ at the persistence layer.

\begin{declaration}[Schema extension and mandatory fields \textbf{[Spec]}]\label{dec:12-schema}
The following \texttt{run.yaml} fields are \emph{mandatory} for auditability (synchronized with Appendix~G):
\begin{itemize}
  \item \textbf{Quantale block} \texttt{quantale:\{name, op, unit, order, mode\}} (e.g.\ $[0,\infty]_{+}$, $(\max,+)$, probabilistic/product modes).
  \item \textbf{Layered $\delta$} \texttt{layered\_delta:\{deltaGal, deltaTr, deltaFun\}}.
  \item \textbf{Definable windows} \texttt{definable:\{structure, window\_formulae\}} (structure $\in\{\texttt{R\_an,exp},\texttt{Denef\!-\!Pas}\}$).
  \item \textbf{Iwasawa} \texttt{iwasawa:\{tower\_level, control\_finite\_bounds\}}.
  \item \textbf{AWFS/2\hyp cell} \texttt{awfs:\{enabled: bool, two\_cell\_bounds: value\}}.
  \item \textbf{Overlap Gate} \texttt{overlap\_checks:\{local\_equiv, cech\_ext1\_ok, stability\_band\_ok\}}.
  \item \textbf{Length spectrum} \texttt{Lambda\_len} per degree on $[0,\tau]$ (list or hash) for T15.
  \item \textbf{Spectral policy} \texttt{spectral\_policy:\{order: "ascending", norm: "op"|"fro"\}}, and \texttt{spectral\_bounds:\{lambda\_min, lambda\_max, lip\_tol?\}}.
  \item \textbf{Persistence} explicit $(\mu,\nu)$ totals and tail\hyp isomorphism flag \texttt{phi\_iso\_tail}.
  \item \textbf{Budget} \texttt{sum\_delta}, \texttt{safety\_margin}; for Restart, per\hyp window \texttt{gap\_tau}.
  \item \textbf{A/B} \texttt{ab\_test:\{eta, policy, fallback\}}.
\end{itemize}
\end{declaration}

\begin{remark}[Audit checklist]\label{rk:12-audit}
(i) Constructibility verified;\;
(ii) Coefficient field fixed (Novikov allowed at \textbf{[Spec]});\;
(iii) Deletion– vs.\ inclusion–type correctly labeled;\;
(iv) Uniform interleaving shifts $\varepsilon_n$ bounded;\;
(v) Same window for $\mathrm{PE}$, spectral, aux\hyp bars, $\Ext^1$, and $(\mu,\nu)$ after $\Ttau$;\;
(vi) LC order: $\Ctau$ then $\Rfun$ (one\hyp way bridge only);\;
(vii) PF/BC prechecks for derived transfers (App.~N); non\hyp expansiveness only \emph{after} truncation;\;
(viii) Spectra ascending; norm declared; terminal symbol consistent;\;
(ix) MECE coverage and event accounting satisfied (finite or countable);\;
(x) A/B commutativity configured (T10) and logged;\;
(xi) Restart/Summability evidenced (T11/T–Delta–Sum–Converges);\;
(xii) Overlap Gate fields complete (T14);\;
(xiii) Length spectrum audit recorded (T15/T–$\Lambda_{\mathrm{len}}$);\;
(xiv) Quantale/definable/Iwasawa/AWFS blocks complete (T16–T18);\;
(xv) T–PFBC–AfterCollapse flags present when PF/BC is used.
\end{remark}

\noindent\emph{Manifest template (YAML).}
\small
\begin{verbatim}
coeff_field: "k"                  # or "Novikov(q)" [Spec-level]
tau_window: [0.05, 1.0]           # start, end
tau_step: 0.05
quantale:
  name: "[0,inf]_plus"
  op: "+"
  unit: 0.0
  order: "<="
  mode: "standard"                # or "probabilistic", "product"
layered_delta: {deltaGal: 0.002, deltaTr: 0.003, deltaFun: 0.002}
definable:
  structure: "R_an,exp"           # or "Denef-Pas"
  window_formulae:
    - "u <= t < u'"
    - "t in union_{j=1..m} (a_j, b_j]"
iwasawa:
  tower_level: 5
  control_finite_bounds: {kernel_le: 2, cokernel_le: 3}
awfs:
  enabled: true
  two_cell_bounds: 0.01
spectral:
  tail_beta: 2
  tail_cutoff_M_of_tau: "floor(0.5 * tau^1.5)"
  heat_t: [0.5*tau^-2, 1.0*tau^-2]
  aux_bins: {a: 0.0, beta: 0.02, bins: 96, boundary: "right-open"}
spectral_policy:
  order: "ascending"
  norm: "op"
spectral_bounds:
  lambda_min: 1.0e-6
  lambda_max: 10.0
  lip_tol: 0.02
tower:
  eps_interleave_max: 0.02
  terminal_symbol: "infty"        # or "lambda_star"
  cone_extension: true
ab_test:
  eta: 0.01
  policy: "soft-commuting"        # or "fallback:A_then_B"
restart_summability:
  kappa_min: 0.8
  sum_delta_bound: 0.05
windows:
  domain: [[0,1), [1,2), [2,3)]   # finite or countable MECE cover
  collapse_tau: 0.08
coverage_check:
  length_sum: 3.0
  length_target: 3.0
  events_sum_equals_global: true
overlap_checks:
  local_equiv: true
  cech_ext1_ok: true
  stability_band_ok: true
pfbc:
  policy: "after_collapse"        # T–PFBC–AfterCollapse
  residual_ledger: ["disc","meas"]
persistence:
  PH1_zero: true
  Ext1_zero: true
  mu: 0
  nu: 0
  phi_iso_tail: true
Lambda_len:
  degree: 1
  tau: 0.08
  audit: "hash:2f4c...d1"
record:
  bars: true
  PE: {report: "per-degree", clipping: "epsilon=0.02"}
  aux: {lifetime_min_frames: 3}
  heat_trace: {ordering: "ascending", norm: "op"}
  ext1: true
  mu_nu: true
budget:
  sum_delta: 0.011
  safety_margin: 0.025
  gap_tau: 0.03
gate:
  accept: true
notes: "deletion-type only for monotonicity; LC with Rfun after truncation; PF/BC verified."
\end{verbatim}
\normalsize

\subsection*{12.4. Open problems (selected)}
\begin{remark}[Open problems]\label{rk:12-open}
\hfill
\begin{enumerate}
  \item \textbf{Quantitative bridge.} Domain\hyp wise sufficient conditions implying $\Ext^1=0$ from decay of $\|\mathbf{CE}^{\le \tau}\|_1$ and $\mathrm{ST}_\beta^{\ge M(\tau)}$ (window\hyp local).
  \item \textbf{Colimit criteria.} Sharp hypotheses guaranteeing $(\muc,\nuc)=(0,0)$ beyond objectwise degreewise colimits.
  \item \textbf{Failure lattice.} Finer invariants separating pure/mixed failures and anticipating Type~IV at nearby scales.
  \item \textbf{Spectral–persistence calibration.} Robust bounds between collapse energy and spectral tails under noise and discretization.
  \item \textbf{Weak group collapse.} Persistence\hyp level proxies vs.\ algebraic invariants without leaving the implementable range.
  \item \textbf{Arithmetic towers.} Templates linking collapse diagnostics to Selmer/class growth while keeping $\muc\neq \mu_{\mathrm{Iwasawa}}$.
  \item \textbf{Langlands layers.} Minimal comparison data for truncated commutativity across Gal$\to$Trans$\to$Funct.
  \item \textbf{PDE program.} Conditions under which persistence\hyp guided regularization predicts classical regimes programmatically.
  \item \textbf{Universality of $\mathbf{T}_\tau$.} Characterization of $\Ttau$ as Serre localization in the implementable range.
\end{enumerate}
\end{remark}

\subsection*{12.5. Final guard\hyp rails}
\begin{remark}[Scope and non\hyp claims]\label{rk:12-guards}
All specifications are confined to the persistence/spectral/categorical layers in the implementable range and are verifiable by the test suite above. No number\hyp theoretic identity, analytic regularity theorem, or group trivialization is asserted. In particular, no claim of $\mathrm{PH}_1\Leftrightarrow \Ext^1$ is made; only the one\hyp way implication under (B1)–(B3) is used. The obstruction $\muc$ is a collapse diagnostic and differs from the classical Iwasawa $\mu$.
\end{remark}

\subsection*{12.6. Effect and auditability}
\begin{remark}[Effect of the extensions]
Relative to v16.0, the harness is strengthened by: (i) Overlap Gate gluing (T14) with post\hyp collapse local equivalence and A/B soft\hyp commuting; (ii) A/B tests with manifest\hyp level tolerance (T10); (iii) Restart/Summability quantification (T11); (iv) Saturation Gate anchoring (T7); \emph{and additionally} (v) $V$\hyp shift verification for quantale\hyp enriched runs (T16); (vi) definable coverage \& Čech finiteness checks (T17); (vii) Iwasawa control integration with $\delta^{\alg}$ absorption (T18); (viii) \textbf{mandatory} named tests T–ExtZero$\Rightarrow$PHZero, T–Countable–Cover, T–Delta–Sum–Converges, T–Lipschitz–AfterCollapse, T–Exactness–Persistence, T–Iwasawa–Alignment; (ix) optional T–PFBC–AfterCollapse and T–$\Lambda_{\mathrm{len}}$. The schema (Decl.~\ref{dec:12-schema}) mandates quantale/definable/Iwasawa/AWFS/PFBC blocks, improving third\hyp party auditability.
\end{remark}

\subsection*{12.7. Conclusion}
This chapter consolidates a complete, testable interface: a precise badge policy, a uniform notation layer, and a formal test suite spanning stability, monotone updates, filtered\hyp colimits, mirror/tropical flows, Langlands triples, PDE pipelines, Overlap Gate gluing, quantale\hyp enriched shifts, definable coverage, and Iwasawa control. All persistence\hyp layer quantities are computed after $\Ttau$; spectral indicators are normalized (eigenvalues ascending; norm declared); categorical checks are performed only in the one\hyp way direction. Reproducibility is enforced by a single manifest with mandatory fields. The \emph{implementable range} is thus executable and auditable, with conservative guard\hyp rails and clear open directions.

\subsection*{12.8. Completion note}
\begin{remark}[No further supplementation required]
This chapter fully integrates: (i) MECE (finite/countable) window tests and event accounting; (ii) A/B commutativity with tolerance and fallback; (iii) Restart/Summability verification; (iv) Trigger pack validation; (v) $\delta$-ledger additivity; (vi) Overlap Gate gluing (T14); (vii) Length spectrum audit (T15/T–$\Lambda_{\mathrm{len}}$); (viii) Quantale $V$-shift tests (T16); (ix) definable coverage/Čech finiteness (T17); (x) Iwasawa control $\Rightarrow$ Overlap Gate (T18/T–Iwasawa–Alignment); (xi) PF/BC enforcement after collapse (T–PFBC–AfterCollapse); (xii) a manifest schema synchronized with Appendix~G. All items are consistent with the v16.0 guard\hyp rails and cross\hyp reference the proven core; no additional supplementation is needed for operational use as a formal test suite.
\end{remark}

% ---------------- Machine-readable badge & test index ----------------
\subsection*{12.9. Machine-readable badge \& test index (for automated extraction)}
\noindent\textit{This block is purely auxiliary for reproducibility tools and can be ignored in print.}
\begin{verbatim}
badge_index:
  proof_labels:   ["prop:stability","lem:shift","prop:operational-coreflection","prop:mu-vanishing"]
  theorem_labels: ["thm:PH1-to-Ext1","thm:E1-local"]
  declaration_chapters: [2,6,7,8,9,10,11,12]
mandatory_tests:
  - name: "T-ExtZero->PHZero"        ; alias: ["T7","T17"]  ; scope: "definable_or_saturation"
  - name: "T-Countable-Cover"        ; alias: ["T9","T11","T14"]
  - name: "T-Delta-Sum-Converges"    ; alias: ["T11"]
  - name: "T-Lipschitz-AfterCollapse"; alias: ["T1","T2"]
  - name: "T-Exactness-Persistence"  ; alias: []
  - name: "T-Iwasawa-Alignment"      ; alias: ["T18"]
optional_tests:
  - name: "T-PFBC-AfterCollapse"     ; alias: ["T5","T13"]
  - name: "T-Lambda_len"             ; alias: ["T15"]
pfbc_policy: "after_collapse"
spectral_policy: {order: "ascending", norm: "op"}
\end{verbatim}



% =========================================================================
% Chapter 13 : High-Dimensional Projection Search and the Map of Validity
% =========================================================================

\section{Chapter 13: The Map of Validity and Defect Potential}
\label{sec:ch13}

\subsection*{13.0. Overview and Motivation}
Part~I established the \emph{Universal Control Contract (UCC)} as a rigorous
auditor: given an input \(F\) and collapse threshold \(\tau\), the system certifies
validity through the inequality
\[
\mathrm{Gap}_\tau > \Sigma\delta.
\]
This certificate is sufficient for verification but inadequate for
\emph{exploration}.  
To study global mathematical families—such as flows of PDEs, arithmetic
families of elliptic curves, or geometric variations—we require
a framework that provides directional information:
a means to navigate the parameter space \(\mathcal{M}\).

This chapter introduces such a mechanism.
We reinterpret the \(\delta\)-ledger as a \textbf{scalar potential}
\(\Phi : \mathcal{M} \to \mathbb{R}_{\ge 0} \cup \{\infty\}\),
providing a navigable landscape over which AI agents—defined
in Chapter~14—may perform gradient descent, local search,
or dimensional lifting.  
Paired with a definable partition of \(\mathcal{M}\) into \emph{Terrain Cells},
the framework evolves from a passive auditor into an active navigation system.

% -------------------------------------------------------------------------
\subsection{13.1. From Collapse Diagnosis to Navigation}
Let \(\mathcal{M}\) denote the moduli space of admissible inputs.
We assume throughout that \(\mathcal{M}\) admits a stratification by definable
sets (Appendix~Q).

\begin{definition}[Navigation Mode]
In \emph{Navigation Mode}, the AK pipeline does not reject an input
\(x \in \mathcal{M}\) when
\(\Sigma\delta(x) \ge \mathrm{gap}_\tau\).
Instead, it returns both the \textbf{magnitude} of the defect and,
when available, an approximate gradient direction
indicating how to reduce the defect within \(\mathcal{M}\).
\end{definition}

Thus the goal is no longer merely to certify a point but to
identify the region:
\[
Z_{\mathrm{Valid}} \ := \ \Phi^{-1}([0,\, \mathrm{gap}_\tau]).
\]

% -------------------------------------------------------------------------
\subsection{13.2. The Defect Potential \texorpdfstring{\(\Phi(x)\)}{Phi(x)}}
The \(\delta\)-ledger encodes algebraic, numerical, and functorial deviations
from ideal collapse. To combine these into a single invariant, we apply
a monotone scalarization.

\begin{definition}[Scalarization]
Let \(V\) be the Quantale of δ-budgets.
A map
\(\|\cdot\|_V : V \to \mathbb{R}_{\ge 0} \cup \{\infty\}\)
is a \emph{scalarization} if:
\begin{enumerate}
    \item \(\|0_V\|_V = 0\);
    \item \(\delta_1 \preceq \delta_2 \Rightarrow \|\delta_1\|_V \le \|\delta_2\|_V\);
    \item \(\|\delta_{\mathrm{alg}} \oplus \delta_{\mathrm{disc}}\|_V
            \ge \|\delta_{\mathrm{alg}}\|_V\)
          (algebraic defects contribute persistently).
\end{enumerate}
Typical examples include \(\ell^1\) or \(\ell^\infty\) norms when
\(V=[0,\infty]^k\).
\end{definition}

\begin{definition}[Defect Potential \(\Phi_\tau\)]
Let \(x\in \mathcal{M}\), and let \(\Sigma\delta(x)\) denote the δ-budget
\emph{after} applying collapse \(T_\tau\).
Let \((\mu(x),\nu(x))\) be the tower obstruction indices (Chapter~4).
The \emph{Defect Potential} is:
\[
\Phi_\tau(x)
\ :=\
\big\|\Sigma\delta(x)\big\|_V
\;+\;
\lambda_{\mathrm{sing}}\, \mathcal{I}_{\mathrm{IV}}(x),
\]
where \(\lambda_{\mathrm{sing}}\gg 1\) and
\[
\mathcal{I}_{\mathrm{IV}}(x)
=
\begin{cases}
1 & \text{if } (\mu(x),\nu(x))\neq(0,0),\\
0 & \text{otherwise}.
\end{cases}
\]
\end{definition}

\begin{remark}[Geometric Stratification and Lifting Trigger]
The potential \(\Phi_\tau\) stratifies the parameter space
\(\mathcal{M}\) into three operational regimes, each prescribing a distinct
AI-agent behavior:

\begin{itemize}
    \item \textbf{Plain of Truth}
    (\(\Phi(x)<\mathrm{gap}_\tau\)).  
    Collapse is certified by UCC.
    The agent \emph{records the region as valid} and explores boundaries.

    \item \textbf{Ridge of Noise}
    (\(\mathrm{gap}_\tau \le \Phi(x) < \lambda_{\mathrm{sing}}\)).  
    Only Types~I--III defects occur.
    The agent performs \textbf{local gradient descent} (or random-walk refinement)
    within the Terrain Cell to search for a descending path.

    \item \textbf{Peak of Singularity}
    (\(\Phi(x)\ge \lambda_{\mathrm{sing}}\)).  
    Indicates essential Type~IV obstruction.
    The agent must trigger a \textbf{Dimensional Lifting} request (Chapter~14),
    adding auxiliary axes to escape the singular fiber.
\end{itemize}

This stratification gives \(\Phi\) an operational semantics and links it directly
to the autonomous behaviors of Chapter~14.
\end{remark}

% -------------------------------------------------------------------------
\subsection{13.3. Terrain Cells and Definable Geometry}
Optimization on \(\mathcal{M}\) requires discretization compatible
with definability and uniformity.

\begin{definition}[Terrain Cell]
A \emph{Terrain Cell} \(W_\alpha\subset\mathcal{M}\) is a definable,
bounded subset satisfying:
\begin{enumerate}
    \item \textbf{Uniform Constructibility:}
    The persistence diagrams \(\mathbf{P}_i(F_x)\) admit uniform
    constructible description for all \(x\in W_\alpha\).

    \item \textbf{Lipschitz Budget:}
    The assignment \(x\mapsto \Sigma\delta(x)\) is Lipschitz on \(W_\alpha\).

    \item \textbf{MECE Partition:}
    The family \(\{W_\alpha\}\) is mutually exclusive and collectively exhaustive.
\end{enumerate}
\end{definition}

\begin{proposition}[Local Convexity Proxy]
Away from Type~IV regions, the potential \(\Phi_\tau(x)\) on a sufficiently
small Terrain Cell is well-approximated by a convex function.
Thus gradient-based search (Hunter Protocol) is admissible inside cells.
\end{proposition}

% -------------------------------------------------------------------------
\subsection{13.4. Structural Regularity Theorem (Conditional)}
We now state the global condition under which the potential landscape
certifies global regularity of the underlying structure.

\begin{theorem}[AK Structural Regularity]\label{thm:ak-reg}
Let \(\mathcal{M}\) be a path-connected definable component.
Assume:
\begin{enumerate}
    \item \textbf{Bounded Potential:}
    \(\sup_{x\in\mathcal{M}} \Phi_\tau(x) < \mathrm{gap}_\tau\).

    \item \textbf{No Essential Obstruction:}
    The Type~IV set
    \(Z_{\mathrm{sing}} := \{x : (\mu(x),\nu(x))\neq(0,0)\}\)
    is empty.

    \item \textbf{Definable Coverage:}
    \(\mathcal{M}\) is covered by countably many Terrain Cells satisfying
    the Summability condition (Appendix~J).
\end{enumerate}
Then the family \(F\) collapses globally to the trivial object in
\(\mathsf{Pers}^{\mathrm{cons}}/\mathsf{E}_\tau\).
In particular:
\begin{itemize}
    \item For PDE families (e.g.\ NSE), no finite-time blow-up occurs.
    \item For arithmetic families (e.g.\ elliptic curves),
          analytic and algebraic ranks coincide.
\end{itemize}
\end{theorem}

\begin{proof}
Bounded potential ensures uniform passage of B-Gate\(^+\).
Absence of Type~IV eliminates essential obstructions.
Definable coverage allows gluing of local collapse certificates
via the Restart Lemma.
Thus collapse is globally coherent.
\end{proof}

% -------------------------------------------------------------------------
\subsection*{13.5. Summary}
This chapter transforms AK-HDPST from a passive diagnostic mechanism
into a geometric navigation framework.
The Defect Potential \(\Phi\) supplies a scalar field governing movement,
and Terrain Cells provide local uniformity for optimization.
Chapter~14 introduces the autonomous agents—Hunter, Mapper, Lifter—
that operate over this landscape.



% =========================================================================
% Chapter 14 : AI Agents — Hunter, Mapper, Lifter
% =========================================================================
% AK-HDPST v17.0 — Fully Revised & Corrected Version (ChatGPT 5.1)
% =========================================================================

\section{Chapter 14: AI Agents — Hunter, Mapper, and Lifter}
\label{sec:ch14}

\subsection*{14.0. Overview and Agent Taxonomy}
The scalar field \(\Phi\) (Chapter~13) transforms collapse diagnostics into
a navigable landscape on the parameter space \(\mathcal{M}\).
To explore this landscape efficiently and safely, we introduce three
autonomous but contract-bounded agents:

\begin{itemize}
    \item \textbf{Hunter} — performs local optimization of \(\Phi\) within
          Terrain Cells, discovering valid regions and local minima.
    \item \textbf{Mapper} — assembles validated Terrain Cells into a coherent
          global structure using Overlap Gates.
    \item \textbf{Lifter} — resolves essential singularities
          (Type~IV points) through controlled dimensional extension.
\end{itemize}

All agents operate under the \emph{Universal Control Contract (UCC)}:
they may propose actions, but acceptance and certification is performed
only by the AK Core.

% -------------------------------------------------------------------------
\subsection{14.1. The Hunter: Regime-Aware Search Strategy}

Navigation Mode (Definition~13.1) assigns each point \(x\in\mathcal{M}\)
to one of the three regimes determined by \(\Phi(x)\):
\[
\text{Plain} \quad (\Phi < \mathrm{gap}_\tau), \qquad
\text{Ridge} \quad (\mathrm{gap}_\tau \le \Phi < \lambda_{\mathrm{sing}}), \qquad
\text{Peak} \quad (\Phi \ge \lambda_{\mathrm{sing}}).
\]

The Hunter realizes these semantics operationally.

\begin{definition}[Hunter State]
A Hunter maintains a state
\[
S_k = (x_k,\; W_k,\; \Phi(x_k),\; \nabla\Phi(x_k)),
\]
where \(W_k\) is the enclosing Terrain Cell.
\end{definition}

\begin{specification}[Hunter Protocol]
At iteration \(k\), the Hunter acts according to the regime at \(x_k\):

\begin{enumerate}

    \item \textbf{Plain Regime (\(\Phi(x_k)\le \mathrm{gap}_\tau\)): Verification.}  
    Trigger the full B-Gate\(^+\) check.  
    If verified, mark \(W_k\) as \texttt{valid} and initiate exploration of
    its boundary \(\partial W_k\).

    \item \textbf{Ridge Regime (\(\mathrm{gap}_\tau < \Phi(x_k) < \lambda_{\mathrm{sing}}\)): Descent.}  
    Compute a descent direction using adjoint differentiation or finite differences:
    \[
        x_{k+1} := x_k - \alpha_k \nabla\Phi(x_k).
    \]
    If \(\nabla\Phi(x_k)\approx 0\), apply a controlled random perturbation.

    \item \textbf{Peak Regime (\(\Phi(x_k)\ge \lambda_{\mathrm{sing}}\)): Escalation.}  
    Terminate local search and invoke the Lifter to escape essential obstructions.

\end{enumerate}

Every action is recorded in the \emph{Hunter Action Log} (Appendix~U)
to guarantee reproducibility.
\end{specification}

% -------------------------------------------------------------------------
\subsection{14.2. The Mapper: Global Assembly of Certificates}

The Mapper ensures that local certificates produced by Hunters
combine into a globally coherent proof artifact.

\begin{definition}[Coverage Graph]
Let \(\mathcal{G}=(\mathcal{V},\mathcal{E})\) be a graph whose vertices
are Terrain Cells marked \texttt{valid}.
An edge \((\alpha,\beta)\in\mathcal{E}\) exists when the Overlap Gate
(Chapter~5) passes on the intersection \(W_\alpha\cap W_\beta\).
\end{definition}

\begin{specification}[Mapper Protocol]
The Mapper performs the following loop:

\begin{enumerate}
    \item \textbf{Ingest Validated Cells.}
          Receive validated \(W_\alpha\) from Hunters.

    \item \textbf{Execute Overlap Gates.}
          Verify consistency on each intersection \(W_\alpha\cap W_\beta\).

    \item \textbf{Update Coverage Graph.}
          Add edges or merge components accordingly.

    \item \textbf{Check Global Coverage.}
          In bounded domains: check whether a connected component
          covers the target domain.  
          In unbounded domains: verify asymptotic stability of certificates.
\end{enumerate}

When a connected component covers the domain of interest,
the Mapper issues a \textbf{Global Certificate}.
\end{specification}

% -------------------------------------------------------------------------
\subsection{14.3. The Lifter: Controlled Dimensional Extension}

The Lifter responds exclusively to essential singularities
(Type~IV points: \(\mu,\nu\neq 0\)).
Unlike the Hunter, its role is not optimization but \emph{geometric escape}.

\begin{definition}[Dimensional Lifting]
If \(\mathcal{M}_n\) is the base parameter space, a \emph{lifting}
is an embedding
\[
\iota: \mathcal{M}_n \hookrightarrow \mathcal{M}_{n+k}
\]
adding \(k\) auxiliary coordinates (e.g.\ smoothing width, spectral softness,
auxiliary weights, or arithmetic depth).
The object \(F\) is pulled back to a lifted object \(\tilde{F}\).
\end{definition}

\begin{specification}[Lifter Protocol]
Given a singular point \(x_{\mathrm{sing}}\):

\begin{enumerate}
    \item \textbf{Select Axis Type.}
          Choose an auxiliary axis appropriate to the obstruction
          (e.g.\ spectral, geometric, arithmetic).

    \item \textbf{Construct Lifted Neighborhood.}
          Build a definable neighborhood \(\tilde{W}\subset\mathcal{M}_{n+1}\)
          around \((x_{\mathrm{sing}},0)\).

    \item \textbf{Spawn New Hunter.}
          Launch a new Hunter on \(\tilde{W}\) to seek descent directions
          in the enlarged space.
\end{enumerate}

All lifting steps are subject to the Lifting Penalty (Section~14.4).
\end{specification}

% -------------------------------------------------------------------------
\subsection{14.4. Safety: The Lifting Penalty \texorpdfstring{\(\delta^{\mathrm{lift}}\)}{delta-lift}}

To prevent trivial “solve-by-infinite-lifting”, dimensional lifting incurs
a cost charged against the global δ-budget.

\begin{definition}[Lifting Penalty]
For a lifting depth \(k\), the penalty is a monotone function
\[
\delta^{\mathrm{lift}}(k)
\quad\in V,
\]
commonly exponential, e.g.
\[
\delta^{\mathrm{lift}}(k) = C_{\mathrm{lift}}\, 2^k.
\]
\end{definition}

\begin{definition}[Augmented Gap Constraint]
Any lifted search path must satisfy:
\[
\Sigma\delta(x) \oplus \delta^{\mathrm{lift}}(k(x))
\ <\ \mathrm{gap}_\tau.
\]
This ensures:
\begin{itemize}
    \item the lifting depth is strictly bounded;
    \item infinite regress is automatically disallowed;
    \item the system favors minimal-dimensional resolutions.
\end{itemize}
\end{definition}


\begin{remark}[Operational Meaning]
The Lifter is not a universal escape mechanism.  
It is a \emph{high-cost extension operator} whose use must be justified
by essential obstructions.  
In practice, only a small number of lifts are admissible under the
UCC budget.
\end{remark}

% -------------------------------------------------------------------------
\subsection*{14.5. Summary}
Hunter, Mapper, and Lifter jointly implement the active exploration
engine of AK-HDPST.
Hunter minimizes the Defect Potential \(\Phi\),
Mapper assembles local certificates into a global proof,
and Lifter resolves essential singularities via controlled dimensional extension.
Together they provide a sound, reproducible, and bounded
search mechanism for the validity landscape described in Chapter~13.



% =========================================================================
% Chapter 15 : Collapse-Based Optimization Protocols
% =========================================================================
% AK-HDPST v17.0 — Revised Version (ChatGPT 5.1 + Gemini)
% =========================================================================

\section{Chapter 15: Collapse-Based Optimization Protocols}
\label{sec:ch15}

\subsection*{15.0. Overview and Motivation}
Chapter~14 defined the agents (Hunter, Mapper, Lifter) as the actors
of the navigation system.
This chapter specifies the \emph{optimization scripts} they follow.

Unlike classical optimization where the cost function is smooth and
explicit, the Defect Potential \(\Phi_\tau(x)\) is derived from
persistence diagnostics and quantale-valued error ledgers.
It can be non-convex, piecewise-defined, and expensive to evaluate.
We therefore require robust protocols combining:
\begin{itemize}
    \item \textbf{Local Descent:} Exploit approximate gradients where
          \(\Phi\) is locally regular (Ridge regime).
    \item \textbf{Restart Logic:} Use the Restart Lemma (Appendix~J)
          to escape shallow local minima caused by window scale.
    \item \textbf{Lifting Heuristics:} Recognize when failure is due
          to essential obstructions rather than numerical noise.
\end{itemize}

% -------------------------------------------------------------------------
\subsection{15.1. Search Strategies and Restart Logic}
Within a Terrain Cell \(W\), the primary operation of a Hunter is to
minimize \(\Phi\) under the Ridge regime
(\(\mathrm{gap}_\tau < \Phi < \lambda_{\mathrm{sing}}\)).

\begin{definition}[Collapse-Based Gradient Descent (CBGD)]
Let \(x_k \in \mathcal{M}\) be the current point.
The \emph{CBGD update rule} is
\[
x_{k+1}
\ =\
x_k - \alpha_k \cdot
\frac{\nabla_{\!\delta}\Phi(x_k)}{\|\nabla_{\!\delta}\Phi(x_k)\|},
\]
where \(\nabla_{\!\delta}\Phi(x_k)\) is a discrete gradient obtained by
evaluating \(\Phi\) at a finite stencil of neighbors in \(W\).
The step size \(\alpha_k\) is adaptive:
if \(\Phi\) oscillates or increases on CBGD steps,
\(\alpha_k\) is reduced.
\end{definition}

\begin{specification}[Restart Logic via Convergence Manager]
To cope with non-convexity and window artifacts, the CBGD loop is
augmented with restart logic based on the Restart Lemma (Appendix~J).

\begin{enumerate}
    \item \textbf{Descent Check.}
          If \(\Phi(x_{k+1}) < \Phi(x_k)\), accept the update.

    \item \textbf{Stagnation.}
          If \(\Phi(x_k) > \mathrm{gap}_\tau\) and
          \(\Phi(x_{k+\ell}) \approx \Phi(x_k)\) for
          \(1\le \ell \le N\),
          declare \emph{stagnation} and perform a
          \textbf{Window Restart}:
          \begin{itemize}
              \item subdivide \(W\) into smaller Terrain Cells \(W'\);
              \item recompute local thresholds \(\mathrm{gap}_\tau(W')\);
              \item if \(\Phi(x) < \mathrm{gap}_\tau(W')\) in some \(W'\),
                    reclassify that region as valid and hand off to Mapper.
          \end{itemize}

    \item \textbf{Local Minimum.}
          If stagnation persists and no refinement \(W'\) yields
          \(\Phi < \mathrm{gap}_\tau(W')\),
          classify \(x_k\) as a \emph{local minimum} of \(\Phi\)
          within the current resolution.
\end{enumerate}
\end{specification}

\begin{remark}[Quantale Awareness]
CBGD minimizes the scalarized quantity \(\|\Sigma\delta\|_V\), but the
full quantale-valued vector \(\boldsymbol{\delta}\) is always retained
in the Hunter Action Log.
This allows post hoc analysis of which component (algebraic vs.\ numerical,
commutation vs.\ truncation) dominates the defect at a local minimum.
\end{remark}

% -------------------------------------------------------------------------
\subsection{15.2. Handling Local Minima via Lifting Heuristics}
When a Hunter is trapped at a local minimum \(x^*\) satisfying
\(\Phi(x^*) \ge \mathrm{gap}_\tau\), the system must decide whether
this reflects:
\begin{itemize}
    \item a genuine candidate counterexample, or
    \item a projection artifact caused by insufficient dimension.
\end{itemize}

\begin{definition}[Lifting Condition]
A local minimum \(x^*\) is a candidate for Dimensional Lifting if:
\begin{enumerate}
    \item \textbf{Persistent Obstruction:}
          \(\Phi(x^*)\) remains above \(\mathrm{gap}_\tau\) under
          a \(\tau\)-sweep (varying collapse scale within admissible bounds).

    \item \textbf{Type IV Signature:}
          The tower diagnostics at \(x^*\) satisfy
          \((\mu(x^*),\nu(x^*)) \neq (0,0)\).

    \item \textbf{Cost Feasibility:}
          The base Lifting Penalty obeys
          \(\delta^{\mathrm{lift}}(1) < \mathrm{gap}_\tau\),
          so that a single lift is admissible under the UCC.
\end{enumerate}
\end{definition}

\begin{specification}[Lifting Heuristic Protocol]
Given a local minimum \(x^*\) satisfying the Lifting Condition:

\begin{enumerate}
    \item \textbf{Freeze Local Search.}
          Suspend further CBGD updates at \(x^*\).

    \item \textbf{Propose Auxiliary Axis.}
          The Lifter proposes an axis \(\mathcal{A}\) in accordance
          with Appendix~U (e.g.\ smoothing parameter, spectral softening,
          arithmetic depth).

    \item \textbf{Test Lift Gradient.}
          Evaluate the one-sided directional derivative (or finite difference)
          of \(\Phi\) along \(\mathcal{A}\) at \((x^*,0)\) in the lifted
          space \(\mathcal{M}\times \mathcal{A}\).

    \item \textbf{Commit or Reject Lift.}
          \begin{itemize}
              \item If a direction with \(\partial_{\mathcal{A}}\Phi < 0\)
                    exists and the augmented budget
                    \(\Sigma\delta \oplus \delta^{\mathrm{lift}}(1)
                    < \mathrm{gap}_\tau\) holds, commit the lift and
                    spawn a new Hunter on \(\mathcal{M}\times\mathcal{A}\).
              \item If all tested axes yield non-decreasing \(\Phi\),
                    label \(x^*\) as a \textbf{Terminal Singularity}
                    (candidate counterexample) and halt lifting.
          \end{itemize}
\end{enumerate}
\end{specification}

\subsection*{15.3. Summary}
This chapter specifies the optimization core of the AK-HDPST search engine:

\begin{itemize}
    \item \textbf{CBGD} provides a regime-aware gradient descent adapted
          to the structure of \(\Phi\).
    \item \textbf{Restart Logic} uses window refinement to distinguish
          spurious local minima from resolution artifacts.
    \item \textbf{Lifting Heuristics} decide when and how to escape
          topological traps by controlled dimensional extension, subject
          to the Lifting Penalty of Chapter~14.
\end{itemize}

Together with Chapters~13 and~14, these protocols endow the Hunter and
Lifter with a mathematically disciplined behavior over the Defect
Potential landscape.



% =========================================================================
% Chapter 16 : Bridge Programs and Spectral-Gap Windows
% =========================================================================

\section{Chapter 16: Bridge Programs and Spectral-Gap Windows}
\label{sec:ch16}

\subsection*{16.0. Overview and Motivation}
Chapter~3 established a \emph{one-way bridge}:
\[
\mathrm{PH}_1 = 0 \;\Longrightarrow\; \mathrm{Ext}^1 = 0
\]
under appropriate amplitude and finiteness hypotheses.
This direction is logically robust and is sufficient for filtering
collapse failures in the Core (Part~I).

However, global regularity programs (Navier--Stokes, BSD, etc.) often
require turning categorical information into topological conclusions.
Operationally, we would like---under strict safety conditions---to treat
\(\mathrm{Ext}^1 \approx 0\) as evidence that \(\mathrm{PH}_1 = 0\).

This chapter defines the \textbf{Bridge Programs} that govern such
\emph{reverse inferences} at the Search Layer (Part~II).
They do not alter the Core theorems of Part~I; instead, they specify
when the search engine is allowed to \emph{treat} Ext-vanishing as a
proxy for PH-vanishing.
The central safeguard is a \textbf{Spectral-Gap Condition} ensuring that
“zero” is spectrally isolated from numerical noise.

% -------------------------------------------------------------------------
\subsection{16.1. Program B1: The Ext \texorpdfstring{\(\to\)}{->} PH Reverse Problem (Spec)}

\begin{definition}[The Reverse Problem]
Let \(W\) be a Terrain Cell and \(\tau>0\) a fixed collapse scale.
We ask for conditions under which the following \emph{operational}
implication is sound:
\[
\mathrm{Ext}^1\big(\mathcal{R}(C_\tau F|_W), k\big) = 0
\quad \Longrightarrow \quad
\mathrm{PH}_1(C_\tau F|_W) = 0.
\]
In finite-dimensional exact arithmetic this is often true, but in
practice numerical errors, near-zero eigenvalues, or limiting phenomena
(Type~IV precursors) can make \(\mathrm{Ext}^1\approx 0\) ambiguous.
\end{definition}

\begin{specification}[Bridge Program B1: Collapse-Consistent Conditions]
We allow the search engine to \emph{accept} the implication
\(\mathrm{Ext}^1\Rightarrow\mathrm{PH}_1\) on a Terrain Cell \(W\) only
if the following \emph{Collapse-Consistent Conditions (CCC)} hold:

\begin{enumerate}
    \item \textbf{UCC Compliance.}
          The cell \(W\) is definable and all computations are
          performed after collapse (\(T_\tau\)).

    \item \textbf{Tower Stability.}
          The tower diagnostics vanish on \(W\):
          \((\mu_{\mathrm{Collapse}},\nu_{\mathrm{Collapse}})=(0,0)\).

    \item \textbf{Potential Bound.}
          The Defect Potential satisfies
          \(\Phi_\tau(x) < \mathrm{gap}_\tau\) for all \(x\in W\).

    \item \textbf{Spectral Safety.}
          The Spectral-Gap Condition of Section~16.2 holds on \(W\).
\end{enumerate}

If CCC is satisfied, the agent may issue a \textbf{Reverse Certificate}
on \(W\), recording that Ext-vanishing is treated as PH-vanishing
at the Search Layer.
\end{specification}

\begin{remark}[Position in the Theory]
Bridge Program~B1 lives entirely in Part~II (Search Layer).
It does \emph{not} assert a new abstract equivalence
\(\mathrm{Ext}^1 \Leftrightarrow \mathrm{PH}_1\) in the Core category.
Instead, it specifies when the computational pipeline is allowed to
propagate “Ext~\(\approx 0\)” as “PH~\(\approx 0\)” in the construction
of the Map of Validity.
\end{remark}

% -------------------------------------------------------------------------
\subsection{16.2. Safety: Spectral-Gap Condition for Reverse Implication}

To prevent the “v14.5 fallacy” (confusing a small but non-zero
eigenvalue with true zero), we require a quantitative separation between
true topological signal and the noise floor induced by \(\delta\)-budgets.

\begin{definition}[Spectral Gap \(\gamma_\tau(x)\)]
Let \(L_\tau(x)\) be the normalized combinatorial Laplacian associated
to the degree-1 chain complex of \(C_\tau F_x\).
The \emph{spectral gap} at \(x\) is
\[
\gamma_\tau(x)
\ :=\ \min\{\lambda>0 \mid \lambda\in\sigma(L_\tau(x))\},
\]
with the convention \(\gamma_\tau(x)=+\infty\) if the spectrum has no
positive part.
\end{definition}

\begin{definition}[Spectral-Gap Condition]
A Terrain Cell \(W\) satisfies the \emph{Spectral-Gap Condition} if
there exists a constant \(c>1\) such that
\[
\inf_{x\in W} \gamma_\tau(x)
\ >\
c \cdot
\sup_{x\in W} \big\|\Sigma\delta(x)\big\|_V.
\]
\end{definition}

\begin{theorem}[Spectral Robustness of Vanishing (Bridge Level)]
\label{thm:spectral-robustness}
Assume CCC and the Spectral-Gap Condition on a Terrain Cell \(W\) with
\(c\ge 2\), and assume the numerical backend respects these spectral
bounds.
Then any computation that returns
\(\mathrm{Ext}^1(\mathcal{R}(C_\tau F_x),k)\approx 0\) for all
\(x\in W\), with residual below
\(\sup_{x\in W}\|\Sigma\delta(x)\|_V\), may be \emph{accepted} as a
certificate \(\mathrm{Ext}^1=0\) on \(W\).
Combining this with the one-way bridge of Chapter~3, the search engine
is justified in treating \(\mathrm{PH}_1=0\) on \(W\) within the
Map-of-Validity construction.
\end{theorem}

\begin{remark}[Interpretation]
This theorem should be read as a \emph{design guarantee} for the
Bridge Program: under a provable spectral separation and UCC-compliant
numerics, the risk of misclassifying a nonzero Ext as zero is controlled
by the \(\delta\)-budget.
The Core theory remains agnostic about any unconditional
Ext/PH equivalence; all reverse uses are explicitly gated by CCC.
\end{remark}

% -------------------------------------------------------------------------
\subsection{16.3. Programs B2 and B3: Global Regularity vs.\ Counterexamples}

Programs B2 and B3 describe how local Bridge conclusions aggregate to
global statements about a conjecture.

\begin{specification}[Program B2: Global Regularity via Coverage]
\textbf{Input:} A parameter space \(\mathcal{M}\) and an AK realization \(F\).

\textbf{Goal:} Show that \(F\) collapses globally to the trivial object
after \(T_\tau\).

\textbf{Procedure:}
\begin{enumerate}
    \item \textbf{Decomposition.}
          Hunters decompose \(\mathcal{M}\) into Terrain Cells \(\{W_\alpha\}\).

    \item \textbf{Local Bridge.}
          On each \(W_\alpha\), execute Program~B1.
          Cells where CCC fails remain undecided or are flagged for lifting.

    \item \textbf{Singularity Handling.}
          For Type~IV regions (peaks of \(\Phi\)), invoke the Lifter
          (Chapter~14) to attempt a resolution in an extended parameter space.
          If lifting fails within the Lifting Penalty, mark the region as a
          \emph{Barrier}.

    \item \textbf{Gluing.}
          Use the Mapper and Overlap Gates (Chapter~5) to assemble all
          validated cells into a connected Coverage Graph.

    \item \textbf{Conclusion.}
          If the union of validated cells covers the target domain in
          \(\mathcal{M}\), and no unresolved Barriers remain, the system
          issues a \textbf{Global Regularity Certificate}.
\end{enumerate}
\end{specification}

\begin{specification}[Program B3: The Counterexample Hunt]
\textbf{Goal:} Search for robust singularities that resist all admissible
repairs and lifts.

\textbf{Procedure:}
\begin{enumerate}
    \item \textbf{Maximization.}
          Hunters seek points \(x\) where \(\Phi_\tau(x)\) is large
          (Peak regime).

    \item \textbf{Lifting Attempts.}
          At each Peak, invoke the Lifter to explore dimensional extensions
          within the Lifting Penalty Policy.

    \item \textbf{Persistence.}
          If a Peak persists across all admissible lifts and
          \(\Phi_\tau(x)\) remains above \(\lambda_{\mathrm{sing}}\),
          record \(x\) as a \textbf{Certified Counterexample Candidate}.
          Such candidates are then subject to independent mathematical
          analysis outside the automated pipeline.
\end{enumerate}
\end{specification}

\subsection*{16.4. Summary}
Bridge Programs B1--B3 elevate the informal heuristic
“Ext behaves like PH” into a carefully gated engineering protocol.

\begin{itemize}
    \item B1 specifies when a local Ext-vanishing computation is allowed
          to propagate as PH-vanishing, guarded by the Spectral-Gap
          Condition and CCC.

    \item B2 lifts local bridge results to a global regularity statement,
          using the Map-of-Validity architecture.

    \item B3 formalizes the dual program: hunting for Peaks of Singularity
          that cannot be resolved under any admissible lift, producing
          counterexample candidates.
\end{itemize}

These programs complete the logical interface between the AK Core
(Part~I) and the AI-driven Search Layer (Part~II).



% =========================================================================
% Chapter 17 : The AK-HDPST AI Platform
% =========================================================================
% AK-HDPST v17.0 — Revised Version (ChatGPT 5.1 + Gemini)
% =========================================================================

\section{Chapter 17: The AK-HDPST AI Platform}
\label{sec:ch17}

\subsection*{17.0. Overview: From Theory to Execution}
The preceding chapters established the mathematical rules (Core, Part~I)
and the exploration strategies (Search Layer, Part~II).
This final chapter specifies the \emph{execution environment}—the
AK-HDPST AI Platform.

In this framework, a “proof” is no longer a static text but a
\emph{reproducible computational process}.
To preserve mathematical rigor, we enforce a strict separation of roles:
\begin{itemize}
    \item \textbf{AI Agents (Proposers):}
          Untrusted agents (Hunters/Lifters) that suggest parameters,
          paths, and lifts to minimize the Defect Potential \(\Phi\).
    \item \textbf{AK Core (Verifier):}
          A trusted, deterministic kernel implementing Part~I and the
          UCC logic, which audits every proposal.
\end{itemize}
This chapter defines the data structures and protocols that bind these
two components, and under which the resulting \emph{Map of Validity}
constitutes a verifiable certificate.

% -------------------------------------------------------------------------
\subsection{17.1. \texttt{run.yaml} as a Formal Proof Object}

\begin{definition}[Proof Object]
A \emph{Proof Object} in the AK-HDPST framework is a cryptographically
hashed manifest (e.g.\ a file \texttt{run.yaml}) that records:
\begin{enumerate}
    \item \textbf{Axioms:}
          The definition of the parameter space \(\mathcal{M}\), the
          target conjecture, and the quantale \(V\) governing \(\delta\).

    \item \textbf{Inference Rules:}
          The versions (hashes) of the Collapse Functor \(T_\tau\),
          Realization \(\mathcal{R}\), and Bridge Programs (Chapter~16)
          used in the run.

    \item \textbf{Trace:}
          A list of Terrain Cells \(\{W_\alpha\}\) covering the domain
          of interest in \(\mathcal{M}\), together with their local
          certificates:
          gap checks, tower diagnostics \((\mu,\nu)\), and any reverse
          certificates (B1).

    \item \textbf{Signatures:}
          Hashes of intermediate artifacts (barcodes, spectra, logs) to
          guarantee immutability and allow independent re-execution.
\end{enumerate}
\end{definition}

\begin{specification}[Validity of a Proof Object]
A manifest \texttt{run.yaml} is \emph{valid} if and only if:
\begin{itemize}
    \item The Terrain Cells it references form a MECE cover of the
          declared domain in \(\mathcal{M}\).
    \item Every cell labeled \texttt{passed: true} has been certified
          by the AK Core under UCC (gap check, tower diagnostics,
          Spectral-Gap Condition when B1 is used).
    \item The global aggregation of \(\delta\)-budgets satisfies the
          Summability condition (Appendix~J), so that no hidden defect
          accumulates at infinity.
\end{itemize}
Verification of a Proof Object reduces to re-executing the manifest
against the Core; this is deterministic given the recorded hashes.
\end{specification}

% -------------------------------------------------------------------------
\subsection{17.2. Reproducibility: Hunter Action Log Schema}

To make the stochastic “discovery phase” scientifically meaningful,
we require that AI-driven exploration be \emph{replayable}.

\begin{definition}[Hunter Action Log]
A \emph{Hunter Action Log} is a sequential record
\(\mathcal{L} = (S_0,A_0,S_1,A_1,\dots)\) where:
\begin{itemize}
    \item \(S_k\) is the state at step \(k\)
          (e.g.\ current parameter \(x_k\), enclosing cell \(W_k\),
          potential \(\Phi(x_k)\)).
    \item \(A_k\) is the action taken
          (e.g.\ \texttt{gradient\_step}, \texttt{restart},
          \texttt{lift\_proposal}).
\end{itemize}
\end{definition}

\begin{specification}[Log Schema Requirements]
A valid Hunter Log must contain at least:
\begin{enumerate}
    \item \textbf{Initialization Data:}
          random seed(s), initial coordinates \(x_0\), choice of Terrain
          Cell \(W_0\), and hyperparameters
          (\(\alpha\), \(\lambda_{\mathrm{sing}}\), step limits).

    \item \textbf{Decision Rationale:}
          for each \(A_k\), the relevant numerical data used
          (approximate gradient \(\nabla\Phi\), local value of \(\Phi\),
          regime classification: Plain/Ridge/Peak).

    \item \textbf{Lifting Trace:}
          whenever a lifting step is committed, the type of auxiliary
          axis, the charged Lifting Penalty \(\delta^{\mathrm{lift}}\),
          and the new lifted coordinates.
\end{enumerate}

\textbf{Replayability Requirement:}
A third party who is given the same initial seed, the same Core
implementation, and the same \texttt{run.yaml} must be able to replay
\(\mathcal{L}\) and recover the same set of validated Terrain Cells
(up to machine-level reproducibility).
\end{specification}

% -------------------------------------------------------------------------
\subsection{17.3. Architecture: Proposer–Verifier Separation}

The AK-HDPST AI Platform adheres to a strict Proposer–Verifier model.

\begin{itemize}
    \item \textbf{Proposer:}
          AI agents (Hunters, Lifters) propose moves in parameter space:
          points \(x\), collapse scales \(\tau\), candidate lifts.

    \item \textbf{Verifier:}
          The AK Core evaluates proposals, computes \(\Phi\),
          updates the \(\delta\)-ledger, checks gates (B-Gate\(^+\),
          Overlap Gates, CCC), and either accepts or rejects the
          proposed certificate.

    \item \textbf{Proof Store:}
          Accepted certificates and coverage data are written into the
          Map of Validity and referenced in \texttt{run.yaml}.
\end{itemize}

\begin{remark}[White-Box Principle]
The platform enforces the following principles:
\begin{enumerate}
    \item \textbf{AI Visibility Only:}
          AI agents \emph{see} the values of \(\Phi\) and diagnostics,
          but cannot alter the Core algorithms or thresholds.

    \item \textbf{Core Sovereignty:}
          The Core has absolute veto power: if
          \(\Sigma\delta > \mathrm{gap}_\tau\) or CCC fails, the
          certificate is rejected irrespective of the AI’s “confidence.”

    \item \textbf{No Black-Box Proofs:}
          A conjecture is never accepted because
          “the AI says so.”
          It is accepted only because every step in
          \texttt{run.yaml} and every log entry passes Core verification
          under the UCC.
\end{enumerate}
\end{remark}

% -------------------------------------------------------------------------
\subsection{17.4. Final Conclusion}

AK-HDPST v17.0 completes the transition from a static theoretical
framework to a \emph{computational proof engine}.
By combining:
\begin{enumerate}
    \item the rigorous \textbf{Collapse Core} (Part~I),
    \item the geometric \textbf{Search Layer} (Part~II),
    \item and the verifiable \textbf{AI Platform} (Chapter~17),
\end{enumerate}
the theory supports large-scale, auditable exploration of parameter
spaces attached to deep conjectures (Navier–Stokes, BSD, RH, etc.).

In this setting, a “solution” is not merely a single linear argument,
but a \textbf{Map of Validity}: a high-dimensional, reproducible object
constructed by AI agents and certified by algebraic and categorical
checks within the AK Core.



% =========================
\section*{Notation and Conventions (reinforced v16.5)}
% =========================
\phantomsection
\addcontentsline{toc}{section}{Notation and Conventions}

\paragraph{Base field and ambient categories.}
Fix a coefficient field \(k\) (Appendices~N/O may instead use a field \(\Lambda\); when used, replace \(k\) by \(\Lambda\) everywhere).
Let \(\mathsf{Vect}_k\) be the abelian category of finite\hyp dimensional \(k\)\hyp vector spaces and write \([\mathbb{R},\mathsf{Vect}_k]\) for functors \((\mathbb{R},\le)\to \mathsf{Vect}_k\).

\paragraph{Constructible persistence and standing identification.}
We write
\[
\Perskft\ \subset\ [\mathbb{R},\mathsf{Vect}_k]
\]
for the full subcategory of \emph{constructible} persistence modules (pointwise finite\hyp dimensional with locally finite critical set on bounded windows).
Throughout the paper we \emph{identify} the “finite\hyp type’’ category with this constructible subcategory and use the symbol \(\Perskft\) uniformly.

\paragraph{Filtered objects and persistence.}
\(\mathsf{FiltCh}(k)\) denotes filtered chain complexes of finite\hyp dimensional \(k\)\hyp spaces; filtered quasi\hyp isomorphism is abbreviated f.q.i.
For \(i\in\mathbb{Z}\) the degree–\(i\) persistence functor is
\[
\mathbf{P}_i:\ \mathsf{FiltCh}(k)\longrightarrow \Perskft,\qquad \mathbf{P}_i(F)(t)=\mathrm{H}_i(F^t).
\]
Realizations from other formalisms into \(\mathsf{FiltCh}(k)\) are denoted \(\mathcal{R}(-)\), \(\mathcal{F}(-)\) as appropriate.

\paragraph{Quantale enrichment and definable windows (UCC layer).}
Unless stated otherwise, the measurement layer is enriched over a fixed \emph{commutative Quantale} \(V\) (write \((V,\oplus,\le,0)\); examples: \(([0,\infty],+,\le,0)\), \((\max,+)\), or product quantales used for error/credence).
Distances are Lawvere \(V\)\hyp valued and aggregates are taken with the single operation \(\oplus\) (``Quantale\hyp sum'').
Window sets are \emph{right\hyp open}, \emph{MECE} along a \(\tau\)\hyp sweep, and—when required—\emph{Denef--Pas definable}; definability guarantees finite event sets and finite Čech depth on bounded windows (Appendix~Q; cf.\ Appendices~H/J for the real o\hyp minimal surrogate \(\mathbb{R}_{\mathrm{an},\exp}\)).
The triple \((V,\text{definable windows},\text{AWFS})\) is referred to as the \emph{Universal Control Contract (UCC)}.

\paragraph{Reflection/truncation vs.\ window clipping.}
For \(\tau\ge 0\), let \(\mathsf{E}_\tau\subset\Perskft\) be the Serre subcategory generated by bars of length \(\le \tau\).
The \emph{reflector} (truncation)
\[
\mathbf{T}_\tau:\ \Perskft\longrightarrow \mathsf{E}_\tau^\perp
\]
is exact, idempotent, and left adjoint to the inclusion \(\iota_\tau:\mathsf{E}_\tau^\perp\hookrightarrow\Perskft\) (Appendix~A, Theorem~\ref{A:thm:localization}); it is \(1\)\hyp Lipschitz for interleavings and, more generally, \(V\)\hyp \(1\)\hyp Lipschitz (Appendix~A, Proposition~\ref{A:prop:lipschitz}).
On filtered complexes we use a collapser \(C_\tau\) with a natural (up to f.q.i.) identification
\[
\mathbf{P}_i(C_\tau F)\ \cong\ \mathbf{T}_\tau(\mathbf{P}_iF)\qquad\text{(natural in \(F,i\)).}
\]
For \(\sigma\ge 0\), the \emph{window clip}
\[
\mathbf{W}_{\le\sigma}=(i_{\le\sigma})^0_!\circ i_{\le\sigma}^\ast:\ \Perskft\to\Perskft
\]
restricts to \([0,\sigma]\) and extends by zero; it is \(1\)\hyp Lipschitz (Appendix~I).
\emph{Warning.} \(\mathbf{T}_\tau\) (delete bars of length \(\le\tau\)) and \(\mathbf{W}_{\le\sigma}\) (clip to a window) play distinct roles and must not be conflated.

\paragraph{Optional low\hyp pass (safe regime; after\hyp collapse).}
An optional operator \(L_\tau\) (even kernel, mass \(1\), bandwidth \(\asymp\sqrt{\tau}\)) may be applied \emph{after} \(\mathbf{T}_\tau\); it \emph{soft\hyp commutes} with \(\mathbf{T}_\tau\) up to \(\delta^{\mathrm{alg}}\) and keeps \(\mathbf{T}_\tau\circ L_\tau\) \(1\)\hyp Lipschitz for \(d_{\mathrm{int}}\) (Ch.~2; Appendix~E). Non\hyp compliant filters are not used for certification.

\paragraph{AWFS viewpoint and \(2\)\hyp cells.}
We use an algebraic weak factorization system on \(\Ho(\mathsf{FiltCh}(k))\),
\[
\mathrm{Id}\ \Rightarrow\ L\ \dashv\ R\ \Rightarrow\ \mathrm{Id},\qquad R=C_\tau,
\]
to organize preprocessing/realization; resulting \(2\)\hyp cells are accounted for as algorithmic defects \(\delta_{\mathrm{alg}}\) in the \(\delta\)\hyp ledger (Ch.~5; Appendices~K/L).

\paragraph{Interleaving metric and shifts.}
On \(\Perskft\) the interleaving metric \(d_{\mathrm{int}}\) equals the bottleneck distance in the constructible \(1\)D setting.
The time shift is \((S^\varepsilon M)(t):=M(t+\varepsilon)\); shifts commute canonically with \(\mathbf{T}_\tau\), hence \(\mathbf{T}_\tau\) is \(1\)\hyp Lipschitz (and \(V\)\hyp \(1\)\hyp Lipschitz when the measurement layer is \(V\)\hyp enriched).

\paragraph{Unique comparison order (after\hyp collapse policy).}
All comparisons (local, overlap, global) follow the unique order
\[
\boxed{\ \text{for each } t\ \Longrightarrow\ \mathbf{P}_i\ \Longrightarrow\ \mathbf{T}_\tau\ \Longrightarrow\ \text{compare in }\Perskft\ }.
\]
This order is mandatory for audits, spectral alignment, and overlap gluing (Ch.~11, \S11.0 bis; Ch.~5).

\paragraph{Barcodes, events, and endpoint convention.}
Barcodes use half\hyp open intervals \(I=[b,d)\) with \(d\in\mathbb{R}\cup\{\infty\}\) and multiplicity \(m(I)\in\mathbb{Z}_{\ge1}\).
Any consistent open/closed choice yields the same clipped lengths and event sets.
For \(\tau\ge 0\) the clipped length is
\[
\ell_{[0,\tau]}(I):=\max\{0,\min\{d,\tau\}-\max\{b,0\}\}.
\]
Given \(\tau_0>0\), the finite event set in degree \(i\) is
\[
\mathsf{Ev}_i(F;\tau_0)=\{0,\tau_0\}\ \cup\ \bigl(\{b\in[0,\tau_0]\}\cap\mathrm{births}\bigr)\ \cup\ \bigl(\{d\in[0,\tau_0]\}\cap\mathrm{deaths}\bigr),
\]
with endpoint conventions and infinite bars as in Appendix~A, Remark~\ref{A:rk:endpoints}.

\paragraph{Betti curves and Betti integral.}
\(\beta_i(F;t):=\dim_k \mathrm{H}_i(F^t)\) is c\`adl\`ag and piecewise constant on bounded windows.
The (clipped) Betti integral is
\[
\mathrm{PE}_i^{\le\tau}(F)=\int_0^\tau \beta_i(F;t)\,dt\ =\ \sum_{I\in\mathcal{B}_i(F)} m(I)\,\ell_{[0,\tau]}(I)\qquad\text{(Appendix~H).}
\]

\paragraph{Length spectrum operator.}
For \(M\in\Perskft\) with barcode \(M\simeq\bigoplus_j I[b_j,d_j)\) and a right\hyp open window \(W=[u,u')\), the \emph{length spectrum} operator \(\Len(M;W)\) (Ch.~2) is diagonal on a bar\hyp basis with eigenvalues \(\ell_W(I[b_j,d_j))\).
Its multiset of eigenvalues equals the multiset of clipped bar lengths and is invariant under isomorphisms \(M\simeq M'\).
In particular, for \(\mathbf{T}_\tau\mathbf{P}_i(F)\) and \(W=[0,\tau]\),
\[
\mathrm{PE}_i^{\le \tau}(F)\ =\ \|\Len(\mathbf{T}_\tau\mathbf{P}_i(F);[0,\tau])\|_{L^1}\,,
\]
hence the total collapse energy is an isomorphism invariant of the truncated persistence (Ch.~11, \S11.0+).

\paragraph{Spectral policy, ordering, and matrix norms (after\hyp collapse only).}
Spectral indicators are computed on \(L(C_\tau F)\) (normalized combinatorial Hodge Laplacian), not on pre\hyp collapse objects.
Positive eigenvalues are reported in \emph{ascending} order.
The matrix norm used for tolerances is declared as \(\|\cdot\|_{\mathrm{op}}\) (operator) or \(\|\cdot\|_{\mathrm{fro}}\) (Frobenius) and recorded in the run manifest.
Optional spectral bounds (\(\lambda_{\min},\lambda_{\max}\)) and a Lipschitz tolerance \texttt{lip\_tol} may be specified (Ch.~11, \S11.2; Appendix~G).

\paragraph{Towers and diagnostics; stability bands.}
A \emph{tower} is a directed system \(F=(F_n)_{n\in I}\) with colimit \(F_\infty\).
For \(i\in\mathbb{Z}\), \(\tau\ge 0\), the comparison map is
\[
\phi_{i,\tau}(F):\ \varinjlim_{n}\ \mathbf{T}_\tau\!\big(\mathbf{P}_i(F_n)\big)\longrightarrow \mathbf{T}_\tau\!\big(\mathbf{P}_i(F_\infty)\big).
\]
Set \(\mu_{i,\tau}(F):=\gdim\ker\phi_{i,\tau}(F)\) and \(\nu_{i,\tau}(F):=\gdim\operatorname{coker}\phi_{i,\tau}(F)\); the totals are
\[
\muc(F):=\sum_{i}\mu_{i,\tau}(F),\qquad \nuc(F):=\sum_{i}\nu_{i,\tau}(F).
\]
Cofinal restriction leaves \((\muc,\nuc)\) unchanged; finite direct sums add; composition is subadditive and the rules extend to \(V\)\hyp distances (Appendix~J).
If \(\phi_{i,\tau}\) is an isomorphism then \((\muc,\nuc)=(0,0)\).
A \emph{stability band} is a contiguous \(\tau\)\hyp range where \(\phi_{i,\tau}\) is an isomorphism; sufficient conditions include (S1)~\(\mathbf{T}_\tau\)\hyp colimit commutation at the apex, (S2)~no near\hyp \(\tau\) accumulation from below, (S3)~\(\mathbf{T}_\tau\)\hyp Cauchy with compatible cocone (Appendices~D/J).

\paragraph{Gate cascade (sequent calculus with cut elimination).}
The default cascade is
\[
E_1{=}0\ \Longrightarrow\ (\mu,\nu){=}(0,0)\ \Longrightarrow\ \Ext^1{=}0\ \Longrightarrow\ \mathrm{PH}_1{=}0,
\]
operated as a sequent calculus with cut elimination (Ch.~1/3/5).
Success at a later stage never overturns failure at an earlier stage.

\paragraph{Overlap Gate (local\texorpdfstring{$\to$}{->}global gluing).}
Given a windowed cover \(\{(X_\alpha,W_\alpha)\}\) by right\hyp open \emph{definable} windows, the \emph{Overlap Gate} (Ch.~1; Ch.~5) requires, after truncation by \(\mathbf{T}_\tau\),
\begin{itemize}
  \item local post\hyp collapse equivalence on overlaps up to the recorded \(\delta\)\hyp budget (quantitative commutation);
  \item Čech–\(\Ext^1\) acyclicity in degree \(1\) (finite Čech depth holds by definability);
  \item stability\hyp band condition \((\mu,\nu)=(0,0)\) and near\hyp \(\tau\) non\hyp accumulation.
\end{itemize}
When all overlaps pass, local truncated objects glue (uniquely up to isomorphism) in \(\Perskft\).
Run manifests record \texttt{overlap\_checks}: \texttt{local\_equiv} (true/false), \texttt{cech\_ext1\_ok}, \texttt{stability\_band\_ok}.

\paragraph{A/B soft\hyp commuting; Mirror/Transfer; pipeline budget; B\hyp Gate\(^{+}\).}
For exact reflectors \(T_A,T_B\) in \(\Perskft\), the commutation defect is
\[
\Delta_{\mathrm{comm}}(M;A,B)=d_{\mathrm{int}}(T_AT_BM,\ T_BT_AM).
\]
Given tolerance \(\eta\), if \(\Delta_{\mathrm{comm}}\le\eta\) we accept \emph{soft\hyp commuting}; else fix an order (e.g.\ \(T_B\circ T_A\)) and \emph{record} \(\Delta_{\mathrm{comm}}\) as \(\delta^{\mathrm{alg}}\) (Appendices~K/L).
For Mirror/Transfer functors \(\Mirror\), assume a natural \(2\)\hyp cell \(\Mirror\circ C_\tau\Rightarrow C_\tau\circ\Mirror\) with a uniform bound \(\delta(i,\tau)\ge 0\) in \(d_{\mathrm{int}}\), additive along pipelines and \emph{non\hyp increasing} under \(1\)\hyp Lipschitz post\hyp processing (Appendix~L).
On a window \(W\) and degree \(i\), the pipeline budget aggregates (in the Quantale)
\[
\Sigma\delta(i,\tau)\ =\ \sum_{\text{Mirror--Collapse}}\delta(i,\tau)\ +\ \sum_{\text{A/B fails}}\Delta_{\mathrm{comm}}\ +\ \sum_{\text{audits}}\bigl(\delta^{\mathrm{disc}}+\delta^{\mathrm{meas}}\bigr).
\]
The \emph{safety margin} \(\mathrm{gap}_\tau>0\) is configured per window and degree; \(\mathrm{B\text{--}Gate}^{+}\) requires \(\mathrm{gap}_\tau>\Sigma\delta(i,\tau)\).
Across windows, Restart and Summability hold if there exists \(\kappa\in(0,1]\) with
\[
\mathrm{gap}_{\tau_{k+1}}\ \ge\ \kappa\bigl(\mathrm{gap}_{\tau_k}-\Sigma\delta_k(i)\bigr),\qquad \sum_k \Sigma\delta_k(i)<\infty
\]
(Appendix~J).

\paragraph{Arithmetic alignment conventions (Iwasawa Gate).}
When applicable, the Iwasawa diagnostic \(\mu_{\mathrm{Iwasawa}}\) is compared with the collapse diagnostic \(\mu_{\mathrm{Collapse}}\) in a three\hyp state regime (lower bound / match / drift\hyp corrected) used by the Gate cascade to suppress Type\hyp IV drift (Ch.~7; Appendix~R). Finite kernel/cokernel effects are absorbed into \(\delta_{\mathrm{alg}}\).

\paragraph{Categorical check and window\hyp local trigger.}
With \(\Qtest=\{k[0]\}\), we test \(\Ext^1(\mathcal{R}(C_\tau F),Q)=0\) \emph{after truncation}.
The one\hyp way bridge \(\mathrm{PH}_1\Rightarrow \Ext^1\) is used only under (B1)–(B3) (field coefficients; constructible range; \(t\)\hyp exact realization of amplitude \(\le 1\)).
On a definable right\hyp open window \(W\) of finite Čech depth we adopt the \emph{window\hyp local trigger} via the first energy page \(E_1\):
\[
E_1(W)=0\ \Longleftrightarrow\ \mathrm{PH}_1\!\big(C_\tau F\!\mid_W\big)=0\ \Longleftrightarrow\ \Ext^1\!\big(\mathcal{R}(C_\tau F)\!\mid_W,k\big)=0,
\]
while globally only the one\hyp way implication is claimed.

\paragraph{Reproducibility (manifest) and mandatory fields.}
The run manifest \texttt{run.yaml} (v17) records: \(\tau\)\hyp sweep and windows (MECE \& coverage checks); spectral policy \((\beta,M(\tau),t)\) with \emph{mandatory} fields \texttt{order="ascending"} and \texttt{norm="op"|"fro"}; optional \(\lambda_{\min},\lambda_{\max}\) and \texttt{lip\_tol}; aux\hyp bar bins \(([a,b],\beta)\) with right\hyp open convention and under/overflow; A/B test parameter \(\eta\), policy, fallback order; tower terminal symbol and cone extension; \(\delta\)\hyp ledger per step and \texttt{sum\_delta}; safety margin and \texttt{gap\_tau}; overlap checks (\texttt{local\_equiv}, \texttt{cech\_ext1\_ok}, \texttt{stability\_band\_ok}); persistence summary (PH/Ext/\(\mu,\nu\), tail isomorphism).
UCC fields are \texttt{quantale} (name/op/unit/order), \texttt{definable} (o\hyp minimal structure and window formulae; Denef--Pas preferred), and \texttt{awfs/2cell} (enabled, bounds).
All persistence\hyp layer quantities are computed after applying \(\mathbf{T}_\tau\) (equivalently on \(C_\tau F\)); spectral indicators are normalized (ascending eigenvalues; declared norm); categorical checks follow the sanctioned one\hyp way direction.

\paragraph{Global guard\hyp rails and non\hyp claims.}
All statements are confined to the persistence/spectral/categorical layers in the implementable range.
No number\hyp theoretic identity, analytic regularity theorem, or group trivialization is asserted.
No claim of \(\mathrm{PH}_1\Leftrightarrow \Ext^1\) is made; only the one\hyp way implication \(\mathrm{PH}_1\Rightarrow \Ext^1\) (under (B1)–(B3)) is used.
The collapse obstruction \(\muc\) is a persistence\hyp level diagnostic and is distinct from the classical Iwasawa \(\mu\).

\medskip
\noindent\emph{Abbreviations.}
f.q.i.\ = filtered quasi\hyp isomorphism;\quad
c\`adl\`ag = right\hyp continuous with left limits;\quad
UCC = Universal Control Contract;\quad
AWFS = algebraic weak factorization system;\quad
“window” \(=\) interval \([0,\tau]\) with \(\tau\ge0\);\quad
DP = Denef--Pas definable.




% =========================
\appendix
\section*{Appendix A. Constructible Persistence: Abelianity, Serre Localization, and the $V$-Nucleus View (reinforced)}
% =========================

\addcontentsline{toc}{section}{Appendix A. Constructible Persistence: Abelianity, Serre Localization, and the $V$-Nucleus View (reinforced)}

\begingroup
Throughout this appendix, fix a field \(k\).
Write \(\Pers_k\) for the category of right-continuous persistence modules
\(M:(\mathbb{R},\le)\to\Vect_k\) with structure maps \(M(t\le t')\).
We denote by \(\Pers^{\mathrm{ft}}_k\subset\Pers_k\) the \emph{constructible} (finite-type) subcategory used in the main text.

\medskip
\noindent\textbf{Global conventions.}
(i) All \(\Ext\)-tests are taken against \(Q=k[0]\) (the unit interval module supported at a point).
(ii) Windowed energies use an exponent \(\alpha>0\) (default \(\alpha=1\)).
(iii) References to appendices use the tilde style (e.g.\ Appendix~D); failure types use the dash style
\(\mathrm{Type\ I\text{--}II}\), \(\mathrm{Type\ III}\), \(\mathrm{Type\ IV}\).
(iv) Notational disambiguation: the reflector (localization) functor is denoted \(\mathbf{T}_\tau\); truncations/clippings on domains are denoted \(\mathbf{Tr}_\tau\) or \(\clip_{[a,b)}\). This resolves the potential collision sometimes found in informal notes where \(\mathbf{T}_\tau\) was used for truncation.
(v) When window partitions are used (MECE; §A.6), \emph{half-open with right-inclusion} is the standing endpoint convention for domain windows and spectral bins; coverage checks are mandatory.
(vi) All results in §§A.2–A.5 are stated and used in the one-parameter (1D), field-coefficient, right-continuous, constructible setting.
(vii) \emph{\(\delta\)-ledger.} Quantitative non-commutation and implementation defects are externalized as
\[
\delta=\delta_{\mathrm{alg}}+\delta_{\mathrm{disc}}+\delta_{\mathrm{meas}}\ \in V,
\]
summed in a fixed commutative quantale \(V\) (see §A.0). Windowed pipelines aggregate \(\delta\)'s additively.

\begin{remark}[Scope / after-collapse alignment]\label{A:rk:scope-after}
Appendix~A asserts equalities and exactness statements purely at the persistence layer \(\Pers^{\mathrm{ft}}_k\).
All \emph{quantitative} comparisons (distances, Lipschitz, monotonicity) that are used operationally in the paper are understood to be evaluated \emph{after} applying the reflector \(\mathbf{T}_\tau\) and on fixed windows; filtered-level claims (chains/complexes) are not made here. This matches the after-collapse policy adopted globally.
\end{remark}

% -------------------------
\subsection*{A.0. $V$-nucleus viewpoint and the $\delta$-interface}
% -------------------------
Let \(V\) be a commutative quantale, and endow \(\Pers^{\mathrm{ft}}_k\) with a Lawvere \(V\)-metric \(d_V\) induced by the interleaving (bottleneck) distance (the canonical choice identifies \(V=[0,\infty]\) with \(+\) and \(\le\)).
\begin{definition}[$V$-nucleus (Lawvere setting)]
A \emph{$V$-nucleus} on a \(V\)-metric category \((\mathcal{C},d_V)\) is an endofunctor \(N:\mathcal{C}\to\mathcal{C}\) together with a natural transformation \(\eta:\mathrm{Id}\Rightarrow N\) such that:
\begin{itemize}
  \item (Idempotence) \(N\circ N\simeq N\).
  \item (Non-expansiveness) \(d_V(NX,NY)\le d_V(X,Y)\) for all \(X,Y\).
  \item (Exactness on the underlying abelian structure, when present) \(N\) preserves finite limits and finite colimits.
\end{itemize}
\end{definition}

\begin{proposition}[$\mathbf{T}_\tau$ is a $V$-nucleus]\label{A:prop:Vnucleus}
In the constructible 1D range, the Serre reflector \(\mathbf{T}_\tau:\Pers^{\mathrm{ft}}_k\to\Pers^{\mathrm{ft}}_{k,\tau\text{-loc}}\) (Theorem~\ref{A:thm:localization}) is a $V$-nucleus:
it is idempotent (Cor.~\ref{A:cor:idemp-cons}), \(1\)-Lipschitz for interleavings (Prop.~\ref{A:prop:lipschitz}), and exact (Theorem~\ref{A:thm:localization}).
\end{proposition}

\begin{remark}[$\delta$-commutation schema]\label{A:rk:delta-schema}
At the categorical level, functors that preserve the \(\tau\)-ephemeral Serre class commute strictly with \(\mathbf{T}_\tau\). Operationally, we record any implementation-level discrepancy by a nonnegative \(\delta\in V\).
For clipping on a right-open window \(W=[u,v)\) we define a \emph{clipping defect} \(\delta_{\mathrm{clip}}(W,\tau)\in V\) with the contract
\[
d_V\bigl(\,\mathbf{T}_\tau\circ\clip_{W}(M),\ \clip_{W}\circ\mathbf{T}_\tau(M)\,\bigr)\ \le\ \delta_{\mathrm{clip}}(W,\tau),
\]
which must be logged and is \emph{subadditive} over pipelines and windows. In the ideal (exact) model, \(\delta_{\mathrm{clip}}(W,\tau)=0\) (Lemma~\ref{A:lem:clip}); any nonzero value reflects discretization/rounding policies and is accounted for in the \(\delta\)-ledger.
\end{remark}

% -------------------------
\subsection*{A.1. Constructible objects}
% -------------------------
\begin{definition}[Constructible / finite-type]\label{A:def:constructible}
A persistence module \(M\in\Pers_k\) is \emph{constructible} (finite-type) if on every bounded interval
\([a,b]\subset\mathbb{R}\) it has a \emph{finite critical set}: there exist
\(a=t_0<t_1<\dots<t_N=b\) such that each structure map
\(M(t\le t')\) is an isomorphism whenever \(t,t'\in (t_j,t_{j+1})\) for some \(j\).
Equivalently, \(M\) is pointwise finite-dimensional and admits a barcode decomposition as a
\emph{locally finite direct sum of interval modules}, i.e.\ only finitely many intervals intersect any bounded window.
We write \(\Pers^{\mathrm{ft}}_k\) for the full subcategory of such modules.
\end{definition}

\begin{remark}
In the 1D, field-coefficient, right-continuous setting, the equivalence above is standard (barcode decomposition). All constructions below (kernels, cokernels, torsion, truncation/clipping) preserve constructibility and are controlled by finitely many events on bounded windows; see the references at the end of this appendix.
\end{remark}

% -------------------------
\subsection*{A.2. Abelianity}
% -------------------------
\begin{proposition}\label{A:prop:abelian}
\(\Pers^{\mathrm{ft}}_k\) is an abelian category.
Moreover, for a morphism \(f:M\to N\) in \(\Pers^{\mathrm{ft}}_k\),
kernels and cokernels are computed pointwise in \(\Vect_k\) and remain constructible.
\end{proposition}

\begin{proof}
Evaluation at each \(t\in\mathbb{R}\) is exact in \(\Vect_k\), hence pointwise kernels and cokernels define functorial sub/quotient persistence modules.
Constructibility is preserved: on any bounded window one refines the break sets of \(M,N\) to a finite set controlling \(\Ker f\) and \(\Coker f\).
Exactness axioms follow objectwise; hence \(\Pers^{\mathrm{ft}}_k\) is abelian with pointwise exactness.
\end{proof}

% -------------------------
\subsection*{A.3. The \texorpdfstring{$\tau$}{tau}-ephemeral Serre subcategory}
% -------------------------
Fix \(\tau>0\).
Let \(I[a,b)\) denote the interval module supported on \([a,b)\) (with the half-open, right-inclusion convention).

\begin{definition}[\(\tau\)-ephemeral subcategory]\label{A:def:Etau}
Let \(\mathsf{E}_\tau\subset\Pers^{\mathrm{ft}}_k\) be the smallest full subcategory
containing all interval modules \(I[a,b)\) with length \(b-a\le \tau\) and closed under subobjects, quotients, and extensions.
We call \(\mathsf{E}_\tau\) the \emph{\(\tau\)-ephemeral} (or \(\tau\)-torsion) subcategory.
\end{definition}

\begin{lemma}\label{A:lem:Serre}
\(\mathsf{E}_\tau\) is a Serre subcategory of \(\Pers^{\mathrm{ft}}_k\), and it is hereditary as a torsion class.
\end{lemma}

\begin{remark}[Endpoint conventions]\label{A:rk:endpoints}
All statements in this appendix are insensitive to the choice of open/closed endpoints on interval modules.
We fix \([a,b)\) for definiteness; changing endpoint conventions does not affect lengths, barcode decompositions, interleaving/bottleneck distances, or any categorical constructions below.
\end{remark}

% -------------------------
\subsection*{A.3.1. The torsion pair and maximal \texorpdfstring{$\tau$}{tau}-ephemeral subobject}
% -------------------------
Define the \emph{\(\tau\)-local (orthogonal)} subcategory
\[
\Pers^{\mathrm{ft}}_{k,\tau\text{-loc}}
\ :=\
\bigl\{\,X\in\Pers^{\mathrm{ft}}_k\ \big|\ \Hom(E,X)=0=\Ext^1(E,X)\ \text{ for all }E\in\mathsf{E}_\tau\,\bigr\}.
\]
Then \((\mathsf{E}_\tau,\ \Pers^{\mathrm{ft}}_{k,\tau\text{-loc}})\) is a torsion pair: for each \(M\) there is a functorial short exact sequence
\[
0\ \longrightarrow\ t_\tau(M)\ \longrightarrow\ M\ \longrightarrow\ f_\tau(M)\ \longrightarrow\ 0
\]
with \(t_\tau(M)\in\mathsf{E}_\tau\) and \(f_\tau(M)\in \Pers^{\mathrm{ft}}_{k,\tau\text{-loc}}\).

% -------------------------
\subsection*{A.4. The reflector \texorpdfstring{$\mathbf{T}_\tau\dashv\iota_\tau$}{T\_\tau ⊣ ι\_\tau}, exactness, and the $V$-nucleus corollary}
% -------------------------
Let \(\iota_\tau:\Pers^{\mathrm{ft}}_{k,\tau\text{-loc}}\hookrightarrow\Pers^{\mathrm{ft}}_k\) be the inclusion.

\begin{theorem}[Exact reflective localization]\label{A:thm:localization}
The Serre quotient functor
\[
\pi_\tau:\ \Pers^{\mathrm{ft}}_k\ \longrightarrow\ \Pers^{\mathrm{ft}}_k/\mathsf{E}_\tau
\]
is exact. In the 1D constructible setting there is a canonical exact equivalence of abelian categories
\[
\Pers^{\mathrm{ft}}_k/\mathsf{E}_\tau \ \simeq\ \Pers^{\mathrm{ft}}_{k,\tau\text{-loc}}.
\]
Composing \(\pi_\tau\) with this equivalence yields a functor
\[
\mathbf{T}_\tau:\ \Pers^{\mathrm{ft}}_k\ \longrightarrow\ \Pers^{\mathrm{ft}}_{k,\tau\text{-loc}}
\]
which is left adjoint to \(\iota_\tau\) and is exact.
\end{theorem}

\begin{corollary}[Idempotence, conservativity, $V$-nucleus]\label{A:cor:idemp-cons}
\(\mathbf{T}_\tau\circ \mathbf{T}_\tau \cong \mathbf{T}_\tau\) and \(\mathbf{T}_\tau|_{\Pers^{\mathrm{ft}}_{k,\tau\text{-loc}}}\cong \mathrm{Id}\).
Together with Proposition~\ref{A:prop:lipschitz}, \(\mathbf{T}_\tau\) is a $V$-nucleus in the sense of Proposition~\ref{A:prop:Vnucleus}.
\end{corollary}

\begin{proposition}[Behavior on barcodes]\label{A:prop:barcode-behavior}
Let \(M\simeq \bigoplus_{j} I[a_j,b_j)\). Then
\[
\mathbf{T}_\tau M\ \simeq\ \bigoplus_{\,b_j-a_j>\tau}\ I[a_j,b_j) \,,\qquad
t_\tau(M)\ \simeq\ \bigoplus_{\,b_j-a_j\le \tau}\ I[a_j,b_j)\,.
\]
\end{proposition}

\begin{remark}[Filtered colimits: functor-category computation and return to constructible]
\label{A:rk:filtered-colimits}
Filtered colimits are computed objectwise in \([\mathbb{R},\Vect_k]\), and \(\mathbf{T}_\tau\) commutes with those colimits there (as a left adjoint).
A filtered colimit of constructible modules may exit \(\Pers^{\mathrm{ft}}_k\).
In applications we either: (i) restrict to towers that remain constructible degreewise; or (ii) compute in \([\mathbb{R},\Vect_k]\), apply \(\mathbf{T}_\tau\), and \emph{verify} return to \(\Pers^{\mathrm{ft}}_k\).
\end{remark}

% -------------------------
\subsection*{A.5. Shift-commutation, monotonicity in \texorpdfstring{$\tau$}{tau}, and 1-Lipschitz continuity}
% -------------------------
For \(\varepsilon\ge 0\), let \(S^\varepsilon\) be the shift \((S^\varepsilon M)(t):=M(t+\varepsilon)\).

\begin{lemma}[Shift commutation]\label{A:lem:shift}
For all \(\varepsilon\ge 0\), \(\mathbf{T}_\tau\circ S^\varepsilon \;\cong\; S^\varepsilon\circ \mathbf{T}_\tau\).
\end{lemma}

\begin{lemma}[Monotonicity in \(\tau\)]\label{A:lem:monotone-tau}
If \(0<\tau\le \tau'\), there is a natural epimorphism \(\mathbf{T}_{\tau'}M\to \mathbf{T}_{\tau}M\), functorial in \(M\).
\end{lemma}

\begin{proposition}[Non-expansiveness (interleaving/bottleneck)]\label{A:prop:lipschitz}
\(\mathbf{T}_\tau\) is \(1\)-Lipschitz for the interleaving (equivalently, bottleneck) distance on \(\Pers^{\mathrm{ft}}_k\).
\end{proposition}

% -------------------------
\subsection*{A.6. Windowing (MECE), coverage checks, and \texorpdfstring{$\tau$}{tau}-adaptation}
% -------------------------
\begin{definition}[MECE domain windowing and coverage]\label{A:def:MECE}
A \emph{domain windowing} is a finite or countable collection of half-open intervals with right-inclusion
\(\{[u_k,u_{k+1})\}_{k\in K}\) such that:
\begin{itemize}
  \item \emph{(Disjointness)} \([u_k,u_{k+1})\cap [u_\ell,u_{\ell+1})=\varnothing\) for \(k\neq \ell\).
  \item \emph{(Contiguity)} \(u_{k+1}=u_k+\len_k\) with \(\len_k>0\).
  \item \emph{(Coverage)} \(\bigsqcup_{k\in K}[u_k,u_{k+1})=[u_0,U)\) for some finite \(U>u_0\).
\end{itemize}
Coverage checks require
\[
\sum_{k\in K}(u_{k+1}-u_k)=U-u_0,\qquad
\#\mathrm{Events}([u_0,U))=\sum_{k\in K}\#\mathrm{Events}([u_k,u_{k+1}))\ (\pm\ \text{rounding}).
\]
\end{definition}

\begin{remark}[Alignment policy]
When persistence and spectral measurements are combined, domain windows \(\{[u_k,u_{k+1})\}\), collapse thresholds \(\tau\), and spectral bins must be \emph{fixed per window} and logged. All B-side measurements are taken \emph{after} applying \(\mathbf{T}_\tau\) and on the same window.
\end{remark}

\begin{definition}[\(\tau\)-adaptation, sweep, and stability bands]\label{A:def:tau-adapt}
A threshold \(\tau\) is \emph{resolution-adapted} if \(\tau=\alpha\cdot \max\{\Delta t,\Delta x\}\) for a fixed \(\alpha>0\).
A \emph{\(\tau\)-sweep} is a discrete set \(\{\tau_\ell\}\) on which diagnostics are evaluated.
A \emph{stability band} is a contiguous range \(B\subset (0,\infty)\) such that chosen natural transformations are isomorphisms for all \(\tau\in B\) (hence the tower diagnostics vanish).
\end{definition}

% -------------------------
\subsection*{A.7. Clipping, strict commutation, and the $\delta$-commutation contract}
% -------------------------
Let \(\clip_{[u,v)}:\Pers^{\mathrm{ft}}_k\to\Pers^{\mathrm{ft}}_k\) denote clipping to \([u,v)\) (half-open, right-inclusion).

\begin{lemma}[Strict commutation in the exact model]\label{A:lem:clip}
\(\mathbf{T}_\tau\) commutes with clipping: \(\mathbf{T}_\tau\circ \clip_{[u,v)} \cong \clip_{[u,v)}\circ \mathbf{T}_\tau\).
\end{lemma}

\begin{proof}
Clipping is exact and preserves interval lengths, hence preserves \(\mathsf{E}_\tau\). It therefore descends to the Serre quotient and commutes with \(\pi_\tau\); transporting across the equivalence in Theorem~\ref{A:thm:localization} gives the claim.
\end{proof}

\begin{proposition}[Operational $\delta$-commutation]\label{A:prop:delta-clip}
In implementations that incur discretization/rounding, there exists a nonnegative defect
\(\delta_{\mathrm{clip}}([u,v),\tau)\in V\) such that for all \(M\in\Pers^{\mathrm{ft}}_k\)
\[
d_V\!\left(\mathbf{T}_\tau\clip_{[u,v)}(M),\ \clip_{[u,v)}\mathbf{T}_\tau(M)\right)\ \le\ \delta_{\mathrm{clip}}([u,v),\tau),
\]
with \(\delta_{\mathrm{clip}}\) \emph{additive} over concatenated windows and \emph{subadditive} along pipelines. In the mathematical model, \(\delta_{\mathrm{clip}}=0\) by Lemma~\ref{A:lem:clip}.
\end{proposition}

% -------------------------
\subsection*{A.8. $V$-enriched metric: benignity and operational role}
% -------------------------
\begin{remark}[$V$-enrichment is benign; \(\mathbf{T}_\tau\) as $V$-nucleus]\label{rem:V-enriched}
Let \(V\) be a commutative quantale. The abelian/Serre-localization statements remain unchanged on the underlying category \(\Pers^{\mathrm{ft}}_k\). The barcode semantics and length calculus (interval Jordan–Hölder and Serre quotient by \(\mathsf{E}_\tau\)) are invariant as \emph{statements in \(\Pers^{\mathrm{ft}}_k\)}. The \(V\)-structure is used only to \emph{measure} distances, sums, and gluing budgets uniformly; it does not alter the algebraic backbone. In particular, \(\mathbf{T}_\tau\) is a $V$-nucleus (Prop.~\ref{A:prop:Vnucleus});
operational non-commutations are summarized by \(\delta\)-contracts (Remark~\ref{A:rk:delta-schema}, Prop.~\ref{A:prop:delta-clip}).
\end{remark}

% -------------------------
\subsection*{A.9. Operational checklist and glued output}
% -------------------------
The reinforcement yields a single, self-contained operational layer compatible with IMRN/AiM standards:
\begin{itemize}
  \item Abelianity and pointwise exactness (Proposition~\ref{A:prop:abelian}).
  \item Hereditary Serre \(\tau\)-ephemeral class \(\mathsf{E}_\tau\) (Lemma~\ref{A:lem:Serre}); exact reflective localization (Theorem~\ref{A:thm:localization}).
  \item \(\mathbf{T}_\tau\) is an exact, idempotent, \(1\)-Lipschitz $V$-nucleus (Corollary~\ref{A:cor:idemp-cons}, Proposition~\ref{A:prop:lipschitz}, Proposition~\ref{A:prop:Vnucleus}).
  \item Strict clipping-commutation in the categorical model (Lemma~\ref{A:lem:clip}); operational $\delta$-commutation contract for audits (Proposition~\ref{A:prop:delta-clip}).
  \item Shift-commutation and monotonicity in \(\tau\) (Lemmas~\ref{A:lem:shift}, \ref{A:lem:monotone-tau}); barcode-level description (Proposition~\ref{A:prop:barcode-behavior}).
  \item MECE windowing and coverage checks (Definition~\ref{A:def:MECE}); \(\tau\)-adaptation and stability bands (Definition~\ref{A:def:tau-adapt}).
  \item Filtered-colimit policy with return-to-constructible verification (Remark~\ref{A:rk:filtered-colimits}).
\end{itemize}
\emph{Output:} the globally glued object obtained from the windowed pipeline (MECE partition, localization, spectral auxiliaries where used, Restart/Summability in the main text) passes the acceptance gate under coverage/adaptation policies; all non-commutations are explicitly accounted for in the \(\delta\)-ledger via the $V$-nucleus contracts above.

\medskip
\noindent\emph{References for Appendix A.}
Crawley–Boevey (2015): Decomposition of pointwise finite-dimensional persistence modules. IMRN. 
Chazal–de~Silva–Glisse–Oudot (2016): The Structure and Stability of Persistence Modules. 
Gabriel (1962): Des catégories abéliennes. 
Popescu: Abelian Categories. 
Stacks Project, Tag~02MO.
\endgroup


% =========================
\appendix
\section*{Appendix B. Lifting \texorpdfstring{$\mathbf{T}_\tau$}{T\_\tau} to \texorpdfstring{$C_\tau$}{C\_\tau} and the Homotopy Setting (reinforced)}
% =========================

\addcontentsline{toc}{section}{Appendix B. Lifting $T_\tau$ to $C_\tau$ and the Homotopy Setting (reinforced)}

\begingroup

Throughout, fix a field \(k\).
Let \(\mathsf{FiltCh}(k)\) denote the category of \emph{bounded-in-degree} filtered chain complexes of finite-dimensional \(k\)-vector spaces with filtration-preserving chain maps.
(“Bounded” refers to homological degree; filtrations are assumed \emph{locally finite on bounded windows} as in Appendix~A.)
For each homological degree \(i\), write
\[
\mathbf{P}_i:\ \mathsf{FiltCh}(k)\longrightarrow \Pers^{\mathrm{ft}}_k,\qquad
F\longmapsto \big(t\mapsto H_i(F^{t}C_\bullet)\big),
\]
the degreewise persistence functor into the constructible subcategory (Appendix~A).

\medskip
\noindent\textbf{Global scope and conventions.}
(i) All claims at the filtered–complex layer hold \emph{up to filtered quasi-isomorphism (f.q.i.)}; all identities at the persistence layer hold \emph{strictly} in \(\Pers^{\mathrm{ft}}_k\).
(ii) Filtered (co)limits, when invoked, are computed objectwise in \([\mathbb{R},\Vect_k]\), and we then \emph{verify} that the result lies in (or returns to) \(\Pers^{\mathrm{ft}}_k\) (Appendix~A, Remark~\ref{A:rk:filtered-colimits}); no claim is made outside this regime.
(iii) Deletion-type updates are non-increasing for windowed energies and spectral tails \emph{after truncation}, whereas inclusion-type updates are only stable (non-expansive) (Appendix~E).
(iv) Endpoint conventions follow Appendix~A (Remark~\ref{A:rk:endpoints}); in particular, infinite bars are not removed by \(\mathbf{T}_\tau\) and their contributions are clipped by windowing.
(v) For notational economy we sometimes write \(\Ttau=\mathbf{T}_\tau\).
(vi) \emph{Amplitude guard-rail.} Realizations used operationally are taken with \emph{amplitude \(\le 1\) after collapse}: the fixed \(t\)-exact realization \(\mathcal{R}\) has cohomological amplitude contained in \([0,1]\) on the after-collapse objects (\S\ref{B:thm:Ctau}); all constructions respect this guard-rail up to f.q.i.

% -------------------------
\subsection*{B.1. The interval-realization assignment \texorpdfstring{$\mathcal{U}$}{U} (up to f.q.i.)}
% -------------------------
\begin{definition}[Elementary interval blocks (two-term/one-term model)]
Let \(I[a,b)\) be an interval module (fixed endpoint convention; Appendix~A, Remark~\ref{A:rk:endpoints}).
\begin{itemize}
  \item If \(b<+\infty\), realize \(I[a,b)\) in homological degree \(i\) by a \emph{two-term filtered block}
  \[
    k\cdot y \xrightarrow{\,d\,} k\cdot x,\qquad |y|=i+1,\ |x|=i,\quad
       \mathrm{fil}(x)=a,\ \mathrm{fil}(y)=b,\ d(y)=x,\ d(x)=0.
  \]
  Then \(x\) contributes a bar born at \(a\) and killed at \(b\).
  \item If \(b=+\infty\), realize \(I[a,\infty)\) by a \emph{one-term} block \(k\cdot x\) in degree \(i\) with \(\mathrm{fil}(x)=a\) and \(d=0\).
\end{itemize}
In all blocks, the differential preserves the filtration: \(d(F^t)\subseteq F^t\) for every \(t\).
Taking \emph{locally finite on bounded windows} direct sums of such blocks and applying degree shifts produces a filtered complex whose persistence recovers the prescribed bars.
We call any such model an \emph{elementary interval complex} and denote a representative by \(\mathcal{I}[a,b)\).
\end{definition}

\begin{proposition}[Barcode realization for bounded families (up to f.q.i.)]\label{B:prop:U}
There exists an assignment
\[
\mathcal{U}:\ \Pers^{\mathrm{ft}}_k\longrightarrow \mathsf{FiltCh}(k)
\]
such that for any \emph{degree-bounded} family \(\{M_i\}_{i\in\mathbb{Z}}\) of constructible persistence modules
(only finitely many \(i\) nonzero) there are natural isomorphisms in \(\Pers^{\mathrm{ft}}_k\),
\[
\mathbf{P}_i\!\Big(\,\bigoplus_{j}\mathcal{U}(M_j)[-j]\Big)\ \cong\ M_i\qquad(\forall i).
\]
The construction is canonical \emph{up to} filtered quasi-isomorphism, additive, and functorial in the homotopy category \(\Ho(\mathsf{FiltCh}(k))\).
In particular, for a single module \(M\) realized in a base degree (say \(0\)) one has \(\mathbf{P}_0(\mathcal{U}(M))\cong M\) and \(\mathbf{P}_j(\mathcal{U}(M))=0\) for all \(j\neq 0\).
\end{proposition}

\begin{remark}[Pseudofunctoriality of \(\mathcal{U}\)]
The assignment \(\mathcal{U}\) extends to a \emph{pseudofunctor}
\(\mathcal{U}:\Pers^{\mathrm{ft}}_k\to \Ho(\mathsf{FiltCh}(k))\): on a morphism of persistence modules, choose interval decompositions and a bar-matching; the induced blockwise filtered chain map is well-defined in \(\Ho\) \emph{up to} f.q.i., and compositions are respected up to coherent isomorphism.
Consequently, constructions below that use \(\mathcal{U}\) on morphisms (e.g.\ \(C_\tau\)) are functorial on \(\Ho(\mathsf{FiltCh}(k))\).
\end{remark}

% -------------------------
\subsection*{B.2. Filtered quasi-isomorphisms and \texorpdfstring{$\Ho(\mathsf{FiltCh}(k))$}{Ho(FiltCh(k))}}
% -------------------------
\begin{definition}[Filtered quasi-isomorphism]
A filtration-preserving chain map \(f:F\to G\) is a \emph{filtered quasi-isomorphism (f.q.i.)} if for every \(t\in\mathbb{R}\) the map \(F^{t}C_\bullet\to G^{t}C_\bullet\) is a quasi-isomorphism.
Equivalently, for all \(i\), \(\mathbf{P}_i(f)\) is an isomorphism in \(\Pers^{\mathrm{ft}}_k\).
\end{definition}

\begin{lemma}[Characterization of f.q.i.]
For bounded-in-degree filtered complexes of finite-dimensional vector spaces, \(f:F\to G\) is an f.q.i. iff \(\mathbf{P}_i(f)\) is an isomorphism in \(\Pers^{\mathrm{ft}}_k\) for all \(i\).
\end{lemma}

\begin{definition}[Homotopy category]
Let \(\Ho(\mathsf{FiltCh}(k))\) be the localization of \(\mathsf{FiltCh}(k)\) at f.q.i.’s.
Identities stated in \(\Ho(\mathsf{FiltCh}(k))\) are to be understood \emph{up to f.q.i.} at the model level.
All endofunctors considered below (e.g.\ \(C_\tau\) and Mirror/Transfer templates) preserve f.q.i.’s; thus they descend to \(\Ho(\mathsf{FiltCh}(k))\).
\end{definition}

% -------------------------
\subsection*{B.2.1. Amplitude \texorpdfstring{$\le 1$}{≤1}: modeling caution and mock proof}
% -------------------------
\begin{remark}[Modeling caution: amplitude \(\le 1\) after collapse]\label{B:rk:ampl}
The operational guard-rail requires that, after applying \(C_\tau\), the realization \(\mathcal{R}(C_\tau F)\) has cohomology concentrated in degrees \([0,1]\).
The \emph{block-diagonal assembly} (no off-diagonal couplings across distinct bars or non-adjacent degrees) ensures this and is enforced throughout.
\end{remark}

\begin{lemma}[Mock proof of amplitude \(\le 1\)]\label{B:lem:ampl}
Let \(F\in\mathsf{FiltCh}(k)\) and construct \(C_\tau(F)\) by replacing each finite bar by a two-term block \((i{+}1)\!\to\! i\) and each infinite bar by a one-term block in degree \(i\), with zero differentials between distinct blocks and between non-adjacent degrees.
Then the spectral sequence of the stupid filtration on \(\mathcal{R}(C_\tau F)\) degenerates at \(E_1\), and \(H^q(\mathcal{R}(C_\tau F))=0\) for \(q\notin\{0,1\}\).
\end{lemma}

\begin{proof}[Proof sketch]
Each bar contributes either a length-one complex or a single object; hence every differential raises degree by at most one.
No off-diagonal maps exist by construction.
Therefore the only potentially nonzero \(d_r\) occur with \(r\le 1\), so the spectral sequence collapses at \(E_1\), forcing cohomology to lie in degrees \(0,1\).
\end{proof}

% -------------------------
\subsection*{B.2.2. f.q.i. checklist (\texorpdfstring{T-Exactness-Persistence}{T-Exactness-Persistence})}
% -------------------------
\begin{definition}[Test \texttt{T-Exactness-Persistence}]\label{B:def:T-Exactness}
Given \(F\in\mathsf{FiltCh}(k)\), the test consists of the following verifications:
\begin{enumerate}\itemsep0.1em
  \item[(E0)] \emph{Degree bound}: \(F\) is bounded in homological degree.
  \item[(E1)] \emph{Local finiteness}: the filtration is locally finite on bounded windows.
  \item[(E2)] \emph{Barcode audit}: for each \(i\), \(\mathbf{P}_i(F)\) is constructible (Appendix~A).
  \item[(E3)] \emph{Realization amplitude}: after \(C_\tau\), \(\mathcal{R}(C_\tau F)\) has amplitude \(\le 1\) (Lemma~\ref{B:lem:ampl}).
  \item[(E4)] \emph{Exactness match}: for every short exact sequence of filtered complexes \(0\to F'\to F\to F''\to 0\), the induced sequence of persistence modules is exact in \(\Pers^{\mathrm{ft}}_k\).
  \item[(E5)] \emph{Functoriality under f.q.i.}: if \(f:F\to G\) is an f.q.i., then \(\mathbf{P}_i(f)\) is iso for all \(i\).
  \item[(E6)] \emph{Shift/clip compatibility}: \(\mathbf{P}_i\) commutes with shifts and clipping; \(\Ttau\) commutes with clipping (Appendix~A, Lemma~\ref{A:lem:clip}).
  \item[(E7)] \emph{Non-expansiveness}: \(d_{\mathrm{int}}(\Ttau\mathbf{P}_i(F),\Ttau\mathbf{P}_i(G))\le d_{\mathrm{int}}(\mathbf{P}_i(F),\mathbf{P}_i(G))\).
\end{enumerate}
A dataset \((F,\tau)\) \emph{passes} \texttt{T-Exactness-Persistence} if (E0)–(E7) hold.
\end{definition}

\begin{proposition}[Effect of \texttt{T-Exactness-Persistence}]\label{B:prop:T-Exactness}
If \(F\) passes \texttt{T-Exactness-Persistence}, then for all \(i\)
\[
\mathbf{P}_i\!\big(C_\tau(F)\big)\ \cong\ \Ttau\,\mathbf{P}_i(F),
\]
functorially in \(\Ho(\mathsf{FiltCh}(k))\).
Moreover, \(C_\tau\) preserves f.q.i. and is idempotent up to f.q.i.
\end{proposition}

\begin{proof}[Mock proof]
(E2) and (E4) ensure that persistence is computed in the abelian, constructible setting where \(\Ttau\) is exact (Appendix~A, Thm.~\ref{A:thm:localization}); (E6) guarantees compatibility with clipping/shifts; (E7) provides the Lipschitz contract after collapse.
By construction of \(C_\tau\) (Theorem~\ref{B:thm:Ctau}) and (E3), the realization stays within amplitude \(\le 1\), and \(\mathbf{P}_i\) detects f.q.i. (E5). The asserted identity then follows degreewise; idempotence follows from \(\Ttau\circ \Ttau=\Ttau\).
\end{proof}

% -------------------------
\subsection*{B.3. Lifting \texorpdfstring{$\mathbf{T}_\tau$}{T\_\tau} to \texorpdfstring{$C_\tau$}{C\_\tau} and (co)limit/pullback compatibilities}
% -------------------------

\paragraph*{Existence, functoriality, and uniqueness (homotopy-functor level): block-diagonal assembly.}
\begin{theorem}[Thresholded collapse in \(\Ho\)]\label{B:thm:Ctau}
For each \(\tau\ge 0\) there exists an endofunctor
\[
C_\tau:\ \Ho(\mathsf{FiltCh}(k))\longrightarrow \Ho(\mathsf{FiltCh}(k))
\]
and natural isomorphisms in \(\Pers^{\mathrm{ft}}_k\)
\[
\mathbf{P}_i\!\big(C_\tau(F)\big)\ \xrightarrow{\ \cong\ }\ \mathbf{T}_\tau\!\big(\mathbf{P}_i(F)\big)\qquad(\forall i,F),
\]
such that:
\begin{enumerate}\itemsep0.2em
  \item (\emph{Idempotence/monotonicity in \(\Ho\)})
  \(C_\tau\circ C_\sigma \simeq C_{\max\{\tau,\sigma\}}\simeq C_\sigma\circ C_\tau\).
  \item (\emph{Non-expansiveness at persistence})
  \(d_{\mathrm{int}}\!\big(\mathbf{P}_i(C_\tau F),\,\mathbf{P}_i(C_\tau G)\big)\ \le\ d_{\mathrm{int}}\!\big(\mathbf{P}_i(F),\,\mathbf{P}_i(G)\big)\).
\end{enumerate}
Any two such lifts are uniquely isomorphic in \(\Ho(\mathsf{FiltCh}(k))\).
For \(\tau=0\), \(C_0\simeq \mathrm{id}\) in \(\Ho(\mathsf{FiltCh}(k))\).
\end{theorem}

\begin{proof}[Construction/Proof]
Replace \(\mathbf{P}_i(F)\) with \(\Ttau(\mathbf{P}_i(F))\) and realize via \(\mathcal{U}\) (Prop.~\ref{B:prop:U}); assemble differentials \emph{block-diagonally} as in Remark~\ref{B:rk:ampl}.
Functoriality and uniqueness follow from pseudofunctoriality of \(\mathcal{U}\) and the universal property of \(\Ttau\);
non-expansiveness reflects Appendix~A, Prop.~\ref{A:prop:lipschitz}.
\end{proof}

\paragraph*{(Co)limits and pullbacks: persistence layer is strict; filtered layer up to f.q.i.}
\begin{proposition}[Compatibility at the persistence layer]\label{B:prop:limits}
Assume filtered colimits in \(\mathsf{FiltCh}(k)\) are computed degreewise and the results return to \(\Pers^{\mathrm{ft}}_k\).
Then for every filtered diagram \(\{F_\lambda\}\) and every \(i\),
\[
\mathbf{P}_i\!\big(C_\tau(\varinjlim\nolimits_\lambda F_\lambda)\big)\ \cong\ \varinjlim\nolimits_\lambda\, \mathbf{P}_i\!\big(C_\tau(F_\lambda)\big)\quad\text{in } \Pers^{\mathrm{ft}}_k.
\]
If, in addition, \textup{[Spec]} finite pullbacks in \(\mathsf{FiltCh}(k)\) are computed degreewise and \(\mathcal{U}\) preserves finite limits up to f.q.i.\ under \emph{lifting–coherence} \((\LC)\), then for any pullback square \(F\times_H G\),
\[
\mathbf{P}_i\!\big(C_\tau(F\times_H G)\big)\ \cong\ \mathbf{P}_i\!\big(C_\tau(F)\times_{C_\tau(H)} C_\tau(G)\big)\quad\text{in } \Pers^{\mathrm{ft}}_k.
\]
At the filtered level, compatibilities hold \emph{up to f.q.i.}
\end{proposition}

\begin{remark}[On \((\LC)\)]
\((\LC)\) is a \emph{finite-diagram} coherence ensuring that interval realizations can be chosen compatibly (up to f.q.i.) with pullback/pushout shapes encountered in practice. It holds for the block model and finite matching diagrams induced by monotone filtrations; we use it only in \textup{[Spec]} statements.
\end{remark}

\begin{remark}[Realization functor; comparison maps \textup{[Spec]}]
Let \(\mathcal{R}:\mathsf{FiltCh}(k)\to D^{\mathrm{b}}(k\text{-mod})\) be the fixed \(t\)-exact realization.
Within the implementable range there are natural comparison morphisms
\[
\mathcal{R}\circ C_\tau\ \Longrightarrow\ \tau_{\ge 0}\circ \mathcal{R},
\]
compatible with \(\mathbf{P}_i\) after homology (Appendix~C). These maps are treated up to f.q.i.\ in \(\Ho(\mathsf{FiltCh}(k))\).
\end{remark}

% -------------------------
\subsection*{B.4. Non-expansive Mirror/Transfer templates \texorpdfstring{[Spec]}{[Spec]}}
% -------------------------
\begin{definition}[Admissible Mirror/Transfer endofunctors]\label{B:def:mirror}
An endofunctor \(\Mirror:\mathsf{FiltCh}(k)\to \mathsf{FiltCh}(k)\) is \emph{admissible} if:
\begin{enumerate}\itemsep0.2em
  \item (\emph{Persistence non-expansiveness})
  \(d_{\mathrm{int}}\!\big(\mathbf{P}_i(\Mirror F),\,\mathbf{P}_i(\Mirror G)\big)\ \le\ d_{\mathrm{int}}\!\big(\mathbf{P}_i(F),\,\mathbf{P}_i(G)\big)\) for all \(F,G,i\).
  \item (\emph{Constructible stability})
  \(\Mirror\) carries finite-type objects to finite-type objects degreewise.
  \item (\emph{f.q.i.-invariance})
  If \(f\) is an f.q.i., then \(\Mirror(f)\) is an f.q.i.; hence \(\Mirror\) descends to \(\Ho(\mathsf{FiltCh}(k))\).
  \item (\emph{Conditional commutation with \(C_\tau\)})
  There exists a natural 2-cell
  \(\theta:\ \Mirror\circ C_\tau\Rightarrow C_\tau\circ \Mirror\)
  whose effect at persistence is \(\delta\)-controlled:
  \[
  d_{\mathrm{int}}\!\Big(\Ttau\,\mathbf{P}_i(\Mirror(C_\tau F)),\ \Ttau\,\mathbf{P}_i(C_\tau(\Mirror F))\Big)\ \le\ \delta(i,\tau),
  \]
  with \(\delta(i,\tau)\) uniform in \(F\) and additive along pipelines.
\end{enumerate}
\end{definition}

\begin{theorem}[Quantitative commutation in \(\Ho\)]\label{B:thm:quant}
Assume \(\Mirror\) is admissible and \(\theta\) exists with bound \(\delta(i,\tau)\).
Then for all \(F,i,\tau\),
\[
  d_{\mathrm{int}}\!\Big(\Ttau\,\mathbf{P}_i(\Mirror(C_\tau F)),\ \Ttau\,\mathbf{P}_i(C_\tau(\Mirror F))\Big)\ \le\ \delta(i,\tau),
\]
and for a pipeline \(\Mirror_m,\dots,\Mirror_1\) the bounds add:
\[
  d_{\mathrm{int}}(\cdots)\ \le\ \sum_{j=1}^m \delta_j(i,\tau_j).
\]
Any subsequent \(1\)-Lipschitz persistence post-processing does not increase the bound.
\end{theorem}

% -------------------------
\subsection*{B.5. Commutable torsion reflectors and A/B policy (homotopy interface)}
% -------------------------
Let \(T_A,T_B:\Pers^{\mathrm{ft}}_k\to\Pers^{\mathrm{ft}}_k\) be exact reflectors with Serre classes \(E_A,E_B\).

\begin{proposition}[Nested torsions \(\Rightarrow\) order independence]\label{B:prop:nested}
If \(E_A\subseteq E_B\) or \(E_B\subseteq E_A\), then
\(T_A\circ T_B= T_B\circ T_A= T_{A\vee B}\).
In particular, for 1D length thresholds, \(\mathbf{T}_\tau\circ \mathbf{T}_\sigma=\mathbf{T}_{\max\{\tau,\sigma\}}\).
\end{proposition}

\begin{definition}[A/B soft-commuting policy]\label{B:def:ab}
For arbitrary reflectors \(T_A,T_B\) and \(M\in\Pers^{\mathrm{ft}}_k\), set
\(\Delta_{\mathrm{comm}}(M;A,B):=d_{\mathrm{int}}(T_A T_B M,\ T_B T_A M)\).
Given a tolerance \(\eta\ge 0\), accept \emph{soft-commuting} if \(\Delta_{\mathrm{comm}}\le \eta\); otherwise fix an order and log \(\Delta_{\mathrm{comm}}\) as \(\delta^{\mathrm{alg}}\) (Appendix~L).
\end{definition}

% -------------------------
\subsection*{B.6. AWFS on \texorpdfstring{$\Ho(\mathsf{FiltCh})$}{Ho(FiltCh)} and 2-cell accounting}
% -------------------------
\begin{declaration}[AWFS on $\Ho(\mathsf{FiltCh})$]\label{dec:awfs}
We adopt an algebraic weak factorization system on $\Ho(\mathsf{FiltCh})$ with $L$ (preprocess/left map) and $R$ (collapse/right map) such that $R \simeq C_\tau$ up to f.q.i.
Triangle/zigzag identities hold up to f.q.i.; all 2-cell deviations are recorded as $\delta_{\mathrm{alg}}$ (Appendix~L).
\end{declaration}

\begin{theorem}[AWFS triangle 2-cells]\label{thm:awfs-triangle}
There exist coherent 2-cells (quantitatively bounded after collapse)
\[
C_\tau\!\circ C_\tau \simeq C_\tau,\qquad
\mathbf{T}_\tau\!\circ \mathbf{T}_\tau = \mathbf{T}_\tau,\qquad
L\!\circ R \simeq R\!\circ L,
\]
whose non-commutation is absorbed into the $\delta_{\mathrm{alg}}$-budget.
\end{theorem}

\begin{corollary}[A/B policy as 2-cell accounting]\label{B:cor:ab-awfs}
For two exact reflectors on persistence, any measured defect $\Delta_{\mathrm{comm}}$ (Definition~\ref{B:def:ab}) is realizable as a 2-cell deviation within the AWFS picture and must be logged as $\delta_{\mathrm{alg}}$; pipeline additivity follows from the quantale-sum rule in Appendix~L.
\end{corollary}

\begin{remark}[V-shifts and $C_\tau$ \emph{in situ}]
If $S^v$ denotes a Lawvere $V$-shift compatible with degreewise filtrations, then $C_\tau\circ S^v\simeq S^v\circ C_\tau$ in $\Ho$ and, after applying $\mathbf{P}_i$, the induced comparison is $V$-1-Lipschitz (Appendix~A, Lemma~\ref{A:lem:shift}).
\end{remark}

% -------------------------
\subsection*{B.7. Completion note and implementation recipe}
% -------------------------
\begin{remark}[No further supplementation required]
This appendix integrates: (i) the lift \(C_\tau\) of the exact reflector \(\Ttau\) to the homotopy setting (existence, functoriality, uniqueness up to f.q.i.; non-expansiveness at persistence); (ii) strict persistence-layer compatibilities with (co)limits and pullbacks (filtered level up to f.q.i.); (iii) admissible Mirror/Transfer templates with a uniform, additive 2-cell bound \(\delta(i,\tau)\) stable under \(1\)-Lipschitz post-processing; (iv) a commutable-torsion policy (nested \(\Rightarrow\) order-independent; else A/B with ledger); and \emph{(v) the amplitude \(\le 1\) guard-rail and the f.q.i. checklist \texttt{T-Exactness-Persistence}}. No additional supplementation is required for operational use under the global scope of Appendix~A.
\end{remark}

\paragraph{Implementation recipe (engineering checklist).}
\begin{itemize}
  \item Build \(C_\tau\) by block-diagonal assembly of interval blocks; forbid off-diagonal couplings across blocks/degrees (ensures amplitude \(\le 1\)).
  \item For each morphism, lift \(\Ttau\mathbf{P}_i(f)\) blockwise via \(\mathcal{U}\) and take direct sums across \(i\); functorial in \(\Ho\).
  \item Enforce \texttt{T-Exactness-Persistence} (Def.~\ref{B:def:T-Exactness}); expose pass/fail and witnesses in logs.
  \item For Mirror/Transfer, provide a 2-cell \(\theta\) with a \emph{uniform} \(\delta(i,\tau)\); accumulate additively along pipelines; ensure post-processors are \(1\)-Lipschitz at persistence.
  \item For reflectors, run A/B tests; if \(\Delta_{\mathrm{comm}}>\eta\), fix an order and log the surplus as \(\delta^{\mathrm{alg}}\).
  \item Keep clipping/windowing separate from localization; use Appendix~A, Lemma~\ref{A:lem:clip} as needed.
\end{itemize}

\medskip
\noindent\emph{References for Appendix B.}
Crawley–Boevey (2015): Decomposition of pointwise finite-dimensional persistence modules. IMRN. 
Chazal–de~Silva–Glisse–Oudot (2016): The Structure and Stability of Persistence Modules. 
Standard sources on AWFS and homotopical algebra (e.g.\ Riehl, \emph{Categorical Homotopy Theory}) are used at the level of \emph{up to f.q.i.} coherence only.

\endgroup



% =========================
\appendix
\section*{Appendix C. The Bridge \texorpdfstring{$\mathrm{PH}_1 \Rightarrow \Ext^1$}{PH1⇒Ext1} and its Local Reverse under \texorpdfstring{$E_1{=}0$}{E1=0} (reinforced, complete)}
% =========================

\addcontentsline{toc}{section}{Appendix C. The Bridge PH$_1\Rightarrow$Ext$^1$ and Local Reverse under $E_1{=}0$}

Throughout, fix a field \(k\).
Let \(\mathsf{FiltCh}(k)\) be the category of \emph{bounded-in-degree} filtered chain complexes of finite-dimensional \(k\)-vector spaces with filtration-preserving maps.
For \(F\in\mathsf{FiltCh}(k)\) and each degree \(i\), the degreewise persistence functor
\[
\mathbf{P}_i(F):\ \mathbb{R}\longrightarrow \mathsf{Vect}_k,\qquad t\longmapsto H_i(F^{t}C_\bullet)
\]
is assumed \emph{constructible} (pointwise finite-dimensional, with finitely many critical parameters on bounded windows), i.e.\ \(\mathbf{P}_i(F)\in \Pers^{\mathrm{ft}}_k\).
We also fix a \(t\)-exact realization functor
\[
\mathcal{R}:\ \mathsf{FiltCh}(k)\longrightarrow D^{\mathrm{b}}(k\text{-mod})
\]
into the bounded derived category of finite-dimensional \(k\)-vector spaces.

\medskip
\noindent\textbf{Bridge hypotheses, scope, and gate policy (final).}
We work under the standing assumptions \textup{(B1)–(B3)}:
\begin{itemize}
  \item \textup{(B1)} field coefficients and constructibility of \(\mathbf{P}_i(F)\);
  \item \textup{(B2)} two-term amplitude for \(\mathcal{R}\) on operative windows: \(\mathcal{R}(C_\tau F)\in D^{[-1,0]}(k\text{-mod})\);
  \item \textup{(B3)} functoriality/naturality of all constructions.
\end{itemize}
All statements operate under \textup{(B1)–(B3)} and within the implementable range of Appendix~A.
Filtered colimits, when used, are computed in the functor category \([\mathbb{R},\mathsf{Vect}_k]\) and must \emph{return} to \(\Pers^{\mathrm{ft}}_k\) once constructibility is verified (Appendix~A, Remark~\ref{A:rk:filtered-colimits}).

\smallskip
\noindent\textbf{Eligibility for B-Gate\(^{+}\) (fixed).}
The \(\Ext^1\)-test is included in B-Gate\(^{+}\) \emph{only} on windows/scales where \(\mathcal{R}(C_\tau F)\in D^{[-1,0]}(k\text{-mod})\) and the test object is \(k[0]\).
Outside this amplitude regime \(\Ext^1\) may be logged but is \emph{not} used for gating.

\smallskip
\noindent\textbf{Operational order (collapse-first, fixed).}
\[
\textbf{collapse}\ \xrightarrow{\ \ C_\tau\ \ } \ \textbf{realize}\ \xrightarrow{\ \ \mathcal{R}\ \ } \ \textbf{test}\ \xrightarrow{\ \ \Ext^1(-,k)\ \ } 0.
\]

\smallskip
\noindent\textbf{Meaning of \(\mathrm{PH}_1(F)=0\).}
\(\mathrm{PH}_1(F)=0\) means the degree-\(1\) persistence module vanishes (equivalently, \(H_1(F^t)=0\) for all \(t\in\mathbb{R}\)).

\smallskip
\noindent\textbf{Windows and clipping.}
For a window \(W\subset\mathbb{R}\) we write \(F|_W\) for the clipped restriction; clipping is exact and commutes with \(\mathbf{T}_\tau\) (Appendix~A, Lemma~\ref{A:lem:clip}).

% -------------------------
\subsection*{C.1. Two-term amplitude and $t$-exactness}
% -------------------------

\begin{proposition}[Two-term amplitude]\label{C:prop:two-term}
Under \textup{(B2)} there is a natural isomorphism in \(D^{\mathrm{b}}(k\text{-mod})\)
\[
\mathcal{R}(F)\ \simeq\ \Big[\, H^{-1}\!\big(\mathcal{R}(F)\big)\ \xrightarrow{\,d\,}\ H^{0}\!\big(\mathcal{R}(F)\big)\,\Big],
\]
concentrated in cohomological degrees \([-1,0]\).
\end{proposition}

\begin{remark}
All statements below are invariant under filtered quasi-isomorphism on \(F\) and under isomorphism in \(D^{\mathrm{b}}(k\text{-mod})\) on \(\mathcal{R}(F)\).
\end{remark}

% -------------------------
\subsection*{C.2. The edge: \texorpdfstring{$H^{-1}(\mathcal{R}(F))\cong \varinjlim_t H_1(F^t)$}{H^{-1}(R(F)) ≅ colim H_1(F^t)} and naturality}
% -------------------------

\begin{proposition}[Edge identification and naturality]\label{C:prop:edge}
Under \textup{(B2)}, for every \(F\in \mathsf{FiltCh}(k)\) there is a natural isomorphism
\[
H^{-1}\!\big(\mathcal{R}(F)\big)\ \cong\ \varinjlim_{t\in\mathbb{R}}\, H_1(F^{t}C_\bullet).
\]
If \(f:F\to G\) preserves filtrations, the obvious square with the induced maps on the right-hand side commutes.
\end{proposition}

% -------------------------
\subsection*{C.3. Computing \texorpdfstring{$\Ext^1$}{Ext1} for amplitude \([-1,0]\)}
% -------------------------

\begin{lemma}[Edge lemma for \(\Ext^1\)]\label{C:lem:edge-ext}
For \(A\in D^{[-1,0]}(k\text{-mod})\) there is a natural isomorphism
\[
\Ext^1(A,k)\ \cong\ \Hom\!\big(H^{-1}(A),\,k\big).
\]
\end{lemma}

\begin{corollary}[Dimension and duality]\label{C:cor:dim}
For any \(F\) with \(\mathcal{R}(F)\in D^{[-1,0]}\),
\[
\dim_k \Ext^1\!\big(\mathcal{R}(F),k\big)\ =\ \dim_k\, H^{-1}\!\big(\mathcal{R}(F)\big)\ =\ \dim_k\Big(\varinjlim_t H_1(F^t)\Big),
\]
and the canonical pairing \(\Ext^1(\mathcal{R}(F),k)\otimes H^{-1}(\mathcal{R}(F))\to k\) is perfect.
\end{corollary}

% -------------------------
\subsection*{C.4. The forward bridge and its windowed form}
% -------------------------

\begin{theorem}[Bridge \(\mathrm{PH}_1\Rightarrow \Ext^1\)]\label{C:thm:bridge}
Let \(F\in \mathsf{FiltCh}(k)\).
If \(\mathrm{PH}_1(F)=0\) (equivalently, \(H_1(F^t)=0\) for all \(t\)), then
\[
\Ext^1\!\big(\mathcal{R}(F),\,k\big)\ =\ 0.
\]
\end{theorem}

\begin{proof}
If \(\mathrm{PH}_1(F)=0\) then \(\varinjlim_t H_1(F^t)=0\).
By Proposition~\ref{C:prop:edge}, \(H^{-1}(\mathcal{R}(F))=0\).
Apply Lemma~\ref{C:lem:edge-ext}.
\end{proof}

\begin{corollary}[Robust/windowed bridge]\label{C:cor:windowed}
Fix \(\tau>0\).
If \(\mathrm{PH}_1\!\big(C_\tau(F)\big)=0\), then \(\Ext^1\!\big(\mathcal{R}(C_\tau(F)),\,k\big)=0\).
\end{corollary}

% -------------------------
\subsection*{C.5. The local reverse under \texorpdfstring{$E_1{=}0$}{E1=0} (formal statement of P3)}
% -------------------------

\begin{definition}[Tail isomorphism on a right-open window]
Let \(W=(a,b]\subset\mathbb{R}\) be right-open.
We say \(F\) has \emph{tail isomorphism} on \(W\) if for all \(t\le t'\le b\) sufficiently close to \(b\), the structure maps \(H_1(F^{t})\to H_1(F^{t'})\) are isomorphisms (i.e.\ the degree-\(1\) tail stabilizes on \(W\)).
\end{definition}

\begin{lemma}[Edge identification on \(E_1\)-degenerate windows]\label{C:lem:E1-edge}
Let \(W\) be a right-open window with \(E_1(W)=0\) (Chapter~11) and tail isomorphism.
Then, after collapse,
\[
H^{-1}\!\big(\mathcal{R}(C_\tau(F)|_W)\big)\ \cong\ \mathrm{PH}_1\!\big(C_\tau(F)|_W\big).
\]
\end{lemma}

\begin{proof}[Proof sketch]
With \(E_1(W)=0\) the spectral sequence computing the stabilized edge on \(W\) degenerates at \(E_1\), identifying the edge group with the persistent \(H_1\) on \(W\).
Tail isomorphism guarantees that the colimit in Proposition~\ref{C:prop:edge} equals the stabilized value on \(W\).
\end{proof}

\begin{theorem}[Local Reverse under \(E_1{=}0\) (P3)]\label{C:thm:local-reverse}
Let \(W\) be a right-open window with \(E_1(W)=0\) and tail isomorphism, and assume \(\mathcal{R}(C_\tau(F)|_W)\in D^{[-1,0]}\).
If
\[
\Ext^1\!\big(\mathcal{R}(C_\tau(F)|_W),\,k\big)=0,
\]
then
\[
\mathrm{PH}_1\!\big(C_\tau(F)|_W\big)=0.
\]
\end{theorem}

\begin{proof}
By Lemma~\ref{C:lem:E1-edge} we have \(H^{-1}(\mathcal{R}(C_\tau(F)|_W))\cong \mathrm{PH}_1(C_\tau(F)|_W)\).
By Lemma~\ref{C:lem:edge-ext}, \(\Ext^1(-,k)\cong \Hom(H^{-1}(-),k)\) in the amplitude \([-1,0]\) regime.
Hence \(\Ext^1=0\Rightarrow H^{-1}=0\Rightarrow \mathrm{PH}_1=0\).
\end{proof}

\begin{corollary}[Local equivalence on suitable windows]\label{C:cor:definable-local-bridge}
If, in addition, \(W\) is right-open and definable (o-minimal) with finite Čech depth, then for every \(F\) and \(\tau>0\),
\[
\mathrm{PH}_1\!\big(C_\tau(F)|_{W}\big)=0
\ \Longleftrightarrow\
\Ext^1\!\big(\mathcal{R}(C_\tau(F)|_{W}),\,k\big)=0,
\]
since the forward direction is Theorem~\ref{C:thm:bridge} and the reverse is Theorem~\ref{C:thm:local-reverse}.
\end{corollary}

% -------------------------
\subsection*{C.6. Naturality, exactness, and stability under admissible updates}
% -------------------------

\begin{proposition}[Naturality of the bridge]\label{C:prop:naturality}
For any filtration-preserving \(f:F\to G\), the natural square
\[
\begin{tikzcd}[row sep=1.2em, column sep=2.4em]
\Ext^1\!\big(\mathcal{R}(F),k\big)  \arrow[r, "\sim"]
  \arrow[d, "{\Ext^1(\mathcal{R}(f),\,k)}"'] &
\Hom\!\big(H^{-1}(\mathcal{R}(F)),k\big)
  \arrow[d, "{\Hom(H^{-1}(\mathcal{R}(f)),\,k)}"]\\
\Ext^1\!\big(\mathcal{R}(G),k\big)
  \arrow[r, "\sim"] &
\Hom\!\big(H^{-1}(\mathcal{R}(G)),k\big)
\end{tikzcd}
\]
commutes.
\end{proposition}

\begin{lemma}[Exact triangles and 2-out-of-3]\label{C:lem:2of3}
Suppose \(F_1\to F_2\to F_3\to F_1[1]\) is a distinguished triangle after realization on a window \([0,\tau]\) with \(\mathcal{R}(C_\tau F_i)\in D^{[-1,0]}\) for all \(i\).
If two of \(\Ext^1(\mathcal{R}(C_\tau F_i),k)\) vanish, then so does the third.
\end{lemma}

\begin{remark}[Stability under admissible updates]
When \(F\mapsto F'\) is non-expansive at persistence (e.g.\ deletion-type or \(\varepsilon\)-continuation post-collapse) and \(\mathcal{R}(C_\tau F),\mathcal{R}(C_\tau F')\in D^{[-1,0]}\), the verdict \(\Ext^1(\mathcal{R}(C_\tau F),k)=0\) persists to \(F'\) provided the edge groups remain isomorphic along the filtered colimit; cf.\ Proposition~\ref{C:prop:edge}, Lemma~\ref{C:lem:edge-ext}.
\end{remark}

% -------------------------
\subsection*{C.7. Implementation details, reproducibility, and logging}
% -------------------------

\begin{remark}[Run-time policy and manifest fields]\label{C:rk:runtime}
Enforce the order
\(\,F\to C_\tau F\to \mathcal{R}(C_\tau F)\to \Ext^1(-,k)\,\).
The manifest \texttt{run.yaml} must include:
\begin{itemize}
  \item \texttt{ext1\_eligible}: boolean; \texttt{true} iff \(\mathcal{R}(C_\tau F)\in D^{[-1,0]}\);
  \item \texttt{ext1\_used\_in\_gate}: boolean; \texttt{true} iff \texttt{ext1\_eligible} and the gate policy enables it;
  \item \texttt{amplitude}: reported as \([-1,0]\) or \texttt{>1};
  \item \texttt{gate\_order}: \texttt{collapse→realize→ext1};
  \item \texttt{q\_test}: \texttt{k[0]};
  \item \texttt{window\_E1\_degenerate}: boolean; if \texttt{true} and \texttt{tail\_iso:true}, the local reverse (Theorem~\ref{C:thm:local-reverse}) is enabled.
\end{itemize}
Outside eligibility, set \texttt{ext1\_eligible:false}, \texttt{ext1\_used\_in\_gate:false}, and continue the gate using persistence/tower/safety criteria only.
\end{remark}

\begin{remark}[Computational recipe]\label{C:rk:recipe}
On an eligible window, computing \(\Ext^1(\mathcal{R}(C_\tau F|_W),k)\) reduces to:
\begin{enumerate}\itemsep0.2em
  \item compute \(\varinjlim_t H_1((C_\tau F|_W)^t)\) by stabilizing the degree-\(1\) barcode on \(W\);
  \item take the dual: \(\Ext^1 \cong \Hom(\varinjlim_t H_1,k)\).
\end{enumerate}
No derived resolutions are needed in the amplitude \([-1,0]\) regime.
\end{remark}

\begin{verbatim}
# run.yaml (excerpt)
ext1_eligible: true
ext1_used_in_gate: true
amplitude: "[-1,0]"
gate_order: "collapse→realize→ext1"
q_test: "k[0]"
window_E1_degenerate: true
tail_iso: true
notes: "Local reverse (P3) enabled on E1=0 window with tail isomorphism."
\end{verbatim}

% -------------------------
\subsection*{C.8. Counterexamples and boundary cases}
% -------------------------

\begin{example}[Failure of the global reverse implication \(\Ext^1\Rightarrow \mathrm{PH}_1\)]
A single finite bar \((a,b)\) in degree \(1\) yields \(\mathrm{PH}_1(F)\neq 0\) though \(\varinjlim_{t\to+\infty}H_1(F^t)=0\), hence \(\Ext^1(\mathcal{R}(F),k)=0\) in the eligible regime. Thus the converse fails globally.
\end{example}

\begin{example}[Amplitude breach: diagnostics only]
If \(\mathcal{R}(C_\tau F)\in D^{[-2,0]}\) with \(H^{-2}\neq 0\), Lemma~\ref{C:lem:edge-ext} is inapplicable.
The \(\Ext^1\)-test is \emph{ineligible} and must not be used for gating.
\end{example}

\begin{example}[Non-constructible tails]
If \(\mathbf{P}_1(F)\) is not constructible, filtered colimits in \([\mathbb{R},\mathsf{Vect}_k]\) may exit \(\Pers^{\mathrm{ft}}_k\).
Eligibility fails by \textup{(B1)}.
\end{example}

% -------------------------
\subsection*{C.9. Additional safeguards and best practices}
% -------------------------

\begin{remark}[Monotonicity across windows]
If \(0<\tau\le \tau'\), eligibility at \(\tau'\) implies eligibility at \(\tau\).
Moreover, if \(\Ext^1(\mathcal{R}(C_{\tau'}F|_W),k)=0\), then \(\Ext^1(\mathcal{R}(C_{\tau}F|_W),k)=0\) by functoriality and Corollary~\ref{C:cor:dim}.
\end{remark}

\begin{remark}[Uniformity under base change]
Eligibility and the conclusion \(\Ext^1=0\) are stable under extending scalars \(k\subset K\).
Thus the gate verdict is field-independent within the amplitude \([-1,0]\) regime.
\end{remark}

% -------------------------
\subsection*{C.10. Scope and non-claims}
% -------------------------

\begin{remark}[Scope and non-claims]\label{C:rk:scope}
\textbf{The bridge \(\mathrm{PH}_1\Rightarrow \Ext^1\) is proved and used in \(D^{\mathrm{b}}(k\text{-mod})\).}
The global converse is \emph{false}.
The \emph{local reverse} (Theorem~\ref{C:thm:local-reverse}, formal statement of P3) holds on right-open windows where \(E_1=0\) and the \(H_1\)-tail stabilizes; definable windows with finite Čech depth are a sufficient regime where these hypotheses are verifiable.
All filtered-colimit uses obey Appendix~A, Remark~\ref{A:rk:filtered-colimits}.
Derived realizations into other targets appear only as \textbf{[Spec]} and are not part of the proved bridge.
\end{remark}

\medskip
\noindent\emph{Summary of Appendix C (reinforced and final).}
Under \((\mathrm{B}1)\)–\((\mathrm{B}3)\), the edge identification \(H^{-1}(\mathcal{R}(F))\cong \varinjlim_t H_1(F^t)\) and the amplitude \([-1,0]\) model yield the forward bridge \(\mathrm{PH}_1(F)=0\Rightarrow \Ext^1(\mathcal{R}(F),k)=0\).
On \(E_1\)-degenerate right-open windows with tail isomorphism, the \emph{local reverse} holds after collapse (Theorem~\ref{C:thm:local-reverse}), giving an equivalence when combined with the forward bridge.
The operational order is \textbf{collapse} \(\to\) \textbf{realize} \(\to\) \textbf{Ext}; the test object is \(k[0]\).
Outside eligibility, \(\Ext^1\) is excluded from gate decisions but may be logged.
All computations respect constructibility and filtered-colimit scope from Appendix~A.



% =========================
\section*{Appendix D. Towers, \texorpdfstring{$\mu,\nu$}{mu,nu}, and Examples [Proof/Example] (reinforced)}
% =========================

\addcontentsline{toc}{section}{Appendix D. Towers, $\mu,\nu$, and Examples}
\refstepcounter{section} % ensure stable cross-references with unnumbered section

\usetikzlibrary{arrows.meta,cd}
Throughout, fix a field \(k\).
We work in the constructible regime (Appendix~A).
For each degree \(i\in\ZZ\), let
\[
\Pfun_i:\ \FiltCh(k)\longrightarrow \Pers^{\mathrm{ft}}_k,\qquad
F\longmapsto \big(t\mapsto H_i(F^{t}C_\bullet)\big)
\]
send a bounded-in-degree filtered chain complex \(F\) to its constructible persistence module.
The truncation \(\T_\tau:\Pers^{\mathrm{ft}}_k\to\Pers^{\mathrm{ft}}_{k,\tau\text{-loc}}\) is exact and \(1\)-Lipschitz (Appendix~A).
Filtered colimits are computed objectwise in \([\RR,\Vect_k]\), with the scope rule of Appendix~A, Remark~\ref{A:rk:filtered-colimits}.
All statements at the filtered–complex layer are \emph{up to filtered quasi-isomorphism (f.q.i.)}; persistence-layer statements take place in \(\Pers^{\mathrm{ft}}_k\).
All quantities below may depend on the threshold \(\tau>0\); no monotonicity in \(\tau\) is asserted.

% -------------------------
% V-metric calculus (small reinforcement)
% -------------------------
\begin{proposition}[Calculus of $\mu,\nu$ in the $V$-metric]\label{prop:V-subadd}
Let \(V\) be a commutative quantale and endow \(\Pers^{\mathrm{ft}}_k\) with a Lawvere \(V\)-metric (Chapter~2, Def.~\ref{A:def:Vlawvere}; Appendix~A, Remark~\ref{rem:V-enriched}).
For any tower with apex and any \(\tau>0\), the obstruction indices computed \emph{after} \(\T_\tau\),
\[
\mu_{i,\tau}:=\gdim\Ker(\phi_{i,\tau}),\qquad
\nu_{i,\tau}:=\gdim\coker(\phi_{i,\tau}),
\]
satisfy, uniformly in the choice of \(V\):
\begin{enumerate}\itemsep0.2em
\item \textbf{Cofinal/f.q.i.\ invariance:} \(\mu,\nu\) are invariant under cofinal reindexing of the tower and under levelwise f.q.i.\ replacements.
\item \textbf{Composition subadditivity:} for composable comparison maps \(\psi,\phi\),
\[
\mu(\psi\!\circ\!\phi)\le \mu(\phi)+\mu(\psi),\qquad
\nu(\psi\!\circ\!\phi)\le \nu(\psi)+\mu(\phi).
\]
\item \textbf{Additivity on finite sums:} \(\mu,\nu\) are additive under finite direct sums of towers.
\item \textbf{Window finiteness on definable covers:} on o\hbox{-}minimal definable windows with finite Čech/Leray depth (Appendix~H/J), all \(\mu_{i,\tau},\nu_{i,\tau}\) are finite and only finitely many \(\tau\)-criticalities occur per window.
\end{enumerate}
All multiplicities are read as the number of \(I[0,\infty)\) summands in the barcode of the relevant kernel/cokernel after \(\T_\tau\).
\end{proposition}

\begin{proof}[Proof sketch]
By Appendix~A, Remark~\ref{rem:V-enriched}, \(V\)-enrichment does not alter the underlying abelian/Serre calculus; kernels, cokernels, and \(\T_\tau\) are computed in \(\Pers^{\mathrm{ft}}_k\) exactly as in the unenriched case.
Hence (1)–(3) follow from the barcode calculus and the finite-dimensional linear algebra bounds used in Propositions~\ref{prop:D-gdim-iff}, \ref{prop:D-subadd}, and \ref{prop:D-additive}.
For (4), definable covers yield finite Čech depth, so only finitely many events contribute on a window (Appendix~H/J), and finiteness follows.
\end{proof}

\medskip
\noindent\textbf{Comparison map and obstruction indices.}
Let \(\{F_n\}_{n\in\NN}\) be a directed system in \(\FiltCh(k)\).
Let \(F_\infty\) be an apex equipped with a cocone \(F_n\to F_\infty\) (indexing category \(\NN\cup\{\infty\}\) with unique morphisms \(n\to\infty\)).
For each \(i\) and \(\tau>0\) set
\[
\phi_{i,\tau}:\ \colim_{n\in\NN}\, \T_\tau\!\big(\Pfun_i(F_n)\big)\ \longrightarrow\ \T_\tau\!\big(\Pfun_i(F_\infty)\big),
\]
the canonical comparison in \([\RR,\Vect_k]\).
Define the \emph{tower obstruction indices}
\[
\mu_{i,\tau}:=\gdim\Ker(\phi_{i,\tau}),\qquad
\nu_{i,\tau}:=\gdim\coker(\phi_{i,\tau}),\qquad
\muc:=\sum_i\mu_{i,\tau},\ \ \nuc:=\sum_i\nu_{i,\tau}.
\]
These sums are finite because complexes are bounded in homological degrees (constructible range).
The pair \((\muc,\nuc)\) detects \emph{Type~IV} (tower-level) failures at scale~\(\tau\).\footnote{For compositions in finite-dimensional linear algebra one has surrogate subadditivity
\(\dim\Ker(g\circ f)\le \dim\Ker f+\dim\Ker g\) and
\(\dim\coker(g\circ f)\le \dim\coker g+\dim\Ker f\).
Appendix~J provides the persistence-layer analogue after applying \(\T_\tau\) and taking generic-fiber dimensions.}

\begin{remark}[Generic dimension after truncation]\label{rem:D-generic-dim}
In \(\Pers^{\mathrm{ft}}_k\), after applying \(\T_\tau\) the kernel and cokernel of any morphism decompose (noncanonically) as finite direct sums of interval modules.
We write \(\gdim(-)\) for the \emph{generic fiber} dimension, i.e.\ the multiplicity of the infinite bar \(I[0,\infty)\) in that decomposition.
Finite bars contribute zero generic fiber.
\end{remark}

\begin{remark}[Invariance of \((\muc,\nuc)\)]
The indices \((\muc,\nuc)\) are invariant under levelwise f.q.i.\ replacements of the tower and apex: if \(F_n\simeq_{\mathrm{f.q.i.}}F'_n\) and \(F_\infty\simeq_{\mathrm{f.q.i.}}F'_\infty\), then \(\Pfun_i\) sends these to isomorphisms in \(\Pers^{\mathrm{ft}}_k\), hence \(\Ker/\coker\) (and thus \(\mu,\nu\)) are unchanged.
They are also invariant under cofinal reindexing of the tower, since filtered colimits over cofinal subdiagrams are canonically isomorphic.
\end{remark}

\begin{figure}[t]
\centering
\begin{tikzcd}[row sep=1.0em, column sep=2.0em]
F_1 \arrow[r] \arrow[dr] &
F_2 \arrow[r] \arrow[dr] &
\cdots \arrow[r] &
F_n \arrow[r] \arrow[dr] &
\cdots \arrow[r] &
F_\infty \\
& \Pfun_i(F_1) \arrow[r] &
\Pfun_i(F_2) \arrow[r] &
\cdots \arrow[r] &
\Pfun_i(F_n) \arrow[r] &
\Pfun_i(F_\infty)
\end{tikzcd}
\caption{A tower with apex \(F_\infty\) and its image under \(\Pfun_i\).
The comparison \(\phi_{i,\tau}\) (defined after applying \(\T_\tau\)) measures the failure of the cocone to exhibit a colimit at scale \(\tau\).}
\end{figure}

\subsection*{D.1. Calculus of defects: generic-fiber interpretation, naturality, and subadditivity}

\begin{proposition}[Generic-fiber interpretation]\label{prop:D-gdim-iff}
Let \(f:M\to N\) be a morphism in \(\Pers^{\mathrm{ft}}_k\) and fix \(\tau>0\).
Then, in the barcode decomposition of \(\Ker(\T_\tau f)\) and \(\coker(\T_\tau f)\), the multiplicity of \(I[0,\infty)\) equals \(\gdim\Ker(\T_\tau f)\) and \(\gdim\coker(\T_\tau f)\), respectively.
Equivalently, \(\gdim\) is the generic fiber dimension of the corresponding functor \(\RR\to\Vect_k\).
\end{proposition}

\begin{proof}
This is standard for constructible 1D persistence over a field: after \(\T_\tau\), objects are pointwise finite-dimensional and split as finite sums of interval modules; see Appendix~A.
The generic fiber (dimension on a cofinal ray) counts infinite bars, i.e.\ copies of \(I[0,\infty)\).
\end{proof}

\begin{definition}[Morphisms of towers and naturality of \(\phi\)]
A \emph{morphism of towers with apex} \((F_\bullet,F_\infty)\to(G_\bullet,G_\infty)\) is a collection of maps \(u_n:F_n\to G_n\) and \(u_\infty:F_\infty\to G_\infty\) commuting with all structure maps to the apex.
Applying \(\Pfun_i\), then \(\T_\tau\), and passing to the filtered colimit yields a commutative square
\[
\begin{tikzcd}
\colim_n \T_\tau \Pfun_i(F_n) \arrow[r,"\phi^{F}_{i,\tau}"] \arrow[d,"\colim\,\T_\tau\Pfun_i(u_n)"'] &
\T_\tau \Pfun_i(F_\infty) \arrow[d,"\T_\tau \Pfun_i(u_\infty)"] \\
\colim_n \T_\tau \Pfun_i(G_n) \arrow[r,"\phi^{G}_{i,\tau}"] &
\T_\tau \Pfun_i(G_\infty)
\end{tikzcd}
\]
i.e.\ \(\phi_{i,\tau}\) is natural in the tower.
\end{definition}

\begin{proposition}[Functoriality and subadditivity under composition]\label{prop:D-subadd}
Let \((F_\bullet,F_\infty)\xrightarrow{u}(G_\bullet,G_\infty)\xrightarrow{v}(H_\bullet,H_\infty)\) be morphisms of towers with apex.
Then, for each \(i,\tau\),
\[
\mu(\phi^{H}_{i,\tau}\!\circ\!\colim\,\T_\tau\Pfun_i(v_n\circ u_n))
\ \le\
\mu(\phi^{G}_{i,\tau}\!\circ\!\colim\,\T_\tau\Pfun_i(u_n))+\mu(\phi^{H}_{i,\tau}\!\circ\!\colim\,\T_\tau\Pfun_i(v_n)),
\]
\[
\nu(\phi^{H}_{i,\tau}\!\circ\!\colim\,\T_\tau\Pfun_i(v_n\circ u_n))
\ \le\
\nu(\phi^{H}_{i,\tau}\!\circ\!\colim\,\T_\tau\Pfun_i(v_n))+\mu(\phi^{G}_{i,\tau}\!\circ\!\colim\,\T_\tau\Pfun_i(u_n)).
\]
In particular, writing simply \(\phi=\phi_{i,\tau}\) for a fixed tower, any factorization of \(\phi\) yields
\(\mu(\phi\circ\psi)\le \mu(\psi)+\mu(\phi)\) and \(\nu(\phi\circ\psi)\le \nu(\phi)+\mu(\psi)\).
\end{proposition}

\begin{proof}
Apply \(\T_\tau\) and use exactness together with the standard inequalities for kernels and cokernels of compositions in finite-dimensional linear algebra, then pass to generic fibers via Proposition~\ref{prop:D-gdim-iff}.
\end{proof}

\begin{proposition}[Additivity on finite direct sums]\label{prop:D-additive}
For two towers \((F_\bullet,F_\infty)\) and \((G_\bullet,G_\infty)\),
\[
\mu\big((F\oplus G)_\bullet,(F\oplus G)_\infty\big)=\mu(F_\bullet,F_\infty)+\mu(G_\bullet,G_\infty),
\qquad
\nu\big((F\oplus G)_\bullet,(F\oplus G)_\infty\big)=\nu(F_\bullet,F_\infty)+\nu(G_\bullet,G_\infty).
\]
\end{proposition}

\begin{proof}
\(\Pfun_i\) and \(\T_\tau\) preserve finite direct sums; kernels and cokernels preserve finite direct sums; \(\gdim\) is additive on direct sums.
\end{proof}

\begin{proposition}[Invariance under f.q.i.\ and cofinal reindexing]\label{prop:D-invariance}
If \(F_n\simeq_{\mathrm{f.q.i.}}F'_n\) levelwise and \(F_\infty\simeq_{\mathrm{f.q.i.}}F'_\infty\), then \(\mu,\nu\) agree for the two towers.
If \(J\subset \NN\) is cofinal, then restricting the tower to \(J\) does not change \(\mu,\nu\).
\end{proposition}

\begin{proof}
Follows from \(\Pfun_i\) sending f.q.i.\ to isomorphisms in \(\Pers^{\mathrm{ft}}_k\), exactness of \(\T_\tau\), and the fact that filtered colimits over cofinal subdiagrams are canonically isomorphic.
\end{proof}

\subsection*{D.2. Toy towers: pure kernel / pure cokernel / mixed}

\begin{example}[Pure cokernel at a fixed scale]\label{D:ex:pure-coker}
Fix \(\tau>0\) and degree \(i=1\).
Let \(\Pfun_1(F_n)=I[0,\tau-\tfrac1n)\) with transition maps the evident inclusions.
Let \(F_\infty\) satisfy \(\Pfun_1(F_\infty)=I[0,\infty)\).
Then \(\T_\tau(\Pfun_1(F_n))=0\) for all \(n\), whereas \(\T_\tau(\Pfun_1(F_\infty))\cong I[0,\infty)\).
Hence \(\phi_{1,\tau}:0\to I[0,\infty)\) has trivial kernel and nontrivial cokernel, so \(\mu_{1,\tau}=0\) and \(\nu_{1,\tau}=1\) (pure cokernel).
\end{example}

\begin{example}[Pure kernel at a fixed scale]\label{D:ex:pure-ker}
Fix \(\tau>0\).
Let \(\Pfun_1(F_n)=I[0,\infty)\) for all \(n\), with transition maps the identities (a stationary directed system).
Let \(F_\infty\) satisfy \(\Pfun_1(F_\infty)=0\), and take the cocone \(\Pfun_1(F_n)\to \Pfun_1(F_\infty)\) to be \(0\) for all \(n\).
Then \(\T_\tau(\Pfun_1(F_n))\cong I[0,\infty)\) for all \(n\), so the source of \(\phi_{1,\tau}\) is \(I[0,\infty)\), while the target is \(0\).
Thus \(\phi_{1,\tau}: I[0,\infty)\to 0\) has nontrivial kernel and zero cokernel, hence \(\mu_{1,\tau}=1\), \(\nu_{1,\tau}=0\) (pure kernel).
\end{example}

\begin{example}[Mixed]\label{D:ex:mixed}
Fix \(\tau>0\) and set
\[
\Pfun_1(F_n)\ =\ I[0,\tau-\tfrac1n)\ \oplus\ I[0,\infty),
\]
with transition maps the obvious inclusions on the first summand and the identities on the second.
Take \(F_\infty\) with \(\Pfun_1(F_\infty)=I[0,\infty)\oplus 0\), using cocone maps that send the second summand to \(0\).
Then the first summand yields a cokernel contribution exactly as in Example~\ref{D:ex:pure-coker}, while the second yields a kernel contribution exactly as in Example~\ref{D:ex:pure-ker}.
Hence \(\mu_{1,\tau}=\nu_{1,\tau}=1\) (mixed).
\end{example}

All three persistence-level towers are realizable by filtered complexes via the interval-realization assignment \(\Ufun\) (Appendix~B), up to f.q.i.; constructibility is preserved.

\subsection*{D.3. When \texorpdfstring{$\phi_{i,\tau}$}{phi} is an isomorphism: \texorpdfstring{$(\muc,\nuc)=(0,0)$}{(mu,nu)=(0,0)}}

\begin{proposition}[Isomorphism criterion]\label{D:prop:iso-zero}
Assume:
\begin{enumerate}[label=(\roman*)]
\item degreewise filtered colimits in \(\FiltCh(k)\) are computed objectwise on chains and filtrations;
\item each \(\Pfun_i(F_n)\) lies in \(\Pers^{\mathrm{ft}}_k\);
\item \(\T_\tau\) commutes with the filtered colimit of \(\{\Pfun_i(F_n)\}\) in \([\RR,\Vect_k]\), and the result is constructible (Appendix~A, Theorem~\ref{A:thm:localization});
\item the cocone exhibits a colimit at persistence level: the canonical map
\(\colim_{n}\Pfun_i(F_n)\xrightarrow{\ \cong\ }\Pfun_i(F_\infty)\)
is an isomorphism in \([\RR,\Vect_k]\).
\end{enumerate}
Then for every \(i\) and \(\tau>0\), the comparison map \(\phi_{i,\tau}\) is an isomorphism in \(\Pers^{\mathrm{ft}}_k\).
Consequently \((\muc,\nuc)=(0,0)\).
\end{proposition}

\begin{remark}
Condition (iv) is automatic if \(F_\infty\) \emph{is} the colimit of \(\{F_n\}\) in a model of filtered complexes for which \(\Pfun_i\) is computed objectwise and the scope rule of Appendix~A applies; no claim is made beyond that regime.
\end{remark}

\subsection*{D.4. Sufficient conditions ensuring \texorpdfstring{$(\muc,\nuc)=(0,0)$}{(mu,nu)=(0,0)}}

The summability condition \(\sum_{n} d_{\mathrm{int}}\!\big(\Pfun_i(F_{n+1}),\Pfun_i(F_n)\big)<\infty\) \emph{alone} does not guarantee \((\muc,\nuc)=(0,0)\); see §D.5.1.
The following hypotheses are sufficient.

\begin{theorem}\label{D:thm:sufficient}
Fix \(i\) and \(\tau>0\).
Each of the following implies that \(\phi_{i,\tau}\) is an isomorphism (hence \((\muc,\nuc)=(0,0)\)):
\begin{enumerate}[label=(S\arabic*)]
\item \textbf{Commutation and apex colimit:} \(\T_\tau\) commutes with the filtered colimit of \(\{\Pfun_i(F_n)\}\) in \([\RR,\Vect_k]\), the outcome is constructible, and the cocone exhibits a colimit at persistence level (i.e.\ Proposition~\ref{D:prop:iso-zero}(iv) holds).
\item \textbf{No \(\tau\)-accumulation from below:} there exists \(\eta>0\) such that, for all sufficiently large \(n\), no bar in \(\Pfun_i(F_n)\) has length in the half-open interval \((\tau-\eta,\tau)\).
Equivalently, there is no sequence of bar lengths strictly increasing to \(\tau\).
\item \textbf{\(\T_\tau\)-Cauchy with compatible cocone:} the sequence \(\T_\tau(\Pfun_i(F_n))\) is Cauchy in the interleaving metric, and the cocone to \(\T_\tau(\Pfun_i(F_\infty))\) identifies the metric limit with the colimit target.
(Here we use only the standard completeness/uniqueness of limits for p.f.d.\ barcodes under the bottleneck/interleaving metric.)
\end{enumerate}
\end{theorem}

\begin{proof}
(S1) is Proposition~\ref{D:prop:iso-zero}.
For (S2), the gap prevents creation at the apex of new bars of length \(>\tau\):
every long bar in \(\T_\tau(\Pfun_i(F_\infty))\) must appear at some finite stage and stabilize, yielding bijectivity on interval factors.
For (S3), completeness of the space of p.f.d.\ persistence modules up to isometry implies a unique metric limit; the stated compatibility identifies it with the colimit target, so \(\phi_{i,\tau}\) is an isometry and hence an isomorphism in \(\Pers^{\mathrm{ft}}_k\).
\end{proof}

\subsection*{D.5. A counterexample: \(\sum d_{\mathrm{int}}<\infty\) yet \((\muc,\nuc)\neq(0,0)\)}\label{subsec:D-counter}

\begin{example}[Summable increments, pure cokernel at the apex]\label{D:ex:summable}
Fix \(\tau>0\) and set \(\ell_n=\tau-\sum_{m\ge n}2^{-m}\uparrow \tau\), so that
\(\sum_{n}(\ell_{n+1}-\ell_{n})=\sum_{n}2^{-n}<\infty\).
Let \(M_n:=I[0,\ell_n)\) with \(M_{n}\into M_{n+1}\) the standard inclusions.
Then
\[
d_{\mathrm{int}}(M_{n},M_{n+1})=\tfrac12(\ell_{n+1}-\ell_{n})=2^{-(n+1)},\qquad \sum_{n} d_{\mathrm{int}}(M_{n},M_{n+1})<\infty.
\]
Let \(\Pfun_1(F_n)=M_n\), and choose an apex with \(\Pfun_1(F_\infty)=I[0,\infty)\).
For every \(n\), \(\T_\tau(M_n)=0\), while \(\T_\tau(\Pfun_1(F_\infty))=I[0,\infty)\).
Thus \(\muc=0\), \(\nuc=1\) (pure cokernel), despite the summable interleaving distances along the tower.
\end{example}

This shows that \(\sum d_{\mathrm{int}}<\infty\) alone is insufficient to force \((\muc,\nuc)=(0,0)\).

\subsection*{D.6. Converse failures and the Type~IV catalog}

\paragraph{D.6.1. \(\Ext^1=0\) does not imply \(\PH_1=0\).}
Let \(A\in D^{[-1,0]}(k\text{-mod})\) with \(H^{-1}(A)=0\) and \(H^0(A)\ne 0\), e.g.\ the stalk complex \(V[0]\) for a nonzero \(k\)-space \(V\).
Then \(\Ext^1(A,k)\cong \Hom(H^{-1}(A),k)=0\) by Appendix~C (Lemma~\ref{C:lem:edge-ext}).
Choose \(F\in\FiltCh(k)\) with \(\Pfun_1(F)\neq 0\) (e.g.\ a single finite interval) and \(\Rfun(F)\simeq A\); this can be arranged up to f.q.i.\ using the realization assignment \(\Ufun\) (Appendix~B).
Hence \(\Ext^1(\Rfun(F),k)=0\) while \(\PH_1(F)\neq 0\), refuting the converse of the bridge.

\paragraph{D.6.2. Type~IV (pure cokernel) at fixed \(\tau\).}
Example~\ref{D:ex:pure-coker} exhibits \(\muc=0\), \(\nuc>0\) with \(\T_\tau(\Pfun_1(F_n))=0\) for all \(n\) but \(\T_\tau(\Pfun_1(F_\infty))\neq 0\).
Thus finite layers appear admissible while the apex fails.

\paragraph{D.6.3. Type~IV (mixed).}
Example~\ref{D:ex:mixed} yields \(\muc>0\) and \(\nuc>0\) simultaneously, demonstrating that both kernel and cokernel defects can occur in the same tower.

\paragraph{D.6.4. Realization notes.}
All persistence-level constructions above are realizable by filtered complexes via \(\Ufun\) (Appendix~B), up to f.q.i.; constructibility is preserved.

\subsection*{D.7. Restart/Summability for window pasting}\label{subsec:D-restart}

All persistence-layer statements below are made \emph{after} applying \(\T_\tau\).

\begin{definition}[Per-window safety margin and pipeline budget]\label{D:def:window-budget}
Let \(\{W_k=[u_k,u_{k+1})\}_{k\in K}\) be a MECE partition (Appendix~A, Definition~\ref{A:def:MECE}). On each window \(W_k\), fix a collapse threshold \(\tau_k>0\). For a degree \(i\), define the \emph{pipeline budget}
\[
\Sigma\delta_k(i)\ :=\ \sum_{U\in W_k}\Big(\delta^{\mathrm{alg}}_{U}(i,\tau_k)+\delta^{\mathrm{disc}}_{U}(i,\tau_k)+\delta^{\mathrm{meas}}_{U}(i,\tau_k)\Big),
\]
and the \emph{safety margin} \(\mathrm{gap}_{\tau_k}(i)>0\) as the configured slack for B\hyp Gate\(^{+}\) on \(W_k\) and degree \(i\).
\end{definition}

\begin{lemma}[Restart inequality]\label{D:lem:restart}
Assume that, on window \(W_k\), B\hyp Gate\(^{+}\) passes with \(\mathrm{gap}_{\tau_k}(i)>\Sigma\delta_k(i)\), and that the transition to \(W_{k+1}\) is realized by a finite composition of \emph{deletion-type} steps and \(\varepsilon\)\hyp continuations (both measured after \(\T_\tau\)). Then there exists \(\kappa\in(0,1]\), depending only on the admissible step class and the \(\tau\)-adaptation policy, such that
\[
\mathrm{gap}_{\tau_{k+1}}(i)\ \ge\ \kappa\bigl(\mathrm{gap}_{\tau_k}(i)-\Sigma\delta_k(i)\bigr).
\]
\end{lemma}

\begin{proof}[Proof sketch]
Deletion-type steps are non\hyp increasing for the monitored indicators after \(\T_\tau\) (Appendix~E), and \(\varepsilon\)\hyp continuations are \(1\)\hyp Lipschitz. Aggregating drifts yields the stated retention factor \(\kappa\).
\end{proof}

\begin{definition}[Summability]\label{D:def:summability}
A run satisfies \emph{Summability} (on a degree set \(I\subset\ZZ\)) if
\[
\sum_{k\in K}\Sigma\delta_k(i)\ <\ \infty\qquad(\forall\,i\in I).
\]
A sufficient design is a geometric decay of \(\tau_k\) (hence of spectral/temporal bins) and bounded per\hyp window step counts.
\end{definition}

\begin{theorem}[Pasting windowed certificates]\label{D:thm:pasting}
Let \(\{W_k\}_k\) be MECE, and on each \(W_k\) let B\hyp Gate\(^{+}\) pass with \(\mathrm{gap}_{\tau_k}(i)>\Sigma\delta_k(i)\) for all \(i\in I\).
If the Restart inequality (Lemma~\ref{D:lem:restart}) holds at every transition and Summability (Definition~\ref{D:def:summability}) holds, then the concatenation of windowed certificates yields a global certificate on \(\bigcup_k W_k\) for the degrees \(i\in I\).
\end{theorem}

% -------------------------
% Σδ convergence pseudocode (added)
% -------------------------
\paragraph{Convergence Manager (Σ\(\delta\)) — auditable pseudocode.}
\begin{verbatim}
# Inputs:
#  windows: list of MECE windows W_k
#  degrees: monitored degree set I
#  tau:     list of collapse thresholds tau_k aligned with windows
#  deltas:  per-window lists of triples (delta_alg, delta_disc, delta_meas) per degree
#  policy:  either "geometric(r<1)" or "p_series(p>1)"
#
def convergence_manager(windows, degrees, tau, deltas, policy):
    total = {i: 0.0 for i in degrees}
    for k, Wk in enumerate(windows):
        # accumulate local budget
        for i in degrees:
            sigma = 0.0
            for (d_alg, d_disc, d_meas) in deltas[k][i]:
                sigma += d_alg + d_disc + d_meas    # Quantale-additive
            assert gap_tau[k][i] > sigma, "Restart inequality failed"
            total[i] += sigma
        # schedule compliance
        if policy.kind == "geometric":
            r = policy.r
            assert 0 < r < 1 and tau[k+1] <= r * tau[k]
        elif policy.kind == "p_series":
            p = policy.p
            assert p > 1
            # (implicit: window widths form a p-series; enforced upstream)
    return {i: (total[i] < float("inf")) for i in degrees}
\end{verbatim}

\subsection*{D.8. Stability bands, \texorpdfstring{$\tau$}{tau}-sweeps, and detection algorithm}\label{subsec:D-stab-bands}

\begin{definition}[Stability band via \(\tau\)-sweep]\label{D:def:stab-band}
Fix a window \(W\) and degree \(i\). Let \(\{\tau_\ell\}_{\ell=1}^L\) be an increasing \(\tau\)-sweep.
A contiguous block \(\{\tau_a,\dots,\tau_b\}\) is a \emph{stability band} if
\[
\mu_{i,\tau_\ell}=\nu_{i,\tau_\ell}=0\quad\text{for all }\ \ell\in\{a,\dots,b\},
\]
and the verdict persists upon \emph{refining} the sweep (inserting new \(\tau\)-values) without introducing \(\mu\) or \(\nu\) in the band.
\end{definition}

\begin{proposition}[Robust detection of stability bands]\label{D:prop:robust-band}
Assume (S1)–(S3) of Theorem~\ref{D:thm:sufficient} hold on \(W\). Then any sufficiently fine \(\tau\)-sweep admits stability bands covering all \(\tau\) at which \(\phi_{i,\tau}\) is an isomorphism; conversely, detecting a stability band by a sweep and its refinement certifies \((\muc,\nuc)=(0,0)\) on the band.
\end{proposition}

\begin{proof}[Proof sketch]
Under (S1)–(S3), \(\phi_{i,\tau}\) is an isomorphism on open neighborhoods of the corresponding \(\tau\)’s. A fine sweep samples each neighborhood; refinement eliminates aliasing. The converse follows by definition.
\end{proof}

\begin{remark}[Caveat: non-monotonicity in \(\tau\)]
There is no general monotonicity of \(\mu_{i,\tau}\) or \(\nu_{i,\tau}\) in \(\tau\).
Stability bands may be separated by isolated \(\tau\)-values where \(\phi_{i,\tau}\) fails to be an isomorphism.
\end{remark}

\paragraph{Detection algorithm (auditable).}
Given a sweep \(\mathsf{TauSweep}=\{\tau_\ell\}_{\ell=1}^L\):
\begin{enumerate}[label=(A\arabic*)]
\item For each \(\ell\), compute \(\phi_{i,\tau_\ell}\) and record \((\mu_{i,\tau_\ell},\nu_{i,\tau_\ell})\).
\item Extract maximal contiguous indices \([a,b]\) with \((\mu,\nu)=(0,0)\).
\item Refine by inserting midpoints \(\tau'=\frac12(\tau_\ell+\tau_{\ell+1})\) within each candidate band and recompute \((\mu,\nu)\).
\item Accept a band if all refined points also yield \((0,0)\).
\item Emit certificate with hashes of inputs, tower metadata, and the flags indicating which of (S1)–(S3) were used.
\end{enumerate}

\subsection*{D.9. Implementation guide: APIs, stubs, and tests}\label{subsec:D-impl}

All persistence-layer computations are understood after applying \(\T_\tau\).

\subsection*{Lean stubs (illustrative).}
\begin{lstlisting}[language=,caption={Lean 4 stubs for towers and obstruction indices}]
namespace PH
structure Tower (ι : Type) :=
  (F    : ι → FiltCh)
  (apex : FiltCh)
  (toApex : ∀ i : ι, ChainMap (F i) apex)

def P_i (i : ℤ) : FiltCh → Pers := -- assumed given
def T_tau (τ : ℝ) : Pers → Pers := -- exact, 1-Lipschitz

def phi (i : ℤ) (τ : ℝ) (T : Tower ℕ) : PersHom :=
  have src := colim (fun n => T_tau τ (P_i i (T.F n)))
  have tgt := T_tau τ (P_i i T.apex)
  comparison src tgt -- canonical

def gdim (M : Pers) : Nat := -- multiplicity of I[0,∞) in barcode

def mu (i : ℤ) (τ : ℝ) (T : Tower ℕ) : Nat := gdim (kernel (phi i τ T))
def nu (i : ℤ) (τ : ℝ) (T : Tower ℕ) : Nat := gdim (cokernel (phi i τ T))
end PH
\end{lstlisting}

\subsection*{Coq stubs (illustrative).}
\begin{lstlisting}[language=,caption={Coq stubs for towers and obstruction indices}]
Module PH.
Record Tower := {
  F      : nat -> FiltCh;
  apex   : FiltCh;
  toApex : forall n, ChainMap (F n) apex
}.

Parameter P_i   : Z -> FiltCh -> Pers.
Parameter T_tau : R -> Pers -> Pers. (* exact, 1-Lipschitz *)

Definition phi (i:Z) (tau:R) (T:Tower) : PersHom :=
  let src := colim (fun n => T_tau tau (P_i i (F T n))) in
  let tgt := T_tau tau (P_i i (apex T)) in
  comparison src tgt.

Parameter gdim : Pers -> nat.

Definition mu (i:Z) (tau:R) (T:Tower) : nat := gdim (kernel (phi i tau T)).
Definition nu (i:Z) (tau:R) (T:Tower) : nat := gdim (cokernel (phi i tau T)).
End PH.
\end{lstlisting}

\paragraph{Sample tests.}
\begin{enumerate}[label=({T}\arabic*)]
\item \textbf{T3 (Filtered-colim stability).} Construct a tower whose apex is the filtered colimit and for which \(\T_\tau\) commutes with colim. Verify \(\phi_{i,\tau}\) is an iso and \((\mu,\nu)=(0,0)\).
\item \textbf{T7 (Toy towers).} Instantiate the pure kernel, pure cokernel, and mixed towers of §D.2 and confirm \((\mu,\nu)=(1,0),(0,1),(1,1)\), respectively.
\item \textbf{T9 (No \(\tau\)-accumulation).} Create barcodes with an \(\eta\)-gap below \(\tau\); confirm \(\phi_{i,\tau}\) is iso.
\item \textbf{T10 (Cauchy+compatibility).} Build a Cauchy sequence in the bottleneck metric whose limit equals the apex post-\(\T_\tau\); confirm iso.
\item \textbf{T11 (Restart+Summability).} Simulate windows and transitions satisfying Lemma~\ref{D:lem:restart} and Definition~\ref{D:def:summability}; verify global certificate via Theorem~\ref{D:thm:pasting}.
\end{enumerate}

\subsection*{D.10. Audit schema: \texttt{run.yaml}, JSON, and HDF5 layout}\label{subsec:D-audit}

\subsection*{YAML fields (mandatory).}
\begin{lstlisting}[language=,caption={Minimal audit fields in run.yaml (bands added)}]
phi:
  idx:
    - i: 1
      tau: 0.75
      iso: true          # φ_{i,τ} isomorphism?
      mu: 0
      nu: 0
      flags:
        S1: true         # used commutation+apex-colim
        S2: false
        S3: false
      iso_tail:
        passed: true     # tail check on refined sweep
        refinement_levels: 2
  bands:
    - i: 1
      start_tau: 0.70
      end_tau: 0.82
      certified: true    # (μ,ν)=(0,0) across band
      method: "sweep+refine"  # provenance of certification
windows:
  collapse:
    tau_sweep: [0.5, 0.6, 0.7, 0.75, 0.8]
persistence:
  phi_iso_tail: "strict" # policy for refinement acceptance
summability:
  policy: "geometric"
  r: 0.8
  total_delta:
    i=0: 0.137
    i=1: 0.092
tower:
  edges:
    - src: 0; dst: 1; kind: "inclusion"
    - src: 1; dst: 2; kind: "inclusion"
hash:
  inputs: "sha256:..."
  code:   "sha256:..."
\end{lstlisting}

\subsection*{JSON snippet (optional).}
\begin{lstlisting}
{
  "phi": {
    "idx": [
      {"i":1, "tau":0.75, "iso":true, "mu":0, "nu":0,
       "flags":{"S1":true,"S2":false,"S3":false},
       "iso_tail":{"passed":true,"refinement_levels":2}}
    ],
    "bands": [
      {"i":1, "start_tau":0.70, "end_tau":0.82,
       "certified":true, "method":"sweep+refine"}
    ]
  },
  "summability":{"policy":"geometric","r":0.8,
                 "total_delta":{"i=0":0.137,"i=1":0.092}}
}
\end{lstlisting}



\paragraph{HDF5 groups (canonical order; bands added).}
We store comparison data under \texttt{/phi} and tower metadata under \texttt{/tower}.
A minimal layout is:

\medskip
\begin{center}
\begin{tabular}{ll}
\toprule
Group/Dataset & Contents \\
\midrule
\texttt{/phi/idx/i} & integer degrees \(i\) \\
\texttt{/phi/idx/tau} & real thresholds \(\tau\) \\
\texttt{/phi/idx/iso} & boolean flags \\
\texttt{/phi/idx/mu} & nonnegative integers \(\mu_{i,\tau}\) \\
\texttt{/phi/idx/nu} & nonnegative integers \(\nu_{i,\tau}\) \\
\texttt{/phi/idx/flags} & bitmask for (S1,S2,S3) \\
\texttt{/phi/idx/iso\_tail/passed} & boolean \\
\texttt{/phi/bands/i} & integer degrees \(i\) \\
\texttt{/phi/bands/start\_tau} & real \(\tau\)-starts of stability bands \\
\texttt{/phi/bands/end\_tau} & real \(\tau\)-ends of stability bands \\
\texttt{/phi/bands/certified} & boolean certification flags \\
\texttt{/phi/bands/method} & string provenance (e.g.\ sweep+refine) \\
\texttt{/tower/edges/src} & integer source indices \\
\texttt{/tower/edges/dst} & integer target indices \\
\texttt{/tower/edges/kind} & categorical: inclusion/deletion/epsilon \\
\bottomrule
\end{tabular}
\end{center}

\subsection*{D.11. Additional formalities: tau-naturality and bandwise certification}

\begin{proposition}[Piecewise constancy off critical thresholds]\label{prop:D-piecewise}
Fix \(i\) and a tower.
There exists a finite set \(S\subset(0,\infty)\) consisting of bar lengths and their finite sums and differences such that \(\mu_{i,\tau}\) and \(\nu_{i,\tau}\) are locally constant on each connected component of \((0,\infty)\setminus S\).
\end{proposition}

\begin{proof}[Proof sketch]
Within the constructible regime, changes in \(\Ker/\coker\) after \(\T_\tau\) can occur only when \(\tau\) crosses endpoints of bars that interact with colim/cocone structure; this yields a finite critical set after fixing a finite window of degrees and using boundedness.
\end{proof}

\begin{corollary}[Bandwise certification]
If \(\phi_{i,\tau_0}\) is an isomorphism at some \(\tau_0\) lying in a component of \((0,\infty)\setminus S\), then it is an isomorphism on the whole component.
\end{corollary}

\subsection*{D.12. Completion note and cross-module conventions}

\begin{remark}[No further supplementation required]
This appendix provides: (i) the definition and calculus of the tower obstruction indices \((\mu,\nu)\) (generic fiber dimensions after truncation), including their $V$-metric invariance and window finiteness on definable covers; (ii) naturality and functoriality of the comparison map \(\phi\), with subadditivity under composition and additivity under direct sums; (iii) illustrative toy towers (pure kernel/cokernel/mixed) and a counterexample showing that \(\sum d_{\mathrm{int}}<\infty\) does not force \((\muc,\nuc)=(0,0)\); (iv) sufficient conditions (S1)–(S3) guaranteeing \((\muc,\nuc)=(0,0)\); (v) a \emph{Restart/Summability} framework with auditable pseudocode to paste windowed certificates into global ones; (vi) a robust \(\tau\)-sweep procedure and \emph{stability bands} to certify \((\muc,\nuc)=(0,0)\) on contiguous \(\tau\)-ranges; and (vii) implementation-grade audit schemas (YAML/JSON/HDF5 with \texttt{bands}), API stubs (Lean/Coq), and test items.
All statements are confined to the v16.0 guard-rails (constructible \(1\)D persistence over a field; persistence-layer equalities after truncation; f.q.i.\ on filtered complexes), and no further supplementation is required for operational use in the proof framework.
\end{remark}

\medskip
\noindent\textbf{Cross-module conventions.}
Ext-tests are always taken against \(k[0]\) (Appendix~C): \(\Ext^1(\Rfun(C_\tau F),k)=0\).
When windowed energy summaries are referenced elsewhere, the exponent is uniformly \(\alpha>0\) (default \(\alpha=1\)).
Update monotonicity follows the global rule: \emph{deletion-type} updates are non-increasing for windowed energies and spectral tails after truncation, whereas \emph{inclusion-type} updates are stable (non-expansive); see Appendix~E.
Type labels follow the global convention \emph{Type~I–II / Type~III / Type~IV}.



% =========================
\section*{Appendix E. Spectral Indicators: Monotonicity, Stability, Counterexamples [Proof/Spec] (reinforced)}
% =========================
\addcontentsline{toc}{section}{Appendix E. Spectral Indicators: Monotonicity, Stability, Counterexamples}
\refstepcounter{section} % stabilize cross-references with an unnumbered section

% Listings (for YAML)
\definecolor{lightgray}{gray}{0.95}
\lstset{
  basicstyle=\ttfamily\small,
  backgroundcolor=\color{lightgray},
  frame=single,
  columns=fullflexible,
  keepspaces=true,
  showstringspaces=false,
  upquote=true,
  breaklines=true
}

For \(\tau>0\), define the \emph{clipped spectrum}
\(\mathrm{clip}_\tau(H):=(\min\{\lambda_j(H),\tau\})_{j=1}^n\), the \emph{clipped sum}
\[
S^{\le \tau}(H):=\sum_{j=1}^n \min\{\lambda_j(H),\tau\},
\]
and the \emph{sub-threshold deficit}
\[
D^{<\tau}(H)\ :=\ \sum_{j=1}^n (\tau-\lambda_j(H))_+,\qquad x_+:=\max\{x,0\}.
\]
We use the operator norm \(\|\cdot\|_{\mathrm{op}}\) and Frobenius norm \(\|\cdot\|_{\mathrm{fro}}\).
All references to filtered colimits follow the scope rule in Appendix~A, Remark~\ref{A:rk:filtered-colimits}.
Cross-module conventions (used globally): Ext-tests are taken against \(k[0]\) (Appendix~C), i.e.\ \(\Ext^1(\mathcal{R}(C_\tau F),k)=0\); energy exponents are uniform \(\alpha>0\) (default \(\alpha=1\)); type labels use \emph{Type~I--II / Type~III / Type~IV}.
When we refer to persistence/filtered complexes, equalities are at the persistence layer (strict in \(\Pers^{\mathrm{ft}}_k\)); filtered–complex claims are \emph{up to f.q.i.}
The \(V\)-enrichment policy follows Appendix~A, Remark~\ref{rem:V-enriched}: metrics/aggregations do not alter the algebraic backbone.

\medskip
\noindent\textbf{Deletion vs.\ inclusion.}
When \(H\) arises by restricting admissible degrees of freedom, imposing Dirichlet constraints, eliminating internal dofs by shorting (Schur complement), or taking principal submatrices, we call conclusions \emph{deletion-type}.
When \(H\) is obtained by adding degrees of freedom, couplings, or enlarging a domain, we call them \emph{inclusion-type}.
Deletion-type updates admit one-sided monotonicity; inclusion-type updates admit only stability (non-expansive), unless additional order hypotheses are imposed. \emph{We label all inclusion-type claims as “non-expansive only.”}

\subsection*{E.0. Scope and window policy}

All spectral audits in this appendix are \emph{windowed} and performed \emph{after} collapse on the B-side single layer.
Concretely, comparisons follow the mandatory order:
\[
\boxed{\ \text{for each } t\ \Longrightarrow\ \text{apply } \mathbf{P}_i \ \Longrightarrow\ \text{apply } \mathbf{T}_\tau \ \Longrightarrow\ \text{compare in }\Pers^{\mathrm{ft}}_k\ }.
\]
When a spectral window \([a,b]\) with bin width \(\beta>0\) is used, bins are \emph{half-open, right-attribution}
\(I_r=[a+r\beta,a+(r+1)\beta)\), and eigenvalues at a right boundary are counted in the next bin.
Underflows/overflows are recorded.
This policy ensures reproducibility and compatibility with Overlap Gate and B–Gate\(^{+}\) (Appendix~G; Chapter~1).

\subsection*{E.1. Deletion-type monotonicity (principal/Dirichlet, Schur complement, Loewner)}

\begin{proposition}[Principal/Dirichlet restriction: interlacing and counting]\label{E:prop:dirichlet}
Let \(A\in\RR^{n\times n}\) be Hermitian and \(B\) a principal \((n-1)\times(n-1)\) submatrix (obtained, e.g., by pinning a coordinate—Dirichlet restriction).
Then Cauchy interlacing holds:
\[
\lambda_1(A)\ \le\ \lambda_1(B)\ \le\ \lambda_2(A)\ \le\ \cdots\ \le\ \lambda_{n-1}(B)\ \le\ \lambda_n(A).
\]
In particular, for every \(\theta\in\RR\),
\(N_\theta(B)\ \le\ N_\theta(A).\)
\end{proposition}

\begin{proposition}[Schur complement (shorting) monotonicity]\label{E:prop:schur}
Partition \(M=\begin{psmallmatrix}A & B\\ B^\top & C\end{psmallmatrix}\succeq 0\) with \(C\succ 0\) and form the Schur complement \(S:=A-BC^{-1}B^\top\).
Then \(S\preceq A\).
Consequently, for all \(j\) and all \(\theta\ge 0\),
\(\lambda_j(S)\ \le\ \lambda_j(A),\quad N_\theta(S)\ \le\ N_\theta(A).\)
\end{proposition}

\begin{proposition}[Loewner-order monotonicity]\label{E:prop:loewner}
If \(0\preceq A\preceq B\) (Loewner order), then for each \(j\),
\(\lambda_j(A)\le \lambda_j(B)\)
and, for every \(\theta\ge 0\), \(N_\theta(A)\le N_\theta(B)\).
\end{proposition}

\begin{remark}[Heat traces and spectral tails]
For PSD matrices and \(t>0\), the heat trace \(\mathrm{HT}(t;H)=\sum_j e^{-t\lambda_j(H)}\) satisfies:
if \(A'\preceq A\) (contraction), then \(\mathrm{HT}(t;A')\ge \mathrm{HT}(t;A)\); if \(A'\succeq A\) (hardening), then \(\mathrm{HT}(t;A')\le \mathrm{HT}(t;A)\).
Likewise, for spectral tails \(\mathrm{ST}_\beta(H)=\sum_{j\ge 1}\lambda_j(H)^{-\beta}\) with \(\beta>0\), one has \(\mathrm{ST}_\beta(A')\ge \mathrm{ST}_\beta(A)\) under \(A'\preceq A\) and the reverse inequality under \(A'\succeq A\), provided all \(\lambda_j>0\).
In practice, tails are computed on \(L(C_\tau F)\) with zero modes removed or handled by pseudoinverses; see Appendix~G.
\end{remark}

\begin{corollary}[Conservative averaging]\label{E:cor:avg}
If \(A_1,\dots,A_m\succeq 0\) satisfy \(A_\ell\preceq A\) for all \(\ell\), then for any convex combination \(\bar A:=\sum_\ell w_\ell A_\ell\) with \(w_\ell\ge 0\), \(\sum_\ell w_\ell=1\),
\(\bar A\ \preceq\ A\).
Therefore \(\lambda_j(\bar A)\le \lambda_j(A)\), \(N_\theta(\bar A)\le N_\theta(A)\) for \(\theta\ge 0\), and the heat trace/tail inequalities above apply.
\end{corollary}

\begin{remark}[Orientation for deletions]
Two Loewner orientations occur in practice.
\emph{Contractions} (e.g.\ Schur complements, Kron reduction) produce \(A'\preceq A\); \emph{hardening} operations (e.g.\ some PDE Dirichlet comparisons across different media) may yield \(A'\succeq A\).
We state deletion-type monotonicities in both orientations.
\end{remark}

\subsection*{E.2. Inclusion-type counterexamples}

Deletion-type monotonicity does \emph{not} extend naively to inclusion-type operations without additional order hypotheses.

\begin{example}[Neumann/domain inclusion reverses direction]\label{E:ex:neumann}
For the Neumann Laplacian on an interval, enlarging the domain decreases the nonzero eigenvalues:
on \([0,L]\), the first nonzero Neumann eigenvalue is \((\pi/L)^2\), so passing \(L:1\to 2\) reduces it from \(\pi^2\) to \((\pi/2)^2\).
Thus any “inclusion \(\Rightarrow\) increase” heuristic fails under Neumann-type constraints.
\end{example}

\begin{example}[Indefinite coupling can move eigenvalues both ways]\label{E:ex:indef}
Let \(A=I_2=\mathrm{diag}(1,1)\) and
\(B=\begin{psmallmatrix}1 & M\\ M & 1\end{psmallmatrix}\) with \(M>1\).
Then \(B\) has eigenvalues \(1-M\) and \(1+M\), so for \(\theta=0\),
\(N_\theta(B)=1<N_\theta(A)=2\), while the top eigenvalue \(\lambda_2\) increases.
Without a Loewner relation (\(B-A\) indefinite), no monotone law survives.
\end{example}

\begin{example}[Principal extension lacks a fixed direction]\label{E:ex:principal-extend}
Let \(B=[0]\) (eigenvalue \(0\)) and
\(A=\begin{psmallmatrix}0 & t\\ t & 0\end{psmallmatrix}\) with \(t\neq 0\).
Going from \(B\) to \(A\) (adding one dof and a coupling) produces eigenvalues \(-|t|\) and \(|t|\):
the maximum increases to \(|t|\), but the minimum decreases to \(-|t|\).
Hence no uniform increase/decrease holds under inclusion.
\end{example}

These examples justify restricting monotone claims to the deletion/Loewner settings formalized in §E.1.

\subsection*{E.3. Continuity, stability, and truncated functionals (with $V$-metric lift)}

Write \(N_{\theta\pm 0}(A)\) for the left/right limits at \(\theta\) (no jump unless \(\theta\) is an eigenvalue).

\begin{proposition}[Weyl and Hoffman–Wielandt]\label{E:prop:weyl}
For Hermitian \(A,B\in\RR^{n\times n}\),
\[
\max_{1\le j\le n}\,|\lambda_j(A)-\lambda_j(B)|\ \le\ \|A-B\|_{\mathrm{op}},\qquad
\Big(\sum_{j=1}^n |\lambda_j(A)-\lambda_j(B)|^2\Big)^{1/2}\ \le\ \|A-B\|_{\mathrm{fro}}.
\]
Hence \(A\mapsto(\lambda_1(A),\dots,\lambda_n(A))\) is \(1\)-Lipschitz from \((\|\cdot\|_{\mathrm{op}})\) into \((\RR^n,\|\cdot\|_\infty)\).
\end{proposition}

\begin{corollary}[Lipschitz stability of clipped spectra]\label{E:cor:clip}
For any \(\tau>0\) and Hermitian \(A,B\),
\[
\sum_{j=1}^n\Big|\min\{\lambda_j(A),\tau\}-\min\{\lambda_j(B),\tau\}\Big|
\ \le\ \sum_{j=1}^n|\lambda_j(A)-\lambda_j(B)|
\ \le\ \sqrt{n}\,\|A-B\|_{\mathrm{fro}}
\ \le\ n\,\|A-B\|_{\mathrm{op}}.
\]
Consequently, \(S^{\le \tau}\) is \(\sqrt{n}\)–Lipschitz in \(\|\cdot\|_{\mathrm{fro}}\) and \(n\)–Lipschitz in \(\|\cdot\|_{\mathrm{op}}\).
\end{corollary}

\begin{proposition}[Semicontinuity of counting indicators]\label{E:prop:count-semi}
If \(A_m\to A\) in operator norm and \(\theta\) is not an eigenvalue of \(A\), then \(N_\theta(A_m)=N_\theta(A)\) for all large \(m\) (local constancy).
In general,
\[
\limsup_{m\to\infty} N_\theta(A_m)\ \le\ N_{\theta-0}(A),\qquad
\liminf_{m\to\infty} N_\theta(A_m)\ \ge\ N_{\theta+0}(A).
\]
\end{proposition}

\begin{proposition}[Truncated functionals: monotonicity and stability]\label{E:prop:trunc-func}
Fix \(\tau>0\).
For an \(n\times n\) positive semidefinite (PSD) matrix \(A\), set
\[
S^{\le\tau}(A):=\sum_{j=1}^n \min\{\lambda_j(A),\tau\},\qquad
D^{<\tau}(A):=\sum_{j=1}^n (\tau-\lambda_j(A))_{+},
\]
and \(N_\theta(A):=\#\{j:\lambda_j(A)\ge \theta\}\) for \(\theta\ge 0\).
Then:
\begin{enumerate}[leftmargin=1.15em,label=(\arabic*)]
\item \emph{Deletion; Loewner contraction \(A'\preceq A\).}
For all \(j\), \(\lambda_j(A')\le \lambda_j(A)\), hence \(N_\theta(A')\le N_\theta(A)\) for every \(\theta\ge 0\), and
\[
S^{\le\tau}(A')\le S^{\le\tau}(A),\qquad D^{<\tau}(A')\ge D^{<\tau}(A).
\]
\item \emph{Deletion; Loewner hardening \(A'\succeq A\).}
All inequalities in \emph{(1)} reverse:
\[
\lambda_j(A')\ge \lambda_j(A),\quad N_\theta(A')\ge N_\theta(A)\ \ (\theta\ge 0),\quad
S^{\le\tau}(A')\ge S^{\le\tau}(A),\quad D^{<\tau}(A')\le D^{<\tau}(A).
\]
\item \emph{Lipschitz stability.} For any Hermitian \(A,B\),
\[
\bigl|D^{<\tau}(A)-D^{<\tau}(B)\bigr|
\ \le \sum_{j=1}^n \bigl|\lambda_j(A)-\lambda_j(B)\bigr|
\ \le \sqrt{n}\,\|A-B\|_{\mathrm{fro}}
\ \le n\,\|A-B\|_{\mathrm{op}}.
\]
\end{enumerate}
\end{proposition}

\begin{theorem}[Deletion-type monotonicity and $V$-Lipschitz after collapse]\label{thm:spec-v-lip}
Let \(V\) be a commutative quantale and endow the per-window spectral dashboard (counts \(N_\theta\), clipped sums \(S^{\le\tau}\), deficits \(D^{<\tau}\), heat traces, tails, cumulative profiles \(C_r\)) with a Lawvere \(V\)-metric aggregator (Appendix~A, Remark~\ref{rem:V-enriched}). Then, measured \emph{after} \(C_\tau\):
\begin{enumerate}[leftmargin=1.15em,label=(\alph*)]
\item For every deletion-type update \(U\), the dashboard is \emph{componentwise monotone} in the appropriate Loewner orientation as in Propositions~\ref{E:prop:dirichlet}–\ref{E:prop:trunc-func}.
\item For every \(\varepsilon\)-continuation with \(\|A'-A\|_{\mathrm{op}}\le\varepsilon\), the dashboard is \(V\)\emph{-1-Lipschitz}: each component changes by at most its scalar Lipschitz bound (Weyl/Hoffman–Wielandt), and any \(V\)-aggregation preserves 1-Lipschitzness.
\end{enumerate}
Spectral indicators are \emph{auxiliary}: they never serve as sole gate criteria.
\end{theorem}

\begin{remark}[Heat traces and spectral tails: stability]
Let \(\mathrm{HT}(t;H)=\sum_j e^{-t\lambda_j(H)}\), \(\mathrm{ST}_\beta(H)=\sum_j \lambda_j(H)^{-\beta}\) with zero modes removed.
If \(\|A-B\|_{\mathrm{op}}\le\varepsilon\), then
\(|\mathrm{HT}(t;A)-\mathrm{HT}(t;B)|\le t\,e^{-t\lambda_{\min}^+}\sum_j |\lambda_j(A)-\lambda_j(B)|\),
yielding a Lipschitz-type bound in terms of \(\|\cdot\|_{\mathrm{fro}}\) (similar for \(\|\cdot\|_{\mathrm{op}}\) with an \(n\) factor).
On windows with \(\lambda_j\ge \lambda_{\min}^+>0\), \(x\mapsto x^{-\beta}\) is Lipschitz on \([\lambda_{\min}^+,\infty)\), giving analogous bounds for \(\mathrm{ST}_\beta\).
In practice, tails/heat traces are evaluated on \(L(C_\tau F)\) under a fixed normalization (Appendix~G).
\end{remark}

% -------------------------
% New: Safe low-pass post-processing (even kernel, unit mass, τ-scale) + 1-Lipschitz verification
% -------------------------
\subsection*{E.3.5. Safe low-pass post-processing and 1-Lipschitz verification (final)}

We record two safe low-pass mechanisms used as \emph{post}–collapse auxiliaries. Both are \emph{non-expansive} and compatible with the dashboard above. They never replace B–Gate\(^{+}\).

\paragraph{(A) Temporal/index low-pass on a sequence \(\{A_j\}_j\) of PSD operators.}
Fix a finitely supported kernel \(h^{(\tau)}:\ZZ\to[0,1]\) with:
\[
\begin{aligned}
\text{\textbf{(LP1) evenness:}} &\quad h^{(\tau)}[r]=h^{(\tau)}[-r],\\
\text{\textbf{(LP2) unit mass:}} &\quad \sum_{r\in\ZZ} h^{(\tau)}[r]=1,\\
\text{\textbf{(LP3) scale:}} &\quad \mathrm{supp}\!\big(h^{(\tau)}\big)\subseteq[-R(\tau),R(\tau)].
\end{aligned}
\]
Define the smoothed sequence \(\widetilde A_j:=\sum_{r} h^{(\tau)}[r]\;A_{j-r}\) (finite sum).
Then:
\begin{enumerate}[leftmargin=1.15em,label=(T\arabic*)]
\item \emph{PSD and Loewner convexity.} \(\widetilde A_j\succeq 0\) and, if \(A_{j-r}\preceq A^\star\) for all \(r\), then \(\widetilde A_j\preceq A^\star\) (Cor.~\ref{E:cor:avg}).
\item \emph{Non-expansive in \(\|\cdot\|_{\mathrm{op}}\).} For any sequences \(\{A_j\},\{B_j\}\),
\(\|\widetilde A_j-\widetilde B_j\|_{\mathrm{op}}\le \sum_r h^{(\tau)}[r]\;\|A_{j-r}-B_{j-r}\|_{\mathrm{op}}\le \sup_m\|A_m-B_m\|_{\mathrm{op}}.\)
Thus the temporal low-pass is \(1\)-Lipschitz in \(\ell_\infty\)-in-\(j\) operator norm.
\item \emph{Dashboard compatibility.} Apply the dashboard to each \(\widetilde A_j\) after \(C_\tau\). Deletion-type monotonicity is preserved when the original updates are deletion-type and the kernel averages only within the same update class; otherwise only non-expansiveness is claimed.
\end{enumerate}

\paragraph{(B) Spectral low-pass via functional calculus.}
Let \(A\succeq 0\) and choose one of the \emph{safe families} (parameter \(\tau>0\)):

\smallskip
\(\bullet\) \emph{Heat kernel: } \(f_\tau(\lambda):=e^{-\tau\lambda}\).
Then
\[
f_\tau(A)-f_\tau(B)=-\!\int_0^\tau e^{-(\tau-s)A}\,(A-B)\,e^{-sB}\,ds,\quad
\Rightarrow\quad \|f_\tau(A)-f_\tau(B)\|_{\mathrm{op}}\le \tau\|A-B\|_{\mathrm{op}}.
\]
Thus \(\tau\le 1\Rightarrow\) operator \(1\)-Lipschitz.

\smallskip
\(\bullet\) \emph{Resolvent: } \(r_\tau(\lambda):=(1+\lambda/\tau)^{-1}\).
By the resolvent identity,
\[
r_\tau(A)-r_\tau(B)=(I+A/\tau)^{-1}\,\frac{B-A}{\tau}\,(I+B/\tau)^{-1},
\]
and \(\|(I+A/\tau)^{-1}\|_{\mathrm{op}},\|(I+B/\tau)^{-1}\|_{\mathrm{op}}\le 1\).
Hence
\(\|r_\tau(A)-r_\tau(B)\|_{\mathrm{op}}\le \frac{1}{\tau}\|A-B\|_{\mathrm{op}}\),
so \(\tau\ge 1\Rightarrow\) operator \(1\)-Lipschitz.

\smallskip
Both filters satisfy \(0\le f_\tau(\lambda),r_\tau(\lambda)\le 1\) and \(f_\tau(0)=r_\tau(0)=1\) (unit mass at DC), and they are operator-monotone decreasing in \(\lambda\).
When applied to \(L(C_\tau F)\), the resulting operators keep the dashboard non-expansive; if a deletion update satisfies a Loewner orientation, the filtered indicators inherit the same orientation.

\paragraph{(C) 1-Lipschitz verification checklist (runtime, per window).}
\begin{enumerate}[leftmargin=1.15em,label=(V\arabic*)]
\item \emph{Temporal low-pass:} check evenness (LP1), unit mass (LP2), finite support (LP3). Report \texttt{lp.kind="temporal"}, \texttt{lp.mass=1.0}, \texttt{lp.support} and \texttt{lp.even=true}. Non-expansiveness holds with constant \(1\).
\item \emph{Heat filter:} record \(\tau\) and ensure \(\tau\le 1\) for \(1\)-Lipschitz; otherwise log \texttt{lip\_const=\(\tau\)} and use it in the \(V\)-aggregator.
\item \emph{Resolvent filter:} ensure \(\tau\ge 1\) for \(1\)-Lipschitz; otherwise log \texttt{lip\_const=\(1/\tau\)}.
\item \emph{Inclusion-type steps:} tag \texttt{inclusion="non-expansive-only"}; no monotone claim is emitted.
\end{enumerate}

\subsection*{(D) Manifest snippet (auditable).}
\begin{lstlisting}[language=,caption={run.yaml (spectral low-pass metadata)}]
spectral_post:
  lowpass:
    mode: "temporal"            # or "heat" | "resolvent"
    kernel:
      even: true
      mass: 1.0
      support: [-2, 2]          # indices r with h[r] ≠ 0
      taps: [0.1, 0.2, 0.4, 0.2, 0.1]
    lipschitz:
      constant: 1.0
      verified: true
    inclusion_policy: "non-expansive-only"
# If mode is "heat":
#   param_tau: 0.75
#   lipschitz.constant: 0.75
# If mode is "resolvent":
#   param_tau: 2.0
#   lipschitz.constant: 0.5
\end{lstlisting}

\subsection*{E.4. Auxiliary spectral bars (aux-bars): definition, stability, and policy [Spec]}

We formalize \emph{auxiliary spectral bars} as diagnostics alongside persistence. They never replace B–Gate\(^{+}\) and are used only as auxiliary evidence.

\paragraph{E.4.1. Binning and endpoint convention.}
Fix a spectral window \([a,b]\) and a bin width \(\beta>0\).
Let \(R:=\lfloor (b-a)/\beta\rfloor\).
Define half-open, right-attribution bins
\(I_r=[a+r\beta,a+(r+1)\beta)\)
for \(r=0,1,\dots,R-1\).
An eigenvalue at the bin’s right boundary is counted in the next bin.
Record underflow \(U(H):=\#\{j:\lambda_j(H)<a\}\) and overflow \(O(H):=\#\{j:\lambda_j(H)\ge b\}\).
For a Hermitian \(H\), define the bin occupancy
\(E_r(H):=\#\{j:\lambda_j(H)\in I_r\}\)
and the cumulative (upper-tail) profile
\(C_r(H):=\sum_{s=r}^{R-1}E_s(H)=N_{a+r\beta}(H)-O(H)\).

\paragraph{E.4.2. Aux-bars across an index (time/tower).}
Let \((H_j)_{j\in J}\) be a sequence (time or tower).
For fixed \(r\), the set \(\{j:E_r(H_j)>0\}\) decomposes into maximal consecutive runs \(J_{r,\ell}\).
Each run \(J_{r,\ell}\) defines an aux-bar \((r,J_{r,\ell})\) with lifetime \(|J_{r,\ell}|\) (or a rescaled duration).
We log \(\mathrm{aux\text{-}count}=\sum_{r,\ell}1\), \(\mathrm{aux\text{-}mass}=\sum_r E_r(H_j)\) (per index \(j\)), and \(\mathrm{active\ bins}=\#\{r:E_r(H_j)>0\}\).

\paragraph{E.4.3. Monotonicity/stability.}
\begin{proposition}[Cumulative-profile monotonicity under Loewner]\label{E:prop:aux-cum-mono}
If \(A'\preceq A\) (PSD contraction) or \(A'\) is a principal/Dirichlet restriction of \(A\), then for every \(r\),
\(C_r(A')\le C_r(A).\)
\end{proposition}

\begin{proposition}[Cumulative-profile stability]\label{E:prop:aux-cum-stab}
If \(\|A-B\|_{\mathrm{op}}\le \varepsilon\) and \(q:=\lceil \varepsilon/\beta\rceil\), then for all \(r\),
\(C_{r+q}(B)\le C_r(A)\le C_{\max\{0,r-q\}}(B).\)
In particular, if \(\varepsilon<\beta\), the cumulative profile can shift by at most one bin.
\end{proposition}

\begin{corollary}[Definable windows: finiteness and piecewise constancy]\label{E:cor:definable}
On an o\hbox{-}minimal definable spectral window with finite Leray/Čech depth (Appendix~H/J), the sequences \(r\mapsto E_r(H_j)\) and \(r\mapsto C_r(H_j)\) admit only finitely many bin-transition events per window and are piecewise constant in \(j\).
Aux-bars are therefore finite in number per window and auditable.
\end{corollary}

\begin{remark}[Policy]
- Deletion-type steps: enforce monotonicity on the \emph{cumulative} profile \(C_r\). Per-bin occupancies and lifetimes are diagnostics only.

- \(\varepsilon\)-continuations: with \(\|A_{j+1}-A_j\|_{\mathrm{op}}\le \varepsilon\), declare stability up to \(\pm q=\lceil \varepsilon/\beta\rceil\) bin shifts; record \texttt{eps\_cont\_shift\_bins} in the manifest.

- Inclusion-type steps: claim no monotonicity; only stability bounds are used (\emph{non-expansive only}).

- Under/overflow must be logged. Optional conservative rules: \(O=0\), \(C_{R-1}=0\) may be enforced as policy (not a proof obligation).
\end{remark}

\paragraph{E.4.4. Reproducibility fields.}
The run manifest \texttt{run.yaml} should include (Appendix~G):
\begin{itemize}[leftmargin=1.25em]
  \item \texttt{spectral.range} \([a,b]\), \texttt{bin\_width} \(\beta\), \texttt{bins} \(R\), endpoint policy \texttt{half-open/right-attribution};
  \item \texttt{underflow}/\texttt{overflow} per index \(j\);
  \item \texttt{cum\_profile}: the sequence \(C_r(H_j)\) per \(j\);
  \item \texttt{aux\_bars} (optional): list of runs \((r,J_{r,\ell})\) with lifetimes;
  \item \texttt{eps\_cont\_bound}: \(\varepsilon\) and derived \texttt{eps\_cont\_shift\_bins} \(= \lceil \varepsilon/\beta\rceil\);
  \item \texttt{spectral\_policy.order}: \texttt{"ascending"}; \texttt{spectral\_policy.norm}: \texttt{"op"} or \texttt{"fro"}; bounds \texttt{lambda\_min}, \texttt{lambda\_max}, optional \texttt{lip\_tol}.
\end{itemize}

\subsection*{E.5. Implementation and reproducibility: JSON/HDF5 schemas}

\subsection*{JSON layout (mandatory fields).}
\begin{lstlisting}[language=,caption={Minimal spec.json layout (one run, multiple operators)}]
{
  "meta": {
    "schema_version": "2025-03-15",
    "eigen_units": "dimensionless",
    "order": "ascending",
    "sorted": true,
    "norm": "op",               // or "fro"
    "Ntheta_convention": { "left": "N_{θ-0}", "right": "N_{θ+0}" },
    "window": { "range": [0.0, 2.0], "semantics": "closed" },
    "clip_tau": 1.0,
    "tol_eig": 1e-8,
    "aux_policy": { "bin": 0.02, "right_attribution": true },
    "coverage_check": { "thetas_in_window": true },
    "links": { "run_id": "...", "run_yaml_hash": "sha256:..." }
  },
  "operators": [
    {
      "id": "sha256:...A",
      "kind": "laplacian_dirichlet",
      "n": 500,
      "spectrum": { "eigs": [0.10, 0.12, 0.45, ...] },  // ascending
      "clip":     { "tau": 1.0, "sum": 37.219, "deficit": 12.004 },
      "underflow": 0, "overflow": 0
    },
    {
      "id": "sha256:...B",
      "kind": "principal_submatrix",
      "parent": "sha256:...A",
      "N_theta": [
        { "theta": 0.20, "left": 17, "right": 16 },
        { "theta": 0.50, "left": 10, "right": 10 }
      ],
      "cum_profile": [ 13, 11, 8, 2, 0 ],
      "underflow": 2, "overflow": 0,
      "monotonicity": { "type": "deletion", "passed": true }
    }
  ],
  "lowpass": {
    "mode": "heat",           // "temporal" | "resolvent"
    "param_tau": 0.75,
    "lipschitz_constant": 0.75,
    "verified": true,
    "inclusion_policy": "non-expansive-only"
  },
  "hash": "sha256:...spec"
}
\end{lstlisting}

\paragraph{HDF5 layout (canonical).}
\begin{itemize}[leftmargin=1.25em]
  \item Datasets:
  \texttt{/spec/ops/\{id\}/eig} (float64 ascending),
  \texttt{/spec/ops/\{id\}/clip/sum} (float64),
  \texttt{/spec/ops/\{id\}/clip/deficit} (float64),
  \texttt{/spec/ops/\{id\}/Ntheta/theta,left,right} (parallel arrays),
  \texttt{/spec/ops/\{id\}/cum\_profile} (int32),
  \texttt{/spec/ops/\{id\}/underflow} (int32),
  \texttt{/spec/ops/\{id\}/overflow} (int32),
  \texttt{/spec/lowpass/mode} (fixed string),
  \texttt{/spec/lowpass/param\_tau} (float64),
  \texttt{/spec/lowpass/lipschitz\_constant} (float64),
  \texttt{/spec/lowpass/verified} (bool).
  \item Attributes: \texttt{order="ascending"}, \texttt{norm="op"|"fro"}, \texttt{eigen\_units}, \texttt{tol\_eig}, \texttt{schema\_version}, bin policy, and canonical HDF5 flags (\texttt{track\_times=false}, UTF-8 fixed strings, fixed chunking); see Appendix~G.
\end{itemize}

\subsection*{E.6. Tests and operational checklist}

\paragraph{Core tests.}
\begin{enumerate}[leftmargin=1.25em]
  \item \textbf{T2 (Deletion-type monotonicity).}
  For principal/Dirichlet or Schur complements, verify \(N_\theta\), \(S^{\le\tau}\), \(D^{<\tau}\), heat trace, and tails satisfy the monotone direction consistent with the Loewner orientation. Log \texttt{monotonicity.passed=true}.
  \item \textbf{T1 (\(\varepsilon\)-continuation stability).}
  Given \(\|A_{j+1}-A_j\|_{\mathrm{op}}\le \varepsilon\), validate the bin-shift bounds in Proposition~\ref{E:prop:aux-cum-stab} and Lipschitz bounds for \(S^{\le\tau}\), \(D^{<\tau}\).
  \item \textbf{T9 (Coverage).}
  Confirm that all \(\theta\)-queries, bin windows, and eigenvalues fall within declared windows; if not, log underflow/overflow and set \texttt{coverage\_check.*=false} with a justification.
  \item \textbf{T12 (Low-pass safety).}
  For temporal low-pass, check (LP1)–(LP3) and \(\sum h=1\); for heat/resolvent, assert the scale choices yield operator Lipschitz constant \(\le 1\) (or record the constant if \(>1\)) and verify against \(\|A-B\|_{\mathrm{op}}\).
\end{enumerate}

\paragraph{Operational checklist (per window).}
\begin{itemize}[leftmargin=1.25em]
  \item Fix spectral policy: \texttt{order="ascending"}, \texttt{norm} selection, bin width \(\beta\), spectral window \([a,b]\).
  \item Compute spectra on \(L(C_\tau F)\) and clip at the same \(\tau\) used by persistence.
  \item Optionally apply a \emph{safe} low-pass (temporal, heat, resolvent) with recorded scale \(\tau\) and verified Lipschitz constant.
  \item Log underflow/overflow, cumulative profiles \(C_r\), optional aux-bars with lifetimes (diagnostic).
  \item For deletion-type steps, assert \(C_r\) monotonicity; for \(\varepsilon\)-continuations, assert bin-shift stability with \(\lceil \varepsilon/\beta\rceil\); for inclusion-type steps, \emph{non-expansive only}.
\end{itemize}

\subsection*{E.7. Completion note}

\begin{remark}[No further supplementation required]
This appendix provides a complete, IMRN/AiM-ready treatment of spectral indicators consistent with the v16.0 guard-rails: (i) deletion-type monotonicities (principal/Dirichlet, Schur, Loewner) for \(N_\theta\), \(S^{\le\tau}\), \(D^{<\tau}\), heat traces, and tails; (ii) inclusion-type counterexamples; (iii) Lipschitz stability via Weyl/Hoffman–Wielandt and induced bounds for clipped sums/deficits, heat traces, and tails; (iv) a windowed, half-open binning policy with cumulative-profile monotonicity and stability under \(\varepsilon\)-continuations; (v) $V$-metric reinforcement (Theorem~\ref{thm:spec-v-lip}) ensuring deletion-type monotonicity and $V$-1-Lipschitz stability after collapse; (vi) \textbf{safe low-pass post-processing} with \emph{even kernel, unit mass, \(\tau\)-scale} (temporal) and \emph{operator-Lipschitz} filters (heat/resolvent), together with an auditable \(1\)-Lipschitz verification checklist; and (vii) reproducibility and canonical schemas (JSON/HDF5) with a minimal test suite (T1/T2/T9/T12).
All claims are made after collapse on the B-side single layer, per-window, and integrate with δ-ledger accounting, Overlap Gate, and B–Gate\(^{+}\) elsewhere in the manuscript. No further supplementation is required for operational deployment or audit.
\end{remark}



% =========================
\section*{Appendix F. Formalization Sketch (Lean/Coq) [Spec] (reinforced)}
% =========================
\addcontentsline{toc}{section}{Appendix F. Formalization Sketch (Lean/Coq)}
\refstepcounter{section} % stabilize cross-references with an unnumbered section

% ---- NEW: interface stub (inserted at the very beginning) ----
\begin{declaration}[Lean/Coq stubs]\label{stub:formal}
Expose minimal, reusable interfaces:
\texttt{pers\_Ttau\_exact}, \texttt{pers\_Ttau\_lipschitz}, \texttt{Ctau\_lift}, \texttt{Ctau\_colim}, \texttt{Ctau\_pullback}, \texttt{mu\_nu\_vanish}, \texttt{PH1\_to\_Ext1\_under\_B}, \texttt{delta\_pipeline\_additivity}, enriched metrics \texttt{V\_metric\_shift}, and the canonical normal form \texttt{cnf\_after\_Ttau} (interval/Smith–type decomposition for kernels/cokernels).
Equalities are confined to the persistence layer; filtered-level facts are packaged ``up to f.q.i.''.
AWFS stubs, o\hbox{-}minimal Čech, and Iwasawa-style control catalysts are provided as portable axioms with clear call sites.
\end{declaration}

This appendix provides a fully integrated, implementation-oriented \emph{Spec} for mechanizing the core claims of
Appendices~A–E in Lean/Coq with a module decomposition tailored for a minimal, portable ``mini-library.''
The categorical spine consists of the \emph{Serre localization} and the \emph{reflector} \(\mathbf{T}_\tau\) (exact, idempotent),
its \emph{$1$-Lipschitz} property on barcodes, tower diagnostics \((\mu,\nu)\) via a comparison map \(\phi_{i,\tau}\),
and the edge identification supporting the one-way bridge \(\mathrm{PH}_1\Rightarrow\Ext^1\).
We work in the \emph{constructible} (p.f.d.) range and adhere to the \emph{filtered colimit scope rule}
(Appendix~A, Remark~\ref{A:rk:filtered-colimits}).
Cross-module conventions: \(\Ext\)-tests are always against \(k[0]\) (Appendix~C), i.e.\ \(\Ext^1(\mathcal{R}(C_\tau F),k)=0\);
the energy exponent is globally \(\alpha>0\) (default \(\alpha=1\));
type labels use \emph{Type I--II / Type III / Type IV} for tower diagnostics.
Spectral monotonicity is invoked only for \emph{deletion-type} operations; inclusion-type operations are used solely with stability bounds (Appendix~E).

\paragraph{Refereeing style (IMRN/AiM).}
Statements are modular with explicit hypotheses and reusable APIs.
All constructions remain within abelian categories, exact localizations, and derived categories with bounded \(t\)-structures.
Proof obligations used in the code stubs are isolated and cited to Appendices~A–E; replacing \texttt{admit}/\texttt{Axiom} by library lemmas yields a fully checked artifact.

% -------------------------
\subsection*{F.0. Reading guide and module map}
% -------------------------

We split the development into five modules and three thin catalysts:

\begin{itemize}
  \item \textbf{AK.Core} (F.1–F.3, F.5–F.6): \(\Pers^{\mathrm{ft}}_k\), the Serre subcategory \(\mathsf{E}_\tau\), the reflector \(\mathbf{T}_\tau\) (exact, idempotent), interleaving/shift calculus and \(1\)-Lipschitz, the collapse on filtered complexes \(C_\tau\), and invariance under filtered quasi-isomorphisms.
  \item \textbf{AK.LocalEquiv} (F.7): Window-local equivalences \(\mathrm{PH}\leftrightarrow \Ext\) under amplitude \(\le 1\), saturation/no-accumulation, and the tail-isomorphism for \(\phi_{i,\tau}\).
  \item \textbf{AK.Tower} (F.8): The comparison map \(\phi_{i,\tau}\), diagnostics \((\mu,\nu)\), functoriality, stability bands, direct sums, compositions, and cofinal invariance.
  \item \textbf{AK.Gluing} (F.9): The Overlap Gate (collapse-compatibility, A/B checks, Čech–\(\Ext^1\), stability bands) and MECE windowing glue to a global verdict.
  \item \textbf{AK.Spectral} (F.10): The spectral operators \(S^{\le\tau}\), \(D^{<\tau}\), and \(C_r\), monotonicity and Lipschitz-type bounds compatible with \(\mathbf{T}_\tau\).
  \item \textbf{Catalysts (thin):} \emph{V-enriched metric/shift} (F.A), \emph{AWFS skeleton} (F.B) for functorial factorization used by \(C_\tau\), and \emph{o\hbox{-}minimal Čech + Iwasawa control} (F.C) to streamline Overlap Gate.
\end{itemize}

Lean and Coq stubs are provided per module. Test fixtures include T7 (saturation gate), T10 (A/B), T13 (\(\delta\)-budget).

% -------------------------
\subsection*{F.1. Environment and objects [Spec] (AK.Core)}
% -------------------------

Fix a field \(k\).
Let \(\mathsf{Vect}_k\) be the abelian category of finite-dimensional \(k\)-vector spaces and
\([\mathbb{R},\mathsf{Vect}_k]\) the functor category (index \((\mathbb{R},\le)\)).
Let \(\Pers^{\mathrm{ft}}_k\subset[\mathbb{R},\mathsf{Vect}_k]\) be the full subcategory of \emph{constructible}
persistence modules (barcodes locally finite on bounded windows).

Let \(\mathsf{FiltCh}(k)\) be filtered chain complexes of finite-dimensional \(k\)-spaces, bounded in homological degree,
with filtration-preserving maps. For \(i\in\mathbb{Z}\) write \(\mathbf{P}_i:\mathsf{FiltCh}(k)\to\Pers^{\mathrm{ft}}_k\)
for the degree–\(i\) persistence functor. The \(\tau\)-\emph{truncation} (collapse) functor
\(\mathbf{T}_\tau:\Pers^{\mathrm{ft}}_k\to\Pers^{\mathrm{ft}}_k\) is recalled from Appendix~A.

\begin{remark}[Generic fiber dimension and stabilization]\label{F:rk:generic-fiber}
We adopt Appendix~D, Remark~\ref{rem:D-generic-dim}. For \(M\in\Pers^{\mathrm{ft}}_k\), the \emph{generic fiber dimension}
is the multiplicity of the infinite interval \(I[0,\infty)\) in the barcode of \(M\); equivalently,
\[
\mathrm{gdim}(M)\ =\ \lim_{t\to +\infty}\dim_k M(t),
\]
which stabilizes in the constructible range.
After applying \(\mathbf{T}_\tau\), kernels and cokernels again lie in \(\Pers^{\mathrm{ft}}_k\), and \(\mathrm{gdim}\)
is computed there.
\end{remark}

\begin{specification}[Stabilization lemma for constructible modules]\label{F:spec:stabilize}
If \(M\in\Pers^{\mathrm{ft}}_k\), then there exist \(T_0\in\mathbb{R}\) and \(c\in\mathbb{N}\) such that
\(\dim_k M(t)=c=\mathrm{gdim}(M)\) for all \(t\ge T_0\).
\emph{Use:} define \(\mathrm{gdim}(M)\) by this stabilized value \(c\) and prove iso-invariance of \(\mathrm{gdim}\).
\end{specification}

% -------------------------
\subsection*{F.2. Serre subcategory and localization [Spec] (AK.Core)}
% -------------------------

Let \(\mathsf{E}_\tau\subset \Pers^{\mathrm{ft}}_k\) be the Serre subcategory generated by intervals of length \(\le\tau\).
By Appendix~A, \(\mathsf{E}_\tau\) is hereditary Serre and the inclusion
\(\iota_\tau:\mathsf{E}_\tau^\perp\hookrightarrow \Pers^{\mathrm{ft}}_k\) admits an exact left adjoint
\(\mathbf{T}_\tau:\Pers^{\mathrm{ft}}_k\to \mathsf{E}_\tau^\perp\) (reflector), inducing an equivalence
\[
\Pers^{\mathrm{ft}}_k/\mathsf{E}_\tau\ \simeq\ \mathsf{E}_\tau^\perp.
\]
Basic laws (API):
\[
\mathbf{T}_\tau\circ \mathbf{T}_\tau \cong \mathbf{T}_\tau,\qquad \mathbf{T}_\tau \dashv \iota_\tau .
\]
Moreover, \(\mathbf{T}_\tau\) is exact and preserves finite (co)limits (Appendix~A).

% -------------------------
\subsection*{F.3. Interleavings, shifts, and \(1\)-Lipschitz [Spec] (AK.Core)}
% -------------------------

The interleaving pseudometric \(d_{\mathrm{int}}\) on \(\Pers^{\mathrm{ft}}_k\) is implemented via shift functors
\(\mathrm{Shift}_\varepsilon\) and \(\varepsilon\)-interleavings. Appendix~A yields natural
isomorphisms \(\mathrm{Shift}_\varepsilon\circ \mathbf{T}_\tau\simeq \mathbf{T}_\tau\circ \mathrm{Shift}_\varepsilon\)
that transport interleavings, hence
\[
d_{\mathrm{int}}(\mathbf{T}_\tau M,\mathbf{T}_\tau N)\ \le\ d_{\mathrm{int}}(M,N).
\]
A \(V\)-enriched (Lawvere) lift is provided in F.A.

% -------------------------
\subsection*{F.4. Canonical Normal Form (CNF) after \texorpdfstring{$\mathbf{T}_\tau$}{Tτ} [Spec] (AK.Core)}
% -------------------------

\begin{definition}[CNF for morphisms after truncation]\label{F:def:cnf}
For \(f:M\to N\) in \(\Pers^{\mathrm{ft}}_k\), define its \emph{canonical normal form}
\(\mathrm{CNF}_\tau(f)\) as the barcode-level decomposition of \(\T_\tau f\) (Appendix~A):
\[
\T_\tau M \ \cong\ \bigoplus_a I_a,\qquad
\T_\tau N \ \cong\ \bigoplus_b I_b,\qquad
\T_\tau f\ \leftrightsquigarrow\ \bigoplus_c \mathrm{id}_{I_c}\ \oplus\ \bigoplus_d 0_{I_d}\ \oplus\ \bigoplus_e \iota_e,
\]
where each summand is either an isomorphism on an interval, the zero map, or a standard inclusion \(\iota_e:I[\ell,\infty)\to I[\ell',\infty)\) with \(\ell\ge \ell'\). This CNF is unique up to permutation of factors.
\end{definition}

\begin{proposition}[Reading \(\mu,\nu\) from CNF]\label{F:prop:cnf-mu-nu}
In \(\mathrm{CNF}_\tau(f)\), the multiplicity of \(I[0,\infty)\) in \(\Ker(\T_\tau f)\) (resp.\ \(\coker(\T_\tau f)\)) equals \(\mu\) (resp.\ \(\nu\)) as in Appendix~D. Finite bars contribute zero to \(\mu,\nu\).
\end{proposition}

\begin{remark}[CNF as a proof/automation device]
The CNF provides a canonical target for normalization tactics (F.19). It is functorial under isomorphism and stable under cofinal tower reindexings.
\end{remark}

% -------------------------
\subsection*{F.5. Filtered colimits, scope rule, and constructibility [Spec] (AK.Core)}
% -------------------------

\textbf{Scope rule.} All filtered (co)limit computations are performed \emph{objectwise} in
\([\mathbb{R},\mathsf{Vect}_k]\), where filtered colimits are exact; they are invoked only under
Appendix~A, Remark~\ref{A:rk:filtered-colimits}. Whenever the result might exit \(\Pers^{\mathrm{ft}}_k\),
we either (i) verify constructibility, or (ii) compute outside and \emph{return} via \(\mathbf{T}_\tau\)
or an explicit finite-type truncation. No claim is made outside this regime.

% -------------------------
\subsection*{F.6. Collapse on filtered complexes and f.q.i. [Spec] (AK.Core)}
% -------------------------

We use a collapse/threshold operation \(C_\tau\) at the level of filtered complexes:
\[
C_\tau:\ \mathsf{FiltCh}(k)\longrightarrow \mathsf{FiltCh}(k),
\]
compatible with \(\mathbf{P}_i\) by construction, and preserving filtered quasi-isomorphisms (f.q.i.) up to localization.
In practice, \(\mathbf{P}_i\circ C_\tau\) factors through \(\mathbf{T}_\tau\circ \mathbf{P}_i\) via a natural transformation that becomes an isomorphism after applying \(\mathbf{T}_\tau\) (Appendix~A).
All \(\Ext\)-tests are taken after \(\mathcal{R}(C_\tau F)\) with \(\mathcal{R}\) of amplitude \([-1,0]\) (Appendix~C).
AWFS scaffolding for \(C_\tau\) is given in F.B.

% -------------------------
\subsection*{F.7. Local equivalences within a window [Spec] (AK.LocalEquiv)}
% -------------------------

We formalize the window-local bridge \(\mathrm{PH}\leftrightarrow \Ext\) under amplitude \(\le 1\) and mild regularity.

Let a right-open window \(W\) be fixed; assume:
(i) the filtered complex \(F\) is concentrated in homological degrees \(\le 1\) on \(W\),
(ii) the collapse \(C_\tau\) is stable on \(W\) (\emph{saturation}),
(iii) no \(\tau\)-accumulation of critical values on \(W\) (Appendix~D),
(iv) tail isomorphism for \(\phi_{i,\tau}\) on \(W\) (Appendix~D, (S1)).

Then for the derived realization \(\mathcal{R}\) with amplitude \([-1,0]\) (Appendix~C) we have natural isomorphisms
\[
H^{-1}\bigl(\mathcal{R}(C_\tau F)\bigr)\ \simeq\ \varinjlim_{t\in W} H_1(F^t),\qquad
\Ext^1\bigl(\mathcal{R}(C_\tau F),k\bigr)\ \simeq\ \Hom\!\bigl(H^{-1}(\mathcal{R}(C_\tau F)),k\bigr).
\]
Thus \(\mathrm{PH}_1(F|_W)=0\Rightarrow \Ext^1(\mathcal{R}(C_\tau F),k)=0\) window-locally.

\begin{remark}[Local reverse under \(E_1(W)=0\)]
Under the definable-window trigger \(E_1(W)=0\) and finite Leray depth (Appendix~C, Cor.~\ref{C:cor:definable-local-bridge}), \(\Ext^1(\mathcal{R}(C_\tau F|_W),k)=0\) also implies \(\mathrm{PH}_1(C_\tau F|_W)=0\).
We encapsulate this as the tactic \emph{reverse\_bridge!} in F.19.
\end{remark}

% -------------------------
\subsection*{F.8. Towers, \texorpdfstring{$\phi_{i,\tau}$}{phi}, and diagnostics \texorpdfstring{$(\mu,\nu)$}{(mu,nu)} [Spec] (AK.Tower)}
% -------------------------

We define towers, the comparison map \(\phi_{i,\tau}\), and the invariants
\[
\mu_{i,\tau}(T)\ :=\ \mathrm{gdim}\,\ker\phi_{i,\tau}(T),\qquad
\nu_{i,\tau}(T)\ :=\ \mathrm{gdim}\,\mathrm{coker}\,\phi_{i,\tau}(T).
\]
They are invariant under filtered quasi-isomorphisms of towers and under cofinal reindexings, and vanish under (S1)–(S3). CNF (F.4) provides a canonical device to read \((\mu,\nu)\).

% -------------------------
\subsection*{F.9. Overlap Gate, MECE windows, and gluing [Spec] (AK.Gluing)}
% -------------------------

We formalize the operational glue from windows to a global verdict. MECE coverage, Čech–\(\Ext^1\) consistency on overlaps, Restart/Summability, and stability bands follow Appendices~C–E.

% -------------------------
\subsection*{F.10. Spectral calculus and Lipschitz bounds [Spec] (AK.Spectral)}
% -------------------------

Deletion-type operators commute with \(\mathbf{T}_\tau\) up to iso; inclusion-type operators are governed by stability bounds (Appendix~E). Safe low-pass filters (heat/resolvent, temporal) are non-expansive (Appendix~E).

% -------------------------
\subsection*{F.11. Edge identification \texorpdfstring{$\mathrm{PH}_1 \Rightarrow \Ext^1$}{PH1⇒Ext1} [Spec] (AK.Core)}
% -------------------------

Let \(\mathcal{R}:\mathsf{FiltCh}(k)\to D(\mathsf{Vect}_k)\) be of amplitude \([-1,0]\).
There is a natural edge isomorphism
\[
H^{-1}(\mathcal{R}(F))\ \cong\ \varinjlim_{t} H_1(F^t).
\]
For \(A\in D^{[-1,0]}\) we have \(\Ext^1(A,k)\cong \Hom(H^{-1}(A),k)\).
Combining these we obtain, for any \(F\),
\[
\mathrm{PH}_1(F)=0\quad\Longrightarrow\quad \Ext^1(\mathcal{R}(F),k)=0,
\]
and the same implication after insertion of the collapse \(C_\tau\).
Under \(E_1(W)=0\) on definable windows, the converse holds after collapse (Appendix~C).

% -------------------------
\subsection*{F.12. Lean~4 sketch (representative stubs) [Spec]}
% -------------------------

\begin{verbatim}
-- AK.Core: categories, Serre reflector, interleavings, scope rule, collapse, CNF
namespace AK.Core
open scoped BigOperators Classical
noncomputable section

variable (k : Type*) [Field k]
abbrev Vect := FinVect k
structure RIdx := (α : Type) (str : Preorder α)
abbrev Diag := (RIdx → Vect k)

abbrev Pers := { M : Diag k // Constructible M }

-- Serre reflector T_τ : Pers → E_τ^⊥
def Eτ (τ : ℝ≥0) : SerreSubcategory (Pers k) := by admit
noncomputable def Tτ (τ : ℝ≥0) : Pers k ⥤ (Eτ k τ).orthogonal := by admit
noncomputable def iotaτ (τ) : (Eτ k τ).orthogonal ⥤ Pers k := by admit
theorem Tτ_exact (τ) : (Tτ k τ).IsExact := by admit
theorem Tτ_idem  (τ) : (Tτ k τ) ⋙ (Tτ k τ) ≅ (Tτ k τ) := by admit
theorem Tτ_adj   (τ) : (Tτ k τ) ⊣ (iotaτ k τ) := by admit

-- Interleaving stability (1-Lipschitz)
class Interleaving (C : Type*) :=
  (dist : C → C → ℝ≥0∞) (isPseudoMetric : PseudoMetricSpace C)
def d_int := (Interleaving.dist : Pers k → Pers k → ℝ≥0∞)
noncomputable def Shift (ε : ℝ≥0) : Pers k ⥤ Pers k := by admit
axiom shift_comm (τ ε) : Shift k ε ⋙ (Tτ k τ) ≅ (Tτ k τ) ⋙ Shift k ε
theorem Tτ_nonexpansive (τ) :
  ∀ M N : Pers k, d_int k ((Tτ k τ).obj M) ((Tτ k τ).obj N) ≤ d_int k M N := by admit

-- Scope rule hooks
theorem filtered_colim_exact :
  ∀ {J} [IsFiltered J] (F : J ⥤ Vect), ExactFilteredColim F := by admit
axiom return_to_constructible :
  ∀ (D : SomeFilteredDiagram), Constructible (colim D)

-- Filtered complexes, persistence, collapse
abbrev FiltCh := FiltChCat k
def P_i (i : ℤ) : FiltCh k ⥤ Pers k := by admit
noncomputable def Cτ (τ : ℝ≥0) : FiltCh k ⥤ FiltCh k := by admit
theorem Cτ_preserves_fqi (τ) : PreservesFQI (Cτ k τ) := by admit

-- CNF after T_τ (barcode-normal form)
structure CNF where
  iso_parts  : List Interval
  zero_parts : List Interval
  incl_parts : List (Interval × Interval)  -- standard inclusions
noncomputable def cnf_after_Tτ {M N : Pers k} (τ) (f : M ⟶ N) : CNF := by admit

end AK.Core
\end{verbatim}

% -------------------------
\subsection*{F.13. Coq sketches (mathcomp/coq-cat-theory) [Spec]}
% -------------------------

\begin{verbatim}
From mathcomp Require Import all_ssreflect all_algebra.
From CoqCT Require Import Category Abelian Functor Limits Colimits.
Set Implicit Arguments. Unset Strict Implicit. Unset Printing Implicit Defensive.

Module AK.

Parameter k : fieldType.
Axiom Vect : AbelianCat.                     (* f.d. k-vector spaces *)
Axiom Rposet : PreOrder.                     (* (ℝ, ≤), schematic *)
Definition Diag := FunctorCat Rposet Vect.
Parameter Constructible : Diag -> Prop.
Record Pers := { M : Diag; pfd : Constructible M }.

Axiom Eτ : SerreSubcat Pers.
Axiom Tτ : Functor Pers (Orthogonal Eτ).
Axiom iotaτ : Functor (Orthogonal Eτ) Pers.
Axiom Tτ_exact : ExactFunctor Tτ.
Axiom Tτ_idem  : FunctorComp Tτ Tτ ≅ Tτ.
Axiom Tτ_adj   : Adjunction Tτ iotaτ.

Parameter dint : Pers -> Pers -> R.
Parameter Shift : R -> Functor Pers Pers.
Axiom shift_comm : forall eps, FunctorComp (Shift eps) Tτ ≅ FunctorComp Tτ (Shift eps).
Axiom Tτ_nonexpansive :
  forall (X Y : Pers), dint (Tτ X) (Tτ Y) <= dint X Y.

(* Collapse and persistence *)
Parameter FiltCh : Type.
Parameter P_i : Z -> Functor FiltCh Pers.
Parameter Cτ : R -> Functor FiltCh FiltCh.
Axiom Cτ_preserves_fqi : forall τ, PreservesFQI (Cτ τ).

(* CNF after T_τ *)
Record CNF := { iso_parts : seq Interval; zero_parts : seq Interval;
                incl_parts : seq (Interval * Interval) }.
Parameter cnf_after_Tτ : forall M N (f : Hom M N) τ, CNF.

End AK.
\end{verbatim}

% -------------------------
\subsection*{F.14. Tests and fixtures (T7, T10, T13) [Spec]}
% -------------------------

\paragraph{T7 (Saturation gate).}
Construct a tower of Type~IV (pure cokernel) and one with a stationary summand (pure kernel),
and their direct sum. Verify that:
(i) \(\mu,\nu\) equal the multiplicity of \(I[0,\infty)\) after \(\mathbf{T}_\tau\) (via CNF);
(ii) cofinal reindexing \(n\mapsto n+1\) preserves \((\mu,\nu)\);
(iii) under (S1) the comparison \(\phi_{i,\tau}\) is an isomorphism and \((\mu,\nu)=(0,0)\).

\paragraph{T10 (A/B).}
Instantiate an \(\eta\)-tolerant A/B test on overlaps, verify soft commuting and that the
window-local \(\Ext^1\)-vanishing glues via Čech–\(\Ext^1\).

\paragraph{T13 (\(\delta\)-budget).}
Generate a \(\delta\)-ledger per window/degree; check additivity/post-stability and the restart inequality
\(\mathrm{gap}_{k+1}\ge \kappa(\mathrm{gap}_k-\Sigma\delta_k)\).
Verify summability \(\sum_k\Sigma\delta_k<\infty\) and that \(\mathrm{B\text{-}Gate}^+\) accepts precisely when
\(\mathrm{gap}>\mathrm{dsum}\) along with \(\mathrm{PH}_1=0\), \(\Ext^1=0\), and \((\mu,\nu)=(0,0)\).

% -------------------------
\subsection*{F.15. What is proved, what is assumed [Spec]}
% -------------------------

\begin{itemize}
\item (\emph{Localization}) \(\mathsf{E}_\tau\) is hereditary Serre; \(\mathbf{T}_\tau\) exists, is exact,
idempotent, and induces \(\Pers^{\mathrm{ft}}_k/\mathsf{E}_\tau\simeq \mathsf{E}_\tau^\perp\).
\item (\emph{Stability}) \(\mathbf{T}_\tau\) is \(1\)-Lipschitz for the interleaving metric; via shift-commutation (Appendix~A).
\item (\emph{Towers}) \(\phi_{i,\tau}\) is functorial; under (S1)–(S3) (Appendix~D) it is an isomorphism, hence
\((\mu,\nu)=(0,0)\). \((\mu,\nu)\) are invariant under f.q.i.\ and cofinal reindexings; finiteness holds by degree bounds.
\item (\emph{Bridge}) For \(F\in\mathsf{FiltCh}(k)\), \(\mathcal{R}\) has amplitude \([-1,0]\);
\(H^{-1}(\mathcal{R}(F))\simeq \varinjlim_t H_1(F^t)\) and
\(\Ext^1(A,k)\simeq \Hom(H^{-1}(A),k)\) for \(A\in D^{[-1,0]}\), hence
\(\mathrm{PH}_1(F)=0\Rightarrow \Ext^1(\mathcal{R}(F),k)=0\).
On definable right-open windows with \(E_1(W)=0\), the local reverse holds after \(C_\tau\) (Appendix~C).
\item (\emph{Spectral}) Deletion-type operators commute with \(\mathbf{T}_\tau\) (up to iso) and are spectrally monotone; inclusion-type operators are controlled via stability bounds; safe low-pass is non-expansive (Appendix~E).
\end{itemize}

% -------------------------
\subsection*{F.16. Notes on libraries and portability [Spec]}
% -------------------------

The Lean sketch targets \textsf{mathlib} (abelian categories, Serre subcategories, localization, derived categories).
The Coq sketch targets \textsf{mathcomp}+\textsf{coq-category-theory} (or \textsf{UniMath}). Nontrivial steps are
isolated behind \texttt{admit}/\texttt{Axiom} with explicit references to Appendices~A–E.
Replacing them by library lemmas yields a complete development.
In Lean, define \(\phi_{i,\tau}\) via \texttt{Limits.colimit.desc} and use \texttt{colimit.hom\_ext} for naturality;
in Coq, use \texttt{Colim.desc}/\texttt{colim\_map} with right-to-left \texttt{compose} convention.

% -------------------------
\subsection*{F.17. Thin catalysts (V-enrichment, AWFS, o-minimal Čech, Iwasawa control)}
% -------------------------

\paragraph{F.A. V-enriched metric/shift (Spec).}
Let \(V\) be a commutative quantale. Equip \(\Pers^{\mathrm{ft}}_k\) with a Lawvere \(V\)-metric \(d_V\) obtained from \(d_{\mathrm{int}}\) via a monotone embedding and a \(V\)-aggregator.
Require:
(i) \(d_V\) extends \(d_{\mathrm{int}}\) on scalars,
(ii) \(\mathrm{Shift}_\varepsilon\) is \(V\)-1-Lipschitz,
(iii) \(\mathbf{T}_\tau\) is \(V\)-1-Lipschitz and commutes with shifts up to enriched natural isomorphism.
\emph{Use:} stability of dashboards and Overlap Gate tolerances.

\paragraph{F.B. AWFS skeleton for \(C_\tau\) (Spec).}
Postulate an algebraic weak factorization system \((\mathcal{L}_\tau,\mathcal{R}_\tau)\) on \(\mathsf{FiltCh}(k)\) with functorial factorization
\(F\xrightarrow{\ell_\tau} C_\tau F \xrightarrow{r_\tau} F\),
where \(\ell_\tau\) is a \(\tau\)-collapse cofibration (acyclic at persistence level after \(\mathbf{T}_\tau\)) and \(r_\tau\) a \(\tau\)-local fibration.
\emph{Use:} functoriality of \(C_\tau\), pullbacks along \(\mathcal{R}_\tau\)-maps, and stability of f.q.i.\ under collapse.

\paragraph{F.C. o\hbox{-}minimal Čech and Iwasawa control (Spec).}
Let \(\mathcal{U}=\{U_i\}\) be a definable right-open cover of a window \(W\) with finite Leray depth.
Then the Čech complex \(\check{C}(\mathcal{U})\) computes window-local \(\Ext^1\) (under amplitude \(\le 1\)) and patches across overlaps.
An \emph{Iwasawa control catalyst} is a tuple \((\Gamma,\rho,\upsilon)\) with a directed index \(\Gamma\), a nonexpansive control \(\rho:\Gamma\to \RR_{\ge 0}\), and a compatibility map \(\upsilon\) such that the windowed diagnostics are \(\rho\)-Cauchy and tail-identify with the apex.
\emph{Use:} Overlap Gate—uniform control across towers implies stability-band persistence and global gluing.

% -------------------------
\subsection*{F.18. Convergence Manager (DP windows) [Spec]}
% -------------------------

We wrap Restart/Summability (Appendix~D) into a \emph{Convergence Manager} for dynamic-programming (DP) windowing.

\begin{specification}[DP-Convergence Manager]
Maintain, per degree \(i\): current safety margin \(\mathrm{gap}_k(i)\), per-window budget \(\Sigma\delta_k(i)\),
and retention factor \(\kappa\in(0,1]\). On transition \(k\to k+1\),
\[
\mathrm{gap}_{k+1}(i)\ \gets\ \max\bigl\{0,\ \kappa\bigl(\mathrm{gap}_k(i)-\Sigma\delta_k(i)\bigr)\bigr\}.
\]
Accept window \(k\) if \(\mathrm{gap}_k(i)>\Sigma\delta_k(i)\) and \((\mu,\nu)=(0,0)\); declare convergence on an interval if the acceptance persists across the MECE chain and \(\sum_k\Sigma\delta_k(i)<\infty\).
\end{specification}

\paragraph{Lean~4 tactic skeleton.}
\begin{verbatim}
namespace AK.Gluing
open AK.Core
meta def converge_windows! :
  Π (κ : ℝ≥0) (deg : ℤ) (gaps budgets : List ℝ≥0),
    tactic (List ℝ≥0) := by admit
-- Intended behavior: compute next gaps via κ⋅(gap - budget)⁺ and
-- produce certificates that BGATE⁺ holds on accepted windows.
end AK.Gluing
\end{verbatim}

\paragraph{Coq hint database (sketch).}
\begin{verbatim}
Create HintDb DPconvergence.
Hint Resolve restart_ok summable_budget : DPconvergence.
(* A Ltac 'converge_windows' computes κ⋅(gap - Σδ)^+ and applies the hints. *)
\end{verbatim}

% -------------------------
\subsection*{F.19. Tactic stubs: \texttt{cnf!}, \texttt{ext1\_hom!}, \texttt{reverse\_bridge!}, \texttt{converge\_windows!}}
% -------------------------

\paragraph{Lean~4 (meta-level).}
\begin{verbatim}
namespace AK.Tactics
open AK.Core AK.Tower AK.LocalEquiv

/-- Put a morphism into CNF after T_τ and read (μ,ν). -/
meta def cnf! (τ : ℝ≥0) : tactic Unit := `[refine (AK.Core.cnf_after_Tτ _ _ ?f τ); all_goals admit]

/-- Solve goals of the form Ext¹(A,k) ≅ Hom(H^{-1}A,k) for A ∈ D^{[-1,0]}. -/
meta def ext1_hom! : tactic Unit := `[apply AK.Core.ext1_edge; try { exact ‹_› }]

/-- Local reverse bridge under E₁(W)=0 (definable window, amplitude ≤ 1). -/
meta def reverse_bridge! : tactic Unit := `
  [ -- consume window hypotheses, reduce to vanishing edge group
    apply AK.LocalEquiv.local_bridge; all_goals admit ]

/-- DP-window convergence manager (Appendix D/E). -/
meta def converge_windows! := AK.Gluing.converge_windows!
end AK.Tactics
\end{verbatim}

\paragraph{Coq (Ltac skeleton).}
\begin{verbatim}
Ltac cnf τ :=
  (* normalize T_τ f to barcode CNF and compute μ, ν *) idtac.

Ltac ext1_hom :=
  (* reduce Ext¹(A,k) to Hom(H^{-1}A,k) when A ∈ D^{[-1,0]} *) idtac.

Ltac reverse_bridge :=
  (* apply local equivalence under E₁(W)=0 to deduce PH₁=0 from Ext¹=0 *) idtac.

Ltac converge_windows :=
  (* apply restart_ok and summable_budget to chain windows *) eauto with DPconvergence.
\end{verbatim}

% -------------------------
\subsection*{F.20. Completion note}
% -------------------------

This appendix delivers IMRN/AiM-ready, concise formalization stubs: \(\mathbf{T}_\tau\) (exact, idempotent, \(1\)-Lipschitz), CNF after truncation to read \((\mu,\nu)\), tower calculus with invariances and sufficient conditions (S1–S3), the bridge \(\mathrm{PH}_1\Rightarrow \Ext^1\) and its local reverse under \(E_1(W)=0\), spectral auxiliaries (non-expansive low-pass), and a DP-style Convergence Manager for window pasting. Lean/Coq tactic skeletons (\texttt{cnf!}, \texttt{ext1\_hom!}, \texttt{reverse\_bridge!}, \texttt{converge\_windows!}) provide an auditable path from the paper’s hypotheses to machine-checked goals. All claims are confined to the constructible, amplitude-\(\le 1\) regime with filtered colimits used under the scope rule; no further supplementation is required for operational use in the proof framework.



% =========================
\section*{Appendix G. Reproducibility: Logs and Schemas [Spec] (reinforced)}
% =========================
\phantomsection
\addcontentsline{toc}{section}{Appendix G. Reproducibility: Logs and Schemas}
\refstepcounter{section}
\label{G:repro}

This appendix specifies the provenance log (\texttt{run.yaml}) and the machine-readable
schemas for artifacts produced in this work—barcodes (\texttt{bars}), spectral indicators
(\texttt{spec}), Ext-tests (\texttt{ext}), tower comparison maps (\texttt{phi}), and the windowed
length spectrum audit (\texttt{Lambda\_len}).
All files may be emitted in either JSON or HDF5; JSON keys coincide with HDF5
group/dataset names. Colimits are used only under the scope policy
(Appendix~A, Remark~\ref{A:rk:filtered-colimits}). Type labels follow \emph{Type I--II / Type III / Type IV}.
Cross-module conventions: the Ext-test is always against \(k[0]\), i.e.\ \(\Ext^1(\mathcal{R}(C_\tau F),k)=0\)
(with \(C_\tau\) understood up to f.q.i.\ on \(\Ho(\mathsf{FiltCh}(k))\)); the energy exponent satisfies
\(\alpha>0\) (default \(\alpha=1\)). Spectral monotonicity is asserted only for \emph{deletion-type} operations
(Dirichlet/principal/Loewner), with directions fixed by Appendix~E; inclusion-type operations
are used solely with stability bounds. All comparisons follow the mandatory order:
\[
\boxed{\ \text{for each } t\ \Longrightarrow\ \text{apply } \mathbf{P}_i \ \Longrightarrow\ \text{apply } \mathbf{T}_\tau \ \Longrightarrow\ \text{compare in }\Pers^{\mathrm{ft}}_k\ }.
\]

% ---- NEW: extended run.yaml schema keys (inserted at the very beginning) ----
\begin{declaration}[run.yaml schema (extended)]\label{dec:runyaml-extended}
The manifest admits the following \emph{optional but normative} sections, used by the
V\hbox{-}enriched metric, definable windowing, layered \(\delta\)-ledger, Iwasawa control, AWFS scaffolding, and tropical binning (cf.\ Appendices~F.A–F.C,E):
\begin{verbatim}
quantale:
  name: "[0,inf]_plus"   # or "max-plus", "product", ...
  op: "+"                # value-level monoid op (aggregator for distances/budgets)
  unit: 0
  order: "<="

definable:
  o_minimal_structure: "R_an,exp"   # or "Denef-Pas"
  window_formulae:
    - "u <= t < u'"      # right-open windows; finite cover
  cech_depth_bound: 2    # Leray/Čech acyclicity depth bound on overlaps

layered_delta:
  delta_Gal: ...         # geometric-algebraic
  delta_Tr:  ...         # discretization/rounding/truncation
  delta_Fun: ...         # functorial/commutation residuals
  compose: "quantale-sum"  # single ⊕ (from quantale.op) to aggregate

iwasawa:
  tower_level: ...       # limiting level/index used for φ-apex
  control_finite_bounds:
    kernel_leq: ...      # bound for μ
    cokernel_leq: ...    # bound for ν

awfs_2cell:
  awfs_enabled: true
  two_cell_bounds:
    mirror_collapse: ...     # C_τ functoriality bound across mirrors
    transfer_collapse: ...   # bound across transfers/pullbacks

tropical:
  bins: { width: ..., range: [a,b] }   # optional diagnostic binning for aux-bars

policy:
  after_collapse_only: true
  windows: "right-open"
\end{verbatim}
\end{declaration}

\paragraph{New in this version (2025-03-15 / suite v17.0).}
Beyond the original specification, this version integrates:
(i) \texttt{windows} (domain/collapse/spectral),
(ii) \texttt{coverage\_check},
(iii) \texttt{operations} (with \texttt{U}, \texttt{type}, \(\tau\), and \(\delta\) breakdown),
(iv) \texttt{persistence} (summaries: \texttt{PH1\_zero}/\texttt{Ext1\_zero}/\(\mu\)/\(\nu\)/\texttt{phi\_iso\_tail}),
(v) \texttt{spectral.aux\_bars\_remaining},
(vi) \texttt{budget} (\texttt{sum\_delta}/\texttt{safety\_margin}/\texttt{gap\_tau}),
(vii) \texttt{gate.accept},
(viii) \texttt{overlap\_checks} (Overlap Gate),
(ix) \texttt{Lambda\_len} (length spectrum audit),
(x) \texttt{spectral\_policy} (norms, ordering, spectral bounds and \emph{safe low-pass} flags),
(xi) \texttt{stability\_bands} (global band list; Appendix~D.8),
(xii) cross-links to \texttt{Lambda\_len} from all artifacts,
(xiii) quantale/definable/Iwasawa/AWFS/tropical hooks,
(xiv) HDF5 canonicalization with fixed-length UTF-8 strings.
These fields ensure bitwise reproducibility and third-party auditability.

% -------------------------
\subsection*{G.1. Provenance, determinism, and gating}
% -------------------------
Each run records (i) source/inputs, (ii) algorithmic choices and thresholds, (iii) numeric
tolerances and units, (iv) code/environment fingerprints, (v) RNG details, (vi) strong identifiers
(content hashes) for all artifacts, and (vii) \emph{gating} decisions that determine acceptance of results.
Randomness is controlled by explicit seeds. Floating-point claims report both an \emph{asserted} tolerance
and a \emph{measured} slack. \emph{Windows} (domain/collapse/spectral) must be declared, and a
\emph{coverage check} attests that all measured quantities fall inside their stated windows.
A \emph{budget} aggregates operation-level error contributions \(\delta\) and yields a
\emph{safety margin} relative to the governing tolerance; finally, \texttt{gate.accept} records the run-level
decision (accept/reject) together with reasons.

% -------------------------
\subsection*{G.2. \texttt{run.yaml} schema (versions, windows, overlap, budget, gate)}
% -------------------------
\paragraph{Intent.} A single file per execution, sufficient to reproduce the pipeline end-to-end, including
all windows, coverage checks, operation logs, overlap checks, and the final acceptance gate.

\noindent\textbf{Canonical layout (YAML).}
\begin{verbatim}
version: 17                      # suite version (v17.0)
schema_version: "2025-03-15"
suite_version: "v17.0"
run_id: "2025-03-15T09:12:07Z-7f5c1b1"
seed: 1337
rng:
  python: "default_rng"
  numpy:  "PCG64"
platform:
  os: "Ubuntu 22.04"
  cpu: "Intel(R) Xeon(R) Platinum 8370C"
  cuda: "12.2"
  blas: "OpenBLAS 0.3.23"
  hdf5: "1.14.3"
  lapack: "OpenBLAS-LAPACK"
  glibc: "2.35"
  kernel: "5.15.0-105"
  locale: "C.UTF-8"
env:
  python: "3.11.7"
  packages:
    numpy: "1.26.4"
    scipy: "1.13.1"
    h5py:  "3.10.0"
    networkx: "3.2.1"
  threads:
    OMP_NUM_THREADS: 1
    MKL_NUM_THREADS: 1
    OPENBLAS_NUM_THREADS: 1
container:
  image: "docker.io/example/persistence:2025.03"
  digest: "sha256:deadbeef..."
git:
  repo: "git@host:ak/persistence.git"
  commit: "a1b2c3d4"
units:
  filtration: "dimensionless"
  eigenvalues: "dimensionless"

# ---- Extended semantics (cf. App.F.A–F.C, App.E) ----
quantale:
  name: "[0,inf]_plus"
  op: "+"
  unit: 0
  order: "<="
definable:
  o_minimal_structure: "R_an,exp"
  window_formulae: ["u <= t < u'"]
  cech_depth_bound: 2
layered_delta:
  delta_Gal: 0.020
  delta_Tr:  0.015
  delta_Fun: 0.015
  compose: "quantale-sum"
iwasawa:
  tower_level: 128
  control_finite_bounds: { kernel_leq: 2, cokernel_leq: 0 }
awfs_2cell:
  awfs_enabled: true
  two_cell_bounds: { mirror_collapse: 0.005, transfer_collapse: 0.005 }
tropical:
  bins: { width: 0.02, range: [0.0, 2.0] }
policy:
  after_collapse_only: true
  windows: "right-open"

windows:
  domain:
    filtration_range: [0.0, 2.0]
    degrees: [0, 2]
  collapse:
    tau_sweep: [0.25, 0.50, 1.00]
  spectral:
    range: [0.0, 2.0]
    order: "ascending"
coverage_check:
  domain_window_covers_bars: true
  spectral_window_covers_thetas: true
  collapse_tau_sweep_covers_reports: true
stability_bands:
  - { i: 1, tau_lo: 0.60, tau_hi: 0.95 }   # Appendix D.8; certified band(s)

overlap_checks:
  local_equiv: true
  cech_ext1_ok: true
  stability_band_ok: true

inputs:
  dataset: "AK-bench-v3"
  graphs:
    - path: "data/G_001.edgelist"
      hash: "sha256:..."
  filters:
    type: "height"
    params: { axis: 2 }

pipeline:
  metric: "interleaving"          # exactly one of: interleaving | bottleneck
  stages:
    - name: "barcode"
      params: { field: "k", reduction: "clearing" }
    - name: "collapse"            # C_τ (up to f.q.i.)
      params: { tau: 0.50 }
    - name: "spec"
      params:
        window: [0.0, 2.0]
        norm: "fro"               # "fro" ≡ ‖·‖_fro, "op" ≡ ‖·‖_op
        order: "ascending"
        clip: 1.00
        loewner_assumption: "A'⪯A"   # enum: "A'⪯A" | "A'⪰A" | "none"
        spectral_bounds:
          lambda_min: 1.0e-12
          lambda_max: 1.0e+05
          lip_tol: 0.02
        low_pass:                 # safety conditions for non-expansive low-pass
          kernel: "heat"
          even: true
          mass: 1.0               # unit mass
          clip_tau_eq_pipeline: true
        eig_solver:
          method: "lanczos"
          k: 128
          maxiter: 1000
          tol: 1e-12
          reorthogonalize: true
          rng_seed: 1337
        tropical_bins: { width: 0.02, range: [0.0, 2.0] }
    - name: "ext-test"            # Ext^1(R(C_τ F), k)
      params: { amplitude_check: true }

operations:
  - step: 1
    U: [0,1,3]
    type: "inclusion"
    tau: 0.50
    delta:
      distance: { interleaving: 0.050 }
      sources:
        discretization: 0.030
        rounding: 1.0e-12
        heuristic: 0.020
      total: 0.050
      note: "Edge contraction in subgraph U"
  - step: 2
    U: "V\\W"
    type: "schur_complement"
    tau: 0.50
    delta:
      distance: { interleaving: 0.100 }
      sources:
        elimination: 0.080
        rounding: 0.020
      total: 0.100

persistence:
  PH1_zero: true
  Ext1_zero: true
  mu: 1
  nu: 0
  phi_iso_tail:
    passed: false
    i: 1
    tau: 0.50

spectral:
  aux_bars_remaining: 0

thresholds:
  alpha: 1.0
  tol:
    distance:
      interleaving: 1e-6
    eig: 1e-8
    witness: 1e-9

budget:
  sum_delta:
    distance:
      interleaving: 0.150
  safety_margin: 0.850
  gap_tau: 0.025
  rationale: "All deltas accounted for; slack remains >0"

Lambda_len:
  degree: 1
  tau: 0.50
  audit: "hash:2f4c...d1"

gate:
  accept: true
  reason: "Coverage ok; safety margin positive; all assertions satisfied"

serialization:
  float_dtype: "ieee754-f64-le"
  json_sort_keys: true
  hdf5_canonical:
    compression: { algo: "gzip", level: 4 }
    shuffle: false
    fletcher32: false
    track_times: false
    fillvalue: 0.0
    string_encoding: "utf8-fixed"
    chunk_shapes:
      bars: { i: 4096, birth: 4096, death: 4096, death_is_inf: 4096, mult: 4096 }
      spec_eigs: { eig: 4096 }
      spec_Ntheta: { theta: 512, left: 512, right: 512 }
      phi_idx: { i: 256, tau: 256, iso: 256, mu: 256, nu: 256 }

cache:
  enabled: true
  dir: ".cache/run_7f5c1b1"

timing:
  wallclock_s: 123.4
  cpu_s: 456.7
  stages:
    barcode: 12.3
    collapse: 4.5
    spec: 80.0
    ext_test: 2.0

status:
  success: true
  errors: []

outputs:
  bars: "out/bars_7f5c1b1.json"
  spec: "out/spec_7f5c1b1.json"
  ext:  "out/ext_7f5c1b1.json"
  phi:  "out/phi_7f5c1b1.h5"
\end{verbatim}

% -------------------------
\subsection*{G.3. \texttt{bars} (barcodes) schema}
% -------------------------
\paragraph{Semantics.} A constructible barcode is a multiset of half-open intervals \(I=[b,d)\) with degree \(i\).
Deaths may be \(+\infty\).

\noindent\textbf{JSON layout (infinity convention, units, cross-links, optional clip report).}
\begin{verbatim}
{
  "meta": {
    "schema_version": "2025-03-15",
    "suite_version": "v17.0",
    "field": "k",
    "filtration_units": "dimensionless",
    "endpoint_convention": "[b,d) (see Chapter~2)",
    "infinity": { "json": "inf" },
    "clip_tau": 0.50,                      // optional: reporting τ
    "float_dtype": "ieee754-f64-le",
    "string_encoding": "utf8-fixed",
    "links": {
      "run_id": "2025-03-15T09:12:07Z-7f5c1b1",
      "run_yaml_hash": "sha256:...run",
      "Lambda_len": "sha256:...Lambda"     // cross-link to length spectrum audit
    }
  },
  "bars": [
    { "i": 0, "birth": 0.0, "death": 0.3,   "mult": 1 },
    { "i": 1, "birth": 0.2, "death": "inf", "mult": 1 }
  ],
  "hash": "sha256:...bars"
}
\end{verbatim}

\noindent\textbf{HDF5 layout (split representation for \(+\infty\); fixed UTF-8).}
\begin{itemize}[leftmargin=1.25em]
  \item Datasets:
    \path{/bars/i} (\texttt{int32}),
    \path{/bars/birth} (\texttt{float64}),
    \path{/bars/death} (\texttt{float64}),
    \path{/bars/death_is_inf} (\texttt{bool}),
    \path{/bars/mult} (\texttt{int32}).
  \item Attributes:
    \path{/bars}.attrs[\texttt{field}=\texttt{"k"}],
    \texttt{filtration\_units},
    \texttt{schema\_version},
    \texttt{suite\_version},
    \texttt{float\_dtype},
    \texttt{death\_encoding}=\texttt{"split\_scalar\_bool"},
    \texttt{string\_encoding}=\texttt{"utf8-fixed"},
    optional \texttt{clip\_tau},
    and \texttt{links/Lambda\_len}.
\end{itemize}

% -------------------------
\subsection*{G.4. \texttt{spec} (spectral indicators) schema}
% -------------------------
\paragraph{Semantics.} Spectral features: clipped sums, counts above/below thresholds with left/right limits,
and deletion-type monotonicity diagnostics. Matrices are identified by content hashes.

\noindent\textbf{JSON layout (ascending storage; \(N_{\theta\pm 0}\); solver and low-pass params; coverage; cross-links).}
\begin{verbatim}
{
  "meta": {
    "schema_version": "2025-03-15",
    "suite_version": "v17.0",
    "eigen_units": "dimensionless",
    "order": "ascending",
    "sorted": true,
    "Ntheta_convention": { "left": "N_{θ-0}", "right": "N_{θ+0}" },
    "window": { "range": [0.0, 2.0], "semantics": "closed" },
    "norm": "fro",
    "clip_tau": 1.0,
    "tol_eig": 1e-8,
    "loewner_assumption": "A'⪯A",
    "low_pass": { "kernel": "heat", "even": true, "mass": 1.0, "safe": true },
    "aux_bars_remaining": 0,
    "coverage_check": { "thetas_in_window": true },
    "tropical_bins": { "width": 0.02, "range": [0.0, 2.0] },
    "string_encoding": "utf8-fixed",
    "eig_solver": {
      "method": "lanczos", "k": 128, "maxiter": 1000,
      "tol": 1e-12, "reorthogonalize": true, "rng_seed": 1337
    },
    "links": {
      "run_id": "2025-03-15T09:12:07Z-7f5c1b1",
      "run_yaml_hash": "sha256:...run",
      "Lambda_len": "sha256:...Lambda"
    }
  },
  "operators": [
    {
      "id": "sha256:...A",
      "kind": "laplacian_dirichlet",
      "n": 500,
      "spectrum": { "eigs": [0.10, 0.12, 0.45, ...] },
      "clip":     { "tau": 1.0, "sum": 37.219, "deficit": 12.004 }
    },
    {
      "id": "sha256:...B",
      "kind": "principal_submatrix",
      "parent": "sha256:...A",
      "N_theta": [
        { "theta": 0.20, "left": 17, "right": 16 },
        { "theta": 0.50, "left": 10, "right": 10 }
      ],
      "monotonicity": { "type": "deletion", "passed": true }
    }
  ],
  "hash": "sha256:...spec"
}
\end{verbatim}

\noindent\textbf{HDF5 layout.}
\begin{itemize}[leftmargin=1.25em]
  \item \path|/spec/ops/{id}/eig| (float64, ascending),
        \path|/spec/ops/{id}/clip/sum| (float64),
        \path|/spec/ops/{id}/clip/deficit| (float64),
        \path|/spec/ops/{id}/Ntheta/theta|, \path|/spec/ops/{id}/Ntheta/left|, \path|/spec/ops/{id}/Ntheta/right|
        (parallel datasets).
  \item Attributes: \texttt{kind}, \texttt{parent}, \texttt{norm} \(\in\)\{\texttt{"fro"},\texttt{"op"}\},
        \texttt{order}=\texttt{"ascending"}, \texttt{sorted} (bool), \texttt{eigen\_units}, \texttt{tol\_eig},
        \texttt{schema\_version}, \texttt{suite\_version}, \texttt{loewner\_assumption},
        \texttt{low\_pass/*}, \texttt{aux\_bars\_remaining}, \texttt{coverage\_thetas\_in\_window},
        optional \texttt{tropical\_bins/width,range}, \texttt{string\_encoding}=\texttt{"utf8-fixed"},
        \texttt{links/Lambda\_len}.
\end{itemize}

% -------------------------
\subsection*{G.5. \texttt{ext} (Ext-test) schema}
% -------------------------
\paragraph{Semantics.} Outcome of the one-way bridge \(\mathrm{PH}_1\Rightarrow \Ext^1\) for \(C_\tau F\), with
amplitude checks for \(\mathcal{R}\) and recorded assumptions.

\noindent\textbf{JSON layout (assumptions, cross-links, suite key).}
\begin{verbatim}
{
  "meta": {
    "schema_version": "2025-03-15",
    "suite_version": "v17.0",
    "field": "k", "alpha": 1.0,
    "assumptions": {
      "field_is_k": true,
      "constructible_verified": true,
      "t_exact_and_amp_le_1": true
    },
    "string_encoding": "utf8-fixed",
    "links": {
      "run_id": "2025-03-15T09:12:07Z-7f5c1b1",
      "run_yaml_hash": "sha256:...run",
      "Lambda_len": "sha256:...Lambda"
    }
  },
  "tau": 0.50,
  "amplitude": { "ok": true, "range": [-1, 0] },
  "Hminus1": { "dim": 0, "witness_norm": 0.0 },
  "Ext1":    { "dim": 0, "passed": true, "tol": 1e-9, "slack": 0.0 },
  "links":   { "bars": "sha256:...bars", "phi": "sha256:...phi" },
  "hash": "sha256:...ext"
}
\end{verbatim}

\noindent\textbf{HDF5 layout.}
\begin{itemize}[leftmargin=1.25em]
\item Scalars: \texttt{/ext/tau} (float64), \texttt{/ext/Hminus1/dim} (int32),
\texttt{/ext/Ext1/dim} (int32), \texttt{/ext/Ext1/passed} (bool), \texttt{/ext/Ext1/tol} (float64),
\texttt{/ext/Ext1/slack} (float64).
\item Attributes: \texttt{field}=\texttt{"k"}, \texttt{alpha}=float64, \texttt{schema\_version},
\texttt{suite\_version}, \texttt{assumptions/*}, \texttt{string\_encoding}=\texttt{"utf8-fixed"},
\texttt{links/Lambda\_len}.
\end{itemize}

% -------------------------
\subsection*{G.6. \texttt{phi} (tower comparison) schema}
% -------------------------
\paragraph{Semantics.} Encodes \(\phi_{i,\tau}\) for towers, together with \((\mu,\nu)\) as generic-fiber
dimensions after truncation, structural flags for the sufficiency hypotheses (Appendix~D §D.4), and explicit
stability bands (Appendix~D.8).

\noindent\textbf{JSON layout (with \(\tau\)-sweep, stability bands, witnesses, iso-tail, cross-link).}
\begin{verbatim}
{
  "meta": {
    "schema_version": "2025-03-15",
    "suite_version": "v17.0",
    "definition": "phi_{i,τ}: colim T_τ P_i(F_n) → T_τ P_i(F_∞)",
    "scope": "colim in [ℝ,Vect_k], return-to-constructible policy",
    "tau_sweep": [0.25, 0.50, 1.00],
    "edge_kinds": ["inclusion","projection","quasi_iso",
                   "filtration_preserving_map","schur_complement","other"],
    "stability_bands": [ { "i": 1, "tau_lo": 0.60, "tau_hi": 0.95 } ],
    "string_encoding": "utf8-fixed",
    "links": {
      "run_id": "2025-03-15T09:12:07Z-7f5c1b1",
      "run_yaml_hash": "sha256:...run",
      "Lambda_len": "sha256:...Lambda"
    }
  },
  "indices": [
    {
      "i": 1, "tau": 0.50,
      "iso": false,
      "mu": 1, "nu": 0,
      "flags": { "S1_commutes": false, "S2_noAccum": true, "S3_Cauchy": false },
      "witness": { "ker_generic_dim": 1, "coker_generic_dim": 0 },
      "iso_tail": { "passed": false }
    }
  ],
  "tower": {
    "nodes": [
      { "n": 0, "id": "sha256:...F0" },
      { "n": 1, "id": "sha256:...F1" }
    ],
    "edges": [
      { "src": 0, "dst": 1, "kind": "inclusion" }
    ],
    "limit": { "id": "sha256:...Finf" }
  },
  "hash": "sha256:...phi"
}
\end{verbatim}

\noindent\textbf{HDF5 layout.}
\begin{itemize}[leftmargin=1.25em]
\item \texttt{/phi/idx/i} (int32), \texttt{/phi/idx/tau} (float64),
\texttt{/phi/idx/iso} (bool), \texttt{/phi/idx/mu} (int32), \texttt{/phi/idx/nu} (int32).
\item \texttt{/phi/idx/flags}: \texttt{S1\_commutes}, \texttt{S2\_noAccum}, \texttt{S3\_Cauchy} (bool).
\item Optional witnesses: \texttt{/phi/idx/ker\_generic\_dim}, \texttt{/phi/idx/coker\_generic\_dim}.
\item Optional tail: \texttt{/phi/idx/iso\_tail/passed} (bool).
\item \texttt{/phi/meta/stability\_bands}: records \((i,\tau_{\mathrm{lo}},\tau_{\mathrm{hi}})\) intervals.
\item Optional tower edges: \texttt{/phi/tower/edges/src,dst} (int32),
\texttt{/phi/tower/edges/kind} (fixed-length UTF-8 string).
\item Attributes: \texttt{schema\_version}, \texttt{suite\_version}, \texttt{string\_encoding}=\texttt{"utf8-fixed"},
\texttt{tau\_sweep} (float64 array), \texttt{links/Lambda\_len}.
\end{itemize}

% -------------------------
\subsection*{G.7. \texttt{Lambda\_len} (windowed length spectrum) schema}
% -------------------------
\paragraph{Semantics.} The length spectrum operator \(\Lambda_{\mathrm{len}}(M;[0,\tau])\) is diagonal on the bar-basis with eigenvalues
the clipped bar lengths on \([0,\tau]\). Its unordered eigenvalue multiset equals the clipped bar-length multiset
(Appendix~H). The \texttt{Lambda\_len} audit records either the eigenvalue list (small instances) or a content hash.

\noindent\textbf{JSON layout (links to all major artifacts).}
\begin{verbatim}
{
  "meta": {
    "schema_version": "2025-03-15",
    "suite_version": "v17.0",
    "definition": "Lambda_len(P_i(C_τ F); [0,τ])",
    "degree": 1,
    "tau": 0.50,
    "string_encoding": "utf8-fixed",
    "links": {
      "bars": "sha256:...bars",
      "phi": "sha256:...phi",
      "spec": "sha256:...spec",
      "ext":  "sha256:...ext",
      "run_yaml_hash": "sha256:...run"
    }
  },
  "eigs": [0.24, 0.51, 0.78],      // optional explicit eigenvalues (small cases)
  "hash": "sha256:2f4c...d1"
}
\end{verbatim}

\noindent\textbf{HDF5 layout.}
\begin{itemize}[leftmargin=1.25em]
  \item \texttt{/Lambda\_len/meta} attributes: \texttt{schema\_version}, \texttt{suite\_version}, \texttt{degree}, \texttt{tau},
  \texttt{string\_encoding}=\texttt{"utf8-fixed"}, links as above.
  \item \texttt{/Lambda\_len/eigs} (optional; float64 array).
  \item \texttt{/Lambda\_len/hash} (fixed-length UTF-8 string).
\end{itemize}

% -------------------------
\subsection*{G.8. Content hashing and canonical serialization}
% -------------------------
Each artifact carries a content hash \texttt{sha256:...} over its canonical serialization
(JSON with sorted keys; HDF5 with \emph{fixed} dataset/attribute creation order, chunk shapes,
compression and filters). All floating datasets are little-endian IEEE-754 double (float64).
Cross-file links (\texttt{bars} \(\leftrightarrow\) \texttt{phi} \(\leftrightarrow\) \texttt{ext} \(\leftrightarrow\) \texttt{spec} \(\leftrightarrow\) \texttt{Lambda\_len})
use these hashes exclusively.
\emph{JSON numeric policy:} finite numbers only; positive infinity is encoded as the string
\texttt{"inf"} where applicable (see \texttt{bars.meta.infinity}). HDF5 encodes \(+\infty\) via the
\emph{split representation} \texttt{/bars/death} (float64) + \texttt{/bars/death\_is\_inf} (bool).
\emph{Strings:} all JSON strings and HDF5 string datasets/attributes are \emph{fixed-length} UTF-8
(\texttt{string\_encoding}=\texttt{"utf8-fixed"}) to ensure bitwise reproducibility.
\emph{HDF5 canonicalization:} set \texttt{track\_times=false}, \texttt{shuffle=false}, \texttt{fletcher32=false},
\texttt{fillvalue=0.0}, compression to GZIP level~4, and use the \texttt{chunk\_shapes} recorded in \texttt{run.yaml};
create datasets and attributes in the order shown in this appendix.

% -------------------------
\subsection*{G.9. Numeric tolerances, \(\delta\)-budgets, and audit trail}
% -------------------------
Every quantitative claim includes:
\begin{itemize}[leftmargin=1.25em]
\item \textbf{tolerance} (\texttt{tol}) declared in \texttt{run.yaml};
\item \textbf{slack} (\texttt{slack}) measured margin to the decision boundary;
\item \textbf{norm} used for spectral bounds (\texttt{fro} or \texttt{op}), consistent with Appendix~E;
\item \textbf{metric} for persistence distances (\texttt{interleaving} or \texttt{bottleneck});
\item \textbf{budget} aggregation: \texttt{operations[*].delta} entries quantale-sum into
\texttt{budget.sum\_delta}, with \texttt{budget.safety\_margin} and \texttt{budget.gap\_tau};
\item \textbf{layered} \(\delta\)-\textbf{ledger:} the stratified entries \texttt{layered\_delta.*} must sum to the
aggregate via \texttt{quantale.op};
\item \textbf{solver} details for spectral computations (Lanczos parameters, RNG seed);
\item \textbf{windows/coverage}: \texttt{windows.*} declare scopes; \texttt{coverage\_check.*} record pass/fail;
\item \textbf{gate} decision: \texttt{gate.accept} with \texttt{gate.reason}.
\end{itemize}

% -------------------------
\subsection*{G.10. Tests T14/T15 and reproducibility checklist}
% -------------------------
\paragraph{T14 (Overlap Gate gluing).}
On a windowed cover, verify:
(i) post-collapse equality (up to budget) on overlaps (\texttt{overlap\_checks.local\_equiv=true}),
(ii) Čech–\(\Ext^1\) acyclicity on overlaps (\texttt{cech\_ext1\_ok=true}),
(iii) stability band detection (\(\mu=\nu=0\), \texttt{stability\_band\_ok=true}),
(iv) A/B soft-commuting with logged residuals (added to the \(\delta\)-ledger),
(v) global gate acceptance with additive budgets.

\paragraph{T15 (Length spectrum audit).}
Compute \(\Lambda_{\mathrm{len}}(\T_\tau \Pfun_i(F);[0,\tau])\) and verify that its eigenvalue multiset equals the clipped bar-length multiset of \(\T_\tau \Pfun_i(F)\) (Appendix~H). Log either the eigenvalue list or a content hash under \texttt{Lambda\_len}, and ensure cross-links (\texttt{bars/spec/ext/phi}) resolve to the same hash.

\paragraph{Minimal reproducibility checklist.}
\begin{enumerate}[leftmargin=1.25em]
\item Preserve \texttt{run.yaml} and all emitted \texttt{bars/spec/ext/phi/Lambda\_len} files (JSON or HDF5).
\item Confirm \(\alpha>0\) (default \(\alpha=1\)) and field \(k\) are consistent across files.
\item Verify content hashes and all cross-links resolve; each artifact carries \texttt{meta.links.run\_id}
and \texttt{run\_yaml\_hash}; all artifacts provide a link to \texttt{Lambda\_len}.
\item Check that \texttt{pipeline.metric} matches \texttt{thresholds.tol.distance} (exactly one of
\texttt{interleaving}/\texttt{bottleneck}); norms/tolerances are consistent across \texttt{spec}.
\item Verify declared \textbf{windows} and that \texttt{coverage\_check.*} is true; for \texttt{definable}, check
\texttt{o\_minimal\_structure}, \texttt{window\_formulae}, and \texttt{cech\_depth\_bound}.
\item Verify eigenvalue \emph{order} metadata: \texttt{spec.meta.order}=\texttt{"ascending"},
\texttt{spec.meta.sorted}=true, and HDF5 eigen arrays are non-decreasing; if \texttt{tropical\_bins} are present,
confirm width/range match \texttt{tropical.bins}.
\item For Dirichlet Laplacians, confirm \(\lambda_{\min}>0\).
\item Ensure each recorded \(\theta\) used for \(N_\theta\) lies within \texttt{spec.meta.window.range}; recompute spectral indicators using recorded norm/tolerance/solver settings; check slack \(\ge 0\).
\item Re-evaluate \(\phi_{i,\tau}\) under the scope policy; confirm \((\mu,\nu)\) match generic-fiber counts (CNF after \(\T_\tau\)); check \texttt{iso\_tail.passed}; if \texttt{iwasawa} present, verify \texttt{kernel\_leq}, \texttt{cokernel\_leq}. Confirm listed \texttt{stability\_bands}.
\item For Ext-tests, verify amplitude \([-1,0]\), the assumption flags in \texttt{ext.meta.assumptions},
and \(\Ext^1(\mathcal{R}(C_\tau F),k)=0\); check \texttt{persistence.Ext1\_zero}.
\item Validate HDF5 canonicalization: chunk shapes, compression, filters, \texttt{string\_encoding="utf8-fixed"},
and \texttt{track\_times=false}; death is represented via the split float/bool fields.
\item Inspect \texttt{operations}, \texttt{budget}, and \texttt{layered\_delta}: the stratified sums match
\texttt{budget.sum\_delta} via \texttt{quantale.op}; compute safety margin and \texttt{gap\_tau}; verify \texttt{gate.accept}.
\item If \texttt{awfs\_2cell.awfs\_enabled}, verify recorded two-cell bounds and that \texttt{policy.after\_collapse\_only} holds.
\end{enumerate}

% -------------------------
\subsection*{G.11. Extended policy notes (concise)}
% -------------------------
\begin{itemize}[leftmargin=1.25em]
\item \textbf{Quantale.} The \texttt{quantale} section fixes the value-level monoid and order used to aggregate
distances and budgets; all layered \(\delta\) entries combine via \texttt{quantale.op}.
\item \textbf{Definable windows.} Right-open windows are specified by first-order formulae in the declared
o\hbox{-}minimal structure; the Čech depth bound controls gluing on overlaps.
\item \textbf{Layered \(\delta\).} The stratified ledger (\(\delta^{\mathrm{Gal}},\delta^{\mathrm{Tr}},\delta^{\mathrm{Fun}}\)) provides a MECE breakdown that must quantale-sum to the aggregate.
\item \textbf{Iwasawa control.} Finite bounds for kernel/cokernel dimensions at the recorded tower level certify control of \((\mu,\nu)\) along the tower.
\item \textbf{AWFS 2\hbox{-}cell.} The optional two-cell bounds certify functoriality/transport of \(C_\tau\) across mirrors/transfers.
\item \textbf{Safe low-pass.} The \texttt{low\_pass} block records kernel parity (even), unit mass, and alignment of \(\tau\) with the pipeline; these ensure non-expansiveness (Appendix~E).
\item \textbf{Tropical bins.} Optional diagnostic binning for aux-bars; if present, the same bin policy is mirrored in \texttt{spec.meta.tropical\_bins}.
\end{itemize}

\medskip
\noindent\textbf{Outcome.}
The versioned schemas above—now with (i) overlap checks for gluing, (ii) a windowed length spectrum audit,
(iii) canonical spectral policy (\texttt{order="ascending"}, \texttt{norm="op"|"fro"}, spectral bounds with safe low-pass),
(iv) δ-ledger extensions with \texttt{layered\_delta} and \texttt{gap\_tau}, (v) quantale/definable/Iwasawa/AWFS/tropical hooks,
(vi) explicit \texttt{stability\_bands}, and (vii) HDF5 canonicalization with fixed-length UTF-8 strings—
are sufficient to regenerate all figures and claims in the main text from first principles,
within the constructible regime and under the filtered-colimit policy, while making accept/reject criteria explicit and auditable.
No further supplementation is required for operational deployment or third-party review.



% =========================
\section*{Appendix H. Betti Integral and Finite \texorpdfstring{$\tau$}{tau}-Events (Reinforced)}
% =========================
\phantomsection
\addcontentsline{toc}{section}{Appendix H. Betti Integral and Finite $\tau$-Events}
\refstepcounter{section}
\label{H:betti-integral}

\paragraph{Standing conventions.}
We work over a field $k$. All persistence modules are constructible (locally finite on bounded windows); filtered colimits, when used, are taken under the scope policy of Appendix~A, Remark~\ref{A:rk:filtered-colimits}.
Endpoint conventions follow Appendix~A, Remark~\ref{A:rk:endpoints}; we use half-open bars $[b,d)$ (any consistent choice is immaterial below).
Global conventions: $\Ext^1$-tests are always against $k[0]$ (we write $\Ext^1(\mathcal{R}(C_\tau F),k)=0$, with $C_\tau$ understood up to f.q.i.\ on $\Ho(\mathsf{FiltCh}(k))$); the \emph{energy exponent} satisfies $\alpha>0$ (default $\alpha=1$).
Tilde references and type dashes (Type I--II / Type III / Type IV) are used uniformly.
All windowed and overlap claims are recorded in the manifest (\S\ref{G:repro}).

\subsection*{H.1. Betti curves and the Betti integral}
Let $F$ be a filtered chain complex (or a filtered object realizing a persistence module) with degree-$i$ persistence module $\mathbf{P}_i(F)$.
Write its barcode as a locally finite multiset
\[
\mathbf{P}_i(F)\ \cong\ \bigoplus_{I\in \mathcal{B}_i(F)} I^{\oplus m(I)},\qquad I=[b,d)\ \ \text{with } d\in\mathbb{R}\cup\{\infty\},\ m(I)\in\mathbb{Z}_{\ge 1}.
\]
Define the Betti curve $\beta_i(t):=\dim_k H_i(F^t)$ and the \emph{Betti integral} up to $\tau\ge 0$ by
\[
\mathrm{PE}_i^{\le \tau}(F)\ :=\ \int_{0}^{\tau} \beta_i(t)\,dt.
\]
Under constructibility, $\beta_i$ is right-continuous and piecewise constant, and on any bounded window only finitely many bars meet.

\begin{theorem}[Betti integral = clipped barcode mass]\label{H:thm:betti-integral}
For every $\tau\ge 0$,
\[
\mathrm{PE}_i^{\le \tau}(F)\ =\ \sum_{I\in \mathcal{B}_i(F)} m(I)\cdot \lambda\!\left(I\cap[0,\tau]\right),
\]
where $\lambda$ is Lebesgue measure and
\[
\lambda([b,d)\cap[0,\tau])=\max\{0,\min\{d,\tau\}-\max\{b,0\}\}\quad(\text{with }\min\{\infty,\tau\}=\tau).
\]
In particular, an infinite bar alive at $0$ contributes its clipped length $\tau$.
\end{theorem}

\begin{proof}
By local finiteness, for each bounded window $[0,\tau]$ only finitely many bars $I$ intersect the window, so
\[
f(t)\ :=\ \sum_{I\in\mathcal{B}_i(F)} m(I)\,\mathbf{1}_I(t)
\]
is a nonnegative measurable function on $[0,\tau]$ given by a finite sum. With the half-open convention $[b,d)$, for all $t$ away from event times (births $b$ and deaths $d$) we have
$\beta_i(t)=f(t)$
and at event times $\beta_i$ is right-continuous, hence equal to $f$ almost everywhere.
Thus, by Tonelli/Fubini on bounded windows,
\[
\int_{0}^{\tau}\!\beta_i(t)\,dt\ =\int_{0}^{\tau}\!f(t)\,dt\ =\sum_{I} m(I)\int_{0}^{\tau}\!\mathbf{1}_{I}(t)\,dt\ =\sum_{I} m(I)\,\lambda\!\left(I\cap[0,\tau]\right).
\]
\end{proof}

\begin{corollary}[Monotonicity, (a.e.) derivative, piecewise linearity]\label{H:cor:pl}
The map $\tau\mapsto \mathrm{PE}_i^{\le \tau}(F)$ is nondecreasing, continuous, and piecewise linear on every bounded interval. Its derivative satisfies
\[
\frac{d}{d\tau}\,\mathrm{PE}_i^{\le \tau}(F)=\beta_i(\tau)\quad\text{for a.e.\ }\tau,
\]
and at event points (births/deaths) the right derivative equals $\beta_i(\tau)$ while the left derivative equals $\beta_i(\tau-)$. All breakpoints on $[0,\tau_0]$ lie in $\{0,\tau_0\}\cup\{b\in[0,\tau_0]\}\cup\{d\in[0,\tau_0]\}$.
\end{corollary}

\begin{remark}[Endpoint and baseline conventions]\label{H:rk:endpoints}
Changing open/closed endpoint conventions modifies $\beta_i$ only on a set of measure zero; the integral and the breakpoint set remain unchanged. The baseline $0$ is a reference; negative births are allowed and handled by intersecting $I$ with $[0,\tau]$.
\end{remark}

\begin{remark}[Energy exponent and $\alpha$-Betti integral]\label{H:rk:alpha}
For $\alpha>0$, define the $\alpha$-Betti integral up to $\tau\ge 0$ by
\[
\mathrm{PE}_{i,\alpha}^{\le \tau}(F)\ :=\ \int_{0}^{\tau} \bigl(\beta_i(t)\bigr)^{\alpha}\,dt.
\]
On each component of $[0,\tau]$ between consecutive event times, $\beta_i$ is constant, hence $\mathrm{PE}_{i,\alpha}^{\le \tau}$ is still continuous, nondecreasing, and piecewise linear in $\tau$ (with slope $\bigl(\beta_i(\tau)\bigr)^{\alpha}$ on right-open pieces). The case $\alpha=1$ recovers Theorem~\ref{H:thm:betti-integral}. For $\alpha\neq 1$ there is no direct ``clipped mass'' formula, but all algorithmic and verification statements below remain valid verbatim (replace $\beta_i$ by $\beta_i^{\alpha}$ when computing segment slopes).
\end{remark}

\subsection*{H.2. Finite \texorpdfstring{$\tau$}{tau}-events and finite checking sets}
Fix $\tau_0>0$ and define the finite $\tau$-event set
\[
\mathsf{Ev}_i(F;\tau_0)\ :=\ \{0,\tau_0\}\ \cup\ \bigl(\{b\mid [b,d)\in\mathcal{B}_i(F)\}\cap[0,\tau_0]\bigr)\ \cup\ \bigl(\{d\mid [b,d)\in\mathcal{B}_i(F)\}\cap[0,\tau_0]\bigr).
\]
By constructibility, $\mathsf{Ev}_i(F;\tau_0)$ is finite.

\begin{proposition}[Finite checking set]\label{H:prop:finite-check}
Let $g:[0,\tau_0]\to\mathbb{R}$ be continuous and affine on each connected component of $[0,\tau_0]\setminus \mathsf{Ev}_i(F;\tau_0)$ (e.g.\ a piecewise linear benchmark with breakpoints in $\mathsf{Ev}_i$). Then, for either inequality direction,
\[
\mathrm{PE}_i^{\le \tau}(F)\ \ge\ g(\tau)\quad(\text{resp.\ }\le)\quad\text{for all }\tau\in[0,\tau_0]
\]
holds if and only if it holds for all $\tau\in \mathsf{Ev}_i(F;\tau_0)$.
\end{proposition}

\begin{proof}
Between consecutive event times, both $\mathrm{PE}_i^{\le \tau}(F)$ and $g(\tau)$ are affine in $\tau$ (Corollary~\ref{H:cor:pl}). Hence their difference $h(\tau):=\mathrm{PE}_i^{\le \tau}(F)-g(\tau)$ is affine on each closed component $J=[u,v]\subset[0,\tau_0]\setminus \mathsf{Ev}_i(F;\tau_0)$.
An affine function on a compact interval attains its extremum at an endpoint, so $h\ge 0$ (resp.\ $h\le 0$) on $J$ iff $h(u)\ge 0$ and $h(v)\ge 0$ (resp.\ $\le 0$). Taking the union over all such $J$ plus the singleton event points yields the claim.
\end{proof}

\begin{remark}[Definable piecewise constancy (o\mbox{-}minimal); finite $E_1$ checks]\label{H:rk:omin}
Under the definable windowing policy of Appendix~G (Declaration~\ref{dec:runyaml-extended}, \texttt{definable.o\_minimal\_structure}/\texttt{window\_formulae}), each $\beta_i$ on a bounded window is a definable, integer-valued function and hence piecewise constant with finitely many jumps. Equivalently, $\mathsf{Ev}_i(F;\tau_0)$ is finite and computable from definable data. Consequently, all constraints for $\mathrm{PE}_{i,\alpha}^{\le \tau}$ and all first-page ($E_1$) checks used by the Overlap Gate reduce to finitely many evaluations at $\tau\in\mathsf{Ev}_i(F;\tau_0)$; record the event list and counts in the manifest.
\end{remark}

\begin{remark}[Algorithmic evaluation]\label{H:rk:algo}
Algorithm for evaluating $\mathrm{PE}_{i,\alpha}^{\le \tau}$ on $[0,\tau_0]$:
\begin{enumerate}
  \item Collect all births and deaths intersecting $[0,\tau_0]$; sort to form $\mathsf{Ev}_i(F;\tau_0)=\{0=t_0<t_1<\cdots<t_M=\tau_0\}$.
  \item For each segment $[t_j,t_{j+1})$, compute $c_j:=\beta_i(t)$ for any $t\in[t_j,t_{j+1})$ and set $s_j:=c_j^{\alpha}$.
  \item For $\tau\in[t_j,t_{j+1})$,
  \[
  \mathrm{PE}_{i,\alpha}^{\le \tau}\ =\ \sum_{\ell<j} s_\ell\,(t_{\ell+1}-t_\ell)\ +\ s_j\,(\tau-t_j).
  \]
\end{enumerate}
Complexity: $O(M\log M)$ to form/sort $\mathsf{Ev}_i$ and $O(M)$ for accumulation. Record window definition and event counts in the manifest (Appendix~G).
\end{remark}

\subsection*{H.3. Consequences for shifts, truncations, and window variation}

\paragraph{(i) Equivariance under shifts.}
For $\varepsilon\in\mathbb{R}$, let $(S^\varepsilon F)^t:=F^{t+\varepsilon}$. Then $\beta_i^{S^\varepsilon}(t)=\beta_i(t+\varepsilon)$ and, for $\sigma\ge 0$,
\[
\mathrm{PE}_{i,\alpha}^{\le \sigma}(S^\varepsilon F)=\mathrm{PE}_{i,\alpha}^{\le \sigma+\varepsilon}(F)-\mathrm{PE}_{i,\alpha}^{\le \varepsilon}(F).
\]

\paragraph{(ii) Truncation monotonicity (deletion-type).}
Let $\mathbf{T}_{\tau'}$ denote bar-deletion at scale $\tau'>0$. Then, for every $\sigma>0$ and $\alpha>0$,
\[
\mathrm{PE}_{i,\alpha}^{\le \sigma}\!\big(\mathbf{T}_{\tau'}(\mathbf{P}_i(F))\big)\ \le\ \mathrm{PE}_{i,\alpha}^{\le \sigma}\!\big(\mathbf{P}_i(F)\big),
\]
since $\beta_i$ decreases pointwise under deletion. Moreover, $\mathbf{T}_{\tau'}$ is $1$-Lipschitz in interleaving distance (Appendix~A).

\paragraph{(iii) Lipschitz in the window parameter.}
For $0\le s\le \tau$ and $\alpha>0$,
\[
\bigl|\mathrm{PE}_{i,\alpha}^{\le \tau}(F)-\mathrm{PE}_{i,\alpha}^{\le s}(F)\bigr|=
\int_{s}^{\tau}\bigl(\beta_i(t)\bigr)^{\alpha}\,dt\ \le\ (\tau-s)\cdot \sup_{t\in[s,\tau]}\bigl(\beta_i(t)\bigr)^{\alpha}.
\]

\subsection*{H.4. Stability under interleavings and perturbations}

\begin{proposition}[Perturbation bound on bounded windows]\label{H:prop:bounded-stability}
Let $\tau_0>0$. Suppose two barcodes $\mathcal{B}_i(F)$ and $\mathcal{B}_i(G)$ are $\delta$-matched in the bottleneck sense: each matched pair $[b,d)\leftrightarrow[b',d')$ satisfies $|b-b'|\le\delta$, $|d-d'|\le\delta$ (with $d=\infty$ allowed), and unmatched bars (if any) have length $\le 2\delta$. Then, for all $\tau\in[0,\tau_0]$ and $\alpha>0$,
\[
\bigl|\mathrm{PE}_{i,\alpha}^{\le \tau}(F)-\mathrm{PE}_{i,\alpha}^{\le \tau}(G)\bigr|\ \le\ C_{i,\alpha}(\tau_0,\delta)\cdot \delta,
\]
where one can take
\[
C_{i,\alpha}(\tau_0,\delta)\ :=\ 2\,N_{i}([-\delta,\tau_0+\delta])\cdot \max\!\Bigl\{1,\ \sup_{t\in[-\delta,\tau_0+\delta]}\bigl(\beta_i^F(t)\bigr)^{\alpha-1},\ \sup_{t\in[-\delta,\tau_0+\delta]}\bigl(\beta_i^G(t)\bigr)^{\alpha-1}\Bigr\}.
\]
In particular, for $\alpha=1$,
\[
\bigl|\mathrm{PE}_{i}^{\le \tau}(F)-\mathrm{PE}_{i}^{\le \tau}(G)\bigr|\ \le\ 2\,\delta\cdot N_{i}([-\delta,\tau_0+\delta]).
\]
\end{proposition}

\begin{proof}[Proof sketch]
By Theorem~\ref{H:thm:betti-integral}, $\mathrm{PE}_i^{\le \tau}$ equals the sum of clipped lengths on $[0,\tau]$. Endpoint $\delta$-perturbations change any matched clipped length by $\le 2\delta$; unmatched bars have length $\le 2\delta$. Summing over bars meeting $[-\delta,\tau_0+\delta]$ yields the case $\alpha=1$. For $\alpha\neq 1$, control the slope change $|c^\alpha-{c'}^\alpha|\le \alpha\max\{c,c'\}^{\alpha-1}|c-c'|$ on each constant segment; the number of crossings on $[-\delta,\tau_0+\delta]$ is bounded by $N_i([-\delta,\tau_0+\delta])$.
\end{proof}

\subsection*{H.5. Implementation notes and numerics}
For large barcodes, the following practices improve reproducibility and numerical stability:
\begin{enumerate}
  \item Event extraction: derive $\mathsf{Ev}_i(F;\tau_0)$ directly from the barcode; for streamed persistence, emit births and deaths as they occur and maintain a running count $c_j$.
  \item Accumulation: use compensated summation (e.g.\ Kahan) when aggregating $s_j\,(t_{j+1}-t_j)$.
  \item Types: store event times as float64; store counts $c_j$ as 64-bit integers; compute slopes as float64.
  \item Idempotence: with $[b,d)$, repeated evaluation on the same event sequence is bitwise deterministic (fixed sort/tie-break rules).
  \item Window policy: record in the manifest (Appendix~G) the baseline, window $[0,\tau_0]$, endpoint convention (right-open), and whether negative births are present.
\end{enumerate}

\subsection*{H.6. Testing and validation}
Minimal tests:
\begin{enumerate}
  \item Synthetic bars: verify Theorem~\ref{H:thm:betti-integral} and cross-check numerical integration vs.\ clipped-length summation.
  \item Endpoint consistency: switch between $[b,d)$ and $(b,d]$ conventions and verify identical $\mathrm{PE}_{i,\alpha}^{\le \tau}$ and breakpoints.
  \item Shift equivariance: random $\varepsilon$; check $\mathrm{PE}_{i,\alpha}^{\le \sigma}(S^\varepsilon F)=\mathrm{PE}_{i,\alpha}^{\le \sigma+\varepsilon}(F)-\mathrm{PE}_{i,\alpha}^{\le \varepsilon}(F)$.
  \item Truncation monotonicity: apply $\mathbf{T}_{\tau'}$; verify pointwise decrease in $\sigma$.
  \item Stability: simulate $\delta$-endpoint perturbations; confirm Proposition~\ref{H:prop:bounded-stability}.
\end{enumerate}

\subsection*{H.7. Variants and generalizations}
\begin{itemize}
  \item Weighted windows: for nonnegative $w\in L^1([0,\tau_0])$,
  \[
  \mathrm{PE}_{i,\alpha}^{w}(F)\ :=\ \int_{0}^{\tau_0} w(t)\,\bigl(\beta_i(t)\bigr)^{\alpha}\,dt\ =\ \sum_{I} m(I)\!\int_{I\cap[0,\tau_0]} \!w(t)\,\bigl(\beta_i(t)\bigr)^{\alpha-1}\,dt,
  \]
  with the same finite-check reduction when $w$ is piecewise constant with breaks in $\mathsf{Ev}_i(F;\tau_0)$.
  \item Alternate baselines: replacing $[0,\tau]$ by $[a,b]$ gives
  $\mathrm{PE}_i^{[a,b]}(F)=\sum_I m(I)\,\lambda(I\cap[a,b])$.
  \item Discrete filtrations: replace integrals by Riemann sums on a grid; all statements adapt with counting measure.
\end{itemize}

\subsection*{H.8. $E_1$-level determinacy and the reverse bridge prerequisites}

\paragraph{Determinacy of $E_1=0$ on finite $\tau$-events.}
In the Overlap Gate, the first-page condition $E_1=0$ (vanishing of the window-local obstruction measured by $\beta_i$-based functionals) is \emph{decidable} from finitely many $\tau$ values:
by Proposition~\ref{H:prop:finite-check} and Remark~\ref{H:rk:omin}, any inequality of the form
\[
\mathrm{PE}_{i,\alpha}^{\le \tau}(F)\ \le\ g(\tau)\quad\text{or}\quad \ge g(\tau)
\]
with $g$ affine on components between events holds on $[0,\tau_0]$ iff it holds at $\tau\in\mathsf{Ev}_i(F;\tau_0)$.
Thus $E_1=0$ can be certified by a \emph{finite checklist} recorded in \texttt{run.yaml} (Appendix~G).

\paragraph{Reverse bridge (window-local) and its prerequisites.}
Let $\mathcal{R}$ be $t$-exact of amplitude $[-1,0]$ (Appendix~C) and let $C_\tau$ be the collapse on filtered complexes (\S F.5).
On a window satisfying the saturation and no-accumulation hypotheses of Appendix~D (Local Equiv), there are natural isomorphisms
\[
H^{-1}\!\bigl(\mathcal{R}(C_\tau F)\bigr)\ \simeq\ \varinjlim_{t\in W} H_1(F^t),\qquad
\Ext^1\!\bigl(\mathcal{R}(C_\tau F),k\bigr)\ \simeq\ \Hom\!\bigl(H^{-1}(\mathcal{R}(C_\tau F)),k\bigr).
\]
Over a field $k$, $\Hom(-,k)$ is faithful on finite-dimensional vector spaces, hence:
\[
\Ext^1\!\bigl(\mathcal{R}(C_\tau F),k\bigr)=0\quad\Longleftrightarrow\quad H^{-1}\!\bigl(\mathcal{R}(C_\tau F)\bigr)=0\quad\Longleftrightarrow\quad \mathrm{PH}_1(F|_W)=0.
\]
\emph{Prerequisites:} constructibility; amplitude $\le 1$; window-local tail isomorphisms (Appendix~D); and post-collapse evaluation (Appendix~A--F).
\emph{Policy note:} the pipeline \emph{uses} only the forward implication $\mathrm{PH}_1\Rightarrow \Ext^1$ for gating; the reverse equivalence is recorded here for completeness and audit when the stated prerequisites are verified and logged (Appendix~G).

\medskip
\noindent\textbf{Summary.}
The Betti integral equals the clipped barcode mass (Theorem~\ref{H:thm:betti-integral}); hence $\mathrm{PE}_{i,\alpha}^{\le \tau}$ is continuous, nondecreasing, and piecewise linear with breakpoints among births/deaths (Corollary~\ref{H:cor:pl}, Remark~\ref{H:rk:alpha}).
Any affine-on-components constraint reduces to the finite event set (Proposition~\ref{H:prop:finite-check}); under the o\mbox{-}minimal definable policy (Appendix~G), these events are \emph{a priori} finite and computable (Remark~\ref{H:rk:omin}), so $E_1$-level checks in Overlap Gate become finite lists of evaluations.
Deletion-type truncations make $\mathrm{PE}_{i,\alpha}^{\le \tau}$ nonincreasing and preserve $1$-Lipschitz stability (Appendix~A). Windowed perturbation bounds (Proposition~\ref{H:prop:bounded-stability}) yield practical stability.
Finally, under the recorded prerequisites (constructible, amplitude $\le 1$, tail isomorphisms, after-collapse), the reverse bridge holds window-locally and globally by gluing (Appendix~G), though the operational gate remains one-way by design.



% =========================
\section*{Appendix I. \texorpdfstring{$\varepsilon$}{epsilon}-Survival Lemma and Grid-to-Continuum [Proof/Spec] (reinforced)}
% =========================
\phantomsection
\addcontentsline{toc}{section}{Appendix I. $\varepsilon$-Survival Lemma and Grid-to-Continuum}
\refstepcounter{section}
\label{I:eps-survival}

\paragraph{Standing conventions.}
We work over a field \(k\). All persistence modules are constructible (locally finite on bounded windows). Any use of filtered colimits follows the scope policy of Appendix~A, Remark~\ref{A:rk:filtered-colimits}. Global conventions: \(\Ext^1\)-tests are always against \(k[0]\) (we write \(\Ext^1(\mathcal{R}(C_\tau F),k)=0\), with \(C_\tau\) understood up to f.q.i.\ on \(\Ho(\mathsf{FiltCh}(k))\)); the energy exponent satisfies \(\alpha>0\) (default \(\alpha=1\)); type dashes (Type~I--II / Type~III / Type~IV) are used uniformly. We write \(d_{\mathrm{int}}\) for the interleaving metric (which equals the bottleneck distance in the 1D constructible case). Window policy and right-attribution conventions follow Appendix~G. All survival/stability statements are applied \emph{after} the mandated order \(\mathbf{P}_i\to \mathbf{T}_\tau\) (Appendix~G).

\begin{remark}[Endpoint conventions]\label{I:rk:endpoints}
Intervals are taken half-open \([b,d)\) with \(d\in\mathbb{R}\cup\{\infty\}\). Any consistent open/closed choice yields the same clipped lengths and event sets; all statements below are invariant under this choice.
\end{remark}

% -------------------------
\subsection*{I.1. Window clipping and nonexpansivity}
% -------------------------

Let \(\Pers^{\mathrm{ft}}_k\) denote constructible persistence modules on \((\mathbb{R},\le)\).
For \(\tau\ge 0\), write \(i_{\le\tau}:[0,\tau]\hookrightarrow\mathbb{R}\) for the inclusion and define the \emph{window clip}
\[
\mathbf{W}_{\le\tau}\ :=\ (i_{\le\tau})^{0}_!\circ i_{\le\tau}^{\ast}:\ \Pers^{\mathrm{ft}}_k\longrightarrow \Pers^{\mathrm{ft}}_k,
\]
restriction to \([0,\tau]\) followed by extension by \(0\). On barcodes, \(\mathbf{W}_{\le\tau}\) is intersection with \([0,\tau]\) (empties discarded). For a bar \(I=[b,d)\), its clipped length on \([0,\tau_0]\) is
\[
\ell_{[0,\tau_0]}(I)=\max\{0,\min\{d,\tau_0\}-\max\{b,0\}\}.
\]

\begin{lemma}[Clipping is \(1\)-Lipschitz]\label{I:lem:clip-1lip}
For all \(M,N\) and \(\tau\ge 0\),
\[
d_{\mathrm{int}}\bigl(\mathbf{W}_{\le\tau}M,\ \mathbf{W}_{\le\tau}N\bigr)\ \le\ d_{\mathrm{int}}(M,N).
\]
\end{lemma}

\begin{remark}[Shifts commute with clipping]\label{I:rk:shift-clip}
For every \(\varepsilon\ge 0\), \(S^\varepsilon\circ \mathbf{W}_{\le\tau}\cong \mathbf{W}_{\le\tau}\circ S^\varepsilon\). Thus interleavings transport through \(\mathbf{W}_{\le\tau}\).
\end{remark}

% -------------------------
\subsection*{I.2. Survival under \texorpdfstring{$\varepsilon$}{epsilon}-interleavings}
% -------------------------

\begin{lemma}[\(\varepsilon\)-survival; sharp two-sided form]\label{I:lem:survive-2eps}
Let \(M,N\in\Pers^{\mathrm{ft}}_k\) with \(d_{\mathrm{int}}(M,N)\le\varepsilon\) and fix \(\tau_0>0\).
\begin{enumerate}\itemsep0.25em
\item If a bar \(I\) in \(M\) is matched to \(J\) in \(N\), then
\(\ \ell_{[0,\tau_0]}(J)\ \ge\ \max\{\ell_{[0,\tau_0]}(I)-2\varepsilon,0\}.\)
\item If \(\ell_{[0,\tau_0]}(I)>2\varepsilon\), then its partner \(J\) satisfies \(\ell_{[0,\tau_0]}(J)>0\).
\item (\emph{Multiplicity}) If at least \(r\) bars in \(M\) have \(\ell_{[0,\tau_0]}>2\varepsilon\), then \(\mathbf{W}_{\le\tau_0}N\) has at least \(r\) nonzero bars (with multiplicity).
\end{enumerate}
\end{lemma}

\begin{proof}
In 1D, \(d_{\mathrm{int}}=d_B\) admits an \(\varepsilon\)-matching of endpoints; clipping to \([0,\tau_0]\) changes length by at most \(\varepsilon\) at each end, giving the \(2\varepsilon\) bound and the consequences.
\end{proof}

\begin{remark}[After-collapse composition]\label{I:rk:after-collapse}
The reflector \(\mathbf{T}_\tau\) is \(1\)-Lipschitz for \(d_{\mathrm{int}}\) (Appendix~A). Hence composing Lemma~\ref{I:lem:survive-2eps} with \(\mathbf{T}_\tau\) (before or after \(\mathbf{W}_{\le\tau_0}\)) preserves all inequalities.
\end{remark}

% -------------------------
\subsection*{I.3. Grid-to-continuum transfer}
% -------------------------

Let \(F\) be filtered with degree-\(i\) persistence \(\mathbf{P}_i(F)\), and \(F_h\) its discretization with
\[
d_{\mathrm{int}}\bigl(\mathbf{P}_i(F_h),\ \mathbf{P}_i(F)\bigr)\ \le\ \varepsilon(h).
\]

\begin{theorem}[Grid-to-continuum survival]\label{I:thm:g2c}
Fix \(\tau_0>0\), \(r\in\mathbb{Z}_{\ge 1}\), and \(\eta>0\). If \(\mathbf{W}_{\le\tau_0}\mathbf{P}_i(F_h)\) has at least \(r\) bars of clipped length \(\ge 2\varepsilon(h)+\eta\), then \(\mathbf{W}_{\le\tau_0}\mathbf{P}_i(F)\) has at least \(r\) nonzero bars, each of clipped length \(\ge \eta\).
\end{theorem}

\begin{proof}
Apply Lemma~\ref{I:lem:survive-2eps} with \(\varepsilon=\varepsilon(h)\).
\end{proof}

% -------------------------
\subsection*{I.4. Budget-adjusted and \texorpdfstring{$V$}{V}-metric variants}
% -------------------------

We now fold in pipeline budgets (Appendix~G) and a \(V\)-enriched nonexpansive distance (Appendices~E–F).

\paragraph{Setup.}
Let \(d_V\) be a Lawvere/quantale distance on \(\Pers^{\mathrm{ft}}_k\) for which both \(\mathbf{W}_{\le\tau}\) and \(\mathbf{T}_\tau\) are \(1\)-Lipschitz. Let \(W=[0,\tau_0]\) be the window and write the additive \(\delta\)-ledger as \(\Delta_W:=\sum \delta\) (sum over \texttt{operations[*].delta} recorded for \(W\) in the manifest). Define the \emph{effective radius}
\[
\boxed{\ \varepsilon_{\mathrm{eff}}\ :=\ d_V\bigl(\mathbf{P}_i(F_h),\mathbf{P}_i(F)\bigr)\ +\ \Delta_W\ }.
\]

\begin{lemma}[\(\varepsilon\)-survival, budgeted \(V\)-metric]\label{I:lem:eps-clip-V}
With notation as above, for any bar \(I\) in \(\mathbf{W}_{\le\tau_0}\mathbf{P}_i(F_h)\):
\begin{enumerate}\itemsep0.25em
\item (\emph{Two-sided, sharp}) If \(\ell_{[0,\tau_0]}(I)>\,2\,\varepsilon_{\mathrm{eff}}\), then its match in \(\mathbf{W}_{\le\tau_0}\mathbf{P}_i(F)\) is nonzero; moreover the clipped length is \(\ge \ell_{[0,\tau_0]}(I)-2\,\varepsilon_{\mathrm{eff}}\).
\item (\emph{One-sided improvement; optional}) If the implementation anchors births (death-side drift only), then the nonvanishing threshold improves to \(\varepsilon_{\mathrm{eff}}\) (and the residual margin degrades by at most \(\varepsilon_{\mathrm{eff}}\)).
\end{enumerate}
\end{lemma}

\begin{proof}
Nonexpansivity of \(\mathbf{W}_{\le\tau_0}\) and \(\mathbf{T}_\tau\) in \(d_V\) pushes the bound \(d_V\le \varepsilon_{\mathrm{eff}}\) through these functors. Two-sided endpoint motion yields the factor \(2\). If births are anchored and only the death coordinate moves by at most \(\varepsilon_{\mathrm{eff}}\), the degradation is one-sided.
\end{proof}

\begin{corollary}[Budgeted grid-to-continuum]\label{I:cor:g2c-V}
If at least \(r\) clipped bars in \(\mathbf{W}_{\le\tau_0}\mathbf{P}_i(F_h)\) have length \(\ge 2\varepsilon_{\mathrm{eff}}+\eta\), then \(\mathbf{W}_{\le\tau_0}\mathbf{P}_i(F)\) has at least \(r\) nonzero bars of clipped length \(\ge \eta\). Under the one-sided hypothesis, replace \(2\) by \(1\).
\end{corollary}

\begin{remark}[Tropical bins and aux-bars]
If a diagnostic bin width \(\beta>0\) is declared (Appendix~E), the cumulative-profile shift is bounded by \(q=\lceil \varepsilon_{\mathrm{eff}}/\beta\rceil\) bins:
this is the persistence analogue of the spectral bin-shift policy; record \texttt{eps\_cont\_shift\_bins} \(=q\) in the manifest.
\end{remark}

\begin{remark}[Logging (minimal contract)]
Record: \texttt{pipeline.metric} or the \(V\)-metric name; \texttt{thresholds.tol.distance}; window \(W=[0,\tau_0]\);
the contributing \(\delta\)-entries and \(\Delta_W\); counts of bars above \((2)\varepsilon_{\mathrm{eff}}+\eta\) (or \(\varepsilon_{\mathrm{eff}}+\eta\) in the one-sided mode); and whether the “after-collapse only” policy was enforced (Appendix~G).
\end{remark}

\begin{corollary}[Betti-integral lower bound (link to Appendix~H)]
If \(\mathbf{W}_{\le\tau_0}\mathbf{P}_i(F_h)\) contains \(r\) bars of clip-length \(\ge 2\varepsilon_{\mathrm{eff}}+\eta\), then
\[
\mathrm{PE}_i^{\le \tau_0}(F)\ \ge\ r\,\eta,
\]
by Theorem~\ref{H:thm:betti-integral} and Corollary~\ref{I:cor:g2c-V}.
\end{corollary}

% -------------------------
\subsection*{I.5. Variants, sharpness, and towers}
% -------------------------

\begin{itemize}\itemsep0.25em
\item \emph{Sharpness.} The constant \(2\) in Lemma~\ref{I:lem:survive-2eps} is optimal: a bar of clip-length \(2\varepsilon\) can be shifted by \(\varepsilon\) at both ends to collapse on \([0,\tau_0]\).
\item \emph{After-collapse pipeline.} Since \(\mathbf{T}_\tau\) is \(1\)-Lipschitz (Appendix~A), composing with \(\mathbf{T}_\tau\) leaves the thresholds in Lemmas~\ref{I:lem:survive-2eps} and \ref{I:lem:eps-clip-V} unchanged.
\item \emph{Towers.} If \(d_V(\mathbf{P}_i(F_n),\mathbf{P}_i(F_\infty))\le\varepsilon_n\to 0\), any fixed positive margin propagates from grid to continuum by Corollary~\ref{I:cor:g2c-V}. This integrates with the tower diagnostics in Appendix~D and the \((\mu,\nu)\)-stability bands.
\end{itemize}

% -------------------------
\subsection*{I.6. Formalization stubs (Lean/Coq) [Spec]}
% -------------------------

\begin{verbatim}
-- Lean-style pseudocode (schematic)
namespace AK.I
open scoped Classical
noncomputable section

/-- Constructible persistence modules (schematic). -/
structure PersModule := (bars : Type)  -- placeholder witnesses of constructibility

def W_le (τ : ℝ≥0) : PersModule → PersModule := sorry
def Tτ  (τ : ℝ≥0) : PersModule → PersModule := sorry

/-- V-distance (quantale-style). -/
def dV : PersModule → PersModule → ℝ≥0 := sorry
axiom W_le_nonexpansive : ∀ τ M N, dV (W_le τ M) (W_le τ N) ≤ dV M N
axiom Tτ_nonexpansive  : ∀ τ M N, dV (Tτ  τ M) (Tτ  τ N) ≤ dV M N

/-- Clipped length of a bar on [0, τ0]. -/
constant Bar : Type
def clipLen_on (τ0 : ℝ≥0) (I : Bar) : ℝ≥0 := sorry
constant BarOf : PersModule → Type    -- bars of a given module

theorem eps_survival_two_sided
  {M N} (τ0 ε Δ : ℝ≥0) (h : dV M N ≤ ε + Δ)
  (I : BarOf M) (hI : clipLen_on τ0 I > 2*(ε+Δ)) :
  ∃ J : BarOf N, clipLen_on τ0 J ≥ clipLen_on τ0 I - 2*(ε+Δ) := by
  -- push h through W_le; endpoint motion ≤ ε+Δ on each side
  admit

theorem eps_survival_one_sided  -- optional death-side control
  {M N} (τ0 ε Δ : ℝ≥0) (h : dV M N ≤ ε + Δ) (births_anchored : True)
  (I : BarOf M) (hI : clipLen_on τ0 I > (ε+Δ)) :
  ∃ J : BarOf N, 0 < clipLen_on τ0 J := by
  admit
end AK.I
\end{verbatim}

% -------------------------
\subsection*{I.7. Summary}
% -------------------------

Clipping \(\mathbf{W}_{\le\tau}\) is restriction to \([0,\tau]\) followed by \(0\)-extension; it preserves constructibility and is \(1\)-Lipschitz (Lemma~\ref{I:lem:clip-1lip}). Under an \(\varepsilon\)-interleaving, clipped lengths degrade by at most \(2\varepsilon\); thus any grid-detected bar with clip-length \(>\!2\varepsilon\) persists in the continuum window (Lemma~\ref{I:lem:survive-2eps}, Theorem~\ref{I:thm:g2c}). In the budgeted \(V\)-metric setting, replace \(\varepsilon\) by the \emph{effective} radius
\[
\varepsilon_{\mathrm{eff}}=d_V\bigl(\mathbf{P}_i(F_h),\mathbf{P}_i(F)\bigr)+\textstyle\sum\delta,
\]
so the survival threshold becomes \(>\!2\,\varepsilon_{\mathrm{eff}}\) (or \(>\!\varepsilon_{\mathrm{eff}}\) under death-side control). These statements commute with the after-collapse policy (\(\mathbf{T}_\tau\) is \(1\)-Lipschitz), tie into Betti-integral lower bounds (Appendix~H), and are logged via the \(\delta\)-ledger and window fields in Appendix~G. No further supplementation is required for operational deployment or audit.



% =========================
\section*{Appendix J. Calculus of $\mu,\nu$ [Proof + Stability Bands + Window Pasting] (reinforced)}
% =========================
\phantomsection
\addcontentsline{toc}{section}{Appendix J. Calculus of $\mu,\nu$}
\refstepcounter{section}
\label{J:calc}

\paragraph{Standing conventions.}
We work over a field \(k\).
All persistence modules are constructible (locally finite on bounded windows).
Filtered colimits are computed in the functor category \([\mathbb{R},\mathsf{Vect}_k]\) under the scope policy of Appendix~A, Remark~\ref{A:rk:filtered-colimits}, and then (when stated) returned to the constructible range.
The reflection \(\mathbf{T}_\tau\dashv \iota_\tau\) is exact and \(1\)-Lipschitz (Appendix~A, Theorem~\ref{A:thm:localization} and Proposition~\ref{A:prop:lipschitz}).
Global conventions: \(\Ext^1\) is always against \(k[0]\); the energy exponent \(\alpha>0\) (default \(\alpha=1\)).
All window policies follow the MECE (mutually exclusive–collectively exhaustive), right-open convention used throughout (Appendix~G).

\paragraph{Setup and notation.}
Fix \(i\in\mathbb{Z}\).
Let \(F=(F_n)_{n\in I}\) be a directed system (``tower'') of filtered objects for which \(\mathbf{P}_i(F_n)\in\Pers^{\mathrm{ft}}_k\), and let \(F_\infty\) be its colimit with cocone maps \(F_n\to F_\infty\).
For \(\tau\ge 0\) consider the \emph{comparison map} in the functor category
\begin{equation}\label{J:eq:phi}
\phi_{i,\tau}(F):\quad
\varinjlim_{n}\ \mathbf{T}_\tau\big(\mathbf{P}_i(F_n)\big)\ \longrightarrow\ \mathbf{T}_\tau\big(\mathbf{P}_i(F_\infty)\big).
\end{equation}
Define
\begin{equation}\label{J:eq:mu-nu}
\mu_{i,\tau}(F)\ :=\ \dim_k \ker \phi_{i,\tau}(F),\qquad
\nu_{i,\tau}(F)\ :=\ \dim_k \operatorname{coker} \phi_{i,\tau}(F),
\end{equation}
where \(\dim_k\) denotes the \emph{generic fiber} dimension in \(\Pers^{\mathrm{ft}}_k\) (i.e.\ the stabilized right-tail dimension \(\lim_{t\to+\infty}\dim_k(-)(t)\), equivalently the multiplicity of \(I[0,\infty)\) after applying \(\mathbf{T}_\tau\); cf.\ Appendix~D, Remark~\ref{rem:D-generic-dim}).\footnote{For a constructible module \(M\), the generic fiber dimension equals \(\dim_k M(t)\) for all sufficiently large \(t\), which stabilizes. After applying \(\mathbf{T}_\tau\), it coincides with the multiplicity of \(I[0,\infty)\) summands. Kernels/cokernels are taken in \(\Pers^{\mathrm{ft}}_k\), where they decompose into finite direct sums of interval modules.}
We also write the \emph{totals}
\begin{equation}\label{J:eq:totals}
\muc(F)\ :=\ \sum_{i}\mu_{i,\tau}(F),\qquad
\nuc(F)\ :=\ \sum_{i}\nu_{i,\tau}(F),
\end{equation}
which are finite because the complexes are bounded in homological degrees (constructible range).
All quantities depend on \(\tau\); no general monotonicity in \(\tau\) is asserted.

% -------------------------
\subsection*{J.1. Functoriality and composition}
% -------------------------

\begin{definition}[Morphisms of towers]
A morphism \(u:F\to G\) consists of maps \(u_n:F_n\to G_n\) commuting with the structure maps (in the index category) and inducing a canonical map on colimits \(u_\infty:F_\infty\to G_\infty\).
Given \(u:F\to G\) and \(v:G\to H\), the composite \(v\circ u:F\to H\) is defined degreewise.
\end{definition}

\begin{lemma}[Functoriality of comparison maps]\label{J:lem:functoriality}
Under the scope policy, each morphism \(u:F\to G\) induces a canonical morphism of comparison maps
\[
\phi_{i,\tau}(u):\ \phi_{i,\tau}(F)\ \Longrightarrow\ \phi_{i,\tau}(G),
\]
natural in both \(i\) and \(\tau\), and for composable \(u,v\),
\(\ \phi_{i,\tau}(v\circ u)=\phi_{i,\tau}(v)\circ\phi_{i,\tau}(u).\)
\end{lemma}

\begin{proof}
Apply \(\mathbf{P}_i\), then \(\mathbf{T}_\tau\), to the systems and pass to filtered colimits in \([\mathbb{R},\mathsf{Vect}_k]\); exactness of \(\mathbf{T}_\tau\) and functoriality of colimits give naturality and compatibility with composition.
\end{proof}

% -------------------------
\subsection*{J.2. Subadditivity under composition (P6) — full proof}
% -------------------------

\begin{theorem}[Subadditivity under composition (P6)]\label{J:thm:subadd}
Let \(u:F\to G\) and \(v:G\to H\) be morphisms of towers and fix \(\tau\ge 0\).
Then, for every \(i\in\mathbb{Z}\),
\[
\mu_{i,\tau}(v\circ u)\ \le\ \mu_{i,\tau}(u)\ +\ \mu_{i,\tau}(v),\qquad
\nu_{i,\tau}(v\circ u)\ \le\ \nu_{i,\tau}(u)\ +\ \nu_{i,\tau}(v).
\]
\end{theorem}

\begin{proof}
Write \(X:=\varinjlim_n \mathbf{T}_\tau\mathbf{P}_i(F_n)\), \(Y:=\varinjlim_n \mathbf{T}_\tau\mathbf{P}_i(G_n)\), \(Z:=\varinjlim_n \mathbf{T}_\tau\mathbf{P}_i(H_n)\), and let
\[
f:X\to Y,\qquad g:Y\to Z
\]
be the morphisms induced by \(u,v\) (Lemma~\ref{J:lem:functoriality}), and
\[
f_\infty:\mathbf{T}_\tau\mathbf{P}_i(F_\infty)\to \mathbf{T}_\tau\mathbf{P}_i(G_\infty),\quad
g_\infty:\mathbf{T}_\tau\mathbf{P}_i(G_\infty)\to \mathbf{T}_\tau\mathbf{P}_i(H_\infty)
\]
their apex maps.
By exactness of \(\mathbf{T}_\tau\) and constructibility, kernels and cokernels of the comparison maps
\[
\phi(u):X\to \mathbf{T}_\tau\mathbf{P}_i(F_\infty),\quad
\phi(v):Y\to \mathbf{T}_\tau\mathbf{P}_i(G_\infty),\quad
\phi(v\circ u):X\to \mathbf{T}_\tau\mathbf{P}_i(H_\infty)
\]
are constructible modules, and their generic fiber dimensions equal the stabilized dimensions of the pointwise linear algebra at large parameters \(t\gg 0\).

Fix such a large \(t\) (common stabilization scale for all objects).
Evaluating at \(t\) turns all objects into finite-dimensional \(k\)-spaces and the three comparison maps into linear maps
\[
\phi(u)_t:X_t\to (F_\infty)_t,\quad
\phi(v)_t:Y_t\to (G_\infty)_t,\quad
\phi(v\circ u)_t:X_t\to (H_\infty)_t,
\]
with \(\phi(v\circ u)_t=(g_\infty)_t\circ \phi(v)_t\circ f_t\) functorially.

\emph{Kernel bound.}
For linear maps \(A:U\to V\), \(B:V\to W\) one has the exact sequence
\[
0\longrightarrow \ker A \longrightarrow \ker (B\!\circ\! A)\xrightarrow{A} \ker B \cap \operatorname{Im}A \longrightarrow 0,
\]
whence \(\dim\ker (B\!\circ\!A)=\dim\ker A+\dim(\ker B\cap \operatorname{Im}A)\le \dim\ker A+\dim\ker B\).
Apply this with \(A:=\phi(v)_t\circ f_t\) and \(B:=(g_\infty)_t\) to conclude
\[
\dim\ker \phi(v\circ u)_t\ \le\ \dim\ker(\phi(v)_t\circ f_t)+\dim\ker(g_\infty)_t
\ \le\ \dim\ker \phi(u)_t+\dim\ker \phi(v)_t,
\]
where the last inequality uses the same argument for the factorization through \(\phi(u)_t\) and \(\phi(v)_t\).
Passing to the stabilized (generic) dimensions yields
\(\ \mu_{i,\tau}(v\circ u)\le \mu_{i,\tau}(u)+\mu_{i,\tau}(v)\).

\emph{Cokernel bound.}
For linear maps \(A:U\to V\), \(B:V\to W\), duality gives
\[
\dim\operatorname{coker}(B\!\circ\!A)
=\dim\ker((B\!\circ\!A)^*)=\dim\ker(A^*\!\circ\!B^*)\le \dim\ker A^*+\dim\ker B^*
=\dim\operatorname{coker}A+\dim\operatorname{coker}B.
\]
Apply this with \(A:=\phi(v)_t\circ f_t\), \(B:=(g_\infty)_t\) and the same factorization argument to conclude
\(\ \dim\operatorname{coker}\phi(v\circ u)_t\le \dim\operatorname{coker}\phi(u)_t+\dim\operatorname{coker}\phi(v)_t\).
Taking stabilized dimensions gives \(\ \nu_{i,\tau}(v\circ u)\le \nu_{i,\tau}(u)+\nu_{i,\tau}(v)\).
\end{proof}

\begin{remark}[Metric-free nature]
The proof uses only exactness after \(\mathbf{T}_\tau\) and constructibility (stabilization), hence holds verbatim in the \(V\)-enriched regime used elsewhere (Appendices~E–F): no metric bounds are needed.
\end{remark}

% -------------------------
\subsection*{J.3. Additivity under finite direct sums}
% -------------------------

\begin{proposition}[Direct-sum additivity]\label{J:prop:sum}
Let \(F=F^{(1)}\oplus F^{(2)}\) be the levelwise direct sum of towers (same index category) and similarly for the colimit.
Then for every \(\tau\ge 0\),
\[
\mu_{i,\tau}(F)=\mu_{i,\tau}\!\big(F^{(1)}\big)+\mu_{i,\tau}\!\big(F^{(2)}\big),\qquad
\nu_{i,\tau}(F)=\nu_{i,\tau}\!\big(F^{(1)}\big)+\nu_{i,\tau}\!\big(F^{(2)}\big),
\]
and therefore \(\muc,\nuc\) are additive as well.
\end{proposition}

\begin{proof}
\(\mathbf{P}_i\) and \(\mathbf{T}_\tau\) preserve finite direct sums; filtered colimits commute with finite direct sums in \([\mathbb{R},\mathsf{Vect}_k]\).
Thus \(\phi_{i,\tau}(F)\) is block-diagonal with blocks \(\phi_{i,\tau}(F^{(1)})\), \(\phi_{i,\tau}(F^{(2)})\); kernels/cokernels and their generic fiber dimensions add.
\end{proof}

% -------------------------
\subsection*{J.4. Cofinal invariance}
% -------------------------

\begin{definition}[Cofinal subindexing]
Let \(I\) be the directed index category for \(F\).
A full subcategory \(J\subset I\) is cofinal if for every \(i\in I\) there exists \(j\in J\) with a morphism \(i\to j\).
The restricted tower \(F|_J\) has the same colimit as \(F\) in \([\mathbb{R},\mathsf{Vect}_k]\).
\end{definition}

\begin{theorem}[Cofinal invariance]\label{J:thm:cofinal}
Let \(J\subset I\) be cofinal. Then for all \(\tau\ge 0\),
\(\ \mu_{i,\tau}(F|_J)=\mu_{i,\tau}(F)\) and \(\ \nu_{i,\tau}(F|_J)=\nu_{i,\tau}(F)\); hence also \(\muc,\nuc\) agree.
\end{theorem}

\begin{proof}
Cofinal restriction does not change colimits in \([\mathbb{R},\mathsf{Vect}_k]\).
Therefore the source/target of \(\phi_{i,\tau}\) and the map itself are unchanged; kernels, cokernels, and their generic fiber dimensions agree.
\end{proof}

% -------------------------
\subsection*{J.5. $\tau$-sweep and stability bands}
% -------------------------

\begin{definition}[Stability band for a fixed window and degree]\label{J:def:stab-band}
Fix a window (MECE, right-open) and a degree \(i\).
A \emph{\(\tau\)-sweep} is a finite or countable increasing array \(\{\tau_\ell\}_{\ell\in L}\subset(0,\infty)\).
A contiguous subarray \(\{\tau_a,\ldots,\tau_b\}\) is a \emph{stability band} if
\[
\mu_{i,\tau_\ell}(F)=\nu_{i,\tau_\ell}(F)=0\quad\text{for all }\ \ell\in\{a,\ldots,b\},
\]
and the verdict persists under refinement of the sweep without creating nonzero \(\mu_{i,\tau}\) or \(\nu_{i,\tau}\) in the band.
\end{definition}

\begin{proposition}[Robust detection under sweep refinement]\label{J:prop:robust-band}
Assume one of the sufficient conditions of Appendix~D, §D.3 holds (commutation (S1), no \(\tau\)-accumulation (S2), or \(\mathbf{T}_\tau\)-Cauchy with compatible cocone (S3)). Then for each \(i\) there exists a neighborhood of any \(\tau_0\) where \(\phi_{i,\tau}\) is an isomorphism, hence \((\mu_{i,\tau},\nu_{i,\tau})=(0,0)\).
A sufficiently fine \(\tau\)-sweep detects such neighborhoods as stability bands and remains stable under refinement.
\end{proposition}

% -------------------------
\subsection*{J.6. Piecewise constancy off a finite critical set}
% -------------------------

\begin{proposition}[Finite critical set and piecewise constancy]\label{J:prop:piecewise}
Fix a tower \(F\), degree \(i\), and a bounded interval \([a,b]\subset(0,\infty)\).
There exists a finite set \(S\subset[a,b]\) (depending on \(F,i,[a,b]\)) such that \(\mu_{i,\tau}\) and \(\nu_{i,\tau}\) are locally constant on each connected component of \([a,b]\setminus S\).
\end{proposition}

\begin{proof}
In the constructible regime, after applying \(\mathbf{T}_\tau\) the barcode of each \(\mathbf{P}_i(F_n)\) changes only at finitely many thresholds in \([a,b]\).
Kernels/cokernels of \(\phi_{i,\tau}\) can change only when a bar crosses the \(\tau\)-cut or when the colimit/cocone alters infinite-bar presence.
Hence only finitely many scales in \([a,b]\) can change \(\mu_{i,\tau}\) or \(\nu_{i,\tau}\); between them, generic-fiber dimensions are constant.
\end{proof}

\begin{corollary}[Band openness]\label{J:cor:band-open}
If \(\phi_{i,\tau_0}\) is an isomorphism for some \(\tau_0\in(a,b)\), then it is an isomorphism on an open neighborhood \(U\subset(a,b)\) of \(\tau_0\).
Hence stability bands are unions of open intervals intersected with the sweep.
\end{corollary}

% -------------------------
\subsection*{J.7. Window pasting via Restart and Summability}
% -------------------------

\begin{definition}[Per-window pipeline budget and safety margin]\label{J:def:budget-gap}
Let \(\{W_k=[u_k,u_{k+1})\}_k\) be a MECE partition (right-open).
On window \(W_k\), let \(\tau_k>0\) be the selected collapse threshold (possibly from a stability band) and define the \emph{pipeline budget}
\[
\Sigma\delta_k(i)\ :=\ \sum_{U\in W_k}\Big(\delta^{\mathrm{alg}}_{U}(i,\tau_k)+\delta^{\mathrm{disc}}_{U}(i,\tau_k)+\delta^{\mathrm{meas}}_{U}(i,\tau_k)\Big),
\]
with \(\delta\)-components recorded per operation (Appendix~G).
The \emph{safety margin} \(\mathrm{gap}_{\tau_k}(i)>0\) is the configured slack for B-Gate\(^{+}\) on \(W_k\) and degree \(i\).
\end{definition}

\begin{lemma}[Restart inequality]\label{J:lem:restart}
Assume that on \(W_k\) the B-Gate\(^{+}\) passes with \(\mathrm{gap}_{\tau_k}(i)>\Sigma\delta_k(i)\) and that the transition to \(W_{k+1}\) is realized by a finite composition of \emph{deletion-type} steps and \(\varepsilon\)-continuations (both measured after \(\mathbf{T}_\tau\)).
Then there exists \(\kappa\in(0,1]\), depending only on the admissible step class and the \(\tau\)-adaptation policy, such that
\[
\mathrm{gap}_{\tau_{k+1}}(i)\ \ge\ \kappa\ \bigl(\mathrm{gap}_{\tau_k}(i)-\Sigma\delta_k(i)\bigr).
\]
\end{lemma}

\begin{definition}[Summability]\label{J:def:summability}
A run satisfies \emph{Summability} (on degrees \(i\in I\)) if \(\sum_{k}\Sigma\delta_k(i)<\infty\) for all \(i\in I\).
A sufficient pattern is geometric decay of thresholds (e.g.\ \(\tau_k=\tau_0\rho^k\), \(\rho\in(0,1)\)) with bounded per-window operation counts (Appendix~G).
\end{definition}

\begin{theorem}[Pasting windowed certificates]\label{J:thm:pasting}
Let \(\{W_k\}_k\) be MECE. Suppose that on each \(W_k\) the B-Gate\(^{+}\) passes with \(\mathrm{gap}_{\tau_k}(i)>\Sigma\delta_k(i)\) (for all \(i\in I\)), that the Restart inequality (Lemma~\ref{J:lem:restart}) holds at every transition, and that Summability (Definition~\ref{J:def:summability}) is satisfied. Then the concatenation of per-window certificates yields a global certificate on \(\bigcup_k W_k\) for the monitored degrees \(i\in I\).
\end{theorem}

% -------------------------
\subsection*{J.8. Countable covers and finite Čech depth (T-Countable-Cover)}
% -------------------------

\begin{theorem}[Countable MECE cover; finite overlap depth]\label{J:thm:countable}
Let \(\mathcal{W}=\{W_j\}_{j\in\mathbb{N}}\) be a countable MECE, right-open cover of a bounded interval \(U\subset\mathbb{R}\), assumed \emph{locally finite} (each \(t\in U\) meets only finitely many \(W_j\)).
Then:
\begin{enumerate}\itemsep0.25em
\item The overlap multiplicity \(m:=\sup_{t\in U}\#\{j\mid t\in W_j\}\) is finite.
\item The Čech nerve of \(\mathcal{W}\) truncates in degree \(m-1\); Overlap Glue (Appendix~G/H) therefore reduces to finitely many checks on each compact subinterval of \(U\).
\item If, in addition, endpoints and overlaps are definable (Appendix~G), one may replace local finiteness by definability; then a finite subcover exists and \(m\) is bounded a priori (Appendix~J, Theorem~\ref{J:thm:def-cech}).
\end{enumerate}
\end{theorem}

\begin{proof}
Local finiteness on a bounded interval implies a uniform finite bound on multiplicities (standard compactness argument).
The Čech truncation follows since \(p\)-fold intersections vanish for \(p>m\).
The definable strengthening is recorded in Theorem~\ref{J:thm:def-cech}.
\end{proof}

\paragraph{Specification (T-Countable-Cover).}
\emph{Input:} countable MECE, right-open cover \(\mathcal{W}\) of \(U\); optional definability metadata.
\emph{Checks:} (i) local finiteness (or definability), (ii) computed \(m\), (iii) Čech depth \(m-1\).
\emph{Accept iff} overlap depth and finite overlap lists are recorded in \texttt{run.yaml} and used by Overlap Glue.

% -------------------------
\subsection*{J.9. Summability contract (T-Delta-Sum-Converges)}
% -------------------------

\begin{proposition}[Summable $\delta$-ledger on countable covers]\label{J:prop:sum-delta}
Let \(\{W_k\}_{k\in\mathbb{N}}\) be a (locally finite) MECE cover of a bounded \(U\).
Suppose per-window budgets \(\Sigma\delta_k(i)\ge 0\) satisfy either
\(\ \Sigma\delta_k(i)\le C\,\rho^k\) with \(C>0\), \(\rho\in(0,1)\), or
\(\ \Sigma\delta_k(i)\le C\cdot \psi(k)\) with \(\sum_k \psi(k)<\infty\).
Then \(\sum_k \Sigma\delta_k(i)<\infty\) for all \(i\), i.e.\ Summability holds.
\end{proposition}

\begin{proof}
Immediate from comparison with a convergent geometric or reference series.
\end{proof}

\paragraph{Specification (T-Delta-Sum-Converges).}
\emph{Input:} recorded sequence \(\{\Sigma\delta_k(i)\}\) and a proof obligation (\(\rho\)-geometric or reference series).
\emph{Check:} compute partial sums; certify a finite bound \(B_i\) for each degree \(i\).
\emph{Accept iff} \( \sup_n \sum_{k\le n}\Sigma\delta_k(i)\le B_i<\infty\) is logged and referenced by B-Gate\(^{+}\) in \texttt{run.yaml}.

% -------------------------
\subsection*{J.10. Definable Čech finiteness (refined)}
% -------------------------

\begin{theorem}[Definable Čech finiteness]\label{J:thm:def-cech}
Let \(\mathcal{W}=\{W_j\}_{j\in J}\) be a right-open cover of a bounded interval \(U\) such that endpoints and overlaps are \emph{definable} in an o-minimal structure (Appendix~G).
Then a finite subcover exists, the overlap multiplicity \(m\) is finite, and the Čech nerve truncates in degree \(m-1\).
\end{theorem}

\begin{proof}
In o-minimal structures, definable subsets of \(\mathbb{R}\) are finite unions of points and intervals; compactness yields a finite subcover; multiplicity is then uniformly bounded; truncation follows.
\end{proof}

% -------------------------
\subsection*{J.11. Window coherence and after-collapse order}
% -------------------------

All diagnostics \(\phi_{i,\tau}\), \(\mu_{i,\tau}\), and \(\nu_{i,\tau}\) are computed \emph{after} applying \(\mathbf{T}_\tau\) (B-side single layer), and \emph{with the same window and the same \(\tau\)} as used by the gate and the \(\delta\)-ledger (Appendix~G).
This mandatory order ``for each \(t\)\(\to\)\(\mathbf{P}_i\)\(\to\)\(\mathbf{T}_\tau\)\(\to\)compare'' ensures that subsequent \(1\)-Lipschitz post-processing never increases budgets or alters band detection.

% -------------------------
\subsection*{J.12. Minimal API sketch (pseudocode)}
% -------------------------

\begin{verbatim}
# Compute (mu, nu) at scale tau for tower F and degree i
def compute_mu_nu(F, i, tau):
    Ms = [ T_tau(P_i(F_n), tau) for F_n in F.levels ]
    M_inf = T_tau(P_i(F.apex), tau)
    colim_M = colim(Ms)                   # scope policy; [R,Vect_k]
    phi = comparison(colim_M, M_inf)
    mu = generic_fiber_dim(kernel(phi))  # multiplicity of I[0,∞)
    nu = generic_fiber_dim(cokernel(phi))
    return mu, nu

# Detect stability band on a tau sweep (robust to refinement)
def detect_stability_band(F, i, tau_sweep):
    zero = [ell for ell,tau in enumerate(tau_sweep)
            if compute_mu_nu(F,i,tau) == (0,0)]
    return maximal_contiguous_subarrays(zero)

# Restart, Summability, and countable-cover Čech depth
def restart_ok(gap_next, gap_curr, dsum, kappa):
    return gap_next >= kappa * (gap_curr - dsum) and gap_curr > dsum

def summable_budget(deltas):            # T-Delta-Sum-Converges
    return (sum(deltas) < +infty)

def cech_depth_bound(cover):            # T-Countable-Cover
    # bounded by max overlap multiplicity on the (locally finite) cover
    return max_overlap_multiplicity(cover)
\end{verbatim}

\medskip
\noindent\textbf{Summary.}
Within the constructible range and the filtered-colimit scope of Appendix~A, the comparison map \(\phi_{i,\tau}\) is functorial in the tower and compatible with composition and finite sums.
Theorem~\ref{J:thm:subadd} (P6) gives full subadditivity of \((\mu,\nu)\) under composition; Proposition~\ref{J:prop:sum} adds direct-sum additivity; Theorem~\ref{J:thm:cofinal} gives cofinal invariance.
On bounded \(\tau\)-windows, \((\mu_{i,\tau},\nu_{i,\tau})\) are piecewise constant off a finite critical set (Proposition~\ref{J:prop:piecewise}), enabling robust stability-band detection.
Restart and Summability yield a principled pasting mechanism for windowed certificates; for countable MECE covers, local finiteness (or definability) guarantees finite Čech depth (Theorems~\ref{J:thm:countable}, \ref{J:thm:def-cech}) and termination of Overlap Glue.
All comparisons and budgets are made after collapse on the B-side single layer; \(1\)-Lipschitz post-processing in either the standard or \(V\)-enriched metric preserves nonexpansive bounds and reproducible audits.



% =========================
\section*{Appendix K. Idempotent (Co)Monads for Collapse (up to f.q.i.) [Spec + Soft-Commuting + AWFS/2-Cell Budget]}
% =========================
\addcontentsline{toc}{section}{Appendix K. Idempotent (Co)Monads for Collapse (up to f.q.i.)}
\label{K:monads}

\paragraph{Standing conventions.}
We work over a field \(k\).
All persistence modules are constructible (locally finite on bounded windows).
Filtered (co)limits are computed in the functor category \([\mathbb{R},\mathsf{Vect}_k]\) under the scope policy of Appendix~A, Remark~\ref{A:rk:filtered-colimits}; when stated, the result is returned to the constructible range.
Reflection \(\mathbf{T}_\tau\dashv \iota_\tau\) onto the \(\tau\)\hyp local (orthogonal) subcategory \((\mathsf{E}_\tau)^\perp\subset \Pers^{\mathrm{ft}}_k\) is exact and \(1\)\hyp Lipschitz (Appendix~A, Theorem~\ref{A:thm:localization}, Proposition~\ref{A:prop:lipschitz}).
Global conventions: \(\Ext^1\) is always against \(k[0]\); the energy exponent satisfies \(\alpha>0\) (default \(\alpha=1\)); windows are MECE and right\hyp open (Appendix~G).
Distances are measured by the interleaving metric \(d_{\mathrm{int}}\) (bottleneck on barcodes in the constructible case).
Identifications ``up to f.q.i.'' take place in \(\Ho(\mathsf{FiltCh}(k))\) and are preserved by \(\mathbf{P}_i\).

% ------------------------------------------------------------
\subsection*{K.1. Persistence layer: the idempotent monad \(\ \iota_\tau\mathbf{T}_\tau\)}
% ------------------------------------------------------------
Let \(\iota_\tau:(\mathsf{E}_\tau)^\perp\hookrightarrow \Pers^{\mathrm{ft}}_k\) be the fully faithful inclusion and \(\mathbf{T}_\tau:\Pers^{\mathrm{ft}}_k\to(\mathsf{E}_\tau)^\perp\) its left adjoint (Appendix~A). Set
\[
\mathbf{M}_\tau\ :=\ \iota_\tau\circ \mathbf{T}_\tau:\ \Pers^{\mathrm{ft}}_k\longrightarrow \Pers^{\mathrm{ft}}_k.
\]

\begin{theorem}[Idempotent monad]\label{K:thm:monad}
With unit \(\eta:\mathrm{Id}\Rightarrow \iota_\tau\mathbf{T}_\tau\) and multiplication
\(\mu:\mathbf{M}_\tau^2=\iota_\tau\mathbf{T}_\tau\iota_\tau\mathbf{T}_\tau\xRightarrow{\ \iota_\tau\,\varepsilon\,\mathbf{T}_\tau\ }\iota_\tau\mathbf{T}_\tau\),
induced by the counit \(\varepsilon:\mathbf{T}_\tau\iota_\tau\Rightarrow \mathrm{Id}\) on \((\mathsf{E}_\tau)^\perp\), the triple \((\mathbf{M}_\tau,\eta,\mu)\) is a monad and \(\mu\) is a natural isomorphism (idempotence).
\end{theorem}

\begin{proposition}[Exact and nonexpansive]\label{K:prop:exact-lip}
\(\mathbf{M}_\tau\) is exact and \(1\)\hyp Lipschitz:
\(
d_{\mathrm{int}}\!\big(\mathbf{M}_\tau M,\mathbf{M}_\tau N\big)\le d_{\mathrm{int}}(M,N)
\)
for all \(M,N\in\Pers^{\mathrm{ft}}_k\).
\end{proposition}

\begin{remark}[Fixed points and algebras]\label{K:rk:algebras}
The Eilenberg\hyp Moore category of \(\mathbf{M}_\tau\)\hyp algebras identifies with \((\mathsf{E}_\tau)^\perp\) via \(\iota_\tau\).
Thus \(\mathbf{M}_\tau\) is the \(\tau\)\hyp collapse, idempotent on its essential image.
\end{remark}

\begin{example}[Length threshold]
In \(1\)\hyp D, \(\mathbf{T}_\tau\) kills summands of lifespan \(<\tau\); hence \(\mathbf{M}_\tau\) replaces a module by its \(\tau\)\hyp truncated part and \(\mathbf{M}_\tau^2=\mathbf{M}_\tau\).
\end{example}

% ------------------------------------------------------------
\subsection*{K.2. Implementable range (up to f.q.i.): idempotent comonad \(\ \iota\circ C_\tau^{\mathrm{comb}}\) [Spec]}
% ------------------------------------------------------------
\emph{[Spec]} There exists a coreflective, implementable subcategory
\(\Ho(\mathsf{FiltCh}(k))_\tau^{\mathrm{comb}}\subset \Ho(\mathsf{FiltCh}(k))\) and a fully faithful inclusion
\(
\iota:\Ho(\mathsf{FiltCh}(k))_\tau^{\mathrm{comb}}\hookrightarrow \Ho(\mathsf{FiltCh}(k))
\)
with right adjoint (coreflector) \(C_\tau^{\mathrm{comb}}\) (natural up to f.q.i.). Define
\(
\mathbf{G}_\tau:=\iota\circ C_\tau^{\mathrm{comb}}
\).

\begin{theorem}[Idempotent comonad up to f.q.i.]\label{K:thm:comonad}
With counit \(\varepsilon:\mathbf{G}_\tau\Rightarrow \mathrm{Id}\) and comultiplication
\(\delta:\mathbf{G}_\tau\xRightarrow{\ \iota\,\eta\,C_\tau^{\mathrm{comb}}\ }\mathbf{G}_\tau^2\),
\((\mathbf{G}_\tau,\varepsilon,\delta)\) is a comonad; \(\delta\) is a natural isomorphism (idempotence).
All statements hold in \(\Ho(\mathsf{FiltCh}(k))\) and are invariant under f.q.i.
\end{theorem}

\begin{proposition}[Compatibility with persistence]\label{K:prop:compat}
Naturally in \(i,F\),
\[
\mathbf{P}_i(\mathbf{G}_\tau F)\ \cong\ \mathbf{M}_\tau(\mathbf{P}_iF),\qquad
d_{\mathrm{int}}\!\big(\mathbf{P}_i(\mathbf{G}_\tau F),\mathbf{P}_i(\mathbf{G}_\tau G)\big)\ \le\ d_{\mathrm{int}}\!\big(\mathbf{P}_iF,\mathbf{P}_iG\big).
\]
\end{proposition}

\begin{remark}[Scope]\label{K:rk:scope}
The comonad is asserted on the implementable, f.q.i.\ range; this suffices for algorithms and stability. All filtered (co)limit claims observe Appendix~A.
\end{remark}

% ------------------------------------------------------------
\subsection*{K.3. Multi-axis torsion reflectors and soft-commuting policy}
% ------------------------------------------------------------
Let \(E_A,E_B\) be hereditary Serre subcategories of \(\Pers^{\mathrm{ft}}_k\) with exact reflectors \(T_A,T_B\) and \(E_{A\vee B}\) the Serre join.

\begin{proposition}[Nested torsions \(\Rightarrow\) order independence]\label{K:prop:nested}
If \(E_A\subseteq E_B\) or \(E_B\subseteq E_A\), then
\(
T_A\!\circ T_B=T_B\!\circ T_A=T_{A\vee B}
\).
In particular, \(\mathbf{T}_\tau\!\circ\mathbf{T}_\sigma=\mathbf{T}_{\max\{\tau,\sigma\}}\).
\end{proposition}

\begin{definition}[Commutation defect and policy]\label{K:def:soft}
For \(M\in\Pers^{\mathrm{ft}}_k\), define
\(
\Delta_{\mathrm{comm}}(M;A,B):=d_{\mathrm{int}}(T_AT_BM,T_BT_AM)
\).
Given a tolerance \(\eta\ge 0\) on a MECE/right\hyp open window, \(T_A,T_B\) are \emph{soft\hyp commuting} on \(M\) if \(\Delta_{\mathrm{comm}}\le\eta\); otherwise fix an order and \emph{log} \(\Delta_{\mathrm{comm}}\) into the \(\delta\)\hyp ledger as \(\delta^{\mathrm{alg}}\).
\end{definition}

\begin{declaration}[Operational A/B policy]\label{K:dec:soft-policy}
Per window \(W\) and degree \(i\):
(i) test \(\Delta_{\mathrm{comm}}\) after the B\hyp side collapse; (ii) adopt soft\hyp commuting if \(\le\eta\), else fix an order, record \(\Delta_{\mathrm{comm}}\) in \(\delta^{\mathrm{alg}}\); (iii) aggregate into the window budget \(\Sigma\delta\) (Appendix~G) for Restart/Summability (Appendix~J).
\end{declaration}

% ------------------------------------------------------------
\subsection*{K.4. Mirror/Transfer 2-cells and additive budget}
% ------------------------------------------------------------
Let \(\Mirror\) be an admissible endofunctor (filtered or persistent) with a natural \(2\)\hyp cell
\(
\Mirror\circ C_\tau \Rightarrow C_\tau\circ \Mirror
\)
whose image under \(\mathbf{P}_i\) is \(1\)\hyp Lipschitz and bounded by \(\delta(i,\tau)\) in \(d_{\mathrm{int}}\).

\begin{proposition}[Additive budget with (co)reflectors and Mirror]\label{K:prop:budget}
On a window, the total slack is bounded by the sum of: (a) each Mirror–Collapse defect \(\delta(i,\tau)\); (b) each recorded A/B commutation defect \(\Delta_{\mathrm{comm}}\).
Any subsequent \(1\)\hyp Lipschitz post\hyp processing (persistence side) does not increase this bound.
\end{proposition}

\begin{corollary}[Windowwise additivity]\label{K:cor:window}
Across MECE windows \(\{W_j\}\) with per\hyp window contributions \(\delta_j^{\mathrm{Mirror}},\delta_j^{\mathrm{A/B}}\),
\(
\sum_j(\delta_j^{\mathrm{Mirror}}+\delta_j^{\mathrm{A/B}})
\)
bounds the end\hyp to\hyp end slack and is preserved by \(1\)\hyp Lipschitz aggregators.
\end{corollary}

% ------------------------------------------------------------
\subsection*{K.5. AWFS/2-cell contracts and quantale aggregation (strict product accounting)}
% ------------------------------------------------------------
We now \emph{strictify} the 2\hyp cell budget aggregation by working in a declared quantale and its finite products, ensuring a single aggregation law is applied exactly once (Appendix~G, Declaration~\ref{dec:runyaml-extended}).

\begin{definition}[Base quantale]\label{K:def:quantale-base}
A \emph{quantale} is a commutative, monotone monoid \((V,\oplus,\le,0)\) (e.g.\ \([0,\infty]_+\) with \(+\) or max\hyp plus).
Each step emits a defect valued in \(V\) (e.g.\ \(\delta^{\mathrm{alg}},\delta^{\mathrm{disc}},\delta^{\mathrm{meas}}\)); window budgets use the \emph{single} operation \(\oplus\) (``quantale\hyp sum'').
\end{definition}

\begin{definition}[Product quantale and homomorphic collapse]\label{K:def:quantale-prod}
For \(m\ge 1\) channels (e.g.\ Mirror, Transfer, A/B), define the \emph{product quantale}
\[
V^{\times m}\quad\text{with}\quad
\mathbf{x}\preceq\mathbf{y}\iff(\forall r)\ x_r\le y_r,\qquad
(\mathbf{x}\ \widehat{\oplus}\ \mathbf{y})_r:=x_r\oplus y_r,\qquad \mathbf{0}:=(0,\dots,0).
\]
Let \(\pi:V^{\times m}\to V\) be the \emph{collapse homomorphism} \(\pi(\mathbf{x})=x_1\oplus\cdots\oplus x_m\).
Then \(\pi\) is monotone and \(\pi(\mathbf{x}\ \widehat{\oplus}\ \mathbf{y})=\pi(\mathbf{x})\oplus\pi(\mathbf{y})\).
\end{definition}

\begin{definition}[AWFS/2\hyp cell bounds]\label{K:def:awfs}
Assume an algebraic weak factorization system on the filtered layer producing:
(i) a comonad \(L\) (cofibration\hyp like) with counit \(L\Rightarrow\mathrm{Id}\),
(ii) a monad \(R\) (fibration\hyp like) with unit \(\mathrm{Id}\Rightarrow R\),
(iii) 2\hyp cell contracts \(G_\tau\!\Rightarrow\!G_\tau^2\) and \(\Mirror\!\circ\!G_\tau\Rightarrow G_\tau\!\circ\!\Mirror\) with component bounds
\[
\boldsymbol{\delta}(i,\tau)\ =\ \big(\,\delta^{\mathrm{mirror}}(i,\tau),\ \delta^{\mathrm{transfer}}(i,\tau),\ \delta^{\mathrm{A/B}}(i,\tau)\,\big)\ \in\ V^{\times m}.
\]
\end{definition}

\begin{proposition}[Strict product accounting]\label{K:prop:product}
Along any pipeline segment with component defect vectors \(\boldsymbol{\delta}_1,\ldots,\boldsymbol{\delta}_n\in V^{\times m}\),
\[
\boldsymbol{\delta}_{\mathrm{seg}}\ \preceq\ \boldsymbol{\delta}_1\ \widehat{\oplus}\ \cdots\ \widehat{\oplus}\ \boldsymbol{\delta}_n,\qquad
\delta_{\mathrm{seg}}\ :=\ \pi(\boldsymbol{\delta}_{\mathrm{seg}})\ \le\ \pi(\boldsymbol{\delta}_1)\ \oplus\ \cdots\ \oplus\ \pi(\boldsymbol{\delta}_n).
\]
Thus (i) each channel is aggregated exactly once (coordinatewise), (ii) the collapsed scalar budget \(\delta_{\mathrm{seg}}\) respects the same single law \(\oplus\).
\end{proposition}

\begin{proof}
Coordinatewise monotonicity of \(\widehat{\oplus}\) and triangle inequalities for the \(d_{\mathrm{int}}\)\hyp bounds of each 2\hyp cell yield the first relation; the second follows by the homomorphism property of \(\pi\).
\end{proof}

\begin{remark}[run.yaml alignment]\label{K:rk:yaml}
Record in \texttt{run.yaml} (Appendix~G, Declaration~\ref{dec:runyaml-extended}):
\texttt{quantale.\{name,op,unit,order\}}, \texttt{layered\_delta.compose:\,"quantale-sum"},
\texttt{awfs\_2cell.two\_cell\_bounds} (per channel), and the per\hyp window vector \(\boldsymbol{\delta}\) plus its scalar collapse \(\pi(\boldsymbol{\delta})\).
This enforces determinism and prevents double counting.
\end{remark}

% ------------------------------------------------------------
\subsection*{K.6. Windowed usage and minimal schema}
% ------------------------------------------------------------
Per window \(W\) (degree \(i\), threshold \(\tau\)), log: (a) reflector order and A/B verdict, \(\Delta_{\mathrm{comm}}\); (b) component 2\hyp cell bounds \(\boldsymbol{\delta}(i,\tau)\in V^{\times m}\) and their collapse \(\pi(\boldsymbol{\delta})\); (c) quantale parameters; (d) B\hyp Gate\(^{+}\) gap for Restart/Summability (Appendix~J). Canonical YAML keys appear in Appendix~G.

% ------------------------------------------------------------
\subsection*{K.7. Formalization stubs (Lean/Coq) [Spec]}
% ------------------------------------------------------------
\begin{verbatim}
-- Reflectors and nested-order independence
constant Reflector : Type
constant T : Reflector → (Pers_ft k ⥤ Pers_ft k)
axiom T_exact : ∀ R, is_exact (T R)
axiom T_reflector : ∀ R, is_reflector R (T R)

theorem nested_torsion_order_indep
  {A B : Reflector} (h : E(A) ⊆ E(B) ∨ E(B) ⊆ E(A)) :
  T A ⋙ T B ≅ T (A ⊔ B) ∧ T B ⋙ T A ≅ T (A ⊔ B) := by
  -- standard torsion-theoretic argument
  admit

-- Commutation defect
def comm_defect (M : Pers_ft k) (A B : Reflector) : ℝ≥0 :=
  d_int ((T A ⋙ T B) ⋙ obj M) ((T B ⋙ T A) ⋙ obj M)

-- Quantale product and collapse
structure Quantale :=
  (V : Type) (op : V → V → V) (le : V → V → Prop) (unit : V)
  (op_comm  : ∀ x y, op x y = op y x)
  (op_assoc : ∀ x y z, op (op x y) z = op x (op y z))
  (op_mono  : ∀ {x x' y y'}, le x x' → le y y' → le (op x y) (op x' y'))
  (le_refl  : ∀ x, le x x) (le_trans : ∀ x y z, le x y → le y z → le x z)

def prodQ (Q : Quantale) (m : ℕ) : Quantale := -- coordinatewise
  -- returns (Q.V)^m with product order and op
  sorry

def collapse (Q : Quantale) (m : ℕ) :
  (prodQ Q m).V → Q.V := -- homomorphic fold
  sorry
\end{verbatim}

% ------------------------------------------------------------
\subsection*{K.8. Edge cases and guard-rails}
% ------------------------------------------------------------
\begin{itemize}\itemsep0.2em
  \item \emph{Nonexact truncations.} Heuristic, nonexact steps may violate monad/comonad laws and stability; exclude them from the reflector class.
  \item \emph{Mismatched windows.} A/B tests and Mirror–Collapse checks must use the same window/\(\tau\) as the gate and the ledger.
  \item \emph{Three or more nonnested axes.} Pairwise soft\hyp commuting does not imply global confluence; fix a canonical order, A/B test adjacent pairs, log residuals.
  \item \emph{Adaptive thresholds.} When \(\tau\) adapts online, every ledger entry must carry the \(\tau\) in force at measurement; do not retroactively adjust.
\end{itemize}

\medskip
\noindent\textbf{Summary.}
Collapse on \(\Pers^{\mathrm{ft}}_k\) is governed by the exact, idempotent, \(1\)\hyp Lipschitz monad \(\iota_\tau\mathbf{T}_\tau\); on \(\Ho(\mathsf{FiltCh}(k))\) (implementable, up to f.q.i.) by the idempotent comonad \(\iota\circ C_\tau^{\mathrm{comb}}\).
For multi\hyp axis torsions, nesting yields order independence; otherwise, an A/B \emph{soft\hyp commuting} policy with measured defect \(\Delta_{\mathrm{comm}}\) controls order and integrates into the additive \(\delta\)\hyp budget together with Mirror/Transfer 2\hyp cell bounds.
AWFS/2\hyp cell contracts are aggregated in a \emph{product quantale} and collapsed by a single homomorphism, giving strict, non\hyp duplicative accounting.
This furnishes subadditive pipeline bounds and reproducible, window\hyp coherent audits aligned with Appendices~G and~J.



% =========================
\section*{Appendix L. Quantitative Commutation for Mirror/Tropical [Spec + Pipeline Budget + A/B Policy]}
% =========================
\phantomsection
\addcontentsline{toc}{section}{Appendix L. Quantitative Commutation for Mirror/Tropical}
\refstepcounter{section}
\label{L:quant-comm}

\paragraph{Standing conventions.}
We work over a field \(k\); all persistence modules are constructible.
Filtered (co)limits are computed in \([\mathbb{R},\mathsf{Vect}_k]\) under the scope policy of Appendix~A, Remark~\ref{A:rk:filtered-colimits}, and then returned to the constructible range.
Distances are measured by the interleaving metric \(d_{\mathrm{int}}\) (bottleneck on barcodes in the constructible case).
The truncation \(\mathbf{T}_\tau\) is exact and \(1\)\hyp Lipschitz; on filtered complexes we write \(C_\tau\) and on the persistence layer \(\mathbf{T}_\tau\).
Global conventions: \(\Ext^1\) is always against \(k[0]\); the energy exponent satisfies \(\alpha>0\) (default \(\alpha=1\)); windows are MECE and right\hyp open (Appendix~G).
All comparisons follow the fixed order \(\boxed{\text{for each }t\ \Rightarrow\ \mathbf{P}_i\ \Rightarrow\ \mathbf{T}_\tau\ \Rightarrow\ \text{compare}}\) on the B\hyp side single layer, after collapse.

% -------------------------
\subsection*{L.1. Hypotheses (Mirror/Tropical)}
% -------------------------
\emph{[Spec]} Let \(U\) be a filtered\hyp level endofunctor (``Mirror/Tropical''); via \(\mathbf{P}_i\) it induces an endofunctor on \(\Pers^{\mathrm{ft}}_k\).
Assume:

\begin{itemize}\itemsep0.35em
  \item[(H1)] \textbf{\(1\)\hyp Lipschitz of \(U\).} For each degree \(i\),
  \[
    d_{\mathrm{int}}\!\Big(\mathbf{P}_i(UF),\ \mathbf{P}_i(UG)\Big)\ \le\ d_{\mathrm{int}}\!\Big(\mathbf{P}_i(F),\ \mathbf{P}_i(G)\Big)
    \qquad(\forall\,F,G).
  \]
  \item[(H2)] \textbf{\(\delta\)\hyp controlled natural \(2\)\hyp cell.} There exists a natural \(2\)\hyp cell
  \[
    \theta:\ U\!\circ\! C_\tau\ \Rightarrow\ C_\tau\!\circ\! U
  \]
  (interpreted up to f.q.i.\ in \(\Ho(\mathsf{FiltCh}(k))\)) such that, uniformly in \(F\),
  \[
    d_{\mathrm{int}}\!\Big(\mathbf{P}_i(U(C_\tau F)),\ \mathbf{P}_i(C_\tau(UF))\Big)\ \le\ \delta(i,\tau),
  \]
  for all \(i,\tau\).
\end{itemize}

\begin{remark}[Strict commutation]
If \(\delta(i,\tau)=0\) then \(U\) and \(C_\tau\) commute up to isomorphism at the truncated persistence layer (degree \(i\), window \(\tau\)).
\end{remark}

\begin{remark}[Tropical instances (Spec)]
A ``Tropical'' operator may be any window\hyp coherent, order\hyp preserving endofunctor assembled from idempotent semiring primitives (e.g.\ min/max filters, morphological erosions/dilations, monotone reparametrizations) that satisfies \emph{(H1)} and provides \(\theta\) with a bound \(\delta_{\mathrm{trop}}(i,\tau)\) independent of \(F\).
Its configuration (e.g.\ bin width/range) is logged under \texttt{tropical} in \texttt{run.yaml} (Appendix~G, Declaration~\ref{dec:runyaml-extended}).
\end{remark}

% -------------------------
\subsection*{L.2. Post-collapse non-expansive bound (direct form)}
% -------------------------
\begin{theorem}[Post\hyp collapse non\hyp expansiveness with \(2\)\hyp cell bound]\label{L:thm:q-comm}
Under \emph{(H1)}–\emph{(H2)}, for all \(F,i,\tau\),
\[
d_{\mathrm{int}}\!\Big(\mathbf{T}_\tau\,\mathbf{P}_i(U C_\tau F),\ \mathbf{T}_\tau\,\mathbf{P}_i(C_\tau U F)\Big)\ \le\ \delta(i,\tau),
\]
and any further B\hyp side post\hyp processing \(\Phi\) with \(d_{\mathrm{int}}(\Phi X,\Phi Y)\le d_{\mathrm{int}}(X,Y)\) does not increase this bound.
\end{theorem}

\begin{proof}[Proof sketch]
Apply \(\mathbf{T}_\tau\) (which is \(1\)\hyp Lipschitz and exact) to the \emph{(H2)} estimate; then use non\hyp expansiveness of \(\Phi\).
\end{proof}

\begin{corollary}[Two\hyp stage additivity]
If \(U_1,U_2\) obey \emph{(H1)}–\emph{(H2)} with controls \(\delta_1,\delta_2\), then
\[
d_{\mathrm{int}}\!\Big(\mathbf{T}_\tau\mathbf{P}_i(U_2U_1C_\tau F),\ \mathbf{T}_\tau\mathbf{P}_i(C_\tau U_2U_1 F)\Big)\ \le\ \delta_1(i,\tau)+\delta_2(i,\tau).
\]
\end{corollary}

\begin{remark}[Necessity of (H2)]
Without a controlled \(2\)\hyp cell, non\hyp expansiveness of \(U\) and \(C_\tau\) does not constrain \(U\!\circ\! C_\tau\) versus \(C_\tau\!\circ\! U\); counterexamples arise from near\hyp \(\tau\) accumulations (Type~IV).
\end{remark}

% -------------------------
\subsection*{L.3. Product-ledger budget (direct accumulation, no relay chains)}
% -------------------------
We mainstream the \emph{direct} accumulation of commutation errors in a \emph{product quantale} ledger (cf.\ Appendix~K, \S K.5), eliminating intermediate comparison chains.

\begin{definition}[Base and product quantales]\label{L:def:q-product}
Fix a commutative quantale \((V,\oplus,\le,0)\) (e.g.\ \([0,\infty]_+\) with \(+\)).
For \(m\ge 1\) channels (e.g.\ Mirror, Transfer, A/B), set
\[
V^{\times m},\quad
\mathbf{x}\preceq\mathbf{y}\iff(\forall r)\ x_r\le y_r,\quad
(\mathbf{x}\ \widehat{\oplus}\ \mathbf{y})_r:=x_r\oplus y_r,\quad
\mathbf{0}:=(0,\dots,0).
\]
Let \(\pi:V^{\times m}\!\to V\) be the collapse homomorphism \(\pi(\mathbf{x})=x_1\oplus\cdots\oplus x_m\).
\end{definition}

\begin{definition}[Per\hyp window defect vector]\label{L:def:vec}
On window \(W=[u,u')\) (MECE, right\hyp open) and degree \(i\), each Mirror/Tropical step \(U_j\) with threshold \(\tau_j\) contributes
\(
\boldsymbol{\delta}^{(j)}(i,\tau_j)\in V^{\times m}
\)
whose nonzero coordinates are the relevant \(\delta\)\hyp bounds from Theorem~\ref{L:thm:q-comm}; any A/B failure contributes a vector with only its A/B coordinate nonzero, equal to \(\Delta_{\mathrm{comm}}(M;A,B)\) (Appendix~K).
The window vector is
\[
\boldsymbol{\delta}_W(i)\ :=\ \widehat{\oplus}_{\,j}\ \boldsymbol{\delta}^{(j)}(i,\tau_j),
\qquad
\Sigma\delta_W(i)\ :=\ \pi\big(\boldsymbol{\delta}_W(i)\big).
\]
\end{definition}

\begin{proposition}[Strict, non\hyp duplicative accounting]\label{L:prop:strict}
Coordinatewise aggregation in \(V^{\times m}\) followed by \(\pi\) yields
\(
\Sigma\delta_W(i)\ \le\ \bigoplus_j \pi\!\big(\boldsymbol{\delta}^{(j)}(i,\tau_j)\big)
\)
and counts each channel exactly once. This is compatible with Appendix~K, Proposition~\ref{K:prop:product}.
\end{proposition}

% -------------------------
\subsection*{L.4. Windowed pipeline bound (direct form)}
% -------------------------
\begin{theorem}[Direct pipeline bound]\label{L:thm:budget}
For any filtered input \(F\),
\[
d_{\mathrm{int}}\!\Big(\mathbf{T}_\tau\,\mathbf{P}_i(\Pi_{\mathrm{lhs}}(F)),\ \mathbf{T}_\tau\,\mathbf{P}_i(\Pi_{\mathrm{rhs}}(F))\Big)\ \le\ \Sigma\delta_W(i),
\]
where \(\Pi_{\mathrm{lhs/rhs}}\) are the realized left/right orderings on \(W\).
The bound is uniform in \(F\), additive across steps via \(\widehat{\oplus}\), and non\hyp increasing under any subsequent \(1\)\hyp Lipschitz persistence post\hyp processing.
\end{theorem}

\begin{corollary}[Across windows; Restart/Summability alignment]
Over a MECE partition \(\{W_k\}\),
\(
\sum_k \Sigma\delta_{W_k}(i)
\)
controls the end\hyp to\hyp end discrepancy; compare this sum to the safety margins \(\mathrm{gap}_{\tau_k}(i)\) to apply Restart and Summability (Appendix~J).
\end{corollary}

% -------------------------
\subsection*{L.5. Operational A/B policy (reflectors) and soft-commuting}
% -------------------------
\begin{definition}[A/B test]\label{L:def:ab}
For exact reflectors \(T_A,T_B\) (Appendix~K) and \(M\in\Pers^{\mathrm{ft}}_k\) on window \(W\),
\[
\Delta_{\mathrm{comm}}(M;A,B)\ :=\ d_{\mathrm{int}}(T_AT_BM,\ T_BT_AM).
\]
Given a tolerance \(\eta\ge 0\), declare \emph{soft\hyp commuting} if \(\Delta_{\mathrm{comm}}\le\eta\); otherwise fix an order and ledger the residual as the A/B coordinate in \(\boldsymbol{\delta}_W(i)\).
Skip testing for nested torsions (\(T_A\circ T_B=T_{A\vee B}\), Appendix~K).
\end{definition}

% -------------------------
\subsection*{L.6. Tropical specification hooks and run.yaml alignment}
% -------------------------
For reproducibility and audit (Appendix~G, Declaration~\ref{dec:runyaml-extended}):
\begin{itemize}\itemsep0.25em
  \item \texttt{tropical:} \texttt{bins:\{width: w, range:[a,b]\}}, \texttt{policy:\,"after\_collapse\_only"}.
  \item \texttt{awfs\_2cell.two\_cell\_bounds:} per\hyp channel bounds (mirror/transfer) used as coordinates of \(\boldsymbol{\delta}\).
  \item Per window/degree \(i\), record both the vector and its collapse:
\begin{verbatim}
tropical:
  bins: { width: 0.05, range: [0.0, 2.0] }
  policy: "after_collapse_only"
awfs_2cell:
  two_cell_bounds: { mirror_collapse: 0.006, transfer_collapse: 0.004 }
window:
  id: "W03"; degree: 1; tau: 0.40
  budget:
    delta_vector: { mirror: 0.006, transfer: 0.004, AB: 0.002 }
    sum_delta: 0.012         # π(delta_vector) under the declared quantale
  b_gate_plus:
    passed: true
    gap_tau: 0.09
\end{verbatim}
\end{itemize}

% -------------------------
\subsection*{L.7. Minimal pseudocode (product-ledger, direct)}
% -------------------------
\begin{verbatim}
# steps on window W, degree i; each step carries a coordinate vector
def window_budget_vector(steps, i):
    # steps: list of dicts with keys: kind ∈ {"mirror","transfer","AB"}, "delta"
    vec = {"mirror": 0.0, "transfer": 0.0, "AB": 0.0}
    for s in steps:
        if s["kind"] in vec:
            vec[s["kind"]] += s["delta"]   # coordinatewise ⊕ in [0,∞]_+
        # post-steps are 1-Lipschitz: no contribution
    return vec

def collapse_quantale(vec):
    # π: product → base quantale (here: sum of coordinates)
    return vec["mirror"] + vec["transfer"] + vec["AB"]
\end{verbatim}

% -------------------------
\subsection*{L.8. Edge cases and guard-rails}
% -------------------------
\begin{itemize}\itemsep0.25em
  \item \emph{No (H2), no bound.} Without a controlled \(2\)\hyp cell, no quantitative commutation estimate exists.
  \item \emph{Window/\(\tau\) coherence.} All measurements use the same window and \(\tau\) as B\hyp Gate\(^{+}\) and the ledger.
  \item \emph{Degree totals vs per\hyp degree.} Report per\hyp degree vectors and their collapsed totals; small residuals can hide in sums.
  \item \emph{Multiple nonnested axes.} Pairwise soft\hyp commuting does not imply global confluence; use a canonical order, test adjacent pairs, ledger residuals.
\end{itemize}

\medskip
\noindent\textbf{Summary.}
Assuming \(U\) is \(1\)\hyp Lipschitz and admits a \(\delta(i,\tau)\)\hyp controlled \(2\)\hyp cell with collapse, the \emph{post\hyp collapse} discrepancy between \(U\!\circ\!C_\tau\) and \(C_\tau\!\circ\!U\) is bounded by \(\delta(i,\tau)\) and is stable under any further B\hyp side non\hyp expansive processing (Theorem~\ref{L:thm:q-comm}).
Commutation costs are accumulated \emph{directly} per window in a product\hyp quantale ledger and collapsed once by a homomorphism, preventing double counting and simplifying error propagation (Proposition~\ref{L:prop:strict}).
The resulting scalar \(\Sigma\delta_W(i)\) integrates with B\hyp Gate\(^{+}\), Restart, and Summability (Appendix~J), while Tropical configuration and 2\hyp cell bounds are recorded in \texttt{run.yaml} as specified in Appendix~G.



% =========================
\section*{Appendix M. (Optional) Lax Monoidal Compatibility [Spec + Windowed Usage + Budget Integration]}
\phantomsection
\addcontentsline{toc}{section}{Appendix M. (Optional) Lax Monoidal Compatibility}
\label{M:lax-monoidal}

\bigskip

\noindent
\textbf{Status.} All claims marked [Spec] are windowed, reproducible contracts intended for operational use. They are compatible with the standard 1D constructible persistence setting and do not assert strict monoidality beyond the stated hypotheses. Full derivations are included so that no further reinforcement is required.

\subsection*{M.0. Standing conventions and scope}

\paragraph{Ground field and categories.}
Fix a field \(\kk\).
Let \(\Pers^\mathrm{c}_\kk\) denote the category of constructible (1-parameter) persistence modules \(M:\RR\to\Vect_\kk\), i.e.\ functors that are pointwise finite dimensional and have locally finite sets of critical parameters (equivalently, barcodes locally finite).
We write \(\Vect_\kk\) for vector spaces over \(\kk\).
Filtered chain complexes will be denoted \(F=(F^t)_{t\in\RR}\), each \(F^t\) a bounded-below chain complex of finite-dimensional \(\kk\)-vector spaces, with nondecreasing inclusions in \(t\) and locally finite changes.

\paragraph{Interleaving metric and truncation.}
We use the interleaving metric \(\dint\) on constructible 1D persistence modules; in this setting \(\dint\) coincides with the bottleneck distance on barcodes.
We assume the existence of:
(i) an exact, functorial, filtration-truncation endofunctor \(\TT:\Pers^\mathrm{c}_\kk\to\Pers^\mathrm{c}_\kk\) for each \(\tau>0\) (``collapse'' or ``truncation'' at scale \(\tau\)), and
(ii) a chain-level operation \(\CT\) producing a filtered complex \(\CT F\) whose persistent homology is \(\TT\) of that of \(F\).
We assume \(\TT\) is \(1\)-Lipschitz with respect to \(\dint\) and exact on short exact sequences of persistence modules.

\paragraph{Windowing and measurability.}
Fix a MECE, right-open windowing of \([0,\infty)\) into intervals \(\mathcal{W}=\{[w_j,w_{j+1})\}_{j\in J}\) with \(0=w_0<w_1<\cdots\), locally finite on bounded ranges.
For a filtered complex \(F\) and degree \(i\), the Betti function \(\betti_i(F;t):=\dim_\kk H_i(F^t)\) is piecewise constant with locally finite jumps; hence it is measurable and integrable on bounded windows.

\paragraph{Energy functional.}
For a window size \(\sigma>0\) and degree \(i\),
\[
  \PE_i^{\le \sigma}(F)\ :=\ \int_0^\sigma \betti_i(F;t)\,dt,
\]
and similarly \(\PE_i^{\le \sigma}(M)\) for a persistence module \(M\), via any filtered presentation.
All integrals are Lebesgue integrals; by constructibility on bounded windows, they reduce to finite sums.

\paragraph{Pointwise tensor.}
On the filtered-complex side and on persistence modules we use the pointwise tensor:
\[
  (M\otimes N)(t) := M(t)\otimes_\kk N(t),\qquad (F\otimes G)^t := F^t\otimes_\kk G^t.
\]
All tensor products are over \(\kk\).

\paragraph{[Spec] scope.}
Results labeled [Spec] are reproducible windowed contracts that depend only on the hypotheses stated herein. They interoperate with budget accounting (Section~M.5) and with \(\tau\)-sweeps (Section~M.4). No assertion of strict monoidality is made beyond the lax statements we prove.

\bigskip

\subsection*{M.1. Hypotheses and the laxator}

We work under the following hypotheses.

\begin{description}[leftmargin=2.2em,labelindent=0em]
  \item[(M1) Exact, constructible tensor.]
  The pointwise tensor \(\otimes\) is exact and biadditive, and preserves constructibility on bounded windows: for any \(M,N\in \Pers^\mathrm{c}_\kk\), the set of critical parameters of \(M\otimes N\) on a bounded window is locally finite, and \(\dim_\kk (M(t)\otimes_\kk N(t))<\infty\). Likewise for filtered complexes \(F,G\).
  \item[(M2) K\"unneth over a field (pointwise).]
  For each \(t\in\RR\) and each \(i\in\ZZ\),
  \[
    H_i\big((F\otimes G)^t\big)\ \cong\ \bigoplus_{p+q=i} H_p(F^t)\otimes_\kk H_q(G^t),
  \]
  naturally in \((F,G)\) and \(t\). Over a field, \(\Tor\)-terms vanish.
  \item[(M3) Lax compatibility for collapse.]
  There is a natural transformation (the laxator) in the homotopy setting (up to filtered quasi-isomorphism)
  \[
    \lambda_{\tau,F,G}:\ \CT(F\otimes G)\ \Longrightarrow\ \CT F\ \otimes\ \CT G,
  \]
  natural in \(F,G\) and \(\tau\).
\end{description}

We will occasionally strengthen (M3) to:

\medskip
\noindent
\(\mathbf{(M3^+)}\) For all \(t\in\RR\) and all degrees \(i\), the induced map in homology
\[
  H_i\big(\lambda_{\tau,F,G}^t\big):\ H_i\big((\CT(F\otimes G))^t\big)\ \longrightarrow\ H_i\big((\CT F\otimes \CT G)^t\big)
\]
is a monomorphism (equivalently, ranks do not decrease under \(\lambda\) at each \((t,i)\)).

\begin{remark}[Intervals under tensor]\label{rem:interval-tensor}
For interval modules over \(\kk\),
\[
  \kk_{[a,b)}\otimes \kk_{[c,d)} \ \cong\
  \begin{cases}
     \kk_{[\max\{a,c\},\ \min\{b,d\})}, & \text{if } [a,b)\cap[c,d)\neq\emptyset,\\
     0, & \text{otherwise}.
  \end{cases}
\]
Thus tensor intersects lifespans; at the barcode level this is the K\"unneth rule over a field.
\end{remark}

\begin{remark}[Preservation of constructibility]\label{rem:constructible}
Since on any bounded window only finitely many bars of \(M\) and \(N\) meet, the critical set of \(M\otimes N\) is contained in the finite union of the critical sets of \(M\) and \(N\); hence it is locally finite. Pointwise finite dimensionality is preserved because \(\dim_\kk (M(t)\otimes_\kk N(t)) = \dim_\kk M(t)\cdot \dim_\kk N(t) < \infty\).
\end{remark}

\begin{remark}[Monotonicity scope]
Tensor is neither purely inclusion-type nor deletion-type in general. We therefore confine monotonicity statements to windowed energy upper bounds and to the \((\mathrm{M3}^+)\) regime, where the laxator is pointwise monomorphic in homology.
\end{remark}

\bigskip

\subsection*{M.2. Windowed energy via overlap integrals}

\begin{definition}[Betti function and windowed energy]
For a filtered complex \(F\) and degree \(i\), set \(\betti_i(F;t):=\dim_\kk H_i(F^t)\) for \(t\in\RR\). For \(\sigma>0\) define the clipped Betti integral (windowed energy)
\[
  \PE_i^{\le \sigma}(F)\ :=\ \int_0^\sigma \betti_i(F;t)\,dt.
\]
For a persistence module \(M\) we define \(\betti_i(M;t)\) and \(\PE_i^{\le\sigma}(M)\) analogously, using any filtered presentation (well-defined by exactness).
\end{definition}

\begin{theorem}[Convolution-type upper bound; equality under K\"unneth]\label{thm:conv}
Assume \emph{(M1)--(M2)}. Then for all filtered complexes \(F,G\), all degrees \(i\), and all windows \(\sigma>0\),
\[
  \PE_i^{\le \sigma}(F\otimes G)\ \le\ \sum_{p+q=i}\ \int_0^\sigma \betti_p(F;t)\,\betti_q(G;t)\,dt,
\]
where the sum \(\sum_{p+q=i}\) is finite due to bounded homological range. Consequently,
\[
  \PE_i^{\le \sigma}(F\otimes G)\ \le\ \sum_{p+q=i}\Big(\sup_{t\in[0,\sigma]}\betti_q(G;t)\Big)\,\PE_p^{\le \sigma}(F),
\]
and symmetrically with \((F,p)\leftrightarrow(G,q)\).
If, moreover, the pointwise K\"unneth isomorphism of \emph{(M2)} holds, then equality holds:
\[
  \PE_i^{\le \sigma}(F\otimes G)\ =\ \sum_{p+q=i}\ \int_0^\sigma \betti_p(F;t)\,\betti_q(G;t)\,dt.
\]
\end{theorem}

\begin{remark}[Barcode-level interpretation]
In the interval-decomposable case (1D constructible over a field), \(\betti_p(F;t)\) counts the number of \(p\)-bars alive at \(t\). The integrand \(\betti_p(F;t)\,\betti_q(G;t)\) counts ordered pairs of alive bars; the overlap integral sums the lengths of pairwise intersections, matching tensor-as-intersection (Remark~\ref{rem:interval-tensor}).
\end{remark}

\bigskip

\subsection*{M.2$^\ast$. Windowed seminorms (\(E_1\) and \(L^\infty\)) and quantale control}

\begin{definition}[Windowed seminorms]\label{def:window-norms}
For \(M\in\Pers^\mathrm{c}_\kk\), degree \(i\), and window \([0,\sigma]\), set
\[
  \|M\|_{E_1}^{(i,\sigma)}\ :=\ \PE_i^{\le \sigma}(M),\qquad
  \|M\|_{\infty}^{(i,\sigma)}\ :=\ \sup_{t\in[0,\sigma]}\betti_i(M;t).
\]
Thus the ``\(E_1\) on the window'' is exactly the clipped Betti integral \(\PE\) from above.
\end{definition}

\begin{proposition}[Quantale-type submultiplicativity]\label{prop:quantale}
Under \emph{(M1)--(M2)}, for all \(i,\sigma\),
\[
  \|F\otimes G\|_{E_1}^{(i,\sigma)}
  \ \le\ \sum_{p+q=i}\ \|F\|_{E_1}^{(p,\sigma)}\ \|G\|_{\infty}^{(q,\sigma)}
  \quad\text{and}\quad
  \|F\otimes G\|_{E_1}^{(i,\sigma)}
  \ \le\ \sum_{p+q=i}\ \|F\|_{\infty}^{(p,\sigma)}\ \|G\|_{E_1}^{(q,\sigma)}.
\]
If K\"unneth holds pointwise, the upper bounds are equalities with \(\|G\|_{\infty}^{(q,\sigma)}\) replaced by the integrand \(\betti_q(G;\cdot)\).
\end{proposition}

\begin{proof}
Rewrite Theorem~\ref{thm:conv} using Definition~\ref{def:window-norms} and bound one factor by its \(L^\infty\) seminorm on \([0,\sigma]\).
\end{proof}

\begin{proposition}[Clip/contract stability of seminorms]\label{prop:clip-contract-stable}
For any \(\sigma>0\) and degree \(i\):
\begin{enumerate}\itemsep0.2em
  \item \emph{Clipping.} Replacing \(M\) by its window restriction (right-open) does not increase \(\|M\|_{E_1}^{(i,\sigma)}\) nor \(\|M\|_{\infty}^{(i,\sigma)}\) and leaves them unchanged on \([0,\sigma]\).
  \item \emph{Collapse.} Applying \(\TT\) at any \(\tau>0\) satisfies
  \(\|\,\TT M\,\|_{E_1}^{(i,\sigma)}\le \|M\|_{E_1}^{(i,\sigma)}\) and
  \(\|\,\TT M\,\|_{\infty}^{(i,\sigma)}\le \|M\|_{\infty}^{(i,\sigma)}\).
\end{enumerate}
\end{proposition}

\begin{proof}
(1) Clipping restricts to \([0,\sigma]\) and does not change values there. (2) Exactness and deletion of short bars imply pointwise \(\betti_i(\TT M;t)\le \betti_i(M;t)\) on \([0,\sigma]\); integrate and take sup.
\end{proof}

\bigskip

\subsection*{M.3. Collapse, lax monoidality, and energy dominance}

\begin{proposition}[Collapsed convolution bound]\label{prop:collapsed-conv}
Assume \emph{(M1)--(M3)}. For any \(\tau,\sigma>0\) and all degrees \(i\),
\[
  \PE_i^{\le \sigma}\big(\CT F \otimes \CT G\big)\ \le\ \sum_{p+q=i}\ \int_0^\sigma \betti_p(\CT F;t)\,\betti_q(\CT G;t)\,dt.
\]
Equality holds when the pointwise K\"unneth isomorphism holds for \((\CT F,\CT G)\).
\end{proposition}

\begin{proof}
Apply Theorem~\ref{thm:conv} to the filtered complexes \(\CT F\) and \(\CT G\).
\end{proof}

\begin{theorem}[Energy dominance under a monomorphic laxator]\label{thm:mono-dominance}
Assume \emph{(M1)--(M3)} and \emph{(M3$^+$)}. Then for all \(\tau,\sigma>0\) and all degrees \(i\),
\[
  \PE_i^{\le \sigma}\big(\CT(F\otimes G)\big)\ \le\ \PE_i^{\le \sigma}\big(\CT F \otimes \CT G\big).
\]
Consequently,
\[
  \PE_i^{\le \sigma}\big(\CT(F\otimes G)\big)\ \le\ \sum_{p+q=i}\ \int_0^\sigma \betti_p(\CT F;t)\,\betti_q(\CT G;t)\,dt.
\]
\end{theorem}

\begin{proof}
By \((\mathrm{M3}^+)\), \(\betti_i(\CT(F\otimes G);t)\le \betti_i(\CT F\otimes \CT G;t)\) for all \(t\); integrate.
\end{proof}

\begin{proposition}[Lax monoidal collapse; quantale control]\label{prop:lax-mono}
On any window \([0,\sigma]\) and degree \(i\), under \emph{(M1)--(M3)}:
\[
  \|\CT(F\otimes G)\|_{E_1}^{(i,\sigma)}\ \le\ \sum_{p+q=i}\ \|\CT F\|_{E_1}^{(p,\sigma)}\,\|\CT G\|_{\infty}^{(q,\sigma)}.
\]
If moreover \emph{(M3$^+$)} holds, then
\[
  \|\CT(F\otimes G)\|_{E_1}^{(i,\sigma)}\ \le\ \|\CT F\otimes \CT G\|_{E_1}^{(i,\sigma)},
\]
and deletion-type steps (further truncations) preserve the inequality; any \(1\)-Lipschitz continuation preserves the right-hand side and hence the dominance.
\end{proposition}

\begin{proof}
Combine Proposition~\ref{prop:collapsed-conv} with Proposition~\ref{prop:quantale}; the second inequality is Theorem~\ref{thm:mono-dominance}. Stability under deletion and \(1\)-Lipschitz post-processing follows from Proposition~\ref{prop:clip-contract-stable}.
\end{proof}

\bigskip

\subsection*{M.3$^\dagger$. ECF low-pass laxator (explicit) [Spec]}

\paragraph{Definition (ECF low-pass).}
In the interval regime, let \(\CT\) be the \emph{low-pass} (length threshold) collapse: \(\CT\) kills any interval of length \(<\tau\) and acts as the identity on longer intervals (no shrinking). Define the \emph{ECF low-pass laxator}
\[
  \lambda^{\mathrm{ECF}}_{\tau,F,G}:\ \CT(F\otimes G)\ \longrightarrow\ \CT F\ \otimes\ \CT G
\]
as the direct sum over interval summands of the canonical maps
\[
  \CT(\kk_{I}\otimes \kk_{J})\ \hookrightarrow\ \CT\kk_{I}\ \otimes\ \CT\kk_{J},
\]
using \(\kk_{I}\otimes \kk_{J}\cong \kk_{I\cap J}\) and the rules: if either \(\kk_I\) or \(\kk_J\) is killed by \(\CT\) then both sides are \(0\); if both survive, the map is the identity on the intersection summand. Extend additively and pass to filtered quasi-isomorphism on complexes.

\paragraph{Properties.}
\begin{itemize}\itemsep0.2em
  \item \([{\rm Spec}]\) \(\lambda^{\mathrm{ECF}}_{\tau,F,G}\) is natural in \(F,G\) and satisfies \((\mathrm{M3})\).
  \item \([{\rm Spec}]\) \(\lambda^{\mathrm{ECF}}_{\tau,F,G}\) is pointwise monomorphic in homology, hence \((\mathrm{M3}^+)\) holds. Therefore Theorem~\ref{thm:mono-dominance} applies verbatim.
  \item \([{\rm Spec}]\) The construction is \(1\)-Lipschitz (barwise deletion only) and window-coherent (uses the same \(\tau\) and MECE right-open windowing as Appendix~G).
\end{itemize}

\begin{remark}[ECF and tower comparisons]
When \(\phi_{i,\tau}\) (Appendix~J) is evaluated after \(\CT\), the ECF laxator supplies a canonical ``low-pass'' candidate for compatibility with tensor products used inside tower morphisms; its use does not increase any windowed budget terms.
\end{remark}

\bigskip

\subsection*{M.4. Interaction with \texorpdfstring{$\tau$}{tau}-sweeps and stability bands}

\begin{definition}[\(\tau\)-sweep and stability band]
For a fixed degree \(i\) and window \([0,\sigma]\), a \(\tau\)-sweep probes \(\tau>0\) to identify stability bands where the functor \(\TT\) is locally constant in \(\tau\) (e.g.\ no bars of length in \((\tau-\eps,\tau+\eps)\) intersect \([0,\sigma]\)).
Within a stability band, small \(\tau\)-variations do not alter \(\betti_i(\TT M; t)\) on \([0,\sigma]\).
\end{definition}

\begin{proposition}[Robustness within stability bands]\label{prop:stability-band}
Fix \(i\) and \(\sigma>0\). Within a stability band for \(\tau\), the inequality
\[
  \PE_i^{\le \sigma}\big(\CT(F\otimes G)\big)\ \le\ \PE_i^{\le \sigma}\big(\CT F \otimes \CT G\big)
\]
is robust under small perturbations of \(\tau\): both sides remain constant in \(\tau\) on the band.
\end{proposition}

\begin{proof}
On a stability band, \(\betti_i(\CT(\cdot);t)\) is locally constant in \(\tau\) for all \(t\in[0,\sigma]\); integrate.
\end{proof}

\bigskip

\subsection*{M.5. Budget integration and quantitative gaps [Spec]}

\begin{definition}[Budget ledger]\label{def:budget-ledger}
Fix a window $[0,\sigma]$. A budget ledger records nonnegative error terms $\delta_\ell$
(from, e.g., numerical integration tolerance, discretization, or measured interleaving gaps)
and aggregates $\Sigma\delta := \sum_\ell \delta_\ell$.
Downstream $1$-Lipschitz operations (e.g.\ $\TT$, homology, degree projections) do not increase $\Sigma\delta$.
\end{definition}

\begin{proposition}[Integrating a measured laxity gap]\label{prop:lax-gap}
Suppose a metric comparison is available at the persistence layer:
\[
  \dint\!\Big(\TT \mathbf{P}_i\big(\CT(F\otimes G)\big),\ \TT \mathbf{P}_i\big(\CT F\otimes \CT G\big)\Big)\ \le\ \delta_{\mathrm{lax}}
\]
for some degree \(i\) and window \([0,\sigma]\). Then \(\delta_{\mathrm{lax}}\) may be recorded in the budget ledger for \([0,\sigma]\). Subsequent \(1\)-Lipschitz processing preserves this bound.
\end{proposition}

\begin{remark}[run.yaml alignment]
Record under \texttt{policy.after\_collapse\_only=true}, the window \([0,\sigma]\), the threshold \(\tau\), and—if used—the measured \(\delta_{\mathrm{lax}}\) in \texttt{budget.sum\_delta}. If ECF low-pass is in force, set \texttt{awfs\_2cell.two\_cell\_bounds.transfer\_collapse} to the configured bound and mirror it in \texttt{layered\_delta} (Appendix~G).
\end{remark}

\begin{remark}[When no metric control is available]
In the absence of such metric control, rely on the energy inequalities (Theorem~\ref{thm:conv}, Theorem~\ref{thm:mono-dominance}) and budget only measurement and A/B residuals arising from the pipeline.
\end{remark}

\bigskip

\subsection*{M.6. Edge cases and pitfalls}

\begin{itemize}[leftmargin=2em]
  \item Non-exact ``tensor'' surrogates. If the operation used in place of \(\otimes\) is not exact and biadditive, K\"unneth may fail and the convolution bound becomes invalid.
  \item Unverified K\"unneth. If (M2) is not verified in the context at hand, use the inequality form of Theorem~\ref{thm:conv}; do not claim equality.
  \item Missing monomorphy. Without (M3$^+$), the comparison \(\CT(F\otimes G)\to \CT F\otimes \CT G\) need not be energy-nonincreasing.
  \item Window mismatch. All measurements (Betti integrals, laxator rank checks) must use the same MECE, right-open windows and the same \(\tau\); otherwise splicing across windows may fail.
\end{itemize}

\bigskip

\subsection*{M.7. Worked examples and tests}

\begin{example}[Single-interval bars]
Let \(\mathbf{P}_p(F)=\kk_{[a,b)}\), \(\mathbf{P}_q(G)=\kk_{[c,d)}\) be the only nontrivial bars (other degrees vanish). Then by Remark~\ref{rem:interval-tensor},
\[
  \mathbf{P}_{p+q}(F\otimes G)\ \cong\ \kk_{[\max\{a,c\},\,\min\{b,d\})},
\]
and for any \(\sigma>0\),
\[
  \PE_{p+q}^{\le \sigma}(F\otimes G)
  \ =\ \lambda\!\Big([\max\{a,c\},\,\min\{b,d\})\cap[0,\sigma]\Big),
\]
where \(\lambda(\cdot)\) denotes Lebesgue measure (length).
\end{example}

\begin{example}[Collapsed intervals]
If \(b-a\le \tau\) or \(d-c\le \tau\), then \(\CT F\) or \(\CT G\) kills the corresponding bar. The right-hand side of the convolution bound reduces accordingly; if both bars are killed, then under (M3$^+$) the left-hand side \(\PE_{p+q}^{\le \sigma}(\CT(F\otimes G))\) must vanish.
\end{example}

\begin{proposition}[Test model for \((\mathrm{M3}^+)\): finite direct sums of intervals]\label{prop:test-m3plus}
Let \(F\simeq \bigoplus_r \kk_{I_r}[-p_r]\) and \(G\simeq \bigoplus_s \kk_{J_s}[-q_s]\) be finite direct sums of interval modules (shifted in homological degree). Suppose \(\CT\) removes all summands with length \(\le \tau\) and acts as the identity on others. Define \(\lambda_{\tau,F,G}\) on summands by the canonical inclusions
\[
  \CT(\kk_{I_r}\otimes\kk_{J_s})\ \hookrightarrow\ \CT\kk_{I_r}\otimes \CT\kk_{J_s},
\]
extended additively. Then \((\mathrm{M3}^+)\) holds, and for all \(i,\sigma>0\),
\[
  \PE_i^{\le \sigma}\big(\CT(F\otimes G)\big)\ \le\ \PE_i^{\le \sigma}\big(\CT F\otimes \CT G\big).
\]
\end{proposition}

\begin{proof}
On intervals, tensor is intersection; \(\CT\) either kills short intervals or leaves them unchanged. If either factor is killed, the target summand on the right is \(0\), and the source summand on the left is also \(0\) (intersection with a killed factor is empty), so the induced map is injective. If both factors survive, \(\CT\) acts as identity and the inclusion is literal. Additivity preserves injectivity. Energy nonincrease follows by integrating ranks of monomorphisms degreewise.
\end{proof}

\bigskip

\subsection*{M.8. Formal underpinnings: constructibility, measurability, exactness}

\begin{lemma}[Constructibility preserved by tensor]\label{lem:constructible-tensor}
Assume (M1). If \(M,N\in \Pers^\mathrm{c}_\kk\), then \(M\otimes N\in \Pers^\mathrm{c}_\kk\). If \(F,G\) are constructible filtered complexes, then \(F\otimes G\) is constructible.
\end{lemma}

\begin{proof}
See Remark~\ref{rem:constructible}. For filtered complexes, apply the same argument degreewise; bounded homological range on windows is preserved by tensor over a field.
\end{proof}

\begin{lemma}[Measurability and finiteness of energy]
For constructible \(F\), each \(\betti_i(F;\cdot)\) is piecewise constant with locally finite jumps. Hence \(\PE_i^{\le \sigma}(F)<\infty\) for all \(\sigma>0\).
\end{lemma}

\begin{proof}
Constructibility implies only finitely many sublevel changes on bounded windows. The Betti function is a finite sum of characteristic functions of intervals; integrability follows.
\end{proof}

\begin{lemma}[Exactness and 1-Lipschitzness of \(\TT\)]
Assume \(\TT\) is exact on short exact sequences and \(1\)-Lipschitz for \(\dint\). Then for any morphism \(f:M\to N\) in \(\Pers^\mathrm{c}_\kk\),
\[
  \dint(\TT M,\TT N)\ \le\ \dint(M,N).
\]
\end{lemma}

\begin{proof}
In barcode terms, \(\TT\) is realized by interval truncation which is nonexpansive for bottleneck distance; exactness follows from functoriality and additivity on intervals.
\end{proof}

\bigskip

\subsection*{M.9. Operational checklist (windowed, reproducible) [Spec]}

For each experiment window \([0,\sigma]\):
\begin{itemize}[leftmargin=2em]
  \item Record the tensor context: pairs \((F,G)\), monitored degrees \(i\), windows \([0,\sigma]\), and collapse thresholds \(\tau\).
  \item State whether K\"unneth (M2) is assumed/verified on the chosen windows; if not, use the \(\le\) form of Theorem~\ref{thm:conv}.
  \item State whether (M3) and optionally (M3$^+$) are assumed/verified on the chosen windows; e.g., check monomorphy of \(H_i(\lambda_{\tau,F,G}^t)\) at sampled \(t\). If using ECF low-pass, cite Section~M.3$^\dagger$.
  \item Fix numerical tolerances for spectra/Betti integrations and any clipping thresholds; enter them in the budget ledger as \(\delta\)-terms (Appendix~G).
  \item If a measured laxity gap \(\delta_{\mathrm{lax}}\) is available (Proposition~\ref{prop:lax-gap}), record it and include it in \(\Sigma\delta\).
\end{itemize}

\bigskip

\subsection*{M.10. Summary of contracts [Spec]}

Under exact pointwise tensor and a collapse-compatible laxator \(\CT(F\otimes G)\Rightarrow \CT F\otimes \CT G\), the clipped Betti integral of a tensor admits robust, windowed upper bounds expressed by overlap integrals of Betti curves (Theorem~\ref{thm:conv}, Proposition~\ref{prop:collapsed-conv}). With the additional monomorphy hypothesis \((\mathrm{M3}^+)\), \(\CT(F\otimes G)\) is energy-dominated by \(\CT F\otimes \CT G\) on each window (Theorem~\ref{thm:mono-dominance}). In the seminorm language, \(\|\cdot\|_{E_1}\) (which equals \(\PE\) on a window) and \(\|\cdot\|_\infty\) satisfy quantale-type submultiplicativity (Proposition~\ref{prop:quantale}); these seminorms are stable under clipping and collapse (Proposition~\ref{prop:clip-contract-stable}). The ECF low-pass laxator (Section~M.3$^\dagger$) provides an explicit, monomorphic \(\lambda\) satisfying the hypotheses, integrates with the windowed budget, and requires no further supplementation.

\bigskip

\subsection*{M.11. Formalization blueprint (Lean/Coq) [Spec]}

A minimal API includes:
\begin{itemize}[leftmargin=2em]
  \item An exact, biadditive \texttt{tensor} on \(\Pers^\mathrm{c}_\kk\) preserving constructibility; a lemma \texttt{kun\_neth} yielding
  \[
    \betti_i(F\otimes G;t)=\sum_{p+q=i}\betti_p(F;t)\,\betti_q(G;t)
  \]
  over a field.
  \item A natural transformation \texttt{laxator} \(\lambda_{\tau,F,G}:\CT(F\otimes G)\Rightarrow \CT F\otimes \CT G\), with an optional predicate \texttt{laxator\_mono}. Provide \texttt{laxator\_ecf} implementing Section~M.3$^\dagger$.
  \item Windowed seminorms \texttt{E1\_norm} and \texttt{Linf\_norm}, and lemmas \texttt{quantale\_submult}, \texttt{clip\_contract\_stable}, \texttt{PE\_conv\_bound}, \texttt{PE\_collapsed\_bound}, \texttt{mono\_dominance}.
  \item Hooks to record optional metric gaps as \(\delta\)-terms for a pipeline budget aligned with Appendix~G.
\end{itemize}



% =========================
\section*{Appendix N. Projection Formula and Base Change [Spec + Windowed Protocol + Budget Integration]}
% =========================
\phantomsection
\addcontentsline{toc}{section}{Appendix N. Projection Formula and Base Change}
\refstepcounter{section}
\label{N:pf-bc}

\paragraph{Standing conventions.}
We work over a coefficient \emph{field} \(\Lambda\) (e.g.\ a base field \(k\) or, at [Spec]-level, a Novikov field), and all statements below are phrased uniformly for \(\Lambda\).
All persistence modules are constructible (locally finite on bounded windows).
Filtered (co)limits are computed objectwise in \([\mathbb{R},\mathsf{Vect}_\Lambda]\) under the scope policy of Appendix~A, Remark~\ref{A:rk:filtered-colimits}, and then (when stated) returned to the constructible range.
The interleaving metric \(d_{\mathrm{int}}\) (=\ bottleneck in the constructible \(1\)D setting) is used throughout.
Truncation \(\mathbf{T}_\tau\) is exact and \(1\)\hyp Lipschitz (Appendix~A, Proposition~\ref{A:prop:lipschitz}); on the filtered\hyp complex side we use \(C_\tau\) \emph{up to f.q.i.}, and write \(\mathbf{P}_i\) for degree–\(i\) persistence.
Global conventions: \(\Ext^1\) is always taken against \(\Lambda[0]\); the energy exponent satisfies \(\alpha>0\) (default \(\alpha=1\)); windows are MECE and right–open (Appendix~G).
References to “infinite bars/generic dimension” point to Appendix~D, Remark~\ref{rem:D-generic-dim}.
Monotonicity claims follow the global policy: deletion-type only (nonincreasing), inclusion-type merely stable/nonexpansive.

% Number subsections in Appendix N as N.1, N.2, ...
\setcounter{subsection}{0}
\renewcommand\thesubsection{N.\arabic{subsection}}
\makeatletter
\renewcommand\@seccntformat[1]{\csname the#1\endcsname.\quad}
\makeatother

% -------------------------
\subsection{Hypotheses (PF/BC layer) and normalizations}
\label{N:hyp}
% -------------------------
We fix a class of filtered spaces and maps \(f:X\to Y\) for which the usual six–functor formalism is available on
\(D^b_c(\mathrm{Shv}_\Lambda(-))\) (bounded derived category of constructible \(\Lambda\)–sheaves), and adopt:

\begin{itemize}\itemsep0.35em
  \item[(N0)] \textbf{Coefficients.} \(\Lambda\) is a field; all objects have \emph{finite Tor-dimension} (no \(\mathrm{Tor}\) corrections).
  \item[(N1)] \textbf{Finiteness/constructibility.} All sheaves are constructible; the standard \(t\)-structure is used; per-object (co)homology is finite dimensional.
  \item[(N2)] \textbf{Proper/smooth hypotheses.} Projection formula (PF) and base change (BC) are taken under the usual hypotheses:
  \begin{itemize}\itemsep0.1em
    \item PF: for \(f\) \emph{proper}, \(Rf_\ast(A\otimes^\mathbf{L} f^\ast B)\simeq Rf_\ast A\otimes^\mathbf{L} B\).
    \item BC: for a Cartesian square with \(f\) proper (or smooth with the appropriate \(f^!\) variant),
    \(Lg^\ast Rf_\ast A \simeq Rf'_\ast Lg'{}^\ast A\).
  \end{itemize}
  \item[(N3)] \textbf{Degree normalization and objectwise evaluation.}
  We use the \emph{cohomological} convention on \(D^b_c\) and evaluate realizations \emph{objectwise in \(t\)}:
  \[
     \mathcal{R}(F)^t \cong \mathcal{R}(F^t),\qquad
        \mathbf{P}_i(F)(t)\ \cong\ H_i(F^t)\ \cong\ H^{-i}\!\big(\mathcal{R}(F^t)\big).
  \]
  Hence \(\mathbf{P}_i\) reads off the \((-i)\)-th cohomology sheaf along the filtration.
  Any geometric shift from \(f^!\) (smooth case) is \emph{absorbed} by this bookkeeping.
  \item[(N4)] \textbf{Tensor.} The tensor is pointwise in \(t\):
  \((A\otimes^\mathbf{L} B)^t \cong A^t\otimes^\mathbf{L} B^t\), exact over the field \(\Lambda\);
  constructibility is preserved (Appendix~H justifies Tonelli on bounded windows).
\end{itemize}

\begin{remark}[Scope and return to constructible]
All PF/BC comparisons below are formed in the derived category, computed objectwise in \(t\), and then passed to the persistence layer via \(\mathbf{P}_i\).
Any filtered colimit is taken in \([\mathbb{R},\mathsf{Vect}_\Lambda]\) under Appendix~A’s scope policy and \emph{returned to} \(\Pers^{\mathrm{ft}}_\Lambda\) (by verification of constructibility or by applying \(\mathbf{T}_\tau\)).
\end{remark}

% -------------------------
\subsection{Projection formula / base change at the persistence layer}
\label{N:pfbc-persistence}
% -------------------------
Let \(f:X\to Y\) and a Cartesian square
\[
\vcenter{\xymatrix{
X'\ar[r]^{g'}\ar[d]_{f'} & X\ar[d]^f\\
Y'\ar[r]^g & Y
}}
\]
satisfy \textup{(N0)–(N2)}.
For filtered complexes \(F\) on \(X\) and \(G\) on \(Y\), write \(\mathcal{R}(F),\mathcal{R}(G)\) for their realizations in \(D^b_c\), computed objectwise in \(t\).

\begin{theorem}[PF/BC transported to \texorpdfstring{$\mathbf{P}_i$ and $\mathbf{T}_\tau$}{P\_i and T\_\tau} \textup{[Spec]}]\label{N:thm:pf-bc}
Under \textup{(N0)--(N4)} the following canonical isomorphisms hold, \emph{natural in}
\((i,\tau,f,g,F,G)\) and \emph{up to f.q.i.} on the filtered–complex side; they are asserted \emph{after truncation by $\mathbf{T}_\tau$}:
\begin{align*}
\textup{(PF)}\quad
& \mathbf{T}_\tau\,\mathbf{P}_i\!\big(Rf_\ast(\mathcal{R}(F)\otimes^{\mathbf{L}} f^\ast\mathcal{R}(G))\big)
 \cong
 \mathbf{T}_\tau\,\mathbf{P}_i\!\big(Rf_\ast\mathcal{R}(F)\otimes^{\mathbf{L}}\mathcal{R}(G)\big),
\\[0.25em]
\textup{(BC)}\quad
& \mathbf{T}_\tau\,\mathbf{P}_i\!\big(Lg^\ast Rf_\ast \mathcal{R}(F)\big)
 \cong
 \mathbf{T}_\tau\,\mathbf{P}_i\!\big(Rf'_\ast Lg'{}^\ast \mathcal{R}(F)\big).
\end{align*}
\end{theorem}

\begin{proof}[Proof sketch]
PF and BC are canonical isomorphisms in \(D^{b}_{c}\) under \textup{(N0)--(N2)}.
Evaluate objectwise in \(t\) (N3), identify \(\mathbf{P}_i\) with \((-i)\)-cohomology in \(t\), and then apply \(\mathbf{T}_\tau\).
Exactness of \(\mathbf{T}_\tau\) (Appendix~A, Theorem~\ref{A:thm:localization}) preserves short exact sequences induced by PF/BC on cohomology sheaves; hence isomorphisms descend to the persistence layer \emph{after truncation}.
Naturality in \((f,g,F,G)\) follows from naturality of PF/BC; naturality in \((i,\tau)\) is clear from functoriality of \(\mathbf{P}_i\) and \(\mathbf{T}_\tau\).
\end{proof}

\begin{declaration}[PF/BC after collapse]\label{N:dec:pfbc-after}
Projection formula/base change are transported to persistence \emph{only after} applying \(C_\tau\) (equivalently, \(\mathbf{T}_\tau\) after \(\mathbf{P}_i\)).
All discretization/measurement errors are aggregated into \(\delta_{\mathrm{disc}}\) and \(\delta_{\mathrm{meas}}\) and combined by the budget’s (commutative) quantale sum.
\end{declaration}

\begin{corollary}[Compatibility with collapse at the filtered-complex layer]
Assume in addition that \(C_\tau\) realizes \(\mathbf{T}_\tau\) after applying \(\mathbf{P}_i\) (Appendix~B) up to f.q.i.
Then the PF/BC isomorphisms of Theorem~\ref{N:thm:pf-bc} hold with \(\mathcal{R}(C_\tau F)\) in place of \(\mathbf{T}_\tau\,\mathbf{P}_i(\mathcal{R}(F))\), \emph{after truncation by \(\mathbf{T}_\tau\) and up to f.q.i.\ on filtered complexes}.
\end{corollary}

\begin{remark}[What is \emph{not} claimed]
No global Lipschitz control for PF/BC is asserted beyond the \(1\)\nobreakdash-Lipschitz behavior of \(\mathbf{T}_\tau\) (Appendix~A) and any additional [Spec] commutation controls (Appendix~L).
PF/BC are exact identities at the sheaf layer; any persistence–level discrepancy indicates violated hypotheses or implementation drift and must be logged (see \S\ref{N:window-protocol}).
\end{remark}

% -------------------------
\subsection{Windowed protocol and reproducible audit}
\label{N:window-protocol}
% -------------------------
All PF/BC audits are \emph{windowed}. The mandatory comparison order is:
\[
\boxed{\ \text{for each } t\ \Longrightarrow\ \text{apply } \mathbf{P}_i\ \Longrightarrow\ \text{apply } \mathbf{T}_\tau\ \Longrightarrow\ \text{compare in }\Pers^{\mathrm{ft}}_\Lambda\ }.
\]
Use the \emph{same} MECE, right–open windows and the \emph{same} \(\tau\) as the rest of the run (Appendix~G).

\begin{declaration}[Audit checklist (per window; [Spec])]
Record in \texttt{run.yaml}:
(i) the PF/BC hypothesis set used (proper/smooth, finite Tor, degree normalization);
(ii) the functors and objects compared (e.g.\ \(Rf_\ast(\mathcal{R}(F)\otimes f^\ast\mathcal{R}(G))\) vs.\ \(Rf_\ast\mathcal{R}(F)\otimes\mathcal{R}(G)\));
(iii) the verdict (isomorphism detected) and, if any numerical/non–ideal drift is observed \emph{after truncation}, its breakdown into \(\delta^{\mathrm{disc}}\) and \(\delta^{\mathrm{meas}}\) together with tolerance(s);
(iv) the window and \(\tau\) used for the comparison, matching those used by B–Gate\(^{+}\) and the \(\delta\)–ledger (Appendix~G).
\end{declaration}

\begin{remark}[Definable coverage]\label{N:rk:def-cov}
If windows are definable in an o–minimal structure, event counts on bounded windows are finite (Appendix~H), coverage is decidable, and all PF/BC checks reduce to finitely many window–wise comparisons recorded in \texttt{run.yaml}.
\end{remark}

% -------------------------
\subsection{Budget integration and window pasting}
\label{N:budget}
% -------------------------
PF/BC isomorphisms themselves contribute \(\delta^{\mathrm{alg}}=0\) when the hypotheses hold.
In a \emph{pipeline} that also includes Mirror/Transfer steps and non–nested reflectors (Appendix~L/K), the \emph{window budget} on \(W\) (degree \(i\)) is
\[
\Sigma\delta_W(i)\ =\ \sum_{\text{Mirror--Collapse}} \delta_j(i,\tau_j)\ +\ \sum_{\text{A/B fails}} \Delta_{\mathrm{comm}}(M;A,B)\ +\ \sum_{\text{PF/BC}} \bigl(\delta^{\mathrm{disc}}+\delta^{\mathrm{meas}}\bigr),
\]
where the sum is taken in the budget’s commutative quantale (additive, monotone).
This budget is \emph{additive}, \emph{uniform in \(F\)} for Mirror–Collapse (Appendix~L), and \emph{non–increasing} under any subsequent \(1\)\hyp Lipschitz persistence post–processing (Appendix~L, Theorem~\ref{L:thm:budget}).
Window pasting then follows from Restart/Summability (Appendix~J): log \(\Sigma\delta_W(i)\) and \(\mathrm{gap}_\tau\) per window and verify \(\mathrm{gap}_\tau>\Sigma\delta_W(i)\).

% -------------------------
\subsection{Ext–tests under change of functor / coefficients}
\label{N:ext-tests}
% -------------------------
PF/BC isomorphisms transport \(\Ext^1\)–tests along canonical identifications:

\begin{proposition}[Portability of the \(\Ext^1\)–test \textup{(sheaf layer)}]
Under \textup{(N0)–(N2)} and Theorem~\ref{N:thm:pf-bc}, for any PF/BC isomorphism \(A \xrightarrow{\sim} B\) in \(D^{b}_{c}\), there is a natural isomorphism
\[
\Ext^1(A,\Lambda)\ \xrightarrow{\ \sim\ }\ \Ext^1(B,\Lambda).
\]
In particular, if \(\Ext^1(\mathcal{R}(C_\tau F),\Lambda)=0\), then \(\Ext^1\) also vanishes for any PF/BC partner of \(\mathcal{R}(C_\tau F)\).
\end{proposition}

\begin{remark}[Bridge stays one–way]
The one–way bridge \(\mathrm{PH}_1\Rightarrow \Ext^1\) (Appendix~C) is unchanged: PF/BC \emph{transport} the test across equivalent sheaf–theoretic descriptions. No converse or new implication is claimed.
\end{remark}

% -------------------------
\subsection{Functoriality and two–out–of–three (windowed Beck–Chevalley)}
\label{N:2of3}
% -------------------------
\begin{proposition}[Two–out–of–three for PF/BC squares \textup{(after collapse)}]
Fix a Cartesian square and \(\tau>0\).
If any two among
\[
\mathbf{T}_\tau\,\mathbf{P}_i(Lg^\ast Rf_\ast \mathcal{R}(F))\xrightarrow{\sim}
\mathbf{T}_\tau\,\mathbf{P}_i(Rf'_\ast Lg'{}^\ast \mathcal{R}(F)),\quad
\mathbf{T}_\tau\,\mathbf{P}_i(Rf_\ast(\mathcal{R}(F)\!\otimes\! f^\ast\mathcal{R}(G)))\xrightarrow{\sim}
\mathbf{T}_\tau\,\mathbf{P}_i(Rf_\ast\mathcal{R}(F)\!\otimes\!\mathcal{R}(G))
\]
and their base–changed analogues are isomorphisms, then the third is an isomorphism as well (all in \(\Pers^{\mathrm{ft}}_\Lambda\)).
\end{proposition}

\begin{proof}[Proof sketch]
Two–out–of–three holds in \(D^b_c\) for PF/BC; exactness of \(\mathbf{T}_\tau\) and objectwise evaluation transport it to \(\Pers^{\mathrm{ft}}_\Lambda\) after collapse.
\end{proof}

% -------------------------
\subsection{Implementation notes and checkpoints}
\label{N:impl}
% -------------------------
\begin{itemize}\itemsep0.35em
  \item \textbf{Finite windows / constructibility.} On bounded \(t\)–windows, bar events are finite (Appendix~H); PF/BC are computed objectwise in \(t\) and preserved by \(\mathbf{T}_\tau\).
  \item \textbf{Exactness bookkeeping.} Reductions to persistence use only:
  (i) PF/BC hold in \(D^b_c\);
  (ii) \(\mathbf{P}_i\) reads off \((-i)\)–cohomology in \(t\);
  (iii) \(\mathbf{T}_\tau\) is exact (Appendix~A, Theorem~\ref{A:thm:localization}) and \(1\)–Lipschitz (Appendix~A, Proposition~\ref{A:prop:lipschitz});
  (iv) filtered colimits respect the scope policy (Appendix~A, Remark~\ref{A:rk:filtered-colimits}).
  \item \textbf{Proper/smooth reminder.} We use the cohomological convention, and any \(f^!\)–induced shifts in the smooth case are absorbed by (N3). PF/BC are invoked under the proper/smooth hypotheses (N2).
  \item \textbf{Window coherence (non–negotiable).} All PF/BC audits must use the \emph{same} windows and \(\tau\) as those employed by B–Gate\(^{+}\) and the \(\delta\)–ledger (Appendix~G); otherwise budget accounting and pasting (Appendix~J) become invalid.
\end{itemize}

% -------------------------
\subsection{Formalization stubs (Lean/Coq) [Spec]}
\label{N:formal}
% -------------------------
A minimal API (cf.\ Appendix~F) includes:
\begin{itemize}\itemsep0.25em
  \item \texttt{pf\_iso}: \(Rf_\ast(A\otimes f^\ast B)\cong Rf_\ast A\otimes B\) under properness; \texttt{bc\_iso}: \(Lg^\ast Rf_\ast A\cong Rf'_\ast Lg'{}^\ast A\) for Cartesian squares (and smooth variants with \(f^!\));
  \item \texttt{to\_pers}: objectwise evaluation in \(t\) plus \(\mathbf{P}_i\) extraction and \(\mathbf{T}_\tau\) application to transport PF/BC to \(\Pers^{\mathrm{ft}}_\Lambda\);
  \item \texttt{pfbc\_pers\_nat}: naturality of these isomorphisms in \((i,\tau,f,g,F,G)\);
  \item hooks to log any residual numeric slack (post–truncation) as \(\delta^{\mathrm{disc}}\), \(\delta^{\mathrm{meas}}\) for the window budget (Appendix~L).
\end{itemize}

\medskip
\noindent\textbf{Summary.}
Under the standard PF/BC hypotheses over a field \(\Lambda\), projection formula and base change descend—via objectwise evaluation in \(t\), \(\mathbf{P}_i\), and the exact truncation \(\mathbf{T}_\tau\)—to canonical, \emph{natural} isomorphisms at the persistence layer, uniformly in \((i,\tau,f,g,F,G)\).
Comparisons follow the windowed protocol “for each \(t\) \(\to\) persistence \(\to\) collapse \(\to\) compare,” with the \emph{same} windows and \(\tau\) as the rest of the run; any residual numerical drift (post–truncation) is accounted for as \(\delta_{\mathrm{disc}}\) and \(\delta_{\mathrm{meas}}\) in the \(\delta\)–ledger and aggregated in the pipeline budget.
These audits integrate with Mirror/Transfer commutation (Appendix~L), multi–axis reflectors (Appendix~K), and Restart/Summability pasting (Appendix~J), while keeping the one–way bridge \(\mathrm{PH}_1\Rightarrow \Ext^1\) intact (Appendix~C).



% =========================
\section*{Appendix O. Fukaya Realization \& Stability [Spec + Permitted Ops + $\delta$-Ledger + B-Gate$^{+}$] (Reinforced)}
% =========================
\phantomsection
\addcontentsline{toc}{section}{Appendix O. Fukaya Realization \& Stability}
\refstepcounter{section}
\label{O:fukaya}

\paragraph{Standing conventions.}
We work over a coefficient \emph{field} \(\Lambda\) (e.g.\ a ground field \(k\) or a Novikov field).
All persistence modules are constructible (locally finite on bounded windows).
Filtered (co)limits are computed objectwise in \([\mathbb{R},\mathsf{Vect}_\Lambda]\) under the scope policy of Appendix~A, Remark~\ref{A:rk:filtered-colimits}, and then (when stated) returned to the constructible range.
The interleaving metric \(d_{\mathrm{int}}\) (which agrees with the bottleneck distance in the constructible \(1\)D setting) is used throughout.
Truncation \(\mathbf{T}_\tau\) is exact and \(1\)\hyp Lipschitz (Appendix~A, Proposition~\ref{A:prop:lipschitz}).
Global conventions: \(\Ext^1\) is always taken against \(\Lambda[0]\); the energy exponent satisfies \(\alpha>0\) (default \(\alpha=1\)).
References to “generic fiber dimension / infinite bars” point to Appendix~D, Remark~\ref{rem:D-generic-dim}.
Monotonicity claims follow the global policy: \emph{deletion-type only} (nonincreasing), inclusion-type merely stable/nonexpansive (Appendix~E).
Windows are MECE and right–open; bars are half-open \([b,d)\) (Appendix~H/G).

\begin{remark}[Right–open windows and half-open bars]\label{O:rmk:windows}
Right–open windows and half-open \([b,d)\) bars enforce MECE coverage and avoid double-counting at endpoints; events at a right boundary are attributed to the subsequent window.
\end{remark}

\begin{definition}[Filtered chain model and persistence]
A filtered chain complex over \(\Lambda\) means a chain complex \(C_\bullet\) equipped with an exhaustive, increasing filtration \(\{F^{t}C_\bullet\}_{t\in\mathbb{R}}\) by subcomplexes such that the differential preserves the filtration and continuation data act by filtered maps. For each homological degree \(i\), the degree–\(i\) persistence module is \(t\mapsto H_i(F^{t}C_\bullet)\in \mathsf{Vect}_\Lambda\). Constructibility on bounded windows means only finitely many jumps (break times) occur and vector-space ranks are finite on bounded intervals.
\end{definition}

\begin{definition}[Deletion-type morphism]\label{O:def:deletion}
A morphism \(M\to N\) in \(\Pers_\Lambda\) is \emph{deletion-type} on a window \(W\subset\mathbb{R}\) if, after restricting to \(W\) and post-composing with the truncation \(\mathbf{T}_\tau\) (for any \(\tau\ge 0\)), the induced map can only shorten or remove existing bars and cannot create new bars on \(W\). Equivalently, all deletion-type monotone indicators of Appendix~E are nonincreasing under this morphism.
\end{definition}

% -------------------------
\subsection*{O.1. Realization functor and hypotheses}
% -------------------------
\emph{[Spec]} Fix a Liouville/Weinstein sector \((X,\lambda)\) with a (possibly empty) system of \emph{stops}.
Write \(\mathsf{Fuk}(X;\mathrm{stops})\) for a wrapped/exact/monotone Fukaya-type category for which Floer-theoretic chain models admit an \emph{action filtration}.
We adopt the convention that the \emph{action value is the filtration parameter \(t\)}, increasing with larger action (so sublevel sets \(F^{t}\) mean action \(\le t\)).
We package the chain-level construction into a realization functor
\[
\mathcal{F}:\ \text{(geometric input)}\longrightarrow \mathsf{FiltCh}(\Lambda),
\]
natural in continuation data and stop operations, with degree–\(i\) persistence
\(\mathbf{P}_i\big(\mathcal{F}(-)\big)\in \Pers^{\mathrm{ft}}_\Lambda\) (constructible on bounded windows).
We assume:

\begin{itemize}\setlength{\itemsep}{0.35em}
  \item[(O0)] \textbf{Coefficients/admissibility.} \(\Lambda\) is a field; in the monotone/exact cases and with admissible almost complex structures, action and index filtrations are well defined; continuation solutions have finite energy.
  \item[(O1)] \textbf{Constructibility (action window).} On every bounded action window \([0,\sigma]\) the Floer complexes have finitely many generators and finitely many break times; hence \(\mathbf{P}_i(\mathcal{F}(-))\) is constructible. This local finiteness equally holds with Novikov coefficients on bounded windows.
  \item[(O2)] \textbf{Continuation shift bound.} Any continuation map for a homotopy of data with controlled action shift \(\varepsilon\) induces a filtered chain map whose filtration increase is \(\le \varepsilon\).
  \item[(O3)] \textbf{Stop operations are deletion-type.} Adding a stop or shrinking a sector removes generators and/or increases differentials in a way that corresponds to a \emph{deletion-type} operation at the persistence layer: in any fixed action window no new bars are created.
  \item[(O4)] \textbf{Up to filtered quasi-isomorphism.} Chain models are considered up to filtered quasi-isomorphism; all claims are invariant under f.q.i.\ by exactness of \(\mathbf{T}_\tau\) and the scope policy.
\end{itemize}

\begin{remark}[Action filtration normalization]
Our choice “action value equals filtration parameter” fixes the direction of filtration; monotone time reparametrizations that preserve order act by reindexings and, after normalization, give isometries in \(d_{\mathrm{int}}\).
\end{remark}

% -------------------------
\subsection*{O.2. Stability: continuation and stops}
% -------------------------
\begin{theorem}[Continuation \(1\)\hyp Lipschitz]\label{O:thm:cont}
Under \textup{(O2)}, for any two realizations related by a continuation with action shift \(\varepsilon\),
\[
d_{\mathrm{int}}\!\Big(\mathbf{P}_i(\mathcal{F}_0),\ \mathbf{P}_i(\mathcal{F}_1)\Big)\ \le\ \varepsilon,\qquad
d_{\mathrm{int}}\!\Big(\mathbf{T}_\tau\mathbf{P}_i(\mathcal{F}_0),\ \mathbf{T}_\tau\mathbf{P}_i(\mathcal{F}_1)\Big)\ \le\ \varepsilon
\]
for all \(i,\tau\ge 0\).
\end{theorem}

\begin{proposition}[Deletion-type monotonicity for stops]\label{O:prop:stops}
Under \textup{(O3)}, adding a stop or shrinking a sector induces, on any window and after \(\mathbf{T}_\tau\), a deletion-type morphism:
for every \(i\) and \(\tau\ge 0\),
\[
\mathbf{T}_\tau\,\mathbf{P}_i\big(\mathcal{F}_{\mathrm{with\ stop}}\big)\ \preceq\ \mathbf{T}_\tau\,\mathbf{P}_i\big(\mathcal{F}_{\mathrm{without\ stop}}\big),
\]
and all deletion-type monotone indicators (Appendix~E) are nonincreasing under this operation.
\end{proposition}

\begin{theorem}[Stop/continuation policy — reinforced]\label{thm:fukaya-policy}
In the action-filtered Fukaya realization, stop addition is deletion-type (post-collapse nonincrease) and \(\varepsilon\)-continuation maps are \(1\)-Lipschitz on \(\mathbf{T}_\tau\)-collapsed persistence. The quantitative \(\varepsilon\) is recorded as \(\delta^{\mathrm{alg}}=\varepsilon\) in the \(\delta\)-ledger; deletion-type steps contribute \(\delta^{\mathrm{alg}}=0\).
\end{theorem}

\begin{declaration}[Gate Cascade placement (after-collapse, B-side single layer)]\label{O:dec:gate-cascade}
Fukaya realizations enter the Gate Cascade at the \emph{B-side} (after applying \(\mathbf{T}_\tau\)), and \emph{before} any Mirror/Tropical post-processing (Appendix~L) or multi-axis reflector interactions (Appendix~K). All audits and budgets for Fukaya steps therefore use the order
\[
\boxed{\ \text{for each } t\ \Longrightarrow\ \mathbf{P}_i\ \Longrightarrow\ \mathbf{T}_\tau\ \Longrightarrow\ \text{Fukaya-op compare}\ },
\]
with the same window and \(\tau\) as B-Gate\(^{+}\) (Appendix~J/G).
\end{declaration}

\begin{remark}[Budget-adjusted continuation radius]
On a window \(W\) with additive \(\delta\)-ledger (Appendix~L/K), use the effective radius
\(\varepsilon_{\mathrm{eff}}:=\varepsilon+\Delta_W\) (Appendix~I) when invoking survival or matching claims; ledger \(\Delta_W\) aggregates only \emph{non-Fukaya} defects (e.g.\ Mirror/Transfer, A/B residuals).
\end{remark}

% -------------------------
\subsection*{O.3. Towers, comparison map, and diagnostics}
% -------------------------
Let \(F=(F_n)_{n\in I}\) be a directed system of geometric inputs (e.g.\ refining Hamiltonians/perturbations or nested stop systems) with colimit \(F_\infty\).
Apply \(\mathcal{F}\) and \(\mathbf{P}_i\) to obtain a tower in \(\Pers^{\mathrm{ft}}_\Lambda\).
For \(\tau\ge 0\) consider the comparison map (Appendix~J)
\[
\phi_{i,\tau}(F):\quad \varinjlim_n\ \mathbf{T}_\tau\!\big(\mathbf{P}_i(\mathcal{F}(F_n))\big)\ \longrightarrow\ \mathbf{T}_\tau\!\big(\mathbf{P}_i(\mathcal{F}(F_\infty))\big).
\]

\begin{definition}[Sufficient tower hypotheses]\label{O:def:tower-conds}
Assume: (S1) commutation up to vanishing error with truncation and reindexing; or (S2) no \(\tau\)-accumulation of break times; or (S3) Cauchy in \(d_{\mathrm{int}}\) with compatible structure maps. Each is made precise in Appendix~D, §D.3 and ensures stability of barcodes under the colimit and truncation.
\end{definition}

\begin{theorem}[When the comparison is an isomorphism]\label{O:thm:phi-iso}
If the tower admits continuation controls \(\varepsilon_n\to 0\) with
\(d_{\mathrm{int}}\!\big(\mathbf{P}_i(\mathcal{F}(F_n)),\mathbf{P}_i(\mathcal{F}(F_\infty))\big)\le \varepsilon_n\)
and satisfies any of (S1)/(S2)/(S3), then \(\phi_{i,\tau}(F)\) is an isomorphism for all \(\tau\ge 0\).
Consequently,
\[
\mu_{i,\tau}(F)=\nu_{i,\tau}(F)=0,
\]
where \((\mu,\nu)\) denote the generic fiber dimensions of the kernel/cokernel (Appendix~D, Remark~\ref{rem:D-generic-dim}).
\end{theorem}

\begin{corollary}[Grid \(\Rightarrow\) continuum survival]\label{O:cor:g2c}
In discretization towers (mesh \(h\to 0\)) with certified continuation bounds \(\varepsilon(h)\to 0\), any bar detected in a fixed window \([0,\tau_0]\) whose \(\mathbf{T}_{\tau_0}\)-clipped length exceeds \(2\varepsilon(h)\) persists in the limit (Appendix~I, Theorem~I:\ref{I:thm:g2c}).
\end{corollary}

% -------------------------
\subsection*{O.4. Permitted-operations table (windowed, post-collapse) and $\delta$-ledger}
% -------------------------
All comparisons follow the protocol “for each \(t\) \(\to\) persistence \(\to\) collapse \(\mathbf{T}_\tau\) \(\to\) compare,” on MECE, right–open windows and a fixed \(\tau\) (Appendix~G/N).
The table summarizes permitted operations, their type, quantitative contracts (after collapse), and how to record them in the \(\delta\)–ledger.

\begin{center}
\footnotesize
\setlength{\tabcolsep}{4pt}
\renewcommand{\arraystretch}{1.1}
\begin{tabularx}{\linewidth}{@{}p{.28\linewidth} p{.16\linewidth} X @{}}
\toprule
Operation & Type & Quantitative contract after $\mathbf{T}_\tau$ and ledger entry \\
\midrule
Add stop / shrink sector & Deletion & Nonincreasing deletion indicators; no new bars on the window; ledger: $\delta^{\mathrm{alg}}=0$, record $\delta^{\mathrm{disc}},\delta^{\mathrm{meas}}$ if any \\
Continuation (tame homotopy) & Shift & $d_{\mathrm{int}}\le \varepsilon$ (Thm.~\ref{O:thm:cont}); ledger: $\delta^{\mathrm{alg}}=\varepsilon$ \\
Hamiltonian change (bounded drift) & Shift & $d_{\mathrm{int}}\le \varepsilon$; ledger: $\delta^{\mathrm{alg}}=\varepsilon$ \\
Almost complex structure change (tame) & Shift & $d_{\mathrm{int}}\le \varepsilon$; ledger: $\delta^{\mathrm{alg}}=\varepsilon$ \\
Regrading / Maslov shift & Bookkeeping & Isometry (degree reindexing); ledger: $\delta^{\mathrm{alg}}=0$ \\
Monotone time reparametrization & Reindex & Isometry after reindex normalization; ledger: $\delta^{\mathrm{alg}}=0$ \\
Mirror/Transfer post-processing & External & As audited (Appendix~L); if commutation defect $\delta(i,\tau)$, ledger: $\delta^{\mathrm{alg}}=\delta(i,\tau)$ \\
Non-nested reflectors (if used) & External & A/B test or soft-commuting fallback (Appendix~K/L); ledger: $\delta^{\mathrm{alg}}=\Delta_{\mathrm{comm}}$ if fallback \\
\bottomrule
\end{tabularx}
\end{center}

\begin{definition}[$\delta$–ledger and budgets]\label{O:def:ledger}
Per window \(W=[u,u')\) and degree \(i\), define the aggregate budget
\[
\Sigma\delta_W(i)\ :=\ \sum_{\text{continuations}}\varepsilon\ +\ \sum_{\text{Mirror/Transfer}}\delta(i,\tau)\ +\ \sum_{\text{A/B fails}}\Delta_{\mathrm{comm}}\ +\ \sum_{\text{audits}}(\delta^{\mathrm{disc}}+\delta^{\mathrm{meas}}),
\]
where “audits” include PF/BC checks (Appendix~N) and numerical tolerances (Appendix~G).
\end{definition}

% -------------------------
\subsection*{O.5. B-Gate$^{+}$, restart, and summability (window pasting)}
% -------------------------
We adopt B-Gate\(^{+}\) with a per-window safety margin \(\mathrm{gap}_\tau(i)>0\) computed after \(\mathbf{T}_\tau\).
On window \(W\) and degree \(i\), the gate \emph{passes} if
\[
\mathrm{gap}_\tau(i)\ >\ \Sigma\delta_W(i).
\]
Across consecutive windows \((W_k)_k\), assume:
(i) transitions are finite compositions of deletion-type steps and \(\varepsilon\)-continuations (measured post-collapse), and
(ii) Summability holds \(\sum_k \Sigma\delta_{W_k}(i)<\infty\) (Appendix~J).
Then the Restart inequality of Appendix~J yields for some \(\kappa\in(0,1]\),
\[
\mathrm{gap}_{\tau_{k+1}}(i)\ \ge\ \kappa\Big(\mathrm{gap}_{\tau_k}(i)-\Sigma\delta_{W_k}(i)\Big),
\]
so positivity of the margin persists along the pipeline.
Per-window certificates therefore paste to a global certificate on \(\bigcup_k W_k\) (Appendix~J, Theorem~J:\ref{J:thm:pasting}).

\begin{remark}[Choice of \(\kappa\)]
The constant \(\kappa\) accounts for uniform losses at window transitions (e.g.\ finite alignment overheads or reindexing coercions). Under exact commutation, one can take \(\kappa=1\).
\end{remark}

% -------------------------
\subsection*{O.6. Windowed usage and run.yaml alignment}
% -------------------------
Record in \texttt{run.yaml} per window and degree:
\begin{itemize}\itemsep0.25em
  \item the operation sequence (stops/sector changes, continuations) with quantitative parameters (\(\varepsilon\), thresholds \(\tau\), sweep settings);
  \item the \(\delta\)–ledger entries and their sum \(\Sigma\delta_W(i)\);
  \item the B-Gate\(^{+}\) safety margin \(\mathrm{gap}_\tau(i)\) and pass/fail verdict;
  \item any external steps (Mirror/Transfer, reflectors) with A/B policy data (\(\eta\), \(\Delta_{\mathrm{comm}}\)) (Appendix~K/L);
  \item constructibility checks: upper bounds on generators and event counts (Appendix~H).
\end{itemize}
All diagnostics (\(\mu,\nu\), comparison maps \(\phi_{i,\tau}\)) are computed \emph{after} truncation \(\mathbf{T}_\tau\) and logged with the same window and \(\tau\).

% -------------------------
\subsection*{O.7. Failure modes and audit checklist}
% -------------------------
\noindent\emph{Failure modes (outside our scope).}
\begin{itemize}\itemsep0.25em
  \item \textbf{Loss of filtration control.} Non-exact data or bubbling may invalidate (O2); interleaving bounds then fail.
  \item \textbf{Near-threshold accumulation.} Type~IV behavior (Appendix~D) may produce bar-length accumulation at \(\tau\); comparison maps need not stabilize without (S2)/(S3).
  \item \textbf{Inclusion-type operations.} Removing stops or enlarging sectors can increase features; only stability (not monotonicity) is claimed, and only when continuation control is present.
\end{itemize}

\noindent\emph{Audit checklist (runtime verifications).}
\begin{enumerate}\itemsep0.25em
  \item Record continuation shift bounds \(\varepsilon\) and certify \(d_{\mathrm{int}}\)-nonexpansivity (Appendix~A).
  \item Verify constructibility on each window \([0,\sigma]\) (finite generators/events; Appendix~H) and log per-window counts (Appendix~G).
  \item For stop additions/sector shrinkage, mark operation as deletion-type and apply Appendix~E indicators.
  \item For towers, log \(\varepsilon_n\), check (S1)/(S2)/(S3) where applicable, and compute \((\mu,\nu)\); \(\phi_{i,\tau}\) iso \(\Rightarrow (\mu,\nu)=(0,0)\) (Appendix~J).
  \item If external functors are used, run Mirror/Transfer commutation audits and A/B tests (Appendix~L/K) and bookkeep residuals in \(\Sigma\delta_W(i)\).
\end{enumerate}

% -------------------------
\subsection*{O.8. Formalization stubs (Lean/Coq) [Spec]}
% -------------------------
A minimal API (cf.\ Appendix~F) includes:
\begin{itemize}\itemsep0.25em
  \item \texttt{fukaya\_realize}: returns an action-filtered chain model \(\mathcal{F}\) up to f.q.i., with constructibility on bounded windows (O1).
  \item \texttt{cont\_eps}: encodes continuation maps with filtration increase \(\le\varepsilon\) and yields \(d_{\mathrm{int}}\le\varepsilon\) (Theorem~\ref{O:thm:cont}).
  \item \texttt{stop\_delete}: produces deletion-type morphisms for stops/sector-shrink (Proposition~\ref{O:prop:stops}).
  \item \texttt{tower\_phi\_iso}: checks sufficient criteria so that \(\phi_{i,\tau}\) is an isomorphism and \((\mu,\nu)=(0,0)\) (Appendix~D/J).
  \item Hooks for \(\delta\)–ledger entries and B-Gate\(^{+}\) checks; restart/summability contracts (Appendix~J).
\end{itemize}

% -------------------------
\subsection*{O.9. Examples and regression tests}
% -------------------------
\begin{example}[Exact wrapped setting with a new stop]
Let \(X=T^*Q\) with its standard exact form and consider a wrapped setting with a stop at infinity. Adding an additional stop supported on a Legendrian subset removes Reeb chords crossing the stop. On any fixed action window, no new generators appear and differentials can only increase; hence the induced map is deletion-type (Proposition~\ref{O:prop:stops}). Deletion indicators (Appendix~E) weakly decrease; regression test: the windowed barcode total variation does not increase after \(\mathbf{T}_\tau\).
\end{example}

\begin{example}[Continuation bound from Hamiltonian drift]
Suppose \(H_0,H_1\) are cofinal Hamiltonians with \(\sup_{x,t}(H_1-H_0)\le \varepsilon\) on the relevant support and homotoped by a tame path. The action change along continuation solutions is bounded by \(\varepsilon\); thus the induced filtered map has filtration increase \(\le \varepsilon\), and \(d_{\mathrm{int}}\le \varepsilon\) (Theorem~\ref{O:thm:cont}). Regression test: pairwise barcode bottleneck distance never exceeds the certified \(\varepsilon\).
\end{example}

\begin{example}[Grid-to-continuum]
Discretize a time-dependent Floer datum with mesh \(h\) and certified continuation bound \(\varepsilon(h)=O(h^\alpha)\). For any fixed \(\tau_0\), bars of length \(>2\varepsilon(h)\) in \(\mathbf{T}_{\tau_0}\)-collapse survive in the limit (Corollary~\ref{O:cor:g2c}). Regression test: survival rates converge monotonically after accounting for \(\Sigma\delta_W(i)\).
\end{example}

% -------------------------
\subsection*{O.10. Summary}
% -------------------------
Floer-theoretic realizations with action filtration yield constructible persistence (O1).
Continuation with shift \(\varepsilon\) is \(1\)\hyp Lipschitz at the persistence level (O2 \(\Rightarrow\) Theorem~\ref{O:thm:cont}).
Adding stops or shrinking sectors is deletion-type and hence nonincreasing for all deletion indicators (Proposition~\ref{O:prop:stops}).
\emph{Placement in Gate Cascade:} Fukaya steps are audited \emph{after collapse} on the B-side single layer (Declaration~\ref{O:dec:gate-cascade}); the same window/\(\tau\) are used for B-Gate\(^{+}\), Mirror/Transfer, and PF/BC checks.
A windowed, post-collapse permitted-ops table prescribes how to assign \(\delta\)–ledger entries; B-Gate\(^{+}\) requires \(\mathrm{gap}_\tau>\Sigma\delta\) per window and pastes via Restart/Summability (Appendix~J).
Under standard tower hypotheses (Appendix~D), the comparison map \(\phi_{i,\tau}\) is an isomorphism and \((\mu,\nu)=(0,0)\) (Theorem~\ref{O:thm:phi-iso}); grid-to-continuum survival follows (Appendix~I).
All items respect the MECE/right–open window policy, are evaluated “\(t\)\(\to\)\(\mathbf{P}_i\)\(\to\)\(\mathbf{T}_\tau\)\(\to\)compare,” and integrate with the pipeline budget and A/B policy (Appendix~L/K) in a reproducible \texttt{run.yaml} workflow (Appendix~G).



% =========================
\section*{Appendix P. Tropical–LMHS Dictionary [Spec; after-collapse indicators only]}
% =========================
\phantomsection
\addcontentsline{toc}{section}{Appendix P. Tropical–LMHS Dictionary}
\refstepcounter{section}
\label{P:trop-lmhs}

\paragraph{Standing conventions.}
We work over a coefficient field \(k\).
All persistence modules are constructible on bounded windows; filtered (co)limits are computed objectwise in \([\mathbb{R},\mathsf{Vect}_k]\) under the scope policy of Appendix~A and returned to the constructible range when stated.
\emph{All numeric comparisons are evaluated after collapse} in the fixed order
\[
\boxed{\ \text{for each }t\ \Longrightarrow\ \mathbf{P}_i\ \Longrightarrow\ \mathbf{T}_\tau\ \Longrightarrow\ \text{compare on } \mathbf{T}_\tau\mathbf{P}_i\ },
\]
with the same MECE, right-open windows and the same \(\tau\) as elsewhere (Appendix~G/N).
Distances/defects aggregate in a declared commutative \emph{quantale} \(V\) with sum \(\oplus\) and scalar action \(\odot\) (Appendix~K/L/S); \(\mathbf{T}_\tau\) is exact and \(1\)-Lipschitz (Appendix~A).
The one-way bridge \(\mathrm{PH}_1\Rightarrow \Ext^1\) is used only in \(D^{\mathrm{b}}(k\text{-mod})\) and only forward (Chapter~3; Appendix~C).

% -------------------------
\subsection*{P.1. Dictionary contract (Spec) and safe use}
% -------------------------

\begin{declaration}[Tropical--LMHS dictionary (Spec; \emph{advisory, after-collapse})]\label{P:dec:dict}
On a definable window \(W\) (Appendix~H/J), a lookup procedure maps tropical inputs
\((\mathrm{val},\mathrm{trop}_\lambda,\text{fan data})\) to coarse LMHS proxies \((W_\bullet,N,h^{p,q}_\infty)\).
These proxies are used \emph{only} to propose \emph{after-collapse} indicators on \(\mathbf{T}_\tau\mathbf{P}_i(F|_W)\).
The dictionary is never a Gate; it can \emph{trigger} certified tests but cannot certify them.
\end{declaration}

\begin{remark}[Safe-use policy]\label{P:rmk:safe}
(i) Proposals are advisory and produce at most a budget entry \(\delta^{\mathrm{spec}}\in V\).  
(ii) Decisions (PF/BC, Overlap Glue, classification) rely only on provable after-collapse statements (Appendix~N/J/L).  
(iii) All claims are windowed and reproducible; events on bounded definable windows are finite and \v{C}ech depth is finite (Appendix~H/J).
\end{remark}

% -------------------------
\subsection*{P.2. Lookup targets and after-collapse indicators}
% -------------------------

\begin{center}
\small
\setlength{\tabcolsep}{7pt}
\renewcommand{\arraystretch}{1.08}
\begin{tabular}{@{}l l l@{}}
\toprule
Tropical input & LMHS proxy (coarse) & After-collapse indicator (advisory) \\
\midrule
\(\mathrm{val}(\text{moduli})\) step loci & weight jumps in \(W_\bullet\) & candidate deletion change-points (stop hints) \\
\(\mathrm{trop}_\lambda\) slopes (\(\lambda\to 0^+\)) & \(\operatorname{rk} N=\operatorname{rk}\log T_u\) & upper bounds for local \(E_1\) bins to test \\
Balanced-fan combinatorics & split vs.\ non-split type & additivity expectations for \((\mu,\nu)\) across overlaps \\
Tropical periods/lengths & limiting \(h^{p,q}_\infty\) pattern & energy-bin priorities; \(\varepsilon\)-survival hints \\
\bottomrule
\end{tabular}
\end{center}

\begin{remark}[Definable coverage]\label{P:rk:def}
Windows \(W\) are finite unions of half-open intervals definable in a fixed o-minimal structure; hence finite event decomposition and finite \v{C}ech depth hold, yielding finite verification cost for all local tests (Appendix~H/J).
\end{remark}

% -------------------------
\subsection*{P.3. Quantitative commutation and 2-cell bounds (after collapse)}
% -------------------------

\begin{declaration}[Tropical 2-cell bound]\label{P:dec:2cell}
Let \(\mathrm{trop}_\lambda\) be a Mirror/Tropical post-processing step that is \(1\)-Lipschitz and admits a controlled \(2\)-cell with collapse (Appendix~L).
On a window \(W\) and for degree \(i\), record a bound \(\delta_{\mathrm{trop}}(i,\tau)\in V\) such that
\[
d_V\!\big(\mathbf{T}_\tau\mathbf{P}_i(\mathrm{trop}_\lambda C_\tau F),\ \mathbf{T}_\tau\mathbf{P}_i(C_\tau \mathrm{trop}_\lambda F)\big)\ \preceq\ \delta_{\mathrm{trop}}(i,\tau).
\]
Enter \(\delta_{\mathrm{trop}}\) into the window budget (Appendix~L).
\end{declaration}

\begin{remark}[Nonexpansive measurement]\label{P:rk:nonexp}
All discrepancies are measured after \(\mathbf{T}_\tau\), hence any contraction in \(\mathrm{trop}_\lambda\) is preserved, and subsequent \(1\)-Lipschitz processing cannot increase recorded bounds (Appendix~L).
\end{remark}

% -------------------------
\subsection*{P.4. Confidence-weighted advisory defect \(\delta^{\mathrm{spec}}\)}
% -------------------------

\begin{definition}[Confidence weight and advisory defect]\label{P:def:conf}
Each dictionary proposal on \(W\) carries a confidence weight \(\omega(W)\in[0,1]\) (dimensionless).
Given a raw advisory magnitude \(\widehat{\delta}_{\mathrm{spec}}(i,\tau;W)\in V\) (e.g.\ from suggested bin gaps or slope tolerances), define the ledgered term
\[
\delta^{\mathrm{spec}}(i,\tau;W)\ :=\ \omega(W)\ \odot\ \widehat{\delta}_{\mathrm{spec}}(i,\tau;W)\ \in V.
\]
Calibration of \(\omega(W)\) may use held-out windows or stability-band cross-checks (Appendix~J/M); absent evidence, set \(\omega(W)=0\) (no cost).
\end{definition}

\begin{remark}[Summability on countable covers]
For a (locally finite) countable definable cover \(\{W_j\}\) of a bounded window, \(\sum_j\delta^{\mathrm{spec}}(i,\tau;W_j)\) is finite whenever \(\sum_j \omega(W_j)\odot \widehat{\delta}_{\mathrm{spec}}(i,\tau;W_j)\) converges in \(V\).
This is the \emph{T-Delta-Sum-Converges} condition, compatible with Restart/Summability (Appendix~J).
\end{remark}

% -------------------------
\subsection*{P.5. Local trigger via \(E_1\) (after-collapse)}
% -------------------------

\begin{declaration}[Advisory trigger]\label{P:dec:trigger}
If the dictionary suggests on \(W\) that the monodromy-rank proxy vanishes (no slope/facet change), it may trigger the certified test \(E_1(W)=0\) on \(\mathbf{T}_\tau\mathbf{P}_i(F|_W)\).
When verified, we may use
\[
E_1(W)=0\ \Longleftrightarrow\ \mathrm{PH}_1(\mathbf{T}_\tau F|_W)=0\ \Longleftrightarrow\ \Ext^1(\mathcal{R}(\mathbf{T}_\tau F)|_W,k)=0,
\]
as in Chapter~11/3. Failure records a nonzero \(\widehat{\delta}_{\mathrm{spec}}\) (e.g.\ minimal positive \(E_1\)-bin) and hence a ledgered \(\delta^{\mathrm{spec}}\) via \(\omega(W)\).
\end{declaration}

\begin{remark}[Overlap Glue with finite depth]\label{P:rk:glue}
On a definable cover of a bounded window, Overlap Glue terminates after finitely many checks (Appendix~J); any residuals are added to the budget by \(\oplus\).
\end{remark}

% -------------------------
\subsection*{P.6. Window budget and pasting}
% -------------------------

\begin{definition}[Window budget with advisory terms]\label{P:def:budget}
For window \(W\) and degree \(i\),
\[
\Sigma\delta_W(i)\ :=\ \sum_{\text{Mirror/Collapse}}\delta_{\mathrm{trop}}(i,\tau)\ \oplus\ \sum_{\text{A/B fails}}\Delta_{\mathrm{comm}}\ \oplus\ \sum_{\text{audits}}(\delta^{\mathrm{disc}}+\delta^{\mathrm{meas}})\ \oplus\ \sum_{\text{advisory}}\delta^{\mathrm{spec}}(i,\tau;W),
\]
where the sums are in \(V\) (Appendix~L/K/N). B-Gate\(^{+}\) requires \(\mathrm{gap}_\tau(i)>\Sigma\delta_W(i)\) (Appendix~J).
\end{definition}

\begin{remark}[T-Countable-Cover]
If a countable (locally finite) definable cover is used, Summability plus \(\sum_W \delta^{\mathrm{spec}}(i,\tau;W)<\infty\) ensures pasting across windows (Appendix~J).
\end{remark}

% -------------------------
\subsection*{P.7. Minimal working example (non-gating)}
% -------------------------

\noindent\emph{Single-facet pass.}
On \(W\), tropical slopes are constant; the dictionary proposes \(N=0\Rightarrow E_1(W)=0\) with \(\omega(W)=0.6\).
The certified test returns \(E_1(W)=0\) \emph{true}; no ledger entry is added.
If the test fails with smallest nonzero \(E_1\)-bin \(\widehat{\delta}_{\mathrm{spec}}=\varepsilon_\star\in V\), then \(\delta^{\mathrm{spec}}=0.6\odot \varepsilon_\star\) is added to \(\Sigma\delta_W(i)\).

% -------------------------
\subsection*{P.8. Reproducibility hooks (run.yaml)}
% -------------------------

\begin{center}
\small
\renewcommand{\arraystretch}{1.05}
\begin{tabular}{@{}l l@{}}
\toprule
Key & Meaning \\
\midrule
\texttt{quantale:\{name,op,unit,order\}} & Quantale \(V\) and its order/aggregation (Appendix~K/L/S). \\
\texttt{definable:\{structure,window\_formulae\}} & o-minimal structure and window formulas (Appendix~G/H/J). \\
\texttt{tropical:\{contraction\_kappa,bins\}} & Mirror/Tropical configuration and binning (Appendix~L/M). \\
\texttt{lmhs:\{proxies\}} & Enabled proxies: \texttt{rankN}, \texttt{weights}, \texttt{hpq\_inf}. \\
\texttt{two\_cell:\{delta\_trop\}} & Recorded \(\delta_{\mathrm{trop}}(i,\tau)\) bounds per window/degree. \\
\texttt{dict:\{omega,delta\_spec\}} & Confidence \(\omega(W)\in[0,1]\) and resulting \(\delta^{\mathrm{spec}}\). \\
\bottomrule
\end{tabular}
\end{center}

% -------------------------
\subsection*{P.9. Minimal API (pseudocode) [Spec]}
% -------------------------

\begin{verbatim}
def advisory_delta(window, degree, tau, raw_spec, omega):
    # raw_spec in V; omega in [0,1]; uses quantale scalar action ⊙
    return omega ⊙ raw_spec

def window_budget(entries):
    # entries: list of V-elements from tropical 2-cells, A/B residuals,
    # audits (disc/meas), and advisory deltas; aggregate by ⊕.
    Sigma = 0_V
    for v in entries: Sigma = Sigma ⊕ v
    return Sigma
\end{verbatim}

% -------------------------
\subsection*{P.10. What this appendix does \emph{not} assert}
No pre-collapse control; no commutation beyond recorded \(2\)-cell bounds; no equivalence beyond the one-way bridge \(\mathrm{PH}_1\Rightarrow \Ext^1\).
The dictionary never acts as a Gate and never overrides certified after-collapse tests.

% -------------------------
\subsection*{P.11. Summary}
% -------------------------
A tropical\(\to\)LMHS lookup provides \emph{after-collapse} advisory signals on definable windows.
Quantitative effects are integrated through (i) controlled Mirror/Tropical \(2\)-cell bounds \(\delta_{\mathrm{trop}}\), and (ii) a confidence-weighted advisory term \(\delta^{\mathrm{spec}}=\omega\odot \widehat{\delta}_{\mathrm{spec}}\).
All contributions enter the window budget via the quantale sum \(\oplus\) and are measured in the fixed order “for each \(t\)\(\to\)\(\mathbf{P}_i\)\(\to\)\(\mathbf{T}_\tau\)\(\to\)compare.”
With Summability (Appendix~J), windowwise certificates paste across (locally finite) countable covers.
The dictionary thus remains a reproducible, non-gating aid to certified after-collapse diagnostics, aligned with Appendices~K/L/N and the global run protocol of Appendix~G.




% =========================
\section*{Appendix Q. p-adic Definable Windows (Denef--Pas) [Spec; finite Čech and local comparisons after collapse]}
% =========================
\phantomsection
\addcontentsline{toc}{section}{Appendix Q. p-adic Definable Windows (Denef--Pas)}
\refstepcounter{section}
\label{Q:denef-pas}

\paragraph{Standing conventions.}
Let \(K\) be a non-archimedean local field with valuation \(\mathrm{val}:K^\times\!\to\!\mathbb{Z}\), residue field \(k\), and angular component \(\mathrm{ac}:K\to k\cup\{0\}\).
We work in the three-sorted Denef--Pas structure \((\mathrm{VF},\mathrm{RF},\mathrm{VG})=(K,k,\mathbb{Z})\) with the usual language and Presburger arithmetic on \(\mathrm{VG}\).
All persistence comparisons are \emph{after collapse}: for each time \(t\) we read \(\mathbf{P}_i(\mathbf{T}_\tau F)\) and compare there.
All distances/defects live in a fixed commutative quantale \(V\) and are recorded in the \(\delta\)-ledger (Appendix~G/K/L).
All definability is in Denef--Pas unless stated; on the \(\mathrm{VG}\)-sort we tacitly use the standard embedding \(\iota:\mathbb{Z}\hookrightarrow\mathbb{R}\) to align with the global time axis.

% -------------------------
\subsection*{Q.1. Denef--Pas definable windows}
% -------------------------
\begin{definition}[DP-definable right-open window]\label{Q:def:dp-window}
A \emph{right-open window} \(W\) is a subset of \(\mathrm{VG}\) definable in Denef--Pas that admits a finite disjoint-union representation
\[
W=\bigsqcup_{m=1}^M \{\,n\in\mathbb{Z}\mid a_m \le n < b_m,\ n\equiv r_m \!\!\!\pmod{c_m}\,\},
\]
with \(a_m\in\mathbb{Z}\cup\{-\infty\}\), \(b_m\in\mathbb{Z}\cup\{+\infty\}\), \(c_m\in\mathbb{Z}_{>0}\).
Via \(\iota\), we regard \(W\) as a right-open, piecewise-Presburger subset of the real time axis.
\end{definition}

\begin{remark}[Parametrised windows]\label{Q:rk:param}
For a DP-definable parameter set \(\Lambda\), a family \(\{W_\lambda\}_{\lambda\in\Lambda}\) is \emph{uniformly} DP-definable if it admits a decomposition as in Definition~\ref{Q:def:dp-window} with Presburger-definable \(a_m(\lambda),b_m(\lambda),r_m(\lambda),c_m(\lambda)\).
All bounds below are then uniform in \(\lambda\).
\end{remark}

% -------------------------
\subsection*{Q.2. Finite event and finite \v{C}ech properties}
% -------------------------
\begin{theorem}[DP windows: finite events and finite \v{C}ech depth]\label{Q:thm:dp-finite}
On any DP-definable right-open window \(W\) (Definition~\ref{Q:def:dp-window}):
\begin{enumerate}\itemsep4pt
  \item \textbf{Finite event decomposition.} By DP cell decomposition on \(\mathrm{VG}\), every integer-valued bookkeeping map used after collapse (bar endpoints in \(\mathbf{T}_\tau\mathbf{P}_i\), defect counters, bin assignments) is Presburger piecewise-constant/linear. Hence \(W\) admits a finite partition into cells on which these readouts are constant/linear, so only finitely many \emph{events} (value changes) occur on \(W\).
  \item \textbf{Finite \v{C}ech depth.} Any DP-definable cover \(\{W_\alpha\}\) of \(W\) has nerve dimension \(\le 1\) (one-parameter Presburger setting). Intersections beyond order \(2\) are empty; Overlap Glue therefore terminates after finitely many checks (Appendix~J).
  \item \textbf{After-collapse comparators.} PF/BC/Control comparisons (Appendix~N/R) are evaluated only on \(\mathbf{P}_i(\mathbf{T}_\tau F|_W)\). Any finite kernel/cokernel errors mandated by local control theorems are entered as \(\delta^{\mathrm{alg}}\) and summed in \(V\).
  \item \textbf{Reproducibility.} Event counts, nerve-depth bounds, and comparator \(\delta\)-budgets \emph{must} be logged in \texttt{run.yaml} under \texttt{definable} and \texttt{layered\_$\delta$} (Appendix~G).
\end{enumerate}
\end{theorem}

\begin{proof}[Proof sketch]
(1) Quantifier elimination and cell decomposition in Presburger arithmetic give piecewise-linearity/constancy on \(\mathrm{VG}\).
(2) In one dimension the nerve has dimension \(\le 1\); hence finite depth.
(3) follows from our mandatory order “for each \(t\)\(\to\)\(\mathbf{P}_i\)\(\to\)\(\mathbf{T}_\tau\)\(\to\)compare” and from exactness/\(1\)-Lipschitz of \(\mathbf{T}_\tau\).
(4) is bookkeeping.
\end{proof}

% -------------------------
\subsection*{Q.3. \(V\)-Lipschitz nonexpansion on DP windows}
% -------------------------
\begin{proposition}[Shift commutation and \(V\)-nonexpansion]\label{Q:prop:Vshift-DP}
Let \(S^v\) be a \(V\)-Lawvere shift (Chapter~2). Then \(\mathbf{T}_\tau\circ S^v \cong S^v\circ\mathbf{T}_\tau\) on \(W\), hence
\[
d_V\!\left(\mathbf{P}_i(\mathbf{T}_\tau F|_W),\ \mathbf{P}_i(\mathbf{T}_\tau G|_W)\right)\ \preceq\ d_V\!\left(\mathbf{P}_i(F|_W),\ \mathbf{P}_i(G|_W)\right).
\]
All distances are measured after collapse; subsequent \(1\)-Lipschitz operations do not increase them.
\end{proposition}

% -------------------------
\subsection*{Q.4. Local bridge trigger via \(E_1\) on DP windows}
% -------------------------
\begin{corollary}[Local bridge on DP windows]\label{Q:cor:dp-bridge}
On any DP-definable cover \(\{W_\alpha\}\) with finite nerve depth, if \(E_1(W_\alpha)=0\) is \emph{verified} after collapse, then
\[
E_1(W_\alpha)=0\ \Longleftrightarrow\ \mathrm{PH}_1(\mathbf{T}_\tau F|_{W_\alpha})=0\ \Longleftrightarrow\ \Ext^1(\mathcal{R}(\mathbf{T}_\tau F)|_{W_\alpha},k)=0,
\]
as in the local bridge (Chapter~3/11). No global converse is claimed.
\end{corollary}

% -------------------------
\subsection*{Q.5. Quantitative commutation for DP transfers}
% -------------------------
\begin{declaration}[DP-transfer \(2\)-cell bound (Spec)]\label{Q:dec:dp-2cell}
For any DP-definable transfer (e.g.\ \(p\)-adic tropicalization or functorial comparison) with contraction factor \(\kappa\in(0,1]\) on \(W\), record a \(2\)-cell bound \(\delta(i,\tau)\in V\) such that, \emph{after} \(\mathbf{T}_\tau\),
\[
d_V\!\left(\mathbf{P}_i(\mathbf{T}_\tau\,\mathrm{Trans}\,F),\ \mathbf{P}_i(\mathbf{T}_\tau\,\mathrm{Trans}\,G)\right)
\ \preceq\ 
\kappa\odot d_V\!\left(\mathbf{P}_i(\mathbf{T}_\tau F),\mathbf{P}_i(\mathbf{T}_\tau G)\right)\ \oplus\ \delta(i,\tau).
\]
Log \(\delta(i,\tau)\) as \(\delta^{\mathrm{Tr}}\) or \(\delta^{\mathrm{Fun}}\) in \texttt{run.yaml}.
\end{declaration}

% -------------------------
\subsection*{Q.6. \texorpdfstring{P6: \(\sum\delta\) is finite}{P6: sum-delta is finite} (Denef--Pas proof)}
% -------------------------
\begin{theorem}[T-Delta-Sum-Converges (P6) on DP windows]\label{Q:thm:P6}
Let \(\{W_j\}_{j\in J}\) be a locally finite DP-definable cover of a bounded valuation range \(I\subset\mathrm{VG}\).
Fix a degree \(i\) and a pipeline whose windowwise ledger decomposes as
\[
\Sigma\delta_{W_j}(i)\ =\ \sum_{\mathrm{Trans}}\!\delta^{\mathrm{Tr}}_{j}(i,\tau)\ \oplus\ \sum_{\mathrm{A/B\ fails}}\!\Delta_{\mathrm{comm},j}\ \oplus\ \sum_{\mathrm{audits}}\!(\delta^{\mathrm{disc}}_j+\delta^{\mathrm{meas}}_j)\ \oplus\ \sum_{\mathrm{advisory}}\!\delta^{\mathrm{spec}}_{j}(i,\tau).
\]
Then
\[
\sum_{j\in J}\ \Sigma\delta_{W_j}(i)\quad\text{converges in }(V,\oplus,\le).
\]
\end{theorem}

\begin{proof}
By Theorem~\ref{Q:thm:dp-finite}(1), each \(W_j\) has finitely many events; each recorded term is supported on either an event (A/B residuals, audits) or on a finite list of steps with uniform bounds (transfer \(2\)-cells).
Hence \(\Sigma\delta_{W_j}(i)\) is a finite \(\oplus\)-sum.
Local finiteness of the cover and nerve dimension \(\le 1\) (Theorem~\ref{Q:thm:dp-finite}(2)) imply bounded overlap multiplicity \(m\le 2\); summing over \(j\) counts each event at most \(m\) times, whence
\[
\bigoplus_{j\in J}\Sigma\delta_{W_j}(i)\ \le\ m\odot
\Big(\bigoplus_{\text{events in }I}\!\!\!\delta_{\text{event}}\ \oplus\ \bigoplus_{\text{step types}}\!\!\!\sup_I \delta_{\text{type}}\Big),
\]
which is a finite \(\oplus\)-sum in \(V\).
\end{proof}

\begin{corollary}[Unbounded ranges with geometric decay]\label{Q:cor:P6-unbounded}
If \(I=\mathrm{VG}\) is unbounded and there exists \(\rho\in(0,1)\), \(C\in V\) with per-level bounds \(\delta_{\text{type}}(n)\ \preceq\ C\odot \rho^{|n|}\) (e.g.\ contraction from valuation scale or transfer), then \(\sum_j\Sigma\delta_{W_j}(i)\) still converges in \(V\).
\end{corollary}

\begin{remark}[Tate/Igusa uniformity]\label{Q:rk:Tate-Igusa}
In DP-uniform families, constants controlling transfer/oscillatory terms admit \emph{uniform} bounds across parameters (Denef--Pas uniformity; Igusa/Tate local theory). Operationally, this provides \(C,\rho\) as in Corollary~\ref{Q:cor:P6-unbounded} independent of the fibre, ensuring \(\sum\delta<\infty\) fibrewise and uniformly.
\end{remark}

% -------------------------
\subsection*{Q.7. Reproducibility (run.yaml) for DP mode}
% -------------------------
\begin{verbatim}
definable:
  structure: "Denef-Pas"
  field: "Q_p"            # or finite extension; state uniformizer
  uniformizer: "pi"
  vg_scale: 1             # t = vg_scale * n + t0
  window_formulae:
    - "a1 <= n < b1 and n % c1 == r1"
    - "a2 <= n < b2"
layered_delta:
  delta^Gal: "<bound-or-rule>"
  delta^Tr:  "<bound-or-rule>"
  delta^Fun: "<bound-or-rule>"
two_cell:
  bound_delta: "<per-(i, tau) cap>"
tests:
  T_Vshift: true
  T_Definable: true
  T_Iwasawa: true
\end{verbatim}

% -------------------------
\subsection*{Q.8. Minimal working recipe (Spec; non-gating)}
% -------------------------
\begin{enumerate}\itemsep3pt
  \item Fix \(W\subset\mathrm{VG}\) by Presburger constraints (Definition~\ref{Q:def:dp-window}); log \texttt{window\_formulae}.
  \item Compute \(\mathbf{P}_i(\mathbf{T}_\tau F|_W)\); extract events by DP cell decomposition; log event count.
  \item Run \(E_1\)-bins after collapse on each cell; when \(E_1=0\) holds, apply Corollary~\ref{Q:cor:dp-bridge}.
  \item Apply PF/BC/Control comparators only after collapse; record finite errors as \(\delta^{\mathrm{alg}}\).
  \item For DP-transfers, register \(\kappa\) and \(\delta(i,\tau)\) per Declaration~\ref{Q:dec:dp-2cell}.
  \item For unbounded ranges, verify geometric decay constants \(C,\rho\) (Remark~\ref{Q:rk:Tate-Igusa}); conclude P6 via Corollary~\ref{Q:cor:P6-unbounded}.
\end{enumerate}

% -------------------------
\subsection*{Q.9. Non-claims}
% -------------------------
No pre-collapse control; no commutation beyond recorded \(2\)-cell bounds; no equivalence beyond the one-way bridge \(\mathrm{PH}_1\Rightarrow\Ext^1\) in \(D^{\mathrm{b}}(k\text{-mod})\); no uniform-in-\(p\) analytic bounds unless logged as \textbf{[Spec]} with explicit \(\delta\)-budgets.

% -------------------------
\subsection*{Q.10. Integration points}
% -------------------------
This appendix is consumed by Chapter~3 (local bridge), Chapter~7 (Iwasawa layers/control on overlaps), Chapter~9 (Langlands layerwise \(\delta\)-boxes), Chapter~10 (BSD/RH local tests on \(p\)-adic windows), Chapter~11 (\(E_1\) as first-class), Chapter~12 (T\_Definable/T\_Vshift/T\_Iwasawa).
All statements conform to the UCC-Contract in Chapter~1 and the AWFS/2-cell bookkeeping in Chapter~5 and Appendices~K/L.



% =========================
\section*{Appendix R. Iwasawa–AK Interface [Spec; Control $\Rightarrow$ Overlap Gate \& $\mu$–alignment guidelines]}
% =========================
\phantomsection
\addcontentsline{toc}{section}{Appendix R. Iwasawa–AK Interface}
\refstepcounter{section}
\label{R:iwasawa}

\paragraph{Standing conventions.}
Fix a prime \(p\). Let \(\Gamma\simeq \mathbb{Z}_p\) be the Galois group of a cyclotomic (or similar one-parameter) tower \(K_\infty/K\) with finite layers \(K_n\) and \(\Gamma_n=\Gamma^{p^n}\).
Write \(\Lambda=\mathbb{Z}_p\llbracket \Gamma\rrbracket\).
Let \((M_n)_{n\ge 0}\) be a compatible system of finite \(p\)-primary modules (e.g.\ class groups or Selmer quotients), and \(M_\infty=\varprojlim M_n\) a cofinitely generated \(\Lambda\)-module.
All persistence measurements are \emph{after collapse}: for each \(t\) we read \(\mathbf{P}_i(\mathbf{T}_\tau F)\) and compare there (Ch.~1/2).
Defects are aggregated in a fixed commutative quantale \(V\) and recorded in the \(\delta\)-ledger (App.~G/K/L).
Windows are definable o\hyp minimally (real side) or in Denef–Pas (App.~Q); finite event/\v{C}ech properties are assumed (App.~H/J/Q).
Every reference to kernels/cokernels being \emph{finite} means finite \(p\)-groups.

% -------------------------
\subsection*{R.1. Control maps and Overlap Gate}
% -------------------------
\begin{definition}[Control map]\label{R:def:control}
A \emph{control map} at level \(n\) is a natural comparison
\[
\Phi_n:\ (M_\infty)_{\Gamma_n}\ \longrightarrow\ M_n
\]
with finite kernel and cokernel. Here \((-)_{\Gamma_n}\) denotes coinvariants.
\end{definition}

\begin{definition}[Overlap Gate (AK)]\label{R:def:overlap-gate}
On a definable cover \(\{W_\alpha\}\) of a bounded time window, an Overlap Gate requires that every comparison on overlaps
\[
\vartheta_{\alpha\beta}:\ \mathbf{P}_i(\mathbf{T}_\tau F|_{W_{\alpha}\cap W_{\beta}})\ \longrightarrow\ \mathbf{P}_i(\mathbf{T}_\tau G|_{W_{\alpha}\cap W_{\beta}})
\]
is an isomorphism \emph{up to finite kernel/cokernel}. The finite parts are entered in the ledger as \(\delta_{\mathrm{alg}}\) and summed in \(V\).
\end{definition}

\begin{definition}[Budget embedding]\label{R:def:eta}
Fix a monotone embedding \(\eta:\mathbb{R}_{\ge 0}\to V\) (e.g.\ \(\eta(x)=x\) for \(V=([0,\infty],+,0)\)).
The \emph{control budget} recorded on a window \(W_\alpha\) is
\[
\delta_{\mathrm{ctrl}}(n;\,W_\alpha)\ :=\ \eta\!\left(\,v_p\!\left|\ker\Phi_n\right|+v_p\!\left|\coker\Phi_n\right|\,\right)\ \in V.
\]
\end{definition}

\begin{proposition}[Control \(\Rightarrow\) Overlap Gate]\label{R:prop:control-overlap}
Assume \emph{uniform} control (Def.~\ref{R:def:control}) on a definable cover \(\{W_\alpha\}\) with finite \v{C}ech depth (App.~J/Q).
Then, after collapse, the induced maps
\[
\vartheta_{n,\alpha}:\ \mathbf{P}_i\!\left(\mathbf{T}_\tau (M_\infty)_{\Gamma_n}\big|_{W_\alpha}\right)\ \longrightarrow\ \mathbf{P}_i\!\left(\mathbf{T}_\tau M_n\big|_{W_\alpha}\right)
\]
satisfy the Overlap Gate (Def.~\ref{R:def:overlap-gate}).
Moreover, for each \((i,\tau,\alpha)\),
\[
\delta_{\mathrm{alg}}(i,\tau;\,W_\alpha)\ \preceq\ \delta_{\mathrm{ctrl}}(n;\,W_\alpha),
\]
and the total algebraic budget over the cover obeys
\[
\bigoplus_{\alpha}\ \delta_{\mathrm{alg}}(i,\tau;\,W_\alpha)\ \preceq\ m\ \odot\ \sup_{\alpha}\delta_{\mathrm{ctrl}}(n;\,W_\alpha),
\]
where \(m\) is the overlap multiplicity (in the 1D setting \(m\le 2\)).
\end{proposition}

\begin{proof}[Proof idea]
Apply \(\mathbf{T}_\tau\) first (exact, \(1\)-Lipschitz). Finite kernels/cokernels survive as bounded defects recorded by \(\eta\). Finite \v{C}ech depth bounds the number of overlap checks; quantale subadditivity (App.~K/L) yields the stated bounds.
\end{proof}

\begin{corollary}[Finite glue with budget]\label{R:cor:glue}
Under Proposition~\ref{R:prop:control-overlap}, Overlap Glue terminates after finitely many checks and the bound
\(
\delta_{\mathrm{alg}}^{\mathrm{tot}}\ \preceq\ m\odot \sup_{\alpha}\delta_{\mathrm{ctrl}}(n;\,W_\alpha)
\)
controls any residual non\hyp isomorphism after collapse.
\end{corollary}

\begin{remark}[Cofinal invariance]\label{R:rk:cofinal}
Passing to a cofinal subsequence in \(n\) preserves the Overlap Gate verdict and the AK diagnostics \((\mu,\nu)\) (App.~D/J). Uniform bounds on \(v_p|\ker\Phi_n|\), \(v_p|\coker\Phi_n|\) keep budgets uniformly bounded.
\end{remark}

% -------------------------
\subsection*{R.2. $\mu$–alignment: decision tree (window–local, non-identification)}
% -------------------------
Let \(\mu_{\mathrm{class}}\) be the classical Iwasawa \(\mu\)–invariant of the torsion \(\Lambda\)\hyp module \(M_\infty\) (when defined), and let \(\mu_{\mathrm{Collapse}}\) be the AK tower diagnostic computed \emph{after} \(\mathbf{T}_\tau\) on definable windows (Ch.~4; App.~D).

\begin{declaration}[$\mu$–alignment decision tree (Spec)]\label{R:dec:mu-align}
Work on a definable cover with finite \v{C}ech depth; all budgets are after collapse and summed in \(V\).
\begin{enumerate}\itemsep3pt
  \item \textbf{Precheck (torsion).} If \(M_\infty\) has nonzero \(\Lambda\)\hyp rank, declare \(\mu_{\mathrm{class}}=\mathrm{NA}\) and skip alignment (report free rank instead). Otherwise proceed.

  \item \textbf{Control\,+\,deletion regime (Coincidence).}
  Assume the control budgets are uniformly bounded
  \[
    \sup_{n}\bigl(v_p\lvert\ker\Phi_n\rvert + v_p\lvert\coker\Phi_n\rvert\bigr) < \infty,
  \]
  and the post\hyp collapse pipeline on the window is \emph{deletion\hyp type only}
  (no inclusion\hyp type amplification; App.~E/O). Then
  \[
    \boxed{\,\mu_{\mathrm{Collapse}}=\mu_{\mathrm{class}}\ \text{ on the window.}\,}
  \]

  \item \textbf{Bounded control, general regime (Lower bound).} Under the bounded control hypothesis alone (no inclusion\hyp type guarantee),
  \[
    \boxed{\,\mu_{\mathrm{Collapse}} \ge \mu_{\mathrm{class}}\ \text{ on the window.}\,}
  \]

  \item \textbf{Sublinear drift (Bracket).} If control budgets grow sublinearly, i.e.\ \(v_p\lvert\ker\Phi_n\rvert + v_p\lvert\coker\Phi_n\rvert = o(p^n)\), then with
  \(
  \mu_{\mathrm{drift}}:=\limsup_{n\to\infty}\frac{1}{p^n}\,\eta^{-1}\!\big(\delta_{\mathrm{ctrl}}(n)\big)
  \)
  we have the certified bracket
  \[
    \boxed{\,\mu_{\mathrm{class}} \le \mu_{\mathrm{Collapse}} \le \mu_{\mathrm{class}}+\mu_{\mathrm{drift}}\ \text{ on the window.}\,}
  \]

  \item \textbf{Uncontrolled growth (No claim).} If \(\delta_{\mathrm{ctrl}}(n)\) is not \(o(p^n)\), do not align; report \(\mu_{\mathrm{Collapse}}\) with the observed ledger trend.
\end{enumerate}

All statements are \textbf{window–local} and \textbf{non-identification} outside (2).
\end{declaration}

\begin{remark}[What each $\mu$ measures]\label{R:rk:mu-meaning}
\(\mu_{\mathrm{class}}\) is the \(p\)–exponent in the characteristic ideal of the torsion part of \(M_\infty\) (growth \(|M_n|\sim p^{\mu p^n+\lambda n+\nu}\)).
\(\mu_{\mathrm{Collapse}}\) is the persistent obstruction rate recorded by the AK diagnostic after \(\mathbf{T}_\tau\) (App.~D).
Equality in (2) follows because bounded control errors and deletion\hyp type steps cannot reduce the persistent obstruction rate, while no inclusion\hyp type amplification is present.
\end{remark}

% -------------------------
\subsection*{R.3. Gate coupling and restart/summability}
% -------------------------
\begin{definition}[Gate coupling]\label{R:def:gate-couple}
Couple the Overlap Gate with B\hyp Gate\(^{+}\) (App.~J): on a window \(W\) and degree \(i\), the gate passes if
\(
\mathrm{gap}_\tau(i)\ >\ \Sigma\delta_W(i)
\)
with \(\Sigma\delta_W(i)\) including \(\delta_{\mathrm{ctrl}}\).
\end{definition}

\begin{proposition}[Restart and P6]\label{R:prop:restart-P6}
If consecutive windows are linked by deletion\hyp type steps and \(\varepsilon\)\hyp continuations (measured after collapse), and the \(\delta\)–budgets satisfy \(\sum_W\Sigma\delta_W(i)<\infty\) (P6; App.~J/Q), then the Restart inequality of App.~J propagates a positive safety margin and windowwise certificates paste to a global one on \(\bigcup W\).
\end{proposition}

% -------------------------
\subsection*{R.4. Reproducibility hooks (run.yaml)}
% -------------------------
\begin{verbatim}
iwasawa:
  tower: "cyclotomic"           # description
  gamma_presentation: "Z_p"     # Γ ≅ Z_p, generator choice
  levels: {start: n0, end: n1}
  control_bounds:
    ker_vp_sup: "<finite/expr>"
    coker_vp_sup: "<finite/expr>"
  mu_targets:
    classical_mu: "<value/NA>"
    collapse_mu_policy: "after-collapse, deletion-type only"
overlap_gate:
  cover: "<o-minimal or Denef-Pas>"
  nerve_depth_bound: 2
layered_delta:
  delta^Gal: "<bound-or-rule>"
  delta^Fun: "<bound-or-rule>"
  delta^alg_control: "<η(v_p|ker|+v_p|coker|)>"
b_gate_plus:
  gap_tau: "<value per window/degree>"
  passed: true/false
mu_decision:
  status: "coincidence | lower_bound | bracket | no_claim | NA"
  mu_drift: "<value if bracket>"
tests:
  T_Iwasawa: true
\end{verbatim}

% -------------------------
\subsection*{R.5. Minimal decision routine (Spec; non-gating)}
% -------------------------
\begin{verbatim}
# inputs: classical_mu (or NA), budgets δ_ctrl(n), pipeline_type in {deletion_only, general}
def mu_alignment(classical_mu, deltas_ctrl, pipeline_type):
    if classical_mu == "NA": return ("NA", None)
    bounded = sup_n(deltas_ctrl[n]) < ∞
    if bounded and pipeline_type == "deletion_only":
        return ("coincidence", 0)
    if bounded:
        return ("lower_bound", 0)
    drift = limsup_n ( η^{-1}(deltas_ctrl[n]) / p**n )
    if drift == 0:
        return ("lower_bound", 0)
    if drift < ∞:
        return ("bracket", drift)
    return ("no_claim", None)
\end{verbatim}

% -------------------------
\subsection*{R.6. Edge cases and guard-rails}
% -------------------------
\begin{itemize}\itemsep0.25em
  \item \textbf{Non-torsion modules.} If \(\mathrm{rank}_\Lambda M_\infty>0\), report \(\mu_{\mathrm{class}}=\mathrm{NA}\) and do not align; optionally report the free rank and \(\lambda\)–like trends separately.
  \item \textbf{Inclusion-type steps.} Any inclusion\hyp type amplification (post-collapse) invalidates Coincidence; fall back to Lower bound/Bracket.
  \item \textbf{Character twists.} Alignment is stable under finite character twists; record twists in \texttt{run.yaml}.
  \item \textbf{Cofinal tails.} Replacing \((M_n)\) by a cofinal subsequence preserves all decisions and budgets (App.~J).
\end{itemize}

% -------------------------
\subsection*{R.7. Formalization stubs (Lean/Coq) [Spec]}
% -------------------------
\begin{verbatim}
structure ControlMap (n : ℕ) :=
  (phi : coinv Γ_n M_∞ ⟶ M_n)
  (ker_fin : finite_p (kernel phi))
  (coker_fin : finite_p (cokernel phi))

def ctrl_budget (η : ℝ≥0 → V) (φ : ControlMap n) : V :=
  η (vp (kernel φ).card + vp (cokernel φ).card)

theorem control_implies_overlap
  (H : ∀ n, ControlMap n) (cover : DefinableCover) :
  OverlapGate_after_collapse H cover
\end{verbatim}

% -------------------------
\subsection*{R.8. Summary}
% -------------------------
Uniform control maps yield Overlap Gates after collapse with algebraic budgets proportional (via \(\eta\)) to \(v_p|\ker|+v_p|\coker|\).
Finite \v{C}ech depth ensures finite glue; budgets aggregate in the quantale \(V\) and satisfy Restart/Summability (P6).
The window–local decision tree aligns \(\mu_{\mathrm{Collapse}}\) with \(\mu_{\mathrm{class}}\) under bounded control and deletion\hyp type pipelines, provides a certified lower bound otherwise, and a bracket in the presence of sublinear drift; no identification is claimed beyond these hypotheses.
All contracts are measured after collapse and logged in \texttt{run.yaml} in a reproducible manner.



% =========================
\section*{Appendix S. Quantale Catalog [Spec; selection policy for the $\delta$-ledger \& $V$-metrics]}
% =========================
\phantomsection
\addcontentsline{toc}{section}{Appendix S. Quantale Catalog}
\refstepcounter{section}
\label{S:quantale}

\paragraph{Standing conventions.}
All \emph{quantitative} measurements are evaluated \emph{after collapse}, i.e.
\[
\boxed{\ \text{for each } t\ \Longrightarrow\ \mathbf{P}_i\ \Longrightarrow\ \mathbf{T}_\tau\ \Longrightarrow\ \text{compare in }\Pers^{\mathrm{ft}}\ },
\]
and all defects are written to a fixed commutative \emph{quantale} \(V\) via the \(\delta\)-ledger (entries \(\delta^{\mathrm{alg}},\delta^{\mathrm{disc}},\delta^{\mathrm{meas}},\delta^{\mathrm{spec}}\)).
Finite definable covers on bounded windows (App.~H/J/Q) ensure that all aggregations and joins below are \emph{finite}.
By Chapter~2, \(\mathbf{T}_\tau\) is \(V\)\nobreakdash-\(1\)\nobreakdash-Lipschitz and idempotent.

% -------------------------
\subsection*{S.1. Axioms in the implementable range}
% -------------------------
A \emph{(finite) commutative quantale} for this pipeline is a tuple \((V,\oplus,0,\le)\) with:
\begin{enumerate}\itemsep4pt
  \item[(Q0)] \((V,\le)\) a poset admitting finite joins \(\bigvee\).
  \item[(Q1)] \((V,\oplus,0)\) a commutative monoid; \(a\oplus 0=a\).
  \item[(Q2)] Monotonicity: \(a\le a'\), \(b\le b'\) \(\Rightarrow\) \(a\oplus b\le a'\oplus b'\).
  \item[(Q3)] Finite distributivity: \(a\oplus(b\vee c)=(a\oplus b)\vee(a\oplus c)\).
  \item[(Q4)] (Optional) scalar action \(\odot\) by a commutative monoid \(R\) (e.g.\ \(\mathbb{R}_{\ge 0}\)) such that \(r\odot(a\vee b)=(r\odot a)\vee(r\odot b)\) and \(r\odot(a\oplus b)=(r\odot a)\oplus(r\odot b)\). Used for contraction factors \(\kappa\in(0,1]\) (App.~L/M/P/Q).
\end{enumerate}
A \emph{Lawvere \(V\)\nobreakdash-distance} on a set \(X\) is a map \(d_V:X\times X\to V\) with \(0\preceq d_V(x,x)\) and
\(d_V(x,z)\preceq d_V(x,y)\oplus d_V(y,z)\).

% -------------------------
\subsection*{S.2. Canonical choices \& intended semantics (all [Spec])}
% -------------------------

\paragraph{S.2.1. Additive budget}
\[
V_{\mathrm{add}}:=([0,\infty],\ \oplus=+,\ 0,\ \le).
\]
\emph{Use when} many small contributors accumulate (triangle\hyp type bounds, numerical tolerances, arithmetic control totals). Compatible with Restart/Summability (App.~J).

\paragraph{S.2.2. Tropical worst\hyp case}
\[
V_{\max}:=([0,\infty],\ \oplus=\max,\ 0,\ \le).
\]
\emph{Use when} a single dominant spike controls acceptance (tightest overlap, largest commutation defect). Fits sup\hyp style audits and safety margins.

\paragraph{S.2.3. Probabilistic amount\(\times\)confidence (finite pre\hyp quantale)}
\[
V_{\mathrm{prob}}:=([0,\infty]\times[0,1],\ \oplus=(+\!,\cdot),\ 0=(0,1),\ \le=\text{product order}).
\]
\emph{Use when} each defect has a magnitude and an independent success probability. Finite distributivity (Q3) holds on our finite covers; this suffices.

\paragraph{S.2.4. Product quantales}
Given \(V_1,V_2\) as above,
\[
V_1\times V_2,\qquad (a_1,a_2)\oplus(b_1,b_2):=(a_1\oplus_1 b_1,\ a_2\oplus_2 b_2),
\]
ordered coordinatewise. \emph{Use when} two audit channels must be retained: e.g.\ \(V_{\mathrm{add}}\times V_{\max}\) (cumulative amount \emph{and} worst local spike), or \(V_{\mathrm{add}}\times V_{\mathrm{prob}}\) (amount with confidence).

\begin{remark}[Embedding of numeric budgets]\label{S:rk:eta}
Arithmetic/control numbers (e.g.\ \(v_p|\ker|+v_p|\coker|\) in App.~R) are injected by a fixed monotone \(\eta:\mathbb{R}_{\ge 0}\to V\): \(\eta(x)=x\) for \(V_{\mathrm{add}}\) or \(V_{\max}\), \(\eta(x)=(x,1)\) for \(V_{\mathrm{prob}}\), and componentwise for products.
\end{remark}

% -------------------------
\subsection*{S.3. Interoperability with the pipeline}
% -------------------------
\begin{itemize}\itemsep3pt
  \item \textbf{Ledger aggregation.} The ledger stores \(V\)\nobreakdash-valued entries and aggregates by \(\oplus\). 2\hyp cell bounds from Mirror/Transfer/Tropical are recorded as \(\delta^{\mathrm{alg}}\) (App.~L/M/P/Q).
  \item \textbf{\(V\)\nobreakdash-metrics and shifts.} \(d_V\) induces \(V\)\nobreakdash-shifts \(S^v\) (Ch.~2). Tests \(T_{\mathrm{Vshift}}\) enforce \(\mathbf{T}_\tau\circ S^v\simeq S^v\circ\mathbf{T}_\tau\) and nonexpansion after collapse.
  \item \textbf{Towers and diagnostics.} Subadditivity/additivity and cofinal invariance for \((\mu,\nu)\) are checked in the chosen \(V\) (App.~D/J). For \(V_{\max}\), “nonincrease’’ means bottleneck nonworsening; for \(V_{\mathrm{add}}\), it means total budget nonincrease under deletion\hyp type moves.
  \item \textbf{Arithmetic layers.} Control\hyp theorem finite parts embed via \(\eta\) (App.~R); PF/BC residuals enter as \(\delta^{\mathrm{disc}},\delta^{\mathrm{meas}}\) (App.~N).
\end{itemize}

% -------------------------
\subsection*{S.4. Selection guidelines (window\hyp local, audit\hyp first)}
% -------------------------
Choose \(V\) \emph{per window} to match the decision objective (record in \texttt{run.yaml.quantale}).
\begin{enumerate}\itemsep3pt
  \item \textbf{Safety\hyp margin dominated} (``did anything blow up?''): pick \(V_{\max}\).
  \item \textbf{Cumulative risk accounting} (``how much budget remains?''): pick \(V_{\mathrm{add}}\).
  \item \textbf{With confidence bookkeeping} (lab/estimator outputs): pick \(V_{\mathrm{prob}}\) or \(V_{\mathrm{add}}\times V_{\mathrm{prob}}\).
  \item \textbf{Dual\hyp view audits} (amount \emph{and} worst spike): pick \(V_{\mathrm{add}}\times V_{\max}\).
\end{enumerate}

% -------------------------
\subsection*{S.5. Change of quantale (safe coercions)}
% -------------------------
A \emph{quantale morphism} \(\phi:(V,\oplus,0,\le)\to(W,\boxplus,0',\preceq)\) is a monotone monoid map with \(\phi(0)=0'\) and \(\phi(a\oplus b)\preceq \phi(a)\boxplus\phi(b)\).
Such \(\phi\) preserves acceptance bounds: any inequality proved in \(V\) remains valid after applying \(\phi\) to the ledger.
\begin{align*}
&V_{\mathrm{add}}\!\to V_{\max}:\ \phi(x)=x,\qquad
V_{\mathrm{prob}}\!\to V_{\mathrm{add}}:\ \phi(x,c)=x,\\
&V\to V\times W:\ a\mapsto (a,\phi'(a))\ \text{(diagonal tracking of a second channel)}.
\end{align*}

\begin{proposition}[Safe coercion monotonicity]
If \(\phi\) is a quantale morphism and \(\Sigma\delta_V\) is a ledger total in \(V\), then for any gate threshold \(\mathrm{gap}\) one has
\(
\Sigma\delta_V\le \mathrm{gap}\ \Rightarrow\ \phi(\Sigma\delta_V)\preceq \phi(\mathrm{gap})
\)
in \(W\).
\end{proposition}

% -------------------------
\subsection*{S.6. Reproducibility keys (mandatory)}
% -------------------------
\begin{verbatim}
quantale:
  name: "V_add"          # or V_max, V_prob, V_addxV_max, ...
  op: "+"                # "+", "max", "(+,·)", "product"
  unit: "0"              # "0" or "(0,1)" etc.
  order: "<="            # coordinatewise when product
  scalar_action: true    # enable κ-scaling for 2-cells if used
  eta_embed: "x->x"      # or "(x,c)->x" etc.
  change_of_base: []     # list of safe coercions φ (if any)
tests:
  T_Vshift: true
  T_Vsubadd: true
  T_ChangeOfBase: true
\end{verbatim}

% -------------------------
\subsection*{S.7. Non-claims}
% -------------------------
We do not assume infinite distributivity nor completeness beyond finite joins (definable finiteness suffices).
No pre\hyp collapse metric statements are made; all \(V\)\nobreakdash-Lipschitz claims are \emph{after} \(\mathbf{T}_\tau\).

% -------------------------
\subsection*{S.8. Integration map}
% -------------------------
Chapter~1 (UCC) pins \(V\) in scope; Chapter~2 uses \(V\)\nobreakdash-metrics and shifts; Chapter~4/App.~D/J compute \((\mu,\nu)\) in \(V\); App.~L/M/P/Q/R inject 2\hyp cell and arithmetic budgets via \(\eta\); App.~N feeds PF/BC residuals; App.~G/Ch.~12 enforce \texttt{run.yaml} compliance and tests.



% =========================
\section*{Appendix T. Implementation Notes / Notebooks [Spec; script skeletons, Gate Cascade, Convergence Manager, counterexample hunter, \& CI demos]}
% =========================
\phantomsection
\addcontentsline{toc}{section}{Appendix T. Implementation Notes / Notebooks}
\refstepcounter{section}
\label{T:notebooks}

\paragraph{Standing conventions.}
All quantitative evaluations are \emph{after collapse} (for each \(t\to \mathbf{P}_i\to \mathbf{T}_\tau\to\) compare; Ch.~1/2).
Windows are definable (o\hyp minimal or Denef--Pas; App.~H/J/Q) so events/\v{C}ech depth are finite.
Distances and \(\delta\)\nobreakdash-budgets live in a fixed commutative quantale \(V\) (App.~S).
Arithmetic overlaps use Control\(\Rightarrow\)Overlap Gate with finite parts logged as \(\delta^{\mathrm{alg}}\) (App.~R).
Notebook templates below are \textbf{[Spec]} and enable the tests of Ch.~12.
Gate Cascade follows the after\hyp collapse order; the Convergence Manager enforces Restart/Summability (App.~J) windowwise.

% -------------------------
\subsection*{T.1. Directory layout (minimal)}
% -------------------------
\begin{verbatim}
ak16/
  run.yaml                   # manifest (Appendix G)
  data/                      # raw inputs (tropical/LMHS/p-adic/Selmer)
  cache/                     # intermediate artifacts (per τ, per window)
  logs/                      # δ-ledger snapshots and audit trails
  notebooks/
    01_tropical.ipynb
    02_lmhs.ipynb
    03_padic.ipynb
    04_selmer.ipynb
    05_convergence_manager.ipynb
    06_gate_cascade.ipynb
    07_counterexample_hunter.ipynb
    demo_gl1_min.ipynb
    demo_gl2_min.ipynb
  scripts/
    tau_sweep.py
    mece_check.py
    delta_aggregate.py
    convergence_manager.py
    gate_cascade.py
    hunter_generate.py
    demo_gl1.py
    demo_gl2.py
\end{verbatim}

% -------------------------
\subsection*{T.2. Manifest fragments (per stage)}
% -------------------------
Record mandatory keys (Appendix~G) with stage-local fields. Use identical window names across stages.

\paragraph{T.2.1. Global, windows, \(\tau\)-sweep}
\begin{verbatim}
quantale:
  name: "V_addxV_max"       # Appendix S
  op: "product"
  unit: "(0,0)"
  order: "coordinatewise"
definable:
  structure: "R_an,exp"     # or "Denef-Pas"
  window_formulae:
    - "W1: a1 <= t < b1"
    - "W2: b1 <= t < b2"
windows:
  base: "W1,W2"             # MECE target; enforced by tests
tau:
  grid: {start: 0.0, stop: 3.0, step: 0.1}  # τ-sweep
tests:
  T_Vshift: true
  T_Definable: true
  T_Vsubadd: true
\end{verbatim}

\paragraph{T.2.2. Tropical \(\to\) LMHS}
\begin{verbatim}
tropical:
  contraction_kappa: 0.9
  bins: [0, 0.25, 0.5, 1.0]
  export: "cache/tropical_{window}_{tau}.json"
lmhs:
  proxies: ["rankN", "weights", "h_infty"]
  export: "cache/lmhs_{window}_{tau}.json"
two_cell:
  bound_delta: "δ^Tr <= 0.05 per window"
\end{verbatim}

\paragraph{T.2.3. p-adic (Denef--Pas) \(\to\) Selmer}
\begin{verbatim}
padic:
  structure: "Denef-Pas"
  field: "Q_p"
  p: 3
  uniformizer: "p"
  vg_scale: 1
  export: "cache/padic_{window}_{tau}.json"
selmer:
  curve: "E: y^2 = x^3 - x"   # placeholder
  level_range: {start: 0, end: 8}
  control_bounds:
    ker_vp_sup: 2
    coker_vp_sup: 3
  export: "cache/selmer_{window}_{tau}.json"
layered_delta:
  delta^Gal: "eta(ker+coker)"     # Appendix S/R
\end{verbatim}

\paragraph{T.2.4. Convergence Manager}
\begin{verbatim}
convergence:
  restart_policy:
    kappa: 1.0                    # App. J/O; window transition factor
    max_restarts: 2
  summability:
    tol: 1e-6                     # threshold on Σδ decrement
    max_iters: 20
  stable_band_scan:
    neighborhood: 0.02            # τ-band half-width
    min_band_width: 0.05
\end{verbatim}

\paragraph{T.2.5. Gate Cascade (after-collapse only)}
\begin{verbatim}
gate_cascade:
  order: ["B_GatePlus", "PF_BC", "OverlapGate"]
  B_GatePlus:
    gap_key: "gap_tau"            # measured after T_tau
  PF_BC:
    enforce: true                  # Appendix N after-collapse isomorphisms
    budget_keys: ["δ^disc", "δ^meas"]
  OverlapGate:
    enforce: true                  # Appendix R (control ⇒ overlap)
    budget_key: "δ^Gal"
\end{verbatim}

\paragraph{T.2.6. Counterexample hunter}
\begin{verbatim}
hunter:
  generators:
    - "typeIV_accumulation"       # near-τ bar-length pileup
    - "mirror_comm_defect"        # large 2-cell δ(i,τ)
    - "inclusion_spike"           # gate-breaking inclusion-type step
  budget_caps:
    δ^Tr: 0.2
    δ^Fun: 0.2
    δ^Gal: 5
  export: "cache/hunter_{window}_{tau}.json"
\end{verbatim}

\paragraph{T.2.7. CI demos (GL(1)/GL(2) minimal workflows)}
\begin{verbatim}
demo:
  gl1:
    tower: "cyclotomic"
    levels: {start: 0, end: 5}
    class_module: "Cl(K_n)[p^\infty]"    # toy surrogate
  gl2:
    curve: "E: y^2 = x^3 - x"            # minimal EC example
    levels: {start: 0, end: 4}
    selmer_p: 3
\end{verbatim}

% -------------------------
\subsection*{T.3. \texorpdfstring{$\tau$}{tau}-sweep driver (skeleton)}
% -------------------------
\begin{verbatim}
# scripts/tau_sweep.py  (pseudocode; after-collapse policy enforced)
from ak16 import pipeline

cfg = pipeline.load_manifest("run.yaml")
for W in cfg.windows.base:
    cells = pipeline.definable_cells(W, cfg)             # finite (App. H/J/Q)
    for tau in pipeline.grid(cfg.tau.grid):
        # 1) Tropical stage (Appendix L/P)
        trop = pipeline.tropical_readout(W, tau, cfg)
        pipeline.ledger.add("delta^Tr", W, tau, trop.delta)

        # 2) LMHS proxies (Appendix P; advisory only)
        lmhs = pipeline.lmhs_proxy(trop, cfg)
        pipeline.cache.save(lmhs, f"cache/lmhs_{W}_{tau}.json")

        # 3) p-adic transfer (Appendix Q)
        pad = pipeline.padic_readout(W, tau, cfg)
        pipeline.ledger.add("delta^Fun", W, tau, pad.delta)

        # 4) AFTER-COLLAPSE persistence and energy
        PiT = pipeline.read_persistence_after_collapse(W, tau, cfg)
        E1  = pipeline.energy_bins(PiT, cfg)
        if E1.is_zero():
            pipeline.bridge_certify(PiT, W, tau)         # local PH1⇒Ext1

        # 5) Arithmetic control on overlaps (Appendix R)
        ctl = pipeline.control_bounds(W, cfg)
        pipeline.ledger.add("delta^Gal", W, tau, ctl.delta)

pipeline.ledger.flush("logs/ledger.json")
\end{verbatim}

% -------------------------
\subsection*{T.4. Convergence Manager (restart \& stability bands)}
% -------------------------
\begin{verbatim}
# scripts/convergence_manager.py (pseudocode; Appendix J/O)
from ak16 import pipeline, quantale as V

cfg = pipeline.load_manifest("run.yaml")
κ     = cfg.convergence.restart_policy.kappa
tol   = cfg.convergence.summability.tol
imax  = cfg.convergence.summability.max_iters

def iterate_until_stable(W, tau0):
    τ = tau0
    Σ_prev = V.infty()
    for it in range(imax):
        PiT = pipeline.read_persistence_after_collapse(W, τ, cfg)
        δTot = pipeline.ledger.window_total(W, τ)   # ⊕-sum (Appendix S)
        gap = pipeline.safety_margin(PiT, τ)        # B-Gate^+ margin
        if V.le(δTot, gap):                         # gate passes
            # Stability band scan
            band = pipeline.find_stability_band(W, τ, cfg.convergence.stable_band_scan)
            return {"tau": τ, "band": band, "delta_total": δTot}
        # Restart step (App. J): shrink τ or adjust window
        τ = pipeline.restart_update(τ, κ)
        if V.distance(δTot, Σ_prev) <= tol:
            break
        Σ_prev = δTot
    return {"tau": τ, "band": None, "delta_total": Σ_prev, "status": "max_iters"}

# Example driver
for W in cfg.windows.base:
    τ0 = cfg.tau.grid.start
    info = iterate_until_stable(W, τ0)
    pipeline.cache.save(info, f"cache/convergence_{W}.json")
\end{verbatim}

% -------------------------
\subsection*{T.5. Gate Cascade (B-Gate$^{+}\to$ PF/BC \texorpdfstring{$\to$}{->} Overlap)}
% -------------------------
\begin{verbatim}
# scripts/gate_cascade.py (pseudocode; Appendix O/N/R, after-collapse)
from ak16 import pipeline, quantale as V

def run_cascade(W, tau, cfg):
    PiT = pipeline.read_persistence_after_collapse(W, tau, cfg)
    gap = pipeline.safety_margin(PiT, tau)
    Σδ  = pipeline.ledger.window_total(W, tau)

    # 1) B-Gate^+ (Appendix O/J)
    if not V.lt(Σδ, gap):
        return {"stage": "B_GatePlus", "pass": False, "gap": gap, "delta": Σδ}

    # 2) PF/BC after-collapse (Appendix N)
    pfbc_ok, drift = pipeline.pfbc_check_after_collapse(W, tau)
    pipeline.ledger.add("δ^disc", W, tau, drift.disc)
    pipeline.ledger.add("δ^meas", W, tau, drift.meas)
    if not pfbc_ok:
        return {"stage": "PF_BC", "pass": False, "drift": drift}

    # 3) Overlap Gate (Appendix R)
    ov_ok, ctl = pipeline.overlap_gate(W, tau)
    pipeline.ledger.add("delta^Gal", W, tau, ctl.delta)
    return {"stage": "OverlapGate", "pass": ov_ok, "control": ctl}

# Notebook cell (06_gate_cascade.ipynb):
# result = run_cascade("W1", tau, cfg)
\end{verbatim}

% -------------------------
\subsection*{T.6. Counterexample hunter (adversarial generators)}
% -------------------------
\begin{verbatim}
# scripts/hunter_generate.py (pseudocode; App. D/E/L/Q)
from ak16 import pipeline

def gen_typeIV_accumulation(W, tau, density=50):
    # Produce near-τ bar-length pileup (finite on bounded window)
    return pipeline.synthetic.pileup(W, tau, density=density)

def gen_mirror_comm_defect(W, tau, size=0.15):
    # Force a large Mirror↔Collapse 2-cell defect (Appendix L/P)
    return pipeline.synthetic.mirror_defect(W, tau, target_delta=size)

def gen_inclusion_spike(W, tau, factor=2.0):
    # Inject inclusion-type step to try to break deletion-only guards
    return pipeline.synthetic.inclusion_spike(W, tau, factor=factor)

def hunt_all(W, tau, cfg):
    cases = []
    for name, gen in [("typeIV", gen_typeIV_accumulation),
                      ("mirror", gen_mirror_comm_defect),
                      ("incl", gen_inclusion_spike)]:
        F = gen(W, tau)
        PiT = pipeline.to_after_collapse(F, W, tau)
        δ  = pipeline.measure_defects(PiT, W, tau)
        cases.append({"name": name, "delta": δ})
        pipeline.cache.save(PiT, f"cache/hunter_{name}_{W}_{tau}.json")
    return cases
\end{verbatim}

% -------------------------
\subsection*{T.7. MECE window coverage check (test hook)}
% -------------------------
\begin{verbatim}
# scripts/mece_check.py (pseudocode)
from ak16 import windows

W = windows.from_manifest("run.yaml")
assert windows.is_right_open(W)                  # half-open convention
assert windows.is_mece(W.base)                   # pairwise disjoint
assert windows.cover_domain(W.base)              # union equals target
windows.log_partition(W, out="logs/windows_partition.txt")
\end{verbatim}

% -------------------------
\subsection*{T.8. \texorpdfstring{$\delta$}{delta}-box aggregation (per window, per \texorpdfstring{$\tau$}{tau})}
% -------------------------
\begin{verbatim}
# scripts/delta_aggregate.py (pseudocode; Appendix S semantics)
from ak16 import ledger, quantale as V

L = ledger.load("logs/ledger.json")
for (W, tau) in L.keys():
    δTr   = L.get("delta^Tr",  W, tau, default=V.zero())
    δFun  = L.get("delta^Fun", W, tau, default=V.zero())
    δGal  = L.get("delta^Gal", W, tau, default=V.zero())
    δDisc = L.get("δ^disc",    W, tau, default=V.zero())
    δMeas = L.get("δ^meas",    W, tau, default=V.zero())
    δAlg  = L.get("delta^alg", W, tau, default=V.zero())
    δTot  = V.op(δTr, V.op(δFun, V.op(δGal, V.op(δDisc, V.op(δMeas, δAlg)))))
    ledger.store_total(W, tau, δTot)
ledger.write("logs/δ_total.json")
\end{verbatim}

% -------------------------
\subsection*{T.9. Notebook cell skeletons (per stage)}
% -------------------------
Each cell assumes \emph{after-collapse} inputs and writes cache + ledger entries.

\paragraph{T.9.1. 01\_tropical.ipynb}
\begin{verbatim}
cfg = load_yaml("run.yaml")
cells = definable_cells("W1", cfg)
trop = tropical_readout("W1", tau=cfg.tau.grid.start, cfg=cfg)
ledger.add("delta^Tr", "W1", cfg.tau.grid.start, trop.delta)
save_json(trop, "cache/tropical_W1_{tau}.json")
\end{verbatim}

\paragraph{T.9.2. 02\_lmhs.ipynb}
\begin{verbatim}
trop = load_json("cache/tropical_W1_{tau}.json")
lmhs = lmhs_proxy(trop, proxies=cfg.lmhs.proxies)  # advisory, non-gating
save_json(lmhs, "cache/lmhs_W1_{tau}.json")
\end{verbatim}

\paragraph{T.9.3. 03\_padic.ipynb}
\begin{verbatim}
pad = padic_readout("W1", tau, field=cfg.padic.field, p=cfg.padic.p)
ledger.add("delta^Fun", "W1", tau, pad.delta)
save_json(pad, "cache/padic_W1_{tau}.json")
\end{verbatim}

\paragraph{T.9.4. 04\_selmer.ipynb}
\begin{verbatim}
ctl = control_bounds(levels=cfg.selmer.level_range)        # Appendix R
ledger.add("delta^Gal", "W1", tau, ctl.delta)
PiT = read_persistence_after_collapse("W1", tau)            # AFTER-COLLAPSE
E1  = energy_bins(PiT)
if E1.is_zero(): bridge_certify(PiT, "W1", tau)             # PH1⇒Ext1
\end{verbatim}

\paragraph{T.9.5. 05\_convergence\_manager.ipynb}
\begin{verbatim}
info = convergence.iterate_until_stable("W1", tau0=cfg.tau.grid.start)
save_json(info, "cache/convergence_W1.json")
\end{verbatim}

\paragraph{T.9.6. 06\_gate\_cascade.ipynb}
\begin{verbatim}
res = run_cascade("W1", tau, cfg)
save_json(res, "cache/gate_W1_{tau}.json")
\end{verbatim}

\paragraph{T.9.7. 07\_counterexample\_hunter.ipynb}
\begin{verbatim}
cases = hunt_all("W1", tau, cfg)
save_json(cases, "cache/hunter_W1_{tau}.json")
\end{verbatim}

% -------------------------
\subsection*{T.10. CI/test integration (Chapter 12)}
% -------------------------
\begin{verbatim}
# Minimal CI matrix (invoke via make/test runner)
tests:
  - name: "T_Vshift"
    script: "pytest tests/test_vshift.py"
  - name: "T_Definable"
    script: "python scripts/mece_check.py"
  - name: "T_Iwasawa"
    script: "pytest tests/test_control_overlap.py"
  - name: "T_GateCascade"
    script: "pytest tests/test_gate_cascade.py"
  - name: "T_Convergence"
    script: "pytest tests/test_convergence_manager.py"
  - name: "T_Hunter"
    script: "pytest tests/test_hunter_regressions.py"
  - name: "Demo_GL1"
    script: "python scripts/demo_gl1.py"
  - name: "Demo_GL2"
    script: "python scripts/demo_gl2.py"
artifacts:
  - "logs/ledger.json"
  - "logs/δ_total.json"
  - "logs/windows_partition.txt"
  - "cache/convergence_W1.json"
\end{verbatim}

% -------------------------
\subsection*{T.11. GL(1) / GL(2) minimal workflows (non-gating demos)}
% -------------------------
\paragraph{T.11.1. GL(1) (cyclotomic class-module surrogate).}
\begin{verbatim}
# scripts/demo_gl1.py (pseudocode)
from ak16 import pipeline
cfg = pipeline.load_manifest("run.yaml")
W, tau = "W1", cfg.tau.grid.start
# synthetic class-module levels with control bounds
ctl = pipeline.demo.gl1_control(cfg.demo.gl1)
pipeline.ledger.add("delta^Gal", W, tau, ctl.delta)
PiT = pipeline.read_persistence_after_collapse(W, tau, cfg)
print("GL(1) demo OK:", pipeline.safety_margin(PiT, tau))
\end{verbatim}

\paragraph{T.11.2. GL(2) (toy elliptic curve over \(\mathbb{Q}\)).}
\begin{verbatim}
# scripts/demo_gl2.py (pseudocode)
from ak16 import pipeline
cfg = pipeline.load_manifest("run.yaml")
W, tau = "W1", cfg.tau.grid.start
sel = pipeline.demo.gl2_selmer(cfg.demo.gl2)
pipeline.ledger.add("delta^Gal", W, tau, sel.delta)
PiT = pipeline.read_persistence_after_collapse(W, tau, cfg)
print("GL(2) demo OK:", pipeline.safety_margin(PiT, tau))
\end{verbatim}

% -------------------------
\subsection*{T.12. Invariants enforced by templates}
% -------------------------
\begin{itemize}\itemsep3pt
  \item \textbf{After-collapse only.} No comparator reads pre-collapse metrics; all distances are on \(\mathbf{P}_i(\mathbf{T}_\tau\cdot)\).
  \item \textbf{Definable finiteness.} All loops iterate over finitely many cells/events (App.~H/J/Q).
  \item \textbf{Quantale compliance.} Ledger aggregation uses the chosen \(\oplus\) with optional safe coercions (App.~S).
  \item \textbf{Gate separation.} LMHS/tropical proxies never gate alone; \(E_1\) remains first-class (Ch.~11; App.~P).
  \item \textbf{Control accounting.} Finite kernel/cokernel contributions are logged as \(\delta^{\mathrm{alg}}\) (App.~R).
  \item \textbf{Restart/Summability.} Convergence Manager applies App.~J bounds and records stability bands (App.~O/M).
\end{itemize}

% -------------------------
\subsection*{T.13. Minimal run example (end-to-end)}
% -------------------------
\begin{verbatim}
# 1) Fill run.yaml per T.2 and choose quantale in Appendix S.
# 2) Validate MECE coverage:
python scripts/mece_check.py
# 3) Execute τ-sweep (produces cache + ledger):
python scripts/tau_sweep.py
# 4) Aggregate δ-boxes and export audit:
python scripts/delta_aggregate.py
# 5) Convergence Manager:
python scripts/convergence_manager.py
# 6) Gate Cascade (windowwise):
python scripts/gate_cascade.py
# 7) (Optional) Counterexample Hunter:
python scripts/hunter_generate.py
# 8) Run CI tests (Chapter 12):
make test
\end{verbatim}

% -------------------------
\subsection*{T.14. Non-claims}
% -------------------------
Templates make no assertion about pre-collapse monotonicity, global \(\mathrm{PH}_1\Leftrightarrow\Ext^1\), or uniformity beyond the logged budgets.
All guarantees are confined to bounded definable windows with after-collapse measurements and the quantale semantics of App.~S.



% ==========================================================================
% Appendix U : AI Agent Specifications (Hunter / Mapper / Lifter)
% ==========================================================================
\section*{Appendix U: AI Agent Specifications}
\label{app:U-agents}

\subsection*{U.0. Purpose and Relation to Part II}
This appendix provides the formal specifications for the AI agents introduced
in Part~II (Chapters~\ref{sec:ch14}--\ref{sec:ch17}). All statements herein
are classified as \textbf{[Spec]}: they define the operational semantics of
the Hunter, Mapper, and Lifter, together with the mandatory logging discipline
required to treat their output as a valid Proof Object. These specifications
do not extend the proven collapse theorems of Part~I; rather, they ensure that
the computational search process respects the Universal Control Contract (UCC)
and remains fully auditable.

% --------------------------------------------------------------------------
\subsection*{U.1. Agent State and Taxonomy}

\begin{definition}[Agent Types and Roles]
We distinguish three classes of autonomous agents, each with a distinct
responsibility in the construction of the Map of Validity:
\begin{itemize}
    \item \textbf{Hunter (H):} The local optimizer. It navigates the
    parameter space \(\mathcal{M}\) within defined Terrain Cells \(W_\alpha\)
    to minimize the Defect Potential \(\Phi_\tau\).
    \item \textbf{Mapper (M):} The global assembler. It verifies local
    certificates produced by Hunters and stitches them into a connected
    Coverage Graph \(\mathcal{G}\) via Overlap Gates.
    \item \textbf{Lifter (L):} The singularity handler. It intervenes when
    a Hunter encounters a Type~IV obstruction (\(\Phi_\tau \ge \lambda_{\mathrm{sing}}\)
    and/or \((\mu,\nu)\neq(0,0)\)), and attempts to resolve it via controlled
    dimensional extension subject to the Lifting Penalty.
\end{itemize}
\end{definition}

\begin{definition}[Hunter State Tuple]
A Hunter operating at step \(k\) maintains a state tuple
\[
  S_k^{\mathrm{H}} \ :=\ (x_k,\ W_k,\ \Phi_\tau(x_k),\ \nabla\Phi_\tau(x_k)),
\]
where:
\begin{itemize}
    \item \(x_k \in \mathcal{M}\) is the current parameter point;
    \item \(W_k\) is the Terrain Cell containing \(x_k\);
    \item \(\Phi_\tau(x_k)\) is the scalar Defect Potential evaluated after collapse;
    \item \(\nabla\Phi_\tau(x_k)\) is a discrete or adjoint gradient (when computable).
\end{itemize}
\end{definition}

\begin{definition}[Mapper State Tuple]
The Mapper maintains the global knowledge state
\[
  S^{\mathrm{M}} \ :=\ (\mathcal{T}_{\mathrm{valid}},\ \mathcal{G}),
\]
where:
\begin{itemize}
    \item \(\mathcal{T}_{\mathrm{valid}}\) is the set of Terrain Cells certified by B-Gate\(^{+}\);
    \item \(\mathcal{G} = (\mathcal{V}, \mathcal{E})\) is the Coverage Graph, where vertices
    \(\mathcal{V} = \mathcal{T}_{\mathrm{valid}}\) and edges \(\mathcal{E}\) represent
    passed Overlap Gates.
\end{itemize}
\end{definition}

\begin{definition}[Lifter State Tuple]
A Lifter is instantiated at a specific singularity with state
\[
  S^{\mathrm{L}} \ :=\ (x_{\mathrm{sing}},\ W_{\mathrm{sing}},\ \Phi_\tau(x_{\mathrm{sing}}),\ (\mu,\nu),\ k),
\]
where:
\begin{itemize}
    \item \(x_{\mathrm{sing}}\) is the singular parameter point where the Hunter escalated;
    \item \(W_{\mathrm{sing}}\) is the enclosing Terrain Cell;
    \item \(\Phi_\tau(x_{\mathrm{sing}})\) is the Defect Potential at the singularity;
    \item \((\mu,\nu)\) are the tower diagnostics characterizing the obstruction;
    \item \(k\) is the current lifting depth (number of auxiliary axes added).
\end{itemize}
\end{definition}

% --------------------------------------------------------------------------
\subsection*{U.2. Hunter Protocol (Operational Semantics)}

\begin{specification}[Hunter Action Semantics]
At any state \(S_k^{\mathrm{H}}\), the Hunter must select exactly one of the
following actions based on the local regime of \(\Phi_\tau\) (cf.\ Chapter~\ref{sec:ch13}):

\begin{itemize}
    \item \texttt{gradient\_step}: If in the \emph{Ridge} regime, apply the
    Collapse-Based Gradient Descent (CBGD) update (Chapter~\ref{sec:ch15})
    to move \(x_k \mapsto x_{k+1}\) within \(W_k\).

    \item \texttt{restart}: If descent stagnates (\(\|\nabla\Phi_\tau(x_k)\|\approx 0\))
    while \(\Phi_\tau(x_k) > \mathrm{gap}_\tau\), invoke the Restart Logic
    (Appendix~J / Chapter~\ref{sec:ch15}) to refine \(W_k\) into sub-cells \(W'_k\).

    \item \texttt{validate}: If in the \emph{Plain} regime
    (\(\Phi_\tau(x_k) \le \mathrm{gap}_\tau\)), request a formal B-Gate\(^{+}\) check
    (Definition~\ref{def:bgate-plus}). If the gate passes, the cell \(W_k\)
    is flagged as \(\texttt{valid}\) and passed to the Mapper.

    \item \texttt{escalate}: If in the \emph{Peak} regime
    (\(\Phi_\tau(x_k) \ge \lambda_{\mathrm{sing}}\)) or if Type~IV diagnostics
    \((\mu,\nu)\neq(0,0)\) are detected, halt local search and invoke the Lifter.
\end{itemize}
Every executed action appends a cryptographically linked entry to the
Hunter Action Log (Section~\ref{app:U-agents}.\ref{sec:U-log}) to ensure
reproducibility and non-repudiation.
\end{specification}

% (ラベルを付けておくと便利)
\subsection{U.3. Mapper Protocol (Coverage Graph)}
\label{sec:U-mapper}

\begin{specification}[Mapper Update Rule]
Upon receiving a newly validated Terrain Cell \(W_\alpha\) (with B-Gate\(^{+}\)
passed and \((\mu,\nu)=(0,0)\)), the Mapper executes:
\begin{enumerate}
    \item \textbf{Vertex Insertion:} Add \(W_\alpha\) to the vertex set of \(\mathcal{G}\).
    \item \textbf{Overlap Verification:} For every existing cell \(W_\beta \in \mathcal{T}_{\mathrm{valid}}\)
    such that \(W_\alpha \cap W_\beta \neq \emptyset\):
    \begin{enumerate}
        \item compute the restriction \(\mathbf{T}_\tau F|_{W_\alpha \cap W_\beta}\);
        \item run the Overlap Gate (Definition~\ref{def:overlap-gate});
        \item if successful, insert edge \((\alpha,\beta)\) into \(\mathcal{G}\).
    \end{enumerate}
    \item \textbf{Global Check:} Periodically verify whether the union of cells in
    the largest connected component of \(\mathcal{G}\) covers the target domain
    \(\mathcal{M}_{\mathrm{target}}\) (as specified in the Proof Object).
\end{enumerate}
\end{specification}

% --------------------------------------------------------------------------
\subsection*{U.4. Lifter Protocol (Dimension Management)}
\label{sec:U-lifter}

\begin{specification}[Lifter Operational Semantics]
When invoked at \(S^{\mathrm{L}}\), the Lifter attempts to resolve the
singularity by extending the parameter space:
\begin{enumerate}
    \item \textbf{Axis Selection:} Select one or more auxiliary axes \(\mathcal{A}_j\)
    from the finite catalog defined in Appendix~S (e.g., smoothing width,
    spectral truncation order, arithmetic level).
    \item \textbf{Gradient Test:} Evaluate the directional derivatives
    \(\partial_{\mathcal{A}_j}\Phi_\tau\) at \((x_{\mathrm{sing}},0)\) in the
    lifted space \(\mathcal{M} \times \mathcal{A}_j\).
    \item \textbf{Commit or Fail:}
    \begin{itemize}
        \item If there exists \(j\) such that \(\partial_{\mathcal{A}_j}\Phi_\tau < 0\)
        (defect decreases) \emph{and} the Augmented Gap Condition
        (Appendix~S) holds:
        \[
        \Sigma\delta(x_{\mathrm{sing}}) \oplus \delta^{\mathrm{lift}}(k+1)
        \ < \ \mathrm{gap}_\tau,
        \]
        then commit the lift, charge the penalty \(\delta^{\mathrm{lift}}(k+1)\),
        and spawn a new Hunter in \(\mathcal{M} \times \mathcal{A}_j\).
        \item Otherwise, mark \(x_{\mathrm{sing}}\) as a \textbf{Terminal Barrier}
        (certified counterexample candidate) and halt escalation at this point.
    \end{itemize}
\end{enumerate}
\end{specification}

% --------------------------------------------------------------------------
\subsection*{U.5. Hunter Action Log Schema (JSON/YAML)}
\label{sec:U-log}

To guarantee that AI-driven proofs are auditable, all agent actions must
conform to a strict schema.

\begin{definition}[Action Log Entry]
A single log entry \(\mathsf{Entry}_k\) is a tuple
\[
  \mathsf{Entry}_k = (t_k,\ \mathrm{seed},\ x_k,\ W_k,\ \Phi_\tau(x_k),\ A_k,\ \mathrm{aux}),
\]
where:
\begin{itemize}
    \item \(t_k\) is the timestamp or step index;
    \item \(\mathrm{seed}\) encodes the RNG state or seed identifier;
    \item \(x_k\) and \(W_k\) are the current point and Terrain Cell;
    \item \(\Phi_\tau(x_k)\) is the current Defect Potential value;
    \item \(A_k\) is the action type (e.g.\ \texttt{gradient\_step}, \texttt{restart},
    \texttt{validate}, \texttt{escalate});
    \item \(\mathrm{aux}\) stores context-specific data (gradient vector, step size,
    chosen lifting axis, restart parameters, etc.).
\end{itemize}
\end{definition}

\begin{specification}[Schema Requirements]
A concrete implementation (e.g.\ in \texttt{run.yaml} or JSON Lines) MUST encode:
\begin{itemize}
    \item \textbf{Global Metadata:} Random seed, quantale choice \(\mathsf{V}\),
    collapse threshold \(\tau\), and the gap policy;
    \item \textbf{State Snapshot:} The current coordinates \(x_k\), the
    evaluated potential \(\Phi_\tau(x_k)\), and tower diagnostics \((\mu,\nu)\);
    \item \textbf{Decision Trace:} The specific logic branch taken (e.g.\ 
    ``gradient magnitude \(> \varepsilon\), performing \texttt{gradient\_step}'');
    \item \textbf{Integrity:} Checksums (hashes) for all referenced certificates
    and persistence barcodes.
\end{itemize}
\textbf{Replayability Condition:} A third party providing the same initial
seed and executing the log against the trusted AK Core (Part~I) MUST obtain
bitwise-identical Terrain Cells, certificates, and Map of Validity.
\end{specification}



% ==========================================================================
% Appendix V : Validity Map Formalism
% ==========================================================================
% Target: AiM Standard (Topological definitions, Data structures)
% Role: Formalizes the "Map of Validity" as a stratified space and defines
%       the Global Certificate structure required for proof acceptance.
% ==========================================================================

\section*{Appendix V: Validity Map Formalism}
\label{app:V-validity}

\subsection*{V.0. Purpose and Scope}
This appendix provides the formalism for the \emph{Map of Validity} introduced in Chapter~\ref{sec:ch13}. It defines the map as a level-set stratification of the parameter space \(\mathcal{M}\) induced by the Defect Potential \(\Phi_\tau\), and specifies the \emph{Global Certificate} data structure used to persist and verify this map.

% --------------------------------------------------------------------------
\subsection*{V.1. Level-Set Topology of \texorpdfstring{\(\Phi_\tau\)}{Phi}}

Let \(\mathcal{M}\) be a definable topological space (e.g.\ in an o-minimal expansion \(\mathbb{R}_{\mathrm{an,exp}}\) or a Denef--Pas structure). The Defect Potential \(\Phi_\tau: \mathcal{M} \to \mathbb{R}_{\ge 0} \cup \{+\infty\}\) induces a natural stratification.

\begin{definition}[Operational Strata]
For a fixed collapse scale \(\tau > 0\) and safety thresholds \(\mathrm{gap}_\tau\) (certification limit) and \(\lambda_{\mathrm{sing}}\) (singularity threshold), we define the following subsets of \(\mathcal{M}\):
\begin{enumerate}
    \item \textbf{The Valid Set (\(Z_{\mathrm{Valid}}\)):}
    \[
    Z_{\mathrm{Valid}} \ := \ \{ x \in \mathcal{M} \mid \Phi_\tau(x) \le \mathrm{gap}_\tau \}.
    \]
    This is the closed region where the conjecture is rigorously certified by the UCC.

    \item \textbf{The Noise Set (\(Z_{\mathrm{Noise}}\)):}
    \[
    Z_{\mathrm{Noise}} \ := \ \{ x \in \mathcal{M} \mid \mathrm{gap}_\tau < \Phi_\tau(x) < \lambda_{\mathrm{sing}} \}.
    \]
    This open region represents solvable defects (Types I--III). Hunter agents operate here to push points into \(Z_{\mathrm{Valid}}\).

    \item \textbf{The Singular Set (\(Z_{\mathrm{Sing}}\)):}
    \[
    Z_{\mathrm{Sing}} \ := \ \{ x \in \mathcal{M} \mid \Phi_\tau(x) \ge \lambda_{\mathrm{sing}} \}.
    \]
    This closed set contains essential Type~IV obstructions. It is the target for Lifter agents.
\end{enumerate}
\end{definition}

\begin{definition}[Validity Map]
The \emph{Validity Map} at scale \(\tau\) is the tuple
\[
  \mathfrak{V}_\tau \ :=\ (\mathcal{M}, \Phi_\tau, \mathcal{T}),
\]
where \(\mathcal{T}\) is a terrain cell decomposition compatible with the strata: for each \(W \in \mathcal{T}\), the restriction \(\Phi_\tau|_W\) has constant regime (Valid / Noise / Sing) or controlled variation (e.g.\ Lipschitz with respect to the ambient metric).
\end{definition}

\begin{specification}[Global Regularity Condition (\textbf{[Spec]})]
Operationally, we \emph{accept} that the underlying conjecture holds globally on \(\mathcal{M}\) once the following conditions are met:
\begin{itemize}
    \item \textbf{No essential singularities:} \(Z_{\mathrm{Sing}} = \emptyset\).
    \item \textbf{Dominance of validity:} \(Z_{\mathrm{Valid}}\) is either
    \begin{itemize}
        \item a deformation retract of \(\mathcal{M}\), or
        \item dense in \(\mathcal{M}\) with a robustness margin controlled by the quantale metric on \(\mathcal{M}\).
    \end{itemize}
\end{itemize}
In practice, these conditions are realized via a Global Certificate (Section~\ref{app:V-global-cert}) built from Terrain Cells whose B-Gate\(^{+}\) has passed and whose overlaps satisfy the Overlap Gate (Definition~\ref{def:overlap-gate}).
\end{specification}

% --------------------------------------------------------------------------
\subsection*{V.2. Global Certificate Format}
\label{app:V-global-cert}

The Mapper agent aggregates local results into a \textbf{Global Certificate}, which serves as the final proof object for the Map of Validity.

\begin{definition}[Global Certificate Structure]
A Global Certificate is a directed acyclic graph (DAG)
\[
  \mathcal{C}_{\mathrm{global}} \ =\ \langle \mathcal{V}, \mathcal{E}, \mathcal{M}_{\mathrm{meta}} \rangle
\]
with the following components:

\begin{itemize}
    \item \textbf{Vertices (\(\mathcal{V}\)):} A collection of validated Terrain Cells
    \(\{W_\alpha\}_{\alpha \in A}\). Each vertex stores:
    \begin{itemize}
        \item a definable formula (or equivalent descriptor) specifying \(W_\alpha\);
        \item the local \(\Phi\)-bound:
        \(\sup_{x \in W_\alpha} \Phi_\tau(x)\);
        \item the B-Gate\(^{+}\) pass token (including hash of the local proof artifacts).
    \end{itemize}

    \item \textbf{Edges (\(\mathcal{E}\)):} Represent verified overlaps. An edge
    \(\alpha \to \beta\) exists when
    \begin{itemize}
        \item \(W_\alpha \cap W_\beta \neq \emptyset\), and
        \item the Overlap Gate (Definition~\ref{def:overlap-gate}) passes on
        \(W_\alpha \cap W_\beta\).
    \end{itemize}

    \item \textbf{Metadata (\(\mathcal{M}_{\mathrm{meta}}\)):} Includes:
    \begin{itemize}
        \item \textbf{Covering Proof:} A formal verification that
        \(\bigcup_{\alpha} W_\alpha \supseteq \mathcal{M}_{\mathrm{target}}\).
        \item \textbf{Summability Check:} Verification that
        \(\bigoplus_{\alpha} \Sigma\delta(W_\alpha)\) converges in the quantale
        \(\mathsf{V}\) (cf.\ Appendix~J).
        \item \textbf{Type IV Clearance:} A certificate that
        \(Z_{\mathrm{Sing}} \cap \big(\bigcup_{\alpha} W_\alpha\big) = \emptyset\),
        i.e.\ no validated cell intersects the singular regime.
    \end{itemize}
\end{itemize}
\end{definition}

\begin{specification}[Verification Protocol]
To verify a Global Certificate \(\mathcal{C}_{\mathrm{global}}\), a trusted kernel performs:
\begin{enumerate}
    \item \textbf{Local Check:} Iterate through \(\mathcal{V}\) and, for each
    \(W_\alpha\), re-verify the B-Gate\(^{+}\) token (including recomputation
    of local collapse, Ext/PH checks, and budget inequalities) against the
    stored hashes.

    \item \textbf{Connectivity and Covering:} Using \(\mathcal{E}\) and the
    associated Overlap Gate results, compute the largest connected component
    and confirm that its union of cells covers \(\mathcal{M}_{\mathrm{target}}\)
    as specified in the Covering Proof.

    \item \textbf{Budget Check:} Re-aggregate the total \(\delta\)-budget
    \(\bigoplus_{\alpha} \Sigma\delta(W_\alpha)\) and confirm that it lies
    within the global safety margin mandated by the UCC (gap policy).
\end{enumerate}
Under these checks, verification of a high-dimensional conjecture reduces to
a finite graph traversal plus a controlled summation in \(\mathsf{V}\).
\end{specification}

% --------------------------------------------------------------------------
\subsection*{V.3. Summary}
Appendix~V provides the topological and data-structural semantics for the
\emph{output} of the AK-HDPST framework. By:
\begin{itemize}
    \item defining the Validity Map as a stratification
    \(\mathcal{M} = Z_{\mathrm{Valid}} \sqcup Z_{\mathrm{Noise}} \sqcup Z_{\mathrm{Sing}}\), and
    \item encoding the Map as a verifiable Global Certificate DAG,
\end{itemize}
we ensure that AI-driven exploration yields a mathematically auditable proof
object, rather than a black-box prediction.



% ==========================================================================
% Appendix W : Bridge Programs — Detailed Specifications
% ==========================================================================
% Target: AiM Standard (Quantitative inequalities, Algorithmic logic)
% Role: Provides the rigorous "runtime checks" required to execute the
%       reverse implication Ext -> PH safely, preventing numerical hallucinations.
% ==========================================================================

\section*{Appendix W: Bridge Programs — Detailed Specifications}
\label{app:W-bridge}

\subsection*{W.0. Purpose and Scope}
This appendix refines the high-level Bridge Programs of Chapter~\ref{sec:ch16}
into executable specifications. It defines the quantitative thresholds for the
\textbf{Spectral-Gap Condition} and details the algorithmic state machines for
Global Regularity (B2) and Counterexample Hunting (B3). These specifications
serve as the acceptance criteria for the AK Core when validating AI-proposed
proofs involving reverse implications. All statements in this appendix are
classified as \textbf{[Spec]}.

% --------------------------------------------------------------------------
\subsection*{W.1. Formalizing the Spectral-Gap Condition}

To distinguish a true topological zero (homological vanishing) from a numerical
near-zero (noise), we require a unified spectral separation bound on each
Terrain Cell.

\begin{definition}[Window Spectral Profile]
For a Terrain Cell \(W\) and collapse scale \(\tau\), let \(L_\tau(x)\) be the
normalized combinatorial Laplacian at \(x \in W\). We define two window-wide
statistics:
\begin{enumerate}
    \item \textbf{Signal Floor (\(\gamma_{\tau,\min}\)):} The infimum of the
    first non-zero eigenvalue across the window:
    \[
    \gamma_{\tau,\min}(W) \ := \ \inf_{x \in W} \min \{ \lambda \in \sigma(L_\tau(x)) \mid \lambda > 0 \}.
    \]
    (If the spectrum is everywhere zero, set \(\gamma_{\tau,\min}(W) = +\infty\).)
    
    \item \textbf{Noise Ceiling (\(\delta_{\max}\)):} The supremum of the
    accumulated scalar error budget:
    \[
    \delta_{\max}(W) \ := \ \sup_{x \in W} \big\lVert \Sigma\delta(x) \big\rVert_V,
    \]
    where \(\lVert\cdot\rVert_V\) is the scalarization of the quantale
    \(\mathsf{V}\) as in Chapter~\ref{sec:ch13}.
\end{enumerate}
\end{definition}

\begin{specification}[Spectral-Gap Condition \(\mathrm{SGC}(c)\)]
A Terrain Cell \(W\) satisfies the Spectral-Gap Condition with safety factor
\(c > 1\) if:
\[
\gamma_{\tau,\min}(W) \ > \ c \cdot \delta_{\max}(W).
\]
\textbf{Default Policy:} The AK Core enforces \(c \ge 2.0\) unless a tighter
bound is proven for the specific realization.

This condition guarantees that the ``gap'' between zero and the first excited
state is wider than the worst-case numerical or commutation noise allowed by
the \(\delta\)-ledger.
\end{specification}

% --------------------------------------------------------------------------
\subsection*{W.2. Program B1: Local Reverse Logic}

Program B1 is the subroutine for certifying \(\mathrm{PH}_1 = 0\) based on an
observed vanishing of \(\mathrm{Ext}^1\) within a Terrain Cell.

\begin{specification}[Logic Flow for Program B1]
\textbf{Context:} Terrain Cell \(W\), scale \(\tau\).
\begin{enumerate}
    \item \textbf{Precondition Check:}
    \begin{itemize}
        \item Is \(W\) definable? (Yes/No)
        \item Does \(\Phi_\tau(x) < \mathrm{gap}_\tau\) hold for all \(x \in W\)? (Yes/No)
        \item Are tower diagnostics \((\mu_{\mathrm{Collapse}},\nu_{\mathrm{Collapse}}) = (0,0)\)? (Yes/No)
    \end{itemize}

    \item \textbf{Spectral Audit:}
    \begin{itemize}
        \item Compute (or bound) \(\gamma_{\tau,\min}(W)\) and \(\delta_{\max}(W)\).
        \item Verify \(\mathrm{SGC}(c)\). If failed, abort with
        \texttt{Status: Indeterminate}.
    \end{itemize}

    \item \textbf{Numerical Verification:}
    \begin{itemize}
        \item Compute
        \[
          \epsilon_{\mathrm{obs}} \ :=\
          \sup_{x \in W} \big\lVert \mathrm{Ext}^1(\mathcal{R}(C_\tau F_x)) \big\rVert.
        \]
        \item Verify \(\epsilon_{\mathrm{obs}} \le \delta_{\max}(W)\).
    \end{itemize}

    \item \textbf{Certification:}
    \begin{itemize}
        \item If all checks pass, issue a \textbf{Reverse Certificate}:
        ``On \(W\), the implication \(\mathrm{Ext}^1 = 0 \Rightarrow \mathrm{PH}_1 = 0\)
        is valid under \(\mathrm{SGC}(c)\) and the UCC guard-rails.''
    \end{itemize}
\end{enumerate}
\end{specification}

% --------------------------------------------------------------------------
\subsection*{W.3. Logic Flow of Programs B2/B3}

These programs orchestrate B1 certifications into global results.

\begin{specification}[Program B2: Global Regularity (Proof Mode)]
\textbf{Goal:} Cover \(\mathcal{M}\) with valid cells and derive a Global Certificate.

\textbf{Algorithm:}
\begin{enumerate}
    \item \textbf{Initialize:}
    \(\mathcal{Q} \gets \{\mathcal{M}\}\) (queue of regions),
    \(\mathcal{G} \gets \emptyset\) (Coverage Graph).

    \item \textbf{Loop} while \(\mathcal{Q}\) is not empty:
    \begin{itemize}
        \item Pop region \(R\) from \(\mathcal{Q}\). A Hunter proposes a Terrain
        Cell \(W \subseteq R\).
        \item Run \textbf{Program B1} on \(W\).
        \item \textbf{If Valid:}
        \begin{itemize}
            \item Add \(W\) (with its B-Gate\(^{+}\) token) to \(\mathcal{G}\).
            \item Invoke the Mapper to verify overlaps via the Overlap Gate
            (Definition~\ref{def:overlap-gate}).
        \end{itemize}
        \item \textbf{If Singular (Type IV):}
        \begin{itemize}
            \item Invoke the Lifter (Chapter~\ref{sec:ch14}).
            \item If the Lift succeeds, add the resulting lifted cells to \(\mathcal{Q}\).
            \item If the Lift fails (Lifting Penalty exceeds gap or no decreasing
            auxiliary axis), \textbf{abort B2} and switch to B3.
        \end{itemize}
    \end{itemize}

    \item \textbf{Terminate:} If \(\mathcal{Q}\) is empty and the Coverage Graph
    \(\mathcal{G}\) has a connected component whose union covers
    \(\mathcal{M}_{\mathrm{target}}\), return
    \textbf{Global Regularity Proven} and export a Global Certificate
    (Appendix~\ref{app:V-validity}).
\end{enumerate}
\end{specification}

\begin{specification}[Program B3: Counterexample Hunt (Disproof Mode)]
\textbf{Goal:} Isolate an irresolvable singularity as a counterexample candidate.

\textbf{Algorithm:}
\begin{enumerate}
    \item \textbf{Search:} A Hunter seeks \(x \in \mathcal{M}\) maximizing
    \(\Phi_\tau(x)\), subject to exploration constraints, and identifies a peak
    \(x^*\).

    \item \textbf{Verify Type IV:} Confirm that
    \((\mu_{\mathrm{Collapse}}(x^*), \nu_{\mathrm{Collapse}}(x^*)) \neq (0,0)\),
    i.e.\ the obstruction has Type~IV signature.

    \item \textbf{Exhaust Lifts:}
    \begin{itemize}
        \item Enumerate all admissible auxiliary axes \(\mathcal{A}_j\) as in
        Appendix~S.
        \item For each \(\mathcal{A}_j\), compute the directional derivative
        \(\partial_{\mathcal{A}_j}\Phi_\tau(x^*,0)\) in the lifted space, and
        estimate the Lifting Penalty \(\delta^{\mathrm{lift}}(k+1)\).
        \item If for all \(\mathcal{A}_j\) either
        \(\partial_{\mathcal{A}_j}\Phi_\tau(x^*,0) \ge 0\) (no decrease into
        the new axis) or
        \(\delta^{\mathrm{lift}}(k+1) > \mathrm{gap}_\tau\),
        then the singularity is deemed \textbf{Essential}.
    \end{itemize}

    \item \textbf{Output:} Return \(x^*\) (together with its local data and
    failed Lifter attempts) as a \textbf{Certified Counterexample Candidate}.
\end{enumerate}
\end{specification}

\subsection*{W.4. Summary}
Appendix~W transforms the theoretical bridge into a concrete decision
procedure. By enforcing the \textbf{Spectral-Gap Condition}, we ensure that
the ``Reverse Bridge'' is never crossed blindly. Programs B2 and B3 provide
the high-level logic for the AI to autonomously construct proofs or disproofs,
while remaining bounded at all times by the UCC budget and the \(\delta\)-ledger.



% ==========================================================================
% Appendix NS : Navier--Stokes Case Study (Independent)
% ==========================================================================
% Target: AiM Standard (Application Template)
% Role: Demonstrates how the abstract machinery of AK-HDPST v17.0 maps
%       onto a concrete Millennium Prize Problem.
%       Defines the specific "Dissipation ECF" and "Type IV" interpretation
%       for 3D Incompressible NSE.
% ==========================================================================

\section*{Appendix NS: Navier--Stokes Case Study}
\label{app:NS}

\subsection*{NS.0. Status and Scope}
This appendix serves as a \textbf{[Spec]-level case study} demonstrating the
application of the AK-HDPST framework to the 3D incompressible Navier--Stokes
Equations (NSE). It translates the classical analytical problem of
``finite-time blow-up vs.\ global regularity'' into the topological language
of ``Type~IV singularity vs.\ global collapse.''

All statements in this appendix are \textbf{application-specific} and do not
modify the Core Theory of Part~I. In particular, nothing in this appendix
should be read as a claimed proof of NSE regularity or blow-up; it is a
template for how an AK-HDPST run could be configured.

% --------------------------------------------------------------------------
\subsection*{NS.1. AK Realization of NSE}

To apply the framework, we first map the PDE data into the category of filtered
chain complexes used in AK-HDPST.

\begin{definition}[NSE Parameter Space \(\mathcal{M}_{\mathrm{NSE}}\)]
The parameter space \(\mathcal{M}_{\mathrm{NSE}}\) is defined by tuples
\[
x \ = \ (u_0, \nu, f, T),
\]
where:
\begin{itemize}
    \item \(u_0 \in H^s(\mathbb{T}^3)\) (or \(\mathbb{R}^3\)) is the initial
    divergence-free velocity field, with \(s \ge \tfrac{1}{2}\).
    \item \(\nu > 0\) is the kinematic viscosity.
    \item \(f : [0,T] \to H^{s-1}\) is the external force (assumed smooth).
    \item \(T > 0\) is the target time horizon for the run.
\end{itemize}
\end{definition}

\begin{specification}[Enstrophy Level-Set Filtration]
Given a (weak or strong) solution \(u(t)\) associated to \(x \in
\mathcal{M}_{\mathrm{NSE}}\), let \(\omega(\cdot,t) = \nabla \times u(\cdot,t)\)
denote the vorticity. For each time \(t\) and level \(\lambda \ge 0\), define
the sublevel set
\[
X^\lambda_t \ := \ \big\{ \mathbf{x} \in \mathbb{T}^3
\ \big|\ |\omega(\mathbf{x}, t)|^2 \le \lambda \big\}.
\]
We then define:
\begin{itemize}
    \item The filtration \(\{X^\lambda_t\}_{\lambda \ge 0}\) as the
    \emph{enstrophy level-set filtration} at time \(t\).
    \item The filtered complex \(F_u(t)\) as the singular chain complex of this
    filtration (or a cubical/simplicial discretization thereof).
\end{itemize}
This construction yields, for each \(x \in \mathcal{M}_{\mathrm{NSE}}\), a
time-indexed family of filtered complexes \(F_u(t)\), which can be fed into the
AK pipeline (collapse at scale \(\tau\), diagnostics, and \(\delta\)-ledger).
\emph{Intuition:} If \(u\) remains smooth, the topology of \(X^\lambda_t\)
evolves in a controlled way. If a singularity forms (blow-up), vorticity
concentrates and creates highly persistent small-scale features in the
filtration that resist collapse.
\end{specification}

% --------------------------------------------------------------------------
\subsection*{NS.2. The Dissipation ECF and Collapse Diagnostics}

We connect the physical notion of energy dissipation to the topological notion
of collapse energy and obstructions.

\begin{definition}[Dissipation ECF]
For a time window \(W = [t_0, t_1) \subset [0,T]\), the
\emph{Dissipation Explicit-Contract Formula (ECF)} is defined as the integrated
enstrophy:
\[
\mathcal{E}_{\mathrm{diss}}(W) \ := \ \nu \int_{t_0}^{t_1}
\big\lVert \omega(\cdot, t) \big\rVert_{L^2}^2 \, dt.
\]
\end{definition}

\begin{conjecture}[Energy--Collapse Inequality \textnormal{[Spec]}]
\label{conj:energy-collapse}
There exists a universal constant \(C > 0\) such that for any time window
\(W = [t_0,t_1)\), the collapse obstruction indices satisfy
\[
\mu_{\mathrm{Collapse}}(W) \ + \ \nu_{\mathrm{Collapse}}(W)
\ \le \ C \cdot \mathcal{E}_{\mathrm{diss}}(W).
\]
\textbf{Heuristic implication:} For any Leray--Hopf weak solution, the total
dissipation \(\int_0^\infty \lVert \omega(\cdot,t)\rVert_{L^2}^2 \, dt\) is
finite. If the conjecture holds, then Type~IV singularities (non-trivial
\((\mu_{\mathrm{Collapse}},\nu_{\mathrm{Collapse}})\)) cannot accumulate
unboundedly in time; they would have to be confined to a set of measure zero in
the time axis. Translating this into a rigorous regularity statement would
require additional analysis beyond this appendix.
\end{conjecture}

% --------------------------------------------------------------------------
\subsection*{NS.3. Blow-up vs.\ Type IV Correspondence}

We formalize the conceptual correspondence between the classical NSE problem
and the AK landscape.

\begin{definition}[NSE--AK Translation Table]
We summarize the intended correspondences as follows:
\[
\begin{array}{l l}
\toprule
\text{\bf Classical NSE Phenomenon} &
\text{\bf AK-HDPST Representation} \\
\midrule
\text{Global regularity on } [0,T] &
\Phi_\tau(x) < \mathrm{gap}_\tau \text{ on all relevant windows} \\
\text{Finite-time blow-up at } T^* < T &
\text{Type~IV singularity (}\Phi_\tau(x) \to \infty\text{ near } T^*) \\
\text{Leray--Hopf weak solution} &
\text{Collapsed object in } \mathsf{Pers}^{\mathrm{cons}}_k / \mathsf{E}_\tau \\
\text{Energy equality} &
\delta\text{-ledger balance }(\Sigma\delta \approx 0) \\
\text{Viscous regularization} &
\text{Deletion-type updates (monotone under }\mathbf{T}_\tau\text{)} \\
\bottomrule
\end{array}
\]
Here \(x = (u_0,\nu,f,T) \in \mathcal{M}_{\mathrm{NSE}}\), and the Defect
Potential \(\Phi_\tau\) is computed from the NSE-specific realization \(F_u\)
via the standard AK collapse pipeline.
\end{definition}

\begin{specification}[Hunter Strategy for NSE \textnormal{[Spec]}]
\textbf{Objective:} For a fixed time horizon \(T\) and scale \(\tau\), either
(1) build a covering of \(\mathcal{M}_{\mathrm{NSE}}\) by valid cells
(Global Regularity mode, Program B2), or (2) isolate a robust Type~IV
singularity (Counterexample mode, Program B3).

A typical Hunter strategy may proceed as follows:
\begin{enumerate}
    \item \textbf{Search over data:} Explore \(\mathcal{M}_{\mathrm{NSE}}\),
    sampling initial data \(u_0\) and parameters \((\nu,f,T)\) to maximize
    the local Defect Potential \(\Phi_\tau(x)\) (Chapter~\ref{sec:ch13}).

    \item \textbf{Focus on high-R\!e regimes:} Bias the search toward regions
    with small viscosity \(\nu\) and complex initial topology (e.g.\ vortex
    tubes with knotted or linked structure), where blow-up is heuristically
    more likely.

    \item \textbf{Monitor Type~IV signatures:} On each time window
    \(W \subset [0,T]\), monitor the tower diagnostics
    \((\mu_{\mathrm{Collapse}}(W),\nu_{\mathrm{Collapse}}(W))\). If these
    remain zero and \(\Phi_\tau < \mathrm{gap}_\tau\), the window is marked
    as regular.

    \item \textbf{Invoke Lifter when needed:} If a Peak
    (\(\Phi_\tau(x) \ge \lambda_{\mathrm{sing}}\)) appears or
    \((\mu_{\mathrm{Collapse}},\nu_{\mathrm{Collapse}}) \neq (0,0)\) on a
    window, invoke the Lifter with auxiliary axes such as:
    \begin{itemize}
        \item a hyperviscosity parameter \(\epsilon(-\Delta)^2\),
        \item spectral truncation levels,
        \item refined spatial resolutions.
    \end{itemize}

    \item \textbf{Disproof candidate criterion:} If, as the auxiliary axes are
    taken back to the physical regime (e.g.\ \(\epsilon \to 0\)), the Defect
    Potential does not decrease and the Lifting Penalty exceeds the global
    gap, Program~B3 (Appendix~\ref{app:W-bridge}) flags the configuration
    as a \textbf{counterexample candidate} for finite-time blow-up.
\end{enumerate}
\end{specification}

\subsection*{NS.4. Summary}
This appendix provides a template for applying AK-HDPST v17.0 to the
Navier--Stokes problem. By translating ``smoothness'' into ``collapsibility''
and ``blow-up'' into ``Type~IV singularity,'' it recasts the analytical
difficulty of detecting singularities as a topological search problem that can
be explored by AI agents (Hunter/Lifter) and certified by the AK Core under
the UCC guard-rails.



% -------------------------
\section*{Concluding Remarks and Acknowledgments (v17.0, AK--HDPST + HDPS)}
% -------------------------
\phantomsection
\addcontentsline{toc}{section}{Concluding Remarks and Acknowledgments}

\paragraph{Standing scope, coefficients, windows, and UCC guard–rails.}
Unless stated otherwise, coefficients lie in a \emph{field} \(k\).
All \emph{Core} statements live in constructible one\hyp parameter persistence
over a field and, when realized, target \(D^{\mathrm{b}}(k\text{-mod})\).
In \textbf{[Spec]} appendices that use sheaf\hyp theoretic, arithmetic, or
Fukaya\hyp categorical realizations we write the coefficient field as
\(\Lambda\); this is only a notational change.
Filtered (co)limits are computed \emph{objectwise} in
\([\mathbb{R},\mathsf{Vect}_\Lambda]\) under the scope policy
(Appendix~A) and returned to the constructible subcategory by verification or
by applying \(\mathbf{T}_\tau\).

All statements are made under the \emph{Unified Collapse Contract (UCC)}
(Thm.~\ref{thm:UCC}, Chapter~1): we work in constructible \(1\)D persistence,
with
\begin{itemize}
  \item a commutative Quantale \((\mathsf{V},\oplus,\le,0)\) enriching the time
  index and hosting all error budgets;
  \item \emph{right\hyp open}, MECE windows along a \(\tau\)\hyp sweep, which,
  when required, are \emph{Denef--Pas definable} to guarantee finite event sets
  and finite Čech depth (Appendix~Q);
  \item an \emph{after\hyp collapse policy}: all equalities, exactness claims,
  monotonicity statements, comparisons, and gluing are asserted only after
  applying \(\mathbf{T}_\tau\) at the persistence layer.
\end{itemize}
Within these guard\hyp rails, the \(\delta\)\hyp ledger aggregates all residuals
in \(\mathsf{V}\), is subadditive under composition, and is non\hyp increasing
under after\hyp collapse \(1\)\hyp Lipschitz post\hyp processing; this allows
\(\Sigma\delta\) to be used both as an audit budget and, in \textbf{[Spec]}
mode, as a scalar \emph{Defect Potential} \(\Phi\) for high\hyp dimensional
search (Ch.~13).

\paragraph{What is proved (Core; F0--F6 \& P1--P10, UCC/bridge extensions).}
Within the above regime we establish machine\hyp checkable results and arrange
them for formalization.
\begin{itemize}
  \item \emph{Exact truncation and filtered lift (F1--F2).}
  The Serre reflector \(\mathbf{T}_\tau\) deletes precisely bars of length
  \(\le\tau\), is exact, idempotent, and \(1\)\hyp Lipschitz (indeed
  \(\mathsf{V}\)\hyp \(1\)\hyp Lipschitz under Quantale enrichment).
  It admits a filtered lift \(C_\tau\) unique up to f.q.i.\ with
  \(\mathbf{P}_i(C_\tau F)\cong \mathbf{T}_\tau(\mathbf{P}_iF)\)
  (Appendices~A/B).
  \item \emph{CNF and field edge (P1--P2).}
  After collapse, objects split in \(D^{\mathrm b}(k\text{-mod})\):
  \(X\simeq\bigoplus_i H^i(X)[-i]\), and
  \(\Ext^1(X,k)\cong\Hom(H^1(X),k)\).
  \item \emph{Bridge and local reverse under safety (P3).}
  Under a \(t\)\hyp exact realization of amplitude \(\le 1\),
  \(\mathrm{PH}_1(F)=0\Rightarrow \Ext^1(\mathcal{R}(F),k)=0\)
  (Appendix~C).
  On definable right\hyp open windows satisfying the UCC hypotheses and the
  Spectral\hyp Gap Condition, the \emph{local reverse bridge} holds after
  collapse (Chapter~16; Thm.~\ref{thm:local-reverse}/\ref{thm:spectral-robustness}):
  numerically vanishing \(\Ext^1\) within the spectral gap implies
  \(\mathrm{PH}_1(C_\tau F|_W)=0\).
  \item \emph{Safe low\hyp pass (P4).}
  Even, mass\hyp \(1\), bandwidth \(\asymp\sqrt{\tau}\) kernels commute with
  \(\mathbf{T}_\tau\) up to a controlled \(\delta^{\mathrm{alg}}\) and keep
  \(\mathbf{T}_\tau\circ L_\tau\) \(1\)\hyp Lipschitz
  (Prop.~\ref{prop:lowpass}; Ch.~2/Appendix~E).
  \item \emph{Monotonicity vs.\ stability (P5).}
  Deletion\hyp type updates are non\hyp increasing after collapse; inclusion\hyp
  type updates are non\hyp expansive.
  \item \emph{Convergence Manager (P6).}
  For countable \emph{Denef--Pas} covers of finite Čech depth, the
  Quantale\hyp summed error satisfies \(\sum\delta<\infty\) and overlap gluing
  holds globally (Thm.~\ref{thm:dp-sum}; Appendices~J/Q).
  \item \emph{AWFS \(2\)\hyp cell additivity (P7).}
  \(2\)\hyp cell defects add subadditively in the Quantale (Ch.~5;
  Appendices~K/L).
  \item \emph{Gate calculus with cut elimination (P8).}
  The default cascade
  \[
  E_1{=}0\ \Longrightarrow\ (\mu,\nu){=}(0,0)\ \Longrightarrow\
  \Ext^1{=}0\ \Longrightarrow\ \mathrm{PH}_1{=}0
  \]
  is operated as a sequent calculus; success later never overturns failure
  earlier.
  \item \emph{Stability bands (P9).}
  Open \(\tau\)\hyp intervals with \((\mu,\nu){=}(0,0)\) certify stability;
  Type\hyp IV is excluded in conjunction with P6 (Ch.~4; Appendix~D/J).
  \item \emph{Reproducibility theorem (P10).}
  From pass\hyp logs of
  \texttt{T-ExtZero-implies-PHZero},
  \texttt{T-Countable-Cover},
  \texttt{T-Delta-Sum-Converges},
  \texttt{T-Lipschitz-AfterCollapse},
  \texttt{T-Exactness-Persistence}
  (and arithmetic \texttt{T-Iwasawa-Alignment}),
  the P3/P6/P8 conclusions are mechanically reconstructed.
  \item \emph{UCC collapse nucleus and Quantale ledger (extension).}
  The Unified Collapse Contract (Thm.~\ref{thm:UCC}) upgrades
  \(\mathbf{T}_\tau\) to a \(\mathsf{V}\)\hyp nucleus and shows that the
  Quantale\hyp valued \(\delta\)\hyp ledger is subadditive, non\hyp increasing
  under after\hyp collapse post\hyp processing, and therefore sound as a
  global potential \(\Sigma\delta\) for both audit and (under \textbf{[Spec]}
  policies) navigation (Ch.~1/Appendix~S).
\end{itemize}

\paragraph{What is specified and how it is audited (\textup{[Spec]}).}
All \textbf{[Spec]} items are explicitly contracted to be \emph{non\hyp
expansive after truncation} and are audited by the windowed diagnostics
\((\mu,\nu)\) together with a Quantale\hyp valued \(\delta\)\hyp ledger (F6).
\begin{itemize}
  \item \emph{UCC search layer and dual mode.}
  The Quantale enrichment \((\mathsf{V},\oplus,\le,0)\), definable windows
  (Denef--Pas preferred), and the AWFS view
  \(\mathrm{Id}\Rightarrow L\dashv R\Rightarrow \mathrm{Id}\) with
  \(R=C_\tau\) (Ch.~1/5; Appendices~K/L) provide the \emph{Audit Mode} in which
  \(\Sigma\delta<\mathrm{gap}\) certifies validity.
  In \emph{Navigation Mode} (Part~II), the same \(\Sigma\delta\) is scalarized
  into a Defect Potential \(\Phi_\tau\) (with Type\hyp IV penalties) used by
  AI agents under Chapter~13.
  \item \emph{Mirror/Transfer pipelines.}
  A natural \(2\)\hyp cell
  \(\Mirror\circ C_\tau\Rightarrow C_\tau\circ\Mirror\) with uniform bounds
  \(\delta(i,\tau)\) yields \(\delta\)\hyp controlled commutation; bounds add
  along pipelines and are non\hyp increasing under \(1\)\hyp Lipschitz
  post\hyp processing (Appendix~L).
  \item \emph{Multi\hyp axis reflectors.}
  For exact reflectors from hereditary Serre subcategories, nesting is
  order\hyp independent; otherwise an A/B test with tolerance \(\eta\) and
  deterministic fallback is used.
  Residuals \(\Delta_{\mathrm{comm}}\) are recorded as \(\delta^{\mathrm{alg}}\)
  (Appendix~K).
  \item \emph{Arithmetic layers (SCTF/ECF/Iwasawa).}
  Local traces (Igusa/Tate) couple to post\hyp collapse measurements;
  discrepancies externalize to \(\delta_{\mathrm{alg}},\delta_{\mathrm{meas}}\).
  The \emph{Explicit\hyp Contract Formula} (ECF) enforces the safety\hyp side
  inequality
  \[
  \bigl|\mu_{\mathrm{Coll}}(W,\tau)
  -\langle \mathrm{Obs}(R_{\mathrm{spec}}(F),W),\varphi_\tau\rangle\bigr|
  \ \le\ \varepsilon_{\mathrm{tot}}(W,\tau),
  \]
  with RHS fully represented in the \(\delta\)\hyp ledger (Appendix~M).
  The \emph{Iwasawa Gate} aligns
  \((\mu_{\mathrm{Collapse}},\mu_{\mathrm{Iwasawa}})\) in a three\hyp state
  regime (lower bound / match / drift\hyp corrected) to suppress Type\hyp IV
  drift (Ch.~7; Appendix~R).
  \item \emph{Fukaya realizations.}
  Action filtration yields constructible persistence on bounded windows;
  continuation is \(1\)\hyp Lipschitz and adding stops is deletion\hyp type
  (Appendix~O).
  \item \emph{PF/BC transport.}
  Projection formula/base change are transported only through the
  after\hyp collapse protocol; residual discretization/measurement slack is
  budgeted as \(\delta^{\mathrm{disc}},\delta^{\mathrm{meas}}\)
  (Appendix~N).
  \item \emph{HDPS engine, Validity Map, and AI agents.}
  Part~II and Appendices~U/V/W specify the \emph{High\hyp Dimensional
  Projection Search (HDPS)} layer:
  \begin{itemize}
      \item the Defect Potential \(\Phi_\tau\) and its stratification into
      Plain/Ridge/Peak regimes (Ch.~13; Appendix~V);
      \item the autonomous agents Hunter/Mapper/Lifter with formally defined
      operational semantics and the Hunter Action Log schema
      (Ch.~14--15; Appendix~U);
      \item the \emph{Validity Map} and Global Certificate as verifiable graph
      structures (Ch.~13/16; Appendix~V);
      \item the Bridge Programs B1/B2/B3 and the Spectral\hyp Gap Condition,
      which govern when reverse implications \(\Ext^1\Rightarrow\mathrm{PH}_1\)
      are allowed in \textbf{[Spec]} mode (Ch.~16; Appendix~W).
  \end{itemize}
  All such search\hyp side components are constrained by the UCC budget:
  they may propose paths and lifts, but the Core vetoes any step with
  \(\Sigma\delta\ge\mathrm{gap}\).
  \item \emph{Case studies.}
  Application templates---e.g.\ the Navier--Stokes case study
  (Appendix~NS)---translate classical questions (finite\hyp time blow\hyp up
  vs.\ global regularity, BSD rank problems, etc.) into the AK\hyp HDPST
  language of collapse, Type~IV singularity, and Validity Maps.
  These remain \textbf{[Spec]} and do not alter the Core theorems.
\end{itemize}

\paragraph{Operational pipeline (end\hyp to\hyp end).}
Per window and degree: enforce \(\mathrm{B\text{--}Gate}^{+}\) with safety
margin \(\mathrm{gap}_\tau>\sum\delta\); run the \emph{Overlap Gate}
(post\hyp collapse equivalence up to budget, Čech–\(\Ext^1\) acyclicity in
degree \(1\), stability bands); and, across windows, apply
Restart/Summability (\(\kappa\)\hyp restart and \(\sum\delta<\infty\)) to paste
local certificates into global ones (Appendix~J).
A single Quantale\hyp sum \(\oplus\) aggregates pipeline budgets:
Mirror–Collapse bounds \(\delta(i,\tau)\), A/B residuals
\(\Delta_{\mathrm{comm}}\), discretization and measurement terms, and, in HDPS
mode, lifting penalties \(\delta^{\mathrm{lift}}\).

\paragraph{Reproducibility, formalization, and tests (\texttt{run.yaml} v17.0).}
Appendix~G specifies the manifest \texttt{run.yaml} (Quantale, definable window
formulae, AWFS/\(2\)\hyp cell bounds, \(\tau\)\hyp sweeps, spectral and
lifting policy, search strategy) and artifact schemas with canonical
serialization and cross\hyp linked hashes
(\texttt{bars/spec/ext/phi/Lambda\_len}).
Appendix~F outlines a Lean/Coq pathway for Core components (Serre
localization; \(1\)\hyp Lipschitz; comparison maps and \((\mu,\nu)\); CNF and
field edge; the local reverse; API stubs for budgets and soft\hyp commuting).
Appendix~U prescribes the Hunter Action Log and Proposer/Verifier split;
Appendix~V formalizes the Global Certificate; Appendix~W the Bridge Programs.
Chapter~12 provides tests for \(\mathsf{V}\)\hyp Lipschitz laws, definable
coverage/finite Čech depth, deletion\hyp type monotonicity, filtered\hyp
colimit behavior, Mirror/tropical pipelines, A/B soft\hyp commuting,
Restart/Summability, arithmetic alignment
(\texttt{T-Iwasawa-Alignment}), and---in HDPS mode---the stability of
\(\Phi_\tau\) under resolution changes.
Optional tests include \texttt{T-PFBC-AfterCollapse} and
\texttt{T-}\(\Lambda_{\!len}\).

\paragraph{Limitations and guard\hyp rails.}
All claims are confined to the implementable persistence/spectral/categorical
layers \emph{after collapse}.
No claim of a global equivalence \(\mathrm{PH}_1\Leftrightarrow \Ext^1\) is
made; only the one\hyp way implication under (B1)–(B3) and the locally
certified reverse under the Spectral\hyp Gap Condition are used, and then only
within definable windows.
Spectral indicators are not f.q.i.\ invariants; they are evaluated under fixed
policies with deletion\hyp type monotonicity and general non\hyp expansiveness.
The collapse diagnostic \(\mu_{\mathrm{Collapse}}\) is distinct from the
classical Iwasawa \(\mu\).
The HDPS/AI layer (Hunter/Mapper/Lifter, Bridge Programs, case studies) is
entirely \textbf{[Spec]}: it can \emph{propose} search trajectories and
Validity Maps, but only the Core/UCC can \emph{certify} them.

\paragraph{Outlook.}
Future work will refine quantitative links between persistence energies and
spectral tails, broaden verifiable criteria for stability bands and
\((\mu,\nu)=(0,0)\), and extend formal libraries (shift/interleaving; PF/BC
transport; budget calculus) together with domain templates
(arithmetic/Langlands/PDE/Fukaya) equipped with auditable \(\delta\)\hyp
controls.
On the HDPS side, further development of Bridge Programs, Validity Maps, and
domain\hyp specific case studies (NSE, BSD, RH, Langlands) may clarify when
AI\hyp assisted search can be safely promoted to Core\hyp level proofs.

\paragraph{Acknowledgments.}
The manuscript was prepared by the author, with assistance from an AI tool
(ChatGPT) through an iterative workflow; any remaining errors are the author’s
responsibility.

\medskip
\noindent\textbf{Final note.}
The separation between the provable \emph{Core} and auditable \textbf{[Spec]}
contracts, together with the after\hyp collapse order, Quantale\hyp aggregated
\(\delta\)\hyp budgets, Restart/Summability, and the HDPS/AI platform, provides
a conservative and extensible methodology for cross\hyp domain reuse within
the guard\hyp rails of \textbf{v17.0 (AK--HDPST + HDPS Engine)}.



\end{document}