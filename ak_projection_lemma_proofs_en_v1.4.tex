
% =====================================================
% AK High-Dimensional Projection Structural Theory - Formal Lemma Proofs (v1.4, English)
% =====================================================

\documentclass[11pt]{article}
\usepackage[utf8]{inputenc}
\usepackage{amsmath,amssymb,amsthm,amsfonts,geometry}
\usepackage{hyperref}
\geometry{margin=1in}

\title{AK High-Dimensional Projection Structural Theory: Formal Proofs of Lemmas}
\author{A. Kobayashi}
\date{Version 1.4 -- May 2025}

\newtheorem{definition}{Definition}[section]
\newtheorem{axiom}[definition]{Axiom}
\newtheorem{theorem}[definition]{Theorem}
\newtheorem{lemma}[definition]{Lemma}
\newtheorem{corollary}[definition]{Corollary}

\begin{document}

\maketitle

\section*{Overview}
This paper provides formal mathematical proofs of the lemmas introduced in the AK High-Dimensional Projection Structural Theory, reinforcing the logical rigor and structural reliability of the framework.

\section{Formal Proofs of Lemmas}

\begin{lemma}[Structural Group Connectivity Lemma]
Given Axiom A2, if each group \( \{G_i\} \) is a connected component and satisfies the adjacency condition \( \overline{G_i} \cap \overline{G_{i+1}} \neq \emptyset \), then the projected space \( \mathcal{P} = \bigcup_i G_i \) is connected.
\end{lemma}

\begin{proof}
Each \(G_i\) is assumed to be a connected closed set. The adjacency ensures that closures of adjacent groups intersect nontrivially. As the union of a finite chain of connected sets with non-empty intersections remains connected, \(\mathcal{P} = \bigcup_i G_i\) is connected.
\end{proof}

\begin{lemma}[Local Smoothness Lemma]
Given Axiom A3, for any group \(G_i\), if the structural stability function \(S_i(t)\) exists and \(\lim_{t \to \infty} S_i(t) = 0\), then the inverse map \(\Phi^{-1}|_{G_i}\) is smooth.
\end{lemma}

\begin{proof}
Assume \(S_i(t)\) quantifies structural features such as topological persistence (e.g., PH distance) or analytical features (e.g., energy gradient). Convergence of \(S_i(t)\) to zero implies the disappearance of structural perturbations over time, yielding a steady geometric/topological regime. Consequently, the inverse \(\Phi^{-1}|_{G_i}\) preserves differentiability and continuity, hence is smooth.
\end{proof}

\begin{lemma}[Inverse Projection Continuity Lemma]
Given Axiom A4 and the above lemmas, the inverse projection \(\Phi^{-1}\) is continuous on each \(G_i\), and hence on the whole space \(\mathcal{P}\).
\end{lemma}

\begin{proof}
Since \(\Phi\) is a structure-preserving continuous map and each \(G_i\) satisfies structural stability, the restriction \(\Phi^{-1}|_{G_i}\) is continuous. As the decomposition \(\{G_i\}\) is MECE, the total map \(\Phi^{-1}\) is a union of local continuous maps and hence globally continuous on \(\mathcal{P}\).
\end{proof}

\section{Main Theorem and Corollary}

\begin{theorem}[Global Smoothness Theorem (Complete Form)]
Given Axioms A1 through A4, and if each structural stability function \(S_i(t)\) is monotonically decreasing and convergent, then the original space \(\mathcal{X}(t)\) evolves smoothly over time.
\end{theorem}

\begin{proof}
Lemma 1 shows that the projected space \(\mathcal{P}\) is connected. Lemmas 2 and 3 establish smoothness on each group and continuity of the inverse map. Together, they imply global smoothness on \(\mathcal{X}(t)\).
\end{proof}

\begin{corollary}[Non-Singularity Corollary]
Under the above conditions, no singularities (blow-up or non-differentiable points) emerge in the evolution of the original space \(\mathcal{X}(t)\).
\end{corollary}

\begin{proof}
The preservation of smoothness implies absence of singularities. Therefore, continuity and differentiability hold globally for all \(t\).
\end{proof}

\end{document}
