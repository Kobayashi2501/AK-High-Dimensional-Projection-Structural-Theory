% =============================================================
% AK高次元射影構造理論:日本語訳付き解説版 (v1.6)
% =============================================================

\documentclass[11pt]{article}
\usepackage[utf8]{inputenc}
\usepackage{amsmath,amssymb,amsthm,amsfonts,geometry,hyperref}
\geometry{margin=1in}

\title{AK高次元射影構造理論\(日本語訳付き解説版\)\thanks{Version 1.6 -- 2025年6月}}
\author{小林篤志 \ ChatGPT共同研究者}
\date{}

\newtheorem{definition}{定義}[section]
\newtheorem{lemma}[definition]{補題}
\newtheorem{proposition}[definition]{命題}
\newtheorem{theorem}[definition]{定理}

\begin{document}

\maketitle

\section*{はじめに}

本稿では、「\textbf{AK高次元射影構造理論}(AK理論)」を提案する。これは、解くのが困難な問題を\textbf{高次元空間に射影}し、\textbf{MECE(Mutually Exclusive, Collectively Exhaustive)なクラスタ構造}に分解することで、局所的に制御可能な形に落とし込む一般的枠組みである。

\begin{quote}
\textbf{「解けない問題は、次元が足りないだけかもしれない。」}
\end{quote}

この思想に基づき、Navier--Stokes方程式v3.2で得られた構造的知見を抽象化し、他の数学的難問にも応用可能な形に整備したものである。

\section{基本定義}

\begin{definition}[AK射影空間]
数学的構造 $X$ に対し、連続写像 $\pi: X \to \mathbb{R}^n$ が存在し、射影先 $\pi(X)$ において群構造やトポロジー構造が解析可能になるとき、$\pi(X)$ を\textbf{AK射影空間}と呼ぶ。
\end{definition}

\begin{definition}[MECEクラスタ構造]
$\pi(X)$ の分割 $\{C_i\}$ が互いに交わらず、かつ全体を覆い、各クラスタ $C_i$ が一貫した解析基準(例えばトポロジー、スペクトル構造など)を持つとき、\textbf{MECEクラスタ構造}と呼ぶ。
\end{definition}

\begin{definition}[AK群構造]
各クラスタ $C_i$ に二項演算 $*_i$ が定義され、それによって $(C_i, *_i)$ が群または準群になるとき、$\{(C_i, *_i)\}$ を\textbf{AK群構造}と呼ぶ。
\end{definition}

\section{構造的補題と命題}

\begin{lemma}[射影による構造保存]
$\pi$ が連続かつトポロジーを保つ(例えば persistent homology により)ならば、$X$ 上の複雑性指標は $\pi(X)$ に移して議論できる。
\end{lemma}

\begin{proposition}[クラスタ単位での証明還元]
問題 $P$ がクラスタごとの命題 $P_i$ に分解でき、各 $P_i$ が独立に証明できるならば、全体の命題 $P$ も成り立つ。
\end{proposition}

\section{主定理群}

\begin{theorem}[トポロジーの崩壊は正則性を示す]
すべてのクラスタ $C_i(t)$ において $\mathrm{PH}_1(C_i(t)) \to 0$ であるならば、元の構造 $X$ は Sobolev的正則性などを持つ。
\end{theorem}

\begin{theorem}[難解性のAK局所化による解決]
証明困難性が特定のクラスタ $C_j$ に局在するならば、AK射影はそれを孤立させ、直接的な解析や退化操作が可能になる。
\end{theorem}

\begin{theorem}[退化幾何による証明条件の境界化]
クラスタ構造がVMHSやトロピカル収束により退化する場合、証明条件はモジュライ空間の境界構造上で定式化できる。
\end{theorem}

\section{フィードバックループ}

AK理論は次の同値性を仮定する:
\[
\text{トポロジー単純化} \Longleftrightarrow \text{軌道の圧縮} \Longleftrightarrow \text{証明の可視化・可解化}。
\]

\section*{謝辞}
この理論は、Navier--Stokes方程式のグローバル正則性に関するv3.2枠組みの経験的成果を抽象化したものである。

\end{document}
